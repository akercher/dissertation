This talk is split into two parts.  The first half focus will be on numerical schemes for ideal magnetohydrodynamics that exhibit pseudo-convergence in which mon-regular waves no longer exist only after heavy grid refinement.  A method is described for obtaining solutions for coplanar and near-coplanar cases that consist of only regular waves, independent of grid refinement.  The method, referred to as Compound Wave Modification (CWM), involves removing the flux associated with non-regular waves and can be used for simulations in two- and three-dimensions because it does not require explicitly tracking an Alfv{\'e}n wave.  For a near-coplanar case, and for grids with $2^{13}$ points or less, the root-square-mean-errors (RSMEs) are as much as 6 times smaller.  For the coplanar case, in which non-regular structures will exist at all levels of grid refinement for the standard FVMs, the RMSE is as much as 25 times smaller.  
 
The second half will cover a multidimensional ideal MHD code that has been implemented for simulations on graphics processing units (GPUs).  Performance measurements are conducted on both a NVIDIA GeForce GTX Titan and Intel Xeon processor. The GPU is shown to perform one to two orders of magnitude faster than for serial runs on the CPU, and two to three times greater than when run in parallel with OpenMP on the CPU.  Performance measurements are also given for two methods of storing data on the GPU.  The first approach stores data as an Array of Structures (AoS), e.g., a point coordinate array of size $3\times n$ is iterated over.  The second approach stores data as a Structure of Arrays (SoA), e.g. three separate arrays of size $n$ are iterated over simultaneously.  For an AoS, coalescing does not occur, reducing memory efficiency.  The SoA approach is shown to improve performance by one and a half to two times depending on the algorithm used.  All results are given for Cartesian grids, but the algorithms are implemented for a general geometry on a unstructured grids.

