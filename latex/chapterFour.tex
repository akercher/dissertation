%
% $Id: chapterFour.tex
%

% A first, optional argument in [ ] is the title as displayed in the table of contents
% The second argument is the title as displayed here.  Use \\ as appropriate in
%   this title to get desired line breaks

%----------------------------------------------------------------------------------------
% Compound wave modification
%----------------------------------------------------------------------------------------
\chapter[Compound wave modification]{Compound wave modification}
\label{chp:cwm}

In Chapter~\ref{chp:num_mhd}, it was shown that FV schemes exhibit pseudo-convergence, where convergence to the solution containing regular waves only occurs at higher grid resolutions, for near coplanar case, and not at all for the coplanar case of a Riemann problem of ideal MHD.  This chapter gives a detailed explanation of the formation of a compound wave and purposes a new method that involves modifying HLLD flux, which produces solutions with only regular waves at all grid resolution for coplanar and non-planar cases.  The process, referred to as compound wave modification (CWM), approximates the flux associated with the compound wave and removes it from the HLLD flux before the solution is advanced in time.  The performance increase from the new HLLD-CWM is demonstrated with root-square-mean-error (RMSE) calculations.  

%**************************************************************************************************************
%----------------------------------------------------------------------------------------
%	Convergence with finite volume schemes
%----------------------------------------------------------------------------------------
\section[Convergence with finite volume schemes]{Convergence with finite volume schemes}
\label{sec:convergeFV}

%-----------------------------------------------------------------
% Compound wave modification
%-----------------------------------------------------------------
%% \subsection[Compound wave modification]{Compound wave modification}
%% \label{sec:CWM}

The influence of a compound wave on the numerical solution can be minimized by limiting the artificial perturbation in pressure caused by numerical diffusion.  We do this by subtracting flux so that only the tangential components of momentum and the magnetic field are affected and thus the upstream and downstream states will still satisfy the jump conditions for a rotational discontinuity.  The subtracted flux is chosen to be proportional to the flux contribution of the compound wave.  We refer to this as compound wave modification (CWM) and its use in conjunction with HLLD as HLLD-CWM.  The flux responsible for the formation of the compound wave is calculated by solving a reduced Riemann problem with initial conditions set to the upstream and downstream states of the $180\dsym$ rotational discontinuity.  These intermediate states correspond to regions $\mbf{U}_{l}^*$ and $\mbf{U}_{2l}^*$ of Figure~\ref{fig:mhd_states} for the coplanar (near-coplanar) problem shown in Figure~\ref{fig:coplanar_a_csol} (\ref{fig:coplanar_b_pcon}).  For test 5a, the exact values in the intermediate region are given by rows 2 and 3 of Table~\ref{tab:AK5}.  The exact values are generally not known, so they are approximated with HLLD.  If the wave is traveling in the positive direction, i.e., the direction of the outward face normal, then intermediate states correspond to regions $\mbf{U}_{2r}^*$ and $\mbf{U}_{r}^*$ of Figure~\ref{fig:mhd_states}.  The direction of the wave is determined by the wave speeds, $S_m$ \eqref{eqn:hlld_sm}, and $S^*_r$ \eqref{eqn:hlld_rd_spd}.  The interface state are $\mbf{U}_{l}^*$ and $\mbf{U}_{2l}^*$, if $S_m > 0$, and $\mbf{U}_{2r}^*$ and $\mbf{U}_{r}^*$, if $S^*_r > 0$. 

%----------------------------------------------------------------
% Coplanar waves
%-----------------------------------------------------------------
\begin{figure}[htbp] 
\begin{tabular}{cc}
\resizebox{0.5\linewidth}{!}{\tikzsetnextfilename{coplanar_a_cwaves_1}\begin{tikzpicture}[gnuplot]
%% generated with GNUPLOT 4.6p4 (Lua 5.1; terminal rev. 99, script rev. 100)
%% Sat 31 May 2014 03:20:24 PM EDT
\path (0.000,0.000) rectangle (8.500,6.000);
\gpfill{rgb color={1.000,1.000,1.000}} (1.196,0.985)--(7.946,0.985)--(7.946,5.630)--(1.196,5.630)--cycle;
\gpcolor{color=gp lt color border}
\gpsetlinetype{gp lt border}
\gpsetlinewidth{1.00}
\draw[gp path] (1.196,0.985)--(1.196,5.630)--(7.946,5.630)--(7.946,0.985)--cycle;
\gpcolor{color=gp lt color axes}
\gpsetlinetype{gp lt axes}
\gpsetlinewidth{2.00}
\draw[gp path] (1.196,0.985)--(7.947,0.985);
\gpcolor{color=gp lt color border}
\gpsetlinetype{gp lt border}
\draw[gp path] (1.196,0.985)--(1.268,0.985);
\draw[gp path] (7.947,0.985)--(7.875,0.985);
\gpcolor{rgb color={0.000,0.000,0.000}}
\node[gp node right,font={\fontsize{10pt}{12pt}\selectfont}] at (1.012,0.985) {0.6};
\gpcolor{color=gp lt color axes}
\gpsetlinetype{gp lt axes}
\draw[gp path] (1.196,1.759)--(7.947,1.759);
\gpcolor{color=gp lt color border}
\gpsetlinetype{gp lt border}
\draw[gp path] (1.196,1.759)--(1.268,1.759);
\draw[gp path] (7.947,1.759)--(7.875,1.759);
\gpcolor{rgb color={0.000,0.000,0.000}}
\node[gp node right,font={\fontsize{10pt}{12pt}\selectfont}] at (1.012,1.759) {0.65};
\gpcolor{color=gp lt color axes}
\gpsetlinetype{gp lt axes}
\draw[gp path] (1.196,2.534)--(7.947,2.534);
\gpcolor{color=gp lt color border}
\gpsetlinetype{gp lt border}
\draw[gp path] (1.196,2.534)--(1.268,2.534);
\draw[gp path] (7.947,2.534)--(7.875,2.534);
\gpcolor{rgb color={0.000,0.000,0.000}}
\node[gp node right,font={\fontsize{10pt}{12pt}\selectfont}] at (1.012,2.534) {0.7};
\gpcolor{color=gp lt color axes}
\gpsetlinetype{gp lt axes}
\draw[gp path] (1.196,3.308)--(7.947,3.308);
\gpcolor{color=gp lt color border}
\gpsetlinetype{gp lt border}
\draw[gp path] (1.196,3.308)--(1.268,3.308);
\draw[gp path] (7.947,3.308)--(7.875,3.308);
\gpcolor{rgb color={0.000,0.000,0.000}}
\node[gp node right,font={\fontsize{10pt}{12pt}\selectfont}] at (1.012,3.308) {0.75};
\gpcolor{color=gp lt color axes}
\gpsetlinetype{gp lt axes}
\draw[gp path] (1.196,4.082)--(7.947,4.082);
\gpcolor{color=gp lt color border}
\gpsetlinetype{gp lt border}
\draw[gp path] (1.196,4.082)--(1.268,4.082);
\draw[gp path] (7.947,4.082)--(7.875,4.082);
\gpcolor{rgb color={0.000,0.000,0.000}}
\node[gp node right,font={\fontsize{10pt}{12pt}\selectfont}] at (1.012,4.082) {0.8};
\gpcolor{color=gp lt color axes}
\gpsetlinetype{gp lt axes}
\draw[gp path] (1.196,4.857)--(7.947,4.857);
\gpcolor{color=gp lt color border}
\gpsetlinetype{gp lt border}
\draw[gp path] (1.196,4.857)--(1.268,4.857);
\draw[gp path] (7.947,4.857)--(7.875,4.857);
\gpcolor{rgb color={0.000,0.000,0.000}}
\node[gp node right,font={\fontsize{10pt}{12pt}\selectfont}] at (1.012,4.857) {0.85};
\gpcolor{color=gp lt color axes}
\gpsetlinetype{gp lt axes}
\draw[gp path] (1.196,5.631)--(7.947,5.631);
\gpcolor{color=gp lt color border}
\gpsetlinetype{gp lt border}
\draw[gp path] (1.196,5.631)--(1.268,5.631);
\draw[gp path] (7.947,5.631)--(7.875,5.631);
\gpcolor{rgb color={0.000,0.000,0.000}}
\node[gp node right,font={\fontsize{10pt}{12pt}\selectfont}] at (1.012,5.631) {0.9};
\gpcolor{color=gp lt color axes}
\gpsetlinetype{gp lt axes}
\draw[gp path] (1.196,0.985)--(1.196,5.631);
\gpcolor{color=gp lt color border}
\gpsetlinetype{gp lt border}
\draw[gp path] (1.196,0.985)--(1.196,1.057);
\draw[gp path] (1.196,5.631)--(1.196,5.559);
\gpcolor{rgb color={0.000,0.000,0.000}}
\node[gp node center,font={\fontsize{10pt}{12pt}\selectfont}] at (1.196,0.677) {0.2};
\gpcolor{color=gp lt color axes}
\gpsetlinetype{gp lt axes}
\draw[gp path] (2.321,0.985)--(2.321,5.631);
\gpcolor{color=gp lt color border}
\gpsetlinetype{gp lt border}
\draw[gp path] (2.321,0.985)--(2.321,1.057);
\draw[gp path] (2.321,5.631)--(2.321,5.559);
\gpcolor{rgb color={0.000,0.000,0.000}}
\node[gp node center,font={\fontsize{10pt}{12pt}\selectfont}] at (2.321,0.677) {0.25};
\gpcolor{color=gp lt color axes}
\gpsetlinetype{gp lt axes}
\draw[gp path] (3.446,0.985)--(3.446,5.631);
\gpcolor{color=gp lt color border}
\gpsetlinetype{gp lt border}
\draw[gp path] (3.446,0.985)--(3.446,1.057);
\draw[gp path] (3.446,5.631)--(3.446,5.559);
\gpcolor{rgb color={0.000,0.000,0.000}}
\node[gp node center,font={\fontsize{10pt}{12pt}\selectfont}] at (3.446,0.677) {0.3};
\gpcolor{color=gp lt color axes}
\gpsetlinetype{gp lt axes}
\draw[gp path] (4.572,0.985)--(4.572,5.631);
\gpcolor{color=gp lt color border}
\gpsetlinetype{gp lt border}
\draw[gp path] (4.572,0.985)--(4.572,1.057);
\draw[gp path] (4.572,5.631)--(4.572,5.559);
\gpcolor{rgb color={0.000,0.000,0.000}}
\node[gp node center,font={\fontsize{10pt}{12pt}\selectfont}] at (4.572,0.677) {0.35};
\gpcolor{color=gp lt color axes}
\gpsetlinetype{gp lt axes}
\draw[gp path] (5.697,0.985)--(5.697,5.631);
\gpcolor{color=gp lt color border}
\gpsetlinetype{gp lt border}
\draw[gp path] (5.697,0.985)--(5.697,1.057);
\draw[gp path] (5.697,5.631)--(5.697,5.559);
\gpcolor{rgb color={0.000,0.000,0.000}}
\node[gp node center,font={\fontsize{10pt}{12pt}\selectfont}] at (5.697,0.677) {0.4};
\gpcolor{color=gp lt color axes}
\gpsetlinetype{gp lt axes}
\draw[gp path] (6.822,0.985)--(6.822,5.631);
\gpcolor{color=gp lt color border}
\gpsetlinetype{gp lt border}
\draw[gp path] (6.822,0.985)--(6.822,1.057);
\draw[gp path] (6.822,5.631)--(6.822,5.559);
\gpcolor{rgb color={0.000,0.000,0.000}}
\node[gp node center,font={\fontsize{10pt}{12pt}\selectfont}] at (6.822,0.677) {0.45};
\gpcolor{color=gp lt color axes}
\gpsetlinetype{gp lt axes}
\draw[gp path] (7.947,0.985)--(7.947,5.631);
\gpcolor{color=gp lt color border}
\gpsetlinetype{gp lt border}
\draw[gp path] (7.947,0.985)--(7.947,1.057);
\draw[gp path] (7.947,5.631)--(7.947,5.559);
\gpcolor{rgb color={0.000,0.000,0.000}}
\node[gp node center,font={\fontsize{10pt}{12pt}\selectfont}] at (7.947,0.677) {0.5};
\gpcolor{color=gp lt color border}
\draw[gp path] (1.196,5.631)--(1.196,0.985)--(7.947,0.985)--(7.947,5.631)--cycle;
\gpcolor{rgb color={0.000,0.000,0.000}}
\node[gp node center,font={\fontsize{10pt}{12pt}\selectfont}] at (4.571,0.215) {\large $x$};
\gpcolor{rgb color={1.000,0.000,0.000}}
\gpsetlinewidth{0.50}
\gpsetpointsize{4.44}
\gppoint{gp mark 7}{(1.206,4.163)}
\gppoint{gp mark 7}{(1.217,4.147)}
\gppoint{gp mark 7}{(1.228,4.131)}
\gppoint{gp mark 7}{(1.239,4.114)}
\gppoint{gp mark 7}{(1.250,4.098)}
\gppoint{gp mark 7}{(1.261,4.081)}
\gppoint{gp mark 7}{(1.272,4.065)}
\gppoint{gp mark 7}{(1.283,4.049)}
\gppoint{gp mark 7}{(1.294,4.032)}
\gppoint{gp mark 7}{(1.305,4.016)}
\gppoint{gp mark 7}{(1.316,4.000)}
\gppoint{gp mark 7}{(1.327,3.983)}
\gppoint{gp mark 7}{(1.338,3.967)}
\gppoint{gp mark 7}{(1.349,3.951)}
\gppoint{gp mark 7}{(1.360,3.935)}
\gppoint{gp mark 7}{(1.371,3.918)}
\gppoint{gp mark 7}{(1.382,3.902)}
\gppoint{gp mark 7}{(1.393,3.886)}
\gppoint{gp mark 7}{(1.404,3.870)}
\gppoint{gp mark 7}{(1.415,3.853)}
\gppoint{gp mark 7}{(1.426,3.837)}
\gppoint{gp mark 7}{(1.437,3.821)}
\gppoint{gp mark 7}{(1.448,3.805)}
\gppoint{gp mark 7}{(1.459,3.789)}
\gppoint{gp mark 7}{(1.470,3.773)}
\gppoint{gp mark 7}{(1.481,3.757)}
\gppoint{gp mark 7}{(1.492,3.740)}
\gppoint{gp mark 7}{(1.503,3.724)}
\gppoint{gp mark 7}{(1.514,3.708)}
\gppoint{gp mark 7}{(1.525,3.692)}
\gppoint{gp mark 7}{(1.536,3.676)}
\gppoint{gp mark 7}{(1.547,3.660)}
\gppoint{gp mark 7}{(1.558,3.644)}
\gppoint{gp mark 7}{(1.568,3.628)}
\gppoint{gp mark 7}{(1.579,3.612)}
\gppoint{gp mark 7}{(1.590,3.596)}
\gppoint{gp mark 7}{(1.601,3.580)}
\gppoint{gp mark 7}{(1.612,3.564)}
\gppoint{gp mark 7}{(1.623,3.548)}
\gppoint{gp mark 7}{(1.634,3.532)}
\gppoint{gp mark 7}{(1.645,3.516)}
\gppoint{gp mark 7}{(1.656,3.500)}
\gppoint{gp mark 7}{(1.667,3.484)}
\gppoint{gp mark 7}{(1.678,3.469)}
\gppoint{gp mark 7}{(1.689,3.453)}
\gppoint{gp mark 7}{(1.700,3.437)}
\gppoint{gp mark 7}{(1.711,3.421)}
\gppoint{gp mark 7}{(1.722,3.405)}
\gppoint{gp mark 7}{(1.733,3.389)}
\gppoint{gp mark 7}{(1.744,3.374)}
\gppoint{gp mark 7}{(1.755,3.358)}
\gppoint{gp mark 7}{(1.766,3.342)}
\gppoint{gp mark 7}{(1.777,3.326)}
\gppoint{gp mark 7}{(1.788,3.311)}
\gppoint{gp mark 7}{(1.799,3.295)}
\gppoint{gp mark 7}{(1.810,3.279)}
\gppoint{gp mark 7}{(1.821,3.263)}
\gppoint{gp mark 7}{(1.832,3.248)}
\gppoint{gp mark 7}{(1.843,3.232)}
\gppoint{gp mark 7}{(1.854,3.216)}
\gppoint{gp mark 7}{(1.865,3.201)}
\gppoint{gp mark 7}{(1.876,3.185)}
\gppoint{gp mark 7}{(1.887,3.169)}
\gppoint{gp mark 7}{(1.898,3.154)}
\gppoint{gp mark 7}{(1.909,3.138)}
\gppoint{gp mark 7}{(1.920,3.123)}
\gppoint{gp mark 7}{(1.931,3.107)}
\gppoint{gp mark 7}{(1.942,3.091)}
\gppoint{gp mark 7}{(1.953,3.076)}
\gppoint{gp mark 7}{(1.964,3.060)}
\gppoint{gp mark 7}{(1.975,3.045)}
\gppoint{gp mark 7}{(1.986,3.029)}
\gppoint{gp mark 7}{(1.997,3.014)}
\gppoint{gp mark 7}{(2.008,2.998)}
\gppoint{gp mark 7}{(2.019,2.983)}
\gppoint{gp mark 7}{(2.030,2.967)}
\gppoint{gp mark 7}{(2.041,2.952)}
\gppoint{gp mark 7}{(2.052,2.937)}
\gppoint{gp mark 7}{(2.063,2.921)}
\gppoint{gp mark 7}{(2.074,2.906)}
\gppoint{gp mark 7}{(2.085,2.890)}
\gppoint{gp mark 7}{(2.096,2.875)}
\gppoint{gp mark 7}{(2.107,2.860)}
\gppoint{gp mark 7}{(2.118,2.844)}
\gppoint{gp mark 7}{(2.129,2.829)}
\gppoint{gp mark 7}{(2.140,2.814)}
\gppoint{gp mark 7}{(2.151,2.798)}
\gppoint{gp mark 7}{(2.162,2.783)}
\gppoint{gp mark 7}{(2.173,2.768)}
\gppoint{gp mark 7}{(2.184,2.753)}
\gppoint{gp mark 7}{(2.195,2.737)}
\gppoint{gp mark 7}{(2.206,2.722)}
\gppoint{gp mark 7}{(2.217,2.707)}
\gppoint{gp mark 7}{(2.228,2.692)}
\gppoint{gp mark 7}{(2.239,2.676)}
\gppoint{gp mark 7}{(2.250,2.661)}
\gppoint{gp mark 7}{(2.261,2.646)}
\gppoint{gp mark 7}{(2.272,2.631)}
\gppoint{gp mark 7}{(2.283,2.616)}
\gppoint{gp mark 7}{(2.294,2.601)}
\gppoint{gp mark 7}{(2.305,2.586)}
\gppoint{gp mark 7}{(2.316,2.570)}
\gppoint{gp mark 7}{(2.327,2.555)}
\gppoint{gp mark 7}{(2.338,2.540)}
\gppoint{gp mark 7}{(2.349,2.525)}
\gppoint{gp mark 7}{(2.360,2.510)}
\gppoint{gp mark 7}{(2.371,2.495)}
\gppoint{gp mark 7}{(2.382,2.480)}
\gppoint{gp mark 7}{(2.393,2.465)}
\gppoint{gp mark 7}{(2.404,2.450)}
\gppoint{gp mark 7}{(2.415,2.435)}
\gppoint{gp mark 7}{(2.426,2.420)}
\gppoint{gp mark 7}{(2.437,2.405)}
\gppoint{gp mark 7}{(2.448,2.390)}
\gppoint{gp mark 7}{(2.459,2.375)}
\gppoint{gp mark 7}{(2.470,2.361)}
\gppoint{gp mark 7}{(2.480,2.346)}
\gppoint{gp mark 7}{(2.491,2.331)}
\gppoint{gp mark 7}{(2.502,2.316)}
\gppoint{gp mark 7}{(2.513,2.301)}
\gppoint{gp mark 7}{(2.524,2.286)}
\gppoint{gp mark 7}{(2.535,2.271)}
\gppoint{gp mark 7}{(2.546,2.257)}
\gppoint{gp mark 7}{(2.557,2.242)}
\gppoint{gp mark 7}{(2.568,2.227)}
\gppoint{gp mark 7}{(2.579,2.212)}
\gppoint{gp mark 7}{(2.590,2.198)}
\gppoint{gp mark 7}{(2.601,2.183)}
\gppoint{gp mark 7}{(2.612,2.168)}
\gppoint{gp mark 7}{(2.623,2.154)}
\gppoint{gp mark 7}{(2.634,2.139)}
\gppoint{gp mark 7}{(2.645,2.124)}
\gppoint{gp mark 7}{(2.656,2.110)}
\gppoint{gp mark 7}{(2.667,2.095)}
\gppoint{gp mark 7}{(2.678,2.080)}
\gppoint{gp mark 7}{(2.689,2.066)}
\gppoint{gp mark 7}{(2.700,2.051)}
\gppoint{gp mark 7}{(2.711,2.037)}
\gppoint{gp mark 7}{(2.722,2.022)}
\gppoint{gp mark 7}{(2.733,2.008)}
\gppoint{gp mark 7}{(2.744,1.993)}
\gppoint{gp mark 7}{(2.755,1.979)}
\gppoint{gp mark 7}{(2.766,1.964)}
\gppoint{gp mark 7}{(2.777,1.950)}
\gppoint{gp mark 7}{(2.788,1.935)}
\gppoint{gp mark 7}{(2.799,1.921)}
\gppoint{gp mark 7}{(2.810,1.907)}
\gppoint{gp mark 7}{(2.821,1.892)}
\gppoint{gp mark 7}{(2.832,1.878)}
\gppoint{gp mark 7}{(2.843,1.864)}
\gppoint{gp mark 7}{(2.854,1.850)}
\gppoint{gp mark 7}{(2.865,1.836)}
\gppoint{gp mark 7}{(2.876,1.822)}
\gppoint{gp mark 7}{(2.887,1.807)}
\gppoint{gp mark 7}{(2.898,1.793)}
\gppoint{gp mark 7}{(2.909,1.779)}
\gppoint{gp mark 7}{(2.920,1.766)}
\gppoint{gp mark 7}{(2.931,1.752)}
\gppoint{gp mark 7}{(2.942,1.738)}
\gppoint{gp mark 7}{(2.953,1.724)}
\gppoint{gp mark 7}{(2.964,1.710)}
\gppoint{gp mark 7}{(2.975,1.697)}
\gppoint{gp mark 7}{(2.986,1.683)}
\gppoint{gp mark 7}{(2.997,1.669)}
\gppoint{gp mark 7}{(3.008,1.655)}
\gppoint{gp mark 7}{(3.019,1.640)}
\gppoint{gp mark 7}{(3.030,1.626)}
\gppoint{gp mark 7}{(3.041,1.613)}
\gppoint{gp mark 7}{(3.052,1.605)}
\gppoint{gp mark 7}{(3.063,1.602)}
\gppoint{gp mark 7}{(3.074,1.601)}
\gppoint{gp mark 7}{(3.085,1.602)}
\gppoint{gp mark 7}{(3.096,1.606)}
\gppoint{gp mark 7}{(3.107,1.626)}
\gppoint{gp mark 7}{(3.118,1.664)}
\gppoint{gp mark 7}{(3.129,1.695)}
\gppoint{gp mark 7}{(3.140,1.703)}
\gppoint{gp mark 7}{(3.151,1.704)}
\gppoint{gp mark 7}{(3.162,1.704)}
\gppoint{gp mark 7}{(3.173,1.704)}
\gppoint{gp mark 7}{(3.184,1.704)}
\gppoint{gp mark 7}{(3.195,1.704)}
\gppoint{gp mark 7}{(3.206,1.703)}
\gppoint{gp mark 7}{(3.217,1.701)}
\gppoint{gp mark 7}{(3.228,1.699)}
\gppoint{gp mark 7}{(3.239,1.696)}
\gppoint{gp mark 7}{(3.250,1.692)}
\gppoint{gp mark 7}{(3.261,1.690)}
\gppoint{gp mark 7}{(3.272,1.687)}
\gppoint{gp mark 7}{(3.283,1.685)}
\gppoint{gp mark 7}{(3.294,1.684)}
\gppoint{gp mark 7}{(3.305,1.682)}
\gppoint{gp mark 7}{(3.316,1.681)}
\gppoint{gp mark 7}{(3.327,1.680)}
\gppoint{gp mark 7}{(3.338,1.680)}
\gppoint{gp mark 7}{(3.349,1.679)}
\gppoint{gp mark 7}{(3.360,1.679)}
\gppoint{gp mark 7}{(3.371,1.679)}
\gppoint{gp mark 7}{(3.382,1.679)}
\gppoint{gp mark 7}{(3.392,1.680)}
\gppoint{gp mark 7}{(3.403,1.680)}
\gppoint{gp mark 7}{(3.414,1.681)}
\gppoint{gp mark 7}{(3.425,1.682)}
\gppoint{gp mark 7}{(3.436,1.683)}
\gppoint{gp mark 7}{(3.447,1.684)}
\gppoint{gp mark 7}{(3.458,1.685)}
\gppoint{gp mark 7}{(3.469,1.685)}
\gppoint{gp mark 7}{(3.480,1.686)}
\gppoint{gp mark 7}{(3.491,1.687)}
\gppoint{gp mark 7}{(3.502,1.706)}
\gppoint{gp mark 7}{(3.513,2.151)}
\gppoint{gp mark 7}{(3.524,4.379)}
\gppoint{gp mark 7}{(3.535,4.982)}
\gppoint{gp mark 7}{(3.546,4.738)}
\gppoint{gp mark 7}{(3.557,4.642)}
\gppoint{gp mark 7}{(3.568,4.603)}
\gppoint{gp mark 7}{(3.579,4.567)}
\gppoint{gp mark 7}{(3.590,4.551)}
\gppoint{gp mark 7}{(3.601,4.544)}
\gppoint{gp mark 7}{(3.612,4.509)}
\gppoint{gp mark 7}{(3.623,4.469)}
\gppoint{gp mark 7}{(3.634,4.449)}
\gppoint{gp mark 7}{(3.645,4.430)}
\gppoint{gp mark 7}{(3.656,4.398)}
\gppoint{gp mark 7}{(3.667,4.361)}
\gppoint{gp mark 7}{(3.678,4.336)}
\gppoint{gp mark 7}{(3.689,4.317)}
\gppoint{gp mark 7}{(3.700,4.291)}
\gppoint{gp mark 7}{(3.711,4.259)}
\gppoint{gp mark 7}{(3.722,4.230)}
\gppoint{gp mark 7}{(3.733,4.205)}
\gppoint{gp mark 7}{(3.744,4.180)}
\gppoint{gp mark 7}{(3.755,4.152)}
\gppoint{gp mark 7}{(3.766,4.125)}
\gppoint{gp mark 7}{(3.777,4.098)}
\gppoint{gp mark 7}{(3.788,4.072)}
\gppoint{gp mark 7}{(3.799,4.045)}
\gppoint{gp mark 7}{(3.810,4.018)}
\gppoint{gp mark 7}{(3.821,3.990)}
\gppoint{gp mark 7}{(3.832,3.963)}
\gppoint{gp mark 7}{(3.843,3.937)}
\gppoint{gp mark 7}{(3.854,3.912)}
\gppoint{gp mark 7}{(3.865,3.885)}
\gppoint{gp mark 7}{(3.876,3.856)}
\gppoint{gp mark 7}{(3.887,3.830)}
\gppoint{gp mark 7}{(3.898,3.805)}
\gppoint{gp mark 7}{(3.909,3.780)}
\gppoint{gp mark 7}{(3.920,3.753)}
\gppoint{gp mark 7}{(3.931,3.724)}
\gppoint{gp mark 7}{(3.942,3.696)}
\gppoint{gp mark 7}{(3.953,3.671)}
\gppoint{gp mark 7}{(3.964,3.647)}
\gppoint{gp mark 7}{(3.975,3.621)}
\gppoint{gp mark 7}{(3.986,3.593)}
\gppoint{gp mark 7}{(3.997,3.567)}
\gppoint{gp mark 7}{(4.008,3.542)}
\gppoint{gp mark 7}{(4.019,3.518)}
\gppoint{gp mark 7}{(4.030,3.492)}
\gppoint{gp mark 7}{(4.041,3.465)}
\gppoint{gp mark 7}{(4.052,3.440)}
\gppoint{gp mark 7}{(4.063,3.417)}
\gppoint{gp mark 7}{(4.074,3.395)}
\gppoint{gp mark 7}{(4.085,3.376)}
\gppoint{gp mark 7}{(4.096,3.363)}
\gppoint{gp mark 7}{(4.107,3.357)}
\gppoint{gp mark 7}{(4.118,3.358)}
\gppoint{gp mark 7}{(4.129,3.358)}
\gppoint{gp mark 7}{(4.140,3.358)}
\gppoint{gp mark 7}{(4.151,3.359)}
\gppoint{gp mark 7}{(4.162,3.361)}
\gppoint{gp mark 7}{(4.173,3.365)}
\gppoint{gp mark 7}{(4.184,3.368)}
\gppoint{gp mark 7}{(4.195,3.369)}
\gppoint{gp mark 7}{(4.206,3.368)}
\gppoint{gp mark 7}{(4.217,3.368)}
\gppoint{gp mark 7}{(4.228,3.368)}
\gppoint{gp mark 7}{(4.239,3.369)}
\gppoint{gp mark 7}{(4.250,3.369)}
\gppoint{gp mark 7}{(4.261,3.369)}
\gppoint{gp mark 7}{(4.272,3.368)}
\gppoint{gp mark 7}{(4.283,3.368)}
\gppoint{gp mark 7}{(4.294,3.368)}
\gppoint{gp mark 7}{(4.304,3.369)}
\gppoint{gp mark 7}{(4.315,3.369)}
\gppoint{gp mark 7}{(4.326,3.368)}
\gppoint{gp mark 7}{(4.337,3.368)}
\gppoint{gp mark 7}{(4.348,3.368)}
\gppoint{gp mark 7}{(4.359,3.368)}
\gppoint{gp mark 7}{(4.370,3.368)}
\gppoint{gp mark 7}{(4.381,3.367)}
\gppoint{gp mark 7}{(4.392,3.367)}
\gppoint{gp mark 7}{(4.403,3.368)}
\gppoint{gp mark 7}{(4.414,3.369)}
\gppoint{gp mark 7}{(4.425,3.369)}
\gppoint{gp mark 7}{(4.436,3.368)}
\gppoint{gp mark 7}{(4.447,3.368)}
\gppoint{gp mark 7}{(4.458,3.368)}
\gppoint{gp mark 7}{(4.469,3.368)}
\gppoint{gp mark 7}{(4.480,3.368)}
\gppoint{gp mark 7}{(4.491,3.368)}
\gppoint{gp mark 7}{(4.502,3.367)}
\gppoint{gp mark 7}{(4.513,3.368)}
\gppoint{gp mark 7}{(4.524,3.368)}
\gppoint{gp mark 7}{(4.535,3.368)}
\gppoint{gp mark 7}{(4.546,3.369)}
\gppoint{gp mark 7}{(4.557,3.369)}
\gppoint{gp mark 7}{(4.568,3.368)}
\gppoint{gp mark 7}{(4.579,3.368)}
\gppoint{gp mark 7}{(4.590,3.367)}
\gppoint{gp mark 7}{(4.601,3.367)}
\gppoint{gp mark 7}{(4.612,3.367)}
\gppoint{gp mark 7}{(4.623,3.367)}
\gppoint{gp mark 7}{(4.634,3.368)}
\gppoint{gp mark 7}{(4.645,3.368)}
\gppoint{gp mark 7}{(4.656,3.368)}
\gppoint{gp mark 7}{(4.667,3.368)}
\gppoint{gp mark 7}{(4.678,3.368)}
\gppoint{gp mark 7}{(4.689,3.368)}
\gppoint{gp mark 7}{(4.700,3.368)}
\gppoint{gp mark 7}{(4.711,3.368)}
\gppoint{gp mark 7}{(4.722,3.367)}
\gppoint{gp mark 7}{(4.733,3.367)}
\gppoint{gp mark 7}{(4.744,3.368)}
\gppoint{gp mark 7}{(4.755,3.368)}
\gppoint{gp mark 7}{(4.766,3.368)}
\gppoint{gp mark 7}{(4.777,3.368)}
\gppoint{gp mark 7}{(4.788,3.368)}
\gppoint{gp mark 7}{(4.799,3.368)}
\gppoint{gp mark 7}{(4.810,3.368)}
\gppoint{gp mark 7}{(4.821,3.368)}
\gppoint{gp mark 7}{(4.832,3.368)}
\gppoint{gp mark 7}{(4.843,3.368)}
\gppoint{gp mark 7}{(4.854,3.368)}
\gppoint{gp mark 7}{(4.865,3.368)}
\gppoint{gp mark 7}{(4.876,3.368)}
\gppoint{gp mark 7}{(4.887,3.368)}
\gppoint{gp mark 7}{(4.898,3.368)}
\gppoint{gp mark 7}{(4.909,3.368)}
\gppoint{gp mark 7}{(4.920,3.368)}
\gppoint{gp mark 7}{(4.931,3.368)}
\gppoint{gp mark 7}{(4.942,3.368)}
\gppoint{gp mark 7}{(4.953,3.368)}
\gppoint{gp mark 7}{(4.964,3.369)}
\gppoint{gp mark 7}{(4.975,3.369)}
\gppoint{gp mark 7}{(4.986,3.368)}
\gppoint{gp mark 7}{(4.997,3.368)}
\gppoint{gp mark 7}{(5.008,3.367)}
\gppoint{gp mark 7}{(5.019,3.367)}
\gppoint{gp mark 7}{(5.030,3.367)}
\gppoint{gp mark 7}{(5.041,3.368)}
\gppoint{gp mark 7}{(5.052,3.368)}
\gppoint{gp mark 7}{(5.063,3.369)}
\gppoint{gp mark 7}{(5.074,3.369)}
\gppoint{gp mark 7}{(5.085,3.369)}
\gppoint{gp mark 7}{(5.096,3.368)}
\gppoint{gp mark 7}{(5.107,3.368)}
\gppoint{gp mark 7}{(5.118,3.368)}
\gppoint{gp mark 7}{(5.129,3.368)}
\gppoint{gp mark 7}{(5.140,3.368)}
\gppoint{gp mark 7}{(5.151,3.368)}
\gppoint{gp mark 7}{(5.162,3.368)}
\gppoint{gp mark 7}{(5.173,3.368)}
\gppoint{gp mark 7}{(5.184,3.369)}
\gppoint{gp mark 7}{(5.195,3.369)}
\gppoint{gp mark 7}{(5.206,3.369)}
\gppoint{gp mark 7}{(5.216,3.369)}
\gppoint{gp mark 7}{(5.227,3.368)}
\gppoint{gp mark 7}{(5.238,3.368)}
\gppoint{gp mark 7}{(5.249,3.368)}
\gppoint{gp mark 7}{(5.260,3.368)}
\gppoint{gp mark 7}{(5.271,3.368)}
\gppoint{gp mark 7}{(5.282,3.368)}
\gppoint{gp mark 7}{(5.293,3.368)}
\gppoint{gp mark 7}{(5.304,3.369)}
\gppoint{gp mark 7}{(5.315,3.369)}
\gppoint{gp mark 7}{(5.326,3.369)}
\gppoint{gp mark 7}{(5.337,3.369)}
\gppoint{gp mark 7}{(5.348,3.369)}
\gppoint{gp mark 7}{(5.359,3.368)}
\gppoint{gp mark 7}{(5.370,3.368)}
\gppoint{gp mark 7}{(5.381,3.368)}
\gppoint{gp mark 7}{(5.392,3.369)}
\gppoint{gp mark 7}{(5.403,3.369)}
\gppoint{gp mark 7}{(5.414,3.369)}
\gppoint{gp mark 7}{(5.425,3.369)}
\gppoint{gp mark 7}{(5.436,3.369)}
\gppoint{gp mark 7}{(5.447,3.369)}
\gppoint{gp mark 7}{(5.458,3.369)}
\gppoint{gp mark 7}{(5.469,3.369)}
\gppoint{gp mark 7}{(5.480,3.369)}
\gppoint{gp mark 7}{(5.491,3.369)}
\gppoint{gp mark 7}{(5.502,3.369)}
\gppoint{gp mark 7}{(5.513,3.369)}
\gppoint{gp mark 7}{(5.524,3.369)}
\gppoint{gp mark 7}{(5.535,3.369)}
\gppoint{gp mark 7}{(5.546,3.368)}
\gppoint{gp mark 7}{(5.557,3.368)}
\gppoint{gp mark 7}{(5.568,3.369)}
\gppoint{gp mark 7}{(5.579,3.369)}
\gppoint{gp mark 7}{(5.590,3.369)}
\gppoint{gp mark 7}{(5.601,3.369)}
\gppoint{gp mark 7}{(5.612,3.369)}
\gppoint{gp mark 7}{(5.623,3.369)}
\gppoint{gp mark 7}{(5.634,3.369)}
\gppoint{gp mark 7}{(5.645,3.369)}
\gppoint{gp mark 7}{(5.656,3.368)}
\gppoint{gp mark 7}{(5.667,3.368)}
\gppoint{gp mark 7}{(5.678,3.368)}
\gppoint{gp mark 7}{(5.689,3.369)}
\gppoint{gp mark 7}{(5.700,3.369)}
\gppoint{gp mark 7}{(5.711,3.369)}
\gppoint{gp mark 7}{(5.722,3.369)}
\gppoint{gp mark 7}{(5.733,3.369)}
\gppoint{gp mark 7}{(5.744,3.369)}
\gppoint{gp mark 7}{(5.755,3.369)}
\gppoint{gp mark 7}{(5.766,3.369)}
\gppoint{gp mark 7}{(5.777,3.369)}
\gppoint{gp mark 7}{(5.788,3.369)}
\gppoint{gp mark 7}{(5.799,3.369)}
\gppoint{gp mark 7}{(5.810,3.369)}
\gppoint{gp mark 7}{(5.821,3.369)}
\gppoint{gp mark 7}{(5.832,3.369)}
\gppoint{gp mark 7}{(5.843,3.369)}
\gppoint{gp mark 7}{(5.854,3.370)}
\gppoint{gp mark 7}{(5.865,3.369)}
\gppoint{gp mark 7}{(5.876,3.369)}
\gppoint{gp mark 7}{(5.887,3.369)}
\gppoint{gp mark 7}{(5.898,3.369)}
\gppoint{gp mark 7}{(5.909,3.369)}
\gppoint{gp mark 7}{(5.920,3.369)}
\gppoint{gp mark 7}{(5.931,3.369)}
\gppoint{gp mark 7}{(5.942,3.369)}
\gppoint{gp mark 7}{(5.953,3.369)}
\gppoint{gp mark 7}{(5.964,3.369)}
\gppoint{gp mark 7}{(5.975,3.369)}
\gppoint{gp mark 7}{(5.986,3.369)}
\gppoint{gp mark 7}{(5.997,3.369)}
\gppoint{gp mark 7}{(6.008,3.369)}
\gppoint{gp mark 7}{(6.019,3.369)}
\gppoint{gp mark 7}{(6.030,3.369)}
\gppoint{gp mark 7}{(6.041,3.369)}
\gppoint{gp mark 7}{(6.052,3.369)}
\gppoint{gp mark 7}{(6.063,3.369)}
\gppoint{gp mark 7}{(6.074,3.369)}
\gppoint{gp mark 7}{(6.085,3.369)}
\gppoint{gp mark 7}{(6.096,3.369)}
\gppoint{gp mark 7}{(6.107,3.369)}
\gppoint{gp mark 7}{(6.118,3.369)}
\gppoint{gp mark 7}{(6.128,3.369)}
\gppoint{gp mark 7}{(6.139,3.369)}
\gppoint{gp mark 7}{(6.150,3.369)}
\gppoint{gp mark 7}{(6.161,3.369)}
\gppoint{gp mark 7}{(6.172,3.369)}
\gppoint{gp mark 7}{(6.183,3.369)}
\gppoint{gp mark 7}{(6.194,3.369)}
\gppoint{gp mark 7}{(6.205,3.369)}
\gppoint{gp mark 7}{(6.216,3.369)}
\gppoint{gp mark 7}{(6.227,3.369)}
\gppoint{gp mark 7}{(6.238,3.369)}
\gppoint{gp mark 7}{(6.249,3.369)}
\gppoint{gp mark 7}{(6.260,3.369)}
\gppoint{gp mark 7}{(6.271,3.369)}
\gppoint{gp mark 7}{(6.282,3.369)}
\gppoint{gp mark 7}{(6.293,3.369)}
\gppoint{gp mark 7}{(6.304,3.369)}
\gppoint{gp mark 7}{(6.315,3.369)}
\gppoint{gp mark 7}{(6.326,3.369)}
\gppoint{gp mark 7}{(6.337,3.369)}
\gppoint{gp mark 7}{(6.348,3.369)}
\gppoint{gp mark 7}{(6.359,3.369)}
\gppoint{gp mark 7}{(6.370,3.369)}
\gppoint{gp mark 7}{(6.381,3.369)}
\gppoint{gp mark 7}{(6.392,3.369)}
\gppoint{gp mark 7}{(6.403,3.369)}
\gppoint{gp mark 7}{(6.414,3.368)}
\gppoint{gp mark 7}{(6.425,3.368)}
\gppoint{gp mark 7}{(6.436,3.368)}
\gppoint{gp mark 7}{(6.447,3.368)}
\gppoint{gp mark 7}{(6.458,3.368)}
\gppoint{gp mark 7}{(6.469,3.368)}
\gppoint{gp mark 7}{(6.480,3.368)}
\gppoint{gp mark 7}{(6.491,3.368)}
\gppoint{gp mark 7}{(6.502,3.368)}
\gppoint{gp mark 7}{(6.513,3.368)}
\gppoint{gp mark 7}{(6.524,3.368)}
\gppoint{gp mark 7}{(6.535,3.368)}
\gppoint{gp mark 7}{(6.546,3.367)}
\gppoint{gp mark 7}{(6.557,3.367)}
\gppoint{gp mark 7}{(6.568,3.367)}
\gppoint{gp mark 7}{(6.579,3.367)}
\gppoint{gp mark 7}{(6.590,3.367)}
\gppoint{gp mark 7}{(6.601,3.367)}
\gppoint{gp mark 7}{(6.612,3.366)}
\gppoint{gp mark 7}{(6.623,3.366)}
\gppoint{gp mark 7}{(6.634,3.366)}
\gppoint{gp mark 7}{(6.645,3.366)}
\gppoint{gp mark 7}{(6.656,3.366)}
\gppoint{gp mark 7}{(6.667,3.366)}
\gppoint{gp mark 7}{(6.678,3.366)}
\gppoint{gp mark 7}{(6.689,3.366)}
\gppoint{gp mark 7}{(6.700,3.365)}
\gppoint{gp mark 7}{(6.711,3.365)}
\gppoint{gp mark 7}{(6.722,3.365)}
\gppoint{gp mark 7}{(6.733,3.365)}
\gppoint{gp mark 7}{(6.744,3.365)}
\gppoint{gp mark 7}{(6.755,3.364)}
\gppoint{gp mark 7}{(6.766,3.364)}
\gppoint{gp mark 7}{(6.777,3.364)}
\gppoint{gp mark 7}{(6.788,3.364)}
\gppoint{gp mark 7}{(6.799,3.364)}
\gppoint{gp mark 7}{(6.810,3.364)}
\gppoint{gp mark 7}{(6.821,3.364)}
\gppoint{gp mark 7}{(6.832,3.364)}
\gppoint{gp mark 7}{(6.843,3.363)}
\gppoint{gp mark 7}{(6.854,3.363)}
\gppoint{gp mark 7}{(6.865,3.363)}
\gppoint{gp mark 7}{(6.876,3.363)}
\gppoint{gp mark 7}{(6.887,3.363)}
\gppoint{gp mark 7}{(6.898,3.363)}
\gppoint{gp mark 7}{(6.909,3.363)}
\gppoint{gp mark 7}{(6.920,3.362)}
\gppoint{gp mark 7}{(6.931,3.362)}
\gppoint{gp mark 7}{(6.942,3.362)}
\gppoint{gp mark 7}{(6.953,3.361)}
\gppoint{gp mark 7}{(6.964,3.361)}
\gppoint{gp mark 7}{(6.975,3.361)}
\gppoint{gp mark 7}{(6.986,3.361)}
\gppoint{gp mark 7}{(6.997,3.361)}
\gppoint{gp mark 7}{(7.008,3.361)}
\gppoint{gp mark 7}{(7.019,3.361)}
\gppoint{gp mark 7}{(7.030,3.360)}
\gppoint{gp mark 7}{(7.040,3.360)}
\gppoint{gp mark 7}{(7.051,3.360)}
\gppoint{gp mark 7}{(7.062,3.360)}
\gppoint{gp mark 7}{(7.073,3.360)}
\gppoint{gp mark 7}{(7.084,3.359)}
\gppoint{gp mark 7}{(7.095,3.359)}
\gppoint{gp mark 7}{(7.106,3.359)}
\gppoint{gp mark 7}{(7.117,3.358)}
\gppoint{gp mark 7}{(7.128,3.358)}
\gppoint{gp mark 7}{(7.139,3.357)}
\gppoint{gp mark 7}{(7.150,3.357)}
\gppoint{gp mark 7}{(7.161,3.356)}
\gppoint{gp mark 7}{(7.172,3.355)}
\gppoint{gp mark 7}{(7.183,3.354)}
\gppoint{gp mark 7}{(7.194,3.354)}
\gppoint{gp mark 7}{(7.205,3.353)}
\gppoint{gp mark 7}{(7.216,3.353)}
\gppoint{gp mark 7}{(7.227,3.352)}
\gppoint{gp mark 7}{(7.238,3.352)}
\gppoint{gp mark 7}{(7.249,3.352)}
\gppoint{gp mark 7}{(7.260,3.352)}
\gppoint{gp mark 7}{(7.271,3.352)}
\gppoint{gp mark 7}{(7.282,3.352)}
\gppoint{gp mark 7}{(7.293,3.352)}
\gppoint{gp mark 7}{(7.304,3.353)}
\gppoint{gp mark 7}{(7.315,3.353)}
\gppoint{gp mark 7}{(7.326,3.354)}
\gppoint{gp mark 7}{(7.337,3.354)}
\gppoint{gp mark 7}{(7.348,3.356)}
\gppoint{gp mark 7}{(7.359,3.356)}
\gppoint{gp mark 7}{(7.370,3.356)}
\gppoint{gp mark 7}{(7.381,3.357)}
\gppoint{gp mark 7}{(7.392,3.358)}
\gppoint{gp mark 7}{(7.403,3.365)}
\gppoint{gp mark 7}{(7.414,3.379)}
\gppoint{gp mark 7}{(7.425,3.393)}
\gppoint{gp mark 7}{(7.436,3.399)}
\gppoint{gp mark 7}{(7.447,3.400)}
\gppoint{gp mark 7}{(7.458,3.401)}
\gppoint{gp mark 7}{(7.469,3.400)}
\gppoint{gp mark 7}{(7.480,3.389)}
\gppoint{gp mark 7}{(7.491,3.301)}
\gppoint{gp mark 7}{(7.502,2.807)}
\gppoint{gp mark 7}{(7.513,1.676)}
\gpcolor{rgb color={0.000,0.000,0.000}}
\gpsetlinetype{gp lt plot 0}
\gpsetlinewidth{4.00}
\draw[gp path] (2.427,2.420)--(3.533,2.420);
\draw[gp path] (3.533,2.420)--(4.326,2.420);
\draw[gp path] (4.326,3.596)--(7.510,3.596);
\draw[gp path] (1.200,4.130)--(1.206,4.121)--(1.212,4.112)--(1.218,4.102)--(1.224,4.093)%
  --(1.230,4.084)--(1.236,4.075)--(1.242,4.066)--(1.248,4.057)--(1.254,4.047)--(1.260,4.038)%
  --(1.267,4.029)--(1.273,4.020)--(1.279,4.011)--(1.285,4.002)--(1.291,3.993)--(1.297,3.984)%
  --(1.303,3.974)--(1.309,3.965)--(1.315,3.956)--(1.321,3.947)--(1.327,3.938)--(1.333,3.929)%
  --(1.339,3.920)--(1.346,3.911)--(1.352,3.902)--(1.358,3.893)--(1.364,3.884)--(1.370,3.875)%
  --(1.376,3.866)--(1.382,3.857)--(1.388,3.848)--(1.394,3.839)--(1.400,3.830)--(1.406,3.821)%
  --(1.412,3.812)--(1.418,3.803)--(1.425,3.794)--(1.431,3.785)--(1.437,3.776)--(1.443,3.767)%
  --(1.449,3.758)--(1.455,3.749)--(1.461,3.740)--(1.467,3.731)--(1.473,3.722)--(1.479,3.713)%
  --(1.485,3.704)--(1.491,3.695)--(1.497,3.686)--(1.504,3.677)--(1.510,3.668)--(1.516,3.660)%
  --(1.522,3.651)--(1.528,3.642)--(1.534,3.633)--(1.540,3.624)--(1.546,3.615)--(1.552,3.606)%
  --(1.558,3.597)--(1.564,3.589)--(1.570,3.580)--(1.576,3.571)--(1.583,3.562)--(1.589,3.553)%
  --(1.595,3.545)--(1.601,3.536)--(1.607,3.527)--(1.613,3.518)--(1.619,3.509)--(1.625,3.501)%
  --(1.631,3.492)--(1.637,3.483)--(1.643,3.474)--(1.649,3.465)--(1.656,3.457)--(1.662,3.448)%
  --(1.668,3.439)--(1.674,3.431)--(1.680,3.422)--(1.686,3.413)--(1.692,3.404)--(1.698,3.396)%
  --(1.704,3.387)--(1.710,3.378)--(1.716,3.370)--(1.722,3.361)--(1.728,3.352)--(1.735,3.344)%
  --(1.741,3.335)--(1.747,3.326)--(1.753,3.318)--(1.759,3.309)--(1.765,3.300)--(1.771,3.292)%
  --(1.777,3.283)--(1.783,3.274)--(1.789,3.266)--(1.795,3.257)--(1.801,3.249)--(1.807,3.240)%
  --(1.814,3.232)--(1.820,3.223)--(1.826,3.214)--(1.832,3.206)--(1.838,3.197)--(1.844,3.189)%
  --(1.850,3.180)--(1.856,3.172)--(1.862,3.163)--(1.868,3.155)--(1.874,3.146)--(1.880,3.138)%
  --(1.886,3.129)--(1.893,3.121)--(1.899,3.112)--(1.905,3.104)--(1.911,3.095)--(1.917,3.087)%
  --(1.923,3.078)--(1.929,3.070)--(1.935,3.061)--(1.941,3.053)--(1.947,3.045)--(1.953,3.036)%
  --(1.959,3.028)--(1.965,3.019)--(1.972,3.011)--(1.978,3.003)--(1.984,2.994)--(1.990,2.986)%
  --(1.996,2.978)--(2.002,2.969)--(2.008,2.961)--(2.014,2.952)--(2.020,2.944)--(2.026,2.936)%
  --(2.032,2.928)--(2.038,2.919)--(2.044,2.911)--(2.051,2.903)--(2.057,2.894)--(2.063,2.886)%
  --(2.069,2.878)--(2.075,2.869)--(2.081,2.861)--(2.087,2.853)--(2.093,2.845)--(2.099,2.837)%
  --(2.105,2.828)--(2.111,2.820)--(2.117,2.812)--(2.124,2.804)--(2.130,2.795)--(2.136,2.787)%
  --(2.142,2.779)--(2.148,2.771)--(2.154,2.763)--(2.160,2.755)--(2.166,2.746)--(2.172,2.738)%
  --(2.178,2.730)--(2.184,2.722)--(2.190,2.714)--(2.196,2.706)--(2.203,2.698)--(2.209,2.690)%
  --(2.215,2.682)--(2.221,2.673)--(2.227,2.665)--(2.233,2.657)--(2.239,2.649)--(2.245,2.641)%
  --(2.251,2.633)--(2.257,2.625)--(2.263,2.617)--(2.269,2.609)--(2.275,2.601)--(2.282,2.593)%
  --(2.288,2.585)--(2.294,2.577)--(2.300,2.569)--(2.306,2.561)--(2.312,2.553)--(2.318,2.545)%
  --(2.324,2.537)--(2.330,2.529)--(2.336,2.522)--(2.342,2.514)--(2.348,2.506)--(2.354,2.498)%
  --(2.361,2.490)--(2.367,2.482)--(2.373,2.474)--(2.379,2.466)--(2.385,2.459)--(2.391,2.451)%
  --(2.397,2.443)--(2.403,2.435)--(2.409,2.427)--(2.415,2.419)--(2.421,2.412)--(2.427,2.420);
\draw[gp path] (4.326,2.420)--(4.326,3.596);
\draw[gp path] (7.510,3.596)--(7.510,0.985);
\gpsetlinetype{gp lt plot 2}
\draw[gp path] (1.206,2.420)--(1.217,2.420)--(1.228,2.420)--(1.239,2.420)--(1.250,2.420)%
  --(1.261,2.420)--(1.272,2.420)--(1.283,2.420)--(1.294,2.420)--(1.305,2.420)--(1.316,2.420)%
  --(1.327,2.420)--(1.338,2.420)--(1.349,2.420)--(1.360,2.420)--(1.371,2.420)--(1.382,2.420)%
  --(1.393,2.420)--(1.404,2.420)--(1.415,2.420)--(1.426,2.420)--(1.437,2.420)--(1.448,2.420)%
  --(1.459,2.420)--(1.470,2.420)--(1.481,2.420)--(1.492,2.420)--(1.503,2.420)--(1.514,2.420)%
  --(1.525,2.420)--(1.536,2.420)--(1.547,2.420)--(1.558,2.420)--(1.568,2.420)--(1.579,2.420)%
  --(1.590,2.420)--(1.601,2.420)--(1.612,2.420)--(1.623,2.420)--(1.634,2.420)--(1.645,2.420)%
  --(1.656,2.420)--(1.667,2.420)--(1.678,2.420)--(1.689,2.420)--(1.700,2.420)--(1.711,2.420)%
  --(1.722,2.420)--(1.733,2.420)--(1.744,2.420)--(1.755,2.420)--(1.766,2.420)--(1.777,2.420)%
  --(1.788,2.420)--(1.799,2.420)--(1.810,2.420)--(1.821,2.420)--(1.832,2.420)--(1.843,2.420)%
  --(1.854,2.420)--(1.865,2.420)--(1.876,2.420)--(1.887,2.420)--(1.898,2.420)--(1.909,2.420)%
  --(1.920,2.420)--(1.931,2.420)--(1.942,2.420)--(1.953,2.420)--(1.964,2.420)--(1.975,2.420)%
  --(1.986,2.420)--(1.997,2.420)--(2.008,2.420)--(2.019,2.420)--(2.030,2.420)--(2.041,2.420)%
  --(2.052,2.420)--(2.063,2.420)--(2.074,2.420)--(2.085,2.420)--(2.096,2.420)--(2.107,2.420)%
  --(2.118,2.420)--(2.129,2.420)--(2.140,2.420)--(2.151,2.420)--(2.162,2.420)--(2.173,2.420)%
  --(2.184,2.420)--(2.195,2.420)--(2.206,2.420)--(2.217,2.420)--(2.228,2.420)--(2.239,2.420)%
  --(2.250,2.420)--(2.261,2.420)--(2.272,2.420)--(2.283,2.420)--(2.294,2.420)--(2.305,2.420)%
  --(2.316,2.420)--(2.327,2.420)--(2.338,2.420)--(2.349,2.420)--(2.360,2.420)--(2.371,2.419)%
  --(2.382,2.417)--(2.393,2.413)--(2.404,2.408)--(2.415,2.401)--(2.426,2.391)--(2.437,2.380)%
  --(2.448,2.368)--(2.459,2.355)--(2.470,2.341)--(2.480,2.327)--(2.491,2.313)--(2.502,2.299)%
  --(2.513,2.285)--(2.524,2.271)--(2.535,2.257)--(2.546,2.242)--(2.557,2.228)--(2.568,2.214)%
  --(2.579,2.199)--(2.590,2.185)--(2.601,2.170)--(2.612,2.156)--(2.623,2.141)--(2.634,2.127)%
  --(2.645,2.112)--(2.656,2.098)--(2.667,2.083)--(2.678,2.069)--(2.689,2.054)--(2.700,2.040)%
  --(2.711,2.025)--(2.722,2.011)--(2.733,1.996)--(2.744,1.982)--(2.755,1.967)--(2.766,1.953)%
  --(2.777,1.939)--(2.788,1.924)--(2.799,1.910)--(2.810,1.895)--(2.821,1.881)--(2.832,1.867)%
  --(2.843,1.853)--(2.854,1.838)--(2.865,1.824)--(2.876,1.810)--(2.887,1.796)--(2.898,1.781)%
  --(2.909,1.767)--(2.920,1.753)--(2.931,1.739)--(2.942,1.725)--(2.953,1.711)--(2.964,1.697)%
  --(2.975,1.683)--(2.986,1.668)--(2.997,1.654)--(3.008,1.639)--(3.019,1.624)--(3.030,1.609)%
  --(3.041,1.592)--(3.052,1.576)--(3.063,1.562)--(3.074,1.556)--(3.085,1.555)--(3.096,1.555)%
  --(3.107,1.559)--(3.118,1.580)--(3.129,1.629)--(3.140,1.679)--(3.151,1.695)--(3.162,1.697)%
  --(3.173,1.697)--(3.184,1.697)--(3.195,1.697)--(3.206,1.697)--(3.217,1.697)--(3.228,1.697)%
  --(3.239,1.697)--(3.250,1.695)--(3.261,1.693)--(3.272,1.691)--(3.283,1.690)--(3.294,1.689)%
  --(3.305,1.689)--(3.316,1.688)--(3.327,1.688)--(3.338,1.688)--(3.349,1.688)--(3.360,1.688)%
  --(3.371,1.688)--(3.382,1.689)--(3.392,1.689)--(3.403,1.691)--(3.414,1.692)--(3.425,1.692)%
  --(3.436,1.693)--(3.447,1.694)--(3.458,1.695)--(3.469,1.695)--(3.480,1.696)--(3.491,1.700)%
  --(3.502,1.789)--(3.513,3.036)--(3.524,5.046)--(3.535,5.082)--(3.546,4.815)--(3.557,4.738)%
  --(3.568,4.710)--(3.579,4.653)--(3.590,4.599)--(3.601,4.570)--(3.612,4.538)--(3.623,4.495)%
  --(3.634,4.466)--(3.645,4.442)--(3.656,4.407)--(3.667,4.375)--(3.678,4.351)--(3.689,4.321)%
  --(3.700,4.289)--(3.711,4.264)--(3.722,4.236)--(3.733,4.206)--(3.744,4.178)--(3.755,4.152)%
  --(3.766,4.123)--(3.777,4.095)--(3.788,4.068)--(3.799,4.041)--(3.810,4.012)--(3.821,3.986)%
  --(3.832,3.959)--(3.843,3.931)--(3.854,3.904)--(3.865,3.877)--(3.876,3.850)--(3.887,3.822)%
  --(3.898,3.796)--(3.909,3.769)--(3.920,3.741)--(3.931,3.715)--(3.942,3.688)--(3.953,3.661)%
  --(3.964,3.634)--(3.975,3.608)--(3.986,3.581)--(3.997,3.554)--(4.008,3.528)--(4.019,3.501)%
  --(4.030,3.475)--(4.041,3.448)--(4.052,3.422)--(4.063,3.396)--(4.074,3.369)--(4.085,3.343)%
  --(4.096,3.317)--(4.107,3.291)--(4.118,3.265)--(4.129,3.239)--(4.140,3.212)--(4.151,3.186)%
  --(4.162,3.160)--(4.173,3.134)--(4.184,3.108)--(4.195,3.082)--(4.206,3.056)--(4.217,3.030)%
  --(4.228,3.005)--(4.239,2.979)--(4.250,2.953)--(4.261,2.927)--(4.272,2.902)--(4.283,2.876)%
  --(4.294,2.851)--(4.304,2.825)--(4.315,2.800)--(4.326,2.774)--(4.337,2.749)--(4.348,2.723)%
  --(4.359,2.698)--(4.370,2.673)--(4.381,2.648)--(4.392,2.623)--(4.403,2.598)--(4.414,2.573)%
  --(4.425,2.548)--(4.436,2.524)--(4.447,2.500)--(4.458,2.475)--(4.469,2.451)--(4.480,2.427)%
  --(4.491,2.404)--(4.502,2.381)--(4.513,2.358)--(4.524,2.336)--(4.535,2.315)--(4.546,2.294)%
  --(4.557,2.275)--(4.568,2.257)--(4.579,2.242)--(4.590,2.231)--(4.601,2.223)--(4.612,2.221)%
  --(4.623,2.220)--(4.634,2.220)--(4.645,2.220)--(4.656,2.220)--(4.667,2.220)--(4.678,2.220)%
  --(4.689,2.220)--(4.700,2.220)--(4.711,2.220)--(4.722,2.220)--(4.733,2.220)--(4.744,2.220)%
  --(4.755,2.220)--(4.766,2.220)--(4.777,2.220)--(4.788,2.220)--(4.799,2.220)--(4.810,2.220)%
  --(4.821,2.220)--(4.832,2.220)--(4.843,2.220)--(4.854,2.220)--(4.865,2.220)--(4.876,2.220)%
  --(4.887,2.220)--(4.898,2.220)--(4.909,2.220)--(4.920,2.220)--(4.931,2.220)--(4.942,2.220)%
  --(4.953,2.220)--(4.964,2.220)--(4.975,2.220)--(4.986,2.220)--(4.997,2.220)--(5.008,2.220)%
  --(5.019,2.220)--(5.030,2.220)--(5.041,2.220)--(5.052,2.220)--(5.063,2.220)--(5.074,2.220)%
  --(5.085,2.220)--(5.096,2.220)--(5.107,2.220)--(5.118,2.220)--(5.129,2.220)--(5.140,2.220)%
  --(5.151,2.220)--(5.162,2.220)--(5.173,2.220)--(5.184,2.220)--(5.195,2.220)--(5.206,2.220)%
  --(5.216,2.220)--(5.227,2.220)--(5.238,2.220)--(5.249,2.220)--(5.260,2.220)--(5.271,2.220)%
  --(5.282,2.220)--(5.293,2.220)--(5.304,2.220)--(5.315,2.220)--(5.326,2.220)--(5.337,2.220)%
  --(5.348,2.220)--(5.359,2.220)--(5.370,2.220)--(5.381,2.220)--(5.392,2.220)--(5.403,2.220)%
  --(5.414,2.220)--(5.425,2.220)--(5.436,2.220)--(5.447,2.220)--(5.458,2.220)--(5.469,2.220)%
  --(5.480,2.220)--(5.491,2.220)--(5.502,2.220)--(5.513,2.220)--(5.524,2.220)--(5.535,2.220)%
  --(5.546,2.220)--(5.557,2.220)--(5.568,2.220)--(5.579,2.220)--(5.590,2.220)--(5.601,2.220)%
  --(5.612,2.220)--(5.623,2.220)--(5.634,2.220)--(5.645,2.220)--(5.656,2.220)--(5.667,2.220)%
  --(5.678,2.220)--(5.689,2.220)--(5.700,2.220)--(5.711,2.220)--(5.722,2.220)--(5.733,2.220)%
  --(5.744,2.220)--(5.755,2.220)--(5.766,2.220)--(5.777,2.220)--(5.788,2.220)--(5.799,2.220)%
  --(5.810,2.220)--(5.821,2.220)--(5.832,2.220)--(5.843,2.220)--(5.854,2.220)--(5.865,2.220)%
  --(5.876,2.220)--(5.887,2.220)--(5.898,2.220)--(5.909,2.220)--(5.920,2.220)--(5.931,2.220)%
  --(5.942,2.220)--(5.953,2.219)--(5.964,2.219)--(5.975,2.219)--(5.986,2.219)--(5.997,2.219)%
  --(6.008,2.219)--(6.019,2.219)--(6.030,2.220)--(6.041,2.220)--(6.052,2.220)--(6.063,2.220)%
  --(6.074,2.220)--(6.085,2.220)--(6.096,2.220)--(6.107,2.220)--(6.118,2.220)--(6.128,2.220)%
  --(6.139,2.220)--(6.150,2.220)--(6.161,2.220)--(6.172,2.220)--(6.183,2.220)--(6.194,2.220)%
  --(6.205,2.220)--(6.216,2.220)--(6.227,2.219)--(6.238,2.219)--(6.249,2.219)--(6.260,2.219)%
  --(6.271,2.219)--(6.282,2.219)--(6.293,2.219)--(6.304,2.219)--(6.315,2.219)--(6.326,2.219)%
  --(6.337,2.219)--(6.348,2.219)--(6.359,2.219)--(6.370,2.219)--(6.381,2.219)--(6.392,2.219)%
  --(6.403,2.219)--(6.414,2.219)--(6.425,2.219)--(6.436,2.219)--(6.447,2.219)--(6.458,2.219)%
  --(6.469,2.219)--(6.480,2.219)--(6.491,2.219)--(6.502,2.219)--(6.513,2.219)--(6.524,2.219)%
  --(6.535,2.219)--(6.546,2.219)--(6.557,2.219)--(6.568,2.219)--(6.579,2.219)--(6.590,2.218)%
  --(6.601,2.218)--(6.612,2.218)--(6.623,2.218)--(6.634,2.218)--(6.645,2.218)--(6.656,2.218)%
  --(6.667,2.218)--(6.678,2.218)--(6.689,2.218)--(6.700,2.218)--(6.711,2.218)--(6.722,2.218)%
  --(6.733,2.218)--(6.744,2.218)--(6.755,2.218)--(6.766,2.218)--(6.777,2.218)--(6.788,2.218)%
  --(6.799,2.218)--(6.810,2.218)--(6.821,2.218)--(6.832,2.218)--(6.843,2.218)--(6.854,2.218)%
  --(6.865,2.218)--(6.876,2.217)--(6.887,2.217)--(6.898,2.217)--(6.909,2.217)--(6.920,2.216)%
  --(6.931,2.216)--(6.942,2.216)--(6.953,2.215)--(6.964,2.215)--(6.975,2.215)--(6.986,2.215)%
  --(6.997,2.215)--(7.008,2.214)--(7.019,2.214)--(7.030,2.214)--(7.040,2.214)--(7.051,2.214)%
  --(7.062,2.214)--(7.073,2.214)--(7.084,2.214)--(7.095,2.214)--(7.106,2.213)--(7.117,2.213)%
  --(7.128,2.213)--(7.139,2.213)--(7.150,2.213)--(7.161,2.213)--(7.172,2.213)--(7.183,2.213)%
  --(7.194,2.212)--(7.205,2.212)--(7.216,2.212)--(7.227,2.211)--(7.238,2.211)--(7.249,2.210)%
  --(7.260,2.209)--(7.271,2.209)--(7.282,2.208)--(7.293,2.207)--(7.304,2.207)--(7.315,2.206)%
  --(7.326,2.205)--(7.337,2.205)--(7.348,2.205)--(7.359,2.205)--(7.370,2.205)--(7.381,2.205)%
  --(7.392,2.205)--(7.403,2.205)--(7.414,2.206)--(7.425,2.206)--(7.436,2.207)--(7.447,2.207)%
  --(7.458,2.207)--(7.469,2.207)--(7.480,2.207)--(7.491,2.207)--(7.502,2.207)--(7.513,2.206)%
  --(7.524,2.205)--(7.535,2.203)--(7.546,2.201)--(7.557,2.200)--(7.568,2.198)--(7.579,2.196)%
  --(7.590,2.194)--(7.601,2.191)--(7.612,2.189)--(7.623,2.186)--(7.634,2.185)--(7.645,2.184)%
  --(7.656,2.184)--(7.667,2.184)--(7.678,2.184)--(7.689,2.184)--(7.700,2.184)--(7.711,2.184)%
  --(7.722,2.186)--(7.733,2.192)--(7.744,2.198)--(7.755,2.200)--(7.766,2.201)--(7.777,2.201)%
  --(7.788,2.200)--(7.799,2.192)--(7.810,2.157)--(7.821,2.107)--(7.832,2.077)--(7.843,2.070)%
  --(7.854,2.069)--(7.865,2.070)--(7.876,2.081)--(7.887,2.137)--(7.898,2.238)--(7.909,2.338)%
  --(7.920,2.406)--(7.931,2.442)--(7.942,2.457);
\node[gp node left,font={\fontsize{10pt}{12pt}\selectfont}] at (1.421,5.244) {\LARGE $\rho$};
\node[gp node left,font={\fontsize{10pt}{12pt}\selectfont}] at (6.147,5.244) {\large $\alpha = \pi$};
%% coordinates of the plot area
\gpdefrectangularnode{gp plot 1}{\pgfpoint{1.196cm}{0.985cm}}{\pgfpoint{7.947cm}{5.631cm}}
\end{tikzpicture}
%% gnuplot variables
} & 
\resizebox{0.5\linewidth}{!}{\tikzsetnextfilename{coplanar_a_cwaves_6}\begin{tikzpicture}[gnuplot]
%% generated with GNUPLOT 4.6p4 (Lua 5.1; terminal rev. 99, script rev. 100)
%% Sat 31 May 2014 03:20:24 PM EDT
\path (0.000,0.000) rectangle (8.500,6.000);
\gpfill{rgb color={1.000,1.000,1.000}} (1.196,0.985)--(7.946,0.985)--(7.946,5.630)--(1.196,5.630)--cycle;
\gpcolor{color=gp lt color border}
\gpsetlinetype{gp lt border}
\gpsetlinewidth{1.00}
\draw[gp path] (1.196,0.985)--(1.196,5.630)--(7.946,5.630)--(7.946,0.985)--cycle;
\gpcolor{color=gp lt color axes}
\gpsetlinetype{gp lt axes}
\gpsetlinewidth{2.00}
\draw[gp path] (1.196,0.985)--(7.947,0.985);
\gpcolor{color=gp lt color border}
\gpsetlinetype{gp lt border}
\draw[gp path] (1.196,0.985)--(1.268,0.985);
\draw[gp path] (7.947,0.985)--(7.875,0.985);
\gpcolor{rgb color={0.000,0.000,0.000}}
\node[gp node right,font={\fontsize{10pt}{12pt}\selectfont}] at (1.012,0.985) {-0.4};
\gpcolor{color=gp lt color axes}
\gpsetlinetype{gp lt axes}
\draw[gp path] (1.196,1.759)--(7.947,1.759);
\gpcolor{color=gp lt color border}
\gpsetlinetype{gp lt border}
\draw[gp path] (1.196,1.759)--(1.268,1.759);
\draw[gp path] (7.947,1.759)--(7.875,1.759);
\gpcolor{rgb color={0.000,0.000,0.000}}
\node[gp node right,font={\fontsize{10pt}{12pt}\selectfont}] at (1.012,1.759) {-0.2};
\gpcolor{color=gp lt color axes}
\gpsetlinetype{gp lt axes}
\draw[gp path] (1.196,2.534)--(7.947,2.534);
\gpcolor{color=gp lt color border}
\gpsetlinetype{gp lt border}
\draw[gp path] (1.196,2.534)--(1.268,2.534);
\draw[gp path] (7.947,2.534)--(7.875,2.534);
\gpcolor{rgb color={0.000,0.000,0.000}}
\node[gp node right,font={\fontsize{10pt}{12pt}\selectfont}] at (1.012,2.534) {0};
\gpcolor{color=gp lt color axes}
\gpsetlinetype{gp lt axes}
\draw[gp path] (1.196,3.308)--(7.947,3.308);
\gpcolor{color=gp lt color border}
\gpsetlinetype{gp lt border}
\draw[gp path] (1.196,3.308)--(1.268,3.308);
\draw[gp path] (7.947,3.308)--(7.875,3.308);
\gpcolor{rgb color={0.000,0.000,0.000}}
\node[gp node right,font={\fontsize{10pt}{12pt}\selectfont}] at (1.012,3.308) {0.2};
\gpcolor{color=gp lt color axes}
\gpsetlinetype{gp lt axes}
\draw[gp path] (1.196,4.082)--(7.947,4.082);
\gpcolor{color=gp lt color border}
\gpsetlinetype{gp lt border}
\draw[gp path] (1.196,4.082)--(1.268,4.082);
\draw[gp path] (7.947,4.082)--(7.875,4.082);
\gpcolor{rgb color={0.000,0.000,0.000}}
\node[gp node right,font={\fontsize{10pt}{12pt}\selectfont}] at (1.012,4.082) {0.4};
\gpcolor{color=gp lt color axes}
\gpsetlinetype{gp lt axes}
\draw[gp path] (1.196,4.857)--(7.947,4.857);
\gpcolor{color=gp lt color border}
\gpsetlinetype{gp lt border}
\draw[gp path] (1.196,4.857)--(1.268,4.857);
\draw[gp path] (7.947,4.857)--(7.875,4.857);
\gpcolor{rgb color={0.000,0.000,0.000}}
\node[gp node right,font={\fontsize{10pt}{12pt}\selectfont}] at (1.012,4.857) {0.6};
\gpcolor{color=gp lt color axes}
\gpsetlinetype{gp lt axes}
\draw[gp path] (1.196,5.631)--(7.947,5.631);
\gpcolor{color=gp lt color border}
\gpsetlinetype{gp lt border}
\draw[gp path] (1.196,5.631)--(1.268,5.631);
\draw[gp path] (7.947,5.631)--(7.875,5.631);
\gpcolor{rgb color={0.000,0.000,0.000}}
\node[gp node right,font={\fontsize{10pt}{12pt}\selectfont}] at (1.012,5.631) {0.8};
\gpcolor{color=gp lt color axes}
\gpsetlinetype{gp lt axes}
\draw[gp path] (1.196,0.985)--(1.196,5.631);
\gpcolor{color=gp lt color border}
\gpsetlinetype{gp lt border}
\draw[gp path] (1.196,0.985)--(1.196,1.057);
\draw[gp path] (1.196,5.631)--(1.196,5.559);
\gpcolor{rgb color={0.000,0.000,0.000}}
\node[gp node center,font={\fontsize{10pt}{12pt}\selectfont}] at (1.196,0.677) {0.2};
\gpcolor{color=gp lt color axes}
\gpsetlinetype{gp lt axes}
\draw[gp path] (2.321,0.985)--(2.321,5.631);
\gpcolor{color=gp lt color border}
\gpsetlinetype{gp lt border}
\draw[gp path] (2.321,0.985)--(2.321,1.057);
\draw[gp path] (2.321,5.631)--(2.321,5.559);
\gpcolor{rgb color={0.000,0.000,0.000}}
\node[gp node center,font={\fontsize{10pt}{12pt}\selectfont}] at (2.321,0.677) {0.25};
\gpcolor{color=gp lt color axes}
\gpsetlinetype{gp lt axes}
\draw[gp path] (3.446,0.985)--(3.446,5.631);
\gpcolor{color=gp lt color border}
\gpsetlinetype{gp lt border}
\draw[gp path] (3.446,0.985)--(3.446,1.057);
\draw[gp path] (3.446,5.631)--(3.446,5.559);
\gpcolor{rgb color={0.000,0.000,0.000}}
\node[gp node center,font={\fontsize{10pt}{12pt}\selectfont}] at (3.446,0.677) {0.3};
\gpcolor{color=gp lt color axes}
\gpsetlinetype{gp lt axes}
\draw[gp path] (4.572,0.985)--(4.572,5.631);
\gpcolor{color=gp lt color border}
\gpsetlinetype{gp lt border}
\draw[gp path] (4.572,0.985)--(4.572,1.057);
\draw[gp path] (4.572,5.631)--(4.572,5.559);
\gpcolor{rgb color={0.000,0.000,0.000}}
\node[gp node center,font={\fontsize{10pt}{12pt}\selectfont}] at (4.572,0.677) {0.35};
\gpcolor{color=gp lt color axes}
\gpsetlinetype{gp lt axes}
\draw[gp path] (5.697,0.985)--(5.697,5.631);
\gpcolor{color=gp lt color border}
\gpsetlinetype{gp lt border}
\draw[gp path] (5.697,0.985)--(5.697,1.057);
\draw[gp path] (5.697,5.631)--(5.697,5.559);
\gpcolor{rgb color={0.000,0.000,0.000}}
\node[gp node center,font={\fontsize{10pt}{12pt}\selectfont}] at (5.697,0.677) {0.4};
\gpcolor{color=gp lt color axes}
\gpsetlinetype{gp lt axes}
\draw[gp path] (6.822,0.985)--(6.822,5.631);
\gpcolor{color=gp lt color border}
\gpsetlinetype{gp lt border}
\draw[gp path] (6.822,0.985)--(6.822,1.057);
\draw[gp path] (6.822,5.631)--(6.822,5.559);
\gpcolor{rgb color={0.000,0.000,0.000}}
\node[gp node center,font={\fontsize{10pt}{12pt}\selectfont}] at (6.822,0.677) {0.45};
\gpcolor{color=gp lt color axes}
\gpsetlinetype{gp lt axes}
\draw[gp path] (7.947,0.985)--(7.947,5.631);
\gpcolor{color=gp lt color border}
\gpsetlinetype{gp lt border}
\draw[gp path] (7.947,0.985)--(7.947,1.057);
\draw[gp path] (7.947,5.631)--(7.947,5.559);
\gpcolor{rgb color={0.000,0.000,0.000}}
\node[gp node center,font={\fontsize{10pt}{12pt}\selectfont}] at (7.947,0.677) {0.5};
\gpcolor{color=gp lt color border}
\draw[gp path] (1.196,5.631)--(1.196,0.985)--(7.947,0.985)--(7.947,5.631)--cycle;
\gpcolor{rgb color={0.000,0.000,0.000}}
\node[gp node center,font={\fontsize{10pt}{12pt}\selectfont}] at (4.571,0.215) {\large $x$};
\gpcolor{rgb color={1.000,0.000,0.000}}
\gpsetlinewidth{0.50}
\gpsetpointsize{4.44}
\gppoint{gp mark 7}{(1.206,4.629)}
\gppoint{gp mark 7}{(1.217,4.623)}
\gppoint{gp mark 7}{(1.228,4.618)}
\gppoint{gp mark 7}{(1.239,4.613)}
\gppoint{gp mark 7}{(1.250,4.607)}
\gppoint{gp mark 7}{(1.261,4.602)}
\gppoint{gp mark 7}{(1.272,4.597)}
\gppoint{gp mark 7}{(1.283,4.592)}
\gppoint{gp mark 7}{(1.294,4.586)}
\gppoint{gp mark 7}{(1.305,4.581)}
\gppoint{gp mark 7}{(1.316,4.576)}
\gppoint{gp mark 7}{(1.327,4.570)}
\gppoint{gp mark 7}{(1.338,4.565)}
\gppoint{gp mark 7}{(1.349,4.560)}
\gppoint{gp mark 7}{(1.360,4.554)}
\gppoint{gp mark 7}{(1.371,4.549)}
\gppoint{gp mark 7}{(1.382,4.544)}
\gppoint{gp mark 7}{(1.393,4.539)}
\gppoint{gp mark 7}{(1.404,4.533)}
\gppoint{gp mark 7}{(1.415,4.528)}
\gppoint{gp mark 7}{(1.426,4.523)}
\gppoint{gp mark 7}{(1.437,4.517)}
\gppoint{gp mark 7}{(1.448,4.512)}
\gppoint{gp mark 7}{(1.459,4.507)}
\gppoint{gp mark 7}{(1.470,4.501)}
\gppoint{gp mark 7}{(1.481,4.496)}
\gppoint{gp mark 7}{(1.492,4.491)}
\gppoint{gp mark 7}{(1.503,4.485)}
\gppoint{gp mark 7}{(1.514,4.480)}
\gppoint{gp mark 7}{(1.525,4.475)}
\gppoint{gp mark 7}{(1.536,4.469)}
\gppoint{gp mark 7}{(1.547,4.464)}
\gppoint{gp mark 7}{(1.558,4.459)}
\gppoint{gp mark 7}{(1.568,4.453)}
\gppoint{gp mark 7}{(1.579,4.448)}
\gppoint{gp mark 7}{(1.590,4.443)}
\gppoint{gp mark 7}{(1.601,4.437)}
\gppoint{gp mark 7}{(1.612,4.432)}
\gppoint{gp mark 7}{(1.623,4.427)}
\gppoint{gp mark 7}{(1.634,4.421)}
\gppoint{gp mark 7}{(1.645,4.416)}
\gppoint{gp mark 7}{(1.656,4.410)}
\gppoint{gp mark 7}{(1.667,4.405)}
\gppoint{gp mark 7}{(1.678,4.400)}
\gppoint{gp mark 7}{(1.689,4.394)}
\gppoint{gp mark 7}{(1.700,4.389)}
\gppoint{gp mark 7}{(1.711,4.384)}
\gppoint{gp mark 7}{(1.722,4.378)}
\gppoint{gp mark 7}{(1.733,4.373)}
\gppoint{gp mark 7}{(1.744,4.367)}
\gppoint{gp mark 7}{(1.755,4.362)}
\gppoint{gp mark 7}{(1.766,4.357)}
\gppoint{gp mark 7}{(1.777,4.351)}
\gppoint{gp mark 7}{(1.788,4.346)}
\gppoint{gp mark 7}{(1.799,4.341)}
\gppoint{gp mark 7}{(1.810,4.335)}
\gppoint{gp mark 7}{(1.821,4.330)}
\gppoint{gp mark 7}{(1.832,4.324)}
\gppoint{gp mark 7}{(1.843,4.319)}
\gppoint{gp mark 7}{(1.854,4.313)}
\gppoint{gp mark 7}{(1.865,4.308)}
\gppoint{gp mark 7}{(1.876,4.303)}
\gppoint{gp mark 7}{(1.887,4.297)}
\gppoint{gp mark 7}{(1.898,4.292)}
\gppoint{gp mark 7}{(1.909,4.286)}
\gppoint{gp mark 7}{(1.920,4.281)}
\gppoint{gp mark 7}{(1.931,4.275)}
\gppoint{gp mark 7}{(1.942,4.270)}
\gppoint{gp mark 7}{(1.953,4.265)}
\gppoint{gp mark 7}{(1.964,4.259)}
\gppoint{gp mark 7}{(1.975,4.254)}
\gppoint{gp mark 7}{(1.986,4.248)}
\gppoint{gp mark 7}{(1.997,4.243)}
\gppoint{gp mark 7}{(2.008,4.237)}
\gppoint{gp mark 7}{(2.019,4.232)}
\gppoint{gp mark 7}{(2.030,4.226)}
\gppoint{gp mark 7}{(2.041,4.221)}
\gppoint{gp mark 7}{(2.052,4.215)}
\gppoint{gp mark 7}{(2.063,4.210)}
\gppoint{gp mark 7}{(2.074,4.204)}
\gppoint{gp mark 7}{(2.085,4.199)}
\gppoint{gp mark 7}{(2.096,4.193)}
\gppoint{gp mark 7}{(2.107,4.188)}
\gppoint{gp mark 7}{(2.118,4.182)}
\gppoint{gp mark 7}{(2.129,4.177)}
\gppoint{gp mark 7}{(2.140,4.171)}
\gppoint{gp mark 7}{(2.151,4.166)}
\gppoint{gp mark 7}{(2.162,4.160)}
\gppoint{gp mark 7}{(2.173,4.155)}
\gppoint{gp mark 7}{(2.184,4.149)}
\gppoint{gp mark 7}{(2.195,4.144)}
\gppoint{gp mark 7}{(2.206,4.138)}
\gppoint{gp mark 7}{(2.217,4.133)}
\gppoint{gp mark 7}{(2.228,4.127)}
\gppoint{gp mark 7}{(2.239,4.121)}
\gppoint{gp mark 7}{(2.250,4.116)}
\gppoint{gp mark 7}{(2.261,4.110)}
\gppoint{gp mark 7}{(2.272,4.105)}
\gppoint{gp mark 7}{(2.283,4.099)}
\gppoint{gp mark 7}{(2.294,4.094)}
\gppoint{gp mark 7}{(2.305,4.088)}
\gppoint{gp mark 7}{(2.316,4.082)}
\gppoint{gp mark 7}{(2.327,4.077)}
\gppoint{gp mark 7}{(2.338,4.071)}
\gppoint{gp mark 7}{(2.349,4.065)}
\gppoint{gp mark 7}{(2.360,4.060)}
\gppoint{gp mark 7}{(2.371,4.054)}
\gppoint{gp mark 7}{(2.382,4.049)}
\gppoint{gp mark 7}{(2.393,4.043)}
\gppoint{gp mark 7}{(2.404,4.037)}
\gppoint{gp mark 7}{(2.415,4.032)}
\gppoint{gp mark 7}{(2.426,4.026)}
\gppoint{gp mark 7}{(2.437,4.020)}
\gppoint{gp mark 7}{(2.448,4.015)}
\gppoint{gp mark 7}{(2.459,4.009)}
\gppoint{gp mark 7}{(2.470,4.003)}
\gppoint{gp mark 7}{(2.480,3.997)}
\gppoint{gp mark 7}{(2.491,3.992)}
\gppoint{gp mark 7}{(2.502,3.986)}
\gppoint{gp mark 7}{(2.513,3.980)}
\gppoint{gp mark 7}{(2.524,3.975)}
\gppoint{gp mark 7}{(2.535,3.969)}
\gppoint{gp mark 7}{(2.546,3.963)}
\gppoint{gp mark 7}{(2.557,3.957)}
\gppoint{gp mark 7}{(2.568,3.952)}
\gppoint{gp mark 7}{(2.579,3.946)}
\gppoint{gp mark 7}{(2.590,3.940)}
\gppoint{gp mark 7}{(2.601,3.934)}
\gppoint{gp mark 7}{(2.612,3.928)}
\gppoint{gp mark 7}{(2.623,3.923)}
\gppoint{gp mark 7}{(2.634,3.917)}
\gppoint{gp mark 7}{(2.645,3.911)}
\gppoint{gp mark 7}{(2.656,3.905)}
\gppoint{gp mark 7}{(2.667,3.899)}
\gppoint{gp mark 7}{(2.678,3.894)}
\gppoint{gp mark 7}{(2.689,3.888)}
\gppoint{gp mark 7}{(2.700,3.882)}
\gppoint{gp mark 7}{(2.711,3.876)}
\gppoint{gp mark 7}{(2.722,3.870)}
\gppoint{gp mark 7}{(2.733,3.864)}
\gppoint{gp mark 7}{(2.744,3.858)}
\gppoint{gp mark 7}{(2.755,3.852)}
\gppoint{gp mark 7}{(2.766,3.847)}
\gppoint{gp mark 7}{(2.777,3.841)}
\gppoint{gp mark 7}{(2.788,3.835)}
\gppoint{gp mark 7}{(2.799,3.829)}
\gppoint{gp mark 7}{(2.810,3.823)}
\gppoint{gp mark 7}{(2.821,3.817)}
\gppoint{gp mark 7}{(2.832,3.811)}
\gppoint{gp mark 7}{(2.843,3.805)}
\gppoint{gp mark 7}{(2.854,3.799)}
\gppoint{gp mark 7}{(2.865,3.793)}
\gppoint{gp mark 7}{(2.876,3.787)}
\gppoint{gp mark 7}{(2.887,3.781)}
\gppoint{gp mark 7}{(2.898,3.775)}
\gppoint{gp mark 7}{(2.909,3.770)}
\gppoint{gp mark 7}{(2.920,3.764)}
\gppoint{gp mark 7}{(2.931,3.758)}
\gppoint{gp mark 7}{(2.942,3.752)}
\gppoint{gp mark 7}{(2.953,3.746)}
\gppoint{gp mark 7}{(2.964,3.740)}
\gppoint{gp mark 7}{(2.975,3.734)}
\gppoint{gp mark 7}{(2.986,3.728)}
\gppoint{gp mark 7}{(2.997,3.722)}
\gppoint{gp mark 7}{(3.008,3.716)}
\gppoint{gp mark 7}{(3.019,3.709)}
\gppoint{gp mark 7}{(3.030,3.703)}
\gppoint{gp mark 7}{(3.041,3.698)}
\gppoint{gp mark 7}{(3.052,3.694)}
\gppoint{gp mark 7}{(3.063,3.692)}
\gppoint{gp mark 7}{(3.074,3.692)}
\gppoint{gp mark 7}{(3.085,3.692)}
\gppoint{gp mark 7}{(3.096,3.694)}
\gppoint{gp mark 7}{(3.107,3.703)}
\gppoint{gp mark 7}{(3.118,3.720)}
\gppoint{gp mark 7}{(3.129,3.733)}
\gppoint{gp mark 7}{(3.140,3.737)}
\gppoint{gp mark 7}{(3.151,3.737)}
\gppoint{gp mark 7}{(3.162,3.737)}
\gppoint{gp mark 7}{(3.173,3.737)}
\gppoint{gp mark 7}{(3.184,3.737)}
\gppoint{gp mark 7}{(3.195,3.737)}
\gppoint{gp mark 7}{(3.206,3.737)}
\gppoint{gp mark 7}{(3.217,3.736)}
\gppoint{gp mark 7}{(3.228,3.735)}
\gppoint{gp mark 7}{(3.239,3.734)}
\gppoint{gp mark 7}{(3.250,3.732)}
\gppoint{gp mark 7}{(3.261,3.731)}
\gppoint{gp mark 7}{(3.272,3.730)}
\gppoint{gp mark 7}{(3.283,3.729)}
\gppoint{gp mark 7}{(3.294,3.728)}
\gppoint{gp mark 7}{(3.305,3.728)}
\gppoint{gp mark 7}{(3.316,3.727)}
\gppoint{gp mark 7}{(3.327,3.727)}
\gppoint{gp mark 7}{(3.338,3.727)}
\gppoint{gp mark 7}{(3.349,3.726)}
\gppoint{gp mark 7}{(3.360,3.726)}
\gppoint{gp mark 7}{(3.371,3.726)}
\gppoint{gp mark 7}{(3.382,3.727)}
\gppoint{gp mark 7}{(3.392,3.727)}
\gppoint{gp mark 7}{(3.403,3.727)}
\gppoint{gp mark 7}{(3.414,3.727)}
\gppoint{gp mark 7}{(3.425,3.728)}
\gppoint{gp mark 7}{(3.436,3.728)}
\gppoint{gp mark 7}{(3.447,3.728)}
\gppoint{gp mark 7}{(3.458,3.729)}
\gppoint{gp mark 7}{(3.469,3.729)}
\gppoint{gp mark 7}{(3.480,3.729)}
\gppoint{gp mark 7}{(3.491,3.730)}
\gppoint{gp mark 7}{(3.502,3.726)}
\gppoint{gp mark 7}{(3.513,3.593)}
\gppoint{gp mark 7}{(3.524,2.513)}
\gppoint{gp mark 7}{(3.535,1.766)}
\gppoint{gp mark 7}{(3.546,1.677)}
\gppoint{gp mark 7}{(3.557,1.654)}
\gppoint{gp mark 7}{(3.568,1.643)}
\gppoint{gp mark 7}{(3.579,1.635)}
\gppoint{gp mark 7}{(3.590,1.629)}
\gppoint{gp mark 7}{(3.601,1.621)}
\gppoint{gp mark 7}{(3.612,1.612)}
\gppoint{gp mark 7}{(3.623,1.604)}
\gppoint{gp mark 7}{(3.634,1.594)}
\gppoint{gp mark 7}{(3.645,1.583)}
\gppoint{gp mark 7}{(3.656,1.575)}
\gppoint{gp mark 7}{(3.667,1.567)}
\gppoint{gp mark 7}{(3.678,1.559)}
\gppoint{gp mark 7}{(3.689,1.549)}
\gppoint{gp mark 7}{(3.700,1.541)}
\gppoint{gp mark 7}{(3.711,1.533)}
\gppoint{gp mark 7}{(3.722,1.525)}
\gppoint{gp mark 7}{(3.733,1.516)}
\gppoint{gp mark 7}{(3.744,1.508)}
\gppoint{gp mark 7}{(3.755,1.501)}
\gppoint{gp mark 7}{(3.766,1.493)}
\gppoint{gp mark 7}{(3.777,1.484)}
\gppoint{gp mark 7}{(3.788,1.477)}
\gppoint{gp mark 7}{(3.799,1.470)}
\gppoint{gp mark 7}{(3.810,1.462)}
\gppoint{gp mark 7}{(3.821,1.454)}
\gppoint{gp mark 7}{(3.832,1.446)}
\gppoint{gp mark 7}{(3.843,1.439)}
\gppoint{gp mark 7}{(3.854,1.431)}
\gppoint{gp mark 7}{(3.865,1.424)}
\gppoint{gp mark 7}{(3.876,1.417)}
\gppoint{gp mark 7}{(3.887,1.410)}
\gppoint{gp mark 7}{(3.898,1.403)}
\gppoint{gp mark 7}{(3.909,1.395)}
\gppoint{gp mark 7}{(3.920,1.389)}
\gppoint{gp mark 7}{(3.931,1.382)}
\gppoint{gp mark 7}{(3.942,1.375)}
\gppoint{gp mark 7}{(3.953,1.368)}
\gppoint{gp mark 7}{(3.964,1.361)}
\gppoint{gp mark 7}{(3.975,1.354)}
\gppoint{gp mark 7}{(3.986,1.348)}
\gppoint{gp mark 7}{(3.997,1.341)}
\gppoint{gp mark 7}{(4.008,1.335)}
\gppoint{gp mark 7}{(4.019,1.328)}
\gppoint{gp mark 7}{(4.030,1.322)}
\gppoint{gp mark 7}{(4.041,1.316)}
\gppoint{gp mark 7}{(4.052,1.309)}
\gppoint{gp mark 7}{(4.063,1.304)}
\gppoint{gp mark 7}{(4.074,1.299)}
\gppoint{gp mark 7}{(4.085,1.294)}
\gppoint{gp mark 7}{(4.096,1.291)}
\gppoint{gp mark 7}{(4.107,1.289)}
\gppoint{gp mark 7}{(4.118,1.289)}
\gppoint{gp mark 7}{(4.129,1.289)}
\gppoint{gp mark 7}{(4.140,1.289)}
\gppoint{gp mark 7}{(4.151,1.290)}
\gppoint{gp mark 7}{(4.162,1.290)}
\gppoint{gp mark 7}{(4.173,1.291)}
\gppoint{gp mark 7}{(4.184,1.291)}
\gppoint{gp mark 7}{(4.195,1.292)}
\gppoint{gp mark 7}{(4.206,1.292)}
\gppoint{gp mark 7}{(4.217,1.292)}
\gppoint{gp mark 7}{(4.228,1.292)}
\gppoint{gp mark 7}{(4.239,1.292)}
\gppoint{gp mark 7}{(4.250,1.292)}
\gppoint{gp mark 7}{(4.261,1.292)}
\gppoint{gp mark 7}{(4.272,1.292)}
\gppoint{gp mark 7}{(4.283,1.292)}
\gppoint{gp mark 7}{(4.294,1.292)}
\gppoint{gp mark 7}{(4.304,1.292)}
\gppoint{gp mark 7}{(4.315,1.292)}
\gppoint{gp mark 7}{(4.326,1.292)}
\gppoint{gp mark 7}{(4.337,1.292)}
\gppoint{gp mark 7}{(4.348,1.292)}
\gppoint{gp mark 7}{(4.359,1.292)}
\gppoint{gp mark 7}{(4.370,1.292)}
\gppoint{gp mark 7}{(4.381,1.292)}
\gppoint{gp mark 7}{(4.392,1.292)}
\gppoint{gp mark 7}{(4.403,1.292)}
\gppoint{gp mark 7}{(4.414,1.292)}
\gppoint{gp mark 7}{(4.425,1.292)}
\gppoint{gp mark 7}{(4.436,1.292)}
\gppoint{gp mark 7}{(4.447,1.292)}
\gppoint{gp mark 7}{(4.458,1.292)}
\gppoint{gp mark 7}{(4.469,1.292)}
\gppoint{gp mark 7}{(4.480,1.292)}
\gppoint{gp mark 7}{(4.491,1.292)}
\gppoint{gp mark 7}{(4.502,1.292)}
\gppoint{gp mark 7}{(4.513,1.292)}
\gppoint{gp mark 7}{(4.524,1.292)}
\gppoint{gp mark 7}{(4.535,1.292)}
\gppoint{gp mark 7}{(4.546,1.292)}
\gppoint{gp mark 7}{(4.557,1.292)}
\gppoint{gp mark 7}{(4.568,1.292)}
\gppoint{gp mark 7}{(4.579,1.292)}
\gppoint{gp mark 7}{(4.590,1.292)}
\gppoint{gp mark 7}{(4.601,1.292)}
\gppoint{gp mark 7}{(4.612,1.292)}
\gppoint{gp mark 7}{(4.623,1.292)}
\gppoint{gp mark 7}{(4.634,1.292)}
\gppoint{gp mark 7}{(4.645,1.292)}
\gppoint{gp mark 7}{(4.656,1.292)}
\gppoint{gp mark 7}{(4.667,1.292)}
\gppoint{gp mark 7}{(4.678,1.292)}
\gppoint{gp mark 7}{(4.689,1.292)}
\gppoint{gp mark 7}{(4.700,1.292)}
\gppoint{gp mark 7}{(4.711,1.292)}
\gppoint{gp mark 7}{(4.722,1.292)}
\gppoint{gp mark 7}{(4.733,1.292)}
\gppoint{gp mark 7}{(4.744,1.292)}
\gppoint{gp mark 7}{(4.755,1.292)}
\gppoint{gp mark 7}{(4.766,1.292)}
\gppoint{gp mark 7}{(4.777,1.292)}
\gppoint{gp mark 7}{(4.788,1.292)}
\gppoint{gp mark 7}{(4.799,1.292)}
\gppoint{gp mark 7}{(4.810,1.292)}
\gppoint{gp mark 7}{(4.821,1.292)}
\gppoint{gp mark 7}{(4.832,1.292)}
\gppoint{gp mark 7}{(4.843,1.292)}
\gppoint{gp mark 7}{(4.854,1.292)}
\gppoint{gp mark 7}{(4.865,1.292)}
\gppoint{gp mark 7}{(4.876,1.292)}
\gppoint{gp mark 7}{(4.887,1.292)}
\gppoint{gp mark 7}{(4.898,1.292)}
\gppoint{gp mark 7}{(4.909,1.292)}
\gppoint{gp mark 7}{(4.920,1.292)}
\gppoint{gp mark 7}{(4.931,1.292)}
\gppoint{gp mark 7}{(4.942,1.292)}
\gppoint{gp mark 7}{(4.953,1.292)}
\gppoint{gp mark 7}{(4.964,1.292)}
\gppoint{gp mark 7}{(4.975,1.292)}
\gppoint{gp mark 7}{(4.986,1.292)}
\gppoint{gp mark 7}{(4.997,1.292)}
\gppoint{gp mark 7}{(5.008,1.292)}
\gppoint{gp mark 7}{(5.019,1.292)}
\gppoint{gp mark 7}{(5.030,1.292)}
\gppoint{gp mark 7}{(5.041,1.292)}
\gppoint{gp mark 7}{(5.052,1.292)}
\gppoint{gp mark 7}{(5.063,1.292)}
\gppoint{gp mark 7}{(5.074,1.292)}
\gppoint{gp mark 7}{(5.085,1.292)}
\gppoint{gp mark 7}{(5.096,1.292)}
\gppoint{gp mark 7}{(5.107,1.292)}
\gppoint{gp mark 7}{(5.118,1.292)}
\gppoint{gp mark 7}{(5.129,1.292)}
\gppoint{gp mark 7}{(5.140,1.292)}
\gppoint{gp mark 7}{(5.151,1.292)}
\gppoint{gp mark 7}{(5.162,1.292)}
\gppoint{gp mark 7}{(5.173,1.292)}
\gppoint{gp mark 7}{(5.184,1.292)}
\gppoint{gp mark 7}{(5.195,1.292)}
\gppoint{gp mark 7}{(5.206,1.292)}
\gppoint{gp mark 7}{(5.216,1.292)}
\gppoint{gp mark 7}{(5.227,1.292)}
\gppoint{gp mark 7}{(5.238,1.292)}
\gppoint{gp mark 7}{(5.249,1.292)}
\gppoint{gp mark 7}{(5.260,1.292)}
\gppoint{gp mark 7}{(5.271,1.292)}
\gppoint{gp mark 7}{(5.282,1.292)}
\gppoint{gp mark 7}{(5.293,1.292)}
\gppoint{gp mark 7}{(5.304,1.292)}
\gppoint{gp mark 7}{(5.315,1.292)}
\gppoint{gp mark 7}{(5.326,1.292)}
\gppoint{gp mark 7}{(5.337,1.292)}
\gppoint{gp mark 7}{(5.348,1.292)}
\gppoint{gp mark 7}{(5.359,1.292)}
\gppoint{gp mark 7}{(5.370,1.292)}
\gppoint{gp mark 7}{(5.381,1.292)}
\gppoint{gp mark 7}{(5.392,1.292)}
\gppoint{gp mark 7}{(5.403,1.292)}
\gppoint{gp mark 7}{(5.414,1.292)}
\gppoint{gp mark 7}{(5.425,1.292)}
\gppoint{gp mark 7}{(5.436,1.292)}
\gppoint{gp mark 7}{(5.447,1.292)}
\gppoint{gp mark 7}{(5.458,1.292)}
\gppoint{gp mark 7}{(5.469,1.292)}
\gppoint{gp mark 7}{(5.480,1.292)}
\gppoint{gp mark 7}{(5.491,1.292)}
\gppoint{gp mark 7}{(5.502,1.292)}
\gppoint{gp mark 7}{(5.513,1.292)}
\gppoint{gp mark 7}{(5.524,1.292)}
\gppoint{gp mark 7}{(5.535,1.292)}
\gppoint{gp mark 7}{(5.546,1.292)}
\gppoint{gp mark 7}{(5.557,1.292)}
\gppoint{gp mark 7}{(5.568,1.292)}
\gppoint{gp mark 7}{(5.579,1.292)}
\gppoint{gp mark 7}{(5.590,1.292)}
\gppoint{gp mark 7}{(5.601,1.292)}
\gppoint{gp mark 7}{(5.612,1.292)}
\gppoint{gp mark 7}{(5.623,1.292)}
\gppoint{gp mark 7}{(5.634,1.292)}
\gppoint{gp mark 7}{(5.645,1.292)}
\gppoint{gp mark 7}{(5.656,1.292)}
\gppoint{gp mark 7}{(5.667,1.292)}
\gppoint{gp mark 7}{(5.678,1.292)}
\gppoint{gp mark 7}{(5.689,1.292)}
\gppoint{gp mark 7}{(5.700,1.292)}
\gppoint{gp mark 7}{(5.711,1.292)}
\gppoint{gp mark 7}{(5.722,1.292)}
\gppoint{gp mark 7}{(5.733,1.292)}
\gppoint{gp mark 7}{(5.744,1.292)}
\gppoint{gp mark 7}{(5.755,1.292)}
\gppoint{gp mark 7}{(5.766,1.292)}
\gppoint{gp mark 7}{(5.777,1.292)}
\gppoint{gp mark 7}{(5.788,1.292)}
\gppoint{gp mark 7}{(5.799,1.292)}
\gppoint{gp mark 7}{(5.810,1.292)}
\gppoint{gp mark 7}{(5.821,1.292)}
\gppoint{gp mark 7}{(5.832,1.292)}
\gppoint{gp mark 7}{(5.843,1.292)}
\gppoint{gp mark 7}{(5.854,1.292)}
\gppoint{gp mark 7}{(5.865,1.292)}
\gppoint{gp mark 7}{(5.876,1.292)}
\gppoint{gp mark 7}{(5.887,1.292)}
\gppoint{gp mark 7}{(5.898,1.292)}
\gppoint{gp mark 7}{(5.909,1.292)}
\gppoint{gp mark 7}{(5.920,1.292)}
\gppoint{gp mark 7}{(5.931,1.292)}
\gppoint{gp mark 7}{(5.942,1.292)}
\gppoint{gp mark 7}{(5.953,1.292)}
\gppoint{gp mark 7}{(5.964,1.292)}
\gppoint{gp mark 7}{(5.975,1.292)}
\gppoint{gp mark 7}{(5.986,1.292)}
\gppoint{gp mark 7}{(5.997,1.292)}
\gppoint{gp mark 7}{(6.008,1.292)}
\gppoint{gp mark 7}{(6.019,1.292)}
\gppoint{gp mark 7}{(6.030,1.292)}
\gppoint{gp mark 7}{(6.041,1.292)}
\gppoint{gp mark 7}{(6.052,1.292)}
\gppoint{gp mark 7}{(6.063,1.292)}
\gppoint{gp mark 7}{(6.074,1.292)}
\gppoint{gp mark 7}{(6.085,1.292)}
\gppoint{gp mark 7}{(6.096,1.292)}
\gppoint{gp mark 7}{(6.107,1.292)}
\gppoint{gp mark 7}{(6.118,1.292)}
\gppoint{gp mark 7}{(6.128,1.292)}
\gppoint{gp mark 7}{(6.139,1.292)}
\gppoint{gp mark 7}{(6.150,1.292)}
\gppoint{gp mark 7}{(6.161,1.292)}
\gppoint{gp mark 7}{(6.172,1.292)}
\gppoint{gp mark 7}{(6.183,1.292)}
\gppoint{gp mark 7}{(6.194,1.292)}
\gppoint{gp mark 7}{(6.205,1.292)}
\gppoint{gp mark 7}{(6.216,1.292)}
\gppoint{gp mark 7}{(6.227,1.292)}
\gppoint{gp mark 7}{(6.238,1.292)}
\gppoint{gp mark 7}{(6.249,1.292)}
\gppoint{gp mark 7}{(6.260,1.292)}
\gppoint{gp mark 7}{(6.271,1.292)}
\gppoint{gp mark 7}{(6.282,1.292)}
\gppoint{gp mark 7}{(6.293,1.292)}
\gppoint{gp mark 7}{(6.304,1.292)}
\gppoint{gp mark 7}{(6.315,1.292)}
\gppoint{gp mark 7}{(6.326,1.292)}
\gppoint{gp mark 7}{(6.337,1.292)}
\gppoint{gp mark 7}{(6.348,1.292)}
\gppoint{gp mark 7}{(6.359,1.292)}
\gppoint{gp mark 7}{(6.370,1.292)}
\gppoint{gp mark 7}{(6.381,1.292)}
\gppoint{gp mark 7}{(6.392,1.292)}
\gppoint{gp mark 7}{(6.403,1.292)}
\gppoint{gp mark 7}{(6.414,1.292)}
\gppoint{gp mark 7}{(6.425,1.292)}
\gppoint{gp mark 7}{(6.436,1.292)}
\gppoint{gp mark 7}{(6.447,1.292)}
\gppoint{gp mark 7}{(6.458,1.292)}
\gppoint{gp mark 7}{(6.469,1.292)}
\gppoint{gp mark 7}{(6.480,1.292)}
\gppoint{gp mark 7}{(6.491,1.292)}
\gppoint{gp mark 7}{(6.502,1.292)}
\gppoint{gp mark 7}{(6.513,1.292)}
\gppoint{gp mark 7}{(6.524,1.292)}
\gppoint{gp mark 7}{(6.535,1.292)}
\gppoint{gp mark 7}{(6.546,1.292)}
\gppoint{gp mark 7}{(6.557,1.292)}
\gppoint{gp mark 7}{(6.568,1.292)}
\gppoint{gp mark 7}{(6.579,1.292)}
\gppoint{gp mark 7}{(6.590,1.292)}
\gppoint{gp mark 7}{(6.601,1.292)}
\gppoint{gp mark 7}{(6.612,1.292)}
\gppoint{gp mark 7}{(6.623,1.292)}
\gppoint{gp mark 7}{(6.634,1.292)}
\gppoint{gp mark 7}{(6.645,1.292)}
\gppoint{gp mark 7}{(6.656,1.292)}
\gppoint{gp mark 7}{(6.667,1.292)}
\gppoint{gp mark 7}{(6.678,1.292)}
\gppoint{gp mark 7}{(6.689,1.292)}
\gppoint{gp mark 7}{(6.700,1.292)}
\gppoint{gp mark 7}{(6.711,1.292)}
\gppoint{gp mark 7}{(6.722,1.292)}
\gppoint{gp mark 7}{(6.733,1.292)}
\gppoint{gp mark 7}{(6.744,1.292)}
\gppoint{gp mark 7}{(6.755,1.292)}
\gppoint{gp mark 7}{(6.766,1.292)}
\gppoint{gp mark 7}{(6.777,1.292)}
\gppoint{gp mark 7}{(6.788,1.292)}
\gppoint{gp mark 7}{(6.799,1.292)}
\gppoint{gp mark 7}{(6.810,1.292)}
\gppoint{gp mark 7}{(6.821,1.292)}
\gppoint{gp mark 7}{(6.832,1.292)}
\gppoint{gp mark 7}{(6.843,1.292)}
\gppoint{gp mark 7}{(6.854,1.292)}
\gppoint{gp mark 7}{(6.865,1.292)}
\gppoint{gp mark 7}{(6.876,1.292)}
\gppoint{gp mark 7}{(6.887,1.292)}
\gppoint{gp mark 7}{(6.898,1.292)}
\gppoint{gp mark 7}{(6.909,1.292)}
\gppoint{gp mark 7}{(6.920,1.292)}
\gppoint{gp mark 7}{(6.931,1.292)}
\gppoint{gp mark 7}{(6.942,1.292)}
\gppoint{gp mark 7}{(6.953,1.292)}
\gppoint{gp mark 7}{(6.964,1.292)}
\gppoint{gp mark 7}{(6.975,1.292)}
\gppoint{gp mark 7}{(6.986,1.292)}
\gppoint{gp mark 7}{(6.997,1.292)}
\gppoint{gp mark 7}{(7.008,1.292)}
\gppoint{gp mark 7}{(7.019,1.292)}
\gppoint{gp mark 7}{(7.030,1.292)}
\gppoint{gp mark 7}{(7.040,1.292)}
\gppoint{gp mark 7}{(7.051,1.292)}
\gppoint{gp mark 7}{(7.062,1.292)}
\gppoint{gp mark 7}{(7.073,1.292)}
\gppoint{gp mark 7}{(7.084,1.292)}
\gppoint{gp mark 7}{(7.095,1.292)}
\gppoint{gp mark 7}{(7.106,1.292)}
\gppoint{gp mark 7}{(7.117,1.292)}
\gppoint{gp mark 7}{(7.128,1.292)}
\gppoint{gp mark 7}{(7.139,1.292)}
\gppoint{gp mark 7}{(7.150,1.292)}
\gppoint{gp mark 7}{(7.161,1.292)}
\gppoint{gp mark 7}{(7.172,1.292)}
\gppoint{gp mark 7}{(7.183,1.292)}
\gppoint{gp mark 7}{(7.194,1.292)}
\gppoint{gp mark 7}{(7.205,1.292)}
\gppoint{gp mark 7}{(7.216,1.292)}
\gppoint{gp mark 7}{(7.227,1.292)}
\gppoint{gp mark 7}{(7.238,1.292)}
\gppoint{gp mark 7}{(7.249,1.292)}
\gppoint{gp mark 7}{(7.260,1.292)}
\gppoint{gp mark 7}{(7.271,1.292)}
\gppoint{gp mark 7}{(7.282,1.292)}
\gppoint{gp mark 7}{(7.293,1.292)}
\gppoint{gp mark 7}{(7.304,1.292)}
\gppoint{gp mark 7}{(7.315,1.292)}
\gppoint{gp mark 7}{(7.326,1.292)}
\gppoint{gp mark 7}{(7.337,1.292)}
\gppoint{gp mark 7}{(7.348,1.292)}
\gppoint{gp mark 7}{(7.359,1.292)}
\gppoint{gp mark 7}{(7.370,1.292)}
\gppoint{gp mark 7}{(7.381,1.292)}
\gppoint{gp mark 7}{(7.392,1.292)}
\gppoint{gp mark 7}{(7.403,1.292)}
\gppoint{gp mark 7}{(7.414,1.292)}
\gppoint{gp mark 7}{(7.425,1.292)}
\gppoint{gp mark 7}{(7.436,1.292)}
\gppoint{gp mark 7}{(7.447,1.292)}
\gppoint{gp mark 7}{(7.458,1.292)}
\gppoint{gp mark 7}{(7.469,1.292)}
\gppoint{gp mark 7}{(7.480,1.292)}
\gppoint{gp mark 7}{(7.491,1.292)}
\gppoint{gp mark 7}{(7.502,1.292)}
\gppoint{gp mark 7}{(7.513,1.292)}
\gppoint{gp mark 7}{(7.524,1.292)}
\gppoint{gp mark 7}{(7.535,1.292)}
\gppoint{gp mark 7}{(7.546,1.292)}
\gppoint{gp mark 7}{(7.557,1.292)}
\gppoint{gp mark 7}{(7.568,1.292)}
\gppoint{gp mark 7}{(7.579,1.292)}
\gppoint{gp mark 7}{(7.590,1.292)}
\gppoint{gp mark 7}{(7.601,1.292)}
\gppoint{gp mark 7}{(7.612,1.292)}
\gppoint{gp mark 7}{(7.623,1.292)}
\gppoint{gp mark 7}{(7.634,1.292)}
\gppoint{gp mark 7}{(7.645,1.292)}
\gppoint{gp mark 7}{(7.656,1.292)}
\gppoint{gp mark 7}{(7.667,1.292)}
\gppoint{gp mark 7}{(7.678,1.292)}
\gppoint{gp mark 7}{(7.689,1.292)}
\gppoint{gp mark 7}{(7.700,1.292)}
\gppoint{gp mark 7}{(7.711,1.292)}
\gppoint{gp mark 7}{(7.722,1.292)}
\gppoint{gp mark 7}{(7.733,1.292)}
\gppoint{gp mark 7}{(7.744,1.292)}
\gppoint{gp mark 7}{(7.755,1.292)}
\gppoint{gp mark 7}{(7.766,1.292)}
\gppoint{gp mark 7}{(7.777,1.292)}
\gppoint{gp mark 7}{(7.788,1.292)}
\gppoint{gp mark 7}{(7.799,1.292)}
\gppoint{gp mark 7}{(7.810,1.292)}
\gppoint{gp mark 7}{(7.821,1.292)}
\gppoint{gp mark 7}{(7.832,1.292)}
\gppoint{gp mark 7}{(7.843,1.292)}
\gppoint{gp mark 7}{(7.854,1.292)}
\gppoint{gp mark 7}{(7.865,1.292)}
\gppoint{gp mark 7}{(7.876,1.292)}
\gppoint{gp mark 7}{(7.887,1.292)}
\gppoint{gp mark 7}{(7.898,1.292)}
\gppoint{gp mark 7}{(7.909,1.292)}
\gppoint{gp mark 7}{(7.920,1.292)}
\gppoint{gp mark 7}{(7.931,1.292)}
\gppoint{gp mark 7}{(7.942,1.292)}
\gpcolor{rgb color={0.000,0.000,0.000}}
\gpsetlinetype{gp lt plot 0}
\gpsetlinewidth{4.00}
\draw[gp path] (2.427,4.020)--(3.533,4.020);
\draw[gp path] (3.533,1.048)--(4.326,1.048);
\draw[gp path] (4.326,1.293)--(7.510,1.293);
\draw[gp path] (7.510,1.293)--(7.947,1.293);
\draw[gp path] (1.200,4.618)--(1.206,4.615)--(1.212,4.612)--(1.218,4.609)--(1.224,4.606)%
  --(1.230,4.603)--(1.236,4.600)--(1.242,4.597)--(1.248,4.594)--(1.254,4.591)--(1.260,4.588)%
  --(1.267,4.585)--(1.273,4.582)--(1.279,4.579)--(1.285,4.576)--(1.291,4.573)--(1.297,4.570)%
  --(1.303,4.567)--(1.309,4.564)--(1.315,4.562)--(1.321,4.559)--(1.327,4.556)--(1.333,4.553)%
  --(1.339,4.550)--(1.346,4.547)--(1.352,4.544)--(1.358,4.541)--(1.364,4.538)--(1.370,4.535)%
  --(1.376,4.532)--(1.382,4.529)--(1.388,4.526)--(1.394,4.523)--(1.400,4.520)--(1.406,4.517)%
  --(1.412,4.514)--(1.418,4.511)--(1.425,4.508)--(1.431,4.505)--(1.437,4.502)--(1.443,4.499)%
  --(1.449,4.496)--(1.455,4.493)--(1.461,4.490)--(1.467,4.488)--(1.473,4.485)--(1.479,4.482)%
  --(1.485,4.479)--(1.491,4.476)--(1.497,4.473)--(1.504,4.470)--(1.510,4.467)--(1.516,4.464)%
  --(1.522,4.461)--(1.528,4.458)--(1.534,4.455)--(1.540,4.452)--(1.546,4.449)--(1.552,4.446)%
  --(1.558,4.443)--(1.564,4.440)--(1.570,4.437)--(1.576,4.434)--(1.583,4.431)--(1.589,4.428)%
  --(1.595,4.425)--(1.601,4.422)--(1.607,4.419)--(1.613,4.416)--(1.619,4.413)--(1.625,4.411)%
  --(1.631,4.408)--(1.637,4.405)--(1.643,4.402)--(1.649,4.399)--(1.656,4.396)--(1.662,4.393)%
  --(1.668,4.390)--(1.674,4.387)--(1.680,4.384)--(1.686,4.381)--(1.692,4.378)--(1.698,4.375)%
  --(1.704,4.372)--(1.710,4.369)--(1.716,4.366)--(1.722,4.363)--(1.728,4.360)--(1.735,4.357)%
  --(1.741,4.354)--(1.747,4.351)--(1.753,4.348)--(1.759,4.345)--(1.765,4.342)--(1.771,4.339)%
  --(1.777,4.337)--(1.783,4.334)--(1.789,4.331)--(1.795,4.328)--(1.801,4.325)--(1.807,4.322)%
  --(1.814,4.319)--(1.820,4.316)--(1.826,4.313)--(1.832,4.310)--(1.838,4.307)--(1.844,4.304)%
  --(1.850,4.301)--(1.856,4.298)--(1.862,4.295)--(1.868,4.292)--(1.874,4.289)--(1.880,4.286)%
  --(1.886,4.283)--(1.893,4.280)--(1.899,4.277)--(1.905,4.274)--(1.911,4.271)--(1.917,4.268)%
  --(1.923,4.265)--(1.929,4.262)--(1.935,4.260)--(1.941,4.257)--(1.947,4.254)--(1.953,4.251)%
  --(1.959,4.248)--(1.965,4.245)--(1.972,4.242)--(1.978,4.239)--(1.984,4.236)--(1.990,4.233)%
  --(1.996,4.230)--(2.002,4.227)--(2.008,4.224)--(2.014,4.221)--(2.020,4.218)--(2.026,4.215)%
  --(2.032,4.212)--(2.038,4.209)--(2.044,4.206)--(2.051,4.203)--(2.057,4.200)--(2.063,4.197)%
  --(2.069,4.194)--(2.075,4.191)--(2.081,4.188)--(2.087,4.186)--(2.093,4.183)--(2.099,4.180)%
  --(2.105,4.177)--(2.111,4.174)--(2.117,4.171)--(2.124,4.168)--(2.130,4.165)--(2.136,4.162)%
  --(2.142,4.159)--(2.148,4.156)--(2.154,4.153)--(2.160,4.150)--(2.166,4.147)--(2.172,4.144)%
  --(2.178,4.141)--(2.184,4.138)--(2.190,4.135)--(2.196,4.132)--(2.203,4.129)--(2.209,4.126)%
  --(2.215,4.123)--(2.221,4.120)--(2.227,4.117)--(2.233,4.114)--(2.239,4.111)--(2.245,4.109)%
  --(2.251,4.106)--(2.257,4.103)--(2.263,4.100)--(2.269,4.097)--(2.275,4.094)--(2.282,4.091)%
  --(2.288,4.088)--(2.294,4.085)--(2.300,4.082)--(2.306,4.079)--(2.312,4.076)--(2.318,4.073)%
  --(2.324,4.070)--(2.330,4.067)--(2.336,4.064)--(2.342,4.061)--(2.348,4.058)--(2.354,4.055)%
  --(2.361,4.052)--(2.367,4.049)--(2.373,4.046)--(2.379,4.043)--(2.385,4.040)--(2.391,4.037)%
  --(2.397,4.035)--(2.403,4.032)--(2.409,4.029)--(2.415,4.026)--(2.421,4.023)--(2.427,4.020);
\draw[gp path] (3.533,4.020)--(3.533,1.048);
\draw[gp path] (4.326,1.048)--(4.326,1.293);
\gpsetlinetype{gp lt plot 2}
\draw[gp path] (1.206,4.020)--(1.217,4.020)--(1.228,4.020)--(1.239,4.020)--(1.250,4.020)%
  --(1.261,4.020)--(1.272,4.020)--(1.283,4.020)--(1.294,4.020)--(1.305,4.020)--(1.316,4.020)%
  --(1.327,4.020)--(1.338,4.020)--(1.349,4.020)--(1.360,4.020)--(1.371,4.020)--(1.382,4.020)%
  --(1.393,4.020)--(1.404,4.020)--(1.415,4.020)--(1.426,4.020)--(1.437,4.020)--(1.448,4.020)%
  --(1.459,4.020)--(1.470,4.020)--(1.481,4.020)--(1.492,4.020)--(1.503,4.020)--(1.514,4.020)%
  --(1.525,4.020)--(1.536,4.020)--(1.547,4.020)--(1.558,4.020)--(1.568,4.020)--(1.579,4.020)%
  --(1.590,4.020)--(1.601,4.020)--(1.612,4.020)--(1.623,4.020)--(1.634,4.020)--(1.645,4.020)%
  --(1.656,4.020)--(1.667,4.020)--(1.678,4.020)--(1.689,4.020)--(1.700,4.020)--(1.711,4.020)%
  --(1.722,4.020)--(1.733,4.020)--(1.744,4.020)--(1.755,4.020)--(1.766,4.020)--(1.777,4.020)%
  --(1.788,4.020)--(1.799,4.020)--(1.810,4.020)--(1.821,4.020)--(1.832,4.020)--(1.843,4.020)%
  --(1.854,4.020)--(1.865,4.020)--(1.876,4.020)--(1.887,4.020)--(1.898,4.020)--(1.909,4.020)%
  --(1.920,4.020)--(1.931,4.020)--(1.942,4.020)--(1.953,4.020)--(1.964,4.020)--(1.975,4.020)%
  --(1.986,4.020)--(1.997,4.020)--(2.008,4.020)--(2.019,4.020)--(2.030,4.020)--(2.041,4.020)%
  --(2.052,4.020)--(2.063,4.020)--(2.074,4.020)--(2.085,4.020)--(2.096,4.020)--(2.107,4.020)%
  --(2.118,4.020)--(2.129,4.020)--(2.140,4.020)--(2.151,4.020)--(2.162,4.020)--(2.173,4.020)%
  --(2.184,4.020)--(2.195,4.020)--(2.206,4.020)--(2.217,4.020)--(2.228,4.020)--(2.239,4.020)%
  --(2.250,4.020)--(2.261,4.020)--(2.272,4.020)--(2.283,4.020)--(2.294,4.020)--(2.305,4.020)%
  --(2.316,4.020)--(2.327,4.020)--(2.338,4.020)--(2.349,4.020)--(2.360,4.019)--(2.371,4.019)%
  --(2.382,4.018)--(2.393,4.017)--(2.404,4.015)--(2.415,4.012)--(2.426,4.009)--(2.437,4.004)%
  --(2.448,4.000)--(2.459,3.995)--(2.470,3.989)--(2.480,3.984)--(2.491,3.979)--(2.502,3.973)%
  --(2.513,3.968)--(2.524,3.962)--(2.535,3.957)--(2.546,3.951)--(2.557,3.946)--(2.568,3.940)%
  --(2.579,3.934)--(2.590,3.929)--(2.601,3.923)--(2.612,3.917)--(2.623,3.912)--(2.634,3.906)%
  --(2.645,3.900)--(2.656,3.894)--(2.667,3.889)--(2.678,3.883)--(2.689,3.877)--(2.700,3.871)%
  --(2.711,3.865)--(2.722,3.859)--(2.733,3.854)--(2.744,3.848)--(2.755,3.842)--(2.766,3.836)%
  --(2.777,3.830)--(2.788,3.824)--(2.799,3.818)--(2.810,3.812)--(2.821,3.806)--(2.832,3.800)%
  --(2.843,3.794)--(2.854,3.788)--(2.865,3.782)--(2.876,3.776)--(2.887,3.770)--(2.898,3.764)%
  --(2.909,3.758)--(2.920,3.752)--(2.931,3.746)--(2.942,3.740)--(2.953,3.734)--(2.964,3.728)%
  --(2.975,3.722)--(2.986,3.716)--(2.997,3.709)--(3.008,3.703)--(3.019,3.696)--(3.030,3.689)%
  --(3.041,3.682)--(3.052,3.675)--(3.063,3.669)--(3.074,3.666)--(3.085,3.666)--(3.096,3.666)%
  --(3.107,3.667)--(3.118,3.676)--(3.129,3.698)--(3.140,3.720)--(3.151,3.727)--(3.162,3.728)%
  --(3.173,3.728)--(3.184,3.728)--(3.195,3.728)--(3.206,3.728)--(3.217,3.728)--(3.228,3.728)%
  --(3.239,3.728)--(3.250,3.727)--(3.261,3.727)--(3.272,3.726)--(3.283,3.725)--(3.294,3.725)%
  --(3.305,3.725)--(3.316,3.724)--(3.327,3.724)--(3.338,3.724)--(3.349,3.724)--(3.360,3.724)%
  --(3.371,3.724)--(3.382,3.724)--(3.392,3.725)--(3.403,3.725)--(3.414,3.726)--(3.425,3.726)%
  --(3.436,3.726)--(3.447,3.727)--(3.458,3.727)--(3.469,3.727)--(3.480,3.728)--(3.491,3.727)%
  --(3.502,3.708)--(3.513,3.323)--(3.524,2.077)--(3.535,1.757)--(3.546,1.721)--(3.557,1.700)%
  --(3.568,1.680)--(3.579,1.665)--(3.590,1.651)--(3.601,1.635)--(3.612,1.622)--(3.623,1.613)%
  --(3.634,1.602)--(3.645,1.591)--(3.656,1.582)--(3.667,1.573)--(3.678,1.563)--(3.689,1.554)%
  --(3.700,1.546)--(3.711,1.536)--(3.722,1.528)--(3.733,1.520)--(3.744,1.511)--(3.755,1.502)%
  --(3.766,1.494)--(3.777,1.486)--(3.788,1.478)--(3.799,1.470)--(3.810,1.463)--(3.821,1.455)%
  --(3.832,1.447)--(3.843,1.439)--(3.854,1.432)--(3.865,1.424)--(3.876,1.417)--(3.887,1.410)%
  --(3.898,1.402)--(3.909,1.395)--(3.920,1.388)--(3.931,1.381)--(3.942,1.374)--(3.953,1.367)%
  --(3.964,1.360)--(3.975,1.353)--(3.986,1.346)--(3.997,1.340)--(4.008,1.333)--(4.019,1.326)%
  --(4.030,1.320)--(4.041,1.313)--(4.052,1.307)--(4.063,1.300)--(4.074,1.294)--(4.085,1.288)%
  --(4.096,1.281)--(4.107,1.275)--(4.118,1.269)--(4.129,1.263)--(4.140,1.257)--(4.151,1.251)%
  --(4.162,1.245)--(4.173,1.239)--(4.184,1.233)--(4.195,1.227)--(4.206,1.221)--(4.217,1.215)%
  --(4.228,1.209)--(4.239,1.204)--(4.250,1.198)--(4.261,1.192)--(4.272,1.187)--(4.283,1.181)%
  --(4.294,1.176)--(4.304,1.170)--(4.315,1.165)--(4.326,1.159)--(4.337,1.154)--(4.348,1.148)%
  --(4.359,1.143)--(4.370,1.138)--(4.381,1.133)--(4.392,1.127)--(4.403,1.122)--(4.414,1.117)%
  --(4.425,1.112)--(4.436,1.107)--(4.447,1.102)--(4.458,1.097)--(4.469,1.092)--(4.480,1.088)%
  --(4.491,1.083)--(4.502,1.078)--(4.513,1.074)--(4.524,1.070)--(4.535,1.065)--(4.546,1.061)%
  --(4.557,1.058)--(4.568,1.054)--(4.579,1.051)--(4.590,1.049)--(4.601,1.048)--(4.612,1.047)%
  --(4.623,1.047)--(4.634,1.047)--(4.645,1.047)--(4.656,1.047)--(4.667,1.047)--(4.678,1.047)%
  --(4.689,1.047)--(4.700,1.047)--(4.711,1.047)--(4.722,1.047)--(4.733,1.047)--(4.744,1.047)%
  --(4.755,1.047)--(4.766,1.047)--(4.777,1.047)--(4.788,1.047)--(4.799,1.047)--(4.810,1.047)%
  --(4.821,1.047)--(4.832,1.047)--(4.843,1.047)--(4.854,1.047)--(4.865,1.047)--(4.876,1.047)%
  --(4.887,1.047)--(4.898,1.047)--(4.909,1.047)--(4.920,1.047)--(4.931,1.047)--(4.942,1.047)%
  --(4.953,1.047)--(4.964,1.047)--(4.975,1.047)--(4.986,1.047)--(4.997,1.047)--(5.008,1.047)%
  --(5.019,1.047)--(5.030,1.047)--(5.041,1.047)--(5.052,1.047)--(5.063,1.047)--(5.074,1.047)%
  --(5.085,1.047)--(5.096,1.047)--(5.107,1.047)--(5.118,1.047)--(5.129,1.047)--(5.140,1.047)%
  --(5.151,1.047)--(5.162,1.047)--(5.173,1.047)--(5.184,1.047)--(5.195,1.047)--(5.206,1.047)%
  --(5.216,1.047)--(5.227,1.047)--(5.238,1.047)--(5.249,1.047)--(5.260,1.047)--(5.271,1.047)%
  --(5.282,1.047)--(5.293,1.047)--(5.304,1.047)--(5.315,1.047)--(5.326,1.047)--(5.337,1.047)%
  --(5.348,1.047)--(5.359,1.047)--(5.370,1.047)--(5.381,1.047)--(5.392,1.047)--(5.403,1.047)%
  --(5.414,1.047)--(5.425,1.047)--(5.436,1.047)--(5.447,1.047)--(5.458,1.047)--(5.469,1.047)%
  --(5.480,1.047)--(5.491,1.047)--(5.502,1.047)--(5.513,1.047)--(5.524,1.047)--(5.535,1.047)%
  --(5.546,1.047)--(5.557,1.047)--(5.568,1.047)--(5.579,1.047)--(5.590,1.047)--(5.601,1.047)%
  --(5.612,1.047)--(5.623,1.047)--(5.634,1.047)--(5.645,1.047)--(5.656,1.047)--(5.667,1.047)%
  --(5.678,1.047)--(5.689,1.047)--(5.700,1.047)--(5.711,1.047)--(5.722,1.047)--(5.733,1.047)%
  --(5.744,1.047)--(5.755,1.047)--(5.766,1.047)--(5.777,1.047)--(5.788,1.047)--(5.799,1.047)%
  --(5.810,1.047)--(5.821,1.047)--(5.832,1.047)--(5.843,1.047)--(5.854,1.047)--(5.865,1.047)%
  --(5.876,1.047)--(5.887,1.047)--(5.898,1.047)--(5.909,1.047)--(5.920,1.047)--(5.931,1.047)%
  --(5.942,1.047)--(5.953,1.047)--(5.964,1.047)--(5.975,1.047)--(5.986,1.047)--(5.997,1.047)%
  --(6.008,1.047)--(6.019,1.047)--(6.030,1.047)--(6.041,1.047)--(6.052,1.047)--(6.063,1.047)%
  --(6.074,1.047)--(6.085,1.047)--(6.096,1.047)--(6.107,1.047)--(6.118,1.047)--(6.128,1.047)%
  --(6.139,1.047)--(6.150,1.047)--(6.161,1.047)--(6.172,1.047)--(6.183,1.047)--(6.194,1.047)%
  --(6.205,1.047)--(6.216,1.047)--(6.227,1.047)--(6.238,1.047)--(6.249,1.047)--(6.260,1.047)%
  --(6.271,1.047)--(6.282,1.047)--(6.293,1.047)--(6.304,1.047)--(6.315,1.047)--(6.326,1.047)%
  --(6.337,1.047)--(6.348,1.047)--(6.359,1.047)--(6.370,1.047)--(6.381,1.047)--(6.392,1.047)%
  --(6.403,1.047)--(6.414,1.047)--(6.425,1.047)--(6.436,1.047)--(6.447,1.047)--(6.458,1.047)%
  --(6.469,1.047)--(6.480,1.047)--(6.491,1.047)--(6.502,1.047)--(6.513,1.047)--(6.524,1.047)%
  --(6.535,1.047)--(6.546,1.047)--(6.557,1.047)--(6.568,1.047)--(6.579,1.047)--(6.590,1.047)%
  --(6.601,1.047)--(6.612,1.047)--(6.623,1.047)--(6.634,1.047)--(6.645,1.047)--(6.656,1.047)%
  --(6.667,1.047)--(6.678,1.047)--(6.689,1.047)--(6.700,1.047)--(6.711,1.047)--(6.722,1.047)%
  --(6.733,1.047)--(6.744,1.047)--(6.755,1.047)--(6.766,1.047)--(6.777,1.047)--(6.788,1.047)%
  --(6.799,1.047)--(6.810,1.047)--(6.821,1.047)--(6.832,1.047)--(6.843,1.047)--(6.854,1.047)%
  --(6.865,1.047)--(6.876,1.047)--(6.887,1.047)--(6.898,1.047)--(6.909,1.047)--(6.920,1.047)%
  --(6.931,1.047)--(6.942,1.047)--(6.953,1.047)--(6.964,1.047)--(6.975,1.047)--(6.986,1.047)%
  --(6.997,1.047)--(7.008,1.047)--(7.019,1.047)--(7.030,1.047)--(7.040,1.047)--(7.051,1.047)%
  --(7.062,1.047)--(7.073,1.047)--(7.084,1.047)--(7.095,1.047)--(7.106,1.047)--(7.117,1.047)%
  --(7.128,1.047)--(7.139,1.047)--(7.150,1.047)--(7.161,1.047)--(7.172,1.047)--(7.183,1.047)%
  --(7.194,1.047)--(7.205,1.047)--(7.216,1.047)--(7.227,1.047)--(7.238,1.047)--(7.249,1.047)%
  --(7.260,1.047)--(7.271,1.047)--(7.282,1.047)--(7.293,1.047)--(7.304,1.047)--(7.315,1.047)%
  --(7.326,1.047)--(7.337,1.047)--(7.348,1.047)--(7.359,1.047)--(7.370,1.047)--(7.381,1.047)%
  --(7.392,1.047)--(7.403,1.047)--(7.414,1.047)--(7.425,1.047)--(7.436,1.047)--(7.447,1.047)%
  --(7.458,1.047)--(7.469,1.047)--(7.480,1.047)--(7.491,1.047)--(7.502,1.047)--(7.513,1.047)%
  --(7.524,1.047)--(7.535,1.047)--(7.546,1.047)--(7.557,1.047)--(7.568,1.047)--(7.579,1.047)%
  --(7.590,1.047)--(7.601,1.047)--(7.612,1.047)--(7.623,1.047)--(7.634,1.047)--(7.645,1.047)%
  --(7.656,1.047)--(7.667,1.047)--(7.678,1.047)--(7.689,1.047)--(7.700,1.047)--(7.711,1.047)%
  --(7.722,1.047)--(7.733,1.047)--(7.744,1.047)--(7.755,1.047)--(7.766,1.047)--(7.777,1.047)%
  --(7.788,1.047)--(7.799,1.047)--(7.810,1.047)--(7.821,1.047)--(7.832,1.047)--(7.843,1.047)%
  --(7.854,1.047)--(7.865,1.047)--(7.876,1.047)--(7.887,1.047)--(7.898,1.047)--(7.909,1.047)%
  --(7.920,1.047)--(7.931,1.047)--(7.942,1.047);
\gpsetlinetype{gp lt plot 0}
\draw[gp path] (3.896,3.695)--(4.572,3.695);
\gpcolor{rgb color={1.000,0.000,0.000}}
\gpsetlinewidth{0.50}
\gppoint{gp mark 7}{(4.234,2.921)}
\gpcolor{rgb color={0.000,0.000,0.000}}
\gpsetlinetype{gp lt plot 2}
\gpsetlinewidth{4.00}
\draw[gp path] (3.896,2.147)--(4.572,2.147);
\node[gp node left,font={\fontsize{10pt}{12pt}\selectfont}] at (1.421,5.166) {\LARGE $B_y$};
\node[gp node left,font={\fontsize{10pt}{12pt}\selectfont}] at (6.147,5.166) {\large $\alpha = \pi$};
\node[gp node left,font={\fontsize{10pt}{12pt}\selectfont}] at (4.797,3.695) {\large exact};
\node[gp node left,font={\fontsize{10pt}{12pt}\selectfont}] at (4.797,2.921) {\large non-converging};
\node[gp node left,font={\fontsize{10pt}{12pt}\selectfont}] at (4.797,2.147) {\large non-converging};
\node[gp node left,font={\fontsize{10pt}{12pt}\selectfont}] at (4.797,1.856) {\large wave};
%% coordinates of the plot area
\gpdefrectangularnode{gp plot 1}{\pgfpoint{1.196cm}{0.985cm}}{\pgfpoint{7.947cm}{5.631cm}}
\end{tikzpicture}
%% gnuplot variables
}
\end{tabular}
\caption{The slow compound wave of Test~5a, approximate non-converging c-solution, and exact r-solution for the coplanar case.}
\label{fig:coplanar_a_cwaves}
\end{figure}

%----------------------------------------------------------------
% Coplanar waves
%-----------------------------------------------------------------
\begin{figure}[htbp] 
\begin{tabular}{cc}
\resizebox{0.5\linewidth}{!}{\tikzsetnextfilename{fast_coplanar_a_cwaves_1}\begin{tikzpicture}[gnuplot]
%% generated with GNUPLOT 4.6p4 (Lua 5.1; terminal rev. 99, script rev. 100)
%% Sat 02 Aug 2014 10:05:19 AM EDT
\path (0.000,0.000) rectangle (8.500,6.000);
\gpfill{rgb color={1.000,1.000,1.000}} (1.196,0.985)--(7.946,0.985)--(7.946,5.630)--(1.196,5.630)--cycle;
\gpcolor{color=gp lt color border}
\gpsetlinetype{gp lt border}
\gpsetlinewidth{1.00}
\draw[gp path] (1.196,0.985)--(1.196,5.630)--(7.946,5.630)--(7.946,0.985)--cycle;
\gpcolor{color=gp lt color axes}
\gpsetlinetype{gp lt axes}
\gpsetlinewidth{2.00}
\draw[gp path] (1.196,1.275)--(7.947,1.275);
\gpcolor{color=gp lt color border}
\gpsetlinetype{gp lt border}
\draw[gp path] (1.196,1.275)--(1.268,1.275);
\draw[gp path] (7.947,1.275)--(7.875,1.275);
\gpcolor{rgb color={0.000,0.000,0.000}}
\node[gp node right,font={\fontsize{10pt}{12pt}\selectfont}] at (1.012,1.275) {0.65};
\gpcolor{color=gp lt color axes}
\gpsetlinetype{gp lt axes}
\draw[gp path] (1.196,2.001)--(7.947,2.001);
\gpcolor{color=gp lt color border}
\gpsetlinetype{gp lt border}
\draw[gp path] (1.196,2.001)--(1.268,2.001);
\draw[gp path] (7.947,2.001)--(7.875,2.001);
\gpcolor{rgb color={0.000,0.000,0.000}}
\node[gp node right,font={\fontsize{10pt}{12pt}\selectfont}] at (1.012,2.001) {0.7};
\gpcolor{color=gp lt color axes}
\gpsetlinetype{gp lt axes}
\draw[gp path] (1.196,2.727)--(7.947,2.727);
\gpcolor{color=gp lt color border}
\gpsetlinetype{gp lt border}
\draw[gp path] (1.196,2.727)--(1.268,2.727);
\draw[gp path] (7.947,2.727)--(7.875,2.727);
\gpcolor{rgb color={0.000,0.000,0.000}}
\node[gp node right,font={\fontsize{10pt}{12pt}\selectfont}] at (1.012,2.727) {0.75};
\gpcolor{color=gp lt color axes}
\gpsetlinetype{gp lt axes}
\draw[gp path] (1.196,3.453)--(7.947,3.453);
\gpcolor{color=gp lt color border}
\gpsetlinetype{gp lt border}
\draw[gp path] (1.196,3.453)--(1.268,3.453);
\draw[gp path] (7.947,3.453)--(7.875,3.453);
\gpcolor{rgb color={0.000,0.000,0.000}}
\node[gp node right,font={\fontsize{10pt}{12pt}\selectfont}] at (1.012,3.453) {0.8};
\gpcolor{color=gp lt color axes}
\gpsetlinetype{gp lt axes}
\draw[gp path] (1.196,4.179)--(7.947,4.179);
\gpcolor{color=gp lt color border}
\gpsetlinetype{gp lt border}
\draw[gp path] (1.196,4.179)--(1.268,4.179);
\draw[gp path] (7.947,4.179)--(7.875,4.179);
\gpcolor{rgb color={0.000,0.000,0.000}}
\node[gp node right,font={\fontsize{10pt}{12pt}\selectfont}] at (1.012,4.179) {0.85};
\gpcolor{color=gp lt color axes}
\gpsetlinetype{gp lt axes}
\draw[gp path] (1.196,4.905)--(7.947,4.905);
\gpcolor{color=gp lt color border}
\gpsetlinetype{gp lt border}
\draw[gp path] (1.196,4.905)--(1.268,4.905);
\draw[gp path] (7.947,4.905)--(7.875,4.905);
\gpcolor{rgb color={0.000,0.000,0.000}}
\node[gp node right,font={\fontsize{10pt}{12pt}\selectfont}] at (1.012,4.905) {0.9};
\gpcolor{color=gp lt color axes}
\gpsetlinetype{gp lt axes}
\draw[gp path] (1.196,5.631)--(7.947,5.631);
\gpcolor{color=gp lt color border}
\gpsetlinetype{gp lt border}
\draw[gp path] (1.196,5.631)--(1.268,5.631);
\draw[gp path] (7.947,5.631)--(7.875,5.631);
\gpcolor{rgb color={0.000,0.000,0.000}}
\node[gp node right,font={\fontsize{10pt}{12pt}\selectfont}] at (1.012,5.631) {0.95};
\gpcolor{color=gp lt color axes}
\gpsetlinetype{gp lt axes}
\draw[gp path] (1.196,0.985)--(1.196,5.631);
\gpcolor{color=gp lt color border}
\gpsetlinetype{gp lt border}
\draw[gp path] (1.196,0.985)--(1.196,1.057);
\draw[gp path] (1.196,5.631)--(1.196,5.559);
\gpcolor{rgb color={0.000,0.000,0.000}}
\node[gp node center,font={\fontsize{10pt}{12pt}\selectfont}] at (1.196,0.677) {0.3};
\gpcolor{color=gp lt color axes}
\gpsetlinetype{gp lt axes}
\draw[gp path] (2.494,0.985)--(2.494,5.631);
\gpcolor{color=gp lt color border}
\gpsetlinetype{gp lt border}
\draw[gp path] (2.494,0.985)--(2.494,1.057);
\draw[gp path] (2.494,5.631)--(2.494,5.559);
\gpcolor{rgb color={0.000,0.000,0.000}}
\node[gp node center,font={\fontsize{10pt}{12pt}\selectfont}] at (2.494,0.677) {0.35};
\gpcolor{color=gp lt color axes}
\gpsetlinetype{gp lt axes}
\draw[gp path] (3.793,0.985)--(3.793,5.631);
\gpcolor{color=gp lt color border}
\gpsetlinetype{gp lt border}
\draw[gp path] (3.793,0.985)--(3.793,1.057);
\draw[gp path] (3.793,5.631)--(3.793,5.559);
\gpcolor{rgb color={0.000,0.000,0.000}}
\node[gp node center,font={\fontsize{10pt}{12pt}\selectfont}] at (3.793,0.677) {0.4};
\gpcolor{color=gp lt color axes}
\gpsetlinetype{gp lt axes}
\draw[gp path] (5.091,0.985)--(5.091,5.631);
\gpcolor{color=gp lt color border}
\gpsetlinetype{gp lt border}
\draw[gp path] (5.091,0.985)--(5.091,1.057);
\draw[gp path] (5.091,5.631)--(5.091,5.559);
\gpcolor{rgb color={0.000,0.000,0.000}}
\node[gp node center,font={\fontsize{10pt}{12pt}\selectfont}] at (5.091,0.677) {0.45};
\gpcolor{color=gp lt color axes}
\gpsetlinetype{gp lt axes}
\draw[gp path] (6.389,0.985)--(6.389,5.631);
\gpcolor{color=gp lt color border}
\gpsetlinetype{gp lt border}
\draw[gp path] (6.389,0.985)--(6.389,1.057);
\draw[gp path] (6.389,5.631)--(6.389,5.559);
\gpcolor{rgb color={0.000,0.000,0.000}}
\node[gp node center,font={\fontsize{10pt}{12pt}\selectfont}] at (6.389,0.677) {0.5};
\gpcolor{color=gp lt color axes}
\gpsetlinetype{gp lt axes}
\draw[gp path] (7.687,0.985)--(7.687,5.631);
\gpcolor{color=gp lt color border}
\gpsetlinetype{gp lt border}
\draw[gp path] (7.687,0.985)--(7.687,1.057);
\draw[gp path] (7.687,5.631)--(7.687,5.559);
\gpcolor{rgb color={0.000,0.000,0.000}}
\node[gp node center,font={\fontsize{10pt}{12pt}\selectfont}] at (7.687,0.677) {0.55};
\gpcolor{color=gp lt color border}
\draw[gp path] (1.196,5.631)--(1.196,0.985)--(7.947,0.985)--(7.947,5.631)--cycle;
\gpcolor{rgb color={0.000,0.000,0.000}}
\node[gp node center,font={\fontsize{10pt}{12pt}\selectfont}] at (4.571,0.215) {\large $x$};
\gpcolor{rgb color={1.000,0.000,0.000}}
\gpsetlinewidth{0.50}
\gpsetpointsize{4.44}
\gppoint{gp mark 7}{(1.229,3.613)}
\gppoint{gp mark 7}{(1.254,3.582)}
\gppoint{gp mark 7}{(1.280,3.551)}
\gppoint{gp mark 7}{(1.305,3.520)}
\gppoint{gp mark 7}{(1.330,3.489)}
\gppoint{gp mark 7}{(1.356,3.458)}
\gppoint{gp mark 7}{(1.381,3.428)}
\gppoint{gp mark 7}{(1.406,3.397)}
\gppoint{gp mark 7}{(1.432,3.367)}
\gppoint{gp mark 7}{(1.457,3.336)}
\gppoint{gp mark 7}{(1.483,3.305)}
\gppoint{gp mark 7}{(1.508,3.275)}
\gppoint{gp mark 7}{(1.533,3.245)}
\gppoint{gp mark 7}{(1.559,3.214)}
\gppoint{gp mark 7}{(1.584,3.184)}
\gppoint{gp mark 7}{(1.609,3.154)}
\gppoint{gp mark 7}{(1.635,3.123)}
\gppoint{gp mark 7}{(1.660,3.093)}
\gppoint{gp mark 7}{(1.685,3.063)}
\gppoint{gp mark 7}{(1.711,3.033)}
\gppoint{gp mark 7}{(1.736,3.003)}
\gppoint{gp mark 7}{(1.761,2.973)}
\gppoint{gp mark 7}{(1.787,2.943)}
\gppoint{gp mark 7}{(1.812,2.913)}
\gppoint{gp mark 7}{(1.838,2.883)}
\gppoint{gp mark 7}{(1.863,2.853)}
\gppoint{gp mark 7}{(1.888,2.824)}
\gppoint{gp mark 7}{(1.914,2.794)}
\gppoint{gp mark 7}{(1.939,2.764)}
\gppoint{gp mark 7}{(1.964,2.735)}
\gppoint{gp mark 7}{(1.990,2.705)}
\gppoint{gp mark 7}{(2.015,2.675)}
\gppoint{gp mark 7}{(2.040,2.646)}
\gppoint{gp mark 7}{(2.066,2.616)}
\gppoint{gp mark 7}{(2.091,2.587)}
\gppoint{gp mark 7}{(2.116,2.557)}
\gppoint{gp mark 7}{(2.142,2.528)}
\gppoint{gp mark 7}{(2.167,2.499)}
\gppoint{gp mark 7}{(2.193,2.469)}
\gppoint{gp mark 7}{(2.218,2.440)}
\gppoint{gp mark 7}{(2.243,2.411)}
\gppoint{gp mark 7}{(2.269,2.382)}
\gppoint{gp mark 7}{(2.294,2.352)}
\gppoint{gp mark 7}{(2.319,2.323)}
\gppoint{gp mark 7}{(2.345,2.294)}
\gppoint{gp mark 7}{(2.370,2.265)}
\gppoint{gp mark 7}{(2.395,2.236)}
\gppoint{gp mark 7}{(2.421,2.207)}
\gppoint{gp mark 7}{(2.446,2.178)}
\gppoint{gp mark 7}{(2.471,2.149)}
\gppoint{gp mark 7}{(2.497,2.120)}
\gppoint{gp mark 7}{(2.522,2.091)}
\gppoint{gp mark 7}{(2.548,2.062)}
\gppoint{gp mark 7}{(2.573,2.034)}
\gppoint{gp mark 7}{(2.598,2.005)}
\gppoint{gp mark 7}{(2.624,1.977)}
\gppoint{gp mark 7}{(2.649,1.949)}
\gppoint{gp mark 7}{(2.674,1.921)}
\gppoint{gp mark 7}{(2.700,1.893)}
\gppoint{gp mark 7}{(2.725,1.866)}
\gppoint{gp mark 7}{(2.750,1.840)}
\gppoint{gp mark 7}{(2.776,1.815)}
\gppoint{gp mark 7}{(2.801,1.791)}
\gppoint{gp mark 7}{(2.826,1.768)}
\gppoint{gp mark 7}{(2.852,1.743)}
\gppoint{gp mark 7}{(2.877,1.700)}
\gppoint{gp mark 7}{(2.903,1.594)}
\gppoint{gp mark 7}{(2.928,1.522)}
\gppoint{gp mark 7}{(2.953,1.812)}
\gppoint{gp mark 7}{(2.979,2.286)}
\gppoint{gp mark 7}{(3.004,2.342)}
\gppoint{gp mark 7}{(3.029,2.337)}
\gppoint{gp mark 7}{(3.055,2.335)}
\gppoint{gp mark 7}{(3.080,2.335)}
\gppoint{gp mark 7}{(3.105,2.334)}
\gppoint{gp mark 7}{(3.131,2.336)}
\gppoint{gp mark 7}{(3.156,2.336)}
\gppoint{gp mark 7}{(3.181,2.336)}
\gppoint{gp mark 7}{(3.207,2.336)}
\gppoint{gp mark 7}{(3.232,2.337)}
\gppoint{gp mark 7}{(3.258,2.340)}
\gppoint{gp mark 7}{(3.283,2.342)}
\gppoint{gp mark 7}{(3.308,2.343)}
\gppoint{gp mark 7}{(3.334,2.342)}
\gppoint{gp mark 7}{(3.359,2.341)}
\gppoint{gp mark 7}{(3.384,2.341)}
\gppoint{gp mark 7}{(3.410,2.341)}
\gppoint{gp mark 7}{(3.435,2.341)}
\gppoint{gp mark 7}{(3.460,2.339)}
\gppoint{gp mark 7}{(3.486,2.336)}
\gppoint{gp mark 7}{(3.511,2.334)}
\gppoint{gp mark 7}{(3.536,2.334)}
\gppoint{gp mark 7}{(3.562,2.334)}
\gppoint{gp mark 7}{(3.587,2.334)}
\gppoint{gp mark 7}{(3.613,2.333)}
\gppoint{gp mark 7}{(3.638,2.332)}
\gppoint{gp mark 7}{(3.663,2.335)}
\gppoint{gp mark 7}{(3.689,2.401)}
\gppoint{gp mark 7}{(3.714,2.994)}
\gppoint{gp mark 7}{(3.739,4.260)}
\gppoint{gp mark 7}{(3.765,4.643)}
\gppoint{gp mark 7}{(3.790,4.662)}
\gppoint{gp mark 7}{(3.815,4.656)}
\gppoint{gp mark 7}{(3.841,4.656)}
\gppoint{gp mark 7}{(3.866,4.658)}
\gppoint{gp mark 7}{(3.891,4.657)}
\gppoint{gp mark 7}{(3.917,4.657)}
\gppoint{gp mark 7}{(3.942,4.658)}
\gppoint{gp mark 7}{(3.968,4.659)}
\gppoint{gp mark 7}{(3.993,4.658)}
\gppoint{gp mark 7}{(4.018,4.657)}
\gppoint{gp mark 7}{(4.044,4.657)}
\gppoint{gp mark 7}{(4.069,4.657)}
\gppoint{gp mark 7}{(4.094,4.657)}
\gppoint{gp mark 7}{(4.120,4.658)}
\gppoint{gp mark 7}{(4.145,4.658)}
\gppoint{gp mark 7}{(4.170,4.658)}
\gppoint{gp mark 7}{(4.196,4.658)}
\gppoint{gp mark 7}{(4.221,4.658)}
\gppoint{gp mark 7}{(4.246,4.658)}
\gppoint{gp mark 7}{(4.272,4.657)}
\gppoint{gp mark 7}{(4.297,4.657)}
\gppoint{gp mark 7}{(4.322,4.657)}
\gppoint{gp mark 7}{(4.348,4.657)}
\gppoint{gp mark 7}{(4.373,4.658)}
\gppoint{gp mark 7}{(4.399,4.658)}
\gppoint{gp mark 7}{(4.424,4.658)}
\gppoint{gp mark 7}{(4.449,4.658)}
\gppoint{gp mark 7}{(4.475,4.658)}
\gppoint{gp mark 7}{(4.500,4.658)}
\gppoint{gp mark 7}{(4.525,4.657)}
\gppoint{gp mark 7}{(4.551,4.657)}
\gppoint{gp mark 7}{(4.576,4.657)}
\gppoint{gp mark 7}{(4.601,4.657)}
\gppoint{gp mark 7}{(4.627,4.658)}
\gppoint{gp mark 7}{(4.652,4.659)}
\gppoint{gp mark 7}{(4.677,4.659)}
\gppoint{gp mark 7}{(4.703,4.659)}
\gppoint{gp mark 7}{(4.728,4.658)}
\gppoint{gp mark 7}{(4.754,4.658)}
\gppoint{gp mark 7}{(4.779,4.658)}
\gppoint{gp mark 7}{(4.804,4.658)}
\gppoint{gp mark 7}{(4.830,4.658)}
\gppoint{gp mark 7}{(4.855,4.658)}
\gppoint{gp mark 7}{(4.880,4.658)}
\gppoint{gp mark 7}{(4.906,4.658)}
\gppoint{gp mark 7}{(4.931,4.658)}
\gppoint{gp mark 7}{(4.956,4.658)}
\gppoint{gp mark 7}{(4.982,4.658)}
\gppoint{gp mark 7}{(5.007,4.658)}
\gppoint{gp mark 7}{(5.032,4.658)}
\gppoint{gp mark 7}{(5.058,4.658)}
\gppoint{gp mark 7}{(5.083,4.659)}
\gppoint{gp mark 7}{(5.109,4.659)}
\gppoint{gp mark 7}{(5.134,4.659)}
\gppoint{gp mark 7}{(5.159,4.659)}
\gppoint{gp mark 7}{(5.185,4.659)}
\gppoint{gp mark 7}{(5.210,4.659)}
\gppoint{gp mark 7}{(5.235,4.658)}
\gppoint{gp mark 7}{(5.261,4.658)}
\gppoint{gp mark 7}{(5.286,4.658)}
\gppoint{gp mark 7}{(5.311,4.658)}
\gppoint{gp mark 7}{(5.337,4.658)}
\gppoint{gp mark 7}{(5.362,4.658)}
\gppoint{gp mark 7}{(5.387,4.658)}
\gppoint{gp mark 7}{(5.413,4.658)}
\gppoint{gp mark 7}{(5.438,4.658)}
\gppoint{gp mark 7}{(5.464,4.658)}
\gppoint{gp mark 7}{(5.489,4.658)}
\gppoint{gp mark 7}{(5.514,4.657)}
\gppoint{gp mark 7}{(5.540,4.657)}
\gppoint{gp mark 7}{(5.565,4.657)}
\gppoint{gp mark 7}{(5.590,4.657)}
\gppoint{gp mark 7}{(5.616,4.656)}
\gppoint{gp mark 7}{(5.641,4.656)}
\gppoint{gp mark 7}{(5.666,4.656)}
\gppoint{gp mark 7}{(5.692,4.656)}
\gppoint{gp mark 7}{(5.717,4.656)}
\gppoint{gp mark 7}{(5.742,4.656)}
\gppoint{gp mark 7}{(5.768,4.656)}
\gppoint{gp mark 7}{(5.793,4.655)}
\gppoint{gp mark 7}{(5.819,4.655)}
\gppoint{gp mark 7}{(5.844,4.655)}
\gppoint{gp mark 7}{(5.869,4.655)}
\gppoint{gp mark 7}{(5.895,4.655)}
\gppoint{gp mark 7}{(5.920,4.655)}
\gppoint{gp mark 7}{(5.945,4.655)}
\gppoint{gp mark 7}{(5.971,4.654)}
\gppoint{gp mark 7}{(5.996,4.654)}
\gppoint{gp mark 7}{(6.021,4.654)}
\gppoint{gp mark 7}{(6.047,4.654)}
\gppoint{gp mark 7}{(6.072,4.654)}
\gppoint{gp mark 7}{(6.097,4.654)}
\gppoint{gp mark 7}{(6.123,4.654)}
\gppoint{gp mark 7}{(6.148,4.653)}
\gppoint{gp mark 7}{(6.174,4.653)}
\gppoint{gp mark 7}{(6.199,4.653)}
\gppoint{gp mark 7}{(6.224,4.653)}
\gppoint{gp mark 7}{(6.250,4.653)}
\gppoint{gp mark 7}{(6.275,4.652)}
\gppoint{gp mark 7}{(6.300,4.652)}
\gppoint{gp mark 7}{(6.326,4.652)}
\gppoint{gp mark 7}{(6.351,4.652)}
\gppoint{gp mark 7}{(6.376,4.653)}
\gppoint{gp mark 7}{(6.402,4.653)}
\gppoint{gp mark 7}{(6.427,4.653)}
\gppoint{gp mark 7}{(6.452,4.654)}
\gppoint{gp mark 7}{(6.478,4.654)}
\gppoint{gp mark 7}{(6.503,4.653)}
\gppoint{gp mark 7}{(6.529,4.653)}
\gppoint{gp mark 7}{(6.554,4.653)}
\gppoint{gp mark 7}{(6.579,4.653)}
\gppoint{gp mark 7}{(6.605,4.654)}
\gppoint{gp mark 7}{(6.630,4.654)}
\gppoint{gp mark 7}{(6.655,4.654)}
\gppoint{gp mark 7}{(6.681,4.654)}
\gppoint{gp mark 7}{(6.706,4.654)}
\gppoint{gp mark 7}{(6.731,4.655)}
\gppoint{gp mark 7}{(6.757,4.657)}
\gppoint{gp mark 7}{(6.782,4.660)}
\gppoint{gp mark 7}{(6.807,4.661)}
\gppoint{gp mark 7}{(6.833,4.662)}
\gppoint{gp mark 7}{(6.858,4.662)}
\gppoint{gp mark 7}{(6.884,4.662)}
\gppoint{gp mark 7}{(6.909,4.663)}
\gppoint{gp mark 7}{(6.934,4.665)}
\gppoint{gp mark 7}{(6.960,4.667)}
\gppoint{gp mark 7}{(6.985,4.668)}
\gppoint{gp mark 7}{(7.010,4.668)}
\gppoint{gp mark 7}{(7.036,4.668)}
\gppoint{gp mark 7}{(7.061,4.668)}
\gppoint{gp mark 7}{(7.086,4.668)}
\gppoint{gp mark 7}{(7.112,4.666)}
\gppoint{gp mark 7}{(7.137,4.664)}
\gppoint{gp mark 7}{(7.162,4.662)}
\gppoint{gp mark 7}{(7.188,4.659)}
\gppoint{gp mark 7}{(7.213,4.656)}
\gppoint{gp mark 7}{(7.239,4.651)}
\gppoint{gp mark 7}{(7.264,4.645)}
\gppoint{gp mark 7}{(7.289,4.639)}
\gppoint{gp mark 7}{(7.315,4.632)}
\gppoint{gp mark 7}{(7.340,4.623)}
\gppoint{gp mark 7}{(7.365,4.614)}
\gppoint{gp mark 7}{(7.391,4.608)}
\gppoint{gp mark 7}{(7.416,4.603)}
\gppoint{gp mark 7}{(7.441,4.598)}
\gppoint{gp mark 7}{(7.467,4.591)}
\gppoint{gp mark 7}{(7.492,4.582)}
\gppoint{gp mark 7}{(7.517,4.574)}
\gppoint{gp mark 7}{(7.543,4.570)}
\gppoint{gp mark 7}{(7.568,4.569)}
\gppoint{gp mark 7}{(7.594,4.567)}
\gppoint{gp mark 7}{(7.619,4.559)}
\gppoint{gp mark 7}{(7.644,4.504)}
\gppoint{gp mark 7}{(7.670,4.157)}
\gppoint{gp mark 7}{(7.695,3.305)}
\gppoint{gp mark 7}{(7.720,2.308)}
\gppoint{gp mark 7}{(7.746,1.605)}
\gppoint{gp mark 7}{(7.771,1.279)}
\gppoint{gp mark 7}{(7.796,1.174)}
\gppoint{gp mark 7}{(7.822,1.150)}
\gppoint{gp mark 7}{(7.847,1.147)}
\gppoint{gp mark 7}{(7.872,1.146)}
\gppoint{gp mark 7}{(7.898,1.146)}
\gppoint{gp mark 7}{(7.923,1.148)}
\gpcolor{rgb color={0.000,0.000,0.000}}
\gpsetlinetype{gp lt plot 0}
\gpsetlinewidth{4.00}
\draw[gp path] (2.440,2.203)--(2.899,2.203);
\draw[gp path] (2.899,2.203)--(3.805,2.203);
\draw[gp path] (3.805,4.694)--(7.720,4.694);
\draw[gp path] (7.720,1.153)--(7.947,1.153);
\draw[gp path] (1.204,3.442)--(1.217,3.426)--(1.230,3.411)--(1.243,3.396)--(1.256,3.380)%
  --(1.269,3.365)--(1.282,3.350)--(1.295,3.334)--(1.308,3.319)--(1.321,3.304)--(1.334,3.289)%
  --(1.347,3.274)--(1.360,3.259)--(1.373,3.244)--(1.386,3.229)--(1.399,3.214)--(1.412,3.199)%
  --(1.425,3.184)--(1.438,3.169)--(1.451,3.155)--(1.464,3.140)--(1.477,3.125)--(1.490,3.111)%
  --(1.503,3.096)--(1.516,3.082)--(1.529,3.067)--(1.542,3.053)--(1.555,3.038)--(1.568,3.024)%
  --(1.581,3.010)--(1.594,2.996)--(1.607,2.982)--(1.620,2.967)--(1.633,2.953)--(1.646,2.939)%
  --(1.659,2.925)--(1.672,2.911)--(1.686,2.898)--(1.699,2.884)--(1.712,2.870)--(1.725,2.856)%
  --(1.738,2.843)--(1.751,2.829)--(1.764,2.815)--(1.777,2.802)--(1.790,2.788)--(1.803,2.775)%
  --(1.816,2.762)--(1.829,2.748)--(1.842,2.735)--(1.855,2.722)--(1.868,2.709)--(1.881,2.696)%
  --(1.894,2.683)--(1.907,2.670)--(1.920,2.657)--(1.933,2.644)--(1.946,2.631)--(1.959,2.618)%
  --(1.972,2.605)--(1.985,2.593)--(1.998,2.580)--(2.011,2.567)--(2.024,2.555)--(2.037,2.542)%
  --(2.050,2.530)--(2.063,2.518)--(2.076,2.505)--(2.089,2.493)--(2.102,2.481)--(2.115,2.469)%
  --(2.128,2.457)--(2.141,2.445)--(2.154,2.433)--(2.167,2.421)--(2.180,2.409)--(2.193,2.397)%
  --(2.206,2.386)--(2.219,2.374)--(2.232,2.362)--(2.245,2.351)--(2.258,2.339)--(2.271,2.328)%
  --(2.284,2.317)--(2.297,2.305)--(2.310,2.294)--(2.323,2.283)--(2.336,2.272)--(2.349,2.261)%
  --(2.362,2.250)--(2.375,2.239)--(2.388,2.228)--(2.401,2.217)--(2.414,2.206)--(2.427,2.196)%
  --(2.440,2.203);
\draw[gp path] (3.805,2.203)--(3.805,4.694);
\draw[gp path] (7.720,4.694)--(7.720,1.153);
\gpsetlinetype{gp lt plot 2}
\draw[gp path] (1.229,2.203)--(1.254,2.203)--(1.280,2.203)--(1.305,2.203)--(1.330,2.203)%
  --(1.356,2.203)--(1.381,2.203)--(1.406,2.203)--(1.432,2.203)--(1.457,2.203)--(1.483,2.203)%
  --(1.508,2.203)--(1.533,2.203)--(1.559,2.203)--(1.584,2.203)--(1.609,2.203)--(1.635,2.203)%
  --(1.660,2.203)--(1.685,2.203)--(1.711,2.203)--(1.736,2.203)--(1.761,2.203)--(1.787,2.203)%
  --(1.812,2.203)--(1.838,2.203)--(1.863,2.203)--(1.888,2.203)--(1.914,2.203)--(1.939,2.203)%
  --(1.964,2.203)--(1.990,2.203)--(2.015,2.203)--(2.040,2.203)--(2.066,2.203)--(2.091,2.203)%
  --(2.116,2.203)--(2.142,2.203)--(2.167,2.203)--(2.193,2.203)--(2.218,2.203)--(2.243,2.203)%
  --(2.269,2.203)--(2.294,2.203)--(2.319,2.202)--(2.345,2.202)--(2.370,2.200)--(2.395,2.193)%
  --(2.421,2.180)--(2.446,2.162)--(2.471,2.141)--(2.497,2.117)--(2.522,2.091)--(2.548,2.066)%
  --(2.573,2.039)--(2.598,2.013)--(2.624,1.986)--(2.649,1.960)--(2.674,1.933)--(2.700,1.908)%
  --(2.725,1.882)--(2.750,1.859)--(2.776,1.837)--(2.801,1.817)--(2.826,1.799)--(2.852,1.778)%
  --(2.877,1.729)--(2.903,1.600)--(2.928,1.558)--(2.953,1.985)--(2.979,2.333)--(3.004,2.347)%
  --(3.029,2.345)--(3.055,2.345)--(3.080,2.348)--(3.105,2.349)--(3.131,2.347)--(3.156,2.348)%
  --(3.181,2.348)--(3.207,2.346)--(3.232,2.344)--(3.258,2.344)--(3.283,2.344)--(3.308,2.344)%
  --(3.334,2.344)--(3.359,2.346)--(3.384,2.347)--(3.410,2.348)--(3.435,2.348)--(3.460,2.348)%
  --(3.486,2.347)--(3.511,2.347)--(3.536,2.346)--(3.562,2.345)--(3.587,2.344)--(3.613,2.344)%
  --(3.638,2.343)--(3.663,2.343)--(3.689,2.344)--(3.714,2.344)--(3.739,2.343)--(3.765,2.343)%
  --(3.790,2.343)--(3.815,2.345)--(3.841,2.348)--(3.866,2.351)--(3.891,2.354)--(3.917,2.355)%
  --(3.942,2.356)--(3.968,2.357)--(3.993,2.356)--(4.018,2.353)--(4.044,2.349)--(4.069,2.346)%
  --(4.094,2.345)--(4.120,2.349)--(4.145,2.370)--(4.170,2.433)--(4.196,2.534)--(4.221,2.618)%
  --(4.246,2.637)--(4.272,2.636)--(4.297,2.616)--(4.322,2.573)--(4.348,2.516)--(4.373,2.460)%
  --(4.399,2.408)--(4.424,2.360)--(4.449,2.317)--(4.475,2.277)--(4.500,2.240)--(4.525,2.206)%
  --(4.551,2.179)--(4.576,2.165)--(4.601,2.163)--(4.627,2.161)--(4.652,2.163)--(4.677,2.167)%
  --(4.703,2.171)--(4.728,2.173)--(4.754,2.174)--(4.779,2.175)--(4.804,2.176)--(4.830,2.176)%
  --(4.855,2.176)--(4.880,2.175)--(4.906,2.175)--(4.931,2.174)--(4.956,2.174)--(4.982,2.173)%
  --(5.007,2.174)--(5.032,2.174)--(5.058,2.175)--(5.083,2.176)--(5.109,2.176)--(5.134,2.176)%
  --(5.159,2.175)--(5.185,2.175)--(5.210,2.174)--(5.235,2.173)--(5.261,2.173)--(5.286,2.173)%
  --(5.311,2.173)--(5.337,2.174)--(5.362,2.175)--(5.387,2.176)--(5.413,2.176)--(5.438,2.176)%
  --(5.464,2.176)--(5.489,2.175)--(5.514,2.174)--(5.540,2.173)--(5.565,2.173)--(5.590,2.173)%
  --(5.616,2.174)--(5.641,2.175)--(5.666,2.175)--(5.692,2.176)--(5.717,2.176)--(5.742,2.175)%
  --(5.768,2.174)--(5.793,2.173)--(5.819,2.173)--(5.844,2.173)--(5.869,2.173)--(5.895,2.174)%
  --(5.920,2.175)--(5.945,2.176)--(5.971,2.176)--(5.996,2.176)--(6.021,2.176)--(6.047,2.176)%
  --(6.072,2.174)--(6.097,2.174)--(6.123,2.173)--(6.148,2.173)--(6.174,2.173)--(6.199,2.174)%
  --(6.224,2.175)--(6.250,2.175)--(6.275,2.175)--(6.300,2.175)--(6.326,2.175)--(6.351,2.174)%
  --(6.376,2.174)--(6.402,2.173)--(6.427,2.173)--(6.452,2.173)--(6.478,2.174)--(6.503,2.175)%
  --(6.529,2.175)--(6.554,2.176)--(6.579,2.176)--(6.605,2.175)--(6.630,2.175)--(6.655,2.173)%
  --(6.681,2.172)--(6.706,2.172)--(6.731,2.172)--(6.757,2.172)--(6.782,2.173)--(6.807,2.174)%
  --(6.833,2.175)--(6.858,2.175)--(6.884,2.175)--(6.909,2.175)--(6.934,2.174)--(6.960,2.173)%
  --(6.985,2.172)--(7.010,2.172)--(7.036,2.172)--(7.061,2.173)--(7.086,2.173)--(7.112,2.174)%
  --(7.137,2.174)--(7.162,2.174)--(7.188,2.174)--(7.213,2.174)--(7.239,2.173)--(7.264,2.172)%
  --(7.289,2.172)--(7.315,2.172)--(7.340,2.172)--(7.365,2.172)--(7.391,2.173)--(7.416,2.174)%
  --(7.441,2.174)--(7.467,2.174)--(7.492,2.174)--(7.517,2.174)--(7.543,2.174)--(7.568,2.173)%
  --(7.594,2.173)--(7.619,2.173)--(7.644,2.173)--(7.670,2.173)--(7.695,2.173)--(7.720,2.173)%
  --(7.746,2.173)--(7.771,2.172)--(7.796,2.172)--(7.822,2.172)--(7.847,2.172)--(7.872,2.172)%
  --(7.898,2.171)--(7.923,2.171);
\node[gp node left,font={\fontsize{10pt}{12pt}\selectfont}] at (1.456,5.268) {\LARGE $\rho$};
\node[gp node left,font={\fontsize{10pt}{12pt}\selectfont}] at (5.740,5.268) {\large $\alpha = \pi$};
%% coordinates of the plot area
\gpdefrectangularnode{gp plot 1}{\pgfpoint{1.196cm}{0.985cm}}{\pgfpoint{7.947cm}{5.631cm}}
\end{tikzpicture}
%% gnuplot variables
} & 
\resizebox{0.5\linewidth}{!}{\tikzsetnextfilename{fast_coplanar_a_cwaves_6}\begin{tikzpicture}[gnuplot]
%% generated with GNUPLOT 4.6p4 (Lua 5.1; terminal rev. 99, script rev. 100)
%% Sat 02 Aug 2014 10:05:20 AM EDT
\path (0.000,0.000) rectangle (8.500,6.000);
\gpfill{rgb color={1.000,1.000,1.000}} (1.196,0.985)--(7.946,0.985)--(7.946,5.630)--(1.196,5.630)--cycle;
\gpcolor{color=gp lt color border}
\gpsetlinetype{gp lt border}
\gpsetlinewidth{1.00}
\draw[gp path] (1.196,0.985)--(1.196,5.630)--(7.946,5.630)--(7.946,0.985)--cycle;
\gpcolor{color=gp lt color axes}
\gpsetlinetype{gp lt axes}
\gpsetlinewidth{2.00}
\draw[gp path] (1.196,0.985)--(7.947,0.985);
\gpcolor{color=gp lt color border}
\gpsetlinetype{gp lt border}
\draw[gp path] (1.196,0.985)--(1.268,0.985);
\draw[gp path] (7.947,0.985)--(7.875,0.985);
\gpcolor{rgb color={0.000,0.000,0.000}}
\node[gp node right,font={\fontsize{10pt}{12pt}\selectfont}] at (1.012,0.985) {-0.4};
\gpcolor{color=gp lt color axes}
\gpsetlinetype{gp lt axes}
\draw[gp path] (1.196,1.759)--(7.947,1.759);
\gpcolor{color=gp lt color border}
\gpsetlinetype{gp lt border}
\draw[gp path] (1.196,1.759)--(1.268,1.759);
\draw[gp path] (7.947,1.759)--(7.875,1.759);
\gpcolor{rgb color={0.000,0.000,0.000}}
\node[gp node right,font={\fontsize{10pt}{12pt}\selectfont}] at (1.012,1.759) {-0.2};
\gpcolor{color=gp lt color axes}
\gpsetlinetype{gp lt axes}
\draw[gp path] (1.196,2.534)--(7.947,2.534);
\gpcolor{color=gp lt color border}
\gpsetlinetype{gp lt border}
\draw[gp path] (1.196,2.534)--(1.268,2.534);
\draw[gp path] (7.947,2.534)--(7.875,2.534);
\gpcolor{rgb color={0.000,0.000,0.000}}
\node[gp node right,font={\fontsize{10pt}{12pt}\selectfont}] at (1.012,2.534) {0};
\gpcolor{color=gp lt color axes}
\gpsetlinetype{gp lt axes}
\draw[gp path] (1.196,3.308)--(7.947,3.308);
\gpcolor{color=gp lt color border}
\gpsetlinetype{gp lt border}
\draw[gp path] (1.196,3.308)--(1.268,3.308);
\draw[gp path] (7.947,3.308)--(7.875,3.308);
\gpcolor{rgb color={0.000,0.000,0.000}}
\node[gp node right,font={\fontsize{10pt}{12pt}\selectfont}] at (1.012,3.308) {0.2};
\gpcolor{color=gp lt color axes}
\gpsetlinetype{gp lt axes}
\draw[gp path] (1.196,4.082)--(7.947,4.082);
\gpcolor{color=gp lt color border}
\gpsetlinetype{gp lt border}
\draw[gp path] (1.196,4.082)--(1.268,4.082);
\draw[gp path] (7.947,4.082)--(7.875,4.082);
\gpcolor{rgb color={0.000,0.000,0.000}}
\node[gp node right,font={\fontsize{10pt}{12pt}\selectfont}] at (1.012,4.082) {0.4};
\gpcolor{color=gp lt color axes}
\gpsetlinetype{gp lt axes}
\draw[gp path] (1.196,4.857)--(7.947,4.857);
\gpcolor{color=gp lt color border}
\gpsetlinetype{gp lt border}
\draw[gp path] (1.196,4.857)--(1.268,4.857);
\draw[gp path] (7.947,4.857)--(7.875,4.857);
\gpcolor{rgb color={0.000,0.000,0.000}}
\node[gp node right,font={\fontsize{10pt}{12pt}\selectfont}] at (1.012,4.857) {0.6};
\gpcolor{color=gp lt color axes}
\gpsetlinetype{gp lt axes}
\draw[gp path] (1.196,5.631)--(7.947,5.631);
\gpcolor{color=gp lt color border}
\gpsetlinetype{gp lt border}
\draw[gp path] (1.196,5.631)--(1.268,5.631);
\draw[gp path] (7.947,5.631)--(7.875,5.631);
\gpcolor{rgb color={0.000,0.000,0.000}}
\node[gp node right,font={\fontsize{10pt}{12pt}\selectfont}] at (1.012,5.631) {0.8};
\gpcolor{color=gp lt color axes}
\gpsetlinetype{gp lt axes}
\draw[gp path] (1.196,0.985)--(1.196,5.631);
\gpcolor{color=gp lt color border}
\gpsetlinetype{gp lt border}
\draw[gp path] (1.196,0.985)--(1.196,1.057);
\draw[gp path] (1.196,5.631)--(1.196,5.559);
\gpcolor{rgb color={0.000,0.000,0.000}}
\node[gp node center,font={\fontsize{10pt}{12pt}\selectfont}] at (1.196,0.677) {0.3};
\gpcolor{color=gp lt color axes}
\gpsetlinetype{gp lt axes}
\draw[gp path] (2.494,0.985)--(2.494,5.631);
\gpcolor{color=gp lt color border}
\gpsetlinetype{gp lt border}
\draw[gp path] (2.494,0.985)--(2.494,1.057);
\draw[gp path] (2.494,5.631)--(2.494,5.559);
\gpcolor{rgb color={0.000,0.000,0.000}}
\node[gp node center,font={\fontsize{10pt}{12pt}\selectfont}] at (2.494,0.677) {0.35};
\gpcolor{color=gp lt color axes}
\gpsetlinetype{gp lt axes}
\draw[gp path] (3.793,0.985)--(3.793,5.631);
\gpcolor{color=gp lt color border}
\gpsetlinetype{gp lt border}
\draw[gp path] (3.793,0.985)--(3.793,1.057);
\draw[gp path] (3.793,5.631)--(3.793,5.559);
\gpcolor{rgb color={0.000,0.000,0.000}}
\node[gp node center,font={\fontsize{10pt}{12pt}\selectfont}] at (3.793,0.677) {0.4};
\gpcolor{color=gp lt color axes}
\gpsetlinetype{gp lt axes}
\draw[gp path] (5.091,0.985)--(5.091,5.631);
\gpcolor{color=gp lt color border}
\gpsetlinetype{gp lt border}
\draw[gp path] (5.091,0.985)--(5.091,1.057);
\draw[gp path] (5.091,5.631)--(5.091,5.559);
\gpcolor{rgb color={0.000,0.000,0.000}}
\node[gp node center,font={\fontsize{10pt}{12pt}\selectfont}] at (5.091,0.677) {0.45};
\gpcolor{color=gp lt color axes}
\gpsetlinetype{gp lt axes}
\draw[gp path] (6.389,0.985)--(6.389,5.631);
\gpcolor{color=gp lt color border}
\gpsetlinetype{gp lt border}
\draw[gp path] (6.389,0.985)--(6.389,1.057);
\draw[gp path] (6.389,5.631)--(6.389,5.559);
\gpcolor{rgb color={0.000,0.000,0.000}}
\node[gp node center,font={\fontsize{10pt}{12pt}\selectfont}] at (6.389,0.677) {0.5};
\gpcolor{color=gp lt color axes}
\gpsetlinetype{gp lt axes}
\draw[gp path] (7.687,0.985)--(7.687,5.631);
\gpcolor{color=gp lt color border}
\gpsetlinetype{gp lt border}
\draw[gp path] (7.687,0.985)--(7.687,1.057);
\draw[gp path] (7.687,5.631)--(7.687,5.559);
\gpcolor{rgb color={0.000,0.000,0.000}}
\node[gp node center,font={\fontsize{10pt}{12pt}\selectfont}] at (7.687,0.677) {0.55};
\gpcolor{color=gp lt color border}
\draw[gp path] (1.196,5.631)--(1.196,0.985)--(7.947,0.985)--(7.947,5.631)--cycle;
\gpcolor{rgb color={0.000,0.000,0.000}}
\node[gp node center,font={\fontsize{10pt}{12pt}\selectfont}] at (4.571,0.215) {\large $x$};
\gpcolor{rgb color={1.000,0.000,0.000}}
\gpsetlinewidth{0.50}
\gpsetpointsize{4.44}
\gppoint{gp mark 7}{(1.229,4.938)}
\gppoint{gp mark 7}{(1.254,4.919)}
\gppoint{gp mark 7}{(1.280,4.900)}
\gppoint{gp mark 7}{(1.305,4.880)}
\gppoint{gp mark 7}{(1.330,4.861)}
\gppoint{gp mark 7}{(1.356,4.842)}
\gppoint{gp mark 7}{(1.381,4.822)}
\gppoint{gp mark 7}{(1.406,4.803)}
\gppoint{gp mark 7}{(1.432,4.783)}
\gppoint{gp mark 7}{(1.457,4.764)}
\gppoint{gp mark 7}{(1.483,4.744)}
\gppoint{gp mark 7}{(1.508,4.724)}
\gppoint{gp mark 7}{(1.533,4.704)}
\gppoint{gp mark 7}{(1.559,4.684)}
\gppoint{gp mark 7}{(1.584,4.664)}
\gppoint{gp mark 7}{(1.609,4.644)}
\gppoint{gp mark 7}{(1.635,4.623)}
\gppoint{gp mark 7}{(1.660,4.603)}
\gppoint{gp mark 7}{(1.685,4.583)}
\gppoint{gp mark 7}{(1.711,4.562)}
\gppoint{gp mark 7}{(1.736,4.541)}
\gppoint{gp mark 7}{(1.761,4.520)}
\gppoint{gp mark 7}{(1.787,4.499)}
\gppoint{gp mark 7}{(1.812,4.478)}
\gppoint{gp mark 7}{(1.838,4.457)}
\gppoint{gp mark 7}{(1.863,4.436)}
\gppoint{gp mark 7}{(1.888,4.414)}
\gppoint{gp mark 7}{(1.914,4.393)}
\gppoint{gp mark 7}{(1.939,4.371)}
\gppoint{gp mark 7}{(1.964,4.349)}
\gppoint{gp mark 7}{(1.990,4.327)}
\gppoint{gp mark 7}{(2.015,4.304)}
\gppoint{gp mark 7}{(2.040,4.282)}
\gppoint{gp mark 7}{(2.066,4.259)}
\gppoint{gp mark 7}{(2.091,4.236)}
\gppoint{gp mark 7}{(2.116,4.213)}
\gppoint{gp mark 7}{(2.142,4.190)}
\gppoint{gp mark 7}{(2.167,4.167)}
\gppoint{gp mark 7}{(2.193,4.143)}
\gppoint{gp mark 7}{(2.218,4.119)}
\gppoint{gp mark 7}{(2.243,4.095)}
\gppoint{gp mark 7}{(2.269,4.070)}
\gppoint{gp mark 7}{(2.294,4.046)}
\gppoint{gp mark 7}{(2.319,4.020)}
\gppoint{gp mark 7}{(2.345,3.995)}
\gppoint{gp mark 7}{(2.370,3.969)}
\gppoint{gp mark 7}{(2.395,3.943)}
\gppoint{gp mark 7}{(2.421,3.917)}
\gppoint{gp mark 7}{(2.446,3.890)}
\gppoint{gp mark 7}{(2.471,3.863)}
\gppoint{gp mark 7}{(2.497,3.835)}
\gppoint{gp mark 7}{(2.522,3.807)}
\gppoint{gp mark 7}{(2.548,3.779)}
\gppoint{gp mark 7}{(2.573,3.750)}
\gppoint{gp mark 7}{(2.598,3.720)}
\gppoint{gp mark 7}{(2.624,3.690)}
\gppoint{gp mark 7}{(2.649,3.660)}
\gppoint{gp mark 7}{(2.674,3.629)}
\gppoint{gp mark 7}{(2.700,3.598)}
\gppoint{gp mark 7}{(2.725,3.567)}
\gppoint{gp mark 7}{(2.750,3.536)}
\gppoint{gp mark 7}{(2.776,3.505)}
\gppoint{gp mark 7}{(2.801,3.475)}
\gppoint{gp mark 7}{(2.826,3.445)}
\gppoint{gp mark 7}{(2.852,3.411)}
\gppoint{gp mark 7}{(2.877,3.357)}
\gppoint{gp mark 7}{(2.903,3.228)}
\gppoint{gp mark 7}{(2.928,2.829)}
\gppoint{gp mark 7}{(2.953,1.927)}
\gppoint{gp mark 7}{(2.979,1.282)}
\gppoint{gp mark 7}{(3.004,1.199)}
\gppoint{gp mark 7}{(3.029,1.194)}
\gppoint{gp mark 7}{(3.055,1.193)}
\gppoint{gp mark 7}{(3.080,1.192)}
\gppoint{gp mark 7}{(3.105,1.193)}
\gppoint{gp mark 7}{(3.131,1.194)}
\gppoint{gp mark 7}{(3.156,1.193)}
\gppoint{gp mark 7}{(3.181,1.193)}
\gppoint{gp mark 7}{(3.207,1.192)}
\gppoint{gp mark 7}{(3.232,1.191)}
\gppoint{gp mark 7}{(3.258,1.191)}
\gppoint{gp mark 7}{(3.283,1.192)}
\gppoint{gp mark 7}{(3.308,1.192)}
\gppoint{gp mark 7}{(3.334,1.192)}
\gppoint{gp mark 7}{(3.359,1.192)}
\gppoint{gp mark 7}{(3.384,1.193)}
\gppoint{gp mark 7}{(3.410,1.194)}
\gppoint{gp mark 7}{(3.435,1.194)}
\gppoint{gp mark 7}{(3.460,1.194)}
\gppoint{gp mark 7}{(3.486,1.193)}
\gppoint{gp mark 7}{(3.511,1.192)}
\gppoint{gp mark 7}{(3.536,1.191)}
\gppoint{gp mark 7}{(3.562,1.190)}
\gppoint{gp mark 7}{(3.587,1.190)}
\gppoint{gp mark 7}{(3.613,1.190)}
\gppoint{gp mark 7}{(3.638,1.190)}
\gppoint{gp mark 7}{(3.663,1.191)}
\gppoint{gp mark 7}{(3.689,1.205)}
\gppoint{gp mark 7}{(3.714,1.314)}
\gppoint{gp mark 7}{(3.739,1.537)}
\gppoint{gp mark 7}{(3.765,1.605)}
\gppoint{gp mark 7}{(3.790,1.609)}
\gppoint{gp mark 7}{(3.815,1.610)}
\gppoint{gp mark 7}{(3.841,1.610)}
\gppoint{gp mark 7}{(3.866,1.609)}
\gppoint{gp mark 7}{(3.891,1.609)}
\gppoint{gp mark 7}{(3.917,1.609)}
\gppoint{gp mark 7}{(3.942,1.609)}
\gppoint{gp mark 7}{(3.968,1.610)}
\gppoint{gp mark 7}{(3.993,1.610)}
\gppoint{gp mark 7}{(4.018,1.610)}
\gppoint{gp mark 7}{(4.044,1.610)}
\gppoint{gp mark 7}{(4.069,1.610)}
\gppoint{gp mark 7}{(4.094,1.610)}
\gppoint{gp mark 7}{(4.120,1.610)}
\gppoint{gp mark 7}{(4.145,1.610)}
\gppoint{gp mark 7}{(4.170,1.609)}
\gppoint{gp mark 7}{(4.196,1.609)}
\gppoint{gp mark 7}{(4.221,1.609)}
\gppoint{gp mark 7}{(4.246,1.609)}
\gppoint{gp mark 7}{(4.272,1.610)}
\gppoint{gp mark 7}{(4.297,1.609)}
\gppoint{gp mark 7}{(4.322,1.609)}
\gppoint{gp mark 7}{(4.348,1.609)}
\gppoint{gp mark 7}{(4.373,1.609)}
\gppoint{gp mark 7}{(4.399,1.610)}
\gppoint{gp mark 7}{(4.424,1.610)}
\gppoint{gp mark 7}{(4.449,1.609)}
\gppoint{gp mark 7}{(4.475,1.609)}
\gppoint{gp mark 7}{(4.500,1.609)}
\gppoint{gp mark 7}{(4.525,1.609)}
\gppoint{gp mark 7}{(4.551,1.609)}
\gppoint{gp mark 7}{(4.576,1.609)}
\gppoint{gp mark 7}{(4.601,1.609)}
\gppoint{gp mark 7}{(4.627,1.609)}
\gppoint{gp mark 7}{(4.652,1.609)}
\gppoint{gp mark 7}{(4.677,1.609)}
\gppoint{gp mark 7}{(4.703,1.609)}
\gppoint{gp mark 7}{(4.728,1.609)}
\gppoint{gp mark 7}{(4.754,1.609)}
\gppoint{gp mark 7}{(4.779,1.609)}
\gppoint{gp mark 7}{(4.804,1.609)}
\gppoint{gp mark 7}{(4.830,1.609)}
\gppoint{gp mark 7}{(4.855,1.609)}
\gppoint{gp mark 7}{(4.880,1.609)}
\gppoint{gp mark 7}{(4.906,1.609)}
\gppoint{gp mark 7}{(4.931,1.609)}
\gppoint{gp mark 7}{(4.956,1.609)}
\gppoint{gp mark 7}{(4.982,1.609)}
\gppoint{gp mark 7}{(5.007,1.609)}
\gppoint{gp mark 7}{(5.032,1.610)}
\gppoint{gp mark 7}{(5.058,1.610)}
\gppoint{gp mark 7}{(5.083,1.610)}
\gppoint{gp mark 7}{(5.109,1.610)}
\gppoint{gp mark 7}{(5.134,1.610)}
\gppoint{gp mark 7}{(5.159,1.610)}
\gppoint{gp mark 7}{(5.185,1.610)}
\gppoint{gp mark 7}{(5.210,1.609)}
\gppoint{gp mark 7}{(5.235,1.609)}
\gppoint{gp mark 7}{(5.261,1.609)}
\gppoint{gp mark 7}{(5.286,1.609)}
\gppoint{gp mark 7}{(5.311,1.609)}
\gppoint{gp mark 7}{(5.337,1.609)}
\gppoint{gp mark 7}{(5.362,1.609)}
\gppoint{gp mark 7}{(5.387,1.609)}
\gppoint{gp mark 7}{(5.413,1.609)}
\gppoint{gp mark 7}{(5.438,1.609)}
\gppoint{gp mark 7}{(5.464,1.609)}
\gppoint{gp mark 7}{(5.489,1.609)}
\gppoint{gp mark 7}{(5.514,1.609)}
\gppoint{gp mark 7}{(5.540,1.609)}
\gppoint{gp mark 7}{(5.565,1.609)}
\gppoint{gp mark 7}{(5.590,1.609)}
\gppoint{gp mark 7}{(5.616,1.609)}
\gppoint{gp mark 7}{(5.641,1.609)}
\gppoint{gp mark 7}{(5.666,1.609)}
\gppoint{gp mark 7}{(5.692,1.609)}
\gppoint{gp mark 7}{(5.717,1.609)}
\gppoint{gp mark 7}{(5.742,1.609)}
\gppoint{gp mark 7}{(5.768,1.609)}
\gppoint{gp mark 7}{(5.793,1.609)}
\gppoint{gp mark 7}{(5.819,1.609)}
\gppoint{gp mark 7}{(5.844,1.609)}
\gppoint{gp mark 7}{(5.869,1.609)}
\gppoint{gp mark 7}{(5.895,1.609)}
\gppoint{gp mark 7}{(5.920,1.609)}
\gppoint{gp mark 7}{(5.945,1.609)}
\gppoint{gp mark 7}{(5.971,1.609)}
\gppoint{gp mark 7}{(5.996,1.609)}
\gppoint{gp mark 7}{(6.021,1.609)}
\gppoint{gp mark 7}{(6.047,1.609)}
\gppoint{gp mark 7}{(6.072,1.609)}
\gppoint{gp mark 7}{(6.097,1.609)}
\gppoint{gp mark 7}{(6.123,1.609)}
\gppoint{gp mark 7}{(6.148,1.609)}
\gppoint{gp mark 7}{(6.174,1.610)}
\gppoint{gp mark 7}{(6.199,1.610)}
\gppoint{gp mark 7}{(6.224,1.610)}
\gppoint{gp mark 7}{(6.250,1.609)}
\gppoint{gp mark 7}{(6.275,1.609)}
\gppoint{gp mark 7}{(6.300,1.609)}
\gppoint{gp mark 7}{(6.326,1.609)}
\gppoint{gp mark 7}{(6.351,1.609)}
\gppoint{gp mark 7}{(6.376,1.609)}
\gppoint{gp mark 7}{(6.402,1.609)}
\gppoint{gp mark 7}{(6.427,1.609)}
\gppoint{gp mark 7}{(6.452,1.609)}
\gppoint{gp mark 7}{(6.478,1.609)}
\gppoint{gp mark 7}{(6.503,1.609)}
\gppoint{gp mark 7}{(6.529,1.609)}
\gppoint{gp mark 7}{(6.554,1.609)}
\gppoint{gp mark 7}{(6.579,1.609)}
\gppoint{gp mark 7}{(6.605,1.609)}
\gppoint{gp mark 7}{(6.630,1.609)}
\gppoint{gp mark 7}{(6.655,1.609)}
\gppoint{gp mark 7}{(6.681,1.609)}
\gppoint{gp mark 7}{(6.706,1.609)}
\gppoint{gp mark 7}{(6.731,1.609)}
\gppoint{gp mark 7}{(6.757,1.609)}
\gppoint{gp mark 7}{(6.782,1.609)}
\gppoint{gp mark 7}{(6.807,1.609)}
\gppoint{gp mark 7}{(6.833,1.609)}
\gppoint{gp mark 7}{(6.858,1.609)}
\gppoint{gp mark 7}{(6.884,1.609)}
\gppoint{gp mark 7}{(6.909,1.609)}
\gppoint{gp mark 7}{(6.934,1.609)}
\gppoint{gp mark 7}{(6.960,1.609)}
\gppoint{gp mark 7}{(6.985,1.609)}
\gppoint{gp mark 7}{(7.010,1.609)}
\gppoint{gp mark 7}{(7.036,1.609)}
\gppoint{gp mark 7}{(7.061,1.609)}
\gppoint{gp mark 7}{(7.086,1.609)}
\gppoint{gp mark 7}{(7.112,1.609)}
\gppoint{gp mark 7}{(7.137,1.609)}
\gppoint{gp mark 7}{(7.162,1.609)}
\gppoint{gp mark 7}{(7.188,1.609)}
\gppoint{gp mark 7}{(7.213,1.609)}
\gppoint{gp mark 7}{(7.239,1.609)}
\gppoint{gp mark 7}{(7.264,1.609)}
\gppoint{gp mark 7}{(7.289,1.609)}
\gppoint{gp mark 7}{(7.315,1.609)}
\gppoint{gp mark 7}{(7.340,1.609)}
\gppoint{gp mark 7}{(7.365,1.609)}
\gppoint{gp mark 7}{(7.391,1.609)}
\gppoint{gp mark 7}{(7.416,1.609)}
\gppoint{gp mark 7}{(7.441,1.609)}
\gppoint{gp mark 7}{(7.467,1.609)}
\gppoint{gp mark 7}{(7.492,1.609)}
\gppoint{gp mark 7}{(7.517,1.609)}
\gppoint{gp mark 7}{(7.543,1.609)}
\gppoint{gp mark 7}{(7.568,1.609)}
\gppoint{gp mark 7}{(7.594,1.609)}
\gppoint{gp mark 7}{(7.619,1.609)}
\gppoint{gp mark 7}{(7.644,1.609)}
\gppoint{gp mark 7}{(7.670,1.609)}
\gppoint{gp mark 7}{(7.695,1.609)}
\gppoint{gp mark 7}{(7.720,1.609)}
\gppoint{gp mark 7}{(7.746,1.609)}
\gppoint{gp mark 7}{(7.771,1.609)}
\gppoint{gp mark 7}{(7.796,1.609)}
\gppoint{gp mark 7}{(7.822,1.609)}
\gppoint{gp mark 7}{(7.847,1.609)}
\gppoint{gp mark 7}{(7.872,1.609)}
\gppoint{gp mark 7}{(7.898,1.609)}
\gppoint{gp mark 7}{(7.923,1.609)}
\gpcolor{rgb color={0.000,0.000,0.000}}
\gpsetlinetype{gp lt plot 0}
\gpsetlinewidth{4.00}
\draw[gp path] (2.440,3.897)--(2.899,3.897);
\draw[gp path] (2.899,1.171)--(3.805,1.171);
\draw[gp path] (3.805,1.610)--(7.720,1.610);
\draw[gp path] (7.720,1.610)--(7.947,1.610);
\draw[gp path] (1.204,4.831)--(1.217,4.821)--(1.230,4.812)--(1.243,4.802)--(1.256,4.792)%
  --(1.269,4.782)--(1.282,4.772)--(1.295,4.762)--(1.308,4.753)--(1.321,4.743)--(1.334,4.733)%
  --(1.347,4.723)--(1.360,4.713)--(1.373,4.703)--(1.386,4.694)--(1.399,4.684)--(1.412,4.674)%
  --(1.425,4.664)--(1.438,4.654)--(1.451,4.644)--(1.464,4.635)--(1.477,4.625)--(1.490,4.615)%
  --(1.503,4.605)--(1.516,4.595)--(1.529,4.585)--(1.542,4.576)--(1.555,4.566)--(1.568,4.556)%
  --(1.581,4.546)--(1.594,4.536)--(1.607,4.526)--(1.620,4.517)--(1.633,4.507)--(1.646,4.497)%
  --(1.659,4.487)--(1.672,4.477)--(1.686,4.467)--(1.699,4.457)--(1.712,4.448)--(1.725,4.438)%
  --(1.738,4.428)--(1.751,4.418)--(1.764,4.408)--(1.777,4.398)--(1.790,4.389)--(1.803,4.379)%
  --(1.816,4.369)--(1.829,4.359)--(1.842,4.349)--(1.855,4.339)--(1.868,4.330)--(1.881,4.320)%
  --(1.894,4.310)--(1.907,4.300)--(1.920,4.290)--(1.933,4.280)--(1.946,4.271)--(1.959,4.261)%
  --(1.972,4.251)--(1.985,4.241)--(1.998,4.231)--(2.011,4.221)--(2.024,4.212)--(2.037,4.202)%
  --(2.050,4.192)--(2.063,4.182)--(2.076,4.172)--(2.089,4.162)--(2.102,4.153)--(2.115,4.143)%
  --(2.128,4.133)--(2.141,4.123)--(2.154,4.113)--(2.167,4.103)--(2.180,4.093)--(2.193,4.084)%
  --(2.206,4.074)--(2.219,4.064)--(2.232,4.054)--(2.245,4.044)--(2.258,4.034)--(2.271,4.025)%
  --(2.284,4.015)--(2.297,4.005)--(2.310,3.995)--(2.323,3.985)--(2.336,3.975)--(2.349,3.966)%
  --(2.362,3.956)--(2.375,3.946)--(2.388,3.936)--(2.401,3.926)--(2.414,3.916)--(2.427,3.907)%
  --(2.440,3.897);
\draw[gp path] (2.899,3.897)--(2.899,1.171);
\draw[gp path] (3.805,1.171)--(3.805,1.610);
\gpsetlinetype{gp lt plot 2}
\draw[gp path] (1.229,3.897)--(1.254,3.897)--(1.280,3.897)--(1.305,3.897)--(1.330,3.897)%
  --(1.356,3.897)--(1.381,3.897)--(1.406,3.897)--(1.432,3.897)--(1.457,3.897)--(1.483,3.897)%
  --(1.508,3.897)--(1.533,3.897)--(1.559,3.897)--(1.584,3.897)--(1.609,3.897)--(1.635,3.897)%
  --(1.660,3.897)--(1.685,3.897)--(1.711,3.897)--(1.736,3.897)--(1.761,3.897)--(1.787,3.897)%
  --(1.812,3.897)--(1.838,3.897)--(1.863,3.897)--(1.888,3.897)--(1.914,3.897)--(1.939,3.897)%
  --(1.964,3.897)--(1.990,3.897)--(2.015,3.897)--(2.040,3.897)--(2.066,3.897)--(2.091,3.897)%
  --(2.116,3.897)--(2.142,3.897)--(2.167,3.897)--(2.193,3.897)--(2.218,3.897)--(2.243,3.897)%
  --(2.269,3.897)--(2.294,3.897)--(2.319,3.897)--(2.345,3.896)--(2.370,3.894)--(2.395,3.888)%
  --(2.421,3.876)--(2.446,3.859)--(2.471,3.838)--(2.497,3.815)--(2.522,3.790)--(2.548,3.765)%
  --(2.573,3.738)--(2.598,3.710)--(2.624,3.682)--(2.649,3.654)--(2.674,3.625)--(2.700,3.595)%
  --(2.725,3.566)--(2.750,3.538)--(2.776,3.511)--(2.801,3.487)--(2.826,3.463)--(2.852,3.435)%
  --(2.877,3.381)--(2.903,3.221)--(2.928,2.693)--(2.953,1.715)--(2.979,1.250)--(3.004,1.198)%
  --(3.029,1.197)--(3.055,1.197)--(3.080,1.194)--(3.105,1.194)--(3.131,1.196)--(3.156,1.196)%
  --(3.181,1.195)--(3.207,1.195)--(3.232,1.196)--(3.258,1.195)--(3.283,1.195)--(3.308,1.195)%
  --(3.334,1.195)--(3.359,1.195)--(3.384,1.195)--(3.410,1.195)--(3.435,1.195)--(3.460,1.195)%
  --(3.486,1.196)--(3.511,1.196)--(3.536,1.195)--(3.562,1.195)--(3.587,1.195)--(3.613,1.195)%
  --(3.638,1.195)--(3.663,1.195)--(3.689,1.195)--(3.714,1.195)--(3.739,1.195)--(3.765,1.195)%
  --(3.790,1.195)--(3.815,1.195)--(3.841,1.196)--(3.866,1.196)--(3.891,1.196)--(3.917,1.197)%
  --(3.942,1.197)--(3.968,1.197)--(3.993,1.196)--(4.018,1.196)--(4.044,1.196)--(4.069,1.195)%
  --(4.094,1.195)--(4.120,1.196)--(4.145,1.199)--(4.170,1.209)--(4.196,1.224)--(4.221,1.236)%
  --(4.246,1.239)--(4.272,1.239)--(4.297,1.235)--(4.322,1.229)--(4.348,1.221)--(4.373,1.212)%
  --(4.399,1.205)--(4.424,1.198)--(4.449,1.191)--(4.475,1.185)--(4.500,1.180)--(4.525,1.175)%
  --(4.551,1.171)--(4.576,1.169)--(4.601,1.168)--(4.627,1.168)--(4.652,1.169)--(4.677,1.169)%
  --(4.703,1.170)--(4.728,1.170)--(4.754,1.170)--(4.779,1.170)--(4.804,1.170)--(4.830,1.170)%
  --(4.855,1.170)--(4.880,1.170)--(4.906,1.170)--(4.931,1.170)--(4.956,1.170)--(4.982,1.170)%
  --(5.007,1.170)--(5.032,1.170)--(5.058,1.170)--(5.083,1.170)--(5.109,1.170)--(5.134,1.170)%
  --(5.159,1.170)--(5.185,1.170)--(5.210,1.170)--(5.235,1.170)--(5.261,1.170)--(5.286,1.170)%
  --(5.311,1.170)--(5.337,1.170)--(5.362,1.170)--(5.387,1.170)--(5.413,1.170)--(5.438,1.170)%
  --(5.464,1.170)--(5.489,1.170)--(5.514,1.170)--(5.540,1.170)--(5.565,1.170)--(5.590,1.170)%
  --(5.616,1.170)--(5.641,1.170)--(5.666,1.170)--(5.692,1.170)--(5.717,1.170)--(5.742,1.170)%
  --(5.768,1.170)--(5.793,1.170)--(5.819,1.170)--(5.844,1.170)--(5.869,1.170)--(5.895,1.170)%
  --(5.920,1.170)--(5.945,1.170)--(5.971,1.170)--(5.996,1.170)--(6.021,1.170)--(6.047,1.170)%
  --(6.072,1.170)--(6.097,1.170)--(6.123,1.170)--(6.148,1.170)--(6.174,1.170)--(6.199,1.170)%
  --(6.224,1.170)--(6.250,1.170)--(6.275,1.170)--(6.300,1.170)--(6.326,1.170)--(6.351,1.170)%
  --(6.376,1.170)--(6.402,1.170)--(6.427,1.170)--(6.452,1.170)--(6.478,1.170)--(6.503,1.170)%
  --(6.529,1.170)--(6.554,1.170)--(6.579,1.170)--(6.605,1.170)--(6.630,1.170)--(6.655,1.170)%
  --(6.681,1.170)--(6.706,1.170)--(6.731,1.170)--(6.757,1.170)--(6.782,1.170)--(6.807,1.170)%
  --(6.833,1.170)--(6.858,1.170)--(6.884,1.170)--(6.909,1.170)--(6.934,1.170)--(6.960,1.170)%
  --(6.985,1.170)--(7.010,1.170)--(7.036,1.170)--(7.061,1.170)--(7.086,1.170)--(7.112,1.170)%
  --(7.137,1.170)--(7.162,1.170)--(7.188,1.170)--(7.213,1.170)--(7.239,1.170)--(7.264,1.170)%
  --(7.289,1.170)--(7.315,1.170)--(7.340,1.170)--(7.365,1.170)--(7.391,1.170)--(7.416,1.170)%
  --(7.441,1.170)--(7.467,1.170)--(7.492,1.170)--(7.517,1.170)--(7.543,1.170)--(7.568,1.170)%
  --(7.594,1.170)--(7.619,1.170)--(7.644,1.170)--(7.670,1.170)--(7.695,1.170)--(7.720,1.170)%
  --(7.746,1.170)--(7.771,1.170)--(7.796,1.170)--(7.822,1.170)--(7.847,1.170)--(7.872,1.170)%
  --(7.898,1.170)--(7.923,1.170);
\gpsetlinetype{gp lt plot 0}
\draw[gp path] (3.793,3.695)--(4.572,3.695);
\gpcolor{rgb color={1.000,0.000,0.000}}
\gpsetlinewidth{0.50}
\gppoint{gp mark 7}{(4.182,2.921)}
\gpcolor{rgb color={0.000,0.000,0.000}}
\gpsetlinetype{gp lt plot 2}
\gpsetlinewidth{4.00}
\draw[gp path] (3.793,2.147)--(4.572,2.147);
\node[gp node left,font={\fontsize{10pt}{12pt}\selectfont}] at (1.456,5.166) {\LARGE $B_y$};
\node[gp node left,font={\fontsize{10pt}{12pt}\selectfont}] at (5.740,5.166) {\large $\alpha = \pi$};
\node[gp node left,font={\fontsize{10pt}{12pt}\selectfont}] at (4.831,3.695) {\large exact};
\node[gp node left,font={\fontsize{10pt}{12pt}\selectfont}] at (4.831,2.921) {\large non-converging};
\node[gp node left,font={\fontsize{10pt}{12pt}\selectfont}] at (4.831,2.147) {\large non-converging};
\node[gp node left,font={\fontsize{10pt}{12pt}\selectfont}] at (4.831,1.856) {\large wave};
%% coordinates of the plot area
\gpdefrectangularnode{gp plot 1}{\pgfpoint{1.196cm}{0.985cm}}{\pgfpoint{7.947cm}{5.631cm}}
\end{tikzpicture}
%% gnuplot variables
}
\end{tabular}
\caption{The fast compound wave of Test~6a, approximate non-converging c-solution, and exact r-solution for the coplanar case.}
\label{fig:fast_coplanar_a_cwaves}
\end{figure}

The solution to the reduced coplanar Riemann problem as well as the solution to the full coplanar Riemann problem are shown in Figure~\ref{fig:coplanar_a_cwaves}.  Three left-going structures are visible in the solution to the reduced Riemann problem.  The intermediate shock at $x\approx 0.3$ is directly followed by a slow rarefaction whose head is located at $x\approx 0.3$ and tail at $x\approx 0.32$.  The speed at the head of the slow rarefaction equals the speed of the intermediate wave, $v - c_a$, and the two structures move together forming a compound wave.  The third structure in the compound wave solution is a fast rarefaction that connects the upstream state of the intermediate shock to the initial conditions.  The speed at the head of the fast rarefaction (dotted black line in Figure \ref{fig:coplanar_a_cwaves}) is equal to the speed at the tail of the fast rarefaction in the solution to the full coplanar Riemann problem (solid black line in Figure \ref{fig:coplanar_a_cwaves}).  These two fast rarefactions, one in the regular solution to the full Riemann problem and one in the compound wave, form a single structure in the compound wave solution to the full coplanar Riemann problem.  The compound wave solution also produces a right-going slow rarefaction wave that connects the state downstream of the left-going slow rarefaction wave to the initial right-state.  The change in density through the right-going slow rarefaction is equal to the difference in density between the exact solution and compound wave solution downstream (left) of the contact discontinuity, at $x\approx 0.48$ in Figure~\ref{fig:coplanar_a_cwaves}.  The compound wave solution requires four structures, as opposed to one in the regular solution (i.e., the rotational discontinuity), to connect two intermediate states in full Riemann problem.  

The intermediate states are calculated with the nonlinear solver described in Section~\ref{sec:mhd_exact}.  The accuracy is determined by the number of iterations.  If the number of iterations is not restricted, the jump conditions can be satisfied to near machine precision.  This precision was not used in the solutions presented here, but may be used when the solution is known to contain at most one non-regular structure.  In that case, the exact solution only needs to be calculated once (at the beginning) and the intermediate states can be used for the remainder of the calculation.  If the number of iterations is set to zero, the intermediate states are those found from the HLLD approximate Riemann solver.  We have found convergence to the correct solution to be independent of the number of iterations used in the exact solver.  This is important because it eliminates the need of a nonlinear solver.  Their implementation can be complex and they have the potential to diverge because of the use of Newton's method.  The implementation of CWM is simplified because it uses the HLLD intermediate states.  

It is important to limit the removal of the flux responsible for producing the compound wave in a way that does not affect the solution in other parts of the domain.  In CWM, the flux is modified if the cell is located in a discontinuity region with a near-$180\dsym$ rotation.  Cell $i$ is considered to be in such a discontinuity region when $| \psi_{i+1} - \psi_{i-1}| > \beta_T$ where $\beta_T$ is a threshold value.  Unless otherwise stated, a value of $\beta_T = 2.0 \text{ radians}$ was used.  This criteria ensures that the flux across regular shocks, waves, or contact discontinuities is unchanged and only rotational discontinuities with a large change in $\alpha$ are affected.  Only rotational discontinuities can change the orientation of the tangential magnetic field if the normal component of the magnetic field is nonzero.  If this criteria is met, the intercell flux at $i$ is modified according to $\mbf{F}^{r}_i = \mbf{F}^{r}_i - A \mbf{F}^c_{i}$, where $\mbf{F}^r$ is the flux for the full Riemann problem, $\mbf{F}^c$ is the flux due to the compound wave, and $A$ is a user-specified constant.  The value of $A$ determines the states upstream and downstream of the rotational discontinuity.  We set $A = 0.1$ in the near-coplanar and coplanar problems shown in Figures~\ref{fig:coplanar_b_rsol_init} and~\ref{fig:coplanar_ab_crsol} and set the Courant number less than $0.4$ is used in order to limit oscillations downstream of the rotational discontinuity.  

%-----------------------------------------------------------------
% Near Coplanar waves initial r-solution
%-----------------------------------------------------------------
\begin{figure}[htbp] 
\begin{tabular}{cc}
\resizebox{0.5\linewidth}{!}{\tikzsetnextfilename{coplanar_b_rsol_init_1}\begin{tikzpicture}[gnuplot]
%% generated with GNUPLOT 4.6p4 (Lua 5.1; terminal rev. 99, script rev. 100)
%% Sun 01 Jun 2014 04:12:29 PM EDT
\path (0.000,0.000) rectangle (8.500,6.000);
\gpfill{rgb color={1.000,1.000,1.000}} (1.196,0.985)--(7.946,0.985)--(7.946,5.630)--(1.196,5.630)--cycle;
\gpcolor{color=gp lt color border}
\gpsetlinetype{gp lt border}
\gpsetlinewidth{1.00}
\draw[gp path] (1.196,0.985)--(1.196,5.630)--(7.946,5.630)--(7.946,0.985)--cycle;
\gpcolor{color=gp lt color axes}
\gpsetlinetype{gp lt axes}
\gpsetlinewidth{2.00}
\draw[gp path] (1.196,0.985)--(7.947,0.985);
\gpcolor{color=gp lt color border}
\gpsetlinetype{gp lt border}
\draw[gp path] (1.196,0.985)--(1.268,0.985);
\draw[gp path] (7.947,0.985)--(7.875,0.985);
\gpcolor{rgb color={0.000,0.000,0.000}}
\node[gp node right,font={\fontsize{10pt}{12pt}\selectfont}] at (1.012,0.985) {0.6};
\gpcolor{color=gp lt color axes}
\gpsetlinetype{gp lt axes}
\draw[gp path] (1.196,1.759)--(7.947,1.759);
\gpcolor{color=gp lt color border}
\gpsetlinetype{gp lt border}
\draw[gp path] (1.196,1.759)--(1.268,1.759);
\draw[gp path] (7.947,1.759)--(7.875,1.759);
\gpcolor{rgb color={0.000,0.000,0.000}}
\node[gp node right,font={\fontsize{10pt}{12pt}\selectfont}] at (1.012,1.759) {0.65};
\gpcolor{color=gp lt color axes}
\gpsetlinetype{gp lt axes}
\draw[gp path] (1.196,2.534)--(7.947,2.534);
\gpcolor{color=gp lt color border}
\gpsetlinetype{gp lt border}
\draw[gp path] (1.196,2.534)--(1.268,2.534);
\draw[gp path] (7.947,2.534)--(7.875,2.534);
\gpcolor{rgb color={0.000,0.000,0.000}}
\node[gp node right,font={\fontsize{10pt}{12pt}\selectfont}] at (1.012,2.534) {0.7};
\gpcolor{color=gp lt color axes}
\gpsetlinetype{gp lt axes}
\draw[gp path] (1.196,3.308)--(7.947,3.308);
\gpcolor{color=gp lt color border}
\gpsetlinetype{gp lt border}
\draw[gp path] (1.196,3.308)--(1.268,3.308);
\draw[gp path] (7.947,3.308)--(7.875,3.308);
\gpcolor{rgb color={0.000,0.000,0.000}}
\node[gp node right,font={\fontsize{10pt}{12pt}\selectfont}] at (1.012,3.308) {0.75};
\gpcolor{color=gp lt color axes}
\gpsetlinetype{gp lt axes}
\draw[gp path] (1.196,4.082)--(7.947,4.082);
\gpcolor{color=gp lt color border}
\gpsetlinetype{gp lt border}
\draw[gp path] (1.196,4.082)--(1.268,4.082);
\draw[gp path] (7.947,4.082)--(7.875,4.082);
\gpcolor{rgb color={0.000,0.000,0.000}}
\node[gp node right,font={\fontsize{10pt}{12pt}\selectfont}] at (1.012,4.082) {0.8};
\gpcolor{color=gp lt color axes}
\gpsetlinetype{gp lt axes}
\draw[gp path] (1.196,4.857)--(7.947,4.857);
\gpcolor{color=gp lt color border}
\gpsetlinetype{gp lt border}
\draw[gp path] (1.196,4.857)--(1.268,4.857);
\draw[gp path] (7.947,4.857)--(7.875,4.857);
\gpcolor{rgb color={0.000,0.000,0.000}}
\node[gp node right,font={\fontsize{10pt}{12pt}\selectfont}] at (1.012,4.857) {0.85};
\gpcolor{color=gp lt color axes}
\gpsetlinetype{gp lt axes}
\draw[gp path] (1.196,5.631)--(7.947,5.631);
\gpcolor{color=gp lt color border}
\gpsetlinetype{gp lt border}
\draw[gp path] (1.196,5.631)--(1.268,5.631);
\draw[gp path] (7.947,5.631)--(7.875,5.631);
\gpcolor{rgb color={0.000,0.000,0.000}}
\node[gp node right,font={\fontsize{10pt}{12pt}\selectfont}] at (1.012,5.631) {0.9};
\gpcolor{color=gp lt color axes}
\gpsetlinetype{gp lt axes}
\draw[gp path] (1.196,0.985)--(1.196,5.631);
\gpcolor{color=gp lt color border}
\gpsetlinetype{gp lt border}
\draw[gp path] (1.196,0.985)--(1.196,1.057);
\draw[gp path] (1.196,5.631)--(1.196,5.559);
\gpcolor{rgb color={0.000,0.000,0.000}}
\node[gp node center,font={\fontsize{10pt}{12pt}\selectfont}] at (1.196,0.677) {0.2};
\gpcolor{color=gp lt color axes}
\gpsetlinetype{gp lt axes}
\draw[gp path] (2.321,0.985)--(2.321,5.631);
\gpcolor{color=gp lt color border}
\gpsetlinetype{gp lt border}
\draw[gp path] (2.321,0.985)--(2.321,1.057);
\draw[gp path] (2.321,5.631)--(2.321,5.559);
\gpcolor{rgb color={0.000,0.000,0.000}}
\node[gp node center,font={\fontsize{10pt}{12pt}\selectfont}] at (2.321,0.677) {0.25};
\gpcolor{color=gp lt color axes}
\gpsetlinetype{gp lt axes}
\draw[gp path] (3.446,0.985)--(3.446,5.631);
\gpcolor{color=gp lt color border}
\gpsetlinetype{gp lt border}
\draw[gp path] (3.446,0.985)--(3.446,1.057);
\draw[gp path] (3.446,5.631)--(3.446,5.559);
\gpcolor{rgb color={0.000,0.000,0.000}}
\node[gp node center,font={\fontsize{10pt}{12pt}\selectfont}] at (3.446,0.677) {0.3};
\gpcolor{color=gp lt color axes}
\gpsetlinetype{gp lt axes}
\draw[gp path] (4.572,0.985)--(4.572,5.631);
\gpcolor{color=gp lt color border}
\gpsetlinetype{gp lt border}
\draw[gp path] (4.572,0.985)--(4.572,1.057);
\draw[gp path] (4.572,5.631)--(4.572,5.559);
\gpcolor{rgb color={0.000,0.000,0.000}}
\node[gp node center,font={\fontsize{10pt}{12pt}\selectfont}] at (4.572,0.677) {0.35};
\gpcolor{color=gp lt color axes}
\gpsetlinetype{gp lt axes}
\draw[gp path] (5.697,0.985)--(5.697,5.631);
\gpcolor{color=gp lt color border}
\gpsetlinetype{gp lt border}
\draw[gp path] (5.697,0.985)--(5.697,1.057);
\draw[gp path] (5.697,5.631)--(5.697,5.559);
\gpcolor{rgb color={0.000,0.000,0.000}}
\node[gp node center,font={\fontsize{10pt}{12pt}\selectfont}] at (5.697,0.677) {0.4};
\gpcolor{color=gp lt color axes}
\gpsetlinetype{gp lt axes}
\draw[gp path] (6.822,0.985)--(6.822,5.631);
\gpcolor{color=gp lt color border}
\gpsetlinetype{gp lt border}
\draw[gp path] (6.822,0.985)--(6.822,1.057);
\draw[gp path] (6.822,5.631)--(6.822,5.559);
\gpcolor{rgb color={0.000,0.000,0.000}}
\node[gp node center,font={\fontsize{10pt}{12pt}\selectfont}] at (6.822,0.677) {0.45};
\gpcolor{color=gp lt color axes}
\gpsetlinetype{gp lt axes}
\draw[gp path] (7.947,0.985)--(7.947,5.631);
\gpcolor{color=gp lt color border}
\gpsetlinetype{gp lt border}
\draw[gp path] (7.947,0.985)--(7.947,1.057);
\draw[gp path] (7.947,5.631)--(7.947,5.559);
\gpcolor{rgb color={0.000,0.000,0.000}}
\node[gp node center,font={\fontsize{10pt}{12pt}\selectfont}] at (7.947,0.677) {0.5};
\gpcolor{color=gp lt color border}
\draw[gp path] (1.196,5.631)--(1.196,0.985)--(7.947,0.985)--(7.947,5.631)--cycle;
\gpcolor{rgb color={0.000,0.000,0.000}}
\node[gp node center,font={\fontsize{10pt}{12pt}\selectfont}] at (4.571,0.215) {\large $x$};
\gpcolor{rgb color={1.000,0.000,0.000}}
\gpsetlinewidth{0.50}
\gpsetpointsize{4.44}
\gppoint{gp mark 7}{(1.203,4.167)}
\gppoint{gp mark 7}{(1.209,4.159)}
\gppoint{gp mark 7}{(1.214,4.150)}
\gppoint{gp mark 7}{(1.220,4.142)}
\gppoint{gp mark 7}{(1.225,4.134)}
\gppoint{gp mark 7}{(1.231,4.126)}
\gppoint{gp mark 7}{(1.236,4.118)}
\gppoint{gp mark 7}{(1.242,4.110)}
\gppoint{gp mark 7}{(1.247,4.101)}
\gppoint{gp mark 7}{(1.253,4.093)}
\gppoint{gp mark 7}{(1.258,4.085)}
\gppoint{gp mark 7}{(1.264,4.077)}
\gppoint{gp mark 7}{(1.269,4.069)}
\gppoint{gp mark 7}{(1.275,4.060)}
\gppoint{gp mark 7}{(1.280,4.052)}
\gppoint{gp mark 7}{(1.286,4.044)}
\gppoint{gp mark 7}{(1.291,4.036)}
\gppoint{gp mark 7}{(1.297,4.028)}
\gppoint{gp mark 7}{(1.302,4.020)}
\gppoint{gp mark 7}{(1.308,4.011)}
\gppoint{gp mark 7}{(1.313,4.003)}
\gppoint{gp mark 7}{(1.319,3.995)}
\gppoint{gp mark 7}{(1.324,3.987)}
\gppoint{gp mark 7}{(1.330,3.979)}
\gppoint{gp mark 7}{(1.335,3.971)}
\gppoint{gp mark 7}{(1.340,3.963)}
\gppoint{gp mark 7}{(1.346,3.955)}
\gppoint{gp mark 7}{(1.351,3.946)}
\gppoint{gp mark 7}{(1.357,3.938)}
\gppoint{gp mark 7}{(1.362,3.930)}
\gppoint{gp mark 7}{(1.368,3.922)}
\gppoint{gp mark 7}{(1.373,3.914)}
\gppoint{gp mark 7}{(1.379,3.906)}
\gppoint{gp mark 7}{(1.384,3.898)}
\gppoint{gp mark 7}{(1.390,3.890)}
\gppoint{gp mark 7}{(1.395,3.882)}
\gppoint{gp mark 7}{(1.401,3.873)}
\gppoint{gp mark 7}{(1.406,3.865)}
\gppoint{gp mark 7}{(1.412,3.857)}
\gppoint{gp mark 7}{(1.417,3.849)}
\gppoint{gp mark 7}{(1.423,3.841)}
\gppoint{gp mark 7}{(1.428,3.833)}
\gppoint{gp mark 7}{(1.434,3.825)}
\gppoint{gp mark 7}{(1.439,3.817)}
\gppoint{gp mark 7}{(1.445,3.809)}
\gppoint{gp mark 7}{(1.450,3.801)}
\gppoint{gp mark 7}{(1.456,3.793)}
\gppoint{gp mark 7}{(1.461,3.785)}
\gppoint{gp mark 7}{(1.467,3.777)}
\gppoint{gp mark 7}{(1.472,3.769)}
\gppoint{gp mark 7}{(1.478,3.761)}
\gppoint{gp mark 7}{(1.483,3.752)}
\gppoint{gp mark 7}{(1.489,3.744)}
\gppoint{gp mark 7}{(1.494,3.736)}
\gppoint{gp mark 7}{(1.500,3.728)}
\gppoint{gp mark 7}{(1.505,3.720)}
\gppoint{gp mark 7}{(1.511,3.712)}
\gppoint{gp mark 7}{(1.516,3.704)}
\gppoint{gp mark 7}{(1.522,3.696)}
\gppoint{gp mark 7}{(1.527,3.688)}
\gppoint{gp mark 7}{(1.533,3.680)}
\gppoint{gp mark 7}{(1.538,3.672)}
\gppoint{gp mark 7}{(1.544,3.664)}
\gppoint{gp mark 7}{(1.549,3.656)}
\gppoint{gp mark 7}{(1.555,3.648)}
\gppoint{gp mark 7}{(1.560,3.640)}
\gppoint{gp mark 7}{(1.566,3.632)}
\gppoint{gp mark 7}{(1.571,3.624)}
\gppoint{gp mark 7}{(1.577,3.616)}
\gppoint{gp mark 7}{(1.582,3.608)}
\gppoint{gp mark 7}{(1.588,3.600)}
\gppoint{gp mark 7}{(1.593,3.592)}
\gppoint{gp mark 7}{(1.599,3.584)}
\gppoint{gp mark 7}{(1.604,3.576)}
\gppoint{gp mark 7}{(1.610,3.569)}
\gppoint{gp mark 7}{(1.615,3.561)}
\gppoint{gp mark 7}{(1.621,3.553)}
\gppoint{gp mark 7}{(1.626,3.545)}
\gppoint{gp mark 7}{(1.632,3.537)}
\gppoint{gp mark 7}{(1.637,3.529)}
\gppoint{gp mark 7}{(1.643,3.521)}
\gppoint{gp mark 7}{(1.648,3.513)}
\gppoint{gp mark 7}{(1.654,3.505)}
\gppoint{gp mark 7}{(1.659,3.497)}
\gppoint{gp mark 7}{(1.665,3.489)}
\gppoint{gp mark 7}{(1.670,3.481)}
\gppoint{gp mark 7}{(1.676,3.473)}
\gppoint{gp mark 7}{(1.681,3.465)}
\gppoint{gp mark 7}{(1.687,3.457)}
\gppoint{gp mark 7}{(1.692,3.450)}
\gppoint{gp mark 7}{(1.698,3.442)}
\gppoint{gp mark 7}{(1.703,3.434)}
\gppoint{gp mark 7}{(1.709,3.426)}
\gppoint{gp mark 7}{(1.714,3.418)}
\gppoint{gp mark 7}{(1.720,3.410)}
\gppoint{gp mark 7}{(1.725,3.402)}
\gppoint{gp mark 7}{(1.731,3.394)}
\gppoint{gp mark 7}{(1.736,3.387)}
\gppoint{gp mark 7}{(1.742,3.379)}
\gppoint{gp mark 7}{(1.747,3.371)}
\gppoint{gp mark 7}{(1.753,3.363)}
\gppoint{gp mark 7}{(1.758,3.355)}
\gppoint{gp mark 7}{(1.764,3.347)}
\gppoint{gp mark 7}{(1.769,3.339)}
\gppoint{gp mark 7}{(1.775,3.332)}
\gppoint{gp mark 7}{(1.780,3.324)}
\gppoint{gp mark 7}{(1.786,3.316)}
\gppoint{gp mark 7}{(1.791,3.308)}
\gppoint{gp mark 7}{(1.796,3.300)}
\gppoint{gp mark 7}{(1.802,3.292)}
\gppoint{gp mark 7}{(1.807,3.285)}
\gppoint{gp mark 7}{(1.813,3.277)}
\gppoint{gp mark 7}{(1.818,3.269)}
\gppoint{gp mark 7}{(1.824,3.261)}
\gppoint{gp mark 7}{(1.829,3.253)}
\gppoint{gp mark 7}{(1.835,3.246)}
\gppoint{gp mark 7}{(1.840,3.238)}
\gppoint{gp mark 7}{(1.846,3.230)}
\gppoint{gp mark 7}{(1.851,3.222)}
\gppoint{gp mark 7}{(1.857,3.214)}
\gppoint{gp mark 7}{(1.862,3.207)}
\gppoint{gp mark 7}{(1.868,3.199)}
\gppoint{gp mark 7}{(1.873,3.191)}
\gppoint{gp mark 7}{(1.879,3.183)}
\gppoint{gp mark 7}{(1.884,3.176)}
\gppoint{gp mark 7}{(1.890,3.168)}
\gppoint{gp mark 7}{(1.895,3.160)}
\gppoint{gp mark 7}{(1.901,3.152)}
\gppoint{gp mark 7}{(1.906,3.145)}
\gppoint{gp mark 7}{(1.912,3.137)}
\gppoint{gp mark 7}{(1.917,3.129)}
\gppoint{gp mark 7}{(1.923,3.121)}
\gppoint{gp mark 7}{(1.928,3.114)}
\gppoint{gp mark 7}{(1.934,3.106)}
\gppoint{gp mark 7}{(1.939,3.098)}
\gppoint{gp mark 7}{(1.945,3.091)}
\gppoint{gp mark 7}{(1.950,3.083)}
\gppoint{gp mark 7}{(1.956,3.075)}
\gppoint{gp mark 7}{(1.961,3.068)}
\gppoint{gp mark 7}{(1.967,3.060)}
\gppoint{gp mark 7}{(1.972,3.052)}
\gppoint{gp mark 7}{(1.978,3.045)}
\gppoint{gp mark 7}{(1.983,3.037)}
\gppoint{gp mark 7}{(1.989,3.029)}
\gppoint{gp mark 7}{(1.994,3.022)}
\gppoint{gp mark 7}{(2.000,3.014)}
\gppoint{gp mark 7}{(2.005,3.006)}
\gppoint{gp mark 7}{(2.011,2.999)}
\gppoint{gp mark 7}{(2.016,2.991)}
\gppoint{gp mark 7}{(2.022,2.983)}
\gppoint{gp mark 7}{(2.027,2.976)}
\gppoint{gp mark 7}{(2.033,2.968)}
\gppoint{gp mark 7}{(2.038,2.961)}
\gppoint{gp mark 7}{(2.044,2.953)}
\gppoint{gp mark 7}{(2.049,2.945)}
\gppoint{gp mark 7}{(2.055,2.938)}
\gppoint{gp mark 7}{(2.060,2.930)}
\gppoint{gp mark 7}{(2.066,2.923)}
\gppoint{gp mark 7}{(2.071,2.915)}
\gppoint{gp mark 7}{(2.077,2.908)}
\gppoint{gp mark 7}{(2.082,2.900)}
\gppoint{gp mark 7}{(2.088,2.893)}
\gppoint{gp mark 7}{(2.093,2.885)}
\gppoint{gp mark 7}{(2.099,2.878)}
\gppoint{gp mark 7}{(2.104,2.870)}
\gppoint{gp mark 7}{(2.110,2.863)}
\gppoint{gp mark 7}{(2.115,2.855)}
\gppoint{gp mark 7}{(2.121,2.848)}
\gppoint{gp mark 7}{(2.126,2.840)}
\gppoint{gp mark 7}{(2.132,2.833)}
\gppoint{gp mark 7}{(2.137,2.825)}
\gppoint{gp mark 7}{(2.143,2.818)}
\gppoint{gp mark 7}{(2.148,2.810)}
\gppoint{gp mark 7}{(2.154,2.803)}
\gppoint{gp mark 7}{(2.159,2.795)}
\gppoint{gp mark 7}{(2.165,2.788)}
\gppoint{gp mark 7}{(2.170,2.781)}
\gppoint{gp mark 7}{(2.176,2.773)}
\gppoint{gp mark 7}{(2.181,2.766)}
\gppoint{gp mark 7}{(2.187,2.759)}
\gppoint{gp mark 7}{(2.192,2.751)}
\gppoint{gp mark 7}{(2.198,2.744)}
\gppoint{gp mark 7}{(2.203,2.736)}
\gppoint{gp mark 7}{(2.209,2.729)}
\gppoint{gp mark 7}{(2.214,2.722)}
\gppoint{gp mark 7}{(2.220,2.715)}
\gppoint{gp mark 7}{(2.225,2.707)}
\gppoint{gp mark 7}{(2.231,2.700)}
\gppoint{gp mark 7}{(2.236,2.693)}
\gppoint{gp mark 7}{(2.242,2.685)}
\gppoint{gp mark 7}{(2.247,2.678)}
\gppoint{gp mark 7}{(2.252,2.671)}
\gppoint{gp mark 7}{(2.258,2.664)}
\gppoint{gp mark 7}{(2.263,2.657)}
\gppoint{gp mark 7}{(2.269,2.649)}
\gppoint{gp mark 7}{(2.274,2.642)}
\gppoint{gp mark 7}{(2.280,2.635)}
\gppoint{gp mark 7}{(2.285,2.628)}
\gppoint{gp mark 7}{(2.291,2.621)}
\gppoint{gp mark 7}{(2.296,2.614)}
\gppoint{gp mark 7}{(2.302,2.607)}
\gppoint{gp mark 7}{(2.307,2.600)}
\gppoint{gp mark 7}{(2.313,2.593)}
\gppoint{gp mark 7}{(2.318,2.586)}
\gppoint{gp mark 7}{(2.324,2.578)}
\gppoint{gp mark 7}{(2.329,2.571)}
\gppoint{gp mark 7}{(2.335,2.564)}
\gppoint{gp mark 7}{(2.340,2.558)}
\gppoint{gp mark 7}{(2.346,2.551)}
\gppoint{gp mark 7}{(2.351,2.544)}
\gppoint{gp mark 7}{(2.357,2.537)}
\gppoint{gp mark 7}{(2.362,2.530)}
\gppoint{gp mark 7}{(2.368,2.523)}
\gppoint{gp mark 7}{(2.373,2.516)}
\gppoint{gp mark 7}{(2.379,2.509)}
\gppoint{gp mark 7}{(2.384,2.502)}
\gppoint{gp mark 7}{(2.390,2.496)}
\gppoint{gp mark 7}{(2.395,2.489)}
\gppoint{gp mark 7}{(2.401,2.482)}
\gppoint{gp mark 7}{(2.406,2.476)}
\gppoint{gp mark 7}{(2.412,2.469)}
\gppoint{gp mark 7}{(2.417,2.462)}
\gppoint{gp mark 7}{(2.423,2.456)}
\gppoint{gp mark 7}{(2.428,2.449)}
\gppoint{gp mark 7}{(2.434,2.443)}
\gppoint{gp mark 7}{(2.439,2.437)}
\gppoint{gp mark 7}{(2.445,2.431)}
\gppoint{gp mark 7}{(2.450,2.426)}
\gppoint{gp mark 7}{(2.456,2.422)}
\gppoint{gp mark 7}{(2.461,2.419)}
\gppoint{gp mark 7}{(2.467,2.418)}
\gppoint{gp mark 7}{(2.472,2.418)}
\gppoint{gp mark 7}{(2.478,2.417)}
\gppoint{gp mark 7}{(2.483,2.417)}
\gppoint{gp mark 7}{(2.489,2.418)}
\gppoint{gp mark 7}{(2.494,2.418)}
\gppoint{gp mark 7}{(2.500,2.419)}
\gppoint{gp mark 7}{(2.505,2.421)}
\gppoint{gp mark 7}{(2.511,2.423)}
\gppoint{gp mark 7}{(2.516,2.424)}
\gppoint{gp mark 7}{(2.522,2.424)}
\gppoint{gp mark 7}{(2.527,2.424)}
\gppoint{gp mark 7}{(2.533,2.424)}
\gppoint{gp mark 7}{(2.538,2.424)}
\gppoint{gp mark 7}{(2.544,2.424)}
\gppoint{gp mark 7}{(2.549,2.424)}
\gppoint{gp mark 7}{(2.555,2.424)}
\gppoint{gp mark 7}{(2.560,2.424)}
\gppoint{gp mark 7}{(2.566,2.424)}
\gppoint{gp mark 7}{(2.571,2.424)}
\gppoint{gp mark 7}{(2.577,2.424)}
\gppoint{gp mark 7}{(2.582,2.424)}
\gppoint{gp mark 7}{(2.588,2.424)}
\gppoint{gp mark 7}{(2.593,2.424)}
\gppoint{gp mark 7}{(2.599,2.424)}
\gppoint{gp mark 7}{(2.604,2.424)}
\gppoint{gp mark 7}{(2.610,2.425)}
\gppoint{gp mark 7}{(2.615,2.425)}
\gppoint{gp mark 7}{(2.621,2.425)}
\gppoint{gp mark 7}{(2.626,2.425)}
\gppoint{gp mark 7}{(2.632,2.425)}
\gppoint{gp mark 7}{(2.637,2.425)}
\gppoint{gp mark 7}{(2.643,2.425)}
\gppoint{gp mark 7}{(2.648,2.425)}
\gppoint{gp mark 7}{(2.654,2.425)}
\gppoint{gp mark 7}{(2.659,2.425)}
\gppoint{gp mark 7}{(2.665,2.425)}
\gppoint{gp mark 7}{(2.670,2.425)}
\gppoint{gp mark 7}{(2.676,2.425)}
\gppoint{gp mark 7}{(2.681,2.425)}
\gppoint{gp mark 7}{(2.687,2.425)}
\gppoint{gp mark 7}{(2.692,2.425)}
\gppoint{gp mark 7}{(2.698,2.425)}
\gppoint{gp mark 7}{(2.703,2.425)}
\gppoint{gp mark 7}{(2.708,2.425)}
\gppoint{gp mark 7}{(2.714,2.425)}
\gppoint{gp mark 7}{(2.719,2.425)}
\gppoint{gp mark 7}{(2.725,2.425)}
\gppoint{gp mark 7}{(2.730,2.425)}
\gppoint{gp mark 7}{(2.736,2.425)}
\gppoint{gp mark 7}{(2.741,2.425)}
\gppoint{gp mark 7}{(2.747,2.425)}
\gppoint{gp mark 7}{(2.752,2.425)}
\gppoint{gp mark 7}{(2.758,2.425)}
\gppoint{gp mark 7}{(2.763,2.425)}
\gppoint{gp mark 7}{(2.769,2.425)}
\gppoint{gp mark 7}{(2.774,2.425)}
\gppoint{gp mark 7}{(2.780,2.425)}
\gppoint{gp mark 7}{(2.785,2.425)}
\gppoint{gp mark 7}{(2.791,2.425)}
\gppoint{gp mark 7}{(2.796,2.425)}
\gppoint{gp mark 7}{(2.802,2.426)}
\gppoint{gp mark 7}{(2.807,2.426)}
\gppoint{gp mark 7}{(2.813,2.426)}
\gppoint{gp mark 7}{(2.818,2.426)}
\gppoint{gp mark 7}{(2.824,2.426)}
\gppoint{gp mark 7}{(2.829,2.426)}
\gppoint{gp mark 7}{(2.835,2.426)}
\gppoint{gp mark 7}{(2.840,2.426)}
\gppoint{gp mark 7}{(2.846,2.426)}
\gppoint{gp mark 7}{(2.851,2.426)}
\gppoint{gp mark 7}{(2.857,2.426)}
\gppoint{gp mark 7}{(2.862,2.426)}
\gppoint{gp mark 7}{(2.868,2.426)}
\gppoint{gp mark 7}{(2.873,2.426)}
\gppoint{gp mark 7}{(2.879,2.426)}
\gppoint{gp mark 7}{(2.884,2.426)}
\gppoint{gp mark 7}{(2.890,2.426)}
\gppoint{gp mark 7}{(2.895,2.426)}
\gppoint{gp mark 7}{(2.901,2.426)}
\gppoint{gp mark 7}{(2.906,2.426)}
\gppoint{gp mark 7}{(2.912,2.426)}
\gppoint{gp mark 7}{(2.917,2.426)}
\gppoint{gp mark 7}{(2.923,2.426)}
\gppoint{gp mark 7}{(2.928,2.426)}
\gppoint{gp mark 7}{(2.934,2.426)}
\gppoint{gp mark 7}{(2.939,2.426)}
\gppoint{gp mark 7}{(2.945,2.426)}
\gppoint{gp mark 7}{(2.950,2.426)}
\gppoint{gp mark 7}{(2.956,2.426)}
\gppoint{gp mark 7}{(2.961,2.426)}
\gppoint{gp mark 7}{(2.967,2.426)}
\gppoint{gp mark 7}{(2.972,2.426)}
\gppoint{gp mark 7}{(2.978,2.426)}
\gppoint{gp mark 7}{(2.983,2.426)}
\gppoint{gp mark 7}{(2.989,2.426)}
\gppoint{gp mark 7}{(2.994,2.426)}
\gppoint{gp mark 7}{(3.000,2.426)}
\gppoint{gp mark 7}{(3.005,2.426)}
\gppoint{gp mark 7}{(3.011,2.426)}
\gppoint{gp mark 7}{(3.016,2.426)}
\gppoint{gp mark 7}{(3.022,2.427)}
\gppoint{gp mark 7}{(3.027,2.427)}
\gppoint{gp mark 7}{(3.033,2.427)}
\gppoint{gp mark 7}{(3.038,2.427)}
\gppoint{gp mark 7}{(3.044,2.427)}
\gppoint{gp mark 7}{(3.049,2.427)}
\gppoint{gp mark 7}{(3.055,2.427)}
\gppoint{gp mark 7}{(3.060,2.427)}
\gppoint{gp mark 7}{(3.066,2.427)}
\gppoint{gp mark 7}{(3.071,2.426)}
\gppoint{gp mark 7}{(3.077,2.426)}
\gppoint{gp mark 7}{(3.082,2.426)}
\gppoint{gp mark 7}{(3.088,2.426)}
\gppoint{gp mark 7}{(3.093,2.426)}
\gppoint{gp mark 7}{(3.099,2.426)}
\gppoint{gp mark 7}{(3.104,2.426)}
\gppoint{gp mark 7}{(3.110,2.427)}
\gppoint{gp mark 7}{(3.115,2.427)}
\gppoint{gp mark 7}{(3.121,2.427)}
\gppoint{gp mark 7}{(3.126,2.427)}
\gppoint{gp mark 7}{(3.132,2.427)}
\gppoint{gp mark 7}{(3.137,2.427)}
\gppoint{gp mark 7}{(3.143,2.427)}
\gppoint{gp mark 7}{(3.148,2.427)}
\gppoint{gp mark 7}{(3.154,2.427)}
\gppoint{gp mark 7}{(3.159,2.427)}
\gppoint{gp mark 7}{(3.164,2.427)}
\gppoint{gp mark 7}{(3.170,2.427)}
\gppoint{gp mark 7}{(3.175,2.427)}
\gppoint{gp mark 7}{(3.181,2.427)}
\gppoint{gp mark 7}{(3.186,2.427)}
\gppoint{gp mark 7}{(3.192,2.427)}
\gppoint{gp mark 7}{(3.197,2.427)}
\gppoint{gp mark 7}{(3.203,2.427)}
\gppoint{gp mark 7}{(3.208,2.427)}
\gppoint{gp mark 7}{(3.214,2.427)}
\gppoint{gp mark 7}{(3.219,2.427)}
\gppoint{gp mark 7}{(3.225,2.427)}
\gppoint{gp mark 7}{(3.230,2.427)}
\gppoint{gp mark 7}{(3.236,2.427)}
\gppoint{gp mark 7}{(3.241,2.427)}
\gppoint{gp mark 7}{(3.247,2.427)}
\gppoint{gp mark 7}{(3.252,2.427)}
\gppoint{gp mark 7}{(3.258,2.427)}
\gppoint{gp mark 7}{(3.263,2.427)}
\gppoint{gp mark 7}{(3.269,2.427)}
\gppoint{gp mark 7}{(3.274,2.427)}
\gppoint{gp mark 7}{(3.280,2.427)}
\gppoint{gp mark 7}{(3.285,2.427)}
\gppoint{gp mark 7}{(3.291,2.427)}
\gppoint{gp mark 7}{(3.296,2.427)}
\gppoint{gp mark 7}{(3.302,2.427)}
\gppoint{gp mark 7}{(3.307,2.427)}
\gppoint{gp mark 7}{(3.313,2.427)}
\gppoint{gp mark 7}{(3.318,2.427)}
\gppoint{gp mark 7}{(3.324,2.427)}
\gppoint{gp mark 7}{(3.329,2.427)}
\gppoint{gp mark 7}{(3.335,2.427)}
\gppoint{gp mark 7}{(3.340,2.427)}
\gppoint{gp mark 7}{(3.346,2.427)}
\gppoint{gp mark 7}{(3.351,2.428)}
\gppoint{gp mark 7}{(3.357,2.428)}
\gppoint{gp mark 7}{(3.362,2.428)}
\gppoint{gp mark 7}{(3.368,2.428)}
\gppoint{gp mark 7}{(3.373,2.428)}
\gppoint{gp mark 7}{(3.379,2.428)}
\gppoint{gp mark 7}{(3.384,2.428)}
\gppoint{gp mark 7}{(3.390,2.428)}
\gppoint{gp mark 7}{(3.395,2.427)}
\gppoint{gp mark 7}{(3.401,2.427)}
\gppoint{gp mark 7}{(3.406,2.427)}
\gppoint{gp mark 7}{(3.412,2.427)}
\gppoint{gp mark 7}{(3.417,2.427)}
\gppoint{gp mark 7}{(3.423,2.427)}
\gppoint{gp mark 7}{(3.428,2.427)}
\gppoint{gp mark 7}{(3.434,2.428)}
\gppoint{gp mark 7}{(3.439,2.428)}
\gppoint{gp mark 7}{(3.445,2.428)}
\gppoint{gp mark 7}{(3.450,2.428)}
\gppoint{gp mark 7}{(3.456,2.428)}
\gppoint{gp mark 7}{(3.461,2.427)}
\gppoint{gp mark 7}{(3.467,2.427)}
\gppoint{gp mark 7}{(3.472,2.427)}
\gppoint{gp mark 7}{(3.478,2.427)}
\gppoint{gp mark 7}{(3.483,2.427)}
\gppoint{gp mark 7}{(3.489,2.428)}
\gppoint{gp mark 7}{(3.494,2.428)}
\gppoint{gp mark 7}{(3.500,2.428)}
\gppoint{gp mark 7}{(3.505,2.428)}
\gppoint{gp mark 7}{(3.511,2.432)}
\gppoint{gp mark 7}{(3.516,2.445)}
\gppoint{gp mark 7}{(3.522,2.964)}
\gppoint{gp mark 7}{(3.527,3.316)}
\gppoint{gp mark 7}{(3.533,3.058)}
\gppoint{gp mark 7}{(3.538,2.510)}
\gppoint{gp mark 7}{(3.544,2.499)}
\gppoint{gp mark 7}{(3.549,2.462)}
\gppoint{gp mark 7}{(3.555,2.450)}
\gppoint{gp mark 7}{(3.560,2.449)}
\gppoint{gp mark 7}{(3.566,2.448)}
\gppoint{gp mark 7}{(3.571,2.454)}
\gppoint{gp mark 7}{(3.577,2.456)}
\gppoint{gp mark 7}{(3.582,2.453)}
\gppoint{gp mark 7}{(3.588,2.449)}
\gppoint{gp mark 7}{(3.593,2.450)}
\gppoint{gp mark 7}{(3.599,2.452)}
\gppoint{gp mark 7}{(3.604,2.451)}
\gppoint{gp mark 7}{(3.610,2.449)}
\gppoint{gp mark 7}{(3.615,2.449)}
\gppoint{gp mark 7}{(3.620,2.453)}
\gppoint{gp mark 7}{(3.626,2.454)}
\gppoint{gp mark 7}{(3.631,2.454)}
\gppoint{gp mark 7}{(3.637,2.454)}
\gppoint{gp mark 7}{(3.642,2.454)}
\gppoint{gp mark 7}{(3.648,2.455)}
\gppoint{gp mark 7}{(3.653,2.455)}
\gppoint{gp mark 7}{(3.659,2.454)}
\gppoint{gp mark 7}{(3.664,2.453)}
\gppoint{gp mark 7}{(3.670,2.452)}
\gppoint{gp mark 7}{(3.675,2.450)}
\gppoint{gp mark 7}{(3.681,2.450)}
\gppoint{gp mark 7}{(3.686,2.450)}
\gppoint{gp mark 7}{(3.692,2.451)}
\gppoint{gp mark 7}{(3.697,2.453)}
\gppoint{gp mark 7}{(3.703,2.455)}
\gppoint{gp mark 7}{(3.708,2.455)}
\gppoint{gp mark 7}{(3.714,2.455)}
\gppoint{gp mark 7}{(3.719,2.455)}
\gppoint{gp mark 7}{(3.725,2.455)}
\gppoint{gp mark 7}{(3.730,2.454)}
\gppoint{gp mark 7}{(3.736,2.453)}
\gppoint{gp mark 7}{(3.741,2.453)}
\gppoint{gp mark 7}{(3.747,2.453)}
\gppoint{gp mark 7}{(3.752,2.451)}
\gppoint{gp mark 7}{(3.758,2.449)}
\gppoint{gp mark 7}{(3.763,2.449)}
\gppoint{gp mark 7}{(3.769,2.450)}
\gppoint{gp mark 7}{(3.774,2.451)}
\gppoint{gp mark 7}{(3.780,2.452)}
\gppoint{gp mark 7}{(3.785,2.455)}
\gppoint{gp mark 7}{(3.791,2.456)}
\gppoint{gp mark 7}{(3.796,2.457)}
\gppoint{gp mark 7}{(3.802,2.456)}
\gppoint{gp mark 7}{(3.807,2.456)}
\gppoint{gp mark 7}{(3.813,2.455)}
\gppoint{gp mark 7}{(3.818,2.455)}
\gppoint{gp mark 7}{(3.824,2.454)}
\gppoint{gp mark 7}{(3.829,2.452)}
\gppoint{gp mark 7}{(3.835,2.451)}
\gppoint{gp mark 7}{(3.840,2.450)}
\gppoint{gp mark 7}{(3.846,2.451)}
\gppoint{gp mark 7}{(3.851,2.451)}
\gppoint{gp mark 7}{(3.857,2.453)}
\gppoint{gp mark 7}{(3.862,2.455)}
\gppoint{gp mark 7}{(3.868,2.456)}
\gppoint{gp mark 7}{(3.873,2.456)}
\gppoint{gp mark 7}{(3.879,2.456)}
\gppoint{gp mark 7}{(3.884,2.456)}
\gppoint{gp mark 7}{(3.890,2.455)}
\gppoint{gp mark 7}{(3.895,2.455)}
\gppoint{gp mark 7}{(3.901,2.454)}
\gppoint{gp mark 7}{(3.906,2.452)}
\gppoint{gp mark 7}{(3.912,2.451)}
\gppoint{gp mark 7}{(3.917,2.450)}
\gppoint{gp mark 7}{(3.923,2.451)}
\gppoint{gp mark 7}{(3.928,2.451)}
\gppoint{gp mark 7}{(3.934,2.453)}
\gppoint{gp mark 7}{(3.939,2.456)}
\gppoint{gp mark 7}{(3.945,2.457)}
\gppoint{gp mark 7}{(3.950,2.458)}
\gppoint{gp mark 7}{(3.956,2.458)}
\gppoint{gp mark 7}{(3.961,2.458)}
\gppoint{gp mark 7}{(3.967,2.458)}
\gppoint{gp mark 7}{(3.972,2.458)}
\gppoint{gp mark 7}{(3.978,2.457)}
\gppoint{gp mark 7}{(3.983,2.457)}
\gppoint{gp mark 7}{(3.989,2.456)}
\gppoint{gp mark 7}{(3.994,2.454)}
\gppoint{gp mark 7}{(4.000,2.451)}
\gppoint{gp mark 7}{(4.005,2.451)}
\gppoint{gp mark 7}{(4.011,2.451)}
\gppoint{gp mark 7}{(4.016,2.452)}
\gppoint{gp mark 7}{(4.022,2.453)}
\gppoint{gp mark 7}{(4.027,2.455)}
\gppoint{gp mark 7}{(4.033,2.457)}
\gppoint{gp mark 7}{(4.038,2.457)}
\gppoint{gp mark 7}{(4.044,2.458)}
\gppoint{gp mark 7}{(4.049,2.458)}
\gppoint{gp mark 7}{(4.055,2.458)}
\gppoint{gp mark 7}{(4.060,2.458)}
\gppoint{gp mark 7}{(4.066,2.457)}
\gppoint{gp mark 7}{(4.071,2.456)}
\gppoint{gp mark 7}{(4.076,2.455)}
\gppoint{gp mark 7}{(4.082,2.455)}
\gppoint{gp mark 7}{(4.087,2.454)}
\gppoint{gp mark 7}{(4.093,2.455)}
\gppoint{gp mark 7}{(4.098,2.456)}
\gppoint{gp mark 7}{(4.104,2.458)}
\gppoint{gp mark 7}{(4.109,2.459)}
\gppoint{gp mark 7}{(4.115,2.458)}
\gppoint{gp mark 7}{(4.120,2.458)}
\gppoint{gp mark 7}{(4.126,2.457)}
\gppoint{gp mark 7}{(4.131,2.456)}
\gppoint{gp mark 7}{(4.137,2.456)}
\gppoint{gp mark 7}{(4.142,2.455)}
\gppoint{gp mark 7}{(4.148,2.454)}
\gppoint{gp mark 7}{(4.153,2.454)}
\gppoint{gp mark 7}{(4.159,2.454)}
\gppoint{gp mark 7}{(4.164,2.454)}
\gppoint{gp mark 7}{(4.170,2.455)}
\gppoint{gp mark 7}{(4.175,2.457)}
\gppoint{gp mark 7}{(4.181,2.459)}
\gppoint{gp mark 7}{(4.186,2.460)}
\gppoint{gp mark 7}{(4.192,2.461)}
\gppoint{gp mark 7}{(4.197,2.461)}
\gppoint{gp mark 7}{(4.203,2.460)}
\gppoint{gp mark 7}{(4.208,2.460)}
\gppoint{gp mark 7}{(4.214,2.458)}
\gppoint{gp mark 7}{(4.219,2.457)}
\gppoint{gp mark 7}{(4.225,2.455)}
\gppoint{gp mark 7}{(4.230,2.455)}
\gppoint{gp mark 7}{(4.236,2.456)}
\gppoint{gp mark 7}{(4.241,2.456)}
\gppoint{gp mark 7}{(4.247,2.457)}
\gppoint{gp mark 7}{(4.252,2.458)}
\gppoint{gp mark 7}{(4.258,2.460)}
\gppoint{gp mark 7}{(4.263,2.461)}
\gppoint{gp mark 7}{(4.269,2.461)}
\gppoint{gp mark 7}{(4.274,2.462)}
\gppoint{gp mark 7}{(4.280,2.462)}
\gppoint{gp mark 7}{(4.285,2.462)}
\gppoint{gp mark 7}{(4.291,2.461)}
\gppoint{gp mark 7}{(4.296,2.459)}
\gppoint{gp mark 7}{(4.302,2.458)}
\gppoint{gp mark 7}{(4.307,2.459)}
\gppoint{gp mark 7}{(4.313,2.477)}
\gppoint{gp mark 7}{(4.318,2.609)}
\gppoint{gp mark 7}{(4.324,3.086)}
\gppoint{gp mark 7}{(4.329,3.522)}
\gppoint{gp mark 7}{(4.335,3.576)}
\gppoint{gp mark 7}{(4.340,3.582)}
\gppoint{gp mark 7}{(4.346,3.588)}
\gppoint{gp mark 7}{(4.351,3.589)}
\gppoint{gp mark 7}{(4.357,3.589)}
\gppoint{gp mark 7}{(4.362,3.589)}
\gppoint{gp mark 7}{(4.368,3.589)}
\gppoint{gp mark 7}{(4.373,3.589)}
\gppoint{gp mark 7}{(4.379,3.587)}
\gppoint{gp mark 7}{(4.384,3.583)}
\gppoint{gp mark 7}{(4.390,3.583)}
\gppoint{gp mark 7}{(4.395,3.584)}
\gppoint{gp mark 7}{(4.401,3.585)}
\gppoint{gp mark 7}{(4.406,3.585)}
\gppoint{gp mark 7}{(4.412,3.586)}
\gppoint{gp mark 7}{(4.417,3.588)}
\gppoint{gp mark 7}{(4.423,3.589)}
\gppoint{gp mark 7}{(4.428,3.589)}
\gppoint{gp mark 7}{(4.434,3.589)}
\gppoint{gp mark 7}{(4.439,3.588)}
\gppoint{gp mark 7}{(4.445,3.587)}
\gppoint{gp mark 7}{(4.450,3.587)}
\gppoint{gp mark 7}{(4.456,3.585)}
\gppoint{gp mark 7}{(4.461,3.583)}
\gppoint{gp mark 7}{(4.467,3.583)}
\gppoint{gp mark 7}{(4.472,3.583)}
\gppoint{gp mark 7}{(4.478,3.583)}
\gppoint{gp mark 7}{(4.483,3.583)}
\gppoint{gp mark 7}{(4.489,3.585)}
\gppoint{gp mark 7}{(4.494,3.587)}
\gppoint{gp mark 7}{(4.500,3.588)}
\gppoint{gp mark 7}{(4.505,3.588)}
\gppoint{gp mark 7}{(4.511,3.588)}
\gppoint{gp mark 7}{(4.516,3.588)}
\gppoint{gp mark 7}{(4.522,3.588)}
\gppoint{gp mark 7}{(4.527,3.585)}
\gppoint{gp mark 7}{(4.532,3.584)}
\gppoint{gp mark 7}{(4.538,3.583)}
\gppoint{gp mark 7}{(4.543,3.582)}
\gppoint{gp mark 7}{(4.549,3.582)}
\gppoint{gp mark 7}{(4.554,3.583)}
\gppoint{gp mark 7}{(4.560,3.585)}
\gppoint{gp mark 7}{(4.565,3.587)}
\gppoint{gp mark 7}{(4.571,3.587)}
\gppoint{gp mark 7}{(4.576,3.587)}
\gppoint{gp mark 7}{(4.582,3.588)}
\gppoint{gp mark 7}{(4.587,3.588)}
\gppoint{gp mark 7}{(4.593,3.587)}
\gppoint{gp mark 7}{(4.598,3.585)}
\gppoint{gp mark 7}{(4.604,3.584)}
\gppoint{gp mark 7}{(4.609,3.584)}
\gppoint{gp mark 7}{(4.615,3.584)}
\gppoint{gp mark 7}{(4.620,3.584)}
\gppoint{gp mark 7}{(4.626,3.585)}
\gppoint{gp mark 7}{(4.631,3.587)}
\gppoint{gp mark 7}{(4.637,3.588)}
\gppoint{gp mark 7}{(4.642,3.588)}
\gppoint{gp mark 7}{(4.648,3.588)}
\gppoint{gp mark 7}{(4.653,3.588)}
\gppoint{gp mark 7}{(4.659,3.587)}
\gppoint{gp mark 7}{(4.664,3.585)}
\gppoint{gp mark 7}{(4.670,3.583)}
\gppoint{gp mark 7}{(4.675,3.582)}
\gppoint{gp mark 7}{(4.681,3.582)}
\gppoint{gp mark 7}{(4.686,3.582)}
\gppoint{gp mark 7}{(4.692,3.583)}
\gppoint{gp mark 7}{(4.697,3.585)}
\gppoint{gp mark 7}{(4.703,3.587)}
\gppoint{gp mark 7}{(4.708,3.588)}
\gppoint{gp mark 7}{(4.714,3.588)}
\gppoint{gp mark 7}{(4.719,3.589)}
\gppoint{gp mark 7}{(4.725,3.589)}
\gppoint{gp mark 7}{(4.730,3.589)}
\gppoint{gp mark 7}{(4.736,3.589)}
\gppoint{gp mark 7}{(4.741,3.587)}
\gppoint{gp mark 7}{(4.747,3.585)}
\gppoint{gp mark 7}{(4.752,3.584)}
\gppoint{gp mark 7}{(4.758,3.584)}
\gppoint{gp mark 7}{(4.763,3.584)}
\gppoint{gp mark 7}{(4.769,3.584)}
\gppoint{gp mark 7}{(4.774,3.584)}
\gppoint{gp mark 7}{(4.780,3.586)}
\gppoint{gp mark 7}{(4.785,3.587)}
\gppoint{gp mark 7}{(4.791,3.587)}
\gppoint{gp mark 7}{(4.796,3.587)}
\gppoint{gp mark 7}{(4.802,3.587)}
\gppoint{gp mark 7}{(4.807,3.586)}
\gppoint{gp mark 7}{(4.813,3.585)}
\gppoint{gp mark 7}{(4.818,3.583)}
\gppoint{gp mark 7}{(4.824,3.583)}
\gppoint{gp mark 7}{(4.829,3.583)}
\gppoint{gp mark 7}{(4.835,3.583)}
\gppoint{gp mark 7}{(4.840,3.584)}
\gppoint{gp mark 7}{(4.846,3.585)}
\gppoint{gp mark 7}{(4.851,3.587)}
\gppoint{gp mark 7}{(4.857,3.588)}
\gppoint{gp mark 7}{(4.862,3.588)}
\gppoint{gp mark 7}{(4.868,3.588)}
\gppoint{gp mark 7}{(4.873,3.588)}
\gppoint{gp mark 7}{(4.879,3.587)}
\gppoint{gp mark 7}{(4.884,3.585)}
\gppoint{gp mark 7}{(4.890,3.584)}
\gppoint{gp mark 7}{(4.895,3.584)}
\gppoint{gp mark 7}{(4.901,3.584)}
\gppoint{gp mark 7}{(4.906,3.584)}
\gppoint{gp mark 7}{(4.912,3.584)}
\gppoint{gp mark 7}{(4.917,3.586)}
\gppoint{gp mark 7}{(4.923,3.587)}
\gppoint{gp mark 7}{(4.928,3.588)}
\gppoint{gp mark 7}{(4.934,3.588)}
\gppoint{gp mark 7}{(4.939,3.588)}
\gppoint{gp mark 7}{(4.945,3.588)}
\gppoint{gp mark 7}{(4.950,3.586)}
\gppoint{gp mark 7}{(4.956,3.584)}
\gppoint{gp mark 7}{(4.961,3.584)}
\gppoint{gp mark 7}{(4.967,3.584)}
\gppoint{gp mark 7}{(4.972,3.584)}
\gppoint{gp mark 7}{(4.978,3.584)}
\gppoint{gp mark 7}{(4.983,3.585)}
\gppoint{gp mark 7}{(4.988,3.587)}
\gppoint{gp mark 7}{(4.994,3.588)}
\gppoint{gp mark 7}{(4.999,3.588)}
\gppoint{gp mark 7}{(5.005,3.588)}
\gppoint{gp mark 7}{(5.010,3.588)}
\gppoint{gp mark 7}{(5.016,3.588)}
\gppoint{gp mark 7}{(5.021,3.587)}
\gppoint{gp mark 7}{(5.027,3.586)}
\gppoint{gp mark 7}{(5.032,3.584)}
\gppoint{gp mark 7}{(5.038,3.583)}
\gppoint{gp mark 7}{(5.043,3.583)}
\gppoint{gp mark 7}{(5.049,3.583)}
\gppoint{gp mark 7}{(5.054,3.583)}
\gppoint{gp mark 7}{(5.060,3.584)}
\gppoint{gp mark 7}{(5.065,3.586)}
\gppoint{gp mark 7}{(5.071,3.588)}
\gppoint{gp mark 7}{(5.076,3.588)}
\gppoint{gp mark 7}{(5.082,3.588)}
\gppoint{gp mark 7}{(5.087,3.588)}
\gppoint{gp mark 7}{(5.093,3.588)}
\gppoint{gp mark 7}{(5.098,3.586)}
\gppoint{gp mark 7}{(5.104,3.585)}
\gppoint{gp mark 7}{(5.109,3.583)}
\gppoint{gp mark 7}{(5.115,3.583)}
\gppoint{gp mark 7}{(5.120,3.583)}
\gppoint{gp mark 7}{(5.126,3.583)}
\gppoint{gp mark 7}{(5.131,3.584)}
\gppoint{gp mark 7}{(5.137,3.586)}
\gppoint{gp mark 7}{(5.142,3.587)}
\gppoint{gp mark 7}{(5.148,3.587)}
\gppoint{gp mark 7}{(5.153,3.588)}
\gppoint{gp mark 7}{(5.159,3.588)}
\gppoint{gp mark 7}{(5.164,3.588)}
\gppoint{gp mark 7}{(5.170,3.588)}
\gppoint{gp mark 7}{(5.175,3.587)}
\gppoint{gp mark 7}{(5.181,3.585)}
\gppoint{gp mark 7}{(5.186,3.584)}
\gppoint{gp mark 7}{(5.192,3.584)}
\gppoint{gp mark 7}{(5.197,3.584)}
\gppoint{gp mark 7}{(5.203,3.584)}
\gppoint{gp mark 7}{(5.208,3.584)}
\gppoint{gp mark 7}{(5.214,3.586)}
\gppoint{gp mark 7}{(5.219,3.587)}
\gppoint{gp mark 7}{(5.225,3.587)}
\gppoint{gp mark 7}{(5.230,3.587)}
\gppoint{gp mark 7}{(5.236,3.587)}
\gppoint{gp mark 7}{(5.241,3.587)}
\gppoint{gp mark 7}{(5.247,3.585)}
\gppoint{gp mark 7}{(5.252,3.584)}
\gppoint{gp mark 7}{(5.258,3.583)}
\gppoint{gp mark 7}{(5.263,3.583)}
\gppoint{gp mark 7}{(5.269,3.583)}
\gppoint{gp mark 7}{(5.274,3.583)}
\gppoint{gp mark 7}{(5.280,3.585)}
\gppoint{gp mark 7}{(5.285,3.586)}
\gppoint{gp mark 7}{(5.291,3.587)}
\gppoint{gp mark 7}{(5.296,3.588)}
\gppoint{gp mark 7}{(5.302,3.588)}
\gppoint{gp mark 7}{(5.307,3.588)}
\gppoint{gp mark 7}{(5.313,3.587)}
\gppoint{gp mark 7}{(5.318,3.586)}
\gppoint{gp mark 7}{(5.324,3.584)}
\gppoint{gp mark 7}{(5.329,3.584)}
\gppoint{gp mark 7}{(5.335,3.584)}
\gppoint{gp mark 7}{(5.340,3.584)}
\gppoint{gp mark 7}{(5.346,3.584)}
\gppoint{gp mark 7}{(5.351,3.585)}
\gppoint{gp mark 7}{(5.357,3.586)}
\gppoint{gp mark 7}{(5.362,3.586)}
\gppoint{gp mark 7}{(5.368,3.586)}
\gppoint{gp mark 7}{(5.373,3.586)}
\gppoint{gp mark 7}{(5.379,3.586)}
\gppoint{gp mark 7}{(5.384,3.585)}
\gppoint{gp mark 7}{(5.390,3.584)}
\gppoint{gp mark 7}{(5.395,3.583)}
\gppoint{gp mark 7}{(5.401,3.583)}
\gppoint{gp mark 7}{(5.406,3.584)}
\gppoint{gp mark 7}{(5.412,3.584)}
\gppoint{gp mark 7}{(5.417,3.585)}
\gppoint{gp mark 7}{(5.423,3.587)}
\gppoint{gp mark 7}{(5.428,3.588)}
\gppoint{gp mark 7}{(5.434,3.589)}
\gppoint{gp mark 7}{(5.439,3.589)}
\gppoint{gp mark 7}{(5.444,3.589)}
\gppoint{gp mark 7}{(5.450,3.589)}
\gppoint{gp mark 7}{(5.455,3.588)}
\gppoint{gp mark 7}{(5.461,3.586)}
\gppoint{gp mark 7}{(5.466,3.584)}
\gppoint{gp mark 7}{(5.472,3.583)}
\gppoint{gp mark 7}{(5.477,3.582)}
\gppoint{gp mark 7}{(5.483,3.582)}
\gppoint{gp mark 7}{(5.488,3.582)}
\gppoint{gp mark 7}{(5.494,3.583)}
\gppoint{gp mark 7}{(5.499,3.584)}
\gppoint{gp mark 7}{(5.505,3.586)}
\gppoint{gp mark 7}{(5.510,3.586)}
\gppoint{gp mark 7}{(5.516,3.587)}
\gppoint{gp mark 7}{(5.521,3.587)}
\gppoint{gp mark 7}{(5.527,3.587)}
\gppoint{gp mark 7}{(5.532,3.586)}
\gppoint{gp mark 7}{(5.538,3.585)}
\gppoint{gp mark 7}{(5.543,3.584)}
\gppoint{gp mark 7}{(5.549,3.584)}
\gppoint{gp mark 7}{(5.554,3.584)}
\gppoint{gp mark 7}{(5.560,3.584)}
\gppoint{gp mark 7}{(5.565,3.584)}
\gppoint{gp mark 7}{(5.571,3.586)}
\gppoint{gp mark 7}{(5.576,3.587)}
\gppoint{gp mark 7}{(5.582,3.588)}
\gppoint{gp mark 7}{(5.587,3.588)}
\gppoint{gp mark 7}{(5.593,3.588)}
\gppoint{gp mark 7}{(5.598,3.588)}
\gppoint{gp mark 7}{(5.604,3.588)}
\gppoint{gp mark 7}{(5.609,3.586)}
\gppoint{gp mark 7}{(5.615,3.585)}
\gppoint{gp mark 7}{(5.620,3.584)}
\gppoint{gp mark 7}{(5.626,3.583)}
\gppoint{gp mark 7}{(5.631,3.583)}
\gppoint{gp mark 7}{(5.637,3.583)}
\gppoint{gp mark 7}{(5.642,3.583)}
\gppoint{gp mark 7}{(5.648,3.584)}
\gppoint{gp mark 7}{(5.653,3.585)}
\gppoint{gp mark 7}{(5.659,3.586)}
\gppoint{gp mark 7}{(5.664,3.586)}
\gppoint{gp mark 7}{(5.670,3.586)}
\gppoint{gp mark 7}{(5.675,3.586)}
\gppoint{gp mark 7}{(5.681,3.585)}
\gppoint{gp mark 7}{(5.686,3.584)}
\gppoint{gp mark 7}{(5.692,3.583)}
\gppoint{gp mark 7}{(5.697,3.583)}
\gppoint{gp mark 7}{(5.703,3.583)}
\gppoint{gp mark 7}{(5.708,3.583)}
\gppoint{gp mark 7}{(5.714,3.584)}
\gppoint{gp mark 7}{(5.719,3.585)}
\gppoint{gp mark 7}{(5.725,3.586)}
\gppoint{gp mark 7}{(5.730,3.586)}
\gppoint{gp mark 7}{(5.736,3.587)}
\gppoint{gp mark 7}{(5.741,3.587)}
\gppoint{gp mark 7}{(5.747,3.587)}
\gppoint{gp mark 7}{(5.752,3.586)}
\gppoint{gp mark 7}{(5.758,3.585)}
\gppoint{gp mark 7}{(5.763,3.584)}
\gppoint{gp mark 7}{(5.769,3.584)}
\gppoint{gp mark 7}{(5.774,3.584)}
\gppoint{gp mark 7}{(5.780,3.585)}
\gppoint{gp mark 7}{(5.785,3.585)}
\gppoint{gp mark 7}{(5.791,3.586)}
\gppoint{gp mark 7}{(5.796,3.587)}
\gppoint{gp mark 7}{(5.802,3.587)}
\gppoint{gp mark 7}{(5.807,3.586)}
\gppoint{gp mark 7}{(5.813,3.586)}
\gppoint{gp mark 7}{(5.818,3.585)}
\gppoint{gp mark 7}{(5.824,3.584)}
\gppoint{gp mark 7}{(5.829,3.583)}
\gppoint{gp mark 7}{(5.835,3.583)}
\gppoint{gp mark 7}{(5.840,3.583)}
\gppoint{gp mark 7}{(5.846,3.583)}
\gppoint{gp mark 7}{(5.851,3.584)}
\gppoint{gp mark 7}{(5.857,3.585)}
\gppoint{gp mark 7}{(5.862,3.587)}
\gppoint{gp mark 7}{(5.868,3.588)}
\gppoint{gp mark 7}{(5.873,3.588)}
\gppoint{gp mark 7}{(5.879,3.588)}
\gppoint{gp mark 7}{(5.884,3.588)}
\gppoint{gp mark 7}{(5.890,3.588)}
\gppoint{gp mark 7}{(5.895,3.587)}
\gppoint{gp mark 7}{(5.900,3.585)}
\gppoint{gp mark 7}{(5.906,3.584)}
\gppoint{gp mark 7}{(5.911,3.583)}
\gppoint{gp mark 7}{(5.917,3.583)}
\gppoint{gp mark 7}{(5.922,3.583)}
\gppoint{gp mark 7}{(5.928,3.583)}
\gppoint{gp mark 7}{(5.933,3.584)}
\gppoint{gp mark 7}{(5.939,3.586)}
\gppoint{gp mark 7}{(5.944,3.587)}
\gppoint{gp mark 7}{(5.950,3.587)}
\gppoint{gp mark 7}{(5.955,3.587)}
\gppoint{gp mark 7}{(5.961,3.587)}
\gppoint{gp mark 7}{(5.966,3.587)}
\gppoint{gp mark 7}{(5.972,3.587)}
\gppoint{gp mark 7}{(5.977,3.585)}
\gppoint{gp mark 7}{(5.983,3.584)}
\gppoint{gp mark 7}{(5.988,3.583)}
\gppoint{gp mark 7}{(5.994,3.583)}
\gppoint{gp mark 7}{(5.999,3.583)}
\gppoint{gp mark 7}{(6.005,3.583)}
\gppoint{gp mark 7}{(6.010,3.584)}
\gppoint{gp mark 7}{(6.016,3.585)}
\gppoint{gp mark 7}{(6.021,3.586)}
\gppoint{gp mark 7}{(6.027,3.587)}
\gppoint{gp mark 7}{(6.032,3.586)}
\gppoint{gp mark 7}{(6.038,3.586)}
\gppoint{gp mark 7}{(6.043,3.586)}
\gppoint{gp mark 7}{(6.049,3.585)}
\gppoint{gp mark 7}{(6.054,3.583)}
\gppoint{gp mark 7}{(6.060,3.583)}
\gppoint{gp mark 7}{(6.065,3.583)}
\gppoint{gp mark 7}{(6.071,3.583)}
\gppoint{gp mark 7}{(6.076,3.583)}
\gppoint{gp mark 7}{(6.082,3.584)}
\gppoint{gp mark 7}{(6.087,3.585)}
\gppoint{gp mark 7}{(6.093,3.586)}
\gppoint{gp mark 7}{(6.098,3.587)}
\gppoint{gp mark 7}{(6.104,3.587)}
\gppoint{gp mark 7}{(6.109,3.587)}
\gppoint{gp mark 7}{(6.115,3.586)}
\gppoint{gp mark 7}{(6.120,3.585)}
\gppoint{gp mark 7}{(6.126,3.584)}
\gppoint{gp mark 7}{(6.131,3.584)}
\gppoint{gp mark 7}{(6.137,3.583)}
\gppoint{gp mark 7}{(6.142,3.583)}
\gppoint{gp mark 7}{(6.148,3.583)}
\gppoint{gp mark 7}{(6.153,3.584)}
\gppoint{gp mark 7}{(6.159,3.585)}
\gppoint{gp mark 7}{(6.164,3.586)}
\gppoint{gp mark 7}{(6.170,3.586)}
\gppoint{gp mark 7}{(6.175,3.586)}
\gppoint{gp mark 7}{(6.181,3.586)}
\gppoint{gp mark 7}{(6.186,3.586)}
\gppoint{gp mark 7}{(6.192,3.585)}
\gppoint{gp mark 7}{(6.197,3.584)}
\gppoint{gp mark 7}{(6.203,3.583)}
\gppoint{gp mark 7}{(6.208,3.583)}
\gppoint{gp mark 7}{(6.214,3.583)}
\gppoint{gp mark 7}{(6.219,3.583)}
\gppoint{gp mark 7}{(6.225,3.584)}
\gppoint{gp mark 7}{(6.230,3.584)}
\gppoint{gp mark 7}{(6.236,3.586)}
\gppoint{gp mark 7}{(6.241,3.587)}
\gppoint{gp mark 7}{(6.247,3.587)}
\gppoint{gp mark 7}{(6.252,3.587)}
\gppoint{gp mark 7}{(6.258,3.587)}
\gppoint{gp mark 7}{(6.263,3.587)}
\gppoint{gp mark 7}{(6.269,3.586)}
\gppoint{gp mark 7}{(6.274,3.585)}
\gppoint{gp mark 7}{(6.280,3.584)}
\gppoint{gp mark 7}{(6.285,3.583)}
\gppoint{gp mark 7}{(6.291,3.583)}
\gppoint{gp mark 7}{(6.296,3.583)}
\gppoint{gp mark 7}{(6.302,3.583)}
\gppoint{gp mark 7}{(6.307,3.583)}
\gppoint{gp mark 7}{(6.313,3.584)}
\gppoint{gp mark 7}{(6.318,3.585)}
\gppoint{gp mark 7}{(6.324,3.586)}
\gppoint{gp mark 7}{(6.329,3.586)}
\gppoint{gp mark 7}{(6.335,3.586)}
\gppoint{gp mark 7}{(6.340,3.585)}
\gppoint{gp mark 7}{(6.346,3.585)}
\gppoint{gp mark 7}{(6.351,3.584)}
\gppoint{gp mark 7}{(6.356,3.583)}
\gppoint{gp mark 7}{(6.362,3.583)}
\gppoint{gp mark 7}{(6.367,3.583)}
\gppoint{gp mark 7}{(6.373,3.584)}
\gppoint{gp mark 7}{(6.378,3.584)}
\gppoint{gp mark 7}{(6.384,3.585)}
\gppoint{gp mark 7}{(6.389,3.586)}
\gppoint{gp mark 7}{(6.395,3.587)}
\gppoint{gp mark 7}{(6.400,3.587)}
\gppoint{gp mark 7}{(6.406,3.587)}
\gppoint{gp mark 7}{(6.411,3.587)}
\gppoint{gp mark 7}{(6.417,3.586)}
\gppoint{gp mark 7}{(6.422,3.585)}
\gppoint{gp mark 7}{(6.428,3.584)}
\gppoint{gp mark 7}{(6.433,3.583)}
\gppoint{gp mark 7}{(6.439,3.583)}
\gppoint{gp mark 7}{(6.444,3.583)}
\gppoint{gp mark 7}{(6.450,3.583)}
\gppoint{gp mark 7}{(6.455,3.584)}
\gppoint{gp mark 7}{(6.461,3.585)}
\gppoint{gp mark 7}{(6.466,3.586)}
\gppoint{gp mark 7}{(6.472,3.586)}
\gppoint{gp mark 7}{(6.477,3.586)}
\gppoint{gp mark 7}{(6.483,3.586)}
\gppoint{gp mark 7}{(6.488,3.585)}
\gppoint{gp mark 7}{(6.494,3.584)}
\gppoint{gp mark 7}{(6.499,3.583)}
\gppoint{gp mark 7}{(6.505,3.582)}
\gppoint{gp mark 7}{(6.510,3.582)}
\gppoint{gp mark 7}{(6.516,3.582)}
\gppoint{gp mark 7}{(6.521,3.582)}
\gppoint{gp mark 7}{(6.527,3.583)}
\gppoint{gp mark 7}{(6.532,3.584)}
\gppoint{gp mark 7}{(6.538,3.586)}
\gppoint{gp mark 7}{(6.543,3.587)}
\gppoint{gp mark 7}{(6.549,3.587)}
\gppoint{gp mark 7}{(6.554,3.588)}
\gppoint{gp mark 7}{(6.560,3.588)}
\gppoint{gp mark 7}{(6.565,3.588)}
\gppoint{gp mark 7}{(6.571,3.588)}
\gppoint{gp mark 7}{(6.576,3.587)}
\gppoint{gp mark 7}{(6.582,3.585)}
\gppoint{gp mark 7}{(6.587,3.584)}
\gppoint{gp mark 7}{(6.593,3.583)}
\gppoint{gp mark 7}{(6.598,3.583)}
\gppoint{gp mark 7}{(6.604,3.582)}
\gppoint{gp mark 7}{(6.609,3.582)}
\gppoint{gp mark 7}{(6.615,3.582)}
\gppoint{gp mark 7}{(6.620,3.583)}
\gppoint{gp mark 7}{(6.626,3.585)}
\gppoint{gp mark 7}{(6.631,3.585)}
\gppoint{gp mark 7}{(6.637,3.586)}
\gppoint{gp mark 7}{(6.642,3.586)}
\gppoint{gp mark 7}{(6.648,3.586)}
\gppoint{gp mark 7}{(6.653,3.586)}
\gppoint{gp mark 7}{(6.659,3.585)}
\gppoint{gp mark 7}{(6.664,3.584)}
\gppoint{gp mark 7}{(6.670,3.583)}
\gppoint{gp mark 7}{(6.675,3.583)}
\gppoint{gp mark 7}{(6.681,3.583)}
\gppoint{gp mark 7}{(6.686,3.583)}
\gppoint{gp mark 7}{(6.692,3.583)}
\gppoint{gp mark 7}{(6.697,3.584)}
\gppoint{gp mark 7}{(6.703,3.585)}
\gppoint{gp mark 7}{(6.708,3.586)}
\gppoint{gp mark 7}{(6.714,3.586)}
\gppoint{gp mark 7}{(6.719,3.586)}
\gppoint{gp mark 7}{(6.725,3.586)}
\gppoint{gp mark 7}{(6.730,3.586)}
\gppoint{gp mark 7}{(6.736,3.585)}
\gppoint{gp mark 7}{(6.741,3.584)}
\gppoint{gp mark 7}{(6.747,3.583)}
\gppoint{gp mark 7}{(6.752,3.582)}
\gppoint{gp mark 7}{(6.758,3.582)}
\gppoint{gp mark 7}{(6.763,3.582)}
\gppoint{gp mark 7}{(6.769,3.582)}
\gppoint{gp mark 7}{(6.774,3.583)}
\gppoint{gp mark 7}{(6.780,3.584)}
\gppoint{gp mark 7}{(6.785,3.585)}
\gppoint{gp mark 7}{(6.791,3.586)}
\gppoint{gp mark 7}{(6.796,3.586)}
\gppoint{gp mark 7}{(6.802,3.587)}
\gppoint{gp mark 7}{(6.807,3.587)}
\gppoint{gp mark 7}{(6.812,3.586)}
\gppoint{gp mark 7}{(6.818,3.585)}
\gppoint{gp mark 7}{(6.823,3.584)}
\gppoint{gp mark 7}{(6.829,3.583)}
\gppoint{gp mark 7}{(6.834,3.582)}
\gppoint{gp mark 7}{(6.840,3.582)}
\gppoint{gp mark 7}{(6.845,3.582)}
\gppoint{gp mark 7}{(6.851,3.582)}
\gppoint{gp mark 7}{(6.856,3.583)}
\gppoint{gp mark 7}{(6.862,3.584)}
\gppoint{gp mark 7}{(6.867,3.585)}
\gppoint{gp mark 7}{(6.873,3.585)}
\gppoint{gp mark 7}{(6.878,3.586)}
\gppoint{gp mark 7}{(6.884,3.586)}
\gppoint{gp mark 7}{(6.889,3.586)}
\gppoint{gp mark 7}{(6.895,3.586)}
\gppoint{gp mark 7}{(6.900,3.586)}
\gppoint{gp mark 7}{(6.906,3.584)}
\gppoint{gp mark 7}{(6.911,3.583)}
\gppoint{gp mark 7}{(6.917,3.583)}
\gppoint{gp mark 7}{(6.922,3.583)}
\gppoint{gp mark 7}{(6.928,3.583)}
\gppoint{gp mark 7}{(6.933,3.583)}
\gppoint{gp mark 7}{(6.939,3.583)}
\gppoint{gp mark 7}{(6.944,3.584)}
\gppoint{gp mark 7}{(6.950,3.585)}
\gppoint{gp mark 7}{(6.955,3.585)}
\gppoint{gp mark 7}{(6.961,3.586)}
\gppoint{gp mark 7}{(6.966,3.586)}
\gppoint{gp mark 7}{(6.972,3.586)}
\gppoint{gp mark 7}{(6.977,3.586)}
\gppoint{gp mark 7}{(6.983,3.586)}
\gppoint{gp mark 7}{(6.988,3.585)}
\gppoint{gp mark 7}{(6.994,3.584)}
\gppoint{gp mark 7}{(6.999,3.583)}
\gppoint{gp mark 7}{(7.005,3.582)}
\gppoint{gp mark 7}{(7.010,3.582)}
\gppoint{gp mark 7}{(7.016,3.582)}
\gppoint{gp mark 7}{(7.021,3.582)}
\gppoint{gp mark 7}{(7.027,3.582)}
\gppoint{gp mark 7}{(7.032,3.583)}
\gppoint{gp mark 7}{(7.038,3.584)}
\gppoint{gp mark 7}{(7.043,3.585)}
\gppoint{gp mark 7}{(7.049,3.586)}
\gppoint{gp mark 7}{(7.054,3.586)}
\gppoint{gp mark 7}{(7.060,3.586)}
\gppoint{gp mark 7}{(7.065,3.586)}
\gppoint{gp mark 7}{(7.071,3.585)}
\gppoint{gp mark 7}{(7.076,3.584)}
\gppoint{gp mark 7}{(7.082,3.583)}
\gppoint{gp mark 7}{(7.087,3.582)}
\gppoint{gp mark 7}{(7.093,3.582)}
\gppoint{gp mark 7}{(7.098,3.582)}
\gppoint{gp mark 7}{(7.104,3.582)}
\gppoint{gp mark 7}{(7.109,3.583)}
\gppoint{gp mark 7}{(7.115,3.584)}
\gppoint{gp mark 7}{(7.120,3.585)}
\gppoint{gp mark 7}{(7.126,3.586)}
\gppoint{gp mark 7}{(7.131,3.586)}
\gppoint{gp mark 7}{(7.137,3.587)}
\gppoint{gp mark 7}{(7.142,3.587)}
\gppoint{gp mark 7}{(7.148,3.587)}
\gppoint{gp mark 7}{(7.153,3.587)}
\gppoint{gp mark 7}{(7.159,3.587)}
\gppoint{gp mark 7}{(7.164,3.587)}
\gppoint{gp mark 7}{(7.170,3.585)}
\gppoint{gp mark 7}{(7.175,3.584)}
\gppoint{gp mark 7}{(7.181,3.583)}
\gppoint{gp mark 7}{(7.186,3.582)}
\gppoint{gp mark 7}{(7.192,3.582)}
\gppoint{gp mark 7}{(7.197,3.582)}
\gppoint{gp mark 7}{(7.203,3.582)}
\gppoint{gp mark 7}{(7.208,3.582)}
\gppoint{gp mark 7}{(7.214,3.583)}
\gppoint{gp mark 7}{(7.219,3.584)}
\gppoint{gp mark 7}{(7.225,3.584)}
\gppoint{gp mark 7}{(7.230,3.585)}
\gppoint{gp mark 7}{(7.236,3.585)}
\gppoint{gp mark 7}{(7.241,3.586)}
\gppoint{gp mark 7}{(7.247,3.586)}
\gppoint{gp mark 7}{(7.252,3.587)}
\gppoint{gp mark 7}{(7.258,3.587)}
\gppoint{gp mark 7}{(7.263,3.587)}
\gppoint{gp mark 7}{(7.268,3.587)}
\gppoint{gp mark 7}{(7.274,3.587)}
\gppoint{gp mark 7}{(7.279,3.587)}
\gppoint{gp mark 7}{(7.285,3.587)}
\gppoint{gp mark 7}{(7.290,3.587)}
\gppoint{gp mark 7}{(7.296,3.587)}
\gppoint{gp mark 7}{(7.301,3.587)}
\gppoint{gp mark 7}{(7.307,3.587)}
\gppoint{gp mark 7}{(7.312,3.587)}
\gppoint{gp mark 7}{(7.318,3.587)}
\gppoint{gp mark 7}{(7.323,3.587)}
\gppoint{gp mark 7}{(7.329,3.587)}
\gppoint{gp mark 7}{(7.334,3.587)}
\gppoint{gp mark 7}{(7.340,3.588)}
\gppoint{gp mark 7}{(7.345,3.588)}
\gppoint{gp mark 7}{(7.351,3.588)}
\gppoint{gp mark 7}{(7.356,3.588)}
\gppoint{gp mark 7}{(7.362,3.588)}
\gppoint{gp mark 7}{(7.367,3.588)}
\gppoint{gp mark 7}{(7.373,3.587)}
\gppoint{gp mark 7}{(7.378,3.587)}
\gppoint{gp mark 7}{(7.384,3.586)}
\gppoint{gp mark 7}{(7.389,3.585)}
\gppoint{gp mark 7}{(7.395,3.584)}
\gppoint{gp mark 7}{(7.400,3.581)}
\gppoint{gp mark 7}{(7.406,3.577)}
\gppoint{gp mark 7}{(7.411,3.572)}
\gppoint{gp mark 7}{(7.417,3.566)}
\gppoint{gp mark 7}{(7.422,3.562)}
\gppoint{gp mark 7}{(7.428,3.559)}
\gppoint{gp mark 7}{(7.433,3.558)}
\gppoint{gp mark 7}{(7.439,3.557)}
\gppoint{gp mark 7}{(7.444,3.557)}
\gppoint{gp mark 7}{(7.450,3.557)}
\gppoint{gp mark 7}{(7.455,3.557)}
\gppoint{gp mark 7}{(7.461,3.557)}
\gppoint{gp mark 7}{(7.466,3.557)}
\gppoint{gp mark 7}{(7.472,3.556)}
\gppoint{gp mark 7}{(7.477,3.554)}
\gppoint{gp mark 7}{(7.483,3.539)}
\gppoint{gp mark 7}{(7.488,3.427)}
\gppoint{gp mark 7}{(7.494,2.953)}
\gppoint{gp mark 7}{(7.499,2.021)}
\gpcolor{rgb color={0.000,0.000,0.000}}
\gpsetlinetype{gp lt plot 0}
\gpsetlinewidth{4.00}
\draw[gp path] (2.411,2.442)--(3.526,2.442);
\draw[gp path] (3.526,2.442)--(4.329,2.442);
\draw[gp path] (4.329,3.580)--(7.511,3.580);
\draw[gp path] (1.202,4.128)--(1.208,4.119)--(1.214,4.109)--(1.220,4.100)--(1.226,4.091)%
  --(1.232,4.082)--(1.238,4.073)--(1.244,4.064)--(1.250,4.055)--(1.256,4.046)--(1.262,4.037)%
  --(1.268,4.027)--(1.275,4.018)--(1.281,4.009)--(1.287,4.000)--(1.293,3.991)--(1.299,3.982)%
  --(1.305,3.973)--(1.311,3.964)--(1.317,3.955)--(1.323,3.946)--(1.329,3.937)--(1.335,3.928)%
  --(1.341,3.919)--(1.347,3.910)--(1.353,3.901)--(1.359,3.892)--(1.365,3.883)--(1.371,3.874)%
  --(1.377,3.865)--(1.383,3.856)--(1.389,3.847)--(1.395,3.838)--(1.402,3.829)--(1.408,3.820)%
  --(1.414,3.811)--(1.420,3.802)--(1.426,3.793)--(1.432,3.784)--(1.438,3.775)--(1.444,3.767)%
  --(1.450,3.758)--(1.456,3.749)--(1.462,3.740)--(1.468,3.731)--(1.474,3.722)--(1.480,3.713)%
  --(1.486,3.704)--(1.492,3.695)--(1.498,3.687)--(1.504,3.678)--(1.510,3.669)--(1.516,3.660)%
  --(1.522,3.651)--(1.528,3.642)--(1.535,3.633)--(1.541,3.625)--(1.547,3.616)--(1.553,3.607)%
  --(1.559,3.598)--(1.565,3.589)--(1.571,3.581)--(1.577,3.572)--(1.583,3.563)--(1.589,3.554)%
  --(1.595,3.546)--(1.601,3.537)--(1.607,3.528)--(1.613,3.519)--(1.619,3.511)--(1.625,3.502)%
  --(1.631,3.493)--(1.637,3.484)--(1.643,3.476)--(1.649,3.467)--(1.655,3.458)--(1.662,3.450)%
  --(1.668,3.441)--(1.674,3.432)--(1.680,3.423)--(1.686,3.415)--(1.692,3.406)--(1.698,3.398)%
  --(1.704,3.389)--(1.710,3.380)--(1.716,3.372)--(1.722,3.363)--(1.728,3.354)--(1.734,3.346)%
  --(1.740,3.337)--(1.746,3.328)--(1.752,3.320)--(1.758,3.311)--(1.764,3.303)--(1.770,3.294)%
  --(1.776,3.286)--(1.782,3.277)--(1.788,3.268)--(1.795,3.260)--(1.801,3.251)--(1.807,3.243)%
  --(1.813,3.234)--(1.819,3.226)--(1.825,3.217)--(1.831,3.209)--(1.837,3.200)--(1.843,3.192)%
  --(1.849,3.183)--(1.855,3.175)--(1.861,3.166)--(1.867,3.158)--(1.873,3.149)--(1.879,3.141)%
  --(1.885,3.132)--(1.891,3.124)--(1.897,3.116)--(1.903,3.107)--(1.909,3.099)--(1.915,3.090)%
  --(1.922,3.082)--(1.928,3.074)--(1.934,3.065)--(1.940,3.057)--(1.946,3.048)--(1.952,3.040)%
  --(1.958,3.032)--(1.964,3.023)--(1.970,3.015)--(1.976,3.007)--(1.982,2.998)--(1.988,2.990)%
  --(1.994,2.982)--(2.000,2.973)--(2.006,2.965)--(2.012,2.957)--(2.018,2.948)--(2.024,2.940)%
  --(2.030,2.932)--(2.036,2.924)--(2.042,2.915)--(2.048,2.907)--(2.055,2.899)--(2.061,2.891)%
  --(2.067,2.882)--(2.073,2.874)--(2.079,2.866)--(2.085,2.858)--(2.091,2.850)--(2.097,2.841)%
  --(2.103,2.833)--(2.109,2.825)--(2.115,2.817)--(2.121,2.809)--(2.127,2.800)--(2.133,2.792)%
  --(2.139,2.784)--(2.145,2.776)--(2.151,2.768)--(2.157,2.760)--(2.163,2.752)--(2.169,2.744)%
  --(2.175,2.736)--(2.182,2.727)--(2.188,2.719)--(2.194,2.711)--(2.200,2.703)--(2.206,2.695)%
  --(2.212,2.687)--(2.218,2.679)--(2.224,2.671)--(2.230,2.663)--(2.236,2.655)--(2.242,2.647)%
  --(2.248,2.639)--(2.254,2.631)--(2.260,2.623)--(2.266,2.615)--(2.272,2.607)--(2.278,2.599)%
  --(2.284,2.591)--(2.290,2.583)--(2.296,2.575)--(2.302,2.567)--(2.308,2.560)--(2.315,2.552)%
  --(2.321,2.544)--(2.327,2.536)--(2.333,2.528)--(2.339,2.520)--(2.345,2.512)--(2.351,2.504)%
  --(2.357,2.497)--(2.363,2.489)--(2.369,2.481)--(2.375,2.473)--(2.381,2.465)--(2.387,2.457)%
  --(2.393,2.450)--(2.399,2.442)--(2.405,2.434)--(2.411,2.442);
\draw[gp path] (4.329,2.442)--(4.329,3.580);
\draw[gp path] (7.511,3.580)--(7.511,0.985);
\node[gp node left,font={\fontsize{10pt}{12pt}\selectfont}] at (1.421,5.244) {\LARGE $\rho$};
\node[gp node left,font={\fontsize{10pt}{12pt}\selectfont}] at (6.147,5.244) {\large $\alpha = 3.0$};
%% coordinates of the plot area
\gpdefrectangularnode{gp plot 1}{\pgfpoint{1.196cm}{0.985cm}}{\pgfpoint{7.947cm}{5.631cm}}
\end{tikzpicture}
%% gnuplot variables
}
& 
\resizebox{0.5\linewidth}{!}{\tikzsetnextfilename{coplanar_b_rsol_init_6} \begin{tikzpicture}[gnuplot]
%% generated with GNUPLOT 4.6p4 (Lua 5.1; terminal rev. 99, script rev. 100)
%% Sun 01 Jun 2014 04:12:30 PM EDT
\path (0.000,0.000) rectangle (8.500,6.000);
\gpfill{rgb color={1.000,1.000,1.000}} (1.196,0.985)--(7.946,0.985)--(7.946,5.630)--(1.196,5.630)--cycle;
\gpcolor{color=gp lt color border}
\gpsetlinetype{gp lt border}
\gpsetlinewidth{1.00}
\draw[gp path] (1.196,0.985)--(1.196,5.630)--(7.946,5.630)--(7.946,0.985)--cycle;
\gpcolor{color=gp lt color axes}
\gpsetlinetype{gp lt axes}
\gpsetlinewidth{2.00}
\draw[gp path] (1.196,0.985)--(7.947,0.985);
\gpcolor{color=gp lt color border}
\gpsetlinetype{gp lt border}
\draw[gp path] (1.196,0.985)--(1.268,0.985);
\draw[gp path] (7.947,0.985)--(7.875,0.985);
\gpcolor{rgb color={0.000,0.000,0.000}}
\node[gp node right,font={\fontsize{10pt}{12pt}\selectfont}] at (1.012,0.985) {-0.4};
\gpcolor{color=gp lt color axes}
\gpsetlinetype{gp lt axes}
\draw[gp path] (1.196,1.759)--(7.947,1.759);
\gpcolor{color=gp lt color border}
\gpsetlinetype{gp lt border}
\draw[gp path] (1.196,1.759)--(1.268,1.759);
\draw[gp path] (7.947,1.759)--(7.875,1.759);
\gpcolor{rgb color={0.000,0.000,0.000}}
\node[gp node right,font={\fontsize{10pt}{12pt}\selectfont}] at (1.012,1.759) {-0.2};
\gpcolor{color=gp lt color axes}
\gpsetlinetype{gp lt axes}
\draw[gp path] (1.196,2.534)--(7.947,2.534);
\gpcolor{color=gp lt color border}
\gpsetlinetype{gp lt border}
\draw[gp path] (1.196,2.534)--(1.268,2.534);
\draw[gp path] (7.947,2.534)--(7.875,2.534);
\gpcolor{rgb color={0.000,0.000,0.000}}
\node[gp node right,font={\fontsize{10pt}{12pt}\selectfont}] at (1.012,2.534) {0};
\gpcolor{color=gp lt color axes}
\gpsetlinetype{gp lt axes}
\draw[gp path] (1.196,3.308)--(7.947,3.308);
\gpcolor{color=gp lt color border}
\gpsetlinetype{gp lt border}
\draw[gp path] (1.196,3.308)--(1.268,3.308);
\draw[gp path] (7.947,3.308)--(7.875,3.308);
\gpcolor{rgb color={0.000,0.000,0.000}}
\node[gp node right,font={\fontsize{10pt}{12pt}\selectfont}] at (1.012,3.308) {0.2};
\gpcolor{color=gp lt color axes}
\gpsetlinetype{gp lt axes}
\draw[gp path] (1.196,4.082)--(7.947,4.082);
\gpcolor{color=gp lt color border}
\gpsetlinetype{gp lt border}
\draw[gp path] (1.196,4.082)--(1.268,4.082);
\draw[gp path] (7.947,4.082)--(7.875,4.082);
\gpcolor{rgb color={0.000,0.000,0.000}}
\node[gp node right,font={\fontsize{10pt}{12pt}\selectfont}] at (1.012,4.082) {0.4};
\gpcolor{color=gp lt color axes}
\gpsetlinetype{gp lt axes}
\draw[gp path] (1.196,4.857)--(7.947,4.857);
\gpcolor{color=gp lt color border}
\gpsetlinetype{gp lt border}
\draw[gp path] (1.196,4.857)--(1.268,4.857);
\draw[gp path] (7.947,4.857)--(7.875,4.857);
\gpcolor{rgb color={0.000,0.000,0.000}}
\node[gp node right,font={\fontsize{10pt}{12pt}\selectfont}] at (1.012,4.857) {0.6};
\gpcolor{color=gp lt color axes}
\gpsetlinetype{gp lt axes}
\draw[gp path] (1.196,5.631)--(7.947,5.631);
\gpcolor{color=gp lt color border}
\gpsetlinetype{gp lt border}
\draw[gp path] (1.196,5.631)--(1.268,5.631);
\draw[gp path] (7.947,5.631)--(7.875,5.631);
\gpcolor{rgb color={0.000,0.000,0.000}}
\node[gp node right,font={\fontsize{10pt}{12pt}\selectfont}] at (1.012,5.631) {0.8};
\gpcolor{color=gp lt color axes}
\gpsetlinetype{gp lt axes}
\draw[gp path] (1.196,0.985)--(1.196,5.631);
\gpcolor{color=gp lt color border}
\gpsetlinetype{gp lt border}
\draw[gp path] (1.196,0.985)--(1.196,1.057);
\draw[gp path] (1.196,5.631)--(1.196,5.559);
\gpcolor{rgb color={0.000,0.000,0.000}}
\node[gp node center,font={\fontsize{10pt}{12pt}\selectfont}] at (1.196,0.677) {0.2};
\gpcolor{color=gp lt color axes}
\gpsetlinetype{gp lt axes}
\draw[gp path] (2.321,0.985)--(2.321,5.631);
\gpcolor{color=gp lt color border}
\gpsetlinetype{gp lt border}
\draw[gp path] (2.321,0.985)--(2.321,1.057);
\draw[gp path] (2.321,5.631)--(2.321,5.559);
\gpcolor{rgb color={0.000,0.000,0.000}}
\node[gp node center,font={\fontsize{10pt}{12pt}\selectfont}] at (2.321,0.677) {0.25};
\gpcolor{color=gp lt color axes}
\gpsetlinetype{gp lt axes}
\draw[gp path] (3.446,0.985)--(3.446,5.631);
\gpcolor{color=gp lt color border}
\gpsetlinetype{gp lt border}
\draw[gp path] (3.446,0.985)--(3.446,1.057);
\draw[gp path] (3.446,5.631)--(3.446,5.559);
\gpcolor{rgb color={0.000,0.000,0.000}}
\node[gp node center,font={\fontsize{10pt}{12pt}\selectfont}] at (3.446,0.677) {0.3};
\gpcolor{color=gp lt color axes}
\gpsetlinetype{gp lt axes}
\draw[gp path] (4.572,0.985)--(4.572,5.631);
\gpcolor{color=gp lt color border}
\gpsetlinetype{gp lt border}
\draw[gp path] (4.572,0.985)--(4.572,1.057);
\draw[gp path] (4.572,5.631)--(4.572,5.559);
\gpcolor{rgb color={0.000,0.000,0.000}}
\node[gp node center,font={\fontsize{10pt}{12pt}\selectfont}] at (4.572,0.677) {0.35};
\gpcolor{color=gp lt color axes}
\gpsetlinetype{gp lt axes}
\draw[gp path] (5.697,0.985)--(5.697,5.631);
\gpcolor{color=gp lt color border}
\gpsetlinetype{gp lt border}
\draw[gp path] (5.697,0.985)--(5.697,1.057);
\draw[gp path] (5.697,5.631)--(5.697,5.559);
\gpcolor{rgb color={0.000,0.000,0.000}}
\node[gp node center,font={\fontsize{10pt}{12pt}\selectfont}] at (5.697,0.677) {0.4};
\gpcolor{color=gp lt color axes}
\gpsetlinetype{gp lt axes}
\draw[gp path] (6.822,0.985)--(6.822,5.631);
\gpcolor{color=gp lt color border}
\gpsetlinetype{gp lt border}
\draw[gp path] (6.822,0.985)--(6.822,1.057);
\draw[gp path] (6.822,5.631)--(6.822,5.559);
\gpcolor{rgb color={0.000,0.000,0.000}}
\node[gp node center,font={\fontsize{10pt}{12pt}\selectfont}] at (6.822,0.677) {0.45};
\gpcolor{color=gp lt color axes}
\gpsetlinetype{gp lt axes}
\draw[gp path] (7.947,0.985)--(7.947,5.631);
\gpcolor{color=gp lt color border}
\gpsetlinetype{gp lt border}
\draw[gp path] (7.947,0.985)--(7.947,1.057);
\draw[gp path] (7.947,5.631)--(7.947,5.559);
\gpcolor{rgb color={0.000,0.000,0.000}}
\node[gp node center,font={\fontsize{10pt}{12pt}\selectfont}] at (7.947,0.677) {0.5};
\gpcolor{color=gp lt color border}
\draw[gp path] (1.196,5.631)--(1.196,0.985)--(7.947,0.985)--(7.947,5.631)--cycle;
\gpcolor{rgb color={0.000,0.000,0.000}}
\node[gp node center,font={\fontsize{10pt}{12pt}\selectfont}] at (4.571,0.215) {\large $x$};
\gpcolor{rgb color={1.000,0.000,0.000}}
\gpsetlinewidth{0.50}
\gpsetpointsize{4.44}
\gppoint{gp mark 7}{(1.203,4.630)}
\gppoint{gp mark 7}{(1.209,4.627)}
\gppoint{gp mark 7}{(1.214,4.624)}
\gppoint{gp mark 7}{(1.220,4.622)}
\gppoint{gp mark 7}{(1.225,4.619)}
\gppoint{gp mark 7}{(1.231,4.616)}
\gppoint{gp mark 7}{(1.236,4.614)}
\gppoint{gp mark 7}{(1.242,4.611)}
\gppoint{gp mark 7}{(1.247,4.609)}
\gppoint{gp mark 7}{(1.253,4.606)}
\gppoint{gp mark 7}{(1.258,4.603)}
\gppoint{gp mark 7}{(1.264,4.601)}
\gppoint{gp mark 7}{(1.269,4.598)}
\gppoint{gp mark 7}{(1.275,4.595)}
\gppoint{gp mark 7}{(1.280,4.593)}
\gppoint{gp mark 7}{(1.286,4.590)}
\gppoint{gp mark 7}{(1.291,4.587)}
\gppoint{gp mark 7}{(1.297,4.585)}
\gppoint{gp mark 7}{(1.302,4.582)}
\gppoint{gp mark 7}{(1.308,4.579)}
\gppoint{gp mark 7}{(1.313,4.577)}
\gppoint{gp mark 7}{(1.319,4.574)}
\gppoint{gp mark 7}{(1.324,4.572)}
\gppoint{gp mark 7}{(1.330,4.569)}
\gppoint{gp mark 7}{(1.335,4.566)}
\gppoint{gp mark 7}{(1.340,4.564)}
\gppoint{gp mark 7}{(1.346,4.561)}
\gppoint{gp mark 7}{(1.351,4.558)}
\gppoint{gp mark 7}{(1.357,4.556)}
\gppoint{gp mark 7}{(1.362,4.553)}
\gppoint{gp mark 7}{(1.368,4.550)}
\gppoint{gp mark 7}{(1.373,4.548)}
\gppoint{gp mark 7}{(1.379,4.545)}
\gppoint{gp mark 7}{(1.384,4.542)}
\gppoint{gp mark 7}{(1.390,4.540)}
\gppoint{gp mark 7}{(1.395,4.537)}
\gppoint{gp mark 7}{(1.401,4.534)}
\gppoint{gp mark 7}{(1.406,4.532)}
\gppoint{gp mark 7}{(1.412,4.529)}
\gppoint{gp mark 7}{(1.417,4.527)}
\gppoint{gp mark 7}{(1.423,4.524)}
\gppoint{gp mark 7}{(1.428,4.521)}
\gppoint{gp mark 7}{(1.434,4.519)}
\gppoint{gp mark 7}{(1.439,4.516)}
\gppoint{gp mark 7}{(1.445,4.513)}
\gppoint{gp mark 7}{(1.450,4.511)}
\gppoint{gp mark 7}{(1.456,4.508)}
\gppoint{gp mark 7}{(1.461,4.505)}
\gppoint{gp mark 7}{(1.467,4.503)}
\gppoint{gp mark 7}{(1.472,4.500)}
\gppoint{gp mark 7}{(1.478,4.497)}
\gppoint{gp mark 7}{(1.483,4.495)}
\gppoint{gp mark 7}{(1.489,4.492)}
\gppoint{gp mark 7}{(1.494,4.489)}
\gppoint{gp mark 7}{(1.500,4.487)}
\gppoint{gp mark 7}{(1.505,4.484)}
\gppoint{gp mark 7}{(1.511,4.481)}
\gppoint{gp mark 7}{(1.516,4.479)}
\gppoint{gp mark 7}{(1.522,4.476)}
\gppoint{gp mark 7}{(1.527,4.473)}
\gppoint{gp mark 7}{(1.533,4.471)}
\gppoint{gp mark 7}{(1.538,4.468)}
\gppoint{gp mark 7}{(1.544,4.465)}
\gppoint{gp mark 7}{(1.549,4.463)}
\gppoint{gp mark 7}{(1.555,4.460)}
\gppoint{gp mark 7}{(1.560,4.457)}
\gppoint{gp mark 7}{(1.566,4.455)}
\gppoint{gp mark 7}{(1.571,4.452)}
\gppoint{gp mark 7}{(1.577,4.449)}
\gppoint{gp mark 7}{(1.582,4.447)}
\gppoint{gp mark 7}{(1.588,4.444)}
\gppoint{gp mark 7}{(1.593,4.441)}
\gppoint{gp mark 7}{(1.599,4.439)}
\gppoint{gp mark 7}{(1.604,4.436)}
\gppoint{gp mark 7}{(1.610,4.433)}
\gppoint{gp mark 7}{(1.615,4.431)}
\gppoint{gp mark 7}{(1.621,4.428)}
\gppoint{gp mark 7}{(1.626,4.425)}
\gppoint{gp mark 7}{(1.632,4.423)}
\gppoint{gp mark 7}{(1.637,4.420)}
\gppoint{gp mark 7}{(1.643,4.417)}
\gppoint{gp mark 7}{(1.648,4.415)}
\gppoint{gp mark 7}{(1.654,4.412)}
\gppoint{gp mark 7}{(1.659,4.409)}
\gppoint{gp mark 7}{(1.665,4.407)}
\gppoint{gp mark 7}{(1.670,4.404)}
\gppoint{gp mark 7}{(1.676,4.401)}
\gppoint{gp mark 7}{(1.681,4.399)}
\gppoint{gp mark 7}{(1.687,4.396)}
\gppoint{gp mark 7}{(1.692,4.393)}
\gppoint{gp mark 7}{(1.698,4.391)}
\gppoint{gp mark 7}{(1.703,4.388)}
\gppoint{gp mark 7}{(1.709,4.385)}
\gppoint{gp mark 7}{(1.714,4.383)}
\gppoint{gp mark 7}{(1.720,4.380)}
\gppoint{gp mark 7}{(1.725,4.377)}
\gppoint{gp mark 7}{(1.731,4.375)}
\gppoint{gp mark 7}{(1.736,4.372)}
\gppoint{gp mark 7}{(1.742,4.369)}
\gppoint{gp mark 7}{(1.747,4.367)}
\gppoint{gp mark 7}{(1.753,4.364)}
\gppoint{gp mark 7}{(1.758,4.361)}
\gppoint{gp mark 7}{(1.764,4.358)}
\gppoint{gp mark 7}{(1.769,4.356)}
\gppoint{gp mark 7}{(1.775,4.353)}
\gppoint{gp mark 7}{(1.780,4.350)}
\gppoint{gp mark 7}{(1.786,4.348)}
\gppoint{gp mark 7}{(1.791,4.345)}
\gppoint{gp mark 7}{(1.796,4.342)}
\gppoint{gp mark 7}{(1.802,4.340)}
\gppoint{gp mark 7}{(1.807,4.337)}
\gppoint{gp mark 7}{(1.813,4.334)}
\gppoint{gp mark 7}{(1.818,4.332)}
\gppoint{gp mark 7}{(1.824,4.329)}
\gppoint{gp mark 7}{(1.829,4.326)}
\gppoint{gp mark 7}{(1.835,4.324)}
\gppoint{gp mark 7}{(1.840,4.321)}
\gppoint{gp mark 7}{(1.846,4.318)}
\gppoint{gp mark 7}{(1.851,4.316)}
\gppoint{gp mark 7}{(1.857,4.313)}
\gppoint{gp mark 7}{(1.862,4.310)}
\gppoint{gp mark 7}{(1.868,4.307)}
\gppoint{gp mark 7}{(1.873,4.305)}
\gppoint{gp mark 7}{(1.879,4.302)}
\gppoint{gp mark 7}{(1.884,4.299)}
\gppoint{gp mark 7}{(1.890,4.297)}
\gppoint{gp mark 7}{(1.895,4.294)}
\gppoint{gp mark 7}{(1.901,4.291)}
\gppoint{gp mark 7}{(1.906,4.289)}
\gppoint{gp mark 7}{(1.912,4.286)}
\gppoint{gp mark 7}{(1.917,4.283)}
\gppoint{gp mark 7}{(1.923,4.281)}
\gppoint{gp mark 7}{(1.928,4.278)}
\gppoint{gp mark 7}{(1.934,4.275)}
\gppoint{gp mark 7}{(1.939,4.272)}
\gppoint{gp mark 7}{(1.945,4.270)}
\gppoint{gp mark 7}{(1.950,4.267)}
\gppoint{gp mark 7}{(1.956,4.264)}
\gppoint{gp mark 7}{(1.961,4.262)}
\gppoint{gp mark 7}{(1.967,4.259)}
\gppoint{gp mark 7}{(1.972,4.256)}
\gppoint{gp mark 7}{(1.978,4.254)}
\gppoint{gp mark 7}{(1.983,4.251)}
\gppoint{gp mark 7}{(1.989,4.248)}
\gppoint{gp mark 7}{(1.994,4.245)}
\gppoint{gp mark 7}{(2.000,4.243)}
\gppoint{gp mark 7}{(2.005,4.240)}
\gppoint{gp mark 7}{(2.011,4.237)}
\gppoint{gp mark 7}{(2.016,4.235)}
\gppoint{gp mark 7}{(2.022,4.232)}
\gppoint{gp mark 7}{(2.027,4.229)}
\gppoint{gp mark 7}{(2.033,4.227)}
\gppoint{gp mark 7}{(2.038,4.224)}
\gppoint{gp mark 7}{(2.044,4.221)}
\gppoint{gp mark 7}{(2.049,4.219)}
\gppoint{gp mark 7}{(2.055,4.216)}
\gppoint{gp mark 7}{(2.060,4.213)}
\gppoint{gp mark 7}{(2.066,4.210)}
\gppoint{gp mark 7}{(2.071,4.208)}
\gppoint{gp mark 7}{(2.077,4.205)}
\gppoint{gp mark 7}{(2.082,4.202)}
\gppoint{gp mark 7}{(2.088,4.200)}
\gppoint{gp mark 7}{(2.093,4.197)}
\gppoint{gp mark 7}{(2.099,4.194)}
\gppoint{gp mark 7}{(2.104,4.192)}
\gppoint{gp mark 7}{(2.110,4.189)}
\gppoint{gp mark 7}{(2.115,4.186)}
\gppoint{gp mark 7}{(2.121,4.184)}
\gppoint{gp mark 7}{(2.126,4.181)}
\gppoint{gp mark 7}{(2.132,4.178)}
\gppoint{gp mark 7}{(2.137,4.176)}
\gppoint{gp mark 7}{(2.143,4.173)}
\gppoint{gp mark 7}{(2.148,4.170)}
\gppoint{gp mark 7}{(2.154,4.167)}
\gppoint{gp mark 7}{(2.159,4.165)}
\gppoint{gp mark 7}{(2.165,4.162)}
\gppoint{gp mark 7}{(2.170,4.159)}
\gppoint{gp mark 7}{(2.176,4.157)}
\gppoint{gp mark 7}{(2.181,4.154)}
\gppoint{gp mark 7}{(2.187,4.151)}
\gppoint{gp mark 7}{(2.192,4.149)}
\gppoint{gp mark 7}{(2.198,4.146)}
\gppoint{gp mark 7}{(2.203,4.143)}
\gppoint{gp mark 7}{(2.209,4.141)}
\gppoint{gp mark 7}{(2.214,4.138)}
\gppoint{gp mark 7}{(2.220,4.135)}
\gppoint{gp mark 7}{(2.225,4.133)}
\gppoint{gp mark 7}{(2.231,4.130)}
\gppoint{gp mark 7}{(2.236,4.127)}
\gppoint{gp mark 7}{(2.242,4.125)}
\gppoint{gp mark 7}{(2.247,4.122)}
\gppoint{gp mark 7}{(2.252,4.119)}
\gppoint{gp mark 7}{(2.258,4.117)}
\gppoint{gp mark 7}{(2.263,4.114)}
\gppoint{gp mark 7}{(2.269,4.112)}
\gppoint{gp mark 7}{(2.274,4.109)}
\gppoint{gp mark 7}{(2.280,4.106)}
\gppoint{gp mark 7}{(2.285,4.104)}
\gppoint{gp mark 7}{(2.291,4.101)}
\gppoint{gp mark 7}{(2.296,4.098)}
\gppoint{gp mark 7}{(2.302,4.096)}
\gppoint{gp mark 7}{(2.307,4.093)}
\gppoint{gp mark 7}{(2.313,4.091)}
\gppoint{gp mark 7}{(2.318,4.088)}
\gppoint{gp mark 7}{(2.324,4.085)}
\gppoint{gp mark 7}{(2.329,4.083)}
\gppoint{gp mark 7}{(2.335,4.080)}
\gppoint{gp mark 7}{(2.340,4.078)}
\gppoint{gp mark 7}{(2.346,4.075)}
\gppoint{gp mark 7}{(2.351,4.072)}
\gppoint{gp mark 7}{(2.357,4.070)}
\gppoint{gp mark 7}{(2.362,4.067)}
\gppoint{gp mark 7}{(2.368,4.065)}
\gppoint{gp mark 7}{(2.373,4.062)}
\gppoint{gp mark 7}{(2.379,4.060)}
\gppoint{gp mark 7}{(2.384,4.057)}
\gppoint{gp mark 7}{(2.390,4.054)}
\gppoint{gp mark 7}{(2.395,4.052)}
\gppoint{gp mark 7}{(2.401,4.049)}
\gppoint{gp mark 7}{(2.406,4.047)}
\gppoint{gp mark 7}{(2.412,4.044)}
\gppoint{gp mark 7}{(2.417,4.042)}
\gppoint{gp mark 7}{(2.423,4.039)}
\gppoint{gp mark 7}{(2.428,4.037)}
\gppoint{gp mark 7}{(2.434,4.034)}
\gppoint{gp mark 7}{(2.439,4.032)}
\gppoint{gp mark 7}{(2.445,4.030)}
\gppoint{gp mark 7}{(2.450,4.028)}
\gppoint{gp mark 7}{(2.456,4.027)}
\gppoint{gp mark 7}{(2.461,4.026)}
\gppoint{gp mark 7}{(2.467,4.025)}
\gppoint{gp mark 7}{(2.472,4.025)}
\gppoint{gp mark 7}{(2.478,4.025)}
\gppoint{gp mark 7}{(2.483,4.025)}
\gppoint{gp mark 7}{(2.489,4.025)}
\gppoint{gp mark 7}{(2.494,4.025)}
\gppoint{gp mark 7}{(2.500,4.025)}
\gppoint{gp mark 7}{(2.505,4.026)}
\gppoint{gp mark 7}{(2.511,4.027)}
\gppoint{gp mark 7}{(2.516,4.027)}
\gppoint{gp mark 7}{(2.522,4.028)}
\gppoint{gp mark 7}{(2.527,4.028)}
\gppoint{gp mark 7}{(2.533,4.028)}
\gppoint{gp mark 7}{(2.538,4.028)}
\gppoint{gp mark 7}{(2.544,4.028)}
\gppoint{gp mark 7}{(2.549,4.028)}
\gppoint{gp mark 7}{(2.555,4.028)}
\gppoint{gp mark 7}{(2.560,4.027)}
\gppoint{gp mark 7}{(2.566,4.027)}
\gppoint{gp mark 7}{(2.571,4.027)}
\gppoint{gp mark 7}{(2.577,4.027)}
\gppoint{gp mark 7}{(2.582,4.027)}
\gppoint{gp mark 7}{(2.588,4.027)}
\gppoint{gp mark 7}{(2.593,4.027)}
\gppoint{gp mark 7}{(2.599,4.028)}
\gppoint{gp mark 7}{(2.604,4.028)}
\gppoint{gp mark 7}{(2.610,4.028)}
\gppoint{gp mark 7}{(2.615,4.028)}
\gppoint{gp mark 7}{(2.621,4.028)}
\gppoint{gp mark 7}{(2.626,4.028)}
\gppoint{gp mark 7}{(2.632,4.028)}
\gppoint{gp mark 7}{(2.637,4.028)}
\gppoint{gp mark 7}{(2.643,4.028)}
\gppoint{gp mark 7}{(2.648,4.028)}
\gppoint{gp mark 7}{(2.654,4.028)}
\gppoint{gp mark 7}{(2.659,4.028)}
\gppoint{gp mark 7}{(2.665,4.028)}
\gppoint{gp mark 7}{(2.670,4.028)}
\gppoint{gp mark 7}{(2.676,4.028)}
\gppoint{gp mark 7}{(2.681,4.028)}
\gppoint{gp mark 7}{(2.687,4.028)}
\gppoint{gp mark 7}{(2.692,4.028)}
\gppoint{gp mark 7}{(2.698,4.028)}
\gppoint{gp mark 7}{(2.703,4.028)}
\gppoint{gp mark 7}{(2.708,4.028)}
\gppoint{gp mark 7}{(2.714,4.028)}
\gppoint{gp mark 7}{(2.719,4.028)}
\gppoint{gp mark 7}{(2.725,4.028)}
\gppoint{gp mark 7}{(2.730,4.028)}
\gppoint{gp mark 7}{(2.736,4.028)}
\gppoint{gp mark 7}{(2.741,4.028)}
\gppoint{gp mark 7}{(2.747,4.028)}
\gppoint{gp mark 7}{(2.752,4.028)}
\gppoint{gp mark 7}{(2.758,4.028)}
\gppoint{gp mark 7}{(2.763,4.028)}
\gppoint{gp mark 7}{(2.769,4.028)}
\gppoint{gp mark 7}{(2.774,4.028)}
\gppoint{gp mark 7}{(2.780,4.028)}
\gppoint{gp mark 7}{(2.785,4.028)}
\gppoint{gp mark 7}{(2.791,4.028)}
\gppoint{gp mark 7}{(2.796,4.028)}
\gppoint{gp mark 7}{(2.802,4.028)}
\gppoint{gp mark 7}{(2.807,4.028)}
\gppoint{gp mark 7}{(2.813,4.028)}
\gppoint{gp mark 7}{(2.818,4.028)}
\gppoint{gp mark 7}{(2.824,4.028)}
\gppoint{gp mark 7}{(2.829,4.028)}
\gppoint{gp mark 7}{(2.835,4.028)}
\gppoint{gp mark 7}{(2.840,4.028)}
\gppoint{gp mark 7}{(2.846,4.028)}
\gppoint{gp mark 7}{(2.851,4.028)}
\gppoint{gp mark 7}{(2.857,4.028)}
\gppoint{gp mark 7}{(2.862,4.028)}
\gppoint{gp mark 7}{(2.868,4.028)}
\gppoint{gp mark 7}{(2.873,4.028)}
\gppoint{gp mark 7}{(2.879,4.028)}
\gppoint{gp mark 7}{(2.884,4.028)}
\gppoint{gp mark 7}{(2.890,4.028)}
\gppoint{gp mark 7}{(2.895,4.028)}
\gppoint{gp mark 7}{(2.901,4.028)}
\gppoint{gp mark 7}{(2.906,4.028)}
\gppoint{gp mark 7}{(2.912,4.028)}
\gppoint{gp mark 7}{(2.917,4.028)}
\gppoint{gp mark 7}{(2.923,4.028)}
\gppoint{gp mark 7}{(2.928,4.028)}
\gppoint{gp mark 7}{(2.934,4.028)}
\gppoint{gp mark 7}{(2.939,4.028)}
\gppoint{gp mark 7}{(2.945,4.028)}
\gppoint{gp mark 7}{(2.950,4.028)}
\gppoint{gp mark 7}{(2.956,4.028)}
\gppoint{gp mark 7}{(2.961,4.028)}
\gppoint{gp mark 7}{(2.967,4.028)}
\gppoint{gp mark 7}{(2.972,4.028)}
\gppoint{gp mark 7}{(2.978,4.028)}
\gppoint{gp mark 7}{(2.983,4.028)}
\gppoint{gp mark 7}{(2.989,4.028)}
\gppoint{gp mark 7}{(2.994,4.028)}
\gppoint{gp mark 7}{(3.000,4.028)}
\gppoint{gp mark 7}{(3.005,4.028)}
\gppoint{gp mark 7}{(3.011,4.028)}
\gppoint{gp mark 7}{(3.016,4.028)}
\gppoint{gp mark 7}{(3.022,4.028)}
\gppoint{gp mark 7}{(3.027,4.028)}
\gppoint{gp mark 7}{(3.033,4.028)}
\gppoint{gp mark 7}{(3.038,4.028)}
\gppoint{gp mark 7}{(3.044,4.028)}
\gppoint{gp mark 7}{(3.049,4.028)}
\gppoint{gp mark 7}{(3.055,4.028)}
\gppoint{gp mark 7}{(3.060,4.028)}
\gppoint{gp mark 7}{(3.066,4.028)}
\gppoint{gp mark 7}{(3.071,4.028)}
\gppoint{gp mark 7}{(3.077,4.028)}
\gppoint{gp mark 7}{(3.082,4.028)}
\gppoint{gp mark 7}{(3.088,4.028)}
\gppoint{gp mark 7}{(3.093,4.028)}
\gppoint{gp mark 7}{(3.099,4.028)}
\gppoint{gp mark 7}{(3.104,4.028)}
\gppoint{gp mark 7}{(3.110,4.028)}
\gppoint{gp mark 7}{(3.115,4.028)}
\gppoint{gp mark 7}{(3.121,4.028)}
\gppoint{gp mark 7}{(3.126,4.028)}
\gppoint{gp mark 7}{(3.132,4.028)}
\gppoint{gp mark 7}{(3.137,4.028)}
\gppoint{gp mark 7}{(3.143,4.029)}
\gppoint{gp mark 7}{(3.148,4.029)}
\gppoint{gp mark 7}{(3.154,4.029)}
\gppoint{gp mark 7}{(3.159,4.029)}
\gppoint{gp mark 7}{(3.164,4.029)}
\gppoint{gp mark 7}{(3.170,4.029)}
\gppoint{gp mark 7}{(3.175,4.029)}
\gppoint{gp mark 7}{(3.181,4.029)}
\gppoint{gp mark 7}{(3.186,4.029)}
\gppoint{gp mark 7}{(3.192,4.028)}
\gppoint{gp mark 7}{(3.197,4.028)}
\gppoint{gp mark 7}{(3.203,4.028)}
\gppoint{gp mark 7}{(3.208,4.028)}
\gppoint{gp mark 7}{(3.214,4.028)}
\gppoint{gp mark 7}{(3.219,4.028)}
\gppoint{gp mark 7}{(3.225,4.029)}
\gppoint{gp mark 7}{(3.230,4.029)}
\gppoint{gp mark 7}{(3.236,4.029)}
\gppoint{gp mark 7}{(3.241,4.029)}
\gppoint{gp mark 7}{(3.247,4.029)}
\gppoint{gp mark 7}{(3.252,4.029)}
\gppoint{gp mark 7}{(3.258,4.029)}
\gppoint{gp mark 7}{(3.263,4.029)}
\gppoint{gp mark 7}{(3.269,4.029)}
\gppoint{gp mark 7}{(3.274,4.029)}
\gppoint{gp mark 7}{(3.280,4.029)}
\gppoint{gp mark 7}{(3.285,4.029)}
\gppoint{gp mark 7}{(3.291,4.029)}
\gppoint{gp mark 7}{(3.296,4.029)}
\gppoint{gp mark 7}{(3.302,4.029)}
\gppoint{gp mark 7}{(3.307,4.029)}
\gppoint{gp mark 7}{(3.313,4.029)}
\gppoint{gp mark 7}{(3.318,4.029)}
\gppoint{gp mark 7}{(3.324,4.029)}
\gppoint{gp mark 7}{(3.329,4.029)}
\gppoint{gp mark 7}{(3.335,4.029)}
\gppoint{gp mark 7}{(3.340,4.029)}
\gppoint{gp mark 7}{(3.346,4.029)}
\gppoint{gp mark 7}{(3.351,4.029)}
\gppoint{gp mark 7}{(3.357,4.029)}
\gppoint{gp mark 7}{(3.362,4.029)}
\gppoint{gp mark 7}{(3.368,4.029)}
\gppoint{gp mark 7}{(3.373,4.029)}
\gppoint{gp mark 7}{(3.379,4.029)}
\gppoint{gp mark 7}{(3.384,4.029)}
\gppoint{gp mark 7}{(3.390,4.029)}
\gppoint{gp mark 7}{(3.395,4.029)}
\gppoint{gp mark 7}{(3.401,4.029)}
\gppoint{gp mark 7}{(3.406,4.029)}
\gppoint{gp mark 7}{(3.412,4.029)}
\gppoint{gp mark 7}{(3.417,4.029)}
\gppoint{gp mark 7}{(3.423,4.029)}
\gppoint{gp mark 7}{(3.428,4.029)}
\gppoint{gp mark 7}{(3.434,4.029)}
\gppoint{gp mark 7}{(3.439,4.029)}
\gppoint{gp mark 7}{(3.445,4.029)}
\gppoint{gp mark 7}{(3.450,4.029)}
\gppoint{gp mark 7}{(3.456,4.029)}
\gppoint{gp mark 7}{(3.461,4.029)}
\gppoint{gp mark 7}{(3.467,4.029)}
\gppoint{gp mark 7}{(3.472,4.029)}
\gppoint{gp mark 7}{(3.478,4.029)}
\gppoint{gp mark 7}{(3.483,4.029)}
\gppoint{gp mark 7}{(3.489,4.029)}
\gppoint{gp mark 7}{(3.494,4.029)}
\gppoint{gp mark 7}{(3.500,4.029)}
\gppoint{gp mark 7}{(3.505,4.029)}
\gppoint{gp mark 7}{(3.511,4.028)}
\gppoint{gp mark 7}{(3.516,4.019)}
\gppoint{gp mark 7}{(3.522,3.993)}
\gppoint{gp mark 7}{(3.527,3.105)}
\gppoint{gp mark 7}{(3.533,1.146)}
\gppoint{gp mark 7}{(3.538,1.083)}
\gppoint{gp mark 7}{(3.544,1.097)}
\gppoint{gp mark 7}{(3.549,1.076)}
\gppoint{gp mark 7}{(3.555,1.082)}
\gppoint{gp mark 7}{(3.560,1.095)}
\gppoint{gp mark 7}{(3.566,1.092)}
\gppoint{gp mark 7}{(3.571,1.084)}
\gppoint{gp mark 7}{(3.577,1.084)}
\gppoint{gp mark 7}{(3.582,1.087)}
\gppoint{gp mark 7}{(3.588,1.088)}
\gppoint{gp mark 7}{(3.593,1.086)}
\gppoint{gp mark 7}{(3.599,1.086)}
\gppoint{gp mark 7}{(3.604,1.086)}
\gppoint{gp mark 7}{(3.610,1.087)}
\gppoint{gp mark 7}{(3.615,1.086)}
\gppoint{gp mark 7}{(3.620,1.086)}
\gppoint{gp mark 7}{(3.626,1.086)}
\gppoint{gp mark 7}{(3.631,1.086)}
\gppoint{gp mark 7}{(3.637,1.086)}
\gppoint{gp mark 7}{(3.642,1.086)}
\gppoint{gp mark 7}{(3.648,1.086)}
\gppoint{gp mark 7}{(3.653,1.086)}
\gppoint{gp mark 7}{(3.659,1.087)}
\gppoint{gp mark 7}{(3.664,1.087)}
\gppoint{gp mark 7}{(3.670,1.087)}
\gppoint{gp mark 7}{(3.675,1.086)}
\gppoint{gp mark 7}{(3.681,1.086)}
\gppoint{gp mark 7}{(3.686,1.086)}
\gppoint{gp mark 7}{(3.692,1.086)}
\gppoint{gp mark 7}{(3.697,1.086)}
\gppoint{gp mark 7}{(3.703,1.086)}
\gppoint{gp mark 7}{(3.708,1.086)}
\gppoint{gp mark 7}{(3.714,1.086)}
\gppoint{gp mark 7}{(3.719,1.086)}
\gppoint{gp mark 7}{(3.725,1.086)}
\gppoint{gp mark 7}{(3.730,1.086)}
\gppoint{gp mark 7}{(3.736,1.086)}
\gppoint{gp mark 7}{(3.741,1.086)}
\gppoint{gp mark 7}{(3.747,1.086)}
\gppoint{gp mark 7}{(3.752,1.086)}
\gppoint{gp mark 7}{(3.758,1.086)}
\gppoint{gp mark 7}{(3.763,1.087)}
\gppoint{gp mark 7}{(3.769,1.087)}
\gppoint{gp mark 7}{(3.774,1.087)}
\gppoint{gp mark 7}{(3.780,1.087)}
\gppoint{gp mark 7}{(3.785,1.087)}
\gppoint{gp mark 7}{(3.791,1.087)}
\gppoint{gp mark 7}{(3.796,1.087)}
\gppoint{gp mark 7}{(3.802,1.087)}
\gppoint{gp mark 7}{(3.807,1.086)}
\gppoint{gp mark 7}{(3.813,1.086)}
\gppoint{gp mark 7}{(3.818,1.086)}
\gppoint{gp mark 7}{(3.824,1.086)}
\gppoint{gp mark 7}{(3.829,1.086)}
\gppoint{gp mark 7}{(3.835,1.086)}
\gppoint{gp mark 7}{(3.840,1.086)}
\gppoint{gp mark 7}{(3.846,1.086)}
\gppoint{gp mark 7}{(3.851,1.086)}
\gppoint{gp mark 7}{(3.857,1.086)}
\gppoint{gp mark 7}{(3.862,1.087)}
\gppoint{gp mark 7}{(3.868,1.087)}
\gppoint{gp mark 7}{(3.873,1.087)}
\gppoint{gp mark 7}{(3.879,1.087)}
\gppoint{gp mark 7}{(3.884,1.086)}
\gppoint{gp mark 7}{(3.890,1.086)}
\gppoint{gp mark 7}{(3.895,1.086)}
\gppoint{gp mark 7}{(3.901,1.086)}
\gppoint{gp mark 7}{(3.906,1.087)}
\gppoint{gp mark 7}{(3.912,1.087)}
\gppoint{gp mark 7}{(3.917,1.087)}
\gppoint{gp mark 7}{(3.923,1.087)}
\gppoint{gp mark 7}{(3.928,1.087)}
\gppoint{gp mark 7}{(3.934,1.087)}
\gppoint{gp mark 7}{(3.939,1.087)}
\gppoint{gp mark 7}{(3.945,1.087)}
\gppoint{gp mark 7}{(3.950,1.087)}
\gppoint{gp mark 7}{(3.956,1.087)}
\gppoint{gp mark 7}{(3.961,1.087)}
\gppoint{gp mark 7}{(3.967,1.087)}
\gppoint{gp mark 7}{(3.972,1.087)}
\gppoint{gp mark 7}{(3.978,1.087)}
\gppoint{gp mark 7}{(3.983,1.087)}
\gppoint{gp mark 7}{(3.989,1.087)}
\gppoint{gp mark 7}{(3.994,1.087)}
\gppoint{gp mark 7}{(4.000,1.087)}
\gppoint{gp mark 7}{(4.005,1.087)}
\gppoint{gp mark 7}{(4.011,1.087)}
\gppoint{gp mark 7}{(4.016,1.087)}
\gppoint{gp mark 7}{(4.022,1.087)}
\gppoint{gp mark 7}{(4.027,1.087)}
\gppoint{gp mark 7}{(4.033,1.087)}
\gppoint{gp mark 7}{(4.038,1.087)}
\gppoint{gp mark 7}{(4.044,1.087)}
\gppoint{gp mark 7}{(4.049,1.087)}
\gppoint{gp mark 7}{(4.055,1.087)}
\gppoint{gp mark 7}{(4.060,1.087)}
\gppoint{gp mark 7}{(4.066,1.087)}
\gppoint{gp mark 7}{(4.071,1.087)}
\gppoint{gp mark 7}{(4.076,1.087)}
\gppoint{gp mark 7}{(4.082,1.087)}
\gppoint{gp mark 7}{(4.087,1.087)}
\gppoint{gp mark 7}{(4.093,1.087)}
\gppoint{gp mark 7}{(4.098,1.087)}
\gppoint{gp mark 7}{(4.104,1.087)}
\gppoint{gp mark 7}{(4.109,1.087)}
\gppoint{gp mark 7}{(4.115,1.087)}
\gppoint{gp mark 7}{(4.120,1.087)}
\gppoint{gp mark 7}{(4.126,1.087)}
\gppoint{gp mark 7}{(4.131,1.087)}
\gppoint{gp mark 7}{(4.137,1.087)}
\gppoint{gp mark 7}{(4.142,1.087)}
\gppoint{gp mark 7}{(4.148,1.087)}
\gppoint{gp mark 7}{(4.153,1.087)}
\gppoint{gp mark 7}{(4.159,1.087)}
\gppoint{gp mark 7}{(4.164,1.087)}
\gppoint{gp mark 7}{(4.170,1.087)}
\gppoint{gp mark 7}{(4.175,1.087)}
\gppoint{gp mark 7}{(4.181,1.087)}
\gppoint{gp mark 7}{(4.186,1.087)}
\gppoint{gp mark 7}{(4.192,1.087)}
\gppoint{gp mark 7}{(4.197,1.087)}
\gppoint{gp mark 7}{(4.203,1.087)}
\gppoint{gp mark 7}{(4.208,1.087)}
\gppoint{gp mark 7}{(4.214,1.087)}
\gppoint{gp mark 7}{(4.219,1.088)}
\gppoint{gp mark 7}{(4.225,1.088)}
\gppoint{gp mark 7}{(4.230,1.087)}
\gppoint{gp mark 7}{(4.236,1.087)}
\gppoint{gp mark 7}{(4.241,1.087)}
\gppoint{gp mark 7}{(4.247,1.087)}
\gppoint{gp mark 7}{(4.252,1.087)}
\gppoint{gp mark 7}{(4.258,1.087)}
\gppoint{gp mark 7}{(4.263,1.088)}
\gppoint{gp mark 7}{(4.269,1.088)}
\gppoint{gp mark 7}{(4.274,1.087)}
\gppoint{gp mark 7}{(4.280,1.087)}
\gppoint{gp mark 7}{(4.285,1.087)}
\gppoint{gp mark 7}{(4.291,1.087)}
\gppoint{gp mark 7}{(4.296,1.087)}
\gppoint{gp mark 7}{(4.302,1.088)}
\gppoint{gp mark 7}{(4.307,1.089)}
\gppoint{gp mark 7}{(4.313,1.092)}
\gppoint{gp mark 7}{(4.318,1.116)}
\gppoint{gp mark 7}{(4.324,1.213)}
\gppoint{gp mark 7}{(4.329,1.302)}
\gppoint{gp mark 7}{(4.335,1.313)}
\gppoint{gp mark 7}{(4.340,1.315)}
\gppoint{gp mark 7}{(4.346,1.316)}
\gppoint{gp mark 7}{(4.351,1.316)}
\gppoint{gp mark 7}{(4.357,1.315)}
\gppoint{gp mark 7}{(4.362,1.315)}
\gppoint{gp mark 7}{(4.368,1.315)}
\gppoint{gp mark 7}{(4.373,1.315)}
\gppoint{gp mark 7}{(4.379,1.315)}
\gppoint{gp mark 7}{(4.384,1.315)}
\gppoint{gp mark 7}{(4.390,1.315)}
\gppoint{gp mark 7}{(4.395,1.315)}
\gppoint{gp mark 7}{(4.401,1.315)}
\gppoint{gp mark 7}{(4.406,1.315)}
\gppoint{gp mark 7}{(4.412,1.315)}
\gppoint{gp mark 7}{(4.417,1.315)}
\gppoint{gp mark 7}{(4.423,1.315)}
\gppoint{gp mark 7}{(4.428,1.315)}
\gppoint{gp mark 7}{(4.434,1.315)}
\gppoint{gp mark 7}{(4.439,1.315)}
\gppoint{gp mark 7}{(4.445,1.315)}
\gppoint{gp mark 7}{(4.450,1.315)}
\gppoint{gp mark 7}{(4.456,1.315)}
\gppoint{gp mark 7}{(4.461,1.315)}
\gppoint{gp mark 7}{(4.467,1.315)}
\gppoint{gp mark 7}{(4.472,1.315)}
\gppoint{gp mark 7}{(4.478,1.315)}
\gppoint{gp mark 7}{(4.483,1.315)}
\gppoint{gp mark 7}{(4.489,1.315)}
\gppoint{gp mark 7}{(4.494,1.315)}
\gppoint{gp mark 7}{(4.500,1.315)}
\gppoint{gp mark 7}{(4.505,1.315)}
\gppoint{gp mark 7}{(4.511,1.315)}
\gppoint{gp mark 7}{(4.516,1.315)}
\gppoint{gp mark 7}{(4.522,1.315)}
\gppoint{gp mark 7}{(4.527,1.315)}
\gppoint{gp mark 7}{(4.532,1.315)}
\gppoint{gp mark 7}{(4.538,1.315)}
\gppoint{gp mark 7}{(4.543,1.315)}
\gppoint{gp mark 7}{(4.549,1.315)}
\gppoint{gp mark 7}{(4.554,1.315)}
\gppoint{gp mark 7}{(4.560,1.315)}
\gppoint{gp mark 7}{(4.565,1.315)}
\gppoint{gp mark 7}{(4.571,1.315)}
\gppoint{gp mark 7}{(4.576,1.315)}
\gppoint{gp mark 7}{(4.582,1.315)}
\gppoint{gp mark 7}{(4.587,1.315)}
\gppoint{gp mark 7}{(4.593,1.315)}
\gppoint{gp mark 7}{(4.598,1.315)}
\gppoint{gp mark 7}{(4.604,1.315)}
\gppoint{gp mark 7}{(4.609,1.315)}
\gppoint{gp mark 7}{(4.615,1.315)}
\gppoint{gp mark 7}{(4.620,1.315)}
\gppoint{gp mark 7}{(4.626,1.315)}
\gppoint{gp mark 7}{(4.631,1.315)}
\gppoint{gp mark 7}{(4.637,1.315)}
\gppoint{gp mark 7}{(4.642,1.315)}
\gppoint{gp mark 7}{(4.648,1.315)}
\gppoint{gp mark 7}{(4.653,1.315)}
\gppoint{gp mark 7}{(4.659,1.315)}
\gppoint{gp mark 7}{(4.664,1.315)}
\gppoint{gp mark 7}{(4.670,1.315)}
\gppoint{gp mark 7}{(4.675,1.315)}
\gppoint{gp mark 7}{(4.681,1.315)}
\gppoint{gp mark 7}{(4.686,1.315)}
\gppoint{gp mark 7}{(4.692,1.315)}
\gppoint{gp mark 7}{(4.697,1.315)}
\gppoint{gp mark 7}{(4.703,1.315)}
\gppoint{gp mark 7}{(4.708,1.315)}
\gppoint{gp mark 7}{(4.714,1.315)}
\gppoint{gp mark 7}{(4.719,1.315)}
\gppoint{gp mark 7}{(4.725,1.315)}
\gppoint{gp mark 7}{(4.730,1.315)}
\gppoint{gp mark 7}{(4.736,1.315)}
\gppoint{gp mark 7}{(4.741,1.315)}
\gppoint{gp mark 7}{(4.747,1.315)}
\gppoint{gp mark 7}{(4.752,1.315)}
\gppoint{gp mark 7}{(4.758,1.315)}
\gppoint{gp mark 7}{(4.763,1.315)}
\gppoint{gp mark 7}{(4.769,1.315)}
\gppoint{gp mark 7}{(4.774,1.315)}
\gppoint{gp mark 7}{(4.780,1.315)}
\gppoint{gp mark 7}{(4.785,1.315)}
\gppoint{gp mark 7}{(4.791,1.315)}
\gppoint{gp mark 7}{(4.796,1.315)}
\gppoint{gp mark 7}{(4.802,1.315)}
\gppoint{gp mark 7}{(4.807,1.315)}
\gppoint{gp mark 7}{(4.813,1.315)}
\gppoint{gp mark 7}{(4.818,1.315)}
\gppoint{gp mark 7}{(4.824,1.315)}
\gppoint{gp mark 7}{(4.829,1.315)}
\gppoint{gp mark 7}{(4.835,1.315)}
\gppoint{gp mark 7}{(4.840,1.315)}
\gppoint{gp mark 7}{(4.846,1.315)}
\gppoint{gp mark 7}{(4.851,1.315)}
\gppoint{gp mark 7}{(4.857,1.315)}
\gppoint{gp mark 7}{(4.862,1.315)}
\gppoint{gp mark 7}{(4.868,1.315)}
\gppoint{gp mark 7}{(4.873,1.315)}
\gppoint{gp mark 7}{(4.879,1.315)}
\gppoint{gp mark 7}{(4.884,1.315)}
\gppoint{gp mark 7}{(4.890,1.315)}
\gppoint{gp mark 7}{(4.895,1.315)}
\gppoint{gp mark 7}{(4.901,1.315)}
\gppoint{gp mark 7}{(4.906,1.315)}
\gppoint{gp mark 7}{(4.912,1.315)}
\gppoint{gp mark 7}{(4.917,1.315)}
\gppoint{gp mark 7}{(4.923,1.315)}
\gppoint{gp mark 7}{(4.928,1.315)}
\gppoint{gp mark 7}{(4.934,1.315)}
\gppoint{gp mark 7}{(4.939,1.315)}
\gppoint{gp mark 7}{(4.945,1.315)}
\gppoint{gp mark 7}{(4.950,1.315)}
\gppoint{gp mark 7}{(4.956,1.315)}
\gppoint{gp mark 7}{(4.961,1.315)}
\gppoint{gp mark 7}{(4.967,1.315)}
\gppoint{gp mark 7}{(4.972,1.315)}
\gppoint{gp mark 7}{(4.978,1.315)}
\gppoint{gp mark 7}{(4.983,1.315)}
\gppoint{gp mark 7}{(4.988,1.315)}
\gppoint{gp mark 7}{(4.994,1.315)}
\gppoint{gp mark 7}{(4.999,1.315)}
\gppoint{gp mark 7}{(5.005,1.315)}
\gppoint{gp mark 7}{(5.010,1.315)}
\gppoint{gp mark 7}{(5.016,1.315)}
\gppoint{gp mark 7}{(5.021,1.315)}
\gppoint{gp mark 7}{(5.027,1.315)}
\gppoint{gp mark 7}{(5.032,1.315)}
\gppoint{gp mark 7}{(5.038,1.315)}
\gppoint{gp mark 7}{(5.043,1.315)}
\gppoint{gp mark 7}{(5.049,1.315)}
\gppoint{gp mark 7}{(5.054,1.315)}
\gppoint{gp mark 7}{(5.060,1.315)}
\gppoint{gp mark 7}{(5.065,1.315)}
\gppoint{gp mark 7}{(5.071,1.315)}
\gppoint{gp mark 7}{(5.076,1.315)}
\gppoint{gp mark 7}{(5.082,1.315)}
\gppoint{gp mark 7}{(5.087,1.315)}
\gppoint{gp mark 7}{(5.093,1.315)}
\gppoint{gp mark 7}{(5.098,1.315)}
\gppoint{gp mark 7}{(5.104,1.315)}
\gppoint{gp mark 7}{(5.109,1.315)}
\gppoint{gp mark 7}{(5.115,1.315)}
\gppoint{gp mark 7}{(5.120,1.315)}
\gppoint{gp mark 7}{(5.126,1.315)}
\gppoint{gp mark 7}{(5.131,1.315)}
\gppoint{gp mark 7}{(5.137,1.315)}
\gppoint{gp mark 7}{(5.142,1.315)}
\gppoint{gp mark 7}{(5.148,1.315)}
\gppoint{gp mark 7}{(5.153,1.315)}
\gppoint{gp mark 7}{(5.159,1.315)}
\gppoint{gp mark 7}{(5.164,1.315)}
\gppoint{gp mark 7}{(5.170,1.315)}
\gppoint{gp mark 7}{(5.175,1.315)}
\gppoint{gp mark 7}{(5.181,1.315)}
\gppoint{gp mark 7}{(5.186,1.315)}
\gppoint{gp mark 7}{(5.192,1.315)}
\gppoint{gp mark 7}{(5.197,1.315)}
\gppoint{gp mark 7}{(5.203,1.315)}
\gppoint{gp mark 7}{(5.208,1.315)}
\gppoint{gp mark 7}{(5.214,1.315)}
\gppoint{gp mark 7}{(5.219,1.315)}
\gppoint{gp mark 7}{(5.225,1.315)}
\gppoint{gp mark 7}{(5.230,1.315)}
\gppoint{gp mark 7}{(5.236,1.315)}
\gppoint{gp mark 7}{(5.241,1.315)}
\gppoint{gp mark 7}{(5.247,1.315)}
\gppoint{gp mark 7}{(5.252,1.315)}
\gppoint{gp mark 7}{(5.258,1.315)}
\gppoint{gp mark 7}{(5.263,1.315)}
\gppoint{gp mark 7}{(5.269,1.315)}
\gppoint{gp mark 7}{(5.274,1.315)}
\gppoint{gp mark 7}{(5.280,1.315)}
\gppoint{gp mark 7}{(5.285,1.315)}
\gppoint{gp mark 7}{(5.291,1.315)}
\gppoint{gp mark 7}{(5.296,1.315)}
\gppoint{gp mark 7}{(5.302,1.315)}
\gppoint{gp mark 7}{(5.307,1.315)}
\gppoint{gp mark 7}{(5.313,1.315)}
\gppoint{gp mark 7}{(5.318,1.315)}
\gppoint{gp mark 7}{(5.324,1.315)}
\gppoint{gp mark 7}{(5.329,1.315)}
\gppoint{gp mark 7}{(5.335,1.315)}
\gppoint{gp mark 7}{(5.340,1.315)}
\gppoint{gp mark 7}{(5.346,1.315)}
\gppoint{gp mark 7}{(5.351,1.315)}
\gppoint{gp mark 7}{(5.357,1.315)}
\gppoint{gp mark 7}{(5.362,1.315)}
\gppoint{gp mark 7}{(5.368,1.315)}
\gppoint{gp mark 7}{(5.373,1.315)}
\gppoint{gp mark 7}{(5.379,1.315)}
\gppoint{gp mark 7}{(5.384,1.315)}
\gppoint{gp mark 7}{(5.390,1.315)}
\gppoint{gp mark 7}{(5.395,1.315)}
\gppoint{gp mark 7}{(5.401,1.315)}
\gppoint{gp mark 7}{(5.406,1.315)}
\gppoint{gp mark 7}{(5.412,1.315)}
\gppoint{gp mark 7}{(5.417,1.315)}
\gppoint{gp mark 7}{(5.423,1.315)}
\gppoint{gp mark 7}{(5.428,1.315)}
\gppoint{gp mark 7}{(5.434,1.315)}
\gppoint{gp mark 7}{(5.439,1.315)}
\gppoint{gp mark 7}{(5.444,1.315)}
\gppoint{gp mark 7}{(5.450,1.315)}
\gppoint{gp mark 7}{(5.455,1.315)}
\gppoint{gp mark 7}{(5.461,1.315)}
\gppoint{gp mark 7}{(5.466,1.315)}
\gppoint{gp mark 7}{(5.472,1.315)}
\gppoint{gp mark 7}{(5.477,1.315)}
\gppoint{gp mark 7}{(5.483,1.315)}
\gppoint{gp mark 7}{(5.488,1.315)}
\gppoint{gp mark 7}{(5.494,1.315)}
\gppoint{gp mark 7}{(5.499,1.315)}
\gppoint{gp mark 7}{(5.505,1.315)}
\gppoint{gp mark 7}{(5.510,1.315)}
\gppoint{gp mark 7}{(5.516,1.315)}
\gppoint{gp mark 7}{(5.521,1.315)}
\gppoint{gp mark 7}{(5.527,1.315)}
\gppoint{gp mark 7}{(5.532,1.315)}
\gppoint{gp mark 7}{(5.538,1.315)}
\gppoint{gp mark 7}{(5.543,1.315)}
\gppoint{gp mark 7}{(5.549,1.315)}
\gppoint{gp mark 7}{(5.554,1.315)}
\gppoint{gp mark 7}{(5.560,1.315)}
\gppoint{gp mark 7}{(5.565,1.315)}
\gppoint{gp mark 7}{(5.571,1.315)}
\gppoint{gp mark 7}{(5.576,1.315)}
\gppoint{gp mark 7}{(5.582,1.315)}
\gppoint{gp mark 7}{(5.587,1.315)}
\gppoint{gp mark 7}{(5.593,1.315)}
\gppoint{gp mark 7}{(5.598,1.315)}
\gppoint{gp mark 7}{(5.604,1.315)}
\gppoint{gp mark 7}{(5.609,1.315)}
\gppoint{gp mark 7}{(5.615,1.315)}
\gppoint{gp mark 7}{(5.620,1.315)}
\gppoint{gp mark 7}{(5.626,1.315)}
\gppoint{gp mark 7}{(5.631,1.315)}
\gppoint{gp mark 7}{(5.637,1.315)}
\gppoint{gp mark 7}{(5.642,1.315)}
\gppoint{gp mark 7}{(5.648,1.315)}
\gppoint{gp mark 7}{(5.653,1.315)}
\gppoint{gp mark 7}{(5.659,1.315)}
\gppoint{gp mark 7}{(5.664,1.315)}
\gppoint{gp mark 7}{(5.670,1.315)}
\gppoint{gp mark 7}{(5.675,1.315)}
\gppoint{gp mark 7}{(5.681,1.315)}
\gppoint{gp mark 7}{(5.686,1.315)}
\gppoint{gp mark 7}{(5.692,1.315)}
\gppoint{gp mark 7}{(5.697,1.315)}
\gppoint{gp mark 7}{(5.703,1.315)}
\gppoint{gp mark 7}{(5.708,1.315)}
\gppoint{gp mark 7}{(5.714,1.315)}
\gppoint{gp mark 7}{(5.719,1.315)}
\gppoint{gp mark 7}{(5.725,1.315)}
\gppoint{gp mark 7}{(5.730,1.315)}
\gppoint{gp mark 7}{(5.736,1.315)}
\gppoint{gp mark 7}{(5.741,1.315)}
\gppoint{gp mark 7}{(5.747,1.315)}
\gppoint{gp mark 7}{(5.752,1.315)}
\gppoint{gp mark 7}{(5.758,1.315)}
\gppoint{gp mark 7}{(5.763,1.315)}
\gppoint{gp mark 7}{(5.769,1.315)}
\gppoint{gp mark 7}{(5.774,1.315)}
\gppoint{gp mark 7}{(5.780,1.315)}
\gppoint{gp mark 7}{(5.785,1.315)}
\gppoint{gp mark 7}{(5.791,1.315)}
\gppoint{gp mark 7}{(5.796,1.315)}
\gppoint{gp mark 7}{(5.802,1.315)}
\gppoint{gp mark 7}{(5.807,1.315)}
\gppoint{gp mark 7}{(5.813,1.315)}
\gppoint{gp mark 7}{(5.818,1.315)}
\gppoint{gp mark 7}{(5.824,1.315)}
\gppoint{gp mark 7}{(5.829,1.315)}
\gppoint{gp mark 7}{(5.835,1.315)}
\gppoint{gp mark 7}{(5.840,1.315)}
\gppoint{gp mark 7}{(5.846,1.315)}
\gppoint{gp mark 7}{(5.851,1.315)}
\gppoint{gp mark 7}{(5.857,1.315)}
\gppoint{gp mark 7}{(5.862,1.315)}
\gppoint{gp mark 7}{(5.868,1.315)}
\gppoint{gp mark 7}{(5.873,1.315)}
\gppoint{gp mark 7}{(5.879,1.315)}
\gppoint{gp mark 7}{(5.884,1.315)}
\gppoint{gp mark 7}{(5.890,1.315)}
\gppoint{gp mark 7}{(5.895,1.315)}
\gppoint{gp mark 7}{(5.900,1.315)}
\gppoint{gp mark 7}{(5.906,1.315)}
\gppoint{gp mark 7}{(5.911,1.315)}
\gppoint{gp mark 7}{(5.917,1.315)}
\gppoint{gp mark 7}{(5.922,1.315)}
\gppoint{gp mark 7}{(5.928,1.315)}
\gppoint{gp mark 7}{(5.933,1.315)}
\gppoint{gp mark 7}{(5.939,1.315)}
\gppoint{gp mark 7}{(5.944,1.315)}
\gppoint{gp mark 7}{(5.950,1.315)}
\gppoint{gp mark 7}{(5.955,1.315)}
\gppoint{gp mark 7}{(5.961,1.315)}
\gppoint{gp mark 7}{(5.966,1.315)}
\gppoint{gp mark 7}{(5.972,1.315)}
\gppoint{gp mark 7}{(5.977,1.315)}
\gppoint{gp mark 7}{(5.983,1.315)}
\gppoint{gp mark 7}{(5.988,1.315)}
\gppoint{gp mark 7}{(5.994,1.315)}
\gppoint{gp mark 7}{(5.999,1.315)}
\gppoint{gp mark 7}{(6.005,1.315)}
\gppoint{gp mark 7}{(6.010,1.315)}
\gppoint{gp mark 7}{(6.016,1.315)}
\gppoint{gp mark 7}{(6.021,1.315)}
\gppoint{gp mark 7}{(6.027,1.315)}
\gppoint{gp mark 7}{(6.032,1.315)}
\gppoint{gp mark 7}{(6.038,1.315)}
\gppoint{gp mark 7}{(6.043,1.315)}
\gppoint{gp mark 7}{(6.049,1.315)}
\gppoint{gp mark 7}{(6.054,1.315)}
\gppoint{gp mark 7}{(6.060,1.315)}
\gppoint{gp mark 7}{(6.065,1.315)}
\gppoint{gp mark 7}{(6.071,1.315)}
\gppoint{gp mark 7}{(6.076,1.315)}
\gppoint{gp mark 7}{(6.082,1.315)}
\gppoint{gp mark 7}{(6.087,1.315)}
\gppoint{gp mark 7}{(6.093,1.315)}
\gppoint{gp mark 7}{(6.098,1.315)}
\gppoint{gp mark 7}{(6.104,1.315)}
\gppoint{gp mark 7}{(6.109,1.315)}
\gppoint{gp mark 7}{(6.115,1.315)}
\gppoint{gp mark 7}{(6.120,1.315)}
\gppoint{gp mark 7}{(6.126,1.315)}
\gppoint{gp mark 7}{(6.131,1.315)}
\gppoint{gp mark 7}{(6.137,1.315)}
\gppoint{gp mark 7}{(6.142,1.315)}
\gppoint{gp mark 7}{(6.148,1.315)}
\gppoint{gp mark 7}{(6.153,1.315)}
\gppoint{gp mark 7}{(6.159,1.315)}
\gppoint{gp mark 7}{(6.164,1.315)}
\gppoint{gp mark 7}{(6.170,1.315)}
\gppoint{gp mark 7}{(6.175,1.315)}
\gppoint{gp mark 7}{(6.181,1.315)}
\gppoint{gp mark 7}{(6.186,1.315)}
\gppoint{gp mark 7}{(6.192,1.315)}
\gppoint{gp mark 7}{(6.197,1.315)}
\gppoint{gp mark 7}{(6.203,1.315)}
\gppoint{gp mark 7}{(6.208,1.315)}
\gppoint{gp mark 7}{(6.214,1.315)}
\gppoint{gp mark 7}{(6.219,1.315)}
\gppoint{gp mark 7}{(6.225,1.315)}
\gppoint{gp mark 7}{(6.230,1.315)}
\gppoint{gp mark 7}{(6.236,1.315)}
\gppoint{gp mark 7}{(6.241,1.315)}
\gppoint{gp mark 7}{(6.247,1.315)}
\gppoint{gp mark 7}{(6.252,1.315)}
\gppoint{gp mark 7}{(6.258,1.315)}
\gppoint{gp mark 7}{(6.263,1.315)}
\gppoint{gp mark 7}{(6.269,1.315)}
\gppoint{gp mark 7}{(6.274,1.315)}
\gppoint{gp mark 7}{(6.280,1.315)}
\gppoint{gp mark 7}{(6.285,1.315)}
\gppoint{gp mark 7}{(6.291,1.315)}
\gppoint{gp mark 7}{(6.296,1.315)}
\gppoint{gp mark 7}{(6.302,1.315)}
\gppoint{gp mark 7}{(6.307,1.315)}
\gppoint{gp mark 7}{(6.313,1.315)}
\gppoint{gp mark 7}{(6.318,1.315)}
\gppoint{gp mark 7}{(6.324,1.315)}
\gppoint{gp mark 7}{(6.329,1.315)}
\gppoint{gp mark 7}{(6.335,1.315)}
\gppoint{gp mark 7}{(6.340,1.315)}
\gppoint{gp mark 7}{(6.346,1.315)}
\gppoint{gp mark 7}{(6.351,1.315)}
\gppoint{gp mark 7}{(6.356,1.315)}
\gppoint{gp mark 7}{(6.362,1.315)}
\gppoint{gp mark 7}{(6.367,1.315)}
\gppoint{gp mark 7}{(6.373,1.315)}
\gppoint{gp mark 7}{(6.378,1.315)}
\gppoint{gp mark 7}{(6.384,1.315)}
\gppoint{gp mark 7}{(6.389,1.315)}
\gppoint{gp mark 7}{(6.395,1.315)}
\gppoint{gp mark 7}{(6.400,1.315)}
\gppoint{gp mark 7}{(6.406,1.315)}
\gppoint{gp mark 7}{(6.411,1.315)}
\gppoint{gp mark 7}{(6.417,1.315)}
\gppoint{gp mark 7}{(6.422,1.315)}
\gppoint{gp mark 7}{(6.428,1.315)}
\gppoint{gp mark 7}{(6.433,1.315)}
\gppoint{gp mark 7}{(6.439,1.315)}
\gppoint{gp mark 7}{(6.444,1.315)}
\gppoint{gp mark 7}{(6.450,1.315)}
\gppoint{gp mark 7}{(6.455,1.315)}
\gppoint{gp mark 7}{(6.461,1.315)}
\gppoint{gp mark 7}{(6.466,1.315)}
\gppoint{gp mark 7}{(6.472,1.315)}
\gppoint{gp mark 7}{(6.477,1.315)}
\gppoint{gp mark 7}{(6.483,1.315)}
\gppoint{gp mark 7}{(6.488,1.315)}
\gppoint{gp mark 7}{(6.494,1.315)}
\gppoint{gp mark 7}{(6.499,1.315)}
\gppoint{gp mark 7}{(6.505,1.315)}
\gppoint{gp mark 7}{(6.510,1.315)}
\gppoint{gp mark 7}{(6.516,1.315)}
\gppoint{gp mark 7}{(6.521,1.315)}
\gppoint{gp mark 7}{(6.527,1.315)}
\gppoint{gp mark 7}{(6.532,1.315)}
\gppoint{gp mark 7}{(6.538,1.315)}
\gppoint{gp mark 7}{(6.543,1.315)}
\gppoint{gp mark 7}{(6.549,1.315)}
\gppoint{gp mark 7}{(6.554,1.315)}
\gppoint{gp mark 7}{(6.560,1.315)}
\gppoint{gp mark 7}{(6.565,1.315)}
\gppoint{gp mark 7}{(6.571,1.315)}
\gppoint{gp mark 7}{(6.576,1.315)}
\gppoint{gp mark 7}{(6.582,1.315)}
\gppoint{gp mark 7}{(6.587,1.315)}
\gppoint{gp mark 7}{(6.593,1.315)}
\gppoint{gp mark 7}{(6.598,1.315)}
\gppoint{gp mark 7}{(6.604,1.315)}
\gppoint{gp mark 7}{(6.609,1.315)}
\gppoint{gp mark 7}{(6.615,1.315)}
\gppoint{gp mark 7}{(6.620,1.315)}
\gppoint{gp mark 7}{(6.626,1.315)}
\gppoint{gp mark 7}{(6.631,1.315)}
\gppoint{gp mark 7}{(6.637,1.315)}
\gppoint{gp mark 7}{(6.642,1.315)}
\gppoint{gp mark 7}{(6.648,1.315)}
\gppoint{gp mark 7}{(6.653,1.315)}
\gppoint{gp mark 7}{(6.659,1.315)}
\gppoint{gp mark 7}{(6.664,1.315)}
\gppoint{gp mark 7}{(6.670,1.315)}
\gppoint{gp mark 7}{(6.675,1.315)}
\gppoint{gp mark 7}{(6.681,1.315)}
\gppoint{gp mark 7}{(6.686,1.315)}
\gppoint{gp mark 7}{(6.692,1.315)}
\gppoint{gp mark 7}{(6.697,1.315)}
\gppoint{gp mark 7}{(6.703,1.315)}
\gppoint{gp mark 7}{(6.708,1.315)}
\gppoint{gp mark 7}{(6.714,1.315)}
\gppoint{gp mark 7}{(6.719,1.315)}
\gppoint{gp mark 7}{(6.725,1.315)}
\gppoint{gp mark 7}{(6.730,1.315)}
\gppoint{gp mark 7}{(6.736,1.315)}
\gppoint{gp mark 7}{(6.741,1.315)}
\gppoint{gp mark 7}{(6.747,1.315)}
\gppoint{gp mark 7}{(6.752,1.315)}
\gppoint{gp mark 7}{(6.758,1.315)}
\gppoint{gp mark 7}{(6.763,1.315)}
\gppoint{gp mark 7}{(6.769,1.315)}
\gppoint{gp mark 7}{(6.774,1.315)}
\gppoint{gp mark 7}{(6.780,1.315)}
\gppoint{gp mark 7}{(6.785,1.315)}
\gppoint{gp mark 7}{(6.791,1.315)}
\gppoint{gp mark 7}{(6.796,1.315)}
\gppoint{gp mark 7}{(6.802,1.315)}
\gppoint{gp mark 7}{(6.807,1.315)}
\gppoint{gp mark 7}{(6.812,1.315)}
\gppoint{gp mark 7}{(6.818,1.315)}
\gppoint{gp mark 7}{(6.823,1.315)}
\gppoint{gp mark 7}{(6.829,1.315)}
\gppoint{gp mark 7}{(6.834,1.315)}
\gppoint{gp mark 7}{(6.840,1.315)}
\gppoint{gp mark 7}{(6.845,1.315)}
\gppoint{gp mark 7}{(6.851,1.315)}
\gppoint{gp mark 7}{(6.856,1.315)}
\gppoint{gp mark 7}{(6.862,1.315)}
\gppoint{gp mark 7}{(6.867,1.315)}
\gppoint{gp mark 7}{(6.873,1.315)}
\gppoint{gp mark 7}{(6.878,1.315)}
\gppoint{gp mark 7}{(6.884,1.315)}
\gppoint{gp mark 7}{(6.889,1.315)}
\gppoint{gp mark 7}{(6.895,1.315)}
\gppoint{gp mark 7}{(6.900,1.315)}
\gppoint{gp mark 7}{(6.906,1.315)}
\gppoint{gp mark 7}{(6.911,1.315)}
\gppoint{gp mark 7}{(6.917,1.315)}
\gppoint{gp mark 7}{(6.922,1.315)}
\gppoint{gp mark 7}{(6.928,1.315)}
\gppoint{gp mark 7}{(6.933,1.315)}
\gppoint{gp mark 7}{(6.939,1.315)}
\gppoint{gp mark 7}{(6.944,1.315)}
\gppoint{gp mark 7}{(6.950,1.315)}
\gppoint{gp mark 7}{(6.955,1.315)}
\gppoint{gp mark 7}{(6.961,1.315)}
\gppoint{gp mark 7}{(6.966,1.315)}
\gppoint{gp mark 7}{(6.972,1.315)}
\gppoint{gp mark 7}{(6.977,1.315)}
\gppoint{gp mark 7}{(6.983,1.315)}
\gppoint{gp mark 7}{(6.988,1.315)}
\gppoint{gp mark 7}{(6.994,1.315)}
\gppoint{gp mark 7}{(6.999,1.315)}
\gppoint{gp mark 7}{(7.005,1.315)}
\gppoint{gp mark 7}{(7.010,1.315)}
\gppoint{gp mark 7}{(7.016,1.315)}
\gppoint{gp mark 7}{(7.021,1.315)}
\gppoint{gp mark 7}{(7.027,1.315)}
\gppoint{gp mark 7}{(7.032,1.315)}
\gppoint{gp mark 7}{(7.038,1.315)}
\gppoint{gp mark 7}{(7.043,1.315)}
\gppoint{gp mark 7}{(7.049,1.315)}
\gppoint{gp mark 7}{(7.054,1.315)}
\gppoint{gp mark 7}{(7.060,1.315)}
\gppoint{gp mark 7}{(7.065,1.315)}
\gppoint{gp mark 7}{(7.071,1.315)}
\gppoint{gp mark 7}{(7.076,1.315)}
\gppoint{gp mark 7}{(7.082,1.315)}
\gppoint{gp mark 7}{(7.087,1.315)}
\gppoint{gp mark 7}{(7.093,1.315)}
\gppoint{gp mark 7}{(7.098,1.315)}
\gppoint{gp mark 7}{(7.104,1.315)}
\gppoint{gp mark 7}{(7.109,1.315)}
\gppoint{gp mark 7}{(7.115,1.315)}
\gppoint{gp mark 7}{(7.120,1.315)}
\gppoint{gp mark 7}{(7.126,1.315)}
\gppoint{gp mark 7}{(7.131,1.315)}
\gppoint{gp mark 7}{(7.137,1.315)}
\gppoint{gp mark 7}{(7.142,1.315)}
\gppoint{gp mark 7}{(7.148,1.315)}
\gppoint{gp mark 7}{(7.153,1.315)}
\gppoint{gp mark 7}{(7.159,1.315)}
\gppoint{gp mark 7}{(7.164,1.315)}
\gppoint{gp mark 7}{(7.170,1.315)}
\gppoint{gp mark 7}{(7.175,1.315)}
\gppoint{gp mark 7}{(7.181,1.315)}
\gppoint{gp mark 7}{(7.186,1.315)}
\gppoint{gp mark 7}{(7.192,1.315)}
\gppoint{gp mark 7}{(7.197,1.315)}
\gppoint{gp mark 7}{(7.203,1.315)}
\gppoint{gp mark 7}{(7.208,1.315)}
\gppoint{gp mark 7}{(7.214,1.315)}
\gppoint{gp mark 7}{(7.219,1.315)}
\gppoint{gp mark 7}{(7.225,1.315)}
\gppoint{gp mark 7}{(7.230,1.315)}
\gppoint{gp mark 7}{(7.236,1.315)}
\gppoint{gp mark 7}{(7.241,1.315)}
\gppoint{gp mark 7}{(7.247,1.315)}
\gppoint{gp mark 7}{(7.252,1.315)}
\gppoint{gp mark 7}{(7.258,1.315)}
\gppoint{gp mark 7}{(7.263,1.315)}
\gppoint{gp mark 7}{(7.268,1.315)}
\gppoint{gp mark 7}{(7.274,1.315)}
\gppoint{gp mark 7}{(7.279,1.315)}
\gppoint{gp mark 7}{(7.285,1.315)}
\gppoint{gp mark 7}{(7.290,1.315)}
\gppoint{gp mark 7}{(7.296,1.315)}
\gppoint{gp mark 7}{(7.301,1.315)}
\gppoint{gp mark 7}{(7.307,1.315)}
\gppoint{gp mark 7}{(7.312,1.315)}
\gppoint{gp mark 7}{(7.318,1.315)}
\gppoint{gp mark 7}{(7.323,1.315)}
\gppoint{gp mark 7}{(7.329,1.315)}
\gppoint{gp mark 7}{(7.334,1.315)}
\gppoint{gp mark 7}{(7.340,1.315)}
\gppoint{gp mark 7}{(7.345,1.315)}
\gppoint{gp mark 7}{(7.351,1.315)}
\gppoint{gp mark 7}{(7.356,1.315)}
\gppoint{gp mark 7}{(7.362,1.315)}
\gppoint{gp mark 7}{(7.367,1.315)}
\gppoint{gp mark 7}{(7.373,1.315)}
\gppoint{gp mark 7}{(7.378,1.315)}
\gppoint{gp mark 7}{(7.384,1.315)}
\gppoint{gp mark 7}{(7.389,1.315)}
\gppoint{gp mark 7}{(7.395,1.315)}
\gppoint{gp mark 7}{(7.400,1.315)}
\gppoint{gp mark 7}{(7.406,1.315)}
\gppoint{gp mark 7}{(7.411,1.315)}
\gppoint{gp mark 7}{(7.417,1.315)}
\gppoint{gp mark 7}{(7.422,1.315)}
\gppoint{gp mark 7}{(7.428,1.315)}
\gppoint{gp mark 7}{(7.433,1.315)}
\gppoint{gp mark 7}{(7.439,1.315)}
\gppoint{gp mark 7}{(7.444,1.315)}
\gppoint{gp mark 7}{(7.450,1.315)}
\gppoint{gp mark 7}{(7.455,1.315)}
\gppoint{gp mark 7}{(7.461,1.315)}
\gppoint{gp mark 7}{(7.466,1.315)}
\gppoint{gp mark 7}{(7.472,1.315)}
\gppoint{gp mark 7}{(7.477,1.315)}
\gppoint{gp mark 7}{(7.483,1.315)}
\gppoint{gp mark 7}{(7.488,1.315)}
\gppoint{gp mark 7}{(7.494,1.315)}
\gppoint{gp mark 7}{(7.499,1.315)}
\gppoint{gp mark 7}{(7.505,1.315)}
\gppoint{gp mark 7}{(7.510,1.315)}
\gppoint{gp mark 7}{(7.516,1.315)}
\gppoint{gp mark 7}{(7.521,1.315)}
\gppoint{gp mark 7}{(7.527,1.315)}
\gppoint{gp mark 7}{(7.532,1.315)}
\gppoint{gp mark 7}{(7.538,1.315)}
\gppoint{gp mark 7}{(7.543,1.315)}
\gppoint{gp mark 7}{(7.549,1.315)}
\gppoint{gp mark 7}{(7.554,1.315)}
\gppoint{gp mark 7}{(7.560,1.315)}
\gppoint{gp mark 7}{(7.565,1.315)}
\gppoint{gp mark 7}{(7.571,1.315)}
\gppoint{gp mark 7}{(7.576,1.315)}
\gppoint{gp mark 7}{(7.582,1.315)}
\gppoint{gp mark 7}{(7.587,1.315)}
\gppoint{gp mark 7}{(7.593,1.315)}
\gppoint{gp mark 7}{(7.598,1.315)}
\gppoint{gp mark 7}{(7.604,1.315)}
\gppoint{gp mark 7}{(7.609,1.315)}
\gppoint{gp mark 7}{(7.615,1.315)}
\gppoint{gp mark 7}{(7.620,1.315)}
\gppoint{gp mark 7}{(7.626,1.315)}
\gppoint{gp mark 7}{(7.631,1.315)}
\gppoint{gp mark 7}{(7.637,1.315)}
\gppoint{gp mark 7}{(7.642,1.315)}
\gppoint{gp mark 7}{(7.648,1.315)}
\gppoint{gp mark 7}{(7.653,1.315)}
\gppoint{gp mark 7}{(7.659,1.315)}
\gppoint{gp mark 7}{(7.664,1.315)}
\gppoint{gp mark 7}{(7.670,1.315)}
\gppoint{gp mark 7}{(7.675,1.315)}
\gppoint{gp mark 7}{(7.681,1.315)}
\gppoint{gp mark 7}{(7.686,1.315)}
\gppoint{gp mark 7}{(7.692,1.315)}
\gppoint{gp mark 7}{(7.697,1.315)}
\gppoint{gp mark 7}{(7.703,1.315)}
\gppoint{gp mark 7}{(7.708,1.315)}
\gppoint{gp mark 7}{(7.714,1.315)}
\gppoint{gp mark 7}{(7.719,1.315)}
\gppoint{gp mark 7}{(7.724,1.315)}
\gppoint{gp mark 7}{(7.730,1.315)}
\gppoint{gp mark 7}{(7.735,1.315)}
\gppoint{gp mark 7}{(7.741,1.315)}
\gppoint{gp mark 7}{(7.746,1.315)}
\gppoint{gp mark 7}{(7.752,1.315)}
\gppoint{gp mark 7}{(7.757,1.315)}
\gppoint{gp mark 7}{(7.763,1.315)}
\gppoint{gp mark 7}{(7.768,1.315)}
\gppoint{gp mark 7}{(7.774,1.315)}
\gppoint{gp mark 7}{(7.779,1.315)}
\gppoint{gp mark 7}{(7.785,1.315)}
\gppoint{gp mark 7}{(7.790,1.315)}
\gppoint{gp mark 7}{(7.796,1.315)}
\gppoint{gp mark 7}{(7.801,1.315)}
\gppoint{gp mark 7}{(7.807,1.315)}
\gppoint{gp mark 7}{(7.812,1.315)}
\gppoint{gp mark 7}{(7.818,1.315)}
\gppoint{gp mark 7}{(7.823,1.315)}
\gppoint{gp mark 7}{(7.829,1.315)}
\gppoint{gp mark 7}{(7.834,1.315)}
\gppoint{gp mark 7}{(7.840,1.315)}
\gppoint{gp mark 7}{(7.845,1.315)}
\gppoint{gp mark 7}{(7.851,1.315)}
\gppoint{gp mark 7}{(7.856,1.315)}
\gppoint{gp mark 7}{(7.862,1.315)}
\gppoint{gp mark 7}{(7.867,1.315)}
\gppoint{gp mark 7}{(7.873,1.315)}
\gppoint{gp mark 7}{(7.878,1.315)}
\gppoint{gp mark 7}{(7.884,1.315)}
\gppoint{gp mark 7}{(7.889,1.315)}
\gppoint{gp mark 7}{(7.895,1.315)}
\gppoint{gp mark 7}{(7.900,1.315)}
\gppoint{gp mark 7}{(7.906,1.315)}
\gppoint{gp mark 7}{(7.911,1.315)}
\gppoint{gp mark 7}{(7.917,1.315)}
\gppoint{gp mark 7}{(7.922,1.315)}
\gppoint{gp mark 7}{(7.928,1.315)}
\gppoint{gp mark 7}{(7.933,1.315)}
\gppoint{gp mark 7}{(7.939,1.315)}
\gppoint{gp mark 7}{(7.944,1.315)}
\gpcolor{rgb color={0.000,0.000,0.000}}
\gpsetlinetype{gp lt plot 0}
\gpsetlinewidth{4.00}
\draw[gp path] (2.411,4.028)--(3.526,4.028);
\draw[gp path] (3.526,1.086)--(4.329,1.086);
\draw[gp path] (4.329,1.315)--(7.511,1.315);
\draw[gp path] (7.511,1.315)--(7.947,1.315);
\draw[gp path] (1.202,4.617)--(1.208,4.614)--(1.214,4.611)--(1.220,4.608)--(1.226,4.605)%
  --(1.232,4.602)--(1.238,4.599)--(1.244,4.596)--(1.250,4.593)--(1.256,4.591)--(1.262,4.588)%
  --(1.268,4.585)--(1.275,4.582)--(1.281,4.579)--(1.287,4.576)--(1.293,4.573)--(1.299,4.570)%
  --(1.305,4.567)--(1.311,4.564)--(1.317,4.561)--(1.323,4.558)--(1.329,4.555)--(1.335,4.552)%
  --(1.341,4.549)--(1.347,4.546)--(1.353,4.543)--(1.359,4.540)--(1.365,4.538)--(1.371,4.535)%
  --(1.377,4.532)--(1.383,4.529)--(1.389,4.526)--(1.395,4.523)--(1.402,4.520)--(1.408,4.517)%
  --(1.414,4.514)--(1.420,4.511)--(1.426,4.508)--(1.432,4.505)--(1.438,4.502)--(1.444,4.499)%
  --(1.450,4.496)--(1.456,4.493)--(1.462,4.490)--(1.468,4.488)--(1.474,4.485)--(1.480,4.482)%
  --(1.486,4.479)--(1.492,4.476)--(1.498,4.473)--(1.504,4.470)--(1.510,4.467)--(1.516,4.464)%
  --(1.522,4.461)--(1.528,4.458)--(1.535,4.455)--(1.541,4.452)--(1.547,4.449)--(1.553,4.446)%
  --(1.559,4.443)--(1.565,4.440)--(1.571,4.437)--(1.577,4.435)--(1.583,4.432)--(1.589,4.429)%
  --(1.595,4.426)--(1.601,4.423)--(1.607,4.420)--(1.613,4.417)--(1.619,4.414)--(1.625,4.411)%
  --(1.631,4.408)--(1.637,4.405)--(1.643,4.402)--(1.649,4.399)--(1.655,4.396)--(1.662,4.393)%
  --(1.668,4.390)--(1.674,4.387)--(1.680,4.384)--(1.686,4.382)--(1.692,4.379)--(1.698,4.376)%
  --(1.704,4.373)--(1.710,4.370)--(1.716,4.367)--(1.722,4.364)--(1.728,4.361)--(1.734,4.358)%
  --(1.740,4.355)--(1.746,4.352)--(1.752,4.349)--(1.758,4.346)--(1.764,4.343)--(1.770,4.340)%
  --(1.776,4.337)--(1.782,4.334)--(1.788,4.331)--(1.795,4.329)--(1.801,4.326)--(1.807,4.323)%
  --(1.813,4.320)--(1.819,4.317)--(1.825,4.314)--(1.831,4.311)--(1.837,4.308)--(1.843,4.305)%
  --(1.849,4.302)--(1.855,4.299)--(1.861,4.296)--(1.867,4.293)--(1.873,4.290)--(1.879,4.287)%
  --(1.885,4.284)--(1.891,4.281)--(1.897,4.278)--(1.903,4.276)--(1.909,4.273)--(1.915,4.270)%
  --(1.922,4.267)--(1.928,4.264)--(1.934,4.261)--(1.940,4.258)--(1.946,4.255)--(1.952,4.252)%
  --(1.958,4.249)--(1.964,4.246)--(1.970,4.243)--(1.976,4.240)--(1.982,4.237)--(1.988,4.234)%
  --(1.994,4.231)--(2.000,4.228)--(2.006,4.225)--(2.012,4.223)--(2.018,4.220)--(2.024,4.217)%
  --(2.030,4.214)--(2.036,4.211)--(2.042,4.208)--(2.048,4.205)--(2.055,4.202)--(2.061,4.199)%
  --(2.067,4.196)--(2.073,4.193)--(2.079,4.190)--(2.085,4.187)--(2.091,4.184)--(2.097,4.181)%
  --(2.103,4.178)--(2.109,4.175)--(2.115,4.172)--(2.121,4.170)--(2.127,4.167)--(2.133,4.164)%
  --(2.139,4.161)--(2.145,4.158)--(2.151,4.155)--(2.157,4.152)--(2.163,4.149)--(2.169,4.146)%
  --(2.175,4.143)--(2.182,4.140)--(2.188,4.137)--(2.194,4.134)--(2.200,4.131)--(2.206,4.128)%
  --(2.212,4.125)--(2.218,4.122)--(2.224,4.119)--(2.230,4.117)--(2.236,4.114)--(2.242,4.111)%
  --(2.248,4.108)--(2.254,4.105)--(2.260,4.102)--(2.266,4.099)--(2.272,4.096)--(2.278,4.093)%
  --(2.284,4.090)--(2.290,4.087)--(2.296,4.084)--(2.302,4.081)--(2.308,4.078)--(2.315,4.075)%
  --(2.321,4.072)--(2.327,4.069)--(2.333,4.067)--(2.339,4.064)--(2.345,4.061)--(2.351,4.058)%
  --(2.357,4.055)--(2.363,4.052)--(2.369,4.049)--(2.375,4.046)--(2.381,4.043)--(2.387,4.040)%
  --(2.393,4.037)--(2.399,4.034)--(2.405,4.031)--(2.411,4.028);
\draw[gp path] (3.526,4.028)--(3.526,1.086);
\draw[gp path] (4.329,1.086)--(4.329,1.315);
\draw[gp path] (4.572,3.695)--(5.247,3.695);
\gpcolor{rgb color={1.000,0.000,0.000}}
\gpsetlinewidth{0.50}
\gppoint{gp mark 7}{(4.909,2.921)}
\gpcolor{rgb color={0.000,0.000,0.000}}
\node[gp node left,font={\fontsize{10pt}{12pt}\selectfont}] at (1.421,5.166) {\LARGE $B_y$};
\node[gp node left,font={\fontsize{10pt}{12pt}\selectfont}] at (6.147,5.166) {\large $\alpha = 3.0$};
\node[gp node left,font={\fontsize{10pt}{12pt}\selectfont}] at (5.472,3.695) {\large exact};
\node[gp node left,font={\fontsize{10pt}{12pt}\selectfont}] at (5.472,2.921) {\large converging};
%% coordinates of the plot area
\gpdefrectangularnode{gp plot 1}{\pgfpoint{1.196cm}{0.985cm}}{\pgfpoint{7.947cm}{5.631cm}}
\end{tikzpicture}
%% gnuplot variables
}
\end{tabular}
\caption{The approximate solution after the first flux correction of HLLD-CWM and exact r-solution to the full Riemann problem for the near-coplanar case with $4096$ grid points.  The compound wave is almost completely removed, except near $x=0.303$ where a weak intermediate shock remains.}
\label{fig:coplanar_b_rsol_init}
\end{figure}

%-----------------------------------------------------------------
% Fast coplanar initial r-solution
%-----------------------------------------------------------------
\begin{figure}[htbp] 
\begin{tabular}{cc}
\resizebox{0.5\linewidth}{!}{\tikzsetnextfilename{fast_coplanar_a_rsol_init_1}\begin{tikzpicture}[gnuplot]
%% generated with GNUPLOT 4.6p4 (Lua 5.1; terminal rev. 99, script rev. 100)
%% Sat 02 Aug 2014 10:07:54 AM EDT
\path (0.000,0.000) rectangle (8.500,6.000);
\gpfill{rgb color={1.000,1.000,1.000}} (1.196,0.985)--(7.946,0.985)--(7.946,5.630)--(1.196,5.630)--cycle;
\gpcolor{color=gp lt color border}
\gpsetlinetype{gp lt border}
\gpsetlinewidth{1.00}
\draw[gp path] (1.196,0.985)--(1.196,5.630)--(7.946,5.630)--(7.946,0.985)--cycle;
\gpcolor{color=gp lt color axes}
\gpsetlinetype{gp lt axes}
\gpsetlinewidth{2.00}
\draw[gp path] (1.196,1.275)--(7.947,1.275);
\gpcolor{color=gp lt color border}
\gpsetlinetype{gp lt border}
\draw[gp path] (1.196,1.275)--(1.268,1.275);
\draw[gp path] (7.947,1.275)--(7.875,1.275);
\gpcolor{rgb color={0.000,0.000,0.000}}
\node[gp node right,font={\fontsize{10pt}{12pt}\selectfont}] at (1.012,1.275) {0.65};
\gpcolor{color=gp lt color axes}
\gpsetlinetype{gp lt axes}
\draw[gp path] (1.196,2.001)--(7.947,2.001);
\gpcolor{color=gp lt color border}
\gpsetlinetype{gp lt border}
\draw[gp path] (1.196,2.001)--(1.268,2.001);
\draw[gp path] (7.947,2.001)--(7.875,2.001);
\gpcolor{rgb color={0.000,0.000,0.000}}
\node[gp node right,font={\fontsize{10pt}{12pt}\selectfont}] at (1.012,2.001) {0.7};
\gpcolor{color=gp lt color axes}
\gpsetlinetype{gp lt axes}
\draw[gp path] (1.196,2.727)--(7.947,2.727);
\gpcolor{color=gp lt color border}
\gpsetlinetype{gp lt border}
\draw[gp path] (1.196,2.727)--(1.268,2.727);
\draw[gp path] (7.947,2.727)--(7.875,2.727);
\gpcolor{rgb color={0.000,0.000,0.000}}
\node[gp node right,font={\fontsize{10pt}{12pt}\selectfont}] at (1.012,2.727) {0.75};
\gpcolor{color=gp lt color axes}
\gpsetlinetype{gp lt axes}
\draw[gp path] (1.196,3.453)--(7.947,3.453);
\gpcolor{color=gp lt color border}
\gpsetlinetype{gp lt border}
\draw[gp path] (1.196,3.453)--(1.268,3.453);
\draw[gp path] (7.947,3.453)--(7.875,3.453);
\gpcolor{rgb color={0.000,0.000,0.000}}
\node[gp node right,font={\fontsize{10pt}{12pt}\selectfont}] at (1.012,3.453) {0.8};
\gpcolor{color=gp lt color axes}
\gpsetlinetype{gp lt axes}
\draw[gp path] (1.196,4.179)--(7.947,4.179);
\gpcolor{color=gp lt color border}
\gpsetlinetype{gp lt border}
\draw[gp path] (1.196,4.179)--(1.268,4.179);
\draw[gp path] (7.947,4.179)--(7.875,4.179);
\gpcolor{rgb color={0.000,0.000,0.000}}
\node[gp node right,font={\fontsize{10pt}{12pt}\selectfont}] at (1.012,4.179) {0.85};
\gpcolor{color=gp lt color axes}
\gpsetlinetype{gp lt axes}
\draw[gp path] (1.196,4.905)--(7.947,4.905);
\gpcolor{color=gp lt color border}
\gpsetlinetype{gp lt border}
\draw[gp path] (1.196,4.905)--(1.268,4.905);
\draw[gp path] (7.947,4.905)--(7.875,4.905);
\gpcolor{rgb color={0.000,0.000,0.000}}
\node[gp node right,font={\fontsize{10pt}{12pt}\selectfont}] at (1.012,4.905) {0.9};
\gpcolor{color=gp lt color axes}
\gpsetlinetype{gp lt axes}
\draw[gp path] (1.196,5.631)--(7.947,5.631);
\gpcolor{color=gp lt color border}
\gpsetlinetype{gp lt border}
\draw[gp path] (1.196,5.631)--(1.268,5.631);
\draw[gp path] (7.947,5.631)--(7.875,5.631);
\gpcolor{rgb color={0.000,0.000,0.000}}
\node[gp node right,font={\fontsize{10pt}{12pt}\selectfont}] at (1.012,5.631) {0.95};
\gpcolor{color=gp lt color axes}
\gpsetlinetype{gp lt axes}
\draw[gp path] (1.196,0.985)--(1.196,5.631);
\gpcolor{color=gp lt color border}
\gpsetlinetype{gp lt border}
\draw[gp path] (1.196,0.985)--(1.196,1.057);
\draw[gp path] (1.196,5.631)--(1.196,5.559);
\gpcolor{rgb color={0.000,0.000,0.000}}
\node[gp node center,font={\fontsize{10pt}{12pt}\selectfont}] at (1.196,0.677) {0.3};
\gpcolor{color=gp lt color axes}
\gpsetlinetype{gp lt axes}
\draw[gp path] (2.494,0.985)--(2.494,5.631);
\gpcolor{color=gp lt color border}
\gpsetlinetype{gp lt border}
\draw[gp path] (2.494,0.985)--(2.494,1.057);
\draw[gp path] (2.494,5.631)--(2.494,5.559);
\gpcolor{rgb color={0.000,0.000,0.000}}
\node[gp node center,font={\fontsize{10pt}{12pt}\selectfont}] at (2.494,0.677) {0.35};
\gpcolor{color=gp lt color axes}
\gpsetlinetype{gp lt axes}
\draw[gp path] (3.793,0.985)--(3.793,5.631);
\gpcolor{color=gp lt color border}
\gpsetlinetype{gp lt border}
\draw[gp path] (3.793,0.985)--(3.793,1.057);
\draw[gp path] (3.793,5.631)--(3.793,5.559);
\gpcolor{rgb color={0.000,0.000,0.000}}
\node[gp node center,font={\fontsize{10pt}{12pt}\selectfont}] at (3.793,0.677) {0.4};
\gpcolor{color=gp lt color axes}
\gpsetlinetype{gp lt axes}
\draw[gp path] (5.091,0.985)--(5.091,5.631);
\gpcolor{color=gp lt color border}
\gpsetlinetype{gp lt border}
\draw[gp path] (5.091,0.985)--(5.091,1.057);
\draw[gp path] (5.091,5.631)--(5.091,5.559);
\gpcolor{rgb color={0.000,0.000,0.000}}
\node[gp node center,font={\fontsize{10pt}{12pt}\selectfont}] at (5.091,0.677) {0.45};
\gpcolor{color=gp lt color axes}
\gpsetlinetype{gp lt axes}
\draw[gp path] (6.389,0.985)--(6.389,5.631);
\gpcolor{color=gp lt color border}
\gpsetlinetype{gp lt border}
\draw[gp path] (6.389,0.985)--(6.389,1.057);
\draw[gp path] (6.389,5.631)--(6.389,5.559);
\gpcolor{rgb color={0.000,0.000,0.000}}
\node[gp node center,font={\fontsize{10pt}{12pt}\selectfont}] at (6.389,0.677) {0.5};
\gpcolor{color=gp lt color axes}
\gpsetlinetype{gp lt axes}
\draw[gp path] (7.687,0.985)--(7.687,5.631);
\gpcolor{color=gp lt color border}
\gpsetlinetype{gp lt border}
\draw[gp path] (7.687,0.985)--(7.687,1.057);
\draw[gp path] (7.687,5.631)--(7.687,5.559);
\gpcolor{rgb color={0.000,0.000,0.000}}
\node[gp node center,font={\fontsize{10pt}{12pt}\selectfont}] at (7.687,0.677) {0.55};
\gpcolor{color=gp lt color border}
\draw[gp path] (1.196,5.631)--(1.196,0.985)--(7.947,0.985)--(7.947,5.631)--cycle;
\gpcolor{rgb color={0.000,0.000,0.000}}
\node[gp node center,font={\fontsize{10pt}{12pt}\selectfont}] at (4.571,0.215) {\large $x$};
\gpcolor{rgb color={1.000,0.000,0.000}}
\gpsetlinewidth{0.50}
\gpsetpointsize{4.44}
\gppoint{gp mark 7}{(1.210,3.636)}
\gppoint{gp mark 7}{(1.223,3.620)}
\gppoint{gp mark 7}{(1.235,3.605)}
\gppoint{gp mark 7}{(1.248,3.590)}
\gppoint{gp mark 7}{(1.261,3.574)}
\gppoint{gp mark 7}{(1.273,3.559)}
\gppoint{gp mark 7}{(1.286,3.543)}
\gppoint{gp mark 7}{(1.299,3.528)}
\gppoint{gp mark 7}{(1.311,3.513)}
\gppoint{gp mark 7}{(1.324,3.497)}
\gppoint{gp mark 7}{(1.337,3.482)}
\gppoint{gp mark 7}{(1.349,3.467)}
\gppoint{gp mark 7}{(1.362,3.451)}
\gppoint{gp mark 7}{(1.375,3.436)}
\gppoint{gp mark 7}{(1.387,3.421)}
\gppoint{gp mark 7}{(1.400,3.405)}
\gppoint{gp mark 7}{(1.413,3.390)}
\gppoint{gp mark 7}{(1.425,3.375)}
\gppoint{gp mark 7}{(1.438,3.360)}
\gppoint{gp mark 7}{(1.451,3.344)}
\gppoint{gp mark 7}{(1.464,3.329)}
\gppoint{gp mark 7}{(1.476,3.314)}
\gppoint{gp mark 7}{(1.489,3.299)}
\gppoint{gp mark 7}{(1.502,3.284)}
\gppoint{gp mark 7}{(1.514,3.269)}
\gppoint{gp mark 7}{(1.527,3.253)}
\gppoint{gp mark 7}{(1.540,3.238)}
\gppoint{gp mark 7}{(1.552,3.223)}
\gppoint{gp mark 7}{(1.565,3.208)}
\gppoint{gp mark 7}{(1.578,3.193)}
\gppoint{gp mark 7}{(1.590,3.178)}
\gppoint{gp mark 7}{(1.603,3.163)}
\gppoint{gp mark 7}{(1.616,3.148)}
\gppoint{gp mark 7}{(1.628,3.133)}
\gppoint{gp mark 7}{(1.641,3.118)}
\gppoint{gp mark 7}{(1.654,3.103)}
\gppoint{gp mark 7}{(1.666,3.088)}
\gppoint{gp mark 7}{(1.679,3.073)}
\gppoint{gp mark 7}{(1.692,3.058)}
\gppoint{gp mark 7}{(1.704,3.043)}
\gppoint{gp mark 7}{(1.717,3.028)}
\gppoint{gp mark 7}{(1.730,3.013)}
\gppoint{gp mark 7}{(1.742,2.998)}
\gppoint{gp mark 7}{(1.755,2.983)}
\gppoint{gp mark 7}{(1.768,2.968)}
\gppoint{gp mark 7}{(1.780,2.953)}
\gppoint{gp mark 7}{(1.793,2.939)}
\gppoint{gp mark 7}{(1.806,2.924)}
\gppoint{gp mark 7}{(1.819,2.909)}
\gppoint{gp mark 7}{(1.831,2.894)}
\gppoint{gp mark 7}{(1.844,2.879)}
\gppoint{gp mark 7}{(1.857,2.865)}
\gppoint{gp mark 7}{(1.869,2.850)}
\gppoint{gp mark 7}{(1.882,2.835)}
\gppoint{gp mark 7}{(1.895,2.820)}
\gppoint{gp mark 7}{(1.907,2.806)}
\gppoint{gp mark 7}{(1.920,2.791)}
\gppoint{gp mark 7}{(1.933,2.776)}
\gppoint{gp mark 7}{(1.945,2.762)}
\gppoint{gp mark 7}{(1.958,2.747)}
\gppoint{gp mark 7}{(1.971,2.732)}
\gppoint{gp mark 7}{(1.983,2.718)}
\gppoint{gp mark 7}{(1.996,2.703)}
\gppoint{gp mark 7}{(2.009,2.688)}
\gppoint{gp mark 7}{(2.021,2.674)}
\gppoint{gp mark 7}{(2.034,2.659)}
\gppoint{gp mark 7}{(2.047,2.645)}
\gppoint{gp mark 7}{(2.059,2.630)}
\gppoint{gp mark 7}{(2.072,2.616)}
\gppoint{gp mark 7}{(2.085,2.601)}
\gppoint{gp mark 7}{(2.097,2.587)}
\gppoint{gp mark 7}{(2.110,2.572)}
\gppoint{gp mark 7}{(2.123,2.558)}
\gppoint{gp mark 7}{(2.135,2.544)}
\gppoint{gp mark 7}{(2.148,2.529)}
\gppoint{gp mark 7}{(2.161,2.515)}
\gppoint{gp mark 7}{(2.174,2.500)}
\gppoint{gp mark 7}{(2.186,2.486)}
\gppoint{gp mark 7}{(2.199,2.472)}
\gppoint{gp mark 7}{(2.212,2.457)}
\gppoint{gp mark 7}{(2.224,2.443)}
\gppoint{gp mark 7}{(2.237,2.429)}
\gppoint{gp mark 7}{(2.250,2.415)}
\gppoint{gp mark 7}{(2.262,2.401)}
\gppoint{gp mark 7}{(2.275,2.386)}
\gppoint{gp mark 7}{(2.288,2.372)}
\gppoint{gp mark 7}{(2.300,2.358)}
\gppoint{gp mark 7}{(2.313,2.344)}
\gppoint{gp mark 7}{(2.326,2.330)}
\gppoint{gp mark 7}{(2.338,2.316)}
\gppoint{gp mark 7}{(2.351,2.302)}
\gppoint{gp mark 7}{(2.364,2.288)}
\gppoint{gp mark 7}{(2.376,2.275)}
\gppoint{gp mark 7}{(2.389,2.261)}
\gppoint{gp mark 7}{(2.402,2.247)}
\gppoint{gp mark 7}{(2.414,2.234)}
\gppoint{gp mark 7}{(2.427,2.221)}
\gppoint{gp mark 7}{(2.440,2.209)}
\gppoint{gp mark 7}{(2.452,2.197)}
\gppoint{gp mark 7}{(2.465,2.188)}
\gppoint{gp mark 7}{(2.478,2.180)}
\gppoint{gp mark 7}{(2.490,2.176)}
\gppoint{gp mark 7}{(2.503,2.174)}
\gppoint{gp mark 7}{(2.516,2.174)}
\gppoint{gp mark 7}{(2.529,2.174)}
\gppoint{gp mark 7}{(2.541,2.174)}
\gppoint{gp mark 7}{(2.554,2.174)}
\gppoint{gp mark 7}{(2.567,2.175)}
\gppoint{gp mark 7}{(2.579,2.177)}
\gppoint{gp mark 7}{(2.592,2.180)}
\gppoint{gp mark 7}{(2.605,2.183)}
\gppoint{gp mark 7}{(2.617,2.184)}
\gppoint{gp mark 7}{(2.630,2.184)}
\gppoint{gp mark 7}{(2.643,2.184)}
\gppoint{gp mark 7}{(2.655,2.184)}
\gppoint{gp mark 7}{(2.668,2.185)}
\gppoint{gp mark 7}{(2.681,2.185)}
\gppoint{gp mark 7}{(2.693,2.186)}
\gppoint{gp mark 7}{(2.706,2.186)}
\gppoint{gp mark 7}{(2.719,2.186)}
\gppoint{gp mark 7}{(2.731,2.186)}
\gppoint{gp mark 7}{(2.744,2.186)}
\gppoint{gp mark 7}{(2.757,2.187)}
\gppoint{gp mark 7}{(2.769,2.187)}
\gppoint{gp mark 7}{(2.782,2.187)}
\gppoint{gp mark 7}{(2.795,2.187)}
\gppoint{gp mark 7}{(2.807,2.187)}
\gppoint{gp mark 7}{(2.820,2.187)}
\gppoint{gp mark 7}{(2.833,2.187)}
\gppoint{gp mark 7}{(2.845,2.187)}
\gppoint{gp mark 7}{(2.858,2.189)}
\gppoint{gp mark 7}{(2.871,2.206)}
\gppoint{gp mark 7}{(2.883,2.305)}
\gppoint{gp mark 7}{(2.896,2.204)}
\gppoint{gp mark 7}{(2.909,2.091)}
\gppoint{gp mark 7}{(2.922,2.199)}
\gppoint{gp mark 7}{(2.934,2.171)}
\gppoint{gp mark 7}{(2.947,2.180)}
\gppoint{gp mark 7}{(2.960,2.187)}
\gppoint{gp mark 7}{(2.972,2.186)}
\gppoint{gp mark 7}{(2.985,2.187)}
\gppoint{gp mark 7}{(2.998,2.190)}
\gppoint{gp mark 7}{(3.010,2.191)}
\gppoint{gp mark 7}{(3.023,2.190)}
\gppoint{gp mark 7}{(3.036,2.189)}
\gppoint{gp mark 7}{(3.048,2.189)}
\gppoint{gp mark 7}{(3.061,2.189)}
\gppoint{gp mark 7}{(3.074,2.189)}
\gppoint{gp mark 7}{(3.086,2.189)}
\gppoint{gp mark 7}{(3.099,2.187)}
\gppoint{gp mark 7}{(3.112,2.186)}
\gppoint{gp mark 7}{(3.124,2.186)}
\gppoint{gp mark 7}{(3.137,2.188)}
\gppoint{gp mark 7}{(3.150,2.189)}
\gppoint{gp mark 7}{(3.162,2.190)}
\gppoint{gp mark 7}{(3.175,2.189)}
\gppoint{gp mark 7}{(3.188,2.188)}
\gppoint{gp mark 7}{(3.200,2.187)}
\gppoint{gp mark 7}{(3.213,2.188)}
\gppoint{gp mark 7}{(3.226,2.189)}
\gppoint{gp mark 7}{(3.238,2.188)}
\gppoint{gp mark 7}{(3.251,2.188)}
\gppoint{gp mark 7}{(3.264,2.187)}
\gppoint{gp mark 7}{(3.277,2.187)}
\gppoint{gp mark 7}{(3.289,2.187)}
\gppoint{gp mark 7}{(3.302,2.187)}
\gppoint{gp mark 7}{(3.315,2.187)}
\gppoint{gp mark 7}{(3.327,2.187)}
\gppoint{gp mark 7}{(3.340,2.187)}
\gppoint{gp mark 7}{(3.353,2.187)}
\gppoint{gp mark 7}{(3.365,2.189)}
\gppoint{gp mark 7}{(3.378,2.189)}
\gppoint{gp mark 7}{(3.391,2.189)}
\gppoint{gp mark 7}{(3.403,2.188)}
\gppoint{gp mark 7}{(3.416,2.187)}
\gppoint{gp mark 7}{(3.429,2.187)}
\gppoint{gp mark 7}{(3.441,2.187)}
\gppoint{gp mark 7}{(3.454,2.188)}
\gppoint{gp mark 7}{(3.467,2.188)}
\gppoint{gp mark 7}{(3.479,2.189)}
\gppoint{gp mark 7}{(3.492,2.189)}
\gppoint{gp mark 7}{(3.505,2.190)}
\gppoint{gp mark 7}{(3.517,2.190)}
\gppoint{gp mark 7}{(3.530,2.191)}
\gppoint{gp mark 7}{(3.543,2.191)}
\gppoint{gp mark 7}{(3.555,2.190)}
\gppoint{gp mark 7}{(3.568,2.189)}
\gppoint{gp mark 7}{(3.581,2.188)}
\gppoint{gp mark 7}{(3.593,2.188)}
\gppoint{gp mark 7}{(3.606,2.188)}
\gppoint{gp mark 7}{(3.619,2.188)}
\gppoint{gp mark 7}{(3.632,2.189)}
\gppoint{gp mark 7}{(3.644,2.190)}
\gppoint{gp mark 7}{(3.657,2.191)}
\gppoint{gp mark 7}{(3.670,2.192)}
\gppoint{gp mark 7}{(3.682,2.192)}
\gppoint{gp mark 7}{(3.695,2.192)}
\gppoint{gp mark 7}{(3.708,2.192)}
\gppoint{gp mark 7}{(3.720,2.190)}
\gppoint{gp mark 7}{(3.733,2.188)}
\gppoint{gp mark 7}{(3.746,2.187)}
\gppoint{gp mark 7}{(3.758,2.191)}
\gppoint{gp mark 7}{(3.771,2.237)}
\gppoint{gp mark 7}{(3.784,2.647)}
\gppoint{gp mark 7}{(3.796,3.988)}
\gppoint{gp mark 7}{(3.809,4.643)}
\gppoint{gp mark 7}{(3.822,4.693)}
\gppoint{gp mark 7}{(3.834,4.692)}
\gppoint{gp mark 7}{(3.847,4.691)}
\gppoint{gp mark 7}{(3.860,4.690)}
\gppoint{gp mark 7}{(3.872,4.691)}
\gppoint{gp mark 7}{(3.885,4.692)}
\gppoint{gp mark 7}{(3.898,4.692)}
\gppoint{gp mark 7}{(3.910,4.691)}
\gppoint{gp mark 7}{(3.923,4.691)}
\gppoint{gp mark 7}{(3.936,4.692)}
\gppoint{gp mark 7}{(3.948,4.692)}
\gppoint{gp mark 7}{(3.961,4.692)}
\gppoint{gp mark 7}{(3.974,4.691)}
\gppoint{gp mark 7}{(3.987,4.691)}
\gppoint{gp mark 7}{(3.999,4.691)}
\gppoint{gp mark 7}{(4.012,4.692)}
\gppoint{gp mark 7}{(4.025,4.692)}
\gppoint{gp mark 7}{(4.037,4.692)}
\gppoint{gp mark 7}{(4.050,4.692)}
\gppoint{gp mark 7}{(4.063,4.693)}
\gppoint{gp mark 7}{(4.075,4.693)}
\gppoint{gp mark 7}{(4.088,4.693)}
\gppoint{gp mark 7}{(4.101,4.691)}
\gppoint{gp mark 7}{(4.113,4.691)}
\gppoint{gp mark 7}{(4.126,4.691)}
\gppoint{gp mark 7}{(4.139,4.692)}
\gppoint{gp mark 7}{(4.151,4.692)}
\gppoint{gp mark 7}{(4.164,4.692)}
\gppoint{gp mark 7}{(4.177,4.692)}
\gppoint{gp mark 7}{(4.189,4.692)}
\gppoint{gp mark 7}{(4.202,4.693)}
\gppoint{gp mark 7}{(4.215,4.693)}
\gppoint{gp mark 7}{(4.227,4.692)}
\gppoint{gp mark 7}{(4.240,4.692)}
\gppoint{gp mark 7}{(4.253,4.692)}
\gppoint{gp mark 7}{(4.265,4.691)}
\gppoint{gp mark 7}{(4.278,4.691)}
\gppoint{gp mark 7}{(4.291,4.691)}
\gppoint{gp mark 7}{(4.303,4.692)}
\gppoint{gp mark 7}{(4.316,4.693)}
\gppoint{gp mark 7}{(4.329,4.693)}
\gppoint{gp mark 7}{(4.342,4.693)}
\gppoint{gp mark 7}{(4.354,4.692)}
\gppoint{gp mark 7}{(4.367,4.692)}
\gppoint{gp mark 7}{(4.380,4.692)}
\gppoint{gp mark 7}{(4.392,4.692)}
\gppoint{gp mark 7}{(4.405,4.692)}
\gppoint{gp mark 7}{(4.418,4.692)}
\gppoint{gp mark 7}{(4.430,4.692)}
\gppoint{gp mark 7}{(4.443,4.692)}
\gppoint{gp mark 7}{(4.456,4.692)}
\gppoint{gp mark 7}{(4.468,4.692)}
\gppoint{gp mark 7}{(4.481,4.692)}
\gppoint{gp mark 7}{(4.494,4.692)}
\gppoint{gp mark 7}{(4.506,4.691)}
\gppoint{gp mark 7}{(4.519,4.691)}
\gppoint{gp mark 7}{(4.532,4.691)}
\gppoint{gp mark 7}{(4.544,4.691)}
\gppoint{gp mark 7}{(4.557,4.692)}
\gppoint{gp mark 7}{(4.570,4.692)}
\gppoint{gp mark 7}{(4.582,4.692)}
\gppoint{gp mark 7}{(4.595,4.692)}
\gppoint{gp mark 7}{(4.608,4.691)}
\gppoint{gp mark 7}{(4.620,4.691)}
\gppoint{gp mark 7}{(4.633,4.691)}
\gppoint{gp mark 7}{(4.646,4.691)}
\gppoint{gp mark 7}{(4.658,4.692)}
\gppoint{gp mark 7}{(4.671,4.692)}
\gppoint{gp mark 7}{(4.684,4.692)}
\gppoint{gp mark 7}{(4.697,4.692)}
\gppoint{gp mark 7}{(4.709,4.692)}
\gppoint{gp mark 7}{(4.722,4.692)}
\gppoint{gp mark 7}{(4.735,4.692)}
\gppoint{gp mark 7}{(4.747,4.691)}
\gppoint{gp mark 7}{(4.760,4.691)}
\gppoint{gp mark 7}{(4.773,4.691)}
\gppoint{gp mark 7}{(4.785,4.691)}
\gppoint{gp mark 7}{(4.798,4.691)}
\gppoint{gp mark 7}{(4.811,4.691)}
\gppoint{gp mark 7}{(4.823,4.691)}
\gppoint{gp mark 7}{(4.836,4.692)}
\gppoint{gp mark 7}{(4.849,4.692)}
\gppoint{gp mark 7}{(4.861,4.691)}
\gppoint{gp mark 7}{(4.874,4.691)}
\gppoint{gp mark 7}{(4.887,4.691)}
\gppoint{gp mark 7}{(4.899,4.692)}
\gppoint{gp mark 7}{(4.912,4.692)}
\gppoint{gp mark 7}{(4.925,4.692)}
\gppoint{gp mark 7}{(4.937,4.692)}
\gppoint{gp mark 7}{(4.950,4.692)}
\gppoint{gp mark 7}{(4.963,4.691)}
\gppoint{gp mark 7}{(4.975,4.691)}
\gppoint{gp mark 7}{(4.988,4.691)}
\gppoint{gp mark 7}{(5.001,4.691)}
\gppoint{gp mark 7}{(5.013,4.691)}
\gppoint{gp mark 7}{(5.026,4.692)}
\gppoint{gp mark 7}{(5.039,4.692)}
\gppoint{gp mark 7}{(5.052,4.692)}
\gppoint{gp mark 7}{(5.064,4.692)}
\gppoint{gp mark 7}{(5.077,4.692)}
\gppoint{gp mark 7}{(5.090,4.692)}
\gppoint{gp mark 7}{(5.102,4.692)}
\gppoint{gp mark 7}{(5.115,4.691)}
\gppoint{gp mark 7}{(5.128,4.691)}
\gppoint{gp mark 7}{(5.140,4.691)}
\gppoint{gp mark 7}{(5.153,4.691)}
\gppoint{gp mark 7}{(5.166,4.691)}
\gppoint{gp mark 7}{(5.178,4.692)}
\gppoint{gp mark 7}{(5.191,4.692)}
\gppoint{gp mark 7}{(5.204,4.692)}
\gppoint{gp mark 7}{(5.216,4.692)}
\gppoint{gp mark 7}{(5.229,4.691)}
\gppoint{gp mark 7}{(5.242,4.691)}
\gppoint{gp mark 7}{(5.254,4.691)}
\gppoint{gp mark 7}{(5.267,4.691)}
\gppoint{gp mark 7}{(5.280,4.691)}
\gppoint{gp mark 7}{(5.292,4.691)}
\gppoint{gp mark 7}{(5.305,4.691)}
\gppoint{gp mark 7}{(5.318,4.692)}
\gppoint{gp mark 7}{(5.330,4.692)}
\gppoint{gp mark 7}{(5.343,4.692)}
\gppoint{gp mark 7}{(5.356,4.691)}
\gppoint{gp mark 7}{(5.368,4.691)}
\gppoint{gp mark 7}{(5.381,4.691)}
\gppoint{gp mark 7}{(5.394,4.691)}
\gppoint{gp mark 7}{(5.407,4.692)}
\gppoint{gp mark 7}{(5.419,4.692)}
\gppoint{gp mark 7}{(5.432,4.692)}
\gppoint{gp mark 7}{(5.445,4.692)}
\gppoint{gp mark 7}{(5.457,4.692)}
\gppoint{gp mark 7}{(5.470,4.691)}
\gppoint{gp mark 7}{(5.483,4.691)}
\gppoint{gp mark 7}{(5.495,4.691)}
\gppoint{gp mark 7}{(5.508,4.691)}
\gppoint{gp mark 7}{(5.521,4.691)}
\gppoint{gp mark 7}{(5.533,4.692)}
\gppoint{gp mark 7}{(5.546,4.692)}
\gppoint{gp mark 7}{(5.559,4.692)}
\gppoint{gp mark 7}{(5.571,4.692)}
\gppoint{gp mark 7}{(5.584,4.692)}
\gppoint{gp mark 7}{(5.597,4.692)}
\gppoint{gp mark 7}{(5.609,4.692)}
\gppoint{gp mark 7}{(5.622,4.691)}
\gppoint{gp mark 7}{(5.635,4.691)}
\gppoint{gp mark 7}{(5.647,4.691)}
\gppoint{gp mark 7}{(5.660,4.691)}
\gppoint{gp mark 7}{(5.673,4.692)}
\gppoint{gp mark 7}{(5.685,4.692)}
\gppoint{gp mark 7}{(5.698,4.692)}
\gppoint{gp mark 7}{(5.711,4.692)}
\gppoint{gp mark 7}{(5.723,4.692)}
\gppoint{gp mark 7}{(5.736,4.691)}
\gppoint{gp mark 7}{(5.749,4.691)}
\gppoint{gp mark 7}{(5.761,4.691)}
\gppoint{gp mark 7}{(5.774,4.691)}
\gppoint{gp mark 7}{(5.787,4.691)}
\gppoint{gp mark 7}{(5.800,4.691)}
\gppoint{gp mark 7}{(5.812,4.692)}
\gppoint{gp mark 7}{(5.825,4.692)}
\gppoint{gp mark 7}{(5.838,4.692)}
\gppoint{gp mark 7}{(5.850,4.692)}
\gppoint{gp mark 7}{(5.863,4.692)}
\gppoint{gp mark 7}{(5.876,4.692)}
\gppoint{gp mark 7}{(5.888,4.692)}
\gppoint{gp mark 7}{(5.901,4.692)}
\gppoint{gp mark 7}{(5.914,4.692)}
\gppoint{gp mark 7}{(5.926,4.692)}
\gppoint{gp mark 7}{(5.939,4.692)}
\gppoint{gp mark 7}{(5.952,4.692)}
\gppoint{gp mark 7}{(5.964,4.692)}
\gppoint{gp mark 7}{(5.977,4.692)}
\gppoint{gp mark 7}{(5.990,4.691)}
\gppoint{gp mark 7}{(6.002,4.691)}
\gppoint{gp mark 7}{(6.015,4.691)}
\gppoint{gp mark 7}{(6.028,4.691)}
\gppoint{gp mark 7}{(6.040,4.691)}
\gppoint{gp mark 7}{(6.053,4.691)}
\gppoint{gp mark 7}{(6.066,4.692)}
\gppoint{gp mark 7}{(6.078,4.692)}
\gppoint{gp mark 7}{(6.091,4.692)}
\gppoint{gp mark 7}{(6.104,4.692)}
\gppoint{gp mark 7}{(6.116,4.691)}
\gppoint{gp mark 7}{(6.129,4.691)}
\gppoint{gp mark 7}{(6.142,4.691)}
\gppoint{gp mark 7}{(6.155,4.691)}
\gppoint{gp mark 7}{(6.167,4.691)}
\gppoint{gp mark 7}{(6.180,4.691)}
\gppoint{gp mark 7}{(6.193,4.691)}
\gppoint{gp mark 7}{(6.205,4.691)}
\gppoint{gp mark 7}{(6.218,4.691)}
\gppoint{gp mark 7}{(6.231,4.691)}
\gppoint{gp mark 7}{(6.243,4.691)}
\gppoint{gp mark 7}{(6.256,4.690)}
\gppoint{gp mark 7}{(6.269,4.690)}
\gppoint{gp mark 7}{(6.281,4.690)}
\gppoint{gp mark 7}{(6.294,4.690)}
\gppoint{gp mark 7}{(6.307,4.690)}
\gppoint{gp mark 7}{(6.319,4.690)}
\gppoint{gp mark 7}{(6.332,4.690)}
\gppoint{gp mark 7}{(6.345,4.690)}
\gppoint{gp mark 7}{(6.357,4.690)}
\gppoint{gp mark 7}{(6.370,4.690)}
\gppoint{gp mark 7}{(6.383,4.690)}
\gppoint{gp mark 7}{(6.395,4.690)}
\gppoint{gp mark 7}{(6.408,4.690)}
\gppoint{gp mark 7}{(6.421,4.690)}
\gppoint{gp mark 7}{(6.433,4.690)}
\gppoint{gp mark 7}{(6.446,4.690)}
\gppoint{gp mark 7}{(6.459,4.690)}
\gppoint{gp mark 7}{(6.471,4.690)}
\gppoint{gp mark 7}{(6.484,4.690)}
\gppoint{gp mark 7}{(6.497,4.690)}
\gppoint{gp mark 7}{(6.510,4.690)}
\gppoint{gp mark 7}{(6.522,4.690)}
\gppoint{gp mark 7}{(6.535,4.690)}
\gppoint{gp mark 7}{(6.548,4.690)}
\gppoint{gp mark 7}{(6.560,4.689)}
\gppoint{gp mark 7}{(6.573,4.689)}
\gppoint{gp mark 7}{(6.586,4.689)}
\gppoint{gp mark 7}{(6.598,4.689)}
\gppoint{gp mark 7}{(6.611,4.689)}
\gppoint{gp mark 7}{(6.624,4.688)}
\gppoint{gp mark 7}{(6.636,4.688)}
\gppoint{gp mark 7}{(6.649,4.688)}
\gppoint{gp mark 7}{(6.662,4.688)}
\gppoint{gp mark 7}{(6.674,4.688)}
\gppoint{gp mark 7}{(6.687,4.688)}
\gppoint{gp mark 7}{(6.700,4.688)}
\gppoint{gp mark 7}{(6.712,4.688)}
\gppoint{gp mark 7}{(6.725,4.687)}
\gppoint{gp mark 7}{(6.738,4.687)}
\gppoint{gp mark 7}{(6.750,4.687)}
\gppoint{gp mark 7}{(6.763,4.687)}
\gppoint{gp mark 7}{(6.776,4.687)}
\gppoint{gp mark 7}{(6.788,4.687)}
\gppoint{gp mark 7}{(6.801,4.687)}
\gppoint{gp mark 7}{(6.814,4.687)}
\gppoint{gp mark 7}{(6.826,4.687)}
\gppoint{gp mark 7}{(6.839,4.686)}
\gppoint{gp mark 7}{(6.852,4.686)}
\gppoint{gp mark 7}{(6.865,4.686)}
\gppoint{gp mark 7}{(6.877,4.686)}
\gppoint{gp mark 7}{(6.890,4.686)}
\gppoint{gp mark 7}{(6.903,4.686)}
\gppoint{gp mark 7}{(6.915,4.686)}
\gppoint{gp mark 7}{(6.928,4.686)}
\gppoint{gp mark 7}{(6.941,4.686)}
\gppoint{gp mark 7}{(6.953,4.686)}
\gppoint{gp mark 7}{(6.966,4.686)}
\gppoint{gp mark 7}{(6.979,4.687)}
\gppoint{gp mark 7}{(6.991,4.687)}
\gppoint{gp mark 7}{(7.004,4.687)}
\gppoint{gp mark 7}{(7.017,4.687)}
\gppoint{gp mark 7}{(7.029,4.687)}
\gppoint{gp mark 7}{(7.042,4.687)}
\gppoint{gp mark 7}{(7.055,4.687)}
\gppoint{gp mark 7}{(7.067,4.687)}
\gppoint{gp mark 7}{(7.080,4.688)}
\gppoint{gp mark 7}{(7.093,4.688)}
\gppoint{gp mark 7}{(7.105,4.689)}
\gppoint{gp mark 7}{(7.118,4.689)}
\gppoint{gp mark 7}{(7.131,4.689)}
\gppoint{gp mark 7}{(7.143,4.690)}
\gppoint{gp mark 7}{(7.156,4.691)}
\gppoint{gp mark 7}{(7.169,4.692)}
\gppoint{gp mark 7}{(7.181,4.692)}
\gppoint{gp mark 7}{(7.194,4.693)}
\gppoint{gp mark 7}{(7.207,4.694)}
\gppoint{gp mark 7}{(7.220,4.696)}
\gppoint{gp mark 7}{(7.232,4.699)}
\gppoint{gp mark 7}{(7.245,4.701)}
\gppoint{gp mark 7}{(7.258,4.703)}
\gppoint{gp mark 7}{(7.270,4.704)}
\gppoint{gp mark 7}{(7.283,4.705)}
\gppoint{gp mark 7}{(7.296,4.706)}
\gppoint{gp mark 7}{(7.308,4.708)}
\gppoint{gp mark 7}{(7.321,4.711)}
\gppoint{gp mark 7}{(7.334,4.715)}
\gppoint{gp mark 7}{(7.346,4.718)}
\gppoint{gp mark 7}{(7.359,4.719)}
\gppoint{gp mark 7}{(7.372,4.720)}
\gppoint{gp mark 7}{(7.384,4.720)}
\gppoint{gp mark 7}{(7.397,4.720)}
\gppoint{gp mark 7}{(7.410,4.720)}
\gppoint{gp mark 7}{(7.422,4.720)}
\gppoint{gp mark 7}{(7.435,4.719)}
\gppoint{gp mark 7}{(7.448,4.717)}
\gppoint{gp mark 7}{(7.460,4.712)}
\gppoint{gp mark 7}{(7.473,4.705)}
\gppoint{gp mark 7}{(7.486,4.698)}
\gppoint{gp mark 7}{(7.498,4.691)}
\gppoint{gp mark 7}{(7.511,4.685)}
\gppoint{gp mark 7}{(7.524,4.679)}
\gppoint{gp mark 7}{(7.536,4.672)}
\gppoint{gp mark 7}{(7.549,4.660)}
\gppoint{gp mark 7}{(7.562,4.643)}
\gppoint{gp mark 7}{(7.575,4.621)}
\gppoint{gp mark 7}{(7.587,4.601)}
\gppoint{gp mark 7}{(7.600,4.590)}
\gppoint{gp mark 7}{(7.613,4.586)}
\gppoint{gp mark 7}{(7.625,4.585)}
\gppoint{gp mark 7}{(7.638,4.584)}
\gppoint{gp mark 7}{(7.651,4.582)}
\gppoint{gp mark 7}{(7.663,4.570)}
\gppoint{gp mark 7}{(7.676,4.484)}
\gppoint{gp mark 7}{(7.689,4.088)}
\gppoint{gp mark 7}{(7.701,3.343)}
\gppoint{gp mark 7}{(7.714,2.509)}
\gppoint{gp mark 7}{(7.727,1.854)}
\gppoint{gp mark 7}{(7.739,1.459)}
\gppoint{gp mark 7}{(7.752,1.264)}
\gppoint{gp mark 7}{(7.765,1.183)}
\gppoint{gp mark 7}{(7.777,1.154)}
\gppoint{gp mark 7}{(7.790,1.146)}
\gppoint{gp mark 7}{(7.803,1.145)}
\gppoint{gp mark 7}{(7.815,1.145)}
\gppoint{gp mark 7}{(7.828,1.145)}
\gppoint{gp mark 7}{(7.841,1.147)}
\gppoint{gp mark 7}{(7.853,1.151)}
\gppoint{gp mark 7}{(7.866,1.153)}
\gppoint{gp mark 7}{(7.879,1.154)}
\gppoint{gp mark 7}{(7.891,1.155)}
\gppoint{gp mark 7}{(7.904,1.155)}
\gppoint{gp mark 7}{(7.917,1.155)}
\gppoint{gp mark 7}{(7.930,1.155)}
\gpcolor{rgb color={0.000,0.000,0.000}}
\gpsetlinetype{gp lt plot 0}
\gpsetlinewidth{4.00}
\draw[gp path] (2.440,2.203)--(2.899,2.203);
\draw[gp path] (2.899,2.203)--(3.805,2.203);
\draw[gp path] (3.805,4.694)--(7.720,4.694);
\draw[gp path] (7.720,1.153)--(7.947,1.153);
\draw[gp path] (1.204,3.442)--(1.217,3.426)--(1.230,3.411)--(1.243,3.396)--(1.256,3.380)%
  --(1.269,3.365)--(1.282,3.350)--(1.295,3.334)--(1.308,3.319)--(1.321,3.304)--(1.334,3.289)%
  --(1.347,3.274)--(1.360,3.259)--(1.373,3.244)--(1.386,3.229)--(1.399,3.214)--(1.412,3.199)%
  --(1.425,3.184)--(1.438,3.169)--(1.451,3.155)--(1.464,3.140)--(1.477,3.125)--(1.490,3.111)%
  --(1.503,3.096)--(1.516,3.082)--(1.529,3.067)--(1.542,3.053)--(1.555,3.038)--(1.568,3.024)%
  --(1.581,3.010)--(1.594,2.996)--(1.607,2.982)--(1.620,2.967)--(1.633,2.953)--(1.646,2.939)%
  --(1.659,2.925)--(1.672,2.911)--(1.686,2.898)--(1.699,2.884)--(1.712,2.870)--(1.725,2.856)%
  --(1.738,2.843)--(1.751,2.829)--(1.764,2.815)--(1.777,2.802)--(1.790,2.788)--(1.803,2.775)%
  --(1.816,2.762)--(1.829,2.748)--(1.842,2.735)--(1.855,2.722)--(1.868,2.709)--(1.881,2.696)%
  --(1.894,2.683)--(1.907,2.670)--(1.920,2.657)--(1.933,2.644)--(1.946,2.631)--(1.959,2.618)%
  --(1.972,2.605)--(1.985,2.593)--(1.998,2.580)--(2.011,2.567)--(2.024,2.555)--(2.037,2.542)%
  --(2.050,2.530)--(2.063,2.518)--(2.076,2.505)--(2.089,2.493)--(2.102,2.481)--(2.115,2.469)%
  --(2.128,2.457)--(2.141,2.445)--(2.154,2.433)--(2.167,2.421)--(2.180,2.409)--(2.193,2.397)%
  --(2.206,2.386)--(2.219,2.374)--(2.232,2.362)--(2.245,2.351)--(2.258,2.339)--(2.271,2.328)%
  --(2.284,2.317)--(2.297,2.305)--(2.310,2.294)--(2.323,2.283)--(2.336,2.272)--(2.349,2.261)%
  --(2.362,2.250)--(2.375,2.239)--(2.388,2.228)--(2.401,2.217)--(2.414,2.206)--(2.427,2.196)%
  --(2.440,2.203);
\draw[gp path] (3.805,2.203)--(3.805,4.694);
\draw[gp path] (7.720,4.694)--(7.720,1.153);
\node[gp node left,font={\fontsize{10pt}{12pt}\selectfont}] at (1.456,5.268) {\LARGE $\rho$};
\node[gp node left,font={\fontsize{10pt}{12pt}\selectfont}] at (5.740,5.268) {\large $\alpha = \pi$};
%% coordinates of the plot area
\gpdefrectangularnode{gp plot 1}{\pgfpoint{1.196cm}{0.985cm}}{\pgfpoint{7.947cm}{5.631cm}}
\end{tikzpicture}
%% gnuplot variables
}
& 
\resizebox{0.5\linewidth}{!}{\tikzsetnextfilename{fast_coplanar_a_rsol_init_6} \begin{tikzpicture}[gnuplot]
%% generated with GNUPLOT 4.6p4 (Lua 5.1; terminal rev. 99, script rev. 100)
%% Sat 02 Aug 2014 10:07:54 AM EDT
\path (0.000,0.000) rectangle (8.500,6.000);
\gpfill{rgb color={1.000,1.000,1.000}} (1.196,0.985)--(7.946,0.985)--(7.946,5.630)--(1.196,5.630)--cycle;
\gpcolor{color=gp lt color border}
\gpsetlinetype{gp lt border}
\gpsetlinewidth{1.00}
\draw[gp path] (1.196,0.985)--(1.196,5.630)--(7.946,5.630)--(7.946,0.985)--cycle;
\gpcolor{color=gp lt color axes}
\gpsetlinetype{gp lt axes}
\gpsetlinewidth{2.00}
\draw[gp path] (1.196,0.985)--(7.947,0.985);
\gpcolor{color=gp lt color border}
\gpsetlinetype{gp lt border}
\draw[gp path] (1.196,0.985)--(1.268,0.985);
\draw[gp path] (7.947,0.985)--(7.875,0.985);
\gpcolor{rgb color={0.000,0.000,0.000}}
\node[gp node right,font={\fontsize{10pt}{12pt}\selectfont}] at (1.012,0.985) {-0.4};
\gpcolor{color=gp lt color axes}
\gpsetlinetype{gp lt axes}
\draw[gp path] (1.196,1.759)--(7.947,1.759);
\gpcolor{color=gp lt color border}
\gpsetlinetype{gp lt border}
\draw[gp path] (1.196,1.759)--(1.268,1.759);
\draw[gp path] (7.947,1.759)--(7.875,1.759);
\gpcolor{rgb color={0.000,0.000,0.000}}
\node[gp node right,font={\fontsize{10pt}{12pt}\selectfont}] at (1.012,1.759) {-0.2};
\gpcolor{color=gp lt color axes}
\gpsetlinetype{gp lt axes}
\draw[gp path] (1.196,2.534)--(7.947,2.534);
\gpcolor{color=gp lt color border}
\gpsetlinetype{gp lt border}
\draw[gp path] (1.196,2.534)--(1.268,2.534);
\draw[gp path] (7.947,2.534)--(7.875,2.534);
\gpcolor{rgb color={0.000,0.000,0.000}}
\node[gp node right,font={\fontsize{10pt}{12pt}\selectfont}] at (1.012,2.534) {0};
\gpcolor{color=gp lt color axes}
\gpsetlinetype{gp lt axes}
\draw[gp path] (1.196,3.308)--(7.947,3.308);
\gpcolor{color=gp lt color border}
\gpsetlinetype{gp lt border}
\draw[gp path] (1.196,3.308)--(1.268,3.308);
\draw[gp path] (7.947,3.308)--(7.875,3.308);
\gpcolor{rgb color={0.000,0.000,0.000}}
\node[gp node right,font={\fontsize{10pt}{12pt}\selectfont}] at (1.012,3.308) {0.2};
\gpcolor{color=gp lt color axes}
\gpsetlinetype{gp lt axes}
\draw[gp path] (1.196,4.082)--(7.947,4.082);
\gpcolor{color=gp lt color border}
\gpsetlinetype{gp lt border}
\draw[gp path] (1.196,4.082)--(1.268,4.082);
\draw[gp path] (7.947,4.082)--(7.875,4.082);
\gpcolor{rgb color={0.000,0.000,0.000}}
\node[gp node right,font={\fontsize{10pt}{12pt}\selectfont}] at (1.012,4.082) {0.4};
\gpcolor{color=gp lt color axes}
\gpsetlinetype{gp lt axes}
\draw[gp path] (1.196,4.857)--(7.947,4.857);
\gpcolor{color=gp lt color border}
\gpsetlinetype{gp lt border}
\draw[gp path] (1.196,4.857)--(1.268,4.857);
\draw[gp path] (7.947,4.857)--(7.875,4.857);
\gpcolor{rgb color={0.000,0.000,0.000}}
\node[gp node right,font={\fontsize{10pt}{12pt}\selectfont}] at (1.012,4.857) {0.6};
\gpcolor{color=gp lt color axes}
\gpsetlinetype{gp lt axes}
\draw[gp path] (1.196,5.631)--(7.947,5.631);
\gpcolor{color=gp lt color border}
\gpsetlinetype{gp lt border}
\draw[gp path] (1.196,5.631)--(1.268,5.631);
\draw[gp path] (7.947,5.631)--(7.875,5.631);
\gpcolor{rgb color={0.000,0.000,0.000}}
\node[gp node right,font={\fontsize{10pt}{12pt}\selectfont}] at (1.012,5.631) {0.8};
\gpcolor{color=gp lt color axes}
\gpsetlinetype{gp lt axes}
\draw[gp path] (1.196,0.985)--(1.196,5.631);
\gpcolor{color=gp lt color border}
\gpsetlinetype{gp lt border}
\draw[gp path] (1.196,0.985)--(1.196,1.057);
\draw[gp path] (1.196,5.631)--(1.196,5.559);
\gpcolor{rgb color={0.000,0.000,0.000}}
\node[gp node center,font={\fontsize{10pt}{12pt}\selectfont}] at (1.196,0.677) {0.3};
\gpcolor{color=gp lt color axes}
\gpsetlinetype{gp lt axes}
\draw[gp path] (2.494,0.985)--(2.494,5.631);
\gpcolor{color=gp lt color border}
\gpsetlinetype{gp lt border}
\draw[gp path] (2.494,0.985)--(2.494,1.057);
\draw[gp path] (2.494,5.631)--(2.494,5.559);
\gpcolor{rgb color={0.000,0.000,0.000}}
\node[gp node center,font={\fontsize{10pt}{12pt}\selectfont}] at (2.494,0.677) {0.35};
\gpcolor{color=gp lt color axes}
\gpsetlinetype{gp lt axes}
\draw[gp path] (3.793,0.985)--(3.793,5.631);
\gpcolor{color=gp lt color border}
\gpsetlinetype{gp lt border}
\draw[gp path] (3.793,0.985)--(3.793,1.057);
\draw[gp path] (3.793,5.631)--(3.793,5.559);
\gpcolor{rgb color={0.000,0.000,0.000}}
\node[gp node center,font={\fontsize{10pt}{12pt}\selectfont}] at (3.793,0.677) {0.4};
\gpcolor{color=gp lt color axes}
\gpsetlinetype{gp lt axes}
\draw[gp path] (5.091,0.985)--(5.091,5.631);
\gpcolor{color=gp lt color border}
\gpsetlinetype{gp lt border}
\draw[gp path] (5.091,0.985)--(5.091,1.057);
\draw[gp path] (5.091,5.631)--(5.091,5.559);
\gpcolor{rgb color={0.000,0.000,0.000}}
\node[gp node center,font={\fontsize{10pt}{12pt}\selectfont}] at (5.091,0.677) {0.45};
\gpcolor{color=gp lt color axes}
\gpsetlinetype{gp lt axes}
\draw[gp path] (6.389,0.985)--(6.389,5.631);
\gpcolor{color=gp lt color border}
\gpsetlinetype{gp lt border}
\draw[gp path] (6.389,0.985)--(6.389,1.057);
\draw[gp path] (6.389,5.631)--(6.389,5.559);
\gpcolor{rgb color={0.000,0.000,0.000}}
\node[gp node center,font={\fontsize{10pt}{12pt}\selectfont}] at (6.389,0.677) {0.5};
\gpcolor{color=gp lt color axes}
\gpsetlinetype{gp lt axes}
\draw[gp path] (7.687,0.985)--(7.687,5.631);
\gpcolor{color=gp lt color border}
\gpsetlinetype{gp lt border}
\draw[gp path] (7.687,0.985)--(7.687,1.057);
\draw[gp path] (7.687,5.631)--(7.687,5.559);
\gpcolor{rgb color={0.000,0.000,0.000}}
\node[gp node center,font={\fontsize{10pt}{12pt}\selectfont}] at (7.687,0.677) {0.55};
\gpcolor{color=gp lt color border}
\draw[gp path] (1.196,5.631)--(1.196,0.985)--(7.947,0.985)--(7.947,5.631)--cycle;
\gpcolor{rgb color={0.000,0.000,0.000}}
\node[gp node center,font={\fontsize{10pt}{12pt}\selectfont}] at (4.571,0.215) {\large $x$};
\gpcolor{rgb color={1.000,0.000,0.000}}
\gpsetlinewidth{0.50}
\gpsetpointsize{4.44}
\gppoint{gp mark 7}{(1.210,4.952)}
\gppoint{gp mark 7}{(1.223,4.943)}
\gppoint{gp mark 7}{(1.235,4.933)}
\gppoint{gp mark 7}{(1.248,4.924)}
\gppoint{gp mark 7}{(1.261,4.914)}
\gppoint{gp mark 7}{(1.273,4.905)}
\gppoint{gp mark 7}{(1.286,4.895)}
\gppoint{gp mark 7}{(1.299,4.885)}
\gppoint{gp mark 7}{(1.311,4.876)}
\gppoint{gp mark 7}{(1.324,4.866)}
\gppoint{gp mark 7}{(1.337,4.857)}
\gppoint{gp mark 7}{(1.349,4.847)}
\gppoint{gp mark 7}{(1.362,4.837)}
\gppoint{gp mark 7}{(1.375,4.828)}
\gppoint{gp mark 7}{(1.387,4.818)}
\gppoint{gp mark 7}{(1.400,4.808)}
\gppoint{gp mark 7}{(1.413,4.798)}
\gppoint{gp mark 7}{(1.425,4.789)}
\gppoint{gp mark 7}{(1.438,4.779)}
\gppoint{gp mark 7}{(1.451,4.769)}
\gppoint{gp mark 7}{(1.464,4.759)}
\gppoint{gp mark 7}{(1.476,4.749)}
\gppoint{gp mark 7}{(1.489,4.740)}
\gppoint{gp mark 7}{(1.502,4.730)}
\gppoint{gp mark 7}{(1.514,4.720)}
\gppoint{gp mark 7}{(1.527,4.710)}
\gppoint{gp mark 7}{(1.540,4.700)}
\gppoint{gp mark 7}{(1.552,4.690)}
\gppoint{gp mark 7}{(1.565,4.680)}
\gppoint{gp mark 7}{(1.578,4.670)}
\gppoint{gp mark 7}{(1.590,4.660)}
\gppoint{gp mark 7}{(1.603,4.650)}
\gppoint{gp mark 7}{(1.616,4.640)}
\gppoint{gp mark 7}{(1.628,4.630)}
\gppoint{gp mark 7}{(1.641,4.620)}
\gppoint{gp mark 7}{(1.654,4.610)}
\gppoint{gp mark 7}{(1.666,4.599)}
\gppoint{gp mark 7}{(1.679,4.589)}
\gppoint{gp mark 7}{(1.692,4.579)}
\gppoint{gp mark 7}{(1.704,4.569)}
\gppoint{gp mark 7}{(1.717,4.558)}
\gppoint{gp mark 7}{(1.730,4.548)}
\gppoint{gp mark 7}{(1.742,4.538)}
\gppoint{gp mark 7}{(1.755,4.528)}
\gppoint{gp mark 7}{(1.768,4.517)}
\gppoint{gp mark 7}{(1.780,4.507)}
\gppoint{gp mark 7}{(1.793,4.496)}
\gppoint{gp mark 7}{(1.806,4.486)}
\gppoint{gp mark 7}{(1.819,4.475)}
\gppoint{gp mark 7}{(1.831,4.465)}
\gppoint{gp mark 7}{(1.844,4.454)}
\gppoint{gp mark 7}{(1.857,4.444)}
\gppoint{gp mark 7}{(1.869,4.433)}
\gppoint{gp mark 7}{(1.882,4.423)}
\gppoint{gp mark 7}{(1.895,4.412)}
\gppoint{gp mark 7}{(1.907,4.401)}
\gppoint{gp mark 7}{(1.920,4.390)}
\gppoint{gp mark 7}{(1.933,4.380)}
\gppoint{gp mark 7}{(1.945,4.369)}
\gppoint{gp mark 7}{(1.958,4.358)}
\gppoint{gp mark 7}{(1.971,4.347)}
\gppoint{gp mark 7}{(1.983,4.336)}
\gppoint{gp mark 7}{(1.996,4.325)}
\gppoint{gp mark 7}{(2.009,4.314)}
\gppoint{gp mark 7}{(2.021,4.303)}
\gppoint{gp mark 7}{(2.034,4.292)}
\gppoint{gp mark 7}{(2.047,4.281)}
\gppoint{gp mark 7}{(2.059,4.270)}
\gppoint{gp mark 7}{(2.072,4.259)}
\gppoint{gp mark 7}{(2.085,4.248)}
\gppoint{gp mark 7}{(2.097,4.236)}
\gppoint{gp mark 7}{(2.110,4.225)}
\gppoint{gp mark 7}{(2.123,4.214)}
\gppoint{gp mark 7}{(2.135,4.203)}
\gppoint{gp mark 7}{(2.148,4.191)}
\gppoint{gp mark 7}{(2.161,4.180)}
\gppoint{gp mark 7}{(2.174,4.168)}
\gppoint{gp mark 7}{(2.186,4.157)}
\gppoint{gp mark 7}{(2.199,4.145)}
\gppoint{gp mark 7}{(2.212,4.133)}
\gppoint{gp mark 7}{(2.224,4.122)}
\gppoint{gp mark 7}{(2.237,4.110)}
\gppoint{gp mark 7}{(2.250,4.098)}
\gppoint{gp mark 7}{(2.262,4.086)}
\gppoint{gp mark 7}{(2.275,4.074)}
\gppoint{gp mark 7}{(2.288,4.063)}
\gppoint{gp mark 7}{(2.300,4.051)}
\gppoint{gp mark 7}{(2.313,4.039)}
\gppoint{gp mark 7}{(2.326,4.027)}
\gppoint{gp mark 7}{(2.338,4.014)}
\gppoint{gp mark 7}{(2.351,4.002)}
\gppoint{gp mark 7}{(2.364,3.990)}
\gppoint{gp mark 7}{(2.376,3.978)}
\gppoint{gp mark 7}{(2.389,3.966)}
\gppoint{gp mark 7}{(2.402,3.954)}
\gppoint{gp mark 7}{(2.414,3.942)}
\gppoint{gp mark 7}{(2.427,3.930)}
\gppoint{gp mark 7}{(2.440,3.919)}
\gppoint{gp mark 7}{(2.452,3.908)}
\gppoint{gp mark 7}{(2.465,3.899)}
\gppoint{gp mark 7}{(2.478,3.892)}
\gppoint{gp mark 7}{(2.490,3.888)}
\gppoint{gp mark 7}{(2.503,3.887)}
\gppoint{gp mark 7}{(2.516,3.887)}
\gppoint{gp mark 7}{(2.529,3.886)}
\gppoint{gp mark 7}{(2.541,3.887)}
\gppoint{gp mark 7}{(2.554,3.887)}
\gppoint{gp mark 7}{(2.567,3.887)}
\gppoint{gp mark 7}{(2.579,3.889)}
\gppoint{gp mark 7}{(2.592,3.892)}
\gppoint{gp mark 7}{(2.605,3.895)}
\gppoint{gp mark 7}{(2.617,3.896)}
\gppoint{gp mark 7}{(2.630,3.896)}
\gppoint{gp mark 7}{(2.643,3.896)}
\gppoint{gp mark 7}{(2.655,3.896)}
\gppoint{gp mark 7}{(2.668,3.896)}
\gppoint{gp mark 7}{(2.681,3.897)}
\gppoint{gp mark 7}{(2.693,3.897)}
\gppoint{gp mark 7}{(2.706,3.898)}
\gppoint{gp mark 7}{(2.719,3.898)}
\gppoint{gp mark 7}{(2.731,3.898)}
\gppoint{gp mark 7}{(2.744,3.898)}
\gppoint{gp mark 7}{(2.757,3.898)}
\gppoint{gp mark 7}{(2.769,3.898)}
\gppoint{gp mark 7}{(2.782,3.898)}
\gppoint{gp mark 7}{(2.795,3.898)}
\gppoint{gp mark 7}{(2.807,3.899)}
\gppoint{gp mark 7}{(2.820,3.899)}
\gppoint{gp mark 7}{(2.833,3.899)}
\gppoint{gp mark 7}{(2.845,3.899)}
\gppoint{gp mark 7}{(2.858,3.898)}
\gppoint{gp mark 7}{(2.871,3.896)}
\gppoint{gp mark 7}{(2.883,3.883)}
\gppoint{gp mark 7}{(2.896,3.379)}
\gppoint{gp mark 7}{(2.909,1.863)}
\gppoint{gp mark 7}{(2.922,1.213)}
\gppoint{gp mark 7}{(2.934,1.174)}
\gppoint{gp mark 7}{(2.947,1.174)}
\gppoint{gp mark 7}{(2.960,1.170)}
\gppoint{gp mark 7}{(2.972,1.168)}
\gppoint{gp mark 7}{(2.985,1.168)}
\gppoint{gp mark 7}{(2.998,1.169)}
\gppoint{gp mark 7}{(3.010,1.168)}
\gppoint{gp mark 7}{(3.023,1.168)}
\gppoint{gp mark 7}{(3.036,1.168)}
\gppoint{gp mark 7}{(3.048,1.168)}
\gppoint{gp mark 7}{(3.061,1.168)}
\gppoint{gp mark 7}{(3.074,1.168)}
\gppoint{gp mark 7}{(3.086,1.168)}
\gppoint{gp mark 7}{(3.099,1.168)}
\gppoint{gp mark 7}{(3.112,1.169)}
\gppoint{gp mark 7}{(3.124,1.169)}
\gppoint{gp mark 7}{(3.137,1.169)}
\gppoint{gp mark 7}{(3.150,1.168)}
\gppoint{gp mark 7}{(3.162,1.168)}
\gppoint{gp mark 7}{(3.175,1.167)}
\gppoint{gp mark 7}{(3.188,1.167)}
\gppoint{gp mark 7}{(3.200,1.167)}
\gppoint{gp mark 7}{(3.213,1.167)}
\gppoint{gp mark 7}{(3.226,1.167)}
\gppoint{gp mark 7}{(3.238,1.167)}
\gppoint{gp mark 7}{(3.251,1.168)}
\gppoint{gp mark 7}{(3.264,1.169)}
\gppoint{gp mark 7}{(3.277,1.169)}
\gppoint{gp mark 7}{(3.289,1.169)}
\gppoint{gp mark 7}{(3.302,1.169)}
\gppoint{gp mark 7}{(3.315,1.168)}
\gppoint{gp mark 7}{(3.327,1.168)}
\gppoint{gp mark 7}{(3.340,1.167)}
\gppoint{gp mark 7}{(3.353,1.167)}
\gppoint{gp mark 7}{(3.365,1.167)}
\gppoint{gp mark 7}{(3.378,1.167)}
\gppoint{gp mark 7}{(3.391,1.167)}
\gppoint{gp mark 7}{(3.403,1.168)}
\gppoint{gp mark 7}{(3.416,1.169)}
\gppoint{gp mark 7}{(3.429,1.169)}
\gppoint{gp mark 7}{(3.441,1.169)}
\gppoint{gp mark 7}{(3.454,1.169)}
\gppoint{gp mark 7}{(3.467,1.169)}
\gppoint{gp mark 7}{(3.479,1.168)}
\gppoint{gp mark 7}{(3.492,1.167)}
\gppoint{gp mark 7}{(3.505,1.167)}
\gppoint{gp mark 7}{(3.517,1.167)}
\gppoint{gp mark 7}{(3.530,1.167)}
\gppoint{gp mark 7}{(3.543,1.167)}
\gppoint{gp mark 7}{(3.555,1.168)}
\gppoint{gp mark 7}{(3.568,1.169)}
\gppoint{gp mark 7}{(3.581,1.170)}
\gppoint{gp mark 7}{(3.593,1.170)}
\gppoint{gp mark 7}{(3.606,1.170)}
\gppoint{gp mark 7}{(3.619,1.169)}
\gppoint{gp mark 7}{(3.632,1.168)}
\gppoint{gp mark 7}{(3.644,1.167)}
\gppoint{gp mark 7}{(3.657,1.167)}
\gppoint{gp mark 7}{(3.670,1.167)}
\gppoint{gp mark 7}{(3.682,1.167)}
\gppoint{gp mark 7}{(3.695,1.167)}
\gppoint{gp mark 7}{(3.708,1.167)}
\gppoint{gp mark 7}{(3.720,1.168)}
\gppoint{gp mark 7}{(3.733,1.169)}
\gppoint{gp mark 7}{(3.746,1.170)}
\gppoint{gp mark 7}{(3.758,1.171)}
\gppoint{gp mark 7}{(3.771,1.178)}
\gppoint{gp mark 7}{(3.784,1.241)}
\gppoint{gp mark 7}{(3.796,1.479)}
\gppoint{gp mark 7}{(3.809,1.600)}
\gppoint{gp mark 7}{(3.822,1.610)}
\gppoint{gp mark 7}{(3.834,1.610)}
\gppoint{gp mark 7}{(3.847,1.610)}
\gppoint{gp mark 7}{(3.860,1.610)}
\gppoint{gp mark 7}{(3.872,1.610)}
\gppoint{gp mark 7}{(3.885,1.610)}
\gppoint{gp mark 7}{(3.898,1.610)}
\gppoint{gp mark 7}{(3.910,1.610)}
\gppoint{gp mark 7}{(3.923,1.610)}
\gppoint{gp mark 7}{(3.936,1.610)}
\gppoint{gp mark 7}{(3.948,1.610)}
\gppoint{gp mark 7}{(3.961,1.610)}
\gppoint{gp mark 7}{(3.974,1.610)}
\gppoint{gp mark 7}{(3.987,1.610)}
\gppoint{gp mark 7}{(3.999,1.610)}
\gppoint{gp mark 7}{(4.012,1.610)}
\gppoint{gp mark 7}{(4.025,1.610)}
\gppoint{gp mark 7}{(4.037,1.610)}
\gppoint{gp mark 7}{(4.050,1.610)}
\gppoint{gp mark 7}{(4.063,1.610)}
\gppoint{gp mark 7}{(4.075,1.610)}
\gppoint{gp mark 7}{(4.088,1.610)}
\gppoint{gp mark 7}{(4.101,1.610)}
\gppoint{gp mark 7}{(4.113,1.610)}
\gppoint{gp mark 7}{(4.126,1.610)}
\gppoint{gp mark 7}{(4.139,1.610)}
\gppoint{gp mark 7}{(4.151,1.610)}
\gppoint{gp mark 7}{(4.164,1.610)}
\gppoint{gp mark 7}{(4.177,1.610)}
\gppoint{gp mark 7}{(4.189,1.610)}
\gppoint{gp mark 7}{(4.202,1.610)}
\gppoint{gp mark 7}{(4.215,1.610)}
\gppoint{gp mark 7}{(4.227,1.610)}
\gppoint{gp mark 7}{(4.240,1.610)}
\gppoint{gp mark 7}{(4.253,1.610)}
\gppoint{gp mark 7}{(4.265,1.610)}
\gppoint{gp mark 7}{(4.278,1.610)}
\gppoint{gp mark 7}{(4.291,1.610)}
\gppoint{gp mark 7}{(4.303,1.610)}
\gppoint{gp mark 7}{(4.316,1.610)}
\gppoint{gp mark 7}{(4.329,1.610)}
\gppoint{gp mark 7}{(4.342,1.610)}
\gppoint{gp mark 7}{(4.354,1.610)}
\gppoint{gp mark 7}{(4.367,1.610)}
\gppoint{gp mark 7}{(4.380,1.610)}
\gppoint{gp mark 7}{(4.392,1.610)}
\gppoint{gp mark 7}{(4.405,1.610)}
\gppoint{gp mark 7}{(4.418,1.610)}
\gppoint{gp mark 7}{(4.430,1.610)}
\gppoint{gp mark 7}{(4.443,1.610)}
\gppoint{gp mark 7}{(4.456,1.610)}
\gppoint{gp mark 7}{(4.468,1.610)}
\gppoint{gp mark 7}{(4.481,1.610)}
\gppoint{gp mark 7}{(4.494,1.610)}
\gppoint{gp mark 7}{(4.506,1.610)}
\gppoint{gp mark 7}{(4.519,1.610)}
\gppoint{gp mark 7}{(4.532,1.610)}
\gppoint{gp mark 7}{(4.544,1.610)}
\gppoint{gp mark 7}{(4.557,1.610)}
\gppoint{gp mark 7}{(4.570,1.610)}
\gppoint{gp mark 7}{(4.582,1.610)}
\gppoint{gp mark 7}{(4.595,1.610)}
\gppoint{gp mark 7}{(4.608,1.610)}
\gppoint{gp mark 7}{(4.620,1.610)}
\gppoint{gp mark 7}{(4.633,1.610)}
\gppoint{gp mark 7}{(4.646,1.610)}
\gppoint{gp mark 7}{(4.658,1.610)}
\gppoint{gp mark 7}{(4.671,1.610)}
\gppoint{gp mark 7}{(4.684,1.610)}
\gppoint{gp mark 7}{(4.697,1.610)}
\gppoint{gp mark 7}{(4.709,1.610)}
\gppoint{gp mark 7}{(4.722,1.610)}
\gppoint{gp mark 7}{(4.735,1.610)}
\gppoint{gp mark 7}{(4.747,1.610)}
\gppoint{gp mark 7}{(4.760,1.610)}
\gppoint{gp mark 7}{(4.773,1.610)}
\gppoint{gp mark 7}{(4.785,1.610)}
\gppoint{gp mark 7}{(4.798,1.610)}
\gppoint{gp mark 7}{(4.811,1.610)}
\gppoint{gp mark 7}{(4.823,1.610)}
\gppoint{gp mark 7}{(4.836,1.610)}
\gppoint{gp mark 7}{(4.849,1.610)}
\gppoint{gp mark 7}{(4.861,1.610)}
\gppoint{gp mark 7}{(4.874,1.610)}
\gppoint{gp mark 7}{(4.887,1.610)}
\gppoint{gp mark 7}{(4.899,1.610)}
\gppoint{gp mark 7}{(4.912,1.610)}
\gppoint{gp mark 7}{(4.925,1.610)}
\gppoint{gp mark 7}{(4.937,1.610)}
\gppoint{gp mark 7}{(4.950,1.610)}
\gppoint{gp mark 7}{(4.963,1.610)}
\gppoint{gp mark 7}{(4.975,1.610)}
\gppoint{gp mark 7}{(4.988,1.610)}
\gppoint{gp mark 7}{(5.001,1.610)}
\gppoint{gp mark 7}{(5.013,1.610)}
\gppoint{gp mark 7}{(5.026,1.610)}
\gppoint{gp mark 7}{(5.039,1.610)}
\gppoint{gp mark 7}{(5.052,1.610)}
\gppoint{gp mark 7}{(5.064,1.610)}
\gppoint{gp mark 7}{(5.077,1.610)}
\gppoint{gp mark 7}{(5.090,1.610)}
\gppoint{gp mark 7}{(5.102,1.610)}
\gppoint{gp mark 7}{(5.115,1.610)}
\gppoint{gp mark 7}{(5.128,1.610)}
\gppoint{gp mark 7}{(5.140,1.610)}
\gppoint{gp mark 7}{(5.153,1.610)}
\gppoint{gp mark 7}{(5.166,1.610)}
\gppoint{gp mark 7}{(5.178,1.610)}
\gppoint{gp mark 7}{(5.191,1.610)}
\gppoint{gp mark 7}{(5.204,1.610)}
\gppoint{gp mark 7}{(5.216,1.610)}
\gppoint{gp mark 7}{(5.229,1.610)}
\gppoint{gp mark 7}{(5.242,1.610)}
\gppoint{gp mark 7}{(5.254,1.610)}
\gppoint{gp mark 7}{(5.267,1.610)}
\gppoint{gp mark 7}{(5.280,1.610)}
\gppoint{gp mark 7}{(5.292,1.610)}
\gppoint{gp mark 7}{(5.305,1.610)}
\gppoint{gp mark 7}{(5.318,1.610)}
\gppoint{gp mark 7}{(5.330,1.610)}
\gppoint{gp mark 7}{(5.343,1.610)}
\gppoint{gp mark 7}{(5.356,1.610)}
\gppoint{gp mark 7}{(5.368,1.610)}
\gppoint{gp mark 7}{(5.381,1.610)}
\gppoint{gp mark 7}{(5.394,1.610)}
\gppoint{gp mark 7}{(5.407,1.610)}
\gppoint{gp mark 7}{(5.419,1.610)}
\gppoint{gp mark 7}{(5.432,1.610)}
\gppoint{gp mark 7}{(5.445,1.610)}
\gppoint{gp mark 7}{(5.457,1.610)}
\gppoint{gp mark 7}{(5.470,1.610)}
\gppoint{gp mark 7}{(5.483,1.610)}
\gppoint{gp mark 7}{(5.495,1.610)}
\gppoint{gp mark 7}{(5.508,1.610)}
\gppoint{gp mark 7}{(5.521,1.610)}
\gppoint{gp mark 7}{(5.533,1.610)}
\gppoint{gp mark 7}{(5.546,1.610)}
\gppoint{gp mark 7}{(5.559,1.610)}
\gppoint{gp mark 7}{(5.571,1.610)}
\gppoint{gp mark 7}{(5.584,1.610)}
\gppoint{gp mark 7}{(5.597,1.610)}
\gppoint{gp mark 7}{(5.609,1.610)}
\gppoint{gp mark 7}{(5.622,1.610)}
\gppoint{gp mark 7}{(5.635,1.610)}
\gppoint{gp mark 7}{(5.647,1.610)}
\gppoint{gp mark 7}{(5.660,1.610)}
\gppoint{gp mark 7}{(5.673,1.610)}
\gppoint{gp mark 7}{(5.685,1.610)}
\gppoint{gp mark 7}{(5.698,1.610)}
\gppoint{gp mark 7}{(5.711,1.610)}
\gppoint{gp mark 7}{(5.723,1.610)}
\gppoint{gp mark 7}{(5.736,1.610)}
\gppoint{gp mark 7}{(5.749,1.610)}
\gppoint{gp mark 7}{(5.761,1.610)}
\gppoint{gp mark 7}{(5.774,1.610)}
\gppoint{gp mark 7}{(5.787,1.610)}
\gppoint{gp mark 7}{(5.800,1.610)}
\gppoint{gp mark 7}{(5.812,1.610)}
\gppoint{gp mark 7}{(5.825,1.610)}
\gppoint{gp mark 7}{(5.838,1.610)}
\gppoint{gp mark 7}{(5.850,1.610)}
\gppoint{gp mark 7}{(5.863,1.610)}
\gppoint{gp mark 7}{(5.876,1.610)}
\gppoint{gp mark 7}{(5.888,1.610)}
\gppoint{gp mark 7}{(5.901,1.610)}
\gppoint{gp mark 7}{(5.914,1.610)}
\gppoint{gp mark 7}{(5.926,1.610)}
\gppoint{gp mark 7}{(5.939,1.610)}
\gppoint{gp mark 7}{(5.952,1.610)}
\gppoint{gp mark 7}{(5.964,1.610)}
\gppoint{gp mark 7}{(5.977,1.610)}
\gppoint{gp mark 7}{(5.990,1.610)}
\gppoint{gp mark 7}{(6.002,1.610)}
\gppoint{gp mark 7}{(6.015,1.610)}
\gppoint{gp mark 7}{(6.028,1.610)}
\gppoint{gp mark 7}{(6.040,1.610)}
\gppoint{gp mark 7}{(6.053,1.610)}
\gppoint{gp mark 7}{(6.066,1.610)}
\gppoint{gp mark 7}{(6.078,1.610)}
\gppoint{gp mark 7}{(6.091,1.610)}
\gppoint{gp mark 7}{(6.104,1.610)}
\gppoint{gp mark 7}{(6.116,1.610)}
\gppoint{gp mark 7}{(6.129,1.610)}
\gppoint{gp mark 7}{(6.142,1.610)}
\gppoint{gp mark 7}{(6.155,1.610)}
\gppoint{gp mark 7}{(6.167,1.610)}
\gppoint{gp mark 7}{(6.180,1.610)}
\gppoint{gp mark 7}{(6.193,1.610)}
\gppoint{gp mark 7}{(6.205,1.610)}
\gppoint{gp mark 7}{(6.218,1.610)}
\gppoint{gp mark 7}{(6.231,1.610)}
\gppoint{gp mark 7}{(6.243,1.610)}
\gppoint{gp mark 7}{(6.256,1.610)}
\gppoint{gp mark 7}{(6.269,1.610)}
\gppoint{gp mark 7}{(6.281,1.610)}
\gppoint{gp mark 7}{(6.294,1.610)}
\gppoint{gp mark 7}{(6.307,1.610)}
\gppoint{gp mark 7}{(6.319,1.610)}
\gppoint{gp mark 7}{(6.332,1.610)}
\gppoint{gp mark 7}{(6.345,1.610)}
\gppoint{gp mark 7}{(6.357,1.610)}
\gppoint{gp mark 7}{(6.370,1.610)}
\gppoint{gp mark 7}{(6.383,1.610)}
\gppoint{gp mark 7}{(6.395,1.610)}
\gppoint{gp mark 7}{(6.408,1.610)}
\gppoint{gp mark 7}{(6.421,1.610)}
\gppoint{gp mark 7}{(6.433,1.610)}
\gppoint{gp mark 7}{(6.446,1.610)}
\gppoint{gp mark 7}{(6.459,1.610)}
\gppoint{gp mark 7}{(6.471,1.610)}
\gppoint{gp mark 7}{(6.484,1.610)}
\gppoint{gp mark 7}{(6.497,1.610)}
\gppoint{gp mark 7}{(6.510,1.610)}
\gppoint{gp mark 7}{(6.522,1.610)}
\gppoint{gp mark 7}{(6.535,1.610)}
\gppoint{gp mark 7}{(6.548,1.610)}
\gppoint{gp mark 7}{(6.560,1.610)}
\gppoint{gp mark 7}{(6.573,1.610)}
\gppoint{gp mark 7}{(6.586,1.610)}
\gppoint{gp mark 7}{(6.598,1.610)}
\gppoint{gp mark 7}{(6.611,1.610)}
\gppoint{gp mark 7}{(6.624,1.610)}
\gppoint{gp mark 7}{(6.636,1.610)}
\gppoint{gp mark 7}{(6.649,1.610)}
\gppoint{gp mark 7}{(6.662,1.610)}
\gppoint{gp mark 7}{(6.674,1.610)}
\gppoint{gp mark 7}{(6.687,1.610)}
\gppoint{gp mark 7}{(6.700,1.610)}
\gppoint{gp mark 7}{(6.712,1.610)}
\gppoint{gp mark 7}{(6.725,1.610)}
\gppoint{gp mark 7}{(6.738,1.610)}
\gppoint{gp mark 7}{(6.750,1.610)}
\gppoint{gp mark 7}{(6.763,1.610)}
\gppoint{gp mark 7}{(6.776,1.610)}
\gppoint{gp mark 7}{(6.788,1.610)}
\gppoint{gp mark 7}{(6.801,1.610)}
\gppoint{gp mark 7}{(6.814,1.610)}
\gppoint{gp mark 7}{(6.826,1.610)}
\gppoint{gp mark 7}{(6.839,1.610)}
\gppoint{gp mark 7}{(6.852,1.610)}
\gppoint{gp mark 7}{(6.865,1.610)}
\gppoint{gp mark 7}{(6.877,1.610)}
\gppoint{gp mark 7}{(6.890,1.610)}
\gppoint{gp mark 7}{(6.903,1.610)}
\gppoint{gp mark 7}{(6.915,1.610)}
\gppoint{gp mark 7}{(6.928,1.610)}
\gppoint{gp mark 7}{(6.941,1.610)}
\gppoint{gp mark 7}{(6.953,1.610)}
\gppoint{gp mark 7}{(6.966,1.610)}
\gppoint{gp mark 7}{(6.979,1.610)}
\gppoint{gp mark 7}{(6.991,1.610)}
\gppoint{gp mark 7}{(7.004,1.610)}
\gppoint{gp mark 7}{(7.017,1.610)}
\gppoint{gp mark 7}{(7.029,1.610)}
\gppoint{gp mark 7}{(7.042,1.610)}
\gppoint{gp mark 7}{(7.055,1.610)}
\gppoint{gp mark 7}{(7.067,1.610)}
\gppoint{gp mark 7}{(7.080,1.610)}
\gppoint{gp mark 7}{(7.093,1.610)}
\gppoint{gp mark 7}{(7.105,1.610)}
\gppoint{gp mark 7}{(7.118,1.610)}
\gppoint{gp mark 7}{(7.131,1.610)}
\gppoint{gp mark 7}{(7.143,1.610)}
\gppoint{gp mark 7}{(7.156,1.610)}
\gppoint{gp mark 7}{(7.169,1.610)}
\gppoint{gp mark 7}{(7.181,1.610)}
\gppoint{gp mark 7}{(7.194,1.610)}
\gppoint{gp mark 7}{(7.207,1.610)}
\gppoint{gp mark 7}{(7.220,1.610)}
\gppoint{gp mark 7}{(7.232,1.610)}
\gppoint{gp mark 7}{(7.245,1.610)}
\gppoint{gp mark 7}{(7.258,1.610)}
\gppoint{gp mark 7}{(7.270,1.610)}
\gppoint{gp mark 7}{(7.283,1.610)}
\gppoint{gp mark 7}{(7.296,1.610)}
\gppoint{gp mark 7}{(7.308,1.610)}
\gppoint{gp mark 7}{(7.321,1.610)}
\gppoint{gp mark 7}{(7.334,1.610)}
\gppoint{gp mark 7}{(7.346,1.610)}
\gppoint{gp mark 7}{(7.359,1.610)}
\gppoint{gp mark 7}{(7.372,1.610)}
\gppoint{gp mark 7}{(7.384,1.610)}
\gppoint{gp mark 7}{(7.397,1.610)}
\gppoint{gp mark 7}{(7.410,1.610)}
\gppoint{gp mark 7}{(7.422,1.610)}
\gppoint{gp mark 7}{(7.435,1.610)}
\gppoint{gp mark 7}{(7.448,1.610)}
\gppoint{gp mark 7}{(7.460,1.610)}
\gppoint{gp mark 7}{(7.473,1.610)}
\gppoint{gp mark 7}{(7.486,1.610)}
\gppoint{gp mark 7}{(7.498,1.610)}
\gppoint{gp mark 7}{(7.511,1.610)}
\gppoint{gp mark 7}{(7.524,1.610)}
\gppoint{gp mark 7}{(7.536,1.610)}
\gppoint{gp mark 7}{(7.549,1.610)}
\gppoint{gp mark 7}{(7.562,1.610)}
\gppoint{gp mark 7}{(7.575,1.610)}
\gppoint{gp mark 7}{(7.587,1.610)}
\gppoint{gp mark 7}{(7.600,1.610)}
\gppoint{gp mark 7}{(7.613,1.610)}
\gppoint{gp mark 7}{(7.625,1.610)}
\gppoint{gp mark 7}{(7.638,1.610)}
\gppoint{gp mark 7}{(7.651,1.610)}
\gppoint{gp mark 7}{(7.663,1.610)}
\gppoint{gp mark 7}{(7.676,1.610)}
\gppoint{gp mark 7}{(7.689,1.610)}
\gppoint{gp mark 7}{(7.701,1.610)}
\gppoint{gp mark 7}{(7.714,1.610)}
\gppoint{gp mark 7}{(7.727,1.610)}
\gppoint{gp mark 7}{(7.739,1.610)}
\gppoint{gp mark 7}{(7.752,1.610)}
\gppoint{gp mark 7}{(7.765,1.610)}
\gppoint{gp mark 7}{(7.777,1.610)}
\gppoint{gp mark 7}{(7.790,1.610)}
\gppoint{gp mark 7}{(7.803,1.610)}
\gppoint{gp mark 7}{(7.815,1.610)}
\gppoint{gp mark 7}{(7.828,1.610)}
\gppoint{gp mark 7}{(7.841,1.610)}
\gppoint{gp mark 7}{(7.853,1.610)}
\gppoint{gp mark 7}{(7.866,1.610)}
\gppoint{gp mark 7}{(7.879,1.610)}
\gppoint{gp mark 7}{(7.891,1.610)}
\gppoint{gp mark 7}{(7.904,1.610)}
\gppoint{gp mark 7}{(7.917,1.610)}
\gppoint{gp mark 7}{(7.930,1.610)}
\gpcolor{rgb color={0.000,0.000,0.000}}
\gpsetlinetype{gp lt plot 0}
\gpsetlinewidth{4.00}
\draw[gp path] (2.440,3.897)--(2.899,3.897);
\draw[gp path] (2.899,1.171)--(3.805,1.171);
\draw[gp path] (3.805,1.610)--(7.720,1.610);
\draw[gp path] (7.720,1.610)--(7.947,1.610);
\draw[gp path] (1.204,4.831)--(1.217,4.821)--(1.230,4.812)--(1.243,4.802)--(1.256,4.792)%
  --(1.269,4.782)--(1.282,4.772)--(1.295,4.762)--(1.308,4.753)--(1.321,4.743)--(1.334,4.733)%
  --(1.347,4.723)--(1.360,4.713)--(1.373,4.703)--(1.386,4.694)--(1.399,4.684)--(1.412,4.674)%
  --(1.425,4.664)--(1.438,4.654)--(1.451,4.644)--(1.464,4.635)--(1.477,4.625)--(1.490,4.615)%
  --(1.503,4.605)--(1.516,4.595)--(1.529,4.585)--(1.542,4.576)--(1.555,4.566)--(1.568,4.556)%
  --(1.581,4.546)--(1.594,4.536)--(1.607,4.526)--(1.620,4.517)--(1.633,4.507)--(1.646,4.497)%
  --(1.659,4.487)--(1.672,4.477)--(1.686,4.467)--(1.699,4.457)--(1.712,4.448)--(1.725,4.438)%
  --(1.738,4.428)--(1.751,4.418)--(1.764,4.408)--(1.777,4.398)--(1.790,4.389)--(1.803,4.379)%
  --(1.816,4.369)--(1.829,4.359)--(1.842,4.349)--(1.855,4.339)--(1.868,4.330)--(1.881,4.320)%
  --(1.894,4.310)--(1.907,4.300)--(1.920,4.290)--(1.933,4.280)--(1.946,4.271)--(1.959,4.261)%
  --(1.972,4.251)--(1.985,4.241)--(1.998,4.231)--(2.011,4.221)--(2.024,4.212)--(2.037,4.202)%
  --(2.050,4.192)--(2.063,4.182)--(2.076,4.172)--(2.089,4.162)--(2.102,4.153)--(2.115,4.143)%
  --(2.128,4.133)--(2.141,4.123)--(2.154,4.113)--(2.167,4.103)--(2.180,4.093)--(2.193,4.084)%
  --(2.206,4.074)--(2.219,4.064)--(2.232,4.054)--(2.245,4.044)--(2.258,4.034)--(2.271,4.025)%
  --(2.284,4.015)--(2.297,4.005)--(2.310,3.995)--(2.323,3.985)--(2.336,3.975)--(2.349,3.966)%
  --(2.362,3.956)--(2.375,3.946)--(2.388,3.936)--(2.401,3.926)--(2.414,3.916)--(2.427,3.907)%
  --(2.440,3.897);
\draw[gp path] (2.899,3.897)--(2.899,1.171);
\draw[gp path] (3.805,1.171)--(3.805,1.610);
\draw[gp path] (3.793,3.695)--(4.572,3.695);
\gpcolor{rgb color={1.000,0.000,0.000}}
\gpsetlinewidth{0.50}
\gppoint{gp mark 7}{(4.182,2.921)}
\gpcolor{rgb color={0.000,0.000,0.000}}
\node[gp node left,font={\fontsize{10pt}{12pt}\selectfont}] at (1.456,5.166) {\LARGE $B_y$};
\node[gp node left,font={\fontsize{10pt}{12pt}\selectfont}] at (5.740,5.166) {\large $\alpha = \pi$};
\node[gp node left,font={\fontsize{10pt}{12pt}\selectfont}] at (4.831,3.695) {\large exact};
\node[gp node left,font={\fontsize{10pt}{12pt}\selectfont}] at (4.831,2.921) {\large converging};
%% coordinates of the plot area
\gpdefrectangularnode{gp plot 1}{\pgfpoint{1.196cm}{0.985cm}}{\pgfpoint{7.947cm}{5.631cm}}
\end{tikzpicture}
%% gnuplot variables
}
\end{tabular}
\caption{The approximate solution after the first flux correction of HLLD-CWM and exact r-solution to the full Riemann problem for the near-coplanar case with $2048$ grid points.  The compound wave is almost completely removed, except near $x=0.365$ a weak intermediate shock remains.}
\label{fig:fast_coplanar_a_rsol_init}
\end{figure}

%-----------------------------------------------------------------
% Fast coplanar initial r-solution
%-----------------------------------------------------------------
\begin{figure}[htbp] 
\begin{tabular}{cc}
\resizebox{0.5\linewidth}{!}{\tikzsetnextfilename{fast_coplanar_b_rsol_init_1}\begin{tikzpicture}[gnuplot]
%% generated with GNUPLOT 4.6p4 (Lua 5.1; terminal rev. 99, script rev. 100)
%% Sat 02 Aug 2014 10:10:57 AM EDT
\path (0.000,0.000) rectangle (8.500,6.000);
\gpfill{rgb color={1.000,1.000,1.000}} (1.196,0.985)--(7.946,0.985)--(7.946,5.630)--(1.196,5.630)--cycle;
\gpcolor{color=gp lt color border}
\gpsetlinetype{gp lt border}
\gpsetlinewidth{1.00}
\draw[gp path] (1.196,0.985)--(1.196,5.630)--(7.946,5.630)--(7.946,0.985)--cycle;
\gpcolor{color=gp lt color axes}
\gpsetlinetype{gp lt axes}
\gpsetlinewidth{2.00}
\draw[gp path] (1.196,1.275)--(7.947,1.275);
\gpcolor{color=gp lt color border}
\gpsetlinetype{gp lt border}
\draw[gp path] (1.196,1.275)--(1.268,1.275);
\draw[gp path] (7.947,1.275)--(7.875,1.275);
\gpcolor{rgb color={0.000,0.000,0.000}}
\node[gp node right,font={\fontsize{10pt}{12pt}\selectfont}] at (1.012,1.275) {0.65};
\gpcolor{color=gp lt color axes}
\gpsetlinetype{gp lt axes}
\draw[gp path] (1.196,2.001)--(7.947,2.001);
\gpcolor{color=gp lt color border}
\gpsetlinetype{gp lt border}
\draw[gp path] (1.196,2.001)--(1.268,2.001);
\draw[gp path] (7.947,2.001)--(7.875,2.001);
\gpcolor{rgb color={0.000,0.000,0.000}}
\node[gp node right,font={\fontsize{10pt}{12pt}\selectfont}] at (1.012,2.001) {0.7};
\gpcolor{color=gp lt color axes}
\gpsetlinetype{gp lt axes}
\draw[gp path] (1.196,2.727)--(7.947,2.727);
\gpcolor{color=gp lt color border}
\gpsetlinetype{gp lt border}
\draw[gp path] (1.196,2.727)--(1.268,2.727);
\draw[gp path] (7.947,2.727)--(7.875,2.727);
\gpcolor{rgb color={0.000,0.000,0.000}}
\node[gp node right,font={\fontsize{10pt}{12pt}\selectfont}] at (1.012,2.727) {0.75};
\gpcolor{color=gp lt color axes}
\gpsetlinetype{gp lt axes}
\draw[gp path] (1.196,3.453)--(7.947,3.453);
\gpcolor{color=gp lt color border}
\gpsetlinetype{gp lt border}
\draw[gp path] (1.196,3.453)--(1.268,3.453);
\draw[gp path] (7.947,3.453)--(7.875,3.453);
\gpcolor{rgb color={0.000,0.000,0.000}}
\node[gp node right,font={\fontsize{10pt}{12pt}\selectfont}] at (1.012,3.453) {0.8};
\gpcolor{color=gp lt color axes}
\gpsetlinetype{gp lt axes}
\draw[gp path] (1.196,4.179)--(7.947,4.179);
\gpcolor{color=gp lt color border}
\gpsetlinetype{gp lt border}
\draw[gp path] (1.196,4.179)--(1.268,4.179);
\draw[gp path] (7.947,4.179)--(7.875,4.179);
\gpcolor{rgb color={0.000,0.000,0.000}}
\node[gp node right,font={\fontsize{10pt}{12pt}\selectfont}] at (1.012,4.179) {0.85};
\gpcolor{color=gp lt color axes}
\gpsetlinetype{gp lt axes}
\draw[gp path] (1.196,4.905)--(7.947,4.905);
\gpcolor{color=gp lt color border}
\gpsetlinetype{gp lt border}
\draw[gp path] (1.196,4.905)--(1.268,4.905);
\draw[gp path] (7.947,4.905)--(7.875,4.905);
\gpcolor{rgb color={0.000,0.000,0.000}}
\node[gp node right,font={\fontsize{10pt}{12pt}\selectfont}] at (1.012,4.905) {0.9};
\gpcolor{color=gp lt color axes}
\gpsetlinetype{gp lt axes}
\draw[gp path] (1.196,5.631)--(7.947,5.631);
\gpcolor{color=gp lt color border}
\gpsetlinetype{gp lt border}
\draw[gp path] (1.196,5.631)--(1.268,5.631);
\draw[gp path] (7.947,5.631)--(7.875,5.631);
\gpcolor{rgb color={0.000,0.000,0.000}}
\node[gp node right,font={\fontsize{10pt}{12pt}\selectfont}] at (1.012,5.631) {0.95};
\gpcolor{color=gp lt color axes}
\gpsetlinetype{gp lt axes}
\draw[gp path] (1.196,0.985)--(1.196,5.631);
\gpcolor{color=gp lt color border}
\gpsetlinetype{gp lt border}
\draw[gp path] (1.196,0.985)--(1.196,1.057);
\draw[gp path] (1.196,5.631)--(1.196,5.559);
\gpcolor{rgb color={0.000,0.000,0.000}}
\node[gp node center,font={\fontsize{10pt}{12pt}\selectfont}] at (1.196,0.677) {0.3};
\gpcolor{color=gp lt color axes}
\gpsetlinetype{gp lt axes}
\draw[gp path] (2.494,0.985)--(2.494,5.631);
\gpcolor{color=gp lt color border}
\gpsetlinetype{gp lt border}
\draw[gp path] (2.494,0.985)--(2.494,1.057);
\draw[gp path] (2.494,5.631)--(2.494,5.559);
\gpcolor{rgb color={0.000,0.000,0.000}}
\node[gp node center,font={\fontsize{10pt}{12pt}\selectfont}] at (2.494,0.677) {0.35};
\gpcolor{color=gp lt color axes}
\gpsetlinetype{gp lt axes}
\draw[gp path] (3.793,0.985)--(3.793,5.631);
\gpcolor{color=gp lt color border}
\gpsetlinetype{gp lt border}
\draw[gp path] (3.793,0.985)--(3.793,1.057);
\draw[gp path] (3.793,5.631)--(3.793,5.559);
\gpcolor{rgb color={0.000,0.000,0.000}}
\node[gp node center,font={\fontsize{10pt}{12pt}\selectfont}] at (3.793,0.677) {0.4};
\gpcolor{color=gp lt color axes}
\gpsetlinetype{gp lt axes}
\draw[gp path] (5.091,0.985)--(5.091,5.631);
\gpcolor{color=gp lt color border}
\gpsetlinetype{gp lt border}
\draw[gp path] (5.091,0.985)--(5.091,1.057);
\draw[gp path] (5.091,5.631)--(5.091,5.559);
\gpcolor{rgb color={0.000,0.000,0.000}}
\node[gp node center,font={\fontsize{10pt}{12pt}\selectfont}] at (5.091,0.677) {0.45};
\gpcolor{color=gp lt color axes}
\gpsetlinetype{gp lt axes}
\draw[gp path] (6.389,0.985)--(6.389,5.631);
\gpcolor{color=gp lt color border}
\gpsetlinetype{gp lt border}
\draw[gp path] (6.389,0.985)--(6.389,1.057);
\draw[gp path] (6.389,5.631)--(6.389,5.559);
\gpcolor{rgb color={0.000,0.000,0.000}}
\node[gp node center,font={\fontsize{10pt}{12pt}\selectfont}] at (6.389,0.677) {0.5};
\gpcolor{color=gp lt color axes}
\gpsetlinetype{gp lt axes}
\draw[gp path] (7.687,0.985)--(7.687,5.631);
\gpcolor{color=gp lt color border}
\gpsetlinetype{gp lt border}
\draw[gp path] (7.687,0.985)--(7.687,1.057);
\draw[gp path] (7.687,5.631)--(7.687,5.559);
\gpcolor{rgb color={0.000,0.000,0.000}}
\node[gp node center,font={\fontsize{10pt}{12pt}\selectfont}] at (7.687,0.677) {0.55};
\gpcolor{color=gp lt color border}
\draw[gp path] (1.196,5.631)--(1.196,0.985)--(7.947,0.985)--(7.947,5.631)--cycle;
\gpcolor{rgb color={0.000,0.000,0.000}}
\node[gp node center,font={\fontsize{10pt}{12pt}\selectfont}] at (4.571,0.215) {\large $x$};
\gpcolor{rgb color={1.000,0.000,0.000}}
\gpsetlinewidth{0.50}
\gpsetpointsize{4.44}
\gppoint{gp mark 7}{(1.210,3.636)}
\gppoint{gp mark 7}{(1.223,3.621)}
\gppoint{gp mark 7}{(1.235,3.605)}
\gppoint{gp mark 7}{(1.248,3.590)}
\gppoint{gp mark 7}{(1.261,3.574)}
\gppoint{gp mark 7}{(1.273,3.559)}
\gppoint{gp mark 7}{(1.286,3.544)}
\gppoint{gp mark 7}{(1.299,3.528)}
\gppoint{gp mark 7}{(1.311,3.513)}
\gppoint{gp mark 7}{(1.324,3.498)}
\gppoint{gp mark 7}{(1.337,3.482)}
\gppoint{gp mark 7}{(1.349,3.467)}
\gppoint{gp mark 7}{(1.362,3.452)}
\gppoint{gp mark 7}{(1.375,3.436)}
\gppoint{gp mark 7}{(1.387,3.421)}
\gppoint{gp mark 7}{(1.400,3.406)}
\gppoint{gp mark 7}{(1.413,3.391)}
\gppoint{gp mark 7}{(1.425,3.375)}
\gppoint{gp mark 7}{(1.438,3.360)}
\gppoint{gp mark 7}{(1.451,3.345)}
\gppoint{gp mark 7}{(1.464,3.330)}
\gppoint{gp mark 7}{(1.476,3.314)}
\gppoint{gp mark 7}{(1.489,3.299)}
\gppoint{gp mark 7}{(1.502,3.284)}
\gppoint{gp mark 7}{(1.514,3.269)}
\gppoint{gp mark 7}{(1.527,3.254)}
\gppoint{gp mark 7}{(1.540,3.239)}
\gppoint{gp mark 7}{(1.552,3.224)}
\gppoint{gp mark 7}{(1.565,3.209)}
\gppoint{gp mark 7}{(1.578,3.193)}
\gppoint{gp mark 7}{(1.590,3.178)}
\gppoint{gp mark 7}{(1.603,3.163)}
\gppoint{gp mark 7}{(1.616,3.148)}
\gppoint{gp mark 7}{(1.628,3.133)}
\gppoint{gp mark 7}{(1.641,3.118)}
\gppoint{gp mark 7}{(1.654,3.103)}
\gppoint{gp mark 7}{(1.666,3.088)}
\gppoint{gp mark 7}{(1.679,3.073)}
\gppoint{gp mark 7}{(1.692,3.058)}
\gppoint{gp mark 7}{(1.704,3.043)}
\gppoint{gp mark 7}{(1.717,3.029)}
\gppoint{gp mark 7}{(1.730,3.014)}
\gppoint{gp mark 7}{(1.742,2.999)}
\gppoint{gp mark 7}{(1.755,2.984)}
\gppoint{gp mark 7}{(1.768,2.969)}
\gppoint{gp mark 7}{(1.780,2.954)}
\gppoint{gp mark 7}{(1.793,2.939)}
\gppoint{gp mark 7}{(1.806,2.924)}
\gppoint{gp mark 7}{(1.819,2.910)}
\gppoint{gp mark 7}{(1.831,2.895)}
\gppoint{gp mark 7}{(1.844,2.880)}
\gppoint{gp mark 7}{(1.857,2.865)}
\gppoint{gp mark 7}{(1.869,2.851)}
\gppoint{gp mark 7}{(1.882,2.836)}
\gppoint{gp mark 7}{(1.895,2.821)}
\gppoint{gp mark 7}{(1.907,2.806)}
\gppoint{gp mark 7}{(1.920,2.792)}
\gppoint{gp mark 7}{(1.933,2.777)}
\gppoint{gp mark 7}{(1.945,2.762)}
\gppoint{gp mark 7}{(1.958,2.748)}
\gppoint{gp mark 7}{(1.971,2.733)}
\gppoint{gp mark 7}{(1.983,2.719)}
\gppoint{gp mark 7}{(1.996,2.704)}
\gppoint{gp mark 7}{(2.009,2.689)}
\gppoint{gp mark 7}{(2.021,2.675)}
\gppoint{gp mark 7}{(2.034,2.660)}
\gppoint{gp mark 7}{(2.047,2.646)}
\gppoint{gp mark 7}{(2.059,2.631)}
\gppoint{gp mark 7}{(2.072,2.617)}
\gppoint{gp mark 7}{(2.085,2.602)}
\gppoint{gp mark 7}{(2.097,2.588)}
\gppoint{gp mark 7}{(2.110,2.573)}
\gppoint{gp mark 7}{(2.123,2.559)}
\gppoint{gp mark 7}{(2.135,2.545)}
\gppoint{gp mark 7}{(2.148,2.530)}
\gppoint{gp mark 7}{(2.161,2.516)}
\gppoint{gp mark 7}{(2.174,2.502)}
\gppoint{gp mark 7}{(2.186,2.487)}
\gppoint{gp mark 7}{(2.199,2.473)}
\gppoint{gp mark 7}{(2.212,2.459)}
\gppoint{gp mark 7}{(2.224,2.445)}
\gppoint{gp mark 7}{(2.237,2.430)}
\gppoint{gp mark 7}{(2.250,2.416)}
\gppoint{gp mark 7}{(2.262,2.402)}
\gppoint{gp mark 7}{(2.275,2.388)}
\gppoint{gp mark 7}{(2.288,2.374)}
\gppoint{gp mark 7}{(2.300,2.360)}
\gppoint{gp mark 7}{(2.313,2.346)}
\gppoint{gp mark 7}{(2.326,2.332)}
\gppoint{gp mark 7}{(2.338,2.319)}
\gppoint{gp mark 7}{(2.351,2.305)}
\gppoint{gp mark 7}{(2.364,2.292)}
\gppoint{gp mark 7}{(2.376,2.278)}
\gppoint{gp mark 7}{(2.389,2.266)}
\gppoint{gp mark 7}{(2.402,2.254)}
\gppoint{gp mark 7}{(2.414,2.243)}
\gppoint{gp mark 7}{(2.427,2.234)}
\gppoint{gp mark 7}{(2.440,2.228)}
\gppoint{gp mark 7}{(2.452,2.224)}
\gppoint{gp mark 7}{(2.465,2.223)}
\gppoint{gp mark 7}{(2.478,2.223)}
\gppoint{gp mark 7}{(2.490,2.223)}
\gppoint{gp mark 7}{(2.503,2.223)}
\gppoint{gp mark 7}{(2.516,2.223)}
\gppoint{gp mark 7}{(2.529,2.224)}
\gppoint{gp mark 7}{(2.541,2.227)}
\gppoint{gp mark 7}{(2.554,2.229)}
\gppoint{gp mark 7}{(2.567,2.231)}
\gppoint{gp mark 7}{(2.579,2.232)}
\gppoint{gp mark 7}{(2.592,2.232)}
\gppoint{gp mark 7}{(2.605,2.232)}
\gppoint{gp mark 7}{(2.617,2.232)}
\gppoint{gp mark 7}{(2.630,2.233)}
\gppoint{gp mark 7}{(2.643,2.233)}
\gppoint{gp mark 7}{(2.655,2.233)}
\gppoint{gp mark 7}{(2.668,2.234)}
\gppoint{gp mark 7}{(2.681,2.234)}
\gppoint{gp mark 7}{(2.693,2.234)}
\gppoint{gp mark 7}{(2.706,2.234)}
\gppoint{gp mark 7}{(2.719,2.234)}
\gppoint{gp mark 7}{(2.731,2.234)}
\gppoint{gp mark 7}{(2.744,2.234)}
\gppoint{gp mark 7}{(2.757,2.233)}
\gppoint{gp mark 7}{(2.769,2.233)}
\gppoint{gp mark 7}{(2.782,2.233)}
\gppoint{gp mark 7}{(2.795,2.233)}
\gppoint{gp mark 7}{(2.807,2.233)}
\gppoint{gp mark 7}{(2.820,2.233)}
\gppoint{gp mark 7}{(2.833,2.233)}
\gppoint{gp mark 7}{(2.845,2.235)}
\gppoint{gp mark 7}{(2.858,2.256)}
\gppoint{gp mark 7}{(2.871,2.373)}
\gppoint{gp mark 7}{(2.883,2.283)}
\gppoint{gp mark 7}{(2.896,2.167)}
\gppoint{gp mark 7}{(2.909,2.232)}
\gppoint{gp mark 7}{(2.922,2.213)}
\gppoint{gp mark 7}{(2.934,2.219)}
\gppoint{gp mark 7}{(2.947,2.227)}
\gppoint{gp mark 7}{(2.960,2.227)}
\gppoint{gp mark 7}{(2.972,2.227)}
\gppoint{gp mark 7}{(2.985,2.228)}
\gppoint{gp mark 7}{(2.998,2.229)}
\gppoint{gp mark 7}{(3.010,2.227)}
\gppoint{gp mark 7}{(3.023,2.227)}
\gppoint{gp mark 7}{(3.036,2.227)}
\gppoint{gp mark 7}{(3.048,2.227)}
\gppoint{gp mark 7}{(3.061,2.226)}
\gppoint{gp mark 7}{(3.074,2.225)}
\gppoint{gp mark 7}{(3.086,2.226)}
\gppoint{gp mark 7}{(3.099,2.227)}
\gppoint{gp mark 7}{(3.112,2.228)}
\gppoint{gp mark 7}{(3.124,2.227)}
\gppoint{gp mark 7}{(3.137,2.227)}
\gppoint{gp mark 7}{(3.150,2.227)}
\gppoint{gp mark 7}{(3.162,2.227)}
\gppoint{gp mark 7}{(3.175,2.228)}
\gppoint{gp mark 7}{(3.188,2.228)}
\gppoint{gp mark 7}{(3.200,2.227)}
\gppoint{gp mark 7}{(3.213,2.225)}
\gppoint{gp mark 7}{(3.226,2.225)}
\gppoint{gp mark 7}{(3.238,2.225)}
\gppoint{gp mark 7}{(3.251,2.225)}
\gppoint{gp mark 7}{(3.264,2.226)}
\gppoint{gp mark 7}{(3.277,2.226)}
\gppoint{gp mark 7}{(3.289,2.226)}
\gppoint{gp mark 7}{(3.302,2.227)}
\gppoint{gp mark 7}{(3.315,2.227)}
\gppoint{gp mark 7}{(3.327,2.227)}
\gppoint{gp mark 7}{(3.340,2.227)}
\gppoint{gp mark 7}{(3.353,2.227)}
\gppoint{gp mark 7}{(3.365,2.226)}
\gppoint{gp mark 7}{(3.378,2.225)}
\gppoint{gp mark 7}{(3.391,2.225)}
\gppoint{gp mark 7}{(3.403,2.224)}
\gppoint{gp mark 7}{(3.416,2.224)}
\gppoint{gp mark 7}{(3.429,2.224)}
\gppoint{gp mark 7}{(3.441,2.226)}
\gppoint{gp mark 7}{(3.454,2.228)}
\gppoint{gp mark 7}{(3.467,2.229)}
\gppoint{gp mark 7}{(3.479,2.229)}
\gppoint{gp mark 7}{(3.492,2.229)}
\gppoint{gp mark 7}{(3.505,2.228)}
\gppoint{gp mark 7}{(3.517,2.227)}
\gppoint{gp mark 7}{(3.530,2.226)}
\gppoint{gp mark 7}{(3.543,2.225)}
\gppoint{gp mark 7}{(3.555,2.225)}
\gppoint{gp mark 7}{(3.568,2.225)}
\gppoint{gp mark 7}{(3.581,2.225)}
\gppoint{gp mark 7}{(3.593,2.226)}
\gppoint{gp mark 7}{(3.606,2.228)}
\gppoint{gp mark 7}{(3.619,2.229)}
\gppoint{gp mark 7}{(3.632,2.229)}
\gppoint{gp mark 7}{(3.644,2.230)}
\gppoint{gp mark 7}{(3.657,2.229)}
\gppoint{gp mark 7}{(3.670,2.229)}
\gppoint{gp mark 7}{(3.682,2.227)}
\gppoint{gp mark 7}{(3.695,2.226)}
\gppoint{gp mark 7}{(3.708,2.225)}
\gppoint{gp mark 7}{(3.720,2.225)}
\gppoint{gp mark 7}{(3.733,2.226)}
\gppoint{gp mark 7}{(3.746,2.227)}
\gppoint{gp mark 7}{(3.758,2.230)}
\gppoint{gp mark 7}{(3.771,2.242)}
\gppoint{gp mark 7}{(3.784,2.362)}
\gppoint{gp mark 7}{(3.796,3.145)}
\gppoint{gp mark 7}{(3.809,4.421)}
\gppoint{gp mark 7}{(3.822,4.660)}
\gppoint{gp mark 7}{(3.834,4.681)}
\gppoint{gp mark 7}{(3.847,4.679)}
\gppoint{gp mark 7}{(3.860,4.677)}
\gppoint{gp mark 7}{(3.872,4.678)}
\gppoint{gp mark 7}{(3.885,4.679)}
\gppoint{gp mark 7}{(3.898,4.679)}
\gppoint{gp mark 7}{(3.910,4.678)}
\gppoint{gp mark 7}{(3.923,4.677)}
\gppoint{gp mark 7}{(3.936,4.677)}
\gppoint{gp mark 7}{(3.948,4.677)}
\gppoint{gp mark 7}{(3.961,4.678)}
\gppoint{gp mark 7}{(3.974,4.678)}
\gppoint{gp mark 7}{(3.987,4.678)}
\gppoint{gp mark 7}{(3.999,4.678)}
\gppoint{gp mark 7}{(4.012,4.678)}
\gppoint{gp mark 7}{(4.025,4.678)}
\gppoint{gp mark 7}{(4.037,4.678)}
\gppoint{gp mark 7}{(4.050,4.678)}
\gppoint{gp mark 7}{(4.063,4.678)}
\gppoint{gp mark 7}{(4.075,4.678)}
\gppoint{gp mark 7}{(4.088,4.678)}
\gppoint{gp mark 7}{(4.101,4.678)}
\gppoint{gp mark 7}{(4.113,4.678)}
\gppoint{gp mark 7}{(4.126,4.677)}
\gppoint{gp mark 7}{(4.139,4.678)}
\gppoint{gp mark 7}{(4.151,4.678)}
\gppoint{gp mark 7}{(4.164,4.679)}
\gppoint{gp mark 7}{(4.177,4.679)}
\gppoint{gp mark 7}{(4.189,4.678)}
\gppoint{gp mark 7}{(4.202,4.678)}
\gppoint{gp mark 7}{(4.215,4.678)}
\gppoint{gp mark 7}{(4.227,4.678)}
\gppoint{gp mark 7}{(4.240,4.678)}
\gppoint{gp mark 7}{(4.253,4.679)}
\gppoint{gp mark 7}{(4.265,4.679)}
\gppoint{gp mark 7}{(4.278,4.679)}
\gppoint{gp mark 7}{(4.291,4.679)}
\gppoint{gp mark 7}{(4.303,4.679)}
\gppoint{gp mark 7}{(4.316,4.679)}
\gppoint{gp mark 7}{(4.329,4.678)}
\gppoint{gp mark 7}{(4.342,4.678)}
\gppoint{gp mark 7}{(4.354,4.678)}
\gppoint{gp mark 7}{(4.367,4.679)}
\gppoint{gp mark 7}{(4.380,4.679)}
\gppoint{gp mark 7}{(4.392,4.679)}
\gppoint{gp mark 7}{(4.405,4.679)}
\gppoint{gp mark 7}{(4.418,4.678)}
\gppoint{gp mark 7}{(4.430,4.678)}
\gppoint{gp mark 7}{(4.443,4.678)}
\gppoint{gp mark 7}{(4.456,4.678)}
\gppoint{gp mark 7}{(4.468,4.678)}
\gppoint{gp mark 7}{(4.481,4.678)}
\gppoint{gp mark 7}{(4.494,4.678)}
\gppoint{gp mark 7}{(4.506,4.678)}
\gppoint{gp mark 7}{(4.519,4.677)}
\gppoint{gp mark 7}{(4.532,4.677)}
\gppoint{gp mark 7}{(4.544,4.678)}
\gppoint{gp mark 7}{(4.557,4.678)}
\gppoint{gp mark 7}{(4.570,4.679)}
\gppoint{gp mark 7}{(4.582,4.679)}
\gppoint{gp mark 7}{(4.595,4.679)}
\gppoint{gp mark 7}{(4.608,4.678)}
\gppoint{gp mark 7}{(4.620,4.678)}
\gppoint{gp mark 7}{(4.633,4.678)}
\gppoint{gp mark 7}{(4.646,4.678)}
\gppoint{gp mark 7}{(4.658,4.678)}
\gppoint{gp mark 7}{(4.671,4.678)}
\gppoint{gp mark 7}{(4.684,4.678)}
\gppoint{gp mark 7}{(4.697,4.678)}
\gppoint{gp mark 7}{(4.709,4.678)}
\gppoint{gp mark 7}{(4.722,4.678)}
\gppoint{gp mark 7}{(4.735,4.679)}
\gppoint{gp mark 7}{(4.747,4.679)}
\gppoint{gp mark 7}{(4.760,4.678)}
\gppoint{gp mark 7}{(4.773,4.678)}
\gppoint{gp mark 7}{(4.785,4.678)}
\gppoint{gp mark 7}{(4.798,4.678)}
\gppoint{gp mark 7}{(4.811,4.678)}
\gppoint{gp mark 7}{(4.823,4.678)}
\gppoint{gp mark 7}{(4.836,4.678)}
\gppoint{gp mark 7}{(4.849,4.678)}
\gppoint{gp mark 7}{(4.861,4.678)}
\gppoint{gp mark 7}{(4.874,4.678)}
\gppoint{gp mark 7}{(4.887,4.678)}
\gppoint{gp mark 7}{(4.899,4.678)}
\gppoint{gp mark 7}{(4.912,4.678)}
\gppoint{gp mark 7}{(4.925,4.678)}
\gppoint{gp mark 7}{(4.937,4.678)}
\gppoint{gp mark 7}{(4.950,4.678)}
\gppoint{gp mark 7}{(4.963,4.678)}
\gppoint{gp mark 7}{(4.975,4.678)}
\gppoint{gp mark 7}{(4.988,4.678)}
\gppoint{gp mark 7}{(5.001,4.678)}
\gppoint{gp mark 7}{(5.013,4.678)}
\gppoint{gp mark 7}{(5.026,4.678)}
\gppoint{gp mark 7}{(5.039,4.678)}
\gppoint{gp mark 7}{(5.052,4.678)}
\gppoint{gp mark 7}{(5.064,4.678)}
\gppoint{gp mark 7}{(5.077,4.678)}
\gppoint{gp mark 7}{(5.090,4.678)}
\gppoint{gp mark 7}{(5.102,4.678)}
\gppoint{gp mark 7}{(5.115,4.678)}
\gppoint{gp mark 7}{(5.128,4.678)}
\gppoint{gp mark 7}{(5.140,4.678)}
\gppoint{gp mark 7}{(5.153,4.678)}
\gppoint{gp mark 7}{(5.166,4.678)}
\gppoint{gp mark 7}{(5.178,4.678)}
\gppoint{gp mark 7}{(5.191,4.678)}
\gppoint{gp mark 7}{(5.204,4.678)}
\gppoint{gp mark 7}{(5.216,4.678)}
\gppoint{gp mark 7}{(5.229,4.678)}
\gppoint{gp mark 7}{(5.242,4.678)}
\gppoint{gp mark 7}{(5.254,4.678)}
\gppoint{gp mark 7}{(5.267,4.678)}
\gppoint{gp mark 7}{(5.280,4.678)}
\gppoint{gp mark 7}{(5.292,4.678)}
\gppoint{gp mark 7}{(5.305,4.678)}
\gppoint{gp mark 7}{(5.318,4.678)}
\gppoint{gp mark 7}{(5.330,4.678)}
\gppoint{gp mark 7}{(5.343,4.678)}
\gppoint{gp mark 7}{(5.356,4.678)}
\gppoint{gp mark 7}{(5.368,4.678)}
\gppoint{gp mark 7}{(5.381,4.678)}
\gppoint{gp mark 7}{(5.394,4.678)}
\gppoint{gp mark 7}{(5.407,4.678)}
\gppoint{gp mark 7}{(5.419,4.678)}
\gppoint{gp mark 7}{(5.432,4.678)}
\gppoint{gp mark 7}{(5.445,4.678)}
\gppoint{gp mark 7}{(5.457,4.678)}
\gppoint{gp mark 7}{(5.470,4.678)}
\gppoint{gp mark 7}{(5.483,4.678)}
\gppoint{gp mark 7}{(5.495,4.678)}
\gppoint{gp mark 7}{(5.508,4.678)}
\gppoint{gp mark 7}{(5.521,4.678)}
\gppoint{gp mark 7}{(5.533,4.678)}
\gppoint{gp mark 7}{(5.546,4.678)}
\gppoint{gp mark 7}{(5.559,4.679)}
\gppoint{gp mark 7}{(5.571,4.679)}
\gppoint{gp mark 7}{(5.584,4.679)}
\gppoint{gp mark 7}{(5.597,4.678)}
\gppoint{gp mark 7}{(5.609,4.678)}
\gppoint{gp mark 7}{(5.622,4.678)}
\gppoint{gp mark 7}{(5.635,4.678)}
\gppoint{gp mark 7}{(5.647,4.678)}
\gppoint{gp mark 7}{(5.660,4.678)}
\gppoint{gp mark 7}{(5.673,4.678)}
\gppoint{gp mark 7}{(5.685,4.678)}
\gppoint{gp mark 7}{(5.698,4.678)}
\gppoint{gp mark 7}{(5.711,4.678)}
\gppoint{gp mark 7}{(5.723,4.678)}
\gppoint{gp mark 7}{(5.736,4.678)}
\gppoint{gp mark 7}{(5.749,4.678)}
\gppoint{gp mark 7}{(5.761,4.678)}
\gppoint{gp mark 7}{(5.774,4.678)}
\gppoint{gp mark 7}{(5.787,4.679)}
\gppoint{gp mark 7}{(5.800,4.679)}
\gppoint{gp mark 7}{(5.812,4.679)}
\gppoint{gp mark 7}{(5.825,4.679)}
\gppoint{gp mark 7}{(5.838,4.678)}
\gppoint{gp mark 7}{(5.850,4.678)}
\gppoint{gp mark 7}{(5.863,4.678)}
\gppoint{gp mark 7}{(5.876,4.678)}
\gppoint{gp mark 7}{(5.888,4.678)}
\gppoint{gp mark 7}{(5.901,4.678)}
\gppoint{gp mark 7}{(5.914,4.678)}
\gppoint{gp mark 7}{(5.926,4.679)}
\gppoint{gp mark 7}{(5.939,4.678)}
\gppoint{gp mark 7}{(5.952,4.678)}
\gppoint{gp mark 7}{(5.964,4.678)}
\gppoint{gp mark 7}{(5.977,4.678)}
\gppoint{gp mark 7}{(5.990,4.678)}
\gppoint{gp mark 7}{(6.002,4.678)}
\gppoint{gp mark 7}{(6.015,4.678)}
\gppoint{gp mark 7}{(6.028,4.678)}
\gppoint{gp mark 7}{(6.040,4.678)}
\gppoint{gp mark 7}{(6.053,4.678)}
\gppoint{gp mark 7}{(6.066,4.678)}
\gppoint{gp mark 7}{(6.078,4.678)}
\gppoint{gp mark 7}{(6.091,4.678)}
\gppoint{gp mark 7}{(6.104,4.678)}
\gppoint{gp mark 7}{(6.116,4.678)}
\gppoint{gp mark 7}{(6.129,4.678)}
\gppoint{gp mark 7}{(6.142,4.678)}
\gppoint{gp mark 7}{(6.155,4.678)}
\gppoint{gp mark 7}{(6.167,4.678)}
\gppoint{gp mark 7}{(6.180,4.678)}
\gppoint{gp mark 7}{(6.193,4.678)}
\gppoint{gp mark 7}{(6.205,4.678)}
\gppoint{gp mark 7}{(6.218,4.678)}
\gppoint{gp mark 7}{(6.231,4.678)}
\gppoint{gp mark 7}{(6.243,4.678)}
\gppoint{gp mark 7}{(6.256,4.678)}
\gppoint{gp mark 7}{(6.269,4.678)}
\gppoint{gp mark 7}{(6.281,4.678)}
\gppoint{gp mark 7}{(6.294,4.678)}
\gppoint{gp mark 7}{(6.307,4.678)}
\gppoint{gp mark 7}{(6.319,4.678)}
\gppoint{gp mark 7}{(6.332,4.678)}
\gppoint{gp mark 7}{(6.345,4.678)}
\gppoint{gp mark 7}{(6.357,4.677)}
\gppoint{gp mark 7}{(6.370,4.677)}
\gppoint{gp mark 7}{(6.383,4.677)}
\gppoint{gp mark 7}{(6.395,4.677)}
\gppoint{gp mark 7}{(6.408,4.677)}
\gppoint{gp mark 7}{(6.421,4.677)}
\gppoint{gp mark 7}{(6.433,4.677)}
\gppoint{gp mark 7}{(6.446,4.677)}
\gppoint{gp mark 7}{(6.459,4.677)}
\gppoint{gp mark 7}{(6.471,4.677)}
\gppoint{gp mark 7}{(6.484,4.677)}
\gppoint{gp mark 7}{(6.497,4.676)}
\gppoint{gp mark 7}{(6.510,4.676)}
\gppoint{gp mark 7}{(6.522,4.676)}
\gppoint{gp mark 7}{(6.535,4.676)}
\gppoint{gp mark 7}{(6.548,4.676)}
\gppoint{gp mark 7}{(6.560,4.676)}
\gppoint{gp mark 7}{(6.573,4.675)}
\gppoint{gp mark 7}{(6.586,4.675)}
\gppoint{gp mark 7}{(6.598,4.675)}
\gppoint{gp mark 7}{(6.611,4.675)}
\gppoint{gp mark 7}{(6.624,4.675)}
\gppoint{gp mark 7}{(6.636,4.675)}
\gppoint{gp mark 7}{(6.649,4.675)}
\gppoint{gp mark 7}{(6.662,4.675)}
\gppoint{gp mark 7}{(6.674,4.675)}
\gppoint{gp mark 7}{(6.687,4.675)}
\gppoint{gp mark 7}{(6.700,4.675)}
\gppoint{gp mark 7}{(6.712,4.674)}
\gppoint{gp mark 7}{(6.725,4.674)}
\gppoint{gp mark 7}{(6.738,4.674)}
\gppoint{gp mark 7}{(6.750,4.673)}
\gppoint{gp mark 7}{(6.763,4.673)}
\gppoint{gp mark 7}{(6.776,4.673)}
\gppoint{gp mark 7}{(6.788,4.673)}
\gppoint{gp mark 7}{(6.801,4.673)}
\gppoint{gp mark 7}{(6.814,4.673)}
\gppoint{gp mark 7}{(6.826,4.673)}
\gppoint{gp mark 7}{(6.839,4.673)}
\gppoint{gp mark 7}{(6.852,4.673)}
\gppoint{gp mark 7}{(6.865,4.673)}
\gppoint{gp mark 7}{(6.877,4.673)}
\gppoint{gp mark 7}{(6.890,4.673)}
\gppoint{gp mark 7}{(6.903,4.673)}
\gppoint{gp mark 7}{(6.915,4.673)}
\gppoint{gp mark 7}{(6.928,4.673)}
\gppoint{gp mark 7}{(6.941,4.673)}
\gppoint{gp mark 7}{(6.953,4.673)}
\gppoint{gp mark 7}{(6.966,4.673)}
\gppoint{gp mark 7}{(6.979,4.673)}
\gppoint{gp mark 7}{(6.991,4.673)}
\gppoint{gp mark 7}{(7.004,4.673)}
\gppoint{gp mark 7}{(7.017,4.674)}
\gppoint{gp mark 7}{(7.029,4.674)}
\gppoint{gp mark 7}{(7.042,4.674)}
\gppoint{gp mark 7}{(7.055,4.674)}
\gppoint{gp mark 7}{(7.067,4.674)}
\gppoint{gp mark 7}{(7.080,4.674)}
\gppoint{gp mark 7}{(7.093,4.675)}
\gppoint{gp mark 7}{(7.105,4.675)}
\gppoint{gp mark 7}{(7.118,4.675)}
\gppoint{gp mark 7}{(7.131,4.676)}
\gppoint{gp mark 7}{(7.143,4.676)}
\gppoint{gp mark 7}{(7.156,4.677)}
\gppoint{gp mark 7}{(7.169,4.678)}
\gppoint{gp mark 7}{(7.181,4.679)}
\gppoint{gp mark 7}{(7.194,4.680)}
\gppoint{gp mark 7}{(7.207,4.680)}
\gppoint{gp mark 7}{(7.220,4.680)}
\gppoint{gp mark 7}{(7.232,4.682)}
\gppoint{gp mark 7}{(7.245,4.685)}
\gppoint{gp mark 7}{(7.258,4.688)}
\gppoint{gp mark 7}{(7.270,4.690)}
\gppoint{gp mark 7}{(7.283,4.691)}
\gppoint{gp mark 7}{(7.296,4.692)}
\gppoint{gp mark 7}{(7.308,4.693)}
\gppoint{gp mark 7}{(7.321,4.695)}
\gppoint{gp mark 7}{(7.334,4.698)}
\gppoint{gp mark 7}{(7.346,4.700)}
\gppoint{gp mark 7}{(7.359,4.702)}
\gppoint{gp mark 7}{(7.372,4.703)}
\gppoint{gp mark 7}{(7.384,4.703)}
\gppoint{gp mark 7}{(7.397,4.703)}
\gppoint{gp mark 7}{(7.410,4.703)}
\gppoint{gp mark 7}{(7.422,4.703)}
\gppoint{gp mark 7}{(7.435,4.701)}
\gppoint{gp mark 7}{(7.448,4.698)}
\gppoint{gp mark 7}{(7.460,4.693)}
\gppoint{gp mark 7}{(7.473,4.689)}
\gppoint{gp mark 7}{(7.486,4.685)}
\gppoint{gp mark 7}{(7.498,4.680)}
\gppoint{gp mark 7}{(7.511,4.676)}
\gppoint{gp mark 7}{(7.524,4.672)}
\gppoint{gp mark 7}{(7.536,4.666)}
\gppoint{gp mark 7}{(7.549,4.657)}
\gppoint{gp mark 7}{(7.562,4.640)}
\gppoint{gp mark 7}{(7.575,4.617)}
\gppoint{gp mark 7}{(7.587,4.596)}
\gppoint{gp mark 7}{(7.600,4.582)}
\gppoint{gp mark 7}{(7.613,4.577)}
\gppoint{gp mark 7}{(7.625,4.575)}
\gppoint{gp mark 7}{(7.638,4.575)}
\gppoint{gp mark 7}{(7.651,4.573)}
\gppoint{gp mark 7}{(7.663,4.563)}
\gppoint{gp mark 7}{(7.676,4.490)}
\gppoint{gp mark 7}{(7.689,4.127)}
\gppoint{gp mark 7}{(7.701,3.401)}
\gppoint{gp mark 7}{(7.714,2.560)}
\gppoint{gp mark 7}{(7.727,1.885)}
\gppoint{gp mark 7}{(7.739,1.473)}
\gppoint{gp mark 7}{(7.752,1.269)}
\gppoint{gp mark 7}{(7.765,1.184)}
\gppoint{gp mark 7}{(7.777,1.154)}
\gppoint{gp mark 7}{(7.790,1.145)}
\gppoint{gp mark 7}{(7.803,1.144)}
\gppoint{gp mark 7}{(7.815,1.144)}
\gppoint{gp mark 7}{(7.828,1.144)}
\gppoint{gp mark 7}{(7.841,1.146)}
\gppoint{gp mark 7}{(7.853,1.149)}
\gppoint{gp mark 7}{(7.866,1.152)}
\gppoint{gp mark 7}{(7.879,1.153)}
\gppoint{gp mark 7}{(7.891,1.154)}
\gppoint{gp mark 7}{(7.904,1.154)}
\gppoint{gp mark 7}{(7.917,1.154)}
\gppoint{gp mark 7}{(7.930,1.154)}
\gpcolor{rgb color={0.000,0.000,0.000}}
\gpsetlinetype{gp lt plot 0}
\gpsetlinewidth{4.00}
\draw[gp path] (2.404,2.243)--(2.886,2.243);
\draw[gp path] (2.886,2.243)--(3.811,2.243);
\draw[gp path] (3.811,4.679)--(7.722,4.679);
\draw[gp path] (7.722,1.152)--(7.947,1.152);
\draw[gp path] (1.207,3.449)--(1.220,3.434)--(1.233,3.418)--(1.246,3.403)--(1.259,3.388)%
  --(1.271,3.373)--(1.284,3.358)--(1.297,3.343)--(1.310,3.328)--(1.323,3.313)--(1.336,3.298)%
  --(1.349,3.283)--(1.361,3.268)--(1.374,3.253)--(1.387,3.238)--(1.400,3.224)--(1.413,3.209)%
  --(1.426,3.194)--(1.439,3.180)--(1.452,3.165)--(1.464,3.151)--(1.477,3.136)--(1.490,3.122)%
  --(1.503,3.108)--(1.516,3.093)--(1.529,3.079)--(1.542,3.065)--(1.555,3.051)--(1.567,3.036)%
  --(1.580,3.022)--(1.593,3.008)--(1.606,2.994)--(1.619,2.980)--(1.632,2.967)--(1.645,2.953)%
  --(1.657,2.939)--(1.670,2.925)--(1.683,2.911)--(1.696,2.898)--(1.709,2.884)--(1.722,2.871)%
  --(1.735,2.857)--(1.748,2.844)--(1.760,2.830)--(1.773,2.817)--(1.786,2.803)--(1.799,2.790)%
  --(1.812,2.777)--(1.825,2.764)--(1.838,2.751)--(1.850,2.738)--(1.863,2.725)--(1.876,2.712)%
  --(1.889,2.699)--(1.902,2.686)--(1.915,2.673)--(1.928,2.660)--(1.941,2.647)--(1.953,2.635)%
  --(1.966,2.622)--(1.979,2.610)--(1.992,2.597)--(2.005,2.585)--(2.018,2.572)--(2.031,2.560)%
  --(2.044,2.548)--(2.056,2.535)--(2.069,2.523)--(2.082,2.511)--(2.095,2.499)--(2.108,2.487)%
  --(2.121,2.475)--(2.134,2.463)--(2.146,2.451)--(2.159,2.439)--(2.172,2.428)--(2.185,2.416)%
  --(2.198,2.404)--(2.211,2.393)--(2.224,2.381)--(2.237,2.370)--(2.249,2.358)--(2.262,2.347)%
  --(2.275,2.336)--(2.288,2.325)--(2.301,2.313)--(2.314,2.302)--(2.327,2.291)--(2.340,2.280)%
  --(2.352,2.269)--(2.365,2.258)--(2.378,2.248)--(2.391,2.237)--(2.404,2.243);
\draw[gp path] (3.811,2.243)--(3.811,4.679);
\draw[gp path] (7.722,4.679)--(7.722,1.152);
\node[gp node left,font={\fontsize{10pt}{12pt}\selectfont}] at (1.456,5.268) {\LARGE $\rho$};
\node[gp node left,font={\fontsize{10pt}{12pt}\selectfont}] at (5.740,5.268) {\large $\alpha = 3.0$};
%% coordinates of the plot area
\gpdefrectangularnode{gp plot 1}{\pgfpoint{1.196cm}{0.985cm}}{\pgfpoint{7.947cm}{5.631cm}}
\end{tikzpicture}
%% gnuplot variables
}
& 
\resizebox{0.5\linewidth}{!}{\tikzsetnextfilename{fast_coplanar_b_rsol_init_6} \begin{tikzpicture}[gnuplot]
%% generated with GNUPLOT 4.6p4 (Lua 5.1; terminal rev. 99, script rev. 100)
%% Sat 02 Aug 2014 10:10:57 AM EDT
\path (0.000,0.000) rectangle (8.500,6.000);
\gpfill{rgb color={1.000,1.000,1.000}} (1.196,0.985)--(7.946,0.985)--(7.946,5.630)--(1.196,5.630)--cycle;
\gpcolor{color=gp lt color border}
\gpsetlinetype{gp lt border}
\gpsetlinewidth{1.00}
\draw[gp path] (1.196,0.985)--(1.196,5.630)--(7.946,5.630)--(7.946,0.985)--cycle;
\gpcolor{color=gp lt color axes}
\gpsetlinetype{gp lt axes}
\gpsetlinewidth{2.00}
\draw[gp path] (1.196,0.985)--(7.947,0.985);
\gpcolor{color=gp lt color border}
\gpsetlinetype{gp lt border}
\draw[gp path] (1.196,0.985)--(1.268,0.985);
\draw[gp path] (7.947,0.985)--(7.875,0.985);
\gpcolor{rgb color={0.000,0.000,0.000}}
\node[gp node right,font={\fontsize{10pt}{12pt}\selectfont}] at (1.012,0.985) {-0.4};
\gpcolor{color=gp lt color axes}
\gpsetlinetype{gp lt axes}
\draw[gp path] (1.196,1.759)--(7.947,1.759);
\gpcolor{color=gp lt color border}
\gpsetlinetype{gp lt border}
\draw[gp path] (1.196,1.759)--(1.268,1.759);
\draw[gp path] (7.947,1.759)--(7.875,1.759);
\gpcolor{rgb color={0.000,0.000,0.000}}
\node[gp node right,font={\fontsize{10pt}{12pt}\selectfont}] at (1.012,1.759) {-0.2};
\gpcolor{color=gp lt color axes}
\gpsetlinetype{gp lt axes}
\draw[gp path] (1.196,2.534)--(7.947,2.534);
\gpcolor{color=gp lt color border}
\gpsetlinetype{gp lt border}
\draw[gp path] (1.196,2.534)--(1.268,2.534);
\draw[gp path] (7.947,2.534)--(7.875,2.534);
\gpcolor{rgb color={0.000,0.000,0.000}}
\node[gp node right,font={\fontsize{10pt}{12pt}\selectfont}] at (1.012,2.534) {0};
\gpcolor{color=gp lt color axes}
\gpsetlinetype{gp lt axes}
\draw[gp path] (1.196,3.308)--(7.947,3.308);
\gpcolor{color=gp lt color border}
\gpsetlinetype{gp lt border}
\draw[gp path] (1.196,3.308)--(1.268,3.308);
\draw[gp path] (7.947,3.308)--(7.875,3.308);
\gpcolor{rgb color={0.000,0.000,0.000}}
\node[gp node right,font={\fontsize{10pt}{12pt}\selectfont}] at (1.012,3.308) {0.2};
\gpcolor{color=gp lt color axes}
\gpsetlinetype{gp lt axes}
\draw[gp path] (1.196,4.082)--(7.947,4.082);
\gpcolor{color=gp lt color border}
\gpsetlinetype{gp lt border}
\draw[gp path] (1.196,4.082)--(1.268,4.082);
\draw[gp path] (7.947,4.082)--(7.875,4.082);
\gpcolor{rgb color={0.000,0.000,0.000}}
\node[gp node right,font={\fontsize{10pt}{12pt}\selectfont}] at (1.012,4.082) {0.4};
\gpcolor{color=gp lt color axes}
\gpsetlinetype{gp lt axes}
\draw[gp path] (1.196,4.857)--(7.947,4.857);
\gpcolor{color=gp lt color border}
\gpsetlinetype{gp lt border}
\draw[gp path] (1.196,4.857)--(1.268,4.857);
\draw[gp path] (7.947,4.857)--(7.875,4.857);
\gpcolor{rgb color={0.000,0.000,0.000}}
\node[gp node right,font={\fontsize{10pt}{12pt}\selectfont}] at (1.012,4.857) {0.6};
\gpcolor{color=gp lt color axes}
\gpsetlinetype{gp lt axes}
\draw[gp path] (1.196,5.631)--(7.947,5.631);
\gpcolor{color=gp lt color border}
\gpsetlinetype{gp lt border}
\draw[gp path] (1.196,5.631)--(1.268,5.631);
\draw[gp path] (7.947,5.631)--(7.875,5.631);
\gpcolor{rgb color={0.000,0.000,0.000}}
\node[gp node right,font={\fontsize{10pt}{12pt}\selectfont}] at (1.012,5.631) {0.8};
\gpcolor{color=gp lt color axes}
\gpsetlinetype{gp lt axes}
\draw[gp path] (1.196,0.985)--(1.196,5.631);
\gpcolor{color=gp lt color border}
\gpsetlinetype{gp lt border}
\draw[gp path] (1.196,0.985)--(1.196,1.057);
\draw[gp path] (1.196,5.631)--(1.196,5.559);
\gpcolor{rgb color={0.000,0.000,0.000}}
\node[gp node center,font={\fontsize{10pt}{12pt}\selectfont}] at (1.196,0.677) {0.3};
\gpcolor{color=gp lt color axes}
\gpsetlinetype{gp lt axes}
\draw[gp path] (2.494,0.985)--(2.494,5.631);
\gpcolor{color=gp lt color border}
\gpsetlinetype{gp lt border}
\draw[gp path] (2.494,0.985)--(2.494,1.057);
\draw[gp path] (2.494,5.631)--(2.494,5.559);
\gpcolor{rgb color={0.000,0.000,0.000}}
\node[gp node center,font={\fontsize{10pt}{12pt}\selectfont}] at (2.494,0.677) {0.35};
\gpcolor{color=gp lt color axes}
\gpsetlinetype{gp lt axes}
\draw[gp path] (3.793,0.985)--(3.793,5.631);
\gpcolor{color=gp lt color border}
\gpsetlinetype{gp lt border}
\draw[gp path] (3.793,0.985)--(3.793,1.057);
\draw[gp path] (3.793,5.631)--(3.793,5.559);
\gpcolor{rgb color={0.000,0.000,0.000}}
\node[gp node center,font={\fontsize{10pt}{12pt}\selectfont}] at (3.793,0.677) {0.4};
\gpcolor{color=gp lt color axes}
\gpsetlinetype{gp lt axes}
\draw[gp path] (5.091,0.985)--(5.091,5.631);
\gpcolor{color=gp lt color border}
\gpsetlinetype{gp lt border}
\draw[gp path] (5.091,0.985)--(5.091,1.057);
\draw[gp path] (5.091,5.631)--(5.091,5.559);
\gpcolor{rgb color={0.000,0.000,0.000}}
\node[gp node center,font={\fontsize{10pt}{12pt}\selectfont}] at (5.091,0.677) {0.45};
\gpcolor{color=gp lt color axes}
\gpsetlinetype{gp lt axes}
\draw[gp path] (6.389,0.985)--(6.389,5.631);
\gpcolor{color=gp lt color border}
\gpsetlinetype{gp lt border}
\draw[gp path] (6.389,0.985)--(6.389,1.057);
\draw[gp path] (6.389,5.631)--(6.389,5.559);
\gpcolor{rgb color={0.000,0.000,0.000}}
\node[gp node center,font={\fontsize{10pt}{12pt}\selectfont}] at (6.389,0.677) {0.5};
\gpcolor{color=gp lt color axes}
\gpsetlinetype{gp lt axes}
\draw[gp path] (7.687,0.985)--(7.687,5.631);
\gpcolor{color=gp lt color border}
\gpsetlinetype{gp lt border}
\draw[gp path] (7.687,0.985)--(7.687,1.057);
\draw[gp path] (7.687,5.631)--(7.687,5.559);
\gpcolor{rgb color={0.000,0.000,0.000}}
\node[gp node center,font={\fontsize{10pt}{12pt}\selectfont}] at (7.687,0.677) {0.55};
\gpcolor{color=gp lt color border}
\draw[gp path] (1.196,5.631)--(1.196,0.985)--(7.947,0.985)--(7.947,5.631)--cycle;
\gpcolor{rgb color={0.000,0.000,0.000}}
\node[gp node center,font={\fontsize{10pt}{12pt}\selectfont}] at (4.571,0.215) {\large $x$};
\gpcolor{rgb color={1.000,0.000,0.000}}
\gpsetlinewidth{0.50}
\gpsetpointsize{4.44}
\gppoint{gp mark 7}{(1.210,4.952)}
\gppoint{gp mark 7}{(1.223,4.943)}
\gppoint{gp mark 7}{(1.235,4.933)}
\gppoint{gp mark 7}{(1.248,4.924)}
\gppoint{gp mark 7}{(1.261,4.914)}
\gppoint{gp mark 7}{(1.273,4.905)}
\gppoint{gp mark 7}{(1.286,4.895)}
\gppoint{gp mark 7}{(1.299,4.886)}
\gppoint{gp mark 7}{(1.311,4.876)}
\gppoint{gp mark 7}{(1.324,4.866)}
\gppoint{gp mark 7}{(1.337,4.857)}
\gppoint{gp mark 7}{(1.349,4.847)}
\gppoint{gp mark 7}{(1.362,4.837)}
\gppoint{gp mark 7}{(1.375,4.828)}
\gppoint{gp mark 7}{(1.387,4.818)}
\gppoint{gp mark 7}{(1.400,4.808)}
\gppoint{gp mark 7}{(1.413,4.799)}
\gppoint{gp mark 7}{(1.425,4.789)}
\gppoint{gp mark 7}{(1.438,4.779)}
\gppoint{gp mark 7}{(1.451,4.769)}
\gppoint{gp mark 7}{(1.464,4.759)}
\gppoint{gp mark 7}{(1.476,4.750)}
\gppoint{gp mark 7}{(1.489,4.740)}
\gppoint{gp mark 7}{(1.502,4.730)}
\gppoint{gp mark 7}{(1.514,4.720)}
\gppoint{gp mark 7}{(1.527,4.710)}
\gppoint{gp mark 7}{(1.540,4.700)}
\gppoint{gp mark 7}{(1.552,4.690)}
\gppoint{gp mark 7}{(1.565,4.680)}
\gppoint{gp mark 7}{(1.578,4.670)}
\gppoint{gp mark 7}{(1.590,4.660)}
\gppoint{gp mark 7}{(1.603,4.650)}
\gppoint{gp mark 7}{(1.616,4.640)}
\gppoint{gp mark 7}{(1.628,4.630)}
\gppoint{gp mark 7}{(1.641,4.620)}
\gppoint{gp mark 7}{(1.654,4.610)}
\gppoint{gp mark 7}{(1.666,4.600)}
\gppoint{gp mark 7}{(1.679,4.590)}
\gppoint{gp mark 7}{(1.692,4.579)}
\gppoint{gp mark 7}{(1.704,4.569)}
\gppoint{gp mark 7}{(1.717,4.559)}
\gppoint{gp mark 7}{(1.730,4.549)}
\gppoint{gp mark 7}{(1.742,4.538)}
\gppoint{gp mark 7}{(1.755,4.528)}
\gppoint{gp mark 7}{(1.768,4.518)}
\gppoint{gp mark 7}{(1.780,4.507)}
\gppoint{gp mark 7}{(1.793,4.497)}
\gppoint{gp mark 7}{(1.806,4.486)}
\gppoint{gp mark 7}{(1.819,4.476)}
\gppoint{gp mark 7}{(1.831,4.465)}
\gppoint{gp mark 7}{(1.844,4.455)}
\gppoint{gp mark 7}{(1.857,4.444)}
\gppoint{gp mark 7}{(1.869,4.434)}
\gppoint{gp mark 7}{(1.882,4.423)}
\gppoint{gp mark 7}{(1.895,4.412)}
\gppoint{gp mark 7}{(1.907,4.402)}
\gppoint{gp mark 7}{(1.920,4.391)}
\gppoint{gp mark 7}{(1.933,4.380)}
\gppoint{gp mark 7}{(1.945,4.370)}
\gppoint{gp mark 7}{(1.958,4.359)}
\gppoint{gp mark 7}{(1.971,4.348)}
\gppoint{gp mark 7}{(1.983,4.337)}
\gppoint{gp mark 7}{(1.996,4.326)}
\gppoint{gp mark 7}{(2.009,4.315)}
\gppoint{gp mark 7}{(2.021,4.304)}
\gppoint{gp mark 7}{(2.034,4.293)}
\gppoint{gp mark 7}{(2.047,4.282)}
\gppoint{gp mark 7}{(2.059,4.271)}
\gppoint{gp mark 7}{(2.072,4.260)}
\gppoint{gp mark 7}{(2.085,4.249)}
\gppoint{gp mark 7}{(2.097,4.237)}
\gppoint{gp mark 7}{(2.110,4.226)}
\gppoint{gp mark 7}{(2.123,4.215)}
\gppoint{gp mark 7}{(2.135,4.203)}
\gppoint{gp mark 7}{(2.148,4.192)}
\gppoint{gp mark 7}{(2.161,4.181)}
\gppoint{gp mark 7}{(2.174,4.169)}
\gppoint{gp mark 7}{(2.186,4.158)}
\gppoint{gp mark 7}{(2.199,4.146)}
\gppoint{gp mark 7}{(2.212,4.134)}
\gppoint{gp mark 7}{(2.224,4.123)}
\gppoint{gp mark 7}{(2.237,4.111)}
\gppoint{gp mark 7}{(2.250,4.099)}
\gppoint{gp mark 7}{(2.262,4.088)}
\gppoint{gp mark 7}{(2.275,4.076)}
\gppoint{gp mark 7}{(2.288,4.064)}
\gppoint{gp mark 7}{(2.300,4.052)}
\gppoint{gp mark 7}{(2.313,4.040)}
\gppoint{gp mark 7}{(2.326,4.028)}
\gppoint{gp mark 7}{(2.338,4.017)}
\gppoint{gp mark 7}{(2.351,4.005)}
\gppoint{gp mark 7}{(2.364,3.993)}
\gppoint{gp mark 7}{(2.376,3.981)}
\gppoint{gp mark 7}{(2.389,3.970)}
\gppoint{gp mark 7}{(2.402,3.959)}
\gppoint{gp mark 7}{(2.414,3.950)}
\gppoint{gp mark 7}{(2.427,3.942)}
\gppoint{gp mark 7}{(2.440,3.936)}
\gppoint{gp mark 7}{(2.452,3.933)}
\gppoint{gp mark 7}{(2.465,3.932)}
\gppoint{gp mark 7}{(2.478,3.932)}
\gppoint{gp mark 7}{(2.490,3.932)}
\gppoint{gp mark 7}{(2.503,3.932)}
\gppoint{gp mark 7}{(2.516,3.932)}
\gppoint{gp mark 7}{(2.529,3.933)}
\gppoint{gp mark 7}{(2.541,3.935)}
\gppoint{gp mark 7}{(2.554,3.937)}
\gppoint{gp mark 7}{(2.567,3.939)}
\gppoint{gp mark 7}{(2.579,3.940)}
\gppoint{gp mark 7}{(2.592,3.940)}
\gppoint{gp mark 7}{(2.605,3.940)}
\gppoint{gp mark 7}{(2.617,3.940)}
\gppoint{gp mark 7}{(2.630,3.940)}
\gppoint{gp mark 7}{(2.643,3.941)}
\gppoint{gp mark 7}{(2.655,3.941)}
\gppoint{gp mark 7}{(2.668,3.941)}
\gppoint{gp mark 7}{(2.681,3.941)}
\gppoint{gp mark 7}{(2.693,3.941)}
\gppoint{gp mark 7}{(2.706,3.941)}
\gppoint{gp mark 7}{(2.719,3.941)}
\gppoint{gp mark 7}{(2.731,3.941)}
\gppoint{gp mark 7}{(2.744,3.941)}
\gppoint{gp mark 7}{(2.757,3.941)}
\gppoint{gp mark 7}{(2.769,3.941)}
\gppoint{gp mark 7}{(2.782,3.941)}
\gppoint{gp mark 7}{(2.795,3.941)}
\gppoint{gp mark 7}{(2.807,3.941)}
\gppoint{gp mark 7}{(2.820,3.941)}
\gppoint{gp mark 7}{(2.833,3.941)}
\gppoint{gp mark 7}{(2.845,3.940)}
\gppoint{gp mark 7}{(2.858,3.936)}
\gppoint{gp mark 7}{(2.871,3.915)}
\gppoint{gp mark 7}{(2.883,3.319)}
\gppoint{gp mark 7}{(2.896,1.748)}
\gppoint{gp mark 7}{(2.909,1.263)}
\gppoint{gp mark 7}{(2.922,1.237)}
\gppoint{gp mark 7}{(2.934,1.237)}
\gppoint{gp mark 7}{(2.947,1.234)}
\gppoint{gp mark 7}{(2.960,1.230)}
\gppoint{gp mark 7}{(2.972,1.228)}
\gppoint{gp mark 7}{(2.985,1.227)}
\gppoint{gp mark 7}{(2.998,1.226)}
\gppoint{gp mark 7}{(3.010,1.226)}
\gppoint{gp mark 7}{(3.023,1.226)}
\gppoint{gp mark 7}{(3.036,1.225)}
\gppoint{gp mark 7}{(3.048,1.225)}
\gppoint{gp mark 7}{(3.061,1.226)}
\gppoint{gp mark 7}{(3.074,1.226)}
\gppoint{gp mark 7}{(3.086,1.226)}
\gppoint{gp mark 7}{(3.099,1.226)}
\gppoint{gp mark 7}{(3.112,1.226)}
\gppoint{gp mark 7}{(3.124,1.225)}
\gppoint{gp mark 7}{(3.137,1.225)}
\gppoint{gp mark 7}{(3.150,1.225)}
\gppoint{gp mark 7}{(3.162,1.225)}
\gppoint{gp mark 7}{(3.175,1.225)}
\gppoint{gp mark 7}{(3.188,1.224)}
\gppoint{gp mark 7}{(3.200,1.225)}
\gppoint{gp mark 7}{(3.213,1.225)}
\gppoint{gp mark 7}{(3.226,1.226)}
\gppoint{gp mark 7}{(3.238,1.227)}
\gppoint{gp mark 7}{(3.251,1.227)}
\gppoint{gp mark 7}{(3.264,1.226)}
\gppoint{gp mark 7}{(3.277,1.226)}
\gppoint{gp mark 7}{(3.289,1.225)}
\gppoint{gp mark 7}{(3.302,1.225)}
\gppoint{gp mark 7}{(3.315,1.225)}
\gppoint{gp mark 7}{(3.327,1.225)}
\gppoint{gp mark 7}{(3.340,1.225)}
\gppoint{gp mark 7}{(3.353,1.225)}
\gppoint{gp mark 7}{(3.365,1.225)}
\gppoint{gp mark 7}{(3.378,1.226)}
\gppoint{gp mark 7}{(3.391,1.226)}
\gppoint{gp mark 7}{(3.403,1.227)}
\gppoint{gp mark 7}{(3.416,1.227)}
\gppoint{gp mark 7}{(3.429,1.226)}
\gppoint{gp mark 7}{(3.441,1.226)}
\gppoint{gp mark 7}{(3.454,1.225)}
\gppoint{gp mark 7}{(3.467,1.224)}
\gppoint{gp mark 7}{(3.479,1.224)}
\gppoint{gp mark 7}{(3.492,1.225)}
\gppoint{gp mark 7}{(3.505,1.225)}
\gppoint{gp mark 7}{(3.517,1.225)}
\gppoint{gp mark 7}{(3.530,1.226)}
\gppoint{gp mark 7}{(3.543,1.226)}
\gppoint{gp mark 7}{(3.555,1.227)}
\gppoint{gp mark 7}{(3.568,1.227)}
\gppoint{gp mark 7}{(3.581,1.227)}
\gppoint{gp mark 7}{(3.593,1.226)}
\gppoint{gp mark 7}{(3.606,1.225)}
\gppoint{gp mark 7}{(3.619,1.224)}
\gppoint{gp mark 7}{(3.632,1.224)}
\gppoint{gp mark 7}{(3.644,1.224)}
\gppoint{gp mark 7}{(3.657,1.224)}
\gppoint{gp mark 7}{(3.670,1.224)}
\gppoint{gp mark 7}{(3.682,1.225)}
\gppoint{gp mark 7}{(3.695,1.227)}
\gppoint{gp mark 7}{(3.708,1.227)}
\gppoint{gp mark 7}{(3.720,1.227)}
\gppoint{gp mark 7}{(3.733,1.227)}
\gppoint{gp mark 7}{(3.746,1.226)}
\gppoint{gp mark 7}{(3.758,1.225)}
\gppoint{gp mark 7}{(3.771,1.226)}
\gppoint{gp mark 7}{(3.784,1.244)}
\gppoint{gp mark 7}{(3.796,1.373)}
\gppoint{gp mark 7}{(3.809,1.590)}
\gppoint{gp mark 7}{(3.822,1.631)}
\gppoint{gp mark 7}{(3.834,1.635)}
\gppoint{gp mark 7}{(3.847,1.635)}
\gppoint{gp mark 7}{(3.860,1.635)}
\gppoint{gp mark 7}{(3.872,1.635)}
\gppoint{gp mark 7}{(3.885,1.635)}
\gppoint{gp mark 7}{(3.898,1.635)}
\gppoint{gp mark 7}{(3.910,1.635)}
\gppoint{gp mark 7}{(3.923,1.635)}
\gppoint{gp mark 7}{(3.936,1.635)}
\gppoint{gp mark 7}{(3.948,1.635)}
\gppoint{gp mark 7}{(3.961,1.635)}
\gppoint{gp mark 7}{(3.974,1.635)}
\gppoint{gp mark 7}{(3.987,1.635)}
\gppoint{gp mark 7}{(3.999,1.635)}
\gppoint{gp mark 7}{(4.012,1.635)}
\gppoint{gp mark 7}{(4.025,1.635)}
\gppoint{gp mark 7}{(4.037,1.635)}
\gppoint{gp mark 7}{(4.050,1.635)}
\gppoint{gp mark 7}{(4.063,1.635)}
\gppoint{gp mark 7}{(4.075,1.635)}
\gppoint{gp mark 7}{(4.088,1.635)}
\gppoint{gp mark 7}{(4.101,1.635)}
\gppoint{gp mark 7}{(4.113,1.635)}
\gppoint{gp mark 7}{(4.126,1.635)}
\gppoint{gp mark 7}{(4.139,1.635)}
\gppoint{gp mark 7}{(4.151,1.635)}
\gppoint{gp mark 7}{(4.164,1.635)}
\gppoint{gp mark 7}{(4.177,1.635)}
\gppoint{gp mark 7}{(4.189,1.635)}
\gppoint{gp mark 7}{(4.202,1.635)}
\gppoint{gp mark 7}{(4.215,1.635)}
\gppoint{gp mark 7}{(4.227,1.635)}
\gppoint{gp mark 7}{(4.240,1.635)}
\gppoint{gp mark 7}{(4.253,1.635)}
\gppoint{gp mark 7}{(4.265,1.635)}
\gppoint{gp mark 7}{(4.278,1.635)}
\gppoint{gp mark 7}{(4.291,1.635)}
\gppoint{gp mark 7}{(4.303,1.635)}
\gppoint{gp mark 7}{(4.316,1.635)}
\gppoint{gp mark 7}{(4.329,1.634)}
\gppoint{gp mark 7}{(4.342,1.634)}
\gppoint{gp mark 7}{(4.354,1.634)}
\gppoint{gp mark 7}{(4.367,1.635)}
\gppoint{gp mark 7}{(4.380,1.635)}
\gppoint{gp mark 7}{(4.392,1.635)}
\gppoint{gp mark 7}{(4.405,1.635)}
\gppoint{gp mark 7}{(4.418,1.635)}
\gppoint{gp mark 7}{(4.430,1.635)}
\gppoint{gp mark 7}{(4.443,1.635)}
\gppoint{gp mark 7}{(4.456,1.635)}
\gppoint{gp mark 7}{(4.468,1.635)}
\gppoint{gp mark 7}{(4.481,1.635)}
\gppoint{gp mark 7}{(4.494,1.635)}
\gppoint{gp mark 7}{(4.506,1.635)}
\gppoint{gp mark 7}{(4.519,1.635)}
\gppoint{gp mark 7}{(4.532,1.635)}
\gppoint{gp mark 7}{(4.544,1.635)}
\gppoint{gp mark 7}{(4.557,1.635)}
\gppoint{gp mark 7}{(4.570,1.635)}
\gppoint{gp mark 7}{(4.582,1.635)}
\gppoint{gp mark 7}{(4.595,1.635)}
\gppoint{gp mark 7}{(4.608,1.635)}
\gppoint{gp mark 7}{(4.620,1.635)}
\gppoint{gp mark 7}{(4.633,1.635)}
\gppoint{gp mark 7}{(4.646,1.635)}
\gppoint{gp mark 7}{(4.658,1.635)}
\gppoint{gp mark 7}{(4.671,1.635)}
\gppoint{gp mark 7}{(4.684,1.635)}
\gppoint{gp mark 7}{(4.697,1.635)}
\gppoint{gp mark 7}{(4.709,1.635)}
\gppoint{gp mark 7}{(4.722,1.635)}
\gppoint{gp mark 7}{(4.735,1.635)}
\gppoint{gp mark 7}{(4.747,1.635)}
\gppoint{gp mark 7}{(4.760,1.635)}
\gppoint{gp mark 7}{(4.773,1.635)}
\gppoint{gp mark 7}{(4.785,1.635)}
\gppoint{gp mark 7}{(4.798,1.635)}
\gppoint{gp mark 7}{(4.811,1.635)}
\gppoint{gp mark 7}{(4.823,1.635)}
\gppoint{gp mark 7}{(4.836,1.635)}
\gppoint{gp mark 7}{(4.849,1.635)}
\gppoint{gp mark 7}{(4.861,1.635)}
\gppoint{gp mark 7}{(4.874,1.635)}
\gppoint{gp mark 7}{(4.887,1.635)}
\gppoint{gp mark 7}{(4.899,1.635)}
\gppoint{gp mark 7}{(4.912,1.635)}
\gppoint{gp mark 7}{(4.925,1.635)}
\gppoint{gp mark 7}{(4.937,1.635)}
\gppoint{gp mark 7}{(4.950,1.635)}
\gppoint{gp mark 7}{(4.963,1.635)}
\gppoint{gp mark 7}{(4.975,1.635)}
\gppoint{gp mark 7}{(4.988,1.635)}
\gppoint{gp mark 7}{(5.001,1.635)}
\gppoint{gp mark 7}{(5.013,1.635)}
\gppoint{gp mark 7}{(5.026,1.635)}
\gppoint{gp mark 7}{(5.039,1.635)}
\gppoint{gp mark 7}{(5.052,1.635)}
\gppoint{gp mark 7}{(5.064,1.635)}
\gppoint{gp mark 7}{(5.077,1.635)}
\gppoint{gp mark 7}{(5.090,1.635)}
\gppoint{gp mark 7}{(5.102,1.635)}
\gppoint{gp mark 7}{(5.115,1.635)}
\gppoint{gp mark 7}{(5.128,1.635)}
\gppoint{gp mark 7}{(5.140,1.635)}
\gppoint{gp mark 7}{(5.153,1.635)}
\gppoint{gp mark 7}{(5.166,1.635)}
\gppoint{gp mark 7}{(5.178,1.635)}
\gppoint{gp mark 7}{(5.191,1.635)}
\gppoint{gp mark 7}{(5.204,1.635)}
\gppoint{gp mark 7}{(5.216,1.635)}
\gppoint{gp mark 7}{(5.229,1.635)}
\gppoint{gp mark 7}{(5.242,1.635)}
\gppoint{gp mark 7}{(5.254,1.635)}
\gppoint{gp mark 7}{(5.267,1.635)}
\gppoint{gp mark 7}{(5.280,1.635)}
\gppoint{gp mark 7}{(5.292,1.635)}
\gppoint{gp mark 7}{(5.305,1.635)}
\gppoint{gp mark 7}{(5.318,1.635)}
\gppoint{gp mark 7}{(5.330,1.635)}
\gppoint{gp mark 7}{(5.343,1.635)}
\gppoint{gp mark 7}{(5.356,1.635)}
\gppoint{gp mark 7}{(5.368,1.635)}
\gppoint{gp mark 7}{(5.381,1.635)}
\gppoint{gp mark 7}{(5.394,1.635)}
\gppoint{gp mark 7}{(5.407,1.635)}
\gppoint{gp mark 7}{(5.419,1.635)}
\gppoint{gp mark 7}{(5.432,1.635)}
\gppoint{gp mark 7}{(5.445,1.635)}
\gppoint{gp mark 7}{(5.457,1.635)}
\gppoint{gp mark 7}{(5.470,1.635)}
\gppoint{gp mark 7}{(5.483,1.635)}
\gppoint{gp mark 7}{(5.495,1.635)}
\gppoint{gp mark 7}{(5.508,1.635)}
\gppoint{gp mark 7}{(5.521,1.635)}
\gppoint{gp mark 7}{(5.533,1.635)}
\gppoint{gp mark 7}{(5.546,1.635)}
\gppoint{gp mark 7}{(5.559,1.635)}
\gppoint{gp mark 7}{(5.571,1.635)}
\gppoint{gp mark 7}{(5.584,1.635)}
\gppoint{gp mark 7}{(5.597,1.635)}
\gppoint{gp mark 7}{(5.609,1.635)}
\gppoint{gp mark 7}{(5.622,1.635)}
\gppoint{gp mark 7}{(5.635,1.635)}
\gppoint{gp mark 7}{(5.647,1.635)}
\gppoint{gp mark 7}{(5.660,1.635)}
\gppoint{gp mark 7}{(5.673,1.635)}
\gppoint{gp mark 7}{(5.685,1.635)}
\gppoint{gp mark 7}{(5.698,1.635)}
\gppoint{gp mark 7}{(5.711,1.635)}
\gppoint{gp mark 7}{(5.723,1.634)}
\gppoint{gp mark 7}{(5.736,1.634)}
\gppoint{gp mark 7}{(5.749,1.634)}
\gppoint{gp mark 7}{(5.761,1.635)}
\gppoint{gp mark 7}{(5.774,1.635)}
\gppoint{gp mark 7}{(5.787,1.635)}
\gppoint{gp mark 7}{(5.800,1.635)}
\gppoint{gp mark 7}{(5.812,1.635)}
\gppoint{gp mark 7}{(5.825,1.635)}
\gppoint{gp mark 7}{(5.838,1.635)}
\gppoint{gp mark 7}{(5.850,1.635)}
\gppoint{gp mark 7}{(5.863,1.635)}
\gppoint{gp mark 7}{(5.876,1.635)}
\gppoint{gp mark 7}{(5.888,1.635)}
\gppoint{gp mark 7}{(5.901,1.635)}
\gppoint{gp mark 7}{(5.914,1.635)}
\gppoint{gp mark 7}{(5.926,1.635)}
\gppoint{gp mark 7}{(5.939,1.635)}
\gppoint{gp mark 7}{(5.952,1.635)}
\gppoint{gp mark 7}{(5.964,1.635)}
\gppoint{gp mark 7}{(5.977,1.635)}
\gppoint{gp mark 7}{(5.990,1.635)}
\gppoint{gp mark 7}{(6.002,1.635)}
\gppoint{gp mark 7}{(6.015,1.635)}
\gppoint{gp mark 7}{(6.028,1.635)}
\gppoint{gp mark 7}{(6.040,1.635)}
\gppoint{gp mark 7}{(6.053,1.635)}
\gppoint{gp mark 7}{(6.066,1.635)}
\gppoint{gp mark 7}{(6.078,1.635)}
\gppoint{gp mark 7}{(6.091,1.635)}
\gppoint{gp mark 7}{(6.104,1.635)}
\gppoint{gp mark 7}{(6.116,1.635)}
\gppoint{gp mark 7}{(6.129,1.635)}
\gppoint{gp mark 7}{(6.142,1.635)}
\gppoint{gp mark 7}{(6.155,1.635)}
\gppoint{gp mark 7}{(6.167,1.635)}
\gppoint{gp mark 7}{(6.180,1.635)}
\gppoint{gp mark 7}{(6.193,1.635)}
\gppoint{gp mark 7}{(6.205,1.635)}
\gppoint{gp mark 7}{(6.218,1.635)}
\gppoint{gp mark 7}{(6.231,1.635)}
\gppoint{gp mark 7}{(6.243,1.635)}
\gppoint{gp mark 7}{(6.256,1.635)}
\gppoint{gp mark 7}{(6.269,1.635)}
\gppoint{gp mark 7}{(6.281,1.635)}
\gppoint{gp mark 7}{(6.294,1.635)}
\gppoint{gp mark 7}{(6.307,1.635)}
\gppoint{gp mark 7}{(6.319,1.635)}
\gppoint{gp mark 7}{(6.332,1.635)}
\gppoint{gp mark 7}{(6.345,1.635)}
\gppoint{gp mark 7}{(6.357,1.635)}
\gppoint{gp mark 7}{(6.370,1.635)}
\gppoint{gp mark 7}{(6.383,1.635)}
\gppoint{gp mark 7}{(6.395,1.635)}
\gppoint{gp mark 7}{(6.408,1.635)}
\gppoint{gp mark 7}{(6.421,1.635)}
\gppoint{gp mark 7}{(6.433,1.635)}
\gppoint{gp mark 7}{(6.446,1.635)}
\gppoint{gp mark 7}{(6.459,1.635)}
\gppoint{gp mark 7}{(6.471,1.635)}
\gppoint{gp mark 7}{(6.484,1.635)}
\gppoint{gp mark 7}{(6.497,1.635)}
\gppoint{gp mark 7}{(6.510,1.635)}
\gppoint{gp mark 7}{(6.522,1.635)}
\gppoint{gp mark 7}{(6.535,1.635)}
\gppoint{gp mark 7}{(6.548,1.635)}
\gppoint{gp mark 7}{(6.560,1.635)}
\gppoint{gp mark 7}{(6.573,1.635)}
\gppoint{gp mark 7}{(6.586,1.635)}
\gppoint{gp mark 7}{(6.598,1.635)}
\gppoint{gp mark 7}{(6.611,1.635)}
\gppoint{gp mark 7}{(6.624,1.635)}
\gppoint{gp mark 7}{(6.636,1.635)}
\gppoint{gp mark 7}{(6.649,1.635)}
\gppoint{gp mark 7}{(6.662,1.635)}
\gppoint{gp mark 7}{(6.674,1.635)}
\gppoint{gp mark 7}{(6.687,1.635)}
\gppoint{gp mark 7}{(6.700,1.635)}
\gppoint{gp mark 7}{(6.712,1.635)}
\gppoint{gp mark 7}{(6.725,1.635)}
\gppoint{gp mark 7}{(6.738,1.635)}
\gppoint{gp mark 7}{(6.750,1.635)}
\gppoint{gp mark 7}{(6.763,1.635)}
\gppoint{gp mark 7}{(6.776,1.635)}
\gppoint{gp mark 7}{(6.788,1.635)}
\gppoint{gp mark 7}{(6.801,1.635)}
\gppoint{gp mark 7}{(6.814,1.635)}
\gppoint{gp mark 7}{(6.826,1.635)}
\gppoint{gp mark 7}{(6.839,1.635)}
\gppoint{gp mark 7}{(6.852,1.635)}
\gppoint{gp mark 7}{(6.865,1.635)}
\gppoint{gp mark 7}{(6.877,1.635)}
\gppoint{gp mark 7}{(6.890,1.635)}
\gppoint{gp mark 7}{(6.903,1.635)}
\gppoint{gp mark 7}{(6.915,1.635)}
\gppoint{gp mark 7}{(6.928,1.635)}
\gppoint{gp mark 7}{(6.941,1.635)}
\gppoint{gp mark 7}{(6.953,1.635)}
\gppoint{gp mark 7}{(6.966,1.635)}
\gppoint{gp mark 7}{(6.979,1.635)}
\gppoint{gp mark 7}{(6.991,1.635)}
\gppoint{gp mark 7}{(7.004,1.635)}
\gppoint{gp mark 7}{(7.017,1.635)}
\gppoint{gp mark 7}{(7.029,1.635)}
\gppoint{gp mark 7}{(7.042,1.635)}
\gppoint{gp mark 7}{(7.055,1.635)}
\gppoint{gp mark 7}{(7.067,1.635)}
\gppoint{gp mark 7}{(7.080,1.635)}
\gppoint{gp mark 7}{(7.093,1.635)}
\gppoint{gp mark 7}{(7.105,1.635)}
\gppoint{gp mark 7}{(7.118,1.635)}
\gppoint{gp mark 7}{(7.131,1.635)}
\gppoint{gp mark 7}{(7.143,1.635)}
\gppoint{gp mark 7}{(7.156,1.635)}
\gppoint{gp mark 7}{(7.169,1.635)}
\gppoint{gp mark 7}{(7.181,1.635)}
\gppoint{gp mark 7}{(7.194,1.635)}
\gppoint{gp mark 7}{(7.207,1.635)}
\gppoint{gp mark 7}{(7.220,1.635)}
\gppoint{gp mark 7}{(7.232,1.635)}
\gppoint{gp mark 7}{(7.245,1.635)}
\gppoint{gp mark 7}{(7.258,1.635)}
\gppoint{gp mark 7}{(7.270,1.635)}
\gppoint{gp mark 7}{(7.283,1.635)}
\gppoint{gp mark 7}{(7.296,1.635)}
\gppoint{gp mark 7}{(7.308,1.635)}
\gppoint{gp mark 7}{(7.321,1.635)}
\gppoint{gp mark 7}{(7.334,1.635)}
\gppoint{gp mark 7}{(7.346,1.635)}
\gppoint{gp mark 7}{(7.359,1.635)}
\gppoint{gp mark 7}{(7.372,1.635)}
\gppoint{gp mark 7}{(7.384,1.635)}
\gppoint{gp mark 7}{(7.397,1.635)}
\gppoint{gp mark 7}{(7.410,1.635)}
\gppoint{gp mark 7}{(7.422,1.635)}
\gppoint{gp mark 7}{(7.435,1.635)}
\gppoint{gp mark 7}{(7.448,1.635)}
\gppoint{gp mark 7}{(7.460,1.635)}
\gppoint{gp mark 7}{(7.473,1.635)}
\gppoint{gp mark 7}{(7.486,1.635)}
\gppoint{gp mark 7}{(7.498,1.635)}
\gppoint{gp mark 7}{(7.511,1.635)}
\gppoint{gp mark 7}{(7.524,1.635)}
\gppoint{gp mark 7}{(7.536,1.635)}
\gppoint{gp mark 7}{(7.549,1.635)}
\gppoint{gp mark 7}{(7.562,1.635)}
\gppoint{gp mark 7}{(7.575,1.635)}
\gppoint{gp mark 7}{(7.587,1.635)}
\gppoint{gp mark 7}{(7.600,1.635)}
\gppoint{gp mark 7}{(7.613,1.635)}
\gppoint{gp mark 7}{(7.625,1.635)}
\gppoint{gp mark 7}{(7.638,1.635)}
\gppoint{gp mark 7}{(7.651,1.635)}
\gppoint{gp mark 7}{(7.663,1.635)}
\gppoint{gp mark 7}{(7.676,1.635)}
\gppoint{gp mark 7}{(7.689,1.635)}
\gppoint{gp mark 7}{(7.701,1.635)}
\gppoint{gp mark 7}{(7.714,1.635)}
\gppoint{gp mark 7}{(7.727,1.635)}
\gppoint{gp mark 7}{(7.739,1.635)}
\gppoint{gp mark 7}{(7.752,1.635)}
\gppoint{gp mark 7}{(7.765,1.635)}
\gppoint{gp mark 7}{(7.777,1.635)}
\gppoint{gp mark 7}{(7.790,1.635)}
\gppoint{gp mark 7}{(7.803,1.635)}
\gppoint{gp mark 7}{(7.815,1.635)}
\gppoint{gp mark 7}{(7.828,1.635)}
\gppoint{gp mark 7}{(7.841,1.635)}
\gppoint{gp mark 7}{(7.853,1.635)}
\gppoint{gp mark 7}{(7.866,1.635)}
\gppoint{gp mark 7}{(7.879,1.635)}
\gppoint{gp mark 7}{(7.891,1.635)}
\gppoint{gp mark 7}{(7.904,1.635)}
\gppoint{gp mark 7}{(7.917,1.635)}
\gppoint{gp mark 7}{(7.930,1.635)}
\gpcolor{rgb color={0.000,0.000,0.000}}
\gpsetlinetype{gp lt plot 0}
\gpsetlinewidth{4.00}
\draw[gp path] (2.404,3.934)--(2.886,3.934);
\draw[gp path] (2.886,1.228)--(3.811,1.228);
\draw[gp path] (3.811,1.635)--(7.722,1.635);
\draw[gp path] (7.722,1.635)--(7.947,1.635);
\draw[gp path] (1.207,4.836)--(1.220,4.826)--(1.233,4.816)--(1.246,4.807)--(1.259,4.797)%
  --(1.271,4.787)--(1.284,4.777)--(1.297,4.768)--(1.310,4.758)--(1.323,4.748)--(1.336,4.739)%
  --(1.349,4.729)--(1.361,4.719)--(1.374,4.710)--(1.387,4.700)--(1.400,4.690)--(1.413,4.681)%
  --(1.426,4.671)--(1.439,4.661)--(1.452,4.652)--(1.464,4.642)--(1.477,4.632)--(1.490,4.622)%
  --(1.503,4.613)--(1.516,4.603)--(1.529,4.593)--(1.542,4.584)--(1.555,4.574)--(1.567,4.564)%
  --(1.580,4.555)--(1.593,4.545)--(1.606,4.535)--(1.619,4.526)--(1.632,4.516)--(1.645,4.506)%
  --(1.657,4.496)--(1.670,4.487)--(1.683,4.477)--(1.696,4.467)--(1.709,4.458)--(1.722,4.448)%
  --(1.735,4.438)--(1.748,4.429)--(1.760,4.419)--(1.773,4.409)--(1.786,4.400)--(1.799,4.390)%
  --(1.812,4.380)--(1.825,4.371)--(1.838,4.361)--(1.850,4.351)--(1.863,4.341)--(1.876,4.332)%
  --(1.889,4.322)--(1.902,4.312)--(1.915,4.303)--(1.928,4.293)--(1.941,4.283)--(1.953,4.274)%
  --(1.966,4.264)--(1.979,4.254)--(1.992,4.245)--(2.005,4.235)--(2.018,4.225)--(2.031,4.215)%
  --(2.044,4.206)--(2.056,4.196)--(2.069,4.186)--(2.082,4.177)--(2.095,4.167)--(2.108,4.157)%
  --(2.121,4.148)--(2.134,4.138)--(2.146,4.128)--(2.159,4.119)--(2.172,4.109)--(2.185,4.099)%
  --(2.198,4.090)--(2.211,4.080)--(2.224,4.070)--(2.237,4.060)--(2.249,4.051)--(2.262,4.041)%
  --(2.275,4.031)--(2.288,4.022)--(2.301,4.012)--(2.314,4.002)--(2.327,3.993)--(2.340,3.983)%
  --(2.352,3.973)--(2.365,3.964)--(2.378,3.954)--(2.391,3.944)--(2.404,3.934);
\draw[gp path] (2.886,3.934)--(2.886,1.228);
\draw[gp path] (3.811,1.228)--(3.811,1.635);
\draw[gp path] (3.793,3.695)--(4.572,3.695);
\gpcolor{rgb color={1.000,0.000,0.000}}
\gpsetlinewidth{0.50}
\gppoint{gp mark 7}{(4.182,2.921)}
\gpcolor{rgb color={0.000,0.000,0.000}}
\node[gp node left,font={\fontsize{10pt}{12pt}\selectfont}] at (1.456,5.166) {\LARGE $B_y$};
\node[gp node left,font={\fontsize{10pt}{12pt}\selectfont}] at (5.740,5.166) {\large $\alpha = 3.0$};
\node[gp node left,font={\fontsize{10pt}{12pt}\selectfont}] at (4.831,3.695) {\large exact};
\node[gp node left,font={\fontsize{10pt}{12pt}\selectfont}] at (4.831,2.921) {\large converging};
%% coordinates of the plot area
\gpdefrectangularnode{gp plot 1}{\pgfpoint{1.196cm}{0.985cm}}{\pgfpoint{7.947cm}{5.631cm}}
\end{tikzpicture}
%% gnuplot variables
}
\end{tabular}
\caption{The approximate solution after the first flux correction of HLLD-CWM and exact r-solution to the full Riemann problem for the near-coplanar case with $2048$ grid points.  The compound wave is almost completely removed, except near $x=0.365$ where a weak intermediate shock remains.}
\label{fig:fast_coplanar_b_rsol_init}
\end{figure}

%-----------------------------------------------------------------
% Two coplanar initial r-solution
%-----------------------------------------------------------------
\begin{figure}[htbp] 
\begin{tabular}{cc}
\resizebox{0.5\linewidth}{!}{\tikzsetnextfilename{AK7_rsol_init_1}\begin{tikzpicture}[gnuplot]
%% generated with GNUPLOT 4.6p4 (Lua 5.1; terminal rev. 99, script rev. 100)
%% Tue 05 Aug 2014 02:52:57 PM EDT
\path (0.000,0.000) rectangle (8.500,6.000);
\gpfill{rgb color={1.000,1.000,1.000}} (1.012,0.985)--(7.946,0.985)--(7.946,5.630)--(1.012,5.630)--cycle;
\gpcolor{color=gp lt color border}
\gpsetlinetype{gp lt border}
\gpsetlinewidth{1.00}
\draw[gp path] (1.012,0.985)--(1.012,5.630)--(7.946,5.630)--(7.946,0.985)--cycle;
\gpcolor{color=gp lt color axes}
\gpsetlinetype{gp lt axes}
\gpsetlinewidth{2.00}
\draw[gp path] (1.012,1.295)--(7.947,1.295);
\gpcolor{color=gp lt color border}
\gpsetlinetype{gp lt border}
\draw[gp path] (1.012,1.295)--(1.084,1.295);
\draw[gp path] (7.947,1.295)--(7.875,1.295);
\gpcolor{rgb color={0.000,0.000,0.000}}
\node[gp node right,font={\fontsize{10pt}{12pt}\selectfont}] at (0.828,1.295) {0.4};
\gpcolor{color=gp lt color axes}
\gpsetlinetype{gp lt axes}
\draw[gp path] (1.012,1.914)--(7.947,1.914);
\gpcolor{color=gp lt color border}
\gpsetlinetype{gp lt border}
\draw[gp path] (1.012,1.914)--(1.084,1.914);
\draw[gp path] (7.947,1.914)--(7.875,1.914);
\gpcolor{rgb color={0.000,0.000,0.000}}
\node[gp node right,font={\fontsize{10pt}{12pt}\selectfont}] at (0.828,1.914) {0.6};
\gpcolor{color=gp lt color axes}
\gpsetlinetype{gp lt axes}
\draw[gp path] (1.012,2.534)--(7.947,2.534);
\gpcolor{color=gp lt color border}
\gpsetlinetype{gp lt border}
\draw[gp path] (1.012,2.534)--(1.084,2.534);
\draw[gp path] (7.947,2.534)--(7.875,2.534);
\gpcolor{rgb color={0.000,0.000,0.000}}
\node[gp node right,font={\fontsize{10pt}{12pt}\selectfont}] at (0.828,2.534) {0.8};
\gpcolor{color=gp lt color axes}
\gpsetlinetype{gp lt axes}
\draw[gp path] (1.012,3.153)--(7.947,3.153);
\gpcolor{color=gp lt color border}
\gpsetlinetype{gp lt border}
\draw[gp path] (1.012,3.153)--(1.084,3.153);
\draw[gp path] (7.947,3.153)--(7.875,3.153);
\gpcolor{rgb color={0.000,0.000,0.000}}
\node[gp node right,font={\fontsize{10pt}{12pt}\selectfont}] at (0.828,3.153) {1};
\gpcolor{color=gp lt color axes}
\gpsetlinetype{gp lt axes}
\draw[gp path] (1.012,3.773)--(7.947,3.773);
\gpcolor{color=gp lt color border}
\gpsetlinetype{gp lt border}
\draw[gp path] (1.012,3.773)--(1.084,3.773);
\draw[gp path] (7.947,3.773)--(7.875,3.773);
\gpcolor{rgb color={0.000,0.000,0.000}}
\node[gp node right,font={\fontsize{10pt}{12pt}\selectfont}] at (0.828,3.773) {1.2};
\gpcolor{color=gp lt color axes}
\gpsetlinetype{gp lt axes}
\draw[gp path] (1.012,4.392)--(7.947,4.392);
\gpcolor{color=gp lt color border}
\gpsetlinetype{gp lt border}
\draw[gp path] (1.012,4.392)--(1.084,4.392);
\draw[gp path] (7.947,4.392)--(7.875,4.392);
\gpcolor{rgb color={0.000,0.000,0.000}}
\node[gp node right,font={\fontsize{10pt}{12pt}\selectfont}] at (0.828,4.392) {1.4};
\gpcolor{color=gp lt color axes}
\gpsetlinetype{gp lt axes}
\draw[gp path] (1.012,5.012)--(7.947,5.012);
\gpcolor{color=gp lt color border}
\gpsetlinetype{gp lt border}
\draw[gp path] (1.012,5.012)--(1.084,5.012);
\draw[gp path] (7.947,5.012)--(7.875,5.012);
\gpcolor{rgb color={0.000,0.000,0.000}}
\node[gp node right,font={\fontsize{10pt}{12pt}\selectfont}] at (0.828,5.012) {1.6};
\gpcolor{color=gp lt color axes}
\gpsetlinetype{gp lt axes}
\draw[gp path] (1.012,5.631)--(7.947,5.631);
\gpcolor{color=gp lt color border}
\gpsetlinetype{gp lt border}
\draw[gp path] (1.012,5.631)--(1.084,5.631);
\draw[gp path] (7.947,5.631)--(7.875,5.631);
\gpcolor{rgb color={0.000,0.000,0.000}}
\node[gp node right,font={\fontsize{10pt}{12pt}\selectfont}] at (0.828,5.631) {1.8};
\gpcolor{color=gp lt color axes}
\gpsetlinetype{gp lt axes}
\draw[gp path] (1.012,0.985)--(1.012,5.631);
\gpcolor{color=gp lt color border}
\gpsetlinetype{gp lt border}
\draw[gp path] (1.012,0.985)--(1.012,1.057);
\draw[gp path] (1.012,5.631)--(1.012,5.559);
\gpcolor{rgb color={0.000,0.000,0.000}}
\node[gp node center,font={\fontsize{10pt}{12pt}\selectfont}] at (1.012,0.677) {0.3};
\gpcolor{color=gp lt color axes}
\gpsetlinetype{gp lt axes}
\draw[gp path] (2.399,0.985)--(2.399,5.631);
\gpcolor{color=gp lt color border}
\gpsetlinetype{gp lt border}
\draw[gp path] (2.399,0.985)--(2.399,1.057);
\draw[gp path] (2.399,5.631)--(2.399,5.559);
\gpcolor{rgb color={0.000,0.000,0.000}}
\node[gp node center,font={\fontsize{10pt}{12pt}\selectfont}] at (2.399,0.677) {0.4};
\gpcolor{color=gp lt color axes}
\gpsetlinetype{gp lt axes}
\draw[gp path] (3.786,0.985)--(3.786,5.631);
\gpcolor{color=gp lt color border}
\gpsetlinetype{gp lt border}
\draw[gp path] (3.786,0.985)--(3.786,1.057);
\draw[gp path] (3.786,5.631)--(3.786,5.559);
\gpcolor{rgb color={0.000,0.000,0.000}}
\node[gp node center,font={\fontsize{10pt}{12pt}\selectfont}] at (3.786,0.677) {0.5};
\gpcolor{color=gp lt color axes}
\gpsetlinetype{gp lt axes}
\draw[gp path] (5.173,0.985)--(5.173,5.631);
\gpcolor{color=gp lt color border}
\gpsetlinetype{gp lt border}
\draw[gp path] (5.173,0.985)--(5.173,1.057);
\draw[gp path] (5.173,5.631)--(5.173,5.559);
\gpcolor{rgb color={0.000,0.000,0.000}}
\node[gp node center,font={\fontsize{10pt}{12pt}\selectfont}] at (5.173,0.677) {0.6};
\gpcolor{color=gp lt color axes}
\gpsetlinetype{gp lt axes}
\draw[gp path] (6.560,0.985)--(6.560,5.631);
\gpcolor{color=gp lt color border}
\gpsetlinetype{gp lt border}
\draw[gp path] (6.560,0.985)--(6.560,1.057);
\draw[gp path] (6.560,5.631)--(6.560,5.559);
\gpcolor{rgb color={0.000,0.000,0.000}}
\node[gp node center,font={\fontsize{10pt}{12pt}\selectfont}] at (6.560,0.677) {0.7};
\gpcolor{color=gp lt color axes}
\gpsetlinetype{gp lt axes}
\draw[gp path] (7.947,0.985)--(7.947,5.631);
\gpcolor{color=gp lt color border}
\gpsetlinetype{gp lt border}
\draw[gp path] (7.947,0.985)--(7.947,1.057);
\draw[gp path] (7.947,5.631)--(7.947,5.559);
\gpcolor{rgb color={0.000,0.000,0.000}}
\node[gp node center,font={\fontsize{10pt}{12pt}\selectfont}] at (7.947,0.677) {0.8};
\gpcolor{color=gp lt color border}
\draw[gp path] (1.012,5.631)--(1.012,0.985)--(7.947,0.985)--(7.947,5.631)--cycle;
\gpcolor{rgb color={0.000,0.000,0.000}}
\node[gp node center,font={\fontsize{10pt}{12pt}\selectfont}] at (4.479,0.215) {\large $x$};
\gpcolor{rgb color={1.000,0.000,0.000}}
\gpsetlinewidth{4.00}
\gpsetpointsize{2.67}
\gppoint{gp mark 7}{(1.014,2.868)}
\gppoint{gp mark 7}{(1.018,2.866)}
\gppoint{gp mark 7}{(1.021,2.864)}
\gppoint{gp mark 7}{(1.025,2.862)}
\gppoint{gp mark 7}{(1.028,2.860)}
\gppoint{gp mark 7}{(1.031,2.858)}
\gppoint{gp mark 7}{(1.035,2.856)}
\gppoint{gp mark 7}{(1.038,2.854)}
\gppoint{gp mark 7}{(1.041,2.852)}
\gppoint{gp mark 7}{(1.045,2.850)}
\gppoint{gp mark 7}{(1.048,2.848)}
\gppoint{gp mark 7}{(1.052,2.846)}
\gppoint{gp mark 7}{(1.055,2.844)}
\gppoint{gp mark 7}{(1.058,2.842)}
\gppoint{gp mark 7}{(1.062,2.840)}
\gppoint{gp mark 7}{(1.065,2.838)}
\gppoint{gp mark 7}{(1.069,2.835)}
\gppoint{gp mark 7}{(1.072,2.833)}
\gppoint{gp mark 7}{(1.075,2.831)}
\gppoint{gp mark 7}{(1.079,2.829)}
\gppoint{gp mark 7}{(1.082,2.827)}
\gppoint{gp mark 7}{(1.085,2.825)}
\gppoint{gp mark 7}{(1.089,2.823)}
\gppoint{gp mark 7}{(1.092,2.821)}
\gppoint{gp mark 7}{(1.096,2.819)}
\gppoint{gp mark 7}{(1.099,2.817)}
\gppoint{gp mark 7}{(1.102,2.815)}
\gppoint{gp mark 7}{(1.106,2.813)}
\gppoint{gp mark 7}{(1.109,2.811)}
\gppoint{gp mark 7}{(1.113,2.809)}
\gppoint{gp mark 7}{(1.116,2.807)}
\gppoint{gp mark 7}{(1.119,2.805)}
\gppoint{gp mark 7}{(1.123,2.803)}
\gppoint{gp mark 7}{(1.126,2.801)}
\gppoint{gp mark 7}{(1.130,2.799)}
\gppoint{gp mark 7}{(1.133,2.797)}
\gppoint{gp mark 7}{(1.136,2.795)}
\gppoint{gp mark 7}{(1.140,2.793)}
\gppoint{gp mark 7}{(1.143,2.791)}
\gppoint{gp mark 7}{(1.146,2.788)}
\gppoint{gp mark 7}{(1.150,2.786)}
\gppoint{gp mark 7}{(1.153,2.784)}
\gppoint{gp mark 7}{(1.157,2.782)}
\gppoint{gp mark 7}{(1.160,2.780)}
\gppoint{gp mark 7}{(1.163,2.778)}
\gppoint{gp mark 7}{(1.167,2.776)}
\gppoint{gp mark 7}{(1.170,2.774)}
\gppoint{gp mark 7}{(1.174,2.772)}
\gppoint{gp mark 7}{(1.177,2.770)}
\gppoint{gp mark 7}{(1.180,2.768)}
\gppoint{gp mark 7}{(1.184,2.766)}
\gppoint{gp mark 7}{(1.187,2.764)}
\gppoint{gp mark 7}{(1.190,2.762)}
\gppoint{gp mark 7}{(1.194,2.760)}
\gppoint{gp mark 7}{(1.197,2.758)}
\gppoint{gp mark 7}{(1.201,2.756)}
\gppoint{gp mark 7}{(1.204,2.754)}
\gppoint{gp mark 7}{(1.207,2.752)}
\gppoint{gp mark 7}{(1.211,2.750)}
\gppoint{gp mark 7}{(1.214,2.748)}
\gppoint{gp mark 7}{(1.218,2.746)}
\gppoint{gp mark 7}{(1.221,2.744)}
\gppoint{gp mark 7}{(1.224,2.742)}
\gppoint{gp mark 7}{(1.228,2.740)}
\gppoint{gp mark 7}{(1.231,2.738)}
\gppoint{gp mark 7}{(1.234,2.736)}
\gppoint{gp mark 7}{(1.238,2.734)}
\gppoint{gp mark 7}{(1.241,2.732)}
\gppoint{gp mark 7}{(1.245,2.730)}
\gppoint{gp mark 7}{(1.248,2.728)}
\gppoint{gp mark 7}{(1.251,2.726)}
\gppoint{gp mark 7}{(1.255,2.724)}
\gppoint{gp mark 7}{(1.258,2.722)}
\gppoint{gp mark 7}{(1.262,2.720)}
\gppoint{gp mark 7}{(1.265,2.718)}
\gppoint{gp mark 7}{(1.268,2.716)}
\gppoint{gp mark 7}{(1.272,2.714)}
\gppoint{gp mark 7}{(1.275,2.712)}
\gppoint{gp mark 7}{(1.278,2.710)}
\gppoint{gp mark 7}{(1.282,2.708)}
\gppoint{gp mark 7}{(1.285,2.706)}
\gppoint{gp mark 7}{(1.289,2.704)}
\gppoint{gp mark 7}{(1.292,2.702)}
\gppoint{gp mark 7}{(1.295,2.700)}
\gppoint{gp mark 7}{(1.299,2.698)}
\gppoint{gp mark 7}{(1.302,2.696)}
\gppoint{gp mark 7}{(1.306,2.694)}
\gppoint{gp mark 7}{(1.309,2.692)}
\gppoint{gp mark 7}{(1.312,2.690)}
\gppoint{gp mark 7}{(1.316,2.688)}
\gppoint{gp mark 7}{(1.319,2.686)}
\gppoint{gp mark 7}{(1.323,2.684)}
\gppoint{gp mark 7}{(1.326,2.682)}
\gppoint{gp mark 7}{(1.329,2.680)}
\gppoint{gp mark 7}{(1.333,2.678)}
\gppoint{gp mark 7}{(1.336,2.676)}
\gppoint{gp mark 7}{(1.339,2.674)}
\gppoint{gp mark 7}{(1.343,2.672)}
\gppoint{gp mark 7}{(1.346,2.670)}
\gppoint{gp mark 7}{(1.350,2.668)}
\gppoint{gp mark 7}{(1.353,2.666)}
\gppoint{gp mark 7}{(1.356,2.664)}
\gppoint{gp mark 7}{(1.360,2.662)}
\gppoint{gp mark 7}{(1.363,2.660)}
\gppoint{gp mark 7}{(1.367,2.658)}
\gppoint{gp mark 7}{(1.370,2.656)}
\gppoint{gp mark 7}{(1.373,2.654)}
\gppoint{gp mark 7}{(1.377,2.652)}
\gppoint{gp mark 7}{(1.380,2.650)}
\gppoint{gp mark 7}{(1.383,2.648)}
\gppoint{gp mark 7}{(1.387,2.646)}
\gppoint{gp mark 7}{(1.390,2.644)}
\gppoint{gp mark 7}{(1.394,2.642)}
\gppoint{gp mark 7}{(1.397,2.640)}
\gppoint{gp mark 7}{(1.400,2.638)}
\gppoint{gp mark 7}{(1.404,2.636)}
\gppoint{gp mark 7}{(1.407,2.634)}
\gppoint{gp mark 7}{(1.411,2.632)}
\gppoint{gp mark 7}{(1.414,2.630)}
\gppoint{gp mark 7}{(1.417,2.628)}
\gppoint{gp mark 7}{(1.421,2.626)}
\gppoint{gp mark 7}{(1.424,2.624)}
\gppoint{gp mark 7}{(1.427,2.622)}
\gppoint{gp mark 7}{(1.431,2.620)}
\gppoint{gp mark 7}{(1.434,2.618)}
\gppoint{gp mark 7}{(1.438,2.616)}
\gppoint{gp mark 7}{(1.441,2.614)}
\gppoint{gp mark 7}{(1.444,2.612)}
\gppoint{gp mark 7}{(1.448,2.610)}
\gppoint{gp mark 7}{(1.451,2.608)}
\gppoint{gp mark 7}{(1.455,2.606)}
\gppoint{gp mark 7}{(1.458,2.604)}
\gppoint{gp mark 7}{(1.461,2.602)}
\gppoint{gp mark 7}{(1.465,2.600)}
\gppoint{gp mark 7}{(1.468,2.598)}
\gppoint{gp mark 7}{(1.472,2.596)}
\gppoint{gp mark 7}{(1.475,2.594)}
\gppoint{gp mark 7}{(1.478,2.592)}
\gppoint{gp mark 7}{(1.482,2.590)}
\gppoint{gp mark 7}{(1.485,2.588)}
\gppoint{gp mark 7}{(1.488,2.586)}
\gppoint{gp mark 7}{(1.492,2.584)}
\gppoint{gp mark 7}{(1.495,2.582)}
\gppoint{gp mark 7}{(1.499,2.580)}
\gppoint{gp mark 7}{(1.502,2.578)}
\gppoint{gp mark 7}{(1.505,2.576)}
\gppoint{gp mark 7}{(1.509,2.574)}
\gppoint{gp mark 7}{(1.512,2.572)}
\gppoint{gp mark 7}{(1.516,2.570)}
\gppoint{gp mark 7}{(1.519,2.568)}
\gppoint{gp mark 7}{(1.522,2.567)}
\gppoint{gp mark 7}{(1.526,2.565)}
\gppoint{gp mark 7}{(1.529,2.563)}
\gppoint{gp mark 7}{(1.532,2.561)}
\gppoint{gp mark 7}{(1.536,2.559)}
\gppoint{gp mark 7}{(1.539,2.558)}
\gppoint{gp mark 7}{(1.543,2.558)}
\gppoint{gp mark 7}{(1.546,2.557)}
\gppoint{gp mark 7}{(1.549,2.557)}
\gppoint{gp mark 7}{(1.553,2.557)}
\gppoint{gp mark 7}{(1.556,2.557)}
\gppoint{gp mark 7}{(1.560,2.557)}
\gppoint{gp mark 7}{(1.563,2.558)}
\gppoint{gp mark 7}{(1.566,2.559)}
\gppoint{gp mark 7}{(1.570,2.560)}
\gppoint{gp mark 7}{(1.573,2.560)}
\gppoint{gp mark 7}{(1.576,2.561)}
\gppoint{gp mark 7}{(1.580,2.561)}
\gppoint{gp mark 7}{(1.583,2.561)}
\gppoint{gp mark 7}{(1.587,2.561)}
\gppoint{gp mark 7}{(1.590,2.561)}
\gppoint{gp mark 7}{(1.593,2.561)}
\gppoint{gp mark 7}{(1.597,2.561)}
\gppoint{gp mark 7}{(1.600,2.561)}
\gppoint{gp mark 7}{(1.604,2.561)}
\gppoint{gp mark 7}{(1.607,2.561)}
\gppoint{gp mark 7}{(1.610,2.561)}
\gppoint{gp mark 7}{(1.614,2.561)}
\gppoint{gp mark 7}{(1.617,2.561)}
\gppoint{gp mark 7}{(1.621,2.561)}
\gppoint{gp mark 7}{(1.624,2.561)}
\gppoint{gp mark 7}{(1.627,2.561)}
\gppoint{gp mark 7}{(1.631,2.561)}
\gppoint{gp mark 7}{(1.634,2.561)}
\gppoint{gp mark 7}{(1.637,2.561)}
\gppoint{gp mark 7}{(1.641,2.562)}
\gppoint{gp mark 7}{(1.644,2.562)}
\gppoint{gp mark 7}{(1.648,2.562)}
\gppoint{gp mark 7}{(1.651,2.562)}
\gppoint{gp mark 7}{(1.654,2.562)}
\gppoint{gp mark 7}{(1.658,2.562)}
\gppoint{gp mark 7}{(1.661,2.563)}
\gppoint{gp mark 7}{(1.665,2.563)}
\gppoint{gp mark 7}{(1.668,2.563)}
\gppoint{gp mark 7}{(1.671,2.563)}
\gppoint{gp mark 7}{(1.675,2.563)}
\gppoint{gp mark 7}{(1.678,2.563)}
\gppoint{gp mark 7}{(1.681,2.563)}
\gppoint{gp mark 7}{(1.685,2.563)}
\gppoint{gp mark 7}{(1.688,2.563)}
\gppoint{gp mark 7}{(1.692,2.563)}
\gppoint{gp mark 7}{(1.695,2.563)}
\gppoint{gp mark 7}{(1.698,2.563)}
\gppoint{gp mark 7}{(1.702,2.562)}
\gppoint{gp mark 7}{(1.705,2.562)}
\gppoint{gp mark 7}{(1.709,2.562)}
\gppoint{gp mark 7}{(1.712,2.562)}
\gppoint{gp mark 7}{(1.715,2.562)}
\gppoint{gp mark 7}{(1.719,2.561)}
\gppoint{gp mark 7}{(1.722,2.561)}
\gppoint{gp mark 7}{(1.725,2.561)}
\gppoint{gp mark 7}{(1.729,2.561)}
\gppoint{gp mark 7}{(1.732,2.561)}
\gppoint{gp mark 7}{(1.736,2.561)}
\gppoint{gp mark 7}{(1.739,2.561)}
\gppoint{gp mark 7}{(1.742,2.561)}
\gppoint{gp mark 7}{(1.746,2.560)}
\gppoint{gp mark 7}{(1.749,2.560)}
\gppoint{gp mark 7}{(1.753,2.560)}
\gppoint{gp mark 7}{(1.756,2.560)}
\gppoint{gp mark 7}{(1.759,2.560)}
\gppoint{gp mark 7}{(1.763,2.560)}
\gppoint{gp mark 7}{(1.766,2.560)}
\gppoint{gp mark 7}{(1.769,2.560)}
\gppoint{gp mark 7}{(1.773,2.560)}
\gppoint{gp mark 7}{(1.776,2.560)}
\gppoint{gp mark 7}{(1.780,2.560)}
\gppoint{gp mark 7}{(1.783,2.560)}
\gppoint{gp mark 7}{(1.786,2.560)}
\gppoint{gp mark 7}{(1.790,2.560)}
\gppoint{gp mark 7}{(1.793,2.560)}
\gppoint{gp mark 7}{(1.797,2.561)}
\gppoint{gp mark 7}{(1.800,2.561)}
\gppoint{gp mark 7}{(1.803,2.561)}
\gppoint{gp mark 7}{(1.807,2.561)}
\gppoint{gp mark 7}{(1.810,2.560)}
\gppoint{gp mark 7}{(1.814,2.560)}
\gppoint{gp mark 7}{(1.817,2.560)}
\gppoint{gp mark 7}{(1.820,2.560)}
\gppoint{gp mark 7}{(1.824,2.560)}
\gppoint{gp mark 7}{(1.827,2.560)}
\gppoint{gp mark 7}{(1.830,2.560)}
\gppoint{gp mark 7}{(1.834,2.560)}
\gppoint{gp mark 7}{(1.837,2.560)}
\gppoint{gp mark 7}{(1.841,2.560)}
\gppoint{gp mark 7}{(1.844,2.560)}
\gppoint{gp mark 7}{(1.847,2.560)}
\gppoint{gp mark 7}{(1.851,2.560)}
\gppoint{gp mark 7}{(1.854,2.559)}
\gppoint{gp mark 7}{(1.858,2.559)}
\gppoint{gp mark 7}{(1.861,2.559)}
\gppoint{gp mark 7}{(1.864,2.559)}
\gppoint{gp mark 7}{(1.868,2.560)}
\gppoint{gp mark 7}{(1.871,2.560)}
\gppoint{gp mark 7}{(1.874,2.560)}
\gppoint{gp mark 7}{(1.878,2.560)}
\gppoint{gp mark 7}{(1.881,2.560)}
\gppoint{gp mark 7}{(1.885,2.560)}
\gppoint{gp mark 7}{(1.888,2.560)}
\gppoint{gp mark 7}{(1.891,2.560)}
\gppoint{gp mark 7}{(1.895,2.559)}
\gppoint{gp mark 7}{(1.898,2.559)}
\gppoint{gp mark 7}{(1.902,2.559)}
\gppoint{gp mark 7}{(1.905,2.559)}
\gppoint{gp mark 7}{(1.908,2.559)}
\gppoint{gp mark 7}{(1.912,2.560)}
\gppoint{gp mark 7}{(1.915,2.560)}
\gppoint{gp mark 7}{(1.918,2.558)}
\gppoint{gp mark 7}{(1.922,2.494)}
\gppoint{gp mark 7}{(1.925,2.191)}
\gppoint{gp mark 7}{(1.929,2.179)}
\gppoint{gp mark 7}{(1.932,2.496)}
\gppoint{gp mark 7}{(1.935,2.547)}
\gppoint{gp mark 7}{(1.939,2.562)}
\gppoint{gp mark 7}{(1.942,2.569)}
\gppoint{gp mark 7}{(1.946,2.584)}
\gppoint{gp mark 7}{(1.949,2.581)}
\gppoint{gp mark 7}{(1.952,2.553)}
\gppoint{gp mark 7}{(1.956,2.541)}
\gppoint{gp mark 7}{(1.959,2.552)}
\gppoint{gp mark 7}{(1.963,2.563)}
\gppoint{gp mark 7}{(1.966,2.573)}
\gppoint{gp mark 7}{(1.969,2.584)}
\gppoint{gp mark 7}{(1.973,2.585)}
\gppoint{gp mark 7}{(1.976,2.579)}
\gppoint{gp mark 7}{(1.979,2.551)}
\gppoint{gp mark 7}{(1.983,2.534)}
\gppoint{gp mark 7}{(1.986,2.536)}
\gppoint{gp mark 7}{(1.990,2.552)}
\gppoint{gp mark 7}{(1.993,2.564)}
\gppoint{gp mark 7}{(1.996,2.577)}
\gppoint{gp mark 7}{(2.000,2.579)}
\gppoint{gp mark 7}{(2.003,2.579)}
\gppoint{gp mark 7}{(2.007,2.576)}
\gppoint{gp mark 7}{(2.010,2.561)}
\gppoint{gp mark 7}{(2.013,2.548)}
\gppoint{gp mark 7}{(2.017,2.545)}
\gppoint{gp mark 7}{(2.020,2.545)}
\gppoint{gp mark 7}{(2.023,2.548)}
\gppoint{gp mark 7}{(2.027,2.558)}
\gppoint{gp mark 7}{(2.030,2.573)}
\gppoint{gp mark 7}{(2.034,2.589)}
\gppoint{gp mark 7}{(2.037,2.595)}
\gppoint{gp mark 7}{(2.040,2.594)}
\gppoint{gp mark 7}{(2.044,2.582)}
\gppoint{gp mark 7}{(2.047,2.563)}
\gppoint{gp mark 7}{(2.051,2.555)}
\gppoint{gp mark 7}{(2.054,2.555)}
\gppoint{gp mark 7}{(2.057,2.557)}
\gppoint{gp mark 7}{(2.061,2.566)}
\gppoint{gp mark 7}{(2.064,2.569)}
\gppoint{gp mark 7}{(2.067,2.567)}
\gppoint{gp mark 7}{(2.071,2.558)}
\gppoint{gp mark 7}{(2.074,2.549)}
\gppoint{gp mark 7}{(2.078,2.549)}
\gppoint{gp mark 7}{(2.081,2.552)}
\gppoint{gp mark 7}{(2.084,2.557)}
\gppoint{gp mark 7}{(2.088,2.564)}
\gppoint{gp mark 7}{(2.091,2.569)}
\gppoint{gp mark 7}{(2.095,2.568)}
\gppoint{gp mark 7}{(2.098,2.564)}
\gppoint{gp mark 7}{(2.101,2.555)}
\gppoint{gp mark 7}{(2.105,2.551)}
\gppoint{gp mark 7}{(2.108,2.551)}
\gppoint{gp mark 7}{(2.112,2.555)}
\gppoint{gp mark 7}{(2.115,2.558)}
\gppoint{gp mark 7}{(2.118,2.566)}
\gppoint{gp mark 7}{(2.122,2.572)}
\gppoint{gp mark 7}{(2.125,2.573)}
\gppoint{gp mark 7}{(2.128,2.571)}
\gppoint{gp mark 7}{(2.132,2.566)}
\gppoint{gp mark 7}{(2.135,2.560)}
\gppoint{gp mark 7}{(2.139,2.555)}
\gppoint{gp mark 7}{(2.142,2.554)}
\gppoint{gp mark 7}{(2.145,2.555)}
\gppoint{gp mark 7}{(2.149,2.561)}
\gppoint{gp mark 7}{(2.152,2.571)}
\gppoint{gp mark 7}{(2.156,2.579)}
\gppoint{gp mark 7}{(2.159,2.581)}
\gppoint{gp mark 7}{(2.162,2.578)}
\gppoint{gp mark 7}{(2.166,2.566)}
\gppoint{gp mark 7}{(2.169,2.556)}
\gppoint{gp mark 7}{(2.172,2.555)}
\gppoint{gp mark 7}{(2.176,2.557)}
\gppoint{gp mark 7}{(2.179,2.567)}
\gppoint{gp mark 7}{(2.183,2.576)}
\gppoint{gp mark 7}{(2.186,2.577)}
\gppoint{gp mark 7}{(2.189,2.576)}
\gppoint{gp mark 7}{(2.193,2.568)}
\gppoint{gp mark 7}{(2.196,2.555)}
\gppoint{gp mark 7}{(2.200,2.551)}
\gppoint{gp mark 7}{(2.203,2.549)}
\gppoint{gp mark 7}{(2.206,2.549)}
\gppoint{gp mark 7}{(2.210,2.556)}
\gppoint{gp mark 7}{(2.213,2.569)}
\gppoint{gp mark 7}{(2.216,2.577)}
\gppoint{gp mark 7}{(2.220,2.581)}
\gppoint{gp mark 7}{(2.223,2.582)}
\gppoint{gp mark 7}{(2.227,2.578)}
\gppoint{gp mark 7}{(2.230,2.570)}
\gppoint{gp mark 7}{(2.233,2.566)}
\gppoint{gp mark 7}{(2.237,2.565)}
\gppoint{gp mark 7}{(2.240,2.565)}
\gppoint{gp mark 7}{(2.244,2.567)}
\gppoint{gp mark 7}{(2.247,2.567)}
\gppoint{gp mark 7}{(2.250,2.565)}
\gppoint{gp mark 7}{(2.254,2.562)}
\gppoint{gp mark 7}{(2.257,2.557)}
\gppoint{gp mark 7}{(2.261,2.556)}
\gppoint{gp mark 7}{(2.264,2.556)}
\gppoint{gp mark 7}{(2.267,2.559)}
\gppoint{gp mark 7}{(2.271,2.569)}
\gppoint{gp mark 7}{(2.274,2.571)}
\gppoint{gp mark 7}{(2.277,2.570)}
\gppoint{gp mark 7}{(2.281,2.566)}
\gppoint{gp mark 7}{(2.284,2.560)}
\gppoint{gp mark 7}{(2.288,2.556)}
\gppoint{gp mark 7}{(2.291,2.556)}
\gppoint{gp mark 7}{(2.294,2.559)}
\gppoint{gp mark 7}{(2.298,2.568)}
\gppoint{gp mark 7}{(2.301,2.570)}
\gppoint{gp mark 7}{(2.305,2.570)}
\gppoint{gp mark 7}{(2.308,2.566)}
\gppoint{gp mark 7}{(2.311,2.555)}
\gppoint{gp mark 7}{(2.315,2.546)}
\gppoint{gp mark 7}{(2.318,2.543)}
\gppoint{gp mark 7}{(2.321,2.544)}
\gppoint{gp mark 7}{(2.325,2.548)}
\gppoint{gp mark 7}{(2.328,2.560)}
\gppoint{gp mark 7}{(2.332,2.571)}
\gppoint{gp mark 7}{(2.335,2.575)}
\gppoint{gp mark 7}{(2.338,2.576)}
\gppoint{gp mark 7}{(2.342,2.575)}
\gppoint{gp mark 7}{(2.345,2.566)}
\gppoint{gp mark 7}{(2.349,2.556)}
\gppoint{gp mark 7}{(2.352,2.551)}
\gppoint{gp mark 7}{(2.355,2.550)}
\gppoint{gp mark 7}{(2.359,2.552)}
\gppoint{gp mark 7}{(2.362,2.562)}
\gppoint{gp mark 7}{(2.365,2.572)}
\gppoint{gp mark 7}{(2.369,2.576)}
\gppoint{gp mark 7}{(2.372,2.576)}
\gppoint{gp mark 7}{(2.376,2.573)}
\gppoint{gp mark 7}{(2.379,2.567)}
\gppoint{gp mark 7}{(2.382,2.564)}
\gppoint{gp mark 7}{(2.386,2.563)}
\gppoint{gp mark 7}{(2.389,2.565)}
\gppoint{gp mark 7}{(2.393,2.567)}
\gppoint{gp mark 7}{(2.396,2.566)}
\gppoint{gp mark 7}{(2.399,2.565)}
\gppoint{gp mark 7}{(2.403,2.563)}
\gppoint{gp mark 7}{(2.406,2.557)}
\gppoint{gp mark 7}{(2.409,2.556)}
\gppoint{gp mark 7}{(2.413,2.557)}
\gppoint{gp mark 7}{(2.416,2.559)}
\gppoint{gp mark 7}{(2.420,2.563)}
\gppoint{gp mark 7}{(2.423,2.571)}
\gppoint{gp mark 7}{(2.426,2.574)}
\gppoint{gp mark 7}{(2.430,2.574)}
\gppoint{gp mark 7}{(2.433,2.571)}
\gppoint{gp mark 7}{(2.437,2.565)}
\gppoint{gp mark 7}{(2.440,2.563)}
\gppoint{gp mark 7}{(2.443,2.563)}
\gppoint{gp mark 7}{(2.447,2.563)}
\gppoint{gp mark 7}{(2.450,2.564)}
\gppoint{gp mark 7}{(2.454,2.564)}
\gppoint{gp mark 7}{(2.457,2.562)}
\gppoint{gp mark 7}{(2.460,2.555)}
\gppoint{gp mark 7}{(2.464,2.548)}
\gppoint{gp mark 7}{(2.467,2.545)}
\gppoint{gp mark 7}{(2.470,2.544)}
\gppoint{gp mark 7}{(2.474,2.545)}
\gppoint{gp mark 7}{(2.477,2.554)}
\gppoint{gp mark 7}{(2.481,2.566)}
\gppoint{gp mark 7}{(2.484,2.573)}
\gppoint{gp mark 7}{(2.487,2.575)}
\gppoint{gp mark 7}{(2.491,2.574)}
\gppoint{gp mark 7}{(2.494,2.570)}
\gppoint{gp mark 7}{(2.498,2.563)}
\gppoint{gp mark 7}{(2.501,2.561)}
\gppoint{gp mark 7}{(2.504,2.561)}
\gppoint{gp mark 7}{(2.508,2.562)}
\gppoint{gp mark 7}{(2.511,2.564)}
\gppoint{gp mark 7}{(2.514,2.565)}
\gppoint{gp mark 7}{(2.518,2.565)}
\gppoint{gp mark 7}{(2.521,2.561)}
\gppoint{gp mark 7}{(2.525,2.554)}
\gppoint{gp mark 7}{(2.528,2.552)}
\gppoint{gp mark 7}{(2.531,2.549)}
\gppoint{gp mark 7}{(2.535,2.548)}
\gppoint{gp mark 7}{(2.538,2.551)}
\gppoint{gp mark 7}{(2.542,2.559)}
\gppoint{gp mark 7}{(2.545,2.563)}
\gppoint{gp mark 7}{(2.548,2.565)}
\gppoint{gp mark 7}{(2.552,2.566)}
\gppoint{gp mark 7}{(2.555,2.564)}
\gppoint{gp mark 7}{(2.558,2.561)}
\gppoint{gp mark 7}{(2.562,2.560)}
\gppoint{gp mark 7}{(2.565,2.560)}
\gppoint{gp mark 7}{(2.569,2.560)}
\gppoint{gp mark 7}{(2.572,2.561)}
\gppoint{gp mark 7}{(2.575,2.561)}
\gppoint{gp mark 7}{(2.579,2.560)}
\gppoint{gp mark 7}{(2.582,2.557)}
\gppoint{gp mark 7}{(2.586,2.557)}
\gppoint{gp mark 7}{(2.589,2.558)}
\gppoint{gp mark 7}{(2.592,2.561)}
\gppoint{gp mark 7}{(2.596,2.567)}
\gppoint{gp mark 7}{(2.599,2.577)}
\gppoint{gp mark 7}{(2.603,2.580)}
\gppoint{gp mark 7}{(2.606,2.579)}
\gppoint{gp mark 7}{(2.609,2.576)}
\gppoint{gp mark 7}{(2.613,2.572)}
\gppoint{gp mark 7}{(2.616,2.568)}
\gppoint{gp mark 7}{(2.619,2.566)}
\gppoint{gp mark 7}{(2.623,2.566)}
\gppoint{gp mark 7}{(2.626,2.567)}
\gppoint{gp mark 7}{(2.630,2.570)}
\gppoint{gp mark 7}{(2.633,2.570)}
\gppoint{gp mark 7}{(2.636,2.569)}
\gppoint{gp mark 7}{(2.640,2.564)}
\gppoint{gp mark 7}{(2.643,2.559)}
\gppoint{gp mark 7}{(2.647,2.557)}
\gppoint{gp mark 7}{(2.650,2.558)}
\gppoint{gp mark 7}{(2.653,2.560)}
\gppoint{gp mark 7}{(2.657,2.564)}
\gppoint{gp mark 7}{(2.660,2.566)}
\gppoint{gp mark 7}{(2.663,2.567)}
\gppoint{gp mark 7}{(2.667,2.565)}
\gppoint{gp mark 7}{(2.670,2.561)}
\gppoint{gp mark 7}{(2.674,2.557)}
\gppoint{gp mark 7}{(2.677,2.554)}
\gppoint{gp mark 7}{(2.680,2.553)}
\gppoint{gp mark 7}{(2.684,2.557)}
\gppoint{gp mark 7}{(2.687,2.565)}
\gppoint{gp mark 7}{(2.691,2.568)}
\gppoint{gp mark 7}{(2.694,2.570)}
\gppoint{gp mark 7}{(2.697,2.571)}
\gppoint{gp mark 7}{(2.701,2.570)}
\gppoint{gp mark 7}{(2.704,2.568)}
\gppoint{gp mark 7}{(2.707,2.568)}
\gppoint{gp mark 7}{(2.711,2.568)}
\gppoint{gp mark 7}{(2.714,2.569)}
\gppoint{gp mark 7}{(2.718,2.571)}
\gppoint{gp mark 7}{(2.721,2.571)}
\gppoint{gp mark 7}{(2.724,2.570)}
\gppoint{gp mark 7}{(2.728,2.569)}
\gppoint{gp mark 7}{(2.731,2.569)}
\gppoint{gp mark 7}{(2.735,2.571)}
\gppoint{gp mark 7}{(2.738,2.572)}
\gppoint{gp mark 7}{(2.741,2.574)}
\gppoint{gp mark 7}{(2.745,2.574)}
\gppoint{gp mark 7}{(2.748,2.572)}
\gppoint{gp mark 7}{(2.752,2.567)}
\gppoint{gp mark 7}{(2.755,2.563)}
\gppoint{gp mark 7}{(2.758,2.561)}
\gppoint{gp mark 7}{(2.762,2.561)}
\gppoint{gp mark 7}{(2.765,2.561)}
\gppoint{gp mark 7}{(2.768,2.564)}
\gppoint{gp mark 7}{(2.772,2.567)}
\gppoint{gp mark 7}{(2.775,2.567)}
\gppoint{gp mark 7}{(2.779,2.566)}
\gppoint{gp mark 7}{(2.782,2.563)}
\gppoint{gp mark 7}{(2.785,2.559)}
\gppoint{gp mark 7}{(2.789,2.557)}
\gppoint{gp mark 7}{(2.792,2.557)}
\gppoint{gp mark 7}{(2.796,2.557)}
\gppoint{gp mark 7}{(2.799,2.558)}
\gppoint{gp mark 7}{(2.802,2.559)}
\gppoint{gp mark 7}{(2.806,2.560)}
\gppoint{gp mark 7}{(2.809,2.558)}
\gppoint{gp mark 7}{(2.812,2.555)}
\gppoint{gp mark 7}{(2.816,2.554)}
\gppoint{gp mark 7}{(2.819,2.554)}
\gppoint{gp mark 7}{(2.823,2.555)}
\gppoint{gp mark 7}{(2.826,2.560)}
\gppoint{gp mark 7}{(2.829,2.567)}
\gppoint{gp mark 7}{(2.833,2.569)}
\gppoint{gp mark 7}{(2.836,2.571)}
\gppoint{gp mark 7}{(2.840,2.571)}
\gppoint{gp mark 7}{(2.843,2.570)}
\gppoint{gp mark 7}{(2.846,2.569)}
\gppoint{gp mark 7}{(2.850,2.568)}
\gppoint{gp mark 7}{(2.853,2.567)}
\gppoint{gp mark 7}{(2.856,2.565)}
\gppoint{gp mark 7}{(2.860,2.563)}
\gppoint{gp mark 7}{(2.863,2.563)}
\gppoint{gp mark 7}{(2.867,2.561)}
\gppoint{gp mark 7}{(2.870,2.559)}
\gppoint{gp mark 7}{(2.873,2.559)}
\gppoint{gp mark 7}{(2.877,2.560)}
\gppoint{gp mark 7}{(2.880,2.561)}
\gppoint{gp mark 7}{(2.884,2.563)}
\gppoint{gp mark 7}{(2.887,2.563)}
\gppoint{gp mark 7}{(2.890,2.562)}
\gppoint{gp mark 7}{(2.894,2.560)}
\gppoint{gp mark 7}{(2.897,2.557)}
\gppoint{gp mark 7}{(2.901,2.556)}
\gppoint{gp mark 7}{(2.904,2.555)}
\gppoint{gp mark 7}{(2.907,2.556)}
\gppoint{gp mark 7}{(2.911,2.558)}
\gppoint{gp mark 7}{(2.914,2.559)}
\gppoint{gp mark 7}{(2.917,2.559)}
\gppoint{gp mark 7}{(2.921,2.557)}
\gppoint{gp mark 7}{(2.924,2.553)}
\gppoint{gp mark 7}{(2.928,2.551)}
\gppoint{gp mark 7}{(2.931,2.550)}
\gppoint{gp mark 7}{(2.934,2.551)}
\gppoint{gp mark 7}{(2.938,2.555)}
\gppoint{gp mark 7}{(2.941,2.559)}
\gppoint{gp mark 7}{(2.945,2.563)}
\gppoint{gp mark 7}{(2.948,2.564)}
\gppoint{gp mark 7}{(2.951,2.564)}
\gppoint{gp mark 7}{(2.955,2.562)}
\gppoint{gp mark 7}{(2.958,2.560)}
\gppoint{gp mark 7}{(2.961,2.560)}
\gppoint{gp mark 7}{(2.965,2.560)}
\gppoint{gp mark 7}{(2.968,2.561)}
\gppoint{gp mark 7}{(2.972,2.562)}
\gppoint{gp mark 7}{(2.975,2.562)}
\gppoint{gp mark 7}{(2.978,2.561)}
\gppoint{gp mark 7}{(2.982,2.560)}
\gppoint{gp mark 7}{(2.985,2.560)}
\gppoint{gp mark 7}{(2.989,2.559)}
\gppoint{gp mark 7}{(2.992,2.559)}
\gppoint{gp mark 7}{(2.995,2.558)}
\gppoint{gp mark 7}{(2.999,2.557)}
\gppoint{gp mark 7}{(3.002,2.554)}
\gppoint{gp mark 7}{(3.005,2.552)}
\gppoint{gp mark 7}{(3.009,2.553)}
\gppoint{gp mark 7}{(3.012,2.553)}
\gppoint{gp mark 7}{(3.016,2.554)}
\gppoint{gp mark 7}{(3.019,2.557)}
\gppoint{gp mark 7}{(3.022,2.558)}
\gppoint{gp mark 7}{(3.026,2.558)}
\gppoint{gp mark 7}{(3.029,2.556)}
\gppoint{gp mark 7}{(3.033,2.552)}
\gppoint{gp mark 7}{(3.036,2.550)}
\gppoint{gp mark 7}{(3.039,2.549)}
\gppoint{gp mark 7}{(3.043,2.549)}
\gppoint{gp mark 7}{(3.046,2.550)}
\gppoint{gp mark 7}{(3.049,2.553)}
\gppoint{gp mark 7}{(3.053,2.554)}
\gppoint{gp mark 7}{(3.056,2.554)}
\gppoint{gp mark 7}{(3.060,2.554)}
\gppoint{gp mark 7}{(3.063,2.554)}
\gppoint{gp mark 7}{(3.066,2.554)}
\gppoint{gp mark 7}{(3.070,2.554)}
\gppoint{gp mark 7}{(3.073,2.556)}
\gppoint{gp mark 7}{(3.077,2.557)}
\gppoint{gp mark 7}{(3.080,2.557)}
\gppoint{gp mark 7}{(3.083,2.557)}
\gppoint{gp mark 7}{(3.087,2.557)}
\gppoint{gp mark 7}{(3.090,2.554)}
\gppoint{gp mark 7}{(3.094,2.553)}
\gppoint{gp mark 7}{(3.097,2.554)}
\gppoint{gp mark 7}{(3.100,2.554)}
\gppoint{gp mark 7}{(3.104,2.555)}
\gppoint{gp mark 7}{(3.107,2.557)}
\gppoint{gp mark 7}{(3.110,2.557)}
\gppoint{gp mark 7}{(3.114,2.556)}
\gppoint{gp mark 7}{(3.117,2.556)}
\gppoint{gp mark 7}{(3.121,2.555)}
\gppoint{gp mark 7}{(3.124,2.555)}
\gppoint{gp mark 7}{(3.127,2.555)}
\gppoint{gp mark 7}{(3.131,2.556)}
\gppoint{gp mark 7}{(3.134,2.556)}
\gppoint{gp mark 7}{(3.138,2.556)}
\gppoint{gp mark 7}{(3.141,2.556)}
\gppoint{gp mark 7}{(3.144,2.557)}
\gppoint{gp mark 7}{(3.148,2.557)}
\gppoint{gp mark 7}{(3.151,2.559)}
\gppoint{gp mark 7}{(3.154,2.561)}
\gppoint{gp mark 7}{(3.158,2.561)}
\gppoint{gp mark 7}{(3.161,2.561)}
\gppoint{gp mark 7}{(3.165,2.560)}
\gppoint{gp mark 7}{(3.168,2.558)}
\gppoint{gp mark 7}{(3.171,2.557)}
\gppoint{gp mark 7}{(3.175,2.557)}
\gppoint{gp mark 7}{(3.178,2.559)}
\gppoint{gp mark 7}{(3.182,2.562)}
\gppoint{gp mark 7}{(3.185,2.567)}
\gppoint{gp mark 7}{(3.188,2.569)}
\gppoint{gp mark 7}{(3.192,2.569)}
\gppoint{gp mark 7}{(3.195,2.569)}
\gppoint{gp mark 7}{(3.198,2.567)}
\gppoint{gp mark 7}{(3.202,2.565)}
\gppoint{gp mark 7}{(3.205,2.564)}
\gppoint{gp mark 7}{(3.209,2.565)}
\gppoint{gp mark 7}{(3.212,2.566)}
\gppoint{gp mark 7}{(3.215,2.568)}
\gppoint{gp mark 7}{(3.219,2.570)}
\gppoint{gp mark 7}{(3.222,2.570)}
\gppoint{gp mark 7}{(3.226,2.569)}
\gppoint{gp mark 7}{(3.229,2.569)}
\gppoint{gp mark 7}{(3.232,2.569)}
\gppoint{gp mark 7}{(3.236,2.568)}
\gppoint{gp mark 7}{(3.239,2.566)}
\gppoint{gp mark 7}{(3.243,2.565)}
\gppoint{gp mark 7}{(3.246,2.563)}
\gppoint{gp mark 7}{(3.249,2.561)}
\gppoint{gp mark 7}{(3.253,2.560)}
\gppoint{gp mark 7}{(3.256,2.559)}
\gppoint{gp mark 7}{(3.259,2.559)}
\gppoint{gp mark 7}{(3.263,2.559)}
\gppoint{gp mark 7}{(3.266,2.559)}
\gppoint{gp mark 7}{(3.270,2.557)}
\gppoint{gp mark 7}{(3.273,2.556)}
\gppoint{gp mark 7}{(3.276,2.555)}
\gppoint{gp mark 7}{(3.280,2.555)}
\gppoint{gp mark 7}{(3.283,2.557)}
\gppoint{gp mark 7}{(3.287,2.560)}
\gppoint{gp mark 7}{(3.290,2.562)}
\gppoint{gp mark 7}{(3.293,2.565)}
\gppoint{gp mark 7}{(3.297,2.567)}
\gppoint{gp mark 7}{(3.300,2.567)}
\gppoint{gp mark 7}{(3.303,2.567)}
\gppoint{gp mark 7}{(3.307,2.567)}
\gppoint{gp mark 7}{(3.310,2.569)}
\gppoint{gp mark 7}{(3.314,2.573)}
\gppoint{gp mark 7}{(3.317,2.579)}
\gppoint{gp mark 7}{(3.320,2.583)}
\gppoint{gp mark 7}{(3.324,2.584)}
\gppoint{gp mark 7}{(3.327,2.584)}
\gppoint{gp mark 7}{(3.331,2.584)}
\gppoint{gp mark 7}{(3.334,2.600)}
\gppoint{gp mark 7}{(3.337,2.863)}
\gppoint{gp mark 7}{(3.341,4.441)}
\gppoint{gp mark 7}{(3.344,5.520)}
\gppoint{gp mark 7}{(3.347,5.550)}
\gppoint{gp mark 7}{(3.351,5.559)}
\gppoint{gp mark 7}{(3.354,5.569)}
\gppoint{gp mark 7}{(3.358,5.569)}
\gppoint{gp mark 7}{(3.361,5.564)}
\gppoint{gp mark 7}{(3.364,5.561)}
\gppoint{gp mark 7}{(3.368,5.556)}
\gppoint{gp mark 7}{(3.371,5.557)}
\gppoint{gp mark 7}{(3.375,5.567)}
\gppoint{gp mark 7}{(3.378,5.574)}
\gppoint{gp mark 7}{(3.381,5.574)}
\gppoint{gp mark 7}{(3.385,5.571)}
\gppoint{gp mark 7}{(3.388,5.562)}
\gppoint{gp mark 7}{(3.392,5.555)}
\gppoint{gp mark 7}{(3.395,5.556)}
\gppoint{gp mark 7}{(3.398,5.563)}
\gppoint{gp mark 7}{(3.402,5.566)}
\gppoint{gp mark 7}{(3.405,5.565)}
\gppoint{gp mark 7}{(3.408,5.559)}
\gppoint{gp mark 7}{(3.412,5.546)}
\gppoint{gp mark 7}{(3.415,5.540)}
\gppoint{gp mark 7}{(3.419,5.541)}
\gppoint{gp mark 7}{(3.422,5.544)}
\gppoint{gp mark 7}{(3.425,5.546)}
\gppoint{gp mark 7}{(3.429,5.545)}
\gppoint{gp mark 7}{(3.432,5.538)}
\gppoint{gp mark 7}{(3.436,5.530)}
\gppoint{gp mark 7}{(3.439,5.529)}
\gppoint{gp mark 7}{(3.442,5.532)}
\gppoint{gp mark 7}{(3.446,5.541)}
\gppoint{gp mark 7}{(3.449,5.545)}
\gppoint{gp mark 7}{(3.452,5.545)}
\gppoint{gp mark 7}{(3.456,5.541)}
\gppoint{gp mark 7}{(3.459,5.536)}
\gppoint{gp mark 7}{(3.463,5.530)}
\gppoint{gp mark 7}{(3.466,5.520)}
\gppoint{gp mark 7}{(3.469,5.515)}
\gppoint{gp mark 7}{(3.473,5.515)}
\gppoint{gp mark 7}{(3.476,5.520)}
\gppoint{gp mark 7}{(3.480,5.532)}
\gppoint{gp mark 7}{(3.483,5.539)}
\gppoint{gp mark 7}{(3.486,5.541)}
\gppoint{gp mark 7}{(3.490,5.540)}
\gppoint{gp mark 7}{(3.493,5.538)}
\gppoint{gp mark 7}{(3.496,5.536)}
\gppoint{gp mark 7}{(3.500,5.533)}
\gppoint{gp mark 7}{(3.503,5.531)}
\gppoint{gp mark 7}{(3.507,5.528)}
\gppoint{gp mark 7}{(3.510,5.525)}
\gppoint{gp mark 7}{(3.513,5.523)}
\gppoint{gp mark 7}{(3.517,5.518)}
\gppoint{gp mark 7}{(3.520,5.513)}
\gppoint{gp mark 7}{(3.524,5.512)}
\gppoint{gp mark 7}{(3.527,5.512)}
\gppoint{gp mark 7}{(3.530,5.515)}
\gppoint{gp mark 7}{(3.534,5.525)}
\gppoint{gp mark 7}{(3.537,5.528)}
\gppoint{gp mark 7}{(3.540,5.527)}
\gppoint{gp mark 7}{(3.544,5.526)}
\gppoint{gp mark 7}{(3.547,5.525)}
\gppoint{gp mark 7}{(3.551,5.522)}
\gppoint{gp mark 7}{(3.554,5.521)}
\gppoint{gp mark 7}{(3.557,5.520)}
\gppoint{gp mark 7}{(3.561,5.516)}
\gppoint{gp mark 7}{(3.564,5.511)}
\gppoint{gp mark 7}{(3.568,5.507)}
\gppoint{gp mark 7}{(3.571,5.506)}
\gppoint{gp mark 7}{(3.574,5.511)}
\gppoint{gp mark 7}{(3.578,5.522)}
\gppoint{gp mark 7}{(3.581,5.528)}
\gppoint{gp mark 7}{(3.585,5.528)}
\gppoint{gp mark 7}{(3.588,5.528)}
\gppoint{gp mark 7}{(3.591,5.527)}
\gppoint{gp mark 7}{(3.595,5.526)}
\gppoint{gp mark 7}{(3.598,5.526)}
\gppoint{gp mark 7}{(3.601,5.525)}
\gppoint{gp mark 7}{(3.605,5.524)}
\gppoint{gp mark 7}{(3.608,5.522)}
\gppoint{gp mark 7}{(3.612,5.522)}
\gppoint{gp mark 7}{(3.615,5.520)}
\gppoint{gp mark 7}{(3.618,5.519)}
\gppoint{gp mark 7}{(3.622,5.519)}
\gppoint{gp mark 7}{(3.625,5.520)}
\gppoint{gp mark 7}{(3.629,5.521)}
\gppoint{gp mark 7}{(3.632,5.524)}
\gppoint{gp mark 7}{(3.635,5.525)}
\gppoint{gp mark 7}{(3.639,5.527)}
\gppoint{gp mark 7}{(3.642,5.530)}
\gppoint{gp mark 7}{(3.645,5.531)}
\gppoint{gp mark 7}{(3.649,5.532)}
\gppoint{gp mark 7}{(3.652,5.533)}
\gppoint{gp mark 7}{(3.656,5.535)}
\gppoint{gp mark 7}{(3.659,5.534)}
\gppoint{gp mark 7}{(3.662,5.534)}
\gppoint{gp mark 7}{(3.666,5.532)}
\gppoint{gp mark 7}{(3.669,5.531)}
\gppoint{gp mark 7}{(3.673,5.529)}
\gppoint{gp mark 7}{(3.676,5.525)}
\gppoint{gp mark 7}{(3.679,5.510)}
\gppoint{gp mark 7}{(3.683,5.493)}
\gppoint{gp mark 7}{(3.686,5.486)}
\gppoint{gp mark 7}{(3.689,5.485)}
\gppoint{gp mark 7}{(3.693,5.482)}
\gppoint{gp mark 7}{(3.696,5.474)}
\gppoint{gp mark 7}{(3.700,5.466)}
\gppoint{gp mark 7}{(3.703,5.463)}
\gppoint{gp mark 7}{(3.706,5.459)}
\gppoint{gp mark 7}{(3.710,5.453)}
\gppoint{gp mark 7}{(3.713,5.447)}
\gppoint{gp mark 7}{(3.717,5.447)}
\gppoint{gp mark 7}{(3.720,5.448)}
\gppoint{gp mark 7}{(3.723,5.455)}
\gppoint{gp mark 7}{(3.727,5.465)}
\gppoint{gp mark 7}{(3.730,5.475)}
\gppoint{gp mark 7}{(3.734,5.480)}
\gppoint{gp mark 7}{(3.737,5.482)}
\gppoint{gp mark 7}{(3.740,5.486)}
\gppoint{gp mark 7}{(3.744,5.499)}
\gppoint{gp mark 7}{(3.747,5.517)}
\gppoint{gp mark 7}{(3.750,5.526)}
\gppoint{gp mark 7}{(3.754,5.530)}
\gppoint{gp mark 7}{(3.757,5.529)}
\gppoint{gp mark 7}{(3.761,5.526)}
\gppoint{gp mark 7}{(3.764,5.510)}
\gppoint{gp mark 7}{(3.767,5.489)}
\gppoint{gp mark 7}{(3.771,5.471)}
\gppoint{gp mark 7}{(3.774,5.442)}
\gppoint{gp mark 7}{(3.778,5.383)}
\gppoint{gp mark 7}{(3.781,5.320)}
\gppoint{gp mark 7}{(3.784,5.292)}
\gppoint{gp mark 7}{(3.788,5.291)}
\gppoint{gp mark 7}{(3.791,5.294)}
\gppoint{gp mark 7}{(3.794,5.313)}
\gppoint{gp mark 7}{(3.798,5.337)}
\gppoint{gp mark 7}{(3.801,5.340)}
\gppoint{gp mark 7}{(3.805,5.334)}
\gppoint{gp mark 7}{(3.808,5.311)}
\gppoint{gp mark 7}{(3.811,5.286)}
\gppoint{gp mark 7}{(3.815,5.277)}
\gppoint{gp mark 7}{(3.818,5.273)}
\gppoint{gp mark 7}{(3.822,5.270)}
\gppoint{gp mark 7}{(3.825,5.246)}
\gppoint{gp mark 7}{(3.828,5.055)}
\gppoint{gp mark 7}{(3.832,4.712)}
\gppoint{gp mark 7}{(3.835,4.536)}
\gppoint{gp mark 7}{(3.838,4.341)}
\gppoint{gp mark 7}{(3.842,3.778)}
\gppoint{gp mark 7}{(3.845,2.772)}
\gppoint{gp mark 7}{(3.849,1.951)}
\gppoint{gp mark 7}{(3.852,1.648)}
\gppoint{gp mark 7}{(3.855,1.607)}
\gppoint{gp mark 7}{(3.859,1.612)}
\gppoint{gp mark 7}{(3.862,1.675)}
\gppoint{gp mark 7}{(3.866,1.883)}
\gppoint{gp mark 7}{(3.869,2.167)}
\gppoint{gp mark 7}{(3.872,2.435)}
\gppoint{gp mark 7}{(3.876,2.629)}
\gppoint{gp mark 7}{(3.879,2.754)}
\gppoint{gp mark 7}{(3.883,2.857)}
\gppoint{gp mark 7}{(3.886,2.966)}
\gppoint{gp mark 7}{(3.889,3.078)}
\gppoint{gp mark 7}{(3.893,3.154)}
\gppoint{gp mark 7}{(3.896,3.184)}
\gppoint{gp mark 7}{(3.899,3.191)}
\gppoint{gp mark 7}{(3.903,3.191)}
\gppoint{gp mark 7}{(3.906,3.190)}
\gppoint{gp mark 7}{(3.910,3.190)}
\gppoint{gp mark 7}{(3.913,3.192)}
\gppoint{gp mark 7}{(3.916,3.201)}
\gppoint{gp mark 7}{(3.920,3.213)}
\gppoint{gp mark 7}{(3.923,3.228)}
\gppoint{gp mark 7}{(3.927,3.242)}
\gppoint{gp mark 7}{(3.930,3.259)}
\gppoint{gp mark 7}{(3.933,3.290)}
\gppoint{gp mark 7}{(3.937,3.334)}
\gppoint{gp mark 7}{(3.940,3.364)}
\gppoint{gp mark 7}{(3.943,3.376)}
\gppoint{gp mark 7}{(3.947,3.386)}
\gppoint{gp mark 7}{(3.950,3.409)}
\gppoint{gp mark 7}{(3.954,3.448)}
\gppoint{gp mark 7}{(3.957,3.475)}
\gppoint{gp mark 7}{(3.960,3.483)}
\gppoint{gp mark 7}{(3.964,3.484)}
\gppoint{gp mark 7}{(3.967,3.483)}
\gppoint{gp mark 7}{(3.971,3.479)}
\gppoint{gp mark 7}{(3.974,3.478)}
\gppoint{gp mark 7}{(3.977,3.478)}
\gppoint{gp mark 7}{(3.981,3.477)}
\gppoint{gp mark 7}{(3.984,3.475)}
\gppoint{gp mark 7}{(3.987,3.474)}
\gppoint{gp mark 7}{(3.991,3.475)}
\gppoint{gp mark 7}{(3.994,3.477)}
\gppoint{gp mark 7}{(3.998,3.489)}
\gppoint{gp mark 7}{(4.001,3.510)}
\gppoint{gp mark 7}{(4.004,3.524)}
\gppoint{gp mark 7}{(4.008,3.528)}
\gppoint{gp mark 7}{(4.011,3.530)}
\gppoint{gp mark 7}{(4.015,3.534)}
\gppoint{gp mark 7}{(4.018,3.538)}
\gppoint{gp mark 7}{(4.021,3.540)}
\gppoint{gp mark 7}{(4.025,3.541)}
\gppoint{gp mark 7}{(4.028,3.541)}
\gppoint{gp mark 7}{(4.032,3.540)}
\gppoint{gp mark 7}{(4.035,3.533)}
\gppoint{gp mark 7}{(4.038,3.526)}
\gppoint{gp mark 7}{(4.042,3.522)}
\gppoint{gp mark 7}{(4.045,3.520)}
\gppoint{gp mark 7}{(4.048,3.519)}
\gppoint{gp mark 7}{(4.052,3.516)}
\gppoint{gp mark 7}{(4.055,3.515)}
\gppoint{gp mark 7}{(4.059,3.513)}
\gppoint{gp mark 7}{(4.062,3.513)}
\gppoint{gp mark 7}{(4.065,3.514)}
\gppoint{gp mark 7}{(4.069,3.514)}
\gppoint{gp mark 7}{(4.072,3.515)}
\gppoint{gp mark 7}{(4.076,3.518)}
\gppoint{gp mark 7}{(4.079,3.521)}
\gppoint{gp mark 7}{(4.082,3.522)}
\gppoint{gp mark 7}{(4.086,3.523)}
\gppoint{gp mark 7}{(4.089,3.525)}
\gppoint{gp mark 7}{(4.092,3.527)}
\gppoint{gp mark 7}{(4.096,3.528)}
\gppoint{gp mark 7}{(4.099,3.528)}
\gppoint{gp mark 7}{(4.103,3.529)}
\gppoint{gp mark 7}{(4.106,3.530)}
\gppoint{gp mark 7}{(4.109,3.532)}
\gppoint{gp mark 7}{(4.113,3.532)}
\gppoint{gp mark 7}{(4.116,3.532)}
\gppoint{gp mark 7}{(4.120,3.532)}
\gppoint{gp mark 7}{(4.123,3.533)}
\gppoint{gp mark 7}{(4.126,3.536)}
\gppoint{gp mark 7}{(4.130,3.538)}
\gppoint{gp mark 7}{(4.133,3.538)}
\gppoint{gp mark 7}{(4.136,3.537)}
\gppoint{gp mark 7}{(4.140,3.537)}
\gppoint{gp mark 7}{(4.143,3.538)}
\gppoint{gp mark 7}{(4.147,3.542)}
\gppoint{gp mark 7}{(4.150,3.549)}
\gppoint{gp mark 7}{(4.153,3.551)}
\gppoint{gp mark 7}{(4.157,3.550)}
\gppoint{gp mark 7}{(4.160,3.548)}
\gppoint{gp mark 7}{(4.164,3.546)}
\gppoint{gp mark 7}{(4.167,3.545)}
\gppoint{gp mark 7}{(4.170,3.545)}
\gppoint{gp mark 7}{(4.174,3.544)}
\gppoint{gp mark 7}{(4.177,3.544)}
\gppoint{gp mark 7}{(4.180,3.544)}
\gppoint{gp mark 7}{(4.184,3.545)}
\gppoint{gp mark 7}{(4.187,3.544)}
\gppoint{gp mark 7}{(4.191,3.542)}
\gppoint{gp mark 7}{(4.194,3.541)}
\gppoint{gp mark 7}{(4.197,3.541)}
\gppoint{gp mark 7}{(4.201,3.545)}
\gppoint{gp mark 7}{(4.204,3.547)}
\gppoint{gp mark 7}{(4.208,3.547)}
\gppoint{gp mark 7}{(4.211,3.545)}
\gppoint{gp mark 7}{(4.214,3.543)}
\gppoint{gp mark 7}{(4.218,3.544)}
\gppoint{gp mark 7}{(4.221,3.545)}
\gppoint{gp mark 7}{(4.225,3.545)}
\gppoint{gp mark 7}{(4.228,3.545)}
\gppoint{gp mark 7}{(4.231,3.545)}
\gppoint{gp mark 7}{(4.235,3.546)}
\gppoint{gp mark 7}{(4.238,3.545)}
\gppoint{gp mark 7}{(4.241,3.542)}
\gppoint{gp mark 7}{(4.245,3.540)}
\gppoint{gp mark 7}{(4.248,3.540)}
\gppoint{gp mark 7}{(4.252,3.540)}
\gppoint{gp mark 7}{(4.255,3.540)}
\gppoint{gp mark 7}{(4.258,3.539)}
\gppoint{gp mark 7}{(4.262,3.539)}
\gppoint{gp mark 7}{(4.265,3.541)}
\gppoint{gp mark 7}{(4.269,3.547)}
\gppoint{gp mark 7}{(4.272,3.551)}
\gppoint{gp mark 7}{(4.275,3.551)}
\gppoint{gp mark 7}{(4.279,3.549)}
\gppoint{gp mark 7}{(4.282,3.547)}
\gppoint{gp mark 7}{(4.285,3.547)}
\gppoint{gp mark 7}{(4.289,3.547)}
\gppoint{gp mark 7}{(4.292,3.547)}
\gppoint{gp mark 7}{(4.296,3.546)}
\gppoint{gp mark 7}{(4.299,3.546)}
\gppoint{gp mark 7}{(4.302,3.549)}
\gppoint{gp mark 7}{(4.306,3.555)}
\gppoint{gp mark 7}{(4.309,3.558)}
\gppoint{gp mark 7}{(4.313,3.558)}
\gppoint{gp mark 7}{(4.316,3.556)}
\gppoint{gp mark 7}{(4.319,3.552)}
\gppoint{gp mark 7}{(4.323,3.552)}
\gppoint{gp mark 7}{(4.326,3.553)}
\gppoint{gp mark 7}{(4.329,3.553)}
\gppoint{gp mark 7}{(4.333,3.551)}
\gppoint{gp mark 7}{(4.336,3.551)}
\gppoint{gp mark 7}{(4.340,3.553)}
\gppoint{gp mark 7}{(4.343,3.554)}
\gppoint{gp mark 7}{(4.346,3.553)}
\gppoint{gp mark 7}{(4.350,3.551)}
\gppoint{gp mark 7}{(4.353,3.552)}
\gppoint{gp mark 7}{(4.357,3.553)}
\gppoint{gp mark 7}{(4.360,3.552)}
\gppoint{gp mark 7}{(4.363,3.548)}
\gppoint{gp mark 7}{(4.367,3.549)}
\gppoint{gp mark 7}{(4.370,3.553)}
\gppoint{gp mark 7}{(4.374,3.554)}
\gppoint{gp mark 7}{(4.377,3.551)}
\gppoint{gp mark 7}{(4.380,3.549)}
\gppoint{gp mark 7}{(4.384,3.551)}
\gppoint{gp mark 7}{(4.387,3.553)}
\gppoint{gp mark 7}{(4.390,3.551)}
\gppoint{gp mark 7}{(4.394,3.548)}
\gppoint{gp mark 7}{(4.397,3.549)}
\gppoint{gp mark 7}{(4.401,3.552)}
\gppoint{gp mark 7}{(4.404,3.553)}
\gppoint{gp mark 7}{(4.407,3.549)}
\gppoint{gp mark 7}{(4.411,3.549)}
\gppoint{gp mark 7}{(4.414,3.554)}
\gppoint{gp mark 7}{(4.418,3.553)}
\gppoint{gp mark 7}{(4.421,3.363)}
\gppoint{gp mark 7}{(4.424,2.012)}
\gppoint{gp mark 7}{(4.428,1.367)}
\gppoint{gp mark 7}{(4.431,1.322)}
\gppoint{gp mark 7}{(4.434,1.319)}
\gppoint{gp mark 7}{(4.438,1.320)}
\gppoint{gp mark 7}{(4.441,1.320)}
\gppoint{gp mark 7}{(4.445,1.321)}
\gppoint{gp mark 7}{(4.448,1.321)}
\gppoint{gp mark 7}{(4.451,1.321)}
\gppoint{gp mark 7}{(4.455,1.320)}
\gppoint{gp mark 7}{(4.458,1.318)}
\gppoint{gp mark 7}{(4.462,1.317)}
\gppoint{gp mark 7}{(4.465,1.317)}
\gppoint{gp mark 7}{(4.468,1.317)}
\gppoint{gp mark 7}{(4.472,1.317)}
\gppoint{gp mark 7}{(4.475,1.317)}
\gppoint{gp mark 7}{(4.478,1.319)}
\gppoint{gp mark 7}{(4.482,1.320)}
\gppoint{gp mark 7}{(4.485,1.321)}
\gppoint{gp mark 7}{(4.489,1.321)}
\gppoint{gp mark 7}{(4.492,1.321)}
\gppoint{gp mark 7}{(4.495,1.321)}
\gppoint{gp mark 7}{(4.499,1.321)}
\gppoint{gp mark 7}{(4.502,1.321)}
\gppoint{gp mark 7}{(4.506,1.321)}
\gppoint{gp mark 7}{(4.509,1.321)}
\gppoint{gp mark 7}{(4.512,1.321)}
\gppoint{gp mark 7}{(4.516,1.320)}
\gppoint{gp mark 7}{(4.519,1.319)}
\gppoint{gp mark 7}{(4.523,1.319)}
\gppoint{gp mark 7}{(4.526,1.319)}
\gppoint{gp mark 7}{(4.529,1.319)}
\gppoint{gp mark 7}{(4.533,1.319)}
\gppoint{gp mark 7}{(4.536,1.320)}
\gppoint{gp mark 7}{(4.539,1.322)}
\gppoint{gp mark 7}{(4.543,1.323)}
\gppoint{gp mark 7}{(4.546,1.323)}
\gppoint{gp mark 7}{(4.550,1.323)}
\gppoint{gp mark 7}{(4.553,1.323)}
\gppoint{gp mark 7}{(4.556,1.322)}
\gppoint{gp mark 7}{(4.560,1.321)}
\gppoint{gp mark 7}{(4.563,1.319)}
\gppoint{gp mark 7}{(4.567,1.319)}
\gppoint{gp mark 7}{(4.570,1.318)}
\gppoint{gp mark 7}{(4.573,1.318)}
\gppoint{gp mark 7}{(4.577,1.319)}
\gppoint{gp mark 7}{(4.580,1.320)}
\gppoint{gp mark 7}{(4.583,1.321)}
\gppoint{gp mark 7}{(4.587,1.322)}
\gppoint{gp mark 7}{(4.590,1.322)}
\gppoint{gp mark 7}{(4.594,1.322)}
\gppoint{gp mark 7}{(4.597,1.322)}
\gppoint{gp mark 7}{(4.600,1.321)}
\gppoint{gp mark 7}{(4.604,1.319)}
\gppoint{gp mark 7}{(4.607,1.319)}
\gppoint{gp mark 7}{(4.611,1.319)}
\gppoint{gp mark 7}{(4.614,1.318)}
\gppoint{gp mark 7}{(4.617,1.319)}
\gppoint{gp mark 7}{(4.621,1.319)}
\gppoint{gp mark 7}{(4.624,1.320)}
\gppoint{gp mark 7}{(4.627,1.321)}
\gppoint{gp mark 7}{(4.631,1.321)}
\gppoint{gp mark 7}{(4.634,1.321)}
\gppoint{gp mark 7}{(4.638,1.321)}
\gppoint{gp mark 7}{(4.641,1.320)}
\gppoint{gp mark 7}{(4.644,1.319)}
\gppoint{gp mark 7}{(4.648,1.319)}
\gppoint{gp mark 7}{(4.651,1.319)}
\gppoint{gp mark 7}{(4.655,1.319)}
\gppoint{gp mark 7}{(4.658,1.319)}
\gppoint{gp mark 7}{(4.661,1.320)}
\gppoint{gp mark 7}{(4.665,1.320)}
\gppoint{gp mark 7}{(4.668,1.320)}
\gppoint{gp mark 7}{(4.671,1.321)}
\gppoint{gp mark 7}{(4.675,1.321)}
\gppoint{gp mark 7}{(4.678,1.320)}
\gppoint{gp mark 7}{(4.682,1.319)}
\gppoint{gp mark 7}{(4.685,1.319)}
\gppoint{gp mark 7}{(4.688,1.319)}
\gppoint{gp mark 7}{(4.692,1.318)}
\gppoint{gp mark 7}{(4.695,1.318)}
\gppoint{gp mark 7}{(4.699,1.319)}
\gppoint{gp mark 7}{(4.702,1.320)}
\gppoint{gp mark 7}{(4.705,1.321)}
\gppoint{gp mark 7}{(4.709,1.322)}
\gppoint{gp mark 7}{(4.712,1.322)}
\gppoint{gp mark 7}{(4.716,1.322)}
\gppoint{gp mark 7}{(4.719,1.321)}
\gppoint{gp mark 7}{(4.722,1.322)}
\gppoint{gp mark 7}{(4.726,1.322)}
\gppoint{gp mark 7}{(4.729,1.322)}
\gppoint{gp mark 7}{(4.732,1.322)}
\gppoint{gp mark 7}{(4.736,1.322)}
\gppoint{gp mark 7}{(4.739,1.321)}
\gppoint{gp mark 7}{(4.743,1.320)}
\gppoint{gp mark 7}{(4.746,1.319)}
\gppoint{gp mark 7}{(4.749,1.318)}
\gppoint{gp mark 7}{(4.753,1.318)}
\gppoint{gp mark 7}{(4.756,1.318)}
\gppoint{gp mark 7}{(4.760,1.319)}
\gppoint{gp mark 7}{(4.763,1.320)}
\gppoint{gp mark 7}{(4.766,1.322)}
\gppoint{gp mark 7}{(4.770,1.323)}
\gppoint{gp mark 7}{(4.773,1.324)}
\gppoint{gp mark 7}{(4.776,1.324)}
\gppoint{gp mark 7}{(4.780,1.324)}
\gppoint{gp mark 7}{(4.783,1.323)}
\gppoint{gp mark 7}{(4.787,1.322)}
\gppoint{gp mark 7}{(4.790,1.320)}
\gppoint{gp mark 7}{(4.793,1.319)}
\gppoint{gp mark 7}{(4.797,1.318)}
\gppoint{gp mark 7}{(4.800,1.316)}
\gppoint{gp mark 7}{(4.804,1.315)}
\gppoint{gp mark 7}{(4.807,1.314)}
\gppoint{gp mark 7}{(4.810,1.314)}
\gppoint{gp mark 7}{(4.814,1.314)}
\gppoint{gp mark 7}{(4.817,1.315)}
\gppoint{gp mark 7}{(4.820,1.317)}
\gppoint{gp mark 7}{(4.824,1.318)}
\gppoint{gp mark 7}{(4.827,1.320)}
\gppoint{gp mark 7}{(4.831,1.321)}
\gppoint{gp mark 7}{(4.834,1.321)}
\gppoint{gp mark 7}{(4.837,1.321)}
\gppoint{gp mark 7}{(4.841,1.320)}
\gppoint{gp mark 7}{(4.844,1.319)}
\gppoint{gp mark 7}{(4.848,1.317)}
\gppoint{gp mark 7}{(4.851,1.316)}
\gppoint{gp mark 7}{(4.854,1.316)}
\gppoint{gp mark 7}{(4.858,1.316)}
\gppoint{gp mark 7}{(4.861,1.317)}
\gppoint{gp mark 7}{(4.865,1.318)}
\gppoint{gp mark 7}{(4.868,1.319)}
\gppoint{gp mark 7}{(4.871,1.319)}
\gppoint{gp mark 7}{(4.875,1.319)}
\gppoint{gp mark 7}{(4.878,1.319)}
\gppoint{gp mark 7}{(4.881,1.318)}
\gppoint{gp mark 7}{(4.885,1.316)}
\gppoint{gp mark 7}{(4.888,1.315)}
\gppoint{gp mark 7}{(4.892,1.315)}
\gppoint{gp mark 7}{(4.895,1.315)}
\gppoint{gp mark 7}{(4.898,1.316)}
\gppoint{gp mark 7}{(4.902,1.317)}
\gppoint{gp mark 7}{(4.905,1.319)}
\gppoint{gp mark 7}{(4.909,1.321)}
\gppoint{gp mark 7}{(4.912,1.321)}
\gppoint{gp mark 7}{(4.915,1.321)}
\gppoint{gp mark 7}{(4.919,1.321)}
\gppoint{gp mark 7}{(4.922,1.320)}
\gppoint{gp mark 7}{(4.925,1.319)}
\gppoint{gp mark 7}{(4.929,1.318)}
\gppoint{gp mark 7}{(4.932,1.318)}
\gppoint{gp mark 7}{(4.936,1.318)}
\gppoint{gp mark 7}{(4.939,1.318)}
\gppoint{gp mark 7}{(4.942,1.318)}
\gppoint{gp mark 7}{(4.946,1.319)}
\gppoint{gp mark 7}{(4.949,1.320)}
\gppoint{gp mark 7}{(4.953,1.320)}
\gppoint{gp mark 7}{(4.956,1.320)}
\gppoint{gp mark 7}{(4.959,1.321)}
\gppoint{gp mark 7}{(4.963,1.322)}
\gppoint{gp mark 7}{(4.966,1.323)}
\gppoint{gp mark 7}{(4.969,1.324)}
\gppoint{gp mark 7}{(4.973,1.324)}
\gppoint{gp mark 7}{(4.976,1.324)}
\gppoint{gp mark 7}{(4.980,1.324)}
\gppoint{gp mark 7}{(4.983,1.323)}
\gppoint{gp mark 7}{(4.986,1.321)}
\gppoint{gp mark 7}{(4.990,1.318)}
\gppoint{gp mark 7}{(4.993,1.316)}
\gppoint{gp mark 7}{(4.997,1.314)}
\gppoint{gp mark 7}{(5.000,1.314)}
\gppoint{gp mark 7}{(5.003,1.314)}
\gppoint{gp mark 7}{(5.007,1.314)}
\gppoint{gp mark 7}{(5.010,1.314)}
\gppoint{gp mark 7}{(5.014,1.314)}
\gppoint{gp mark 7}{(5.017,1.315)}
\gppoint{gp mark 7}{(5.020,1.316)}
\gppoint{gp mark 7}{(5.024,1.317)}
\gppoint{gp mark 7}{(5.027,1.318)}
\gppoint{gp mark 7}{(5.030,1.319)}
\gppoint{gp mark 7}{(5.034,1.319)}
\gppoint{gp mark 7}{(5.037,1.318)}
\gppoint{gp mark 7}{(5.041,1.317)}
\gppoint{gp mark 7}{(5.044,1.316)}
\gppoint{gp mark 7}{(5.047,1.316)}
\gppoint{gp mark 7}{(5.051,1.315)}
\gppoint{gp mark 7}{(5.054,1.316)}
\gppoint{gp mark 7}{(5.058,1.316)}
\gppoint{gp mark 7}{(5.061,1.318)}
\gppoint{gp mark 7}{(5.064,1.320)}
\gppoint{gp mark 7}{(5.068,1.321)}
\gppoint{gp mark 7}{(5.071,1.323)}
\gppoint{gp mark 7}{(5.074,1.323)}
\gppoint{gp mark 7}{(5.078,1.323)}
\gppoint{gp mark 7}{(5.081,1.323)}
\gppoint{gp mark 7}{(5.085,1.323)}
\gppoint{gp mark 7}{(5.088,1.324)}
\gppoint{gp mark 7}{(5.091,1.325)}
\gppoint{gp mark 7}{(5.095,1.325)}
\gppoint{gp mark 7}{(5.098,1.325)}
\gppoint{gp mark 7}{(5.102,1.325)}
\gppoint{gp mark 7}{(5.105,1.324)}
\gppoint{gp mark 7}{(5.108,1.322)}
\gppoint{gp mark 7}{(5.112,1.320)}
\gppoint{gp mark 7}{(5.115,1.318)}
\gppoint{gp mark 7}{(5.118,1.317)}
\gppoint{gp mark 7}{(5.122,1.317)}
\gppoint{gp mark 7}{(5.125,1.317)}
\gppoint{gp mark 7}{(5.129,1.317)}
\gppoint{gp mark 7}{(5.132,1.317)}
\gppoint{gp mark 7}{(5.135,1.318)}
\gppoint{gp mark 7}{(5.139,1.318)}
\gppoint{gp mark 7}{(5.142,1.318)}
\gppoint{gp mark 7}{(5.146,1.319)}
\gppoint{gp mark 7}{(5.149,1.320)}
\gppoint{gp mark 7}{(5.152,1.320)}
\gppoint{gp mark 7}{(5.156,1.319)}
\gppoint{gp mark 7}{(5.159,1.319)}
\gppoint{gp mark 7}{(5.163,1.319)}
\gppoint{gp mark 7}{(5.166,1.319)}
\gppoint{gp mark 7}{(5.169,1.318)}
\gppoint{gp mark 7}{(5.173,1.317)}
\gppoint{gp mark 7}{(5.176,1.318)}
\gppoint{gp mark 7}{(5.179,1.318)}
\gppoint{gp mark 7}{(5.183,1.319)}
\gppoint{gp mark 7}{(5.186,1.319)}
\gppoint{gp mark 7}{(5.190,1.320)}
\gppoint{gp mark 7}{(5.193,1.321)}
\gppoint{gp mark 7}{(5.196,1.321)}
\gppoint{gp mark 7}{(5.200,1.321)}
\gppoint{gp mark 7}{(5.203,1.320)}
\gppoint{gp mark 7}{(5.207,1.319)}
\gppoint{gp mark 7}{(5.210,1.318)}
\gppoint{gp mark 7}{(5.213,1.318)}
\gppoint{gp mark 7}{(5.217,1.317)}
\gppoint{gp mark 7}{(5.220,1.317)}
\gppoint{gp mark 7}{(5.223,1.318)}
\gppoint{gp mark 7}{(5.227,1.319)}
\gppoint{gp mark 7}{(5.230,1.321)}
\gppoint{gp mark 7}{(5.234,1.322)}
\gppoint{gp mark 7}{(5.237,1.323)}
\gppoint{gp mark 7}{(5.240,1.323)}
\gppoint{gp mark 7}{(5.244,1.323)}
\gppoint{gp mark 7}{(5.247,1.322)}
\gppoint{gp mark 7}{(5.251,1.320)}
\gppoint{gp mark 7}{(5.254,1.319)}
\gppoint{gp mark 7}{(5.257,1.318)}
\gppoint{gp mark 7}{(5.261,1.318)}
\gppoint{gp mark 7}{(5.264,1.317)}
\gppoint{gp mark 7}{(5.267,1.316)}
\gppoint{gp mark 7}{(5.271,1.315)}
\gppoint{gp mark 7}{(5.274,1.315)}
\gppoint{gp mark 7}{(5.278,1.315)}
\gppoint{gp mark 7}{(5.281,1.315)}
\gppoint{gp mark 7}{(5.284,1.316)}
\gppoint{gp mark 7}{(5.288,1.317)}
\gppoint{gp mark 7}{(5.291,1.318)}
\gppoint{gp mark 7}{(5.295,1.319)}
\gppoint{gp mark 7}{(5.298,1.319)}
\gppoint{gp mark 7}{(5.301,1.318)}
\gppoint{gp mark 7}{(5.305,1.318)}
\gppoint{gp mark 7}{(5.308,1.317)}
\gppoint{gp mark 7}{(5.311,1.316)}
\gppoint{gp mark 7}{(5.315,1.316)}
\gppoint{gp mark 7}{(5.318,1.316)}
\gppoint{gp mark 7}{(5.322,1.318)}
\gppoint{gp mark 7}{(5.325,1.319)}
\gppoint{gp mark 7}{(5.328,1.322)}
\gppoint{gp mark 7}{(5.332,1.322)}
\gppoint{gp mark 7}{(5.335,1.323)}
\gppoint{gp mark 7}{(5.339,1.322)}
\gppoint{gp mark 7}{(5.342,1.322)}
\gppoint{gp mark 7}{(5.345,1.321)}
\gppoint{gp mark 7}{(5.349,1.319)}
\gppoint{gp mark 7}{(5.352,1.317)}
\gppoint{gp mark 7}{(5.356,1.317)}
\gppoint{gp mark 7}{(5.359,1.316)}
\gppoint{gp mark 7}{(5.362,1.316)}
\gppoint{gp mark 7}{(5.366,1.316)}
\gppoint{gp mark 7}{(5.369,1.316)}
\gppoint{gp mark 7}{(5.372,1.316)}
\gppoint{gp mark 7}{(5.376,1.316)}
\gppoint{gp mark 7}{(5.379,1.316)}
\gppoint{gp mark 7}{(5.383,1.316)}
\gppoint{gp mark 7}{(5.386,1.316)}
\gppoint{gp mark 7}{(5.389,1.317)}
\gppoint{gp mark 7}{(5.393,1.317)}
\gppoint{gp mark 7}{(5.396,1.317)}
\gppoint{gp mark 7}{(5.400,1.318)}
\gppoint{gp mark 7}{(5.403,1.318)}
\gppoint{gp mark 7}{(5.406,1.318)}
\gppoint{gp mark 7}{(5.410,1.317)}
\gppoint{gp mark 7}{(5.413,1.317)}
\gppoint{gp mark 7}{(5.416,1.317)}
\gppoint{gp mark 7}{(5.420,1.317)}
\gppoint{gp mark 7}{(5.423,1.318)}
\gppoint{gp mark 7}{(5.427,1.319)}
\gppoint{gp mark 7}{(5.430,1.321)}
\gppoint{gp mark 7}{(5.433,1.322)}
\gppoint{gp mark 7}{(5.437,1.322)}
\gppoint{gp mark 7}{(5.440,1.322)}
\gppoint{gp mark 7}{(5.444,1.322)}
\gppoint{gp mark 7}{(5.447,1.322)}
\gppoint{gp mark 7}{(5.450,1.321)}
\gppoint{gp mark 7}{(5.454,1.318)}
\gppoint{gp mark 7}{(5.457,1.318)}
\gppoint{gp mark 7}{(5.460,1.318)}
\gppoint{gp mark 7}{(5.464,1.317)}
\gppoint{gp mark 7}{(5.467,1.316)}
\gppoint{gp mark 7}{(5.471,1.315)}
\gppoint{gp mark 7}{(5.474,1.315)}
\gppoint{gp mark 7}{(5.477,1.316)}
\gppoint{gp mark 7}{(5.481,1.317)}
\gppoint{gp mark 7}{(5.484,1.318)}
\gppoint{gp mark 7}{(5.488,1.320)}
\gppoint{gp mark 7}{(5.491,1.320)}
\gppoint{gp mark 7}{(5.494,1.321)}
\gppoint{gp mark 7}{(5.498,1.320)}
\gppoint{gp mark 7}{(5.501,1.319)}
\gppoint{gp mark 7}{(5.505,1.317)}
\gppoint{gp mark 7}{(5.508,1.315)}
\gppoint{gp mark 7}{(5.511,1.315)}
\gppoint{gp mark 7}{(5.515,1.315)}
\gppoint{gp mark 7}{(5.518,1.315)}
\gppoint{gp mark 7}{(5.521,1.317)}
\gppoint{gp mark 7}{(5.525,1.320)}
\gppoint{gp mark 7}{(5.528,1.322)}
\gppoint{gp mark 7}{(5.532,1.324)}
\gppoint{gp mark 7}{(5.535,1.324)}
\gppoint{gp mark 7}{(5.538,1.324)}
\gppoint{gp mark 7}{(5.542,1.324)}
\gppoint{gp mark 7}{(5.545,1.323)}
\gppoint{gp mark 7}{(5.549,1.321)}
\gppoint{gp mark 7}{(5.552,1.320)}
\gppoint{gp mark 7}{(5.555,1.319)}
\gppoint{gp mark 7}{(5.559,1.318)}
\gppoint{gp mark 7}{(5.562,1.317)}
\gppoint{gp mark 7}{(5.565,1.316)}
\gppoint{gp mark 7}{(5.569,1.315)}
\gppoint{gp mark 7}{(5.572,1.314)}
\gppoint{gp mark 7}{(5.576,1.314)}
\gppoint{gp mark 7}{(5.579,1.315)}
\gppoint{gp mark 7}{(5.582,1.315)}
\gppoint{gp mark 7}{(5.586,1.317)}
\gppoint{gp mark 7}{(5.589,1.318)}
\gppoint{gp mark 7}{(5.593,1.319)}
\gppoint{gp mark 7}{(5.596,1.320)}
\gppoint{gp mark 7}{(5.599,1.320)}
\gppoint{gp mark 7}{(5.603,1.319)}
\gppoint{gp mark 7}{(5.606,1.318)}
\gppoint{gp mark 7}{(5.609,1.318)}
\gppoint{gp mark 7}{(5.613,1.319)}
\gppoint{gp mark 7}{(5.616,1.319)}
\gppoint{gp mark 7}{(5.620,1.320)}
\gppoint{gp mark 7}{(5.623,1.321)}
\gppoint{gp mark 7}{(5.626,1.323)}
\gppoint{gp mark 7}{(5.630,1.323)}
\gppoint{gp mark 7}{(5.633,1.323)}
\gppoint{gp mark 7}{(5.637,1.322)}
\gppoint{gp mark 7}{(5.640,1.320)}
\gppoint{gp mark 7}{(5.643,1.319)}
\gppoint{gp mark 7}{(5.647,1.317)}
\gppoint{gp mark 7}{(5.650,1.316)}
\gppoint{gp mark 7}{(5.654,1.316)}
\gppoint{gp mark 7}{(5.657,1.316)}
\gppoint{gp mark 7}{(5.660,1.316)}
\gppoint{gp mark 7}{(5.664,1.316)}
\gppoint{gp mark 7}{(5.667,1.316)}
\gppoint{gp mark 7}{(5.670,1.315)}
\gppoint{gp mark 7}{(5.674,1.314)}
\gppoint{gp mark 7}{(5.677,1.314)}
\gppoint{gp mark 7}{(5.681,1.314)}
\gppoint{gp mark 7}{(5.684,1.315)}
\gppoint{gp mark 7}{(5.687,1.317)}
\gppoint{gp mark 7}{(5.691,1.318)}
\gppoint{gp mark 7}{(5.694,1.319)}
\gppoint{gp mark 7}{(5.698,1.320)}
\gppoint{gp mark 7}{(5.701,1.320)}
\gppoint{gp mark 7}{(5.704,1.320)}
\gppoint{gp mark 7}{(5.708,1.320)}
\gppoint{gp mark 7}{(5.711,1.319)}
\gppoint{gp mark 7}{(5.714,1.319)}
\gppoint{gp mark 7}{(5.718,1.318)}
\gppoint{gp mark 7}{(5.721,1.317)}
\gppoint{gp mark 7}{(5.725,1.316)}
\gppoint{gp mark 7}{(5.728,1.316)}
\gppoint{gp mark 7}{(5.731,1.316)}
\gppoint{gp mark 7}{(5.735,1.316)}
\gppoint{gp mark 7}{(5.738,1.317)}
\gppoint{gp mark 7}{(5.742,1.318)}
\gppoint{gp mark 7}{(5.745,1.320)}
\gppoint{gp mark 7}{(5.748,1.320)}
\gppoint{gp mark 7}{(5.752,1.320)}
\gppoint{gp mark 7}{(5.755,1.319)}
\gppoint{gp mark 7}{(5.758,1.318)}
\gppoint{gp mark 7}{(5.762,1.316)}
\gppoint{gp mark 7}{(5.765,1.316)}
\gppoint{gp mark 7}{(5.769,1.316)}
\gppoint{gp mark 7}{(5.772,1.316)}
\gppoint{gp mark 7}{(5.775,1.317)}
\gppoint{gp mark 7}{(5.779,1.319)}
\gppoint{gp mark 7}{(5.782,1.321)}
\gppoint{gp mark 7}{(5.786,1.323)}
\gppoint{gp mark 7}{(5.789,1.324)}
\gppoint{gp mark 7}{(5.792,1.325)}
\gppoint{gp mark 7}{(5.796,1.325)}
\gppoint{gp mark 7}{(5.799,1.324)}
\gppoint{gp mark 7}{(5.803,1.322)}
\gppoint{gp mark 7}{(5.806,1.319)}
\gppoint{gp mark 7}{(5.809,1.318)}
\gppoint{gp mark 7}{(5.813,1.318)}
\gppoint{gp mark 7}{(5.816,1.317)}
\gppoint{gp mark 7}{(5.819,1.315)}
\gppoint{gp mark 7}{(5.823,1.315)}
\gppoint{gp mark 7}{(5.826,1.315)}
\gppoint{gp mark 7}{(5.830,1.315)}
\gppoint{gp mark 7}{(5.833,1.315)}
\gppoint{gp mark 7}{(5.836,1.317)}
\gppoint{gp mark 7}{(5.840,1.318)}
\gppoint{gp mark 7}{(5.843,1.320)}
\gppoint{gp mark 7}{(5.847,1.321)}
\gppoint{gp mark 7}{(5.850,1.322)}
\gppoint{gp mark 7}{(5.853,1.322)}
\gppoint{gp mark 7}{(5.857,1.321)}
\gppoint{gp mark 7}{(5.860,1.320)}
\gppoint{gp mark 7}{(5.863,1.319)}
\gppoint{gp mark 7}{(5.867,1.318)}
\gppoint{gp mark 7}{(5.870,1.318)}
\gppoint{gp mark 7}{(5.874,1.318)}
\gppoint{gp mark 7}{(5.877,1.318)}
\gppoint{gp mark 7}{(5.880,1.317)}
\gppoint{gp mark 7}{(5.884,1.315)}
\gppoint{gp mark 7}{(5.887,1.314)}
\gppoint{gp mark 7}{(5.891,1.313)}
\gppoint{gp mark 7}{(5.894,1.313)}
\gppoint{gp mark 7}{(5.897,1.313)}
\gppoint{gp mark 7}{(5.901,1.315)}
\gppoint{gp mark 7}{(5.904,1.318)}
\gppoint{gp mark 7}{(5.907,1.320)}
\gppoint{gp mark 7}{(5.911,1.322)}
\gppoint{gp mark 7}{(5.914,1.322)}
\gppoint{gp mark 7}{(5.918,1.322)}
\gppoint{gp mark 7}{(5.921,1.320)}
\gppoint{gp mark 7}{(5.924,1.319)}
\gppoint{gp mark 7}{(5.928,1.319)}
\gppoint{gp mark 7}{(5.931,1.319)}
\gppoint{gp mark 7}{(5.935,1.318)}
\gppoint{gp mark 7}{(5.938,1.317)}
\gppoint{gp mark 7}{(5.941,1.316)}
\gppoint{gp mark 7}{(5.945,1.316)}
\gppoint{gp mark 7}{(5.948,1.315)}
\gppoint{gp mark 7}{(5.951,1.315)}
\gppoint{gp mark 7}{(5.955,1.315)}
\gppoint{gp mark 7}{(5.958,1.316)}
\gppoint{gp mark 7}{(5.962,1.317)}
\gppoint{gp mark 7}{(5.965,1.320)}
\gppoint{gp mark 7}{(5.968,1.321)}
\gppoint{gp mark 7}{(5.972,1.322)}
\gppoint{gp mark 7}{(5.975,1.322)}
\gppoint{gp mark 7}{(5.979,1.321)}
\gppoint{gp mark 7}{(5.982,1.319)}
\gppoint{gp mark 7}{(5.985,1.318)}
\gppoint{gp mark 7}{(5.989,1.317)}
\gppoint{gp mark 7}{(5.992,1.317)}
\gppoint{gp mark 7}{(5.996,1.317)}
\gppoint{gp mark 7}{(5.999,1.316)}
\gppoint{gp mark 7}{(6.002,1.316)}
\gppoint{gp mark 7}{(6.006,1.316)}
\gppoint{gp mark 7}{(6.009,1.317)}
\gppoint{gp mark 7}{(6.012,1.318)}
\gppoint{gp mark 7}{(6.016,1.319)}
\gppoint{gp mark 7}{(6.019,1.321)}
\gppoint{gp mark 7}{(6.023,1.321)}
\gppoint{gp mark 7}{(6.026,1.322)}
\gppoint{gp mark 7}{(6.029,1.321)}
\gppoint{gp mark 7}{(6.033,1.319)}
\gppoint{gp mark 7}{(6.036,1.317)}
\gppoint{gp mark 7}{(6.040,1.317)}
\gppoint{gp mark 7}{(6.043,1.317)}
\gppoint{gp mark 7}{(6.046,1.316)}
\gppoint{gp mark 7}{(6.050,1.316)}
\gppoint{gp mark 7}{(6.053,1.316)}
\gppoint{gp mark 7}{(6.056,1.316)}
\gppoint{gp mark 7}{(6.060,1.316)}
\gppoint{gp mark 7}{(6.063,1.316)}
\gppoint{gp mark 7}{(6.067,1.316)}
\gppoint{gp mark 7}{(6.070,1.315)}
\gppoint{gp mark 7}{(6.073,1.315)}
\gppoint{gp mark 7}{(6.077,1.316)}
\gppoint{gp mark 7}{(6.080,1.317)}
\gppoint{gp mark 7}{(6.084,1.319)}
\gppoint{gp mark 7}{(6.087,1.320)}
\gppoint{gp mark 7}{(6.090,1.320)}
\gppoint{gp mark 7}{(6.094,1.319)}
\gppoint{gp mark 7}{(6.097,1.317)}
\gppoint{gp mark 7}{(6.100,1.317)}
\gppoint{gp mark 7}{(6.104,1.317)}
\gppoint{gp mark 7}{(6.107,1.318)}
\gppoint{gp mark 7}{(6.111,1.319)}
\gppoint{gp mark 7}{(6.114,1.320)}
\gppoint{gp mark 7}{(6.117,1.322)}
\gppoint{gp mark 7}{(6.121,1.324)}
\gppoint{gp mark 7}{(6.124,1.326)}
\gppoint{gp mark 7}{(6.128,1.326)}
\gppoint{gp mark 7}{(6.131,1.325)}
\gppoint{gp mark 7}{(6.134,1.321)}
\gppoint{gp mark 7}{(6.138,1.318)}
\gppoint{gp mark 7}{(6.141,1.316)}
\gppoint{gp mark 7}{(6.145,1.315)}
\gppoint{gp mark 7}{(6.148,1.315)}
\gppoint{gp mark 7}{(6.151,1.315)}
\gppoint{gp mark 7}{(6.155,1.316)}
\gppoint{gp mark 7}{(6.158,1.316)}
\gppoint{gp mark 7}{(6.161,1.316)}
\gppoint{gp mark 7}{(6.165,1.315)}
\gppoint{gp mark 7}{(6.168,1.315)}
\gppoint{gp mark 7}{(6.172,1.314)}
\gppoint{gp mark 7}{(6.175,1.315)}
\gppoint{gp mark 7}{(6.178,1.318)}
\gppoint{gp mark 7}{(6.182,1.321)}
\gppoint{gp mark 7}{(6.185,1.322)}
\gppoint{gp mark 7}{(6.189,1.322)}
\gppoint{gp mark 7}{(6.192,1.321)}
\gppoint{gp mark 7}{(6.195,1.320)}
\gppoint{gp mark 7}{(6.199,1.320)}
\gppoint{gp mark 7}{(6.202,1.320)}
\gppoint{gp mark 7}{(6.205,1.319)}
\gppoint{gp mark 7}{(6.209,1.317)}
\gppoint{gp mark 7}{(6.212,1.314)}
\gppoint{gp mark 7}{(6.216,1.313)}
\gppoint{gp mark 7}{(6.219,1.313)}
\gppoint{gp mark 7}{(6.222,1.314)}
\gppoint{gp mark 7}{(6.226,1.315)}
\gppoint{gp mark 7}{(6.229,1.315)}
\gppoint{gp mark 7}{(6.233,1.315)}
\gppoint{gp mark 7}{(6.236,1.317)}
\gppoint{gp mark 7}{(6.239,1.322)}
\gppoint{gp mark 7}{(6.243,1.324)}
\gppoint{gp mark 7}{(6.246,1.324)}
\gppoint{gp mark 7}{(6.249,1.324)}
\gppoint{gp mark 7}{(6.253,1.321)}
\gppoint{gp mark 7}{(6.256,1.320)}
\gppoint{gp mark 7}{(6.260,1.319)}
\gppoint{gp mark 7}{(6.263,1.320)}
\gppoint{gp mark 7}{(6.266,1.319)}
\gppoint{gp mark 7}{(6.270,1.317)}
\gppoint{gp mark 7}{(6.273,1.315)}
\gppoint{gp mark 7}{(6.277,1.312)}
\gppoint{gp mark 7}{(6.280,1.312)}
\gppoint{gp mark 7}{(6.283,1.312)}
\gppoint{gp mark 7}{(6.287,1.313)}
\gppoint{gp mark 7}{(6.290,1.315)}
\gppoint{gp mark 7}{(6.294,1.317)}
\gppoint{gp mark 7}{(6.297,1.323)}
\gppoint{gp mark 7}{(6.300,1.324)}
\gppoint{gp mark 7}{(6.304,1.324)}
\gppoint{gp mark 7}{(6.307,1.323)}
\gppoint{gp mark 7}{(6.310,1.320)}
\gppoint{gp mark 7}{(6.314,1.319)}
\gppoint{gp mark 7}{(6.317,1.318)}
\gppoint{gp mark 7}{(6.321,1.318)}
\gppoint{gp mark 7}{(6.324,1.318)}
\gppoint{gp mark 7}{(6.327,1.316)}
\gppoint{gp mark 7}{(6.331,1.315)}
\gppoint{gp mark 7}{(6.334,1.316)}
\gppoint{gp mark 7}{(6.338,1.318)}
\gppoint{gp mark 7}{(6.341,1.318)}
\gppoint{gp mark 7}{(6.344,1.314)}
\gppoint{gp mark 7}{(6.348,1.312)}
\gppoint{gp mark 7}{(6.351,1.312)}
\gppoint{gp mark 7}{(6.354,1.319)}
\gppoint{gp mark 7}{(6.358,1.326)}
\gppoint{gp mark 7}{(6.361,1.326)}
\gppoint{gp mark 7}{(6.365,1.323)}
\gppoint{gp mark 7}{(6.368,1.318)}
\gppoint{gp mark 7}{(6.371,1.316)}
\gppoint{gp mark 7}{(6.375,1.315)}
\gppoint{gp mark 7}{(6.378,1.318)}
\gppoint{gp mark 7}{(6.382,1.320)}
\gppoint{gp mark 7}{(6.385,1.320)}
\gppoint{gp mark 7}{(6.388,1.316)}
\gppoint{gp mark 7}{(6.392,1.314)}
\gppoint{gp mark 7}{(6.395,1.315)}
\gppoint{gp mark 7}{(6.398,1.321)}
\gppoint{gp mark 7}{(6.402,1.323)}
\gppoint{gp mark 7}{(6.405,1.321)}
\gppoint{gp mark 7}{(6.409,1.314)}
\gppoint{gp mark 7}{(6.412,1.313)}
\gppoint{gp mark 7}{(6.415,1.318)}
\gppoint{gp mark 7}{(6.419,1.319)}
\gppoint{gp mark 7}{(6.422,1.321)}
\gppoint{gp mark 7}{(6.426,1.318)}
\gppoint{gp mark 7}{(6.429,1.253)}
\gppoint{gp mark 7}{(6.432,1.184)}
\gppoint{gp mark 7}{(6.436,1.235)}
\gppoint{gp mark 7}{(6.439,1.311)}
\gppoint{gp mark 7}{(6.442,1.317)}
\gppoint{gp mark 7}{(6.446,1.317)}
\gppoint{gp mark 7}{(6.449,1.317)}
\gppoint{gp mark 7}{(6.453,1.317)}
\gppoint{gp mark 7}{(6.456,1.317)}
\gppoint{gp mark 7}{(6.459,1.317)}
\gppoint{gp mark 7}{(6.463,1.317)}
\gppoint{gp mark 7}{(6.466,1.317)}
\gppoint{gp mark 7}{(6.470,1.317)}
\gppoint{gp mark 7}{(6.473,1.317)}
\gppoint{gp mark 7}{(6.476,1.317)}
\gppoint{gp mark 7}{(6.480,1.317)}
\gppoint{gp mark 7}{(6.483,1.317)}
\gppoint{gp mark 7}{(6.487,1.317)}
\gppoint{gp mark 7}{(6.490,1.317)}
\gppoint{gp mark 7}{(6.493,1.317)}
\gppoint{gp mark 7}{(6.497,1.317)}
\gppoint{gp mark 7}{(6.500,1.317)}
\gppoint{gp mark 7}{(6.503,1.317)}
\gppoint{gp mark 7}{(6.507,1.317)}
\gppoint{gp mark 7}{(6.510,1.317)}
\gppoint{gp mark 7}{(6.514,1.317)}
\gppoint{gp mark 7}{(6.517,1.317)}
\gppoint{gp mark 7}{(6.520,1.317)}
\gppoint{gp mark 7}{(6.524,1.317)}
\gppoint{gp mark 7}{(6.527,1.317)}
\gppoint{gp mark 7}{(6.531,1.317)}
\gppoint{gp mark 7}{(6.534,1.317)}
\gppoint{gp mark 7}{(6.537,1.317)}
\gppoint{gp mark 7}{(6.541,1.317)}
\gppoint{gp mark 7}{(6.544,1.317)}
\gppoint{gp mark 7}{(6.547,1.317)}
\gppoint{gp mark 7}{(6.551,1.317)}
\gppoint{gp mark 7}{(6.554,1.317)}
\gppoint{gp mark 7}{(6.558,1.317)}
\gppoint{gp mark 7}{(6.561,1.317)}
\gppoint{gp mark 7}{(6.564,1.317)}
\gppoint{gp mark 7}{(6.568,1.317)}
\gppoint{gp mark 7}{(6.571,1.317)}
\gppoint{gp mark 7}{(6.575,1.317)}
\gppoint{gp mark 7}{(6.578,1.317)}
\gppoint{gp mark 7}{(6.581,1.317)}
\gppoint{gp mark 7}{(6.585,1.317)}
\gppoint{gp mark 7}{(6.588,1.317)}
\gppoint{gp mark 7}{(6.591,1.317)}
\gppoint{gp mark 7}{(6.595,1.317)}
\gppoint{gp mark 7}{(6.598,1.317)}
\gppoint{gp mark 7}{(6.602,1.317)}
\gppoint{gp mark 7}{(6.605,1.317)}
\gppoint{gp mark 7}{(6.608,1.317)}
\gppoint{gp mark 7}{(6.612,1.317)}
\gppoint{gp mark 7}{(6.615,1.317)}
\gppoint{gp mark 7}{(6.619,1.317)}
\gppoint{gp mark 7}{(6.622,1.317)}
\gppoint{gp mark 7}{(6.625,1.317)}
\gppoint{gp mark 7}{(6.629,1.317)}
\gppoint{gp mark 7}{(6.632,1.317)}
\gppoint{gp mark 7}{(6.636,1.317)}
\gppoint{gp mark 7}{(6.639,1.317)}
\gppoint{gp mark 7}{(6.642,1.317)}
\gppoint{gp mark 7}{(6.646,1.317)}
\gppoint{gp mark 7}{(6.649,1.317)}
\gppoint{gp mark 7}{(6.652,1.317)}
\gppoint{gp mark 7}{(6.656,1.317)}
\gppoint{gp mark 7}{(6.659,1.317)}
\gppoint{gp mark 7}{(6.663,1.317)}
\gppoint{gp mark 7}{(6.666,1.317)}
\gppoint{gp mark 7}{(6.669,1.317)}
\gppoint{gp mark 7}{(6.673,1.317)}
\gppoint{gp mark 7}{(6.676,1.317)}
\gppoint{gp mark 7}{(6.680,1.317)}
\gppoint{gp mark 7}{(6.683,1.317)}
\gppoint{gp mark 7}{(6.686,1.317)}
\gppoint{gp mark 7}{(6.690,1.317)}
\gppoint{gp mark 7}{(6.693,1.317)}
\gppoint{gp mark 7}{(6.696,1.317)}
\gppoint{gp mark 7}{(6.700,1.317)}
\gppoint{gp mark 7}{(6.703,1.317)}
\gppoint{gp mark 7}{(6.707,1.317)}
\gppoint{gp mark 7}{(6.710,1.317)}
\gppoint{gp mark 7}{(6.713,1.317)}
\gppoint{gp mark 7}{(6.717,1.317)}
\gppoint{gp mark 7}{(6.720,1.317)}
\gppoint{gp mark 7}{(6.724,1.317)}
\gppoint{gp mark 7}{(6.727,1.317)}
\gppoint{gp mark 7}{(6.730,1.317)}
\gppoint{gp mark 7}{(6.734,1.317)}
\gppoint{gp mark 7}{(6.737,1.317)}
\gppoint{gp mark 7}{(6.740,1.317)}
\gppoint{gp mark 7}{(6.744,1.317)}
\gppoint{gp mark 7}{(6.747,1.317)}
\gppoint{gp mark 7}{(6.751,1.317)}
\gppoint{gp mark 7}{(6.754,1.317)}
\gppoint{gp mark 7}{(6.757,1.317)}
\gppoint{gp mark 7}{(6.761,1.317)}
\gppoint{gp mark 7}{(6.764,1.317)}
\gppoint{gp mark 7}{(6.768,1.316)}
\gppoint{gp mark 7}{(6.771,1.316)}
\gppoint{gp mark 7}{(6.774,1.316)}
\gppoint{gp mark 7}{(6.778,1.316)}
\gppoint{gp mark 7}{(6.781,1.316)}
\gppoint{gp mark 7}{(6.785,1.316)}
\gppoint{gp mark 7}{(6.788,1.316)}
\gppoint{gp mark 7}{(6.791,1.316)}
\gppoint{gp mark 7}{(6.795,1.316)}
\gppoint{gp mark 7}{(6.798,1.316)}
\gppoint{gp mark 7}{(6.801,1.316)}
\gppoint{gp mark 7}{(6.805,1.316)}
\gppoint{gp mark 7}{(6.808,1.316)}
\gppoint{gp mark 7}{(6.812,1.316)}
\gppoint{gp mark 7}{(6.815,1.316)}
\gppoint{gp mark 7}{(6.818,1.316)}
\gppoint{gp mark 7}{(6.822,1.316)}
\gppoint{gp mark 7}{(6.825,1.316)}
\gppoint{gp mark 7}{(6.829,1.316)}
\gppoint{gp mark 7}{(6.832,1.316)}
\gppoint{gp mark 7}{(6.835,1.316)}
\gppoint{gp mark 7}{(6.839,1.316)}
\gppoint{gp mark 7}{(6.842,1.316)}
\gppoint{gp mark 7}{(6.845,1.316)}
\gppoint{gp mark 7}{(6.849,1.316)}
\gppoint{gp mark 7}{(6.852,1.316)}
\gppoint{gp mark 7}{(6.856,1.315)}
\gppoint{gp mark 7}{(6.859,1.315)}
\gppoint{gp mark 7}{(6.862,1.315)}
\gppoint{gp mark 7}{(6.866,1.315)}
\gppoint{gp mark 7}{(6.869,1.315)}
\gppoint{gp mark 7}{(6.873,1.315)}
\gppoint{gp mark 7}{(6.876,1.315)}
\gppoint{gp mark 7}{(6.879,1.315)}
\gppoint{gp mark 7}{(6.883,1.314)}
\gppoint{gp mark 7}{(6.886,1.314)}
\gppoint{gp mark 7}{(6.889,1.314)}
\gppoint{gp mark 7}{(6.893,1.314)}
\gppoint{gp mark 7}{(6.896,1.314)}
\gppoint{gp mark 7}{(6.900,1.314)}
\gppoint{gp mark 7}{(6.903,1.314)}
\gppoint{gp mark 7}{(6.906,1.313)}
\gppoint{gp mark 7}{(6.910,1.312)}
\gppoint{gp mark 7}{(6.913,1.309)}
\gppoint{gp mark 7}{(6.917,1.306)}
\gppoint{gp mark 7}{(6.920,1.304)}
\gppoint{gp mark 7}{(6.923,1.304)}
\gppoint{gp mark 7}{(6.927,1.304)}
\gppoint{gp mark 7}{(6.930,1.304)}
\gppoint{gp mark 7}{(6.934,1.304)}
\gppoint{gp mark 7}{(6.937,1.304)}
\gppoint{gp mark 7}{(6.940,1.304)}
\gppoint{gp mark 7}{(6.944,1.305)}
\gppoint{gp mark 7}{(6.947,1.306)}
\gppoint{gp mark 7}{(6.950,1.307)}
\gppoint{gp mark 7}{(6.954,1.308)}
\gppoint{gp mark 7}{(6.957,1.309)}
\gppoint{gp mark 7}{(6.961,1.310)}
\gppoint{gp mark 7}{(6.964,1.311)}
\gppoint{gp mark 7}{(6.967,1.311)}
\gppoint{gp mark 7}{(6.971,1.312)}
\gppoint{gp mark 7}{(6.974,1.313)}
\gppoint{gp mark 7}{(6.978,1.313)}
\gppoint{gp mark 7}{(6.981,1.314)}
\gppoint{gp mark 7}{(6.984,1.315)}
\gppoint{gp mark 7}{(6.988,1.315)}
\gppoint{gp mark 7}{(6.991,1.316)}
\gppoint{gp mark 7}{(6.994,1.317)}
\gppoint{gp mark 7}{(6.998,1.318)}
\gppoint{gp mark 7}{(7.001,1.318)}
\gppoint{gp mark 7}{(7.005,1.319)}
\gppoint{gp mark 7}{(7.008,1.320)}
\gppoint{gp mark 7}{(7.011,1.320)}
\gppoint{gp mark 7}{(7.015,1.321)}
\gppoint{gp mark 7}{(7.018,1.322)}
\gppoint{gp mark 7}{(7.022,1.322)}
\gppoint{gp mark 7}{(7.025,1.323)}
\gppoint{gp mark 7}{(7.028,1.324)}
\gppoint{gp mark 7}{(7.032,1.325)}
\gppoint{gp mark 7}{(7.035,1.325)}
\gppoint{gp mark 7}{(7.038,1.326)}
\gppoint{gp mark 7}{(7.042,1.327)}
\gppoint{gp mark 7}{(7.045,1.327)}
\gppoint{gp mark 7}{(7.049,1.328)}
\gppoint{gp mark 7}{(7.052,1.329)}
\gppoint{gp mark 7}{(7.055,1.329)}
\gppoint{gp mark 7}{(7.059,1.330)}
\gppoint{gp mark 7}{(7.062,1.331)}
\gppoint{gp mark 7}{(7.066,1.331)}
\gppoint{gp mark 7}{(7.069,1.332)}
\gppoint{gp mark 7}{(7.072,1.333)}
\gppoint{gp mark 7}{(7.076,1.334)}
\gppoint{gp mark 7}{(7.079,1.334)}
\gppoint{gp mark 7}{(7.082,1.335)}
\gppoint{gp mark 7}{(7.086,1.336)}
\gppoint{gp mark 7}{(7.089,1.336)}
\gppoint{gp mark 7}{(7.093,1.337)}
\gppoint{gp mark 7}{(7.096,1.338)}
\gppoint{gp mark 7}{(7.099,1.338)}
\gppoint{gp mark 7}{(7.103,1.339)}
\gppoint{gp mark 7}{(7.106,1.340)}
\gppoint{gp mark 7}{(7.110,1.341)}
\gppoint{gp mark 7}{(7.113,1.341)}
\gppoint{gp mark 7}{(7.116,1.342)}
\gppoint{gp mark 7}{(7.120,1.343)}
\gppoint{gp mark 7}{(7.123,1.343)}
\gppoint{gp mark 7}{(7.127,1.344)}
\gppoint{gp mark 7}{(7.130,1.345)}
\gppoint{gp mark 7}{(7.133,1.345)}
\gppoint{gp mark 7}{(7.137,1.346)}
\gppoint{gp mark 7}{(7.140,1.347)}
\gppoint{gp mark 7}{(7.143,1.348)}
\gppoint{gp mark 7}{(7.147,1.348)}
\gppoint{gp mark 7}{(7.150,1.349)}
\gppoint{gp mark 7}{(7.154,1.350)}
\gppoint{gp mark 7}{(7.157,1.350)}
\gppoint{gp mark 7}{(7.160,1.351)}
\gppoint{gp mark 7}{(7.164,1.352)}
\gppoint{gp mark 7}{(7.167,1.352)}
\gppoint{gp mark 7}{(7.171,1.353)}
\gppoint{gp mark 7}{(7.174,1.354)}
\gppoint{gp mark 7}{(7.177,1.355)}
\gppoint{gp mark 7}{(7.181,1.355)}
\gppoint{gp mark 7}{(7.184,1.356)}
\gppoint{gp mark 7}{(7.187,1.357)}
\gppoint{gp mark 7}{(7.191,1.357)}
\gppoint{gp mark 7}{(7.194,1.358)}
\gppoint{gp mark 7}{(7.198,1.359)}
\gppoint{gp mark 7}{(7.201,1.360)}
\gppoint{gp mark 7}{(7.204,1.360)}
\gppoint{gp mark 7}{(7.208,1.361)}
\gppoint{gp mark 7}{(7.211,1.362)}
\gppoint{gp mark 7}{(7.215,1.362)}
\gppoint{gp mark 7}{(7.218,1.363)}
\gppoint{gp mark 7}{(7.221,1.364)}
\gppoint{gp mark 7}{(7.225,1.364)}
\gppoint{gp mark 7}{(7.228,1.365)}
\gppoint{gp mark 7}{(7.231,1.366)}
\gppoint{gp mark 7}{(7.235,1.367)}
\gppoint{gp mark 7}{(7.238,1.367)}
\gppoint{gp mark 7}{(7.242,1.368)}
\gppoint{gp mark 7}{(7.245,1.369)}
\gppoint{gp mark 7}{(7.248,1.369)}
\gppoint{gp mark 7}{(7.252,1.370)}
\gppoint{gp mark 7}{(7.255,1.371)}
\gppoint{gp mark 7}{(7.259,1.372)}
\gppoint{gp mark 7}{(7.262,1.372)}
\gppoint{gp mark 7}{(7.265,1.373)}
\gppoint{gp mark 7}{(7.269,1.374)}
\gppoint{gp mark 7}{(7.272,1.374)}
\gppoint{gp mark 7}{(7.276,1.375)}
\gppoint{gp mark 7}{(7.279,1.376)}
\gppoint{gp mark 7}{(7.282,1.377)}
\gppoint{gp mark 7}{(7.286,1.377)}
\gppoint{gp mark 7}{(7.289,1.378)}
\gppoint{gp mark 7}{(7.292,1.379)}
\gppoint{gp mark 7}{(7.296,1.379)}
\gppoint{gp mark 7}{(7.299,1.380)}
\gppoint{gp mark 7}{(7.303,1.381)}
\gppoint{gp mark 7}{(7.306,1.382)}
\gppoint{gp mark 7}{(7.309,1.382)}
\gppoint{gp mark 7}{(7.313,1.383)}
\gppoint{gp mark 7}{(7.316,1.384)}
\gppoint{gp mark 7}{(7.320,1.384)}
\gppoint{gp mark 7}{(7.323,1.385)}
\gppoint{gp mark 7}{(7.326,1.386)}
\gppoint{gp mark 7}{(7.330,1.387)}
\gppoint{gp mark 7}{(7.333,1.387)}
\gppoint{gp mark 7}{(7.336,1.388)}
\gppoint{gp mark 7}{(7.340,1.389)}
\gppoint{gp mark 7}{(7.343,1.389)}
\gppoint{gp mark 7}{(7.347,1.390)}
\gppoint{gp mark 7}{(7.350,1.391)}
\gppoint{gp mark 7}{(7.353,1.392)}
\gppoint{gp mark 7}{(7.357,1.392)}
\gppoint{gp mark 7}{(7.360,1.393)}
\gppoint{gp mark 7}{(7.364,1.394)}
\gppoint{gp mark 7}{(7.367,1.394)}
\gppoint{gp mark 7}{(7.370,1.395)}
\gppoint{gp mark 7}{(7.374,1.396)}
\gppoint{gp mark 7}{(7.377,1.397)}
\gppoint{gp mark 7}{(7.380,1.397)}
\gppoint{gp mark 7}{(7.384,1.398)}
\gppoint{gp mark 7}{(7.387,1.399)}
\gppoint{gp mark 7}{(7.391,1.399)}
\gppoint{gp mark 7}{(7.394,1.400)}
\gppoint{gp mark 7}{(7.397,1.401)}
\gppoint{gp mark 7}{(7.401,1.402)}
\gppoint{gp mark 7}{(7.404,1.402)}
\gppoint{gp mark 7}{(7.408,1.403)}
\gppoint{gp mark 7}{(7.411,1.404)}
\gppoint{gp mark 7}{(7.414,1.404)}
\gppoint{gp mark 7}{(7.418,1.405)}
\gppoint{gp mark 7}{(7.421,1.406)}
\gppoint{gp mark 7}{(7.425,1.407)}
\gppoint{gp mark 7}{(7.428,1.407)}
\gppoint{gp mark 7}{(7.431,1.408)}
\gppoint{gp mark 7}{(7.435,1.409)}
\gppoint{gp mark 7}{(7.438,1.409)}
\gppoint{gp mark 7}{(7.441,1.410)}
\gppoint{gp mark 7}{(7.445,1.411)}
\gppoint{gp mark 7}{(7.448,1.412)}
\gppoint{gp mark 7}{(7.452,1.412)}
\gppoint{gp mark 7}{(7.455,1.413)}
\gppoint{gp mark 7}{(7.458,1.414)}
\gppoint{gp mark 7}{(7.462,1.415)}
\gppoint{gp mark 7}{(7.465,1.415)}
\gppoint{gp mark 7}{(7.469,1.416)}
\gppoint{gp mark 7}{(7.472,1.417)}
\gppoint{gp mark 7}{(7.475,1.417)}
\gppoint{gp mark 7}{(7.479,1.418)}
\gppoint{gp mark 7}{(7.482,1.419)}
\gppoint{gp mark 7}{(7.485,1.420)}
\gppoint{gp mark 7}{(7.489,1.420)}
\gppoint{gp mark 7}{(7.492,1.421)}
\gppoint{gp mark 7}{(7.496,1.422)}
\gppoint{gp mark 7}{(7.499,1.423)}
\gppoint{gp mark 7}{(7.502,1.423)}
\gppoint{gp mark 7}{(7.506,1.424)}
\gppoint{gp mark 7}{(7.509,1.425)}
\gppoint{gp mark 7}{(7.513,1.425)}
\gppoint{gp mark 7}{(7.516,1.426)}
\gppoint{gp mark 7}{(7.519,1.427)}
\gppoint{gp mark 7}{(7.523,1.428)}
\gppoint{gp mark 7}{(7.526,1.428)}
\gppoint{gp mark 7}{(7.529,1.429)}
\gppoint{gp mark 7}{(7.533,1.430)}
\gppoint{gp mark 7}{(7.536,1.430)}
\gppoint{gp mark 7}{(7.540,1.431)}
\gppoint{gp mark 7}{(7.543,1.432)}
\gppoint{gp mark 7}{(7.546,1.433)}
\gppoint{gp mark 7}{(7.550,1.433)}
\gppoint{gp mark 7}{(7.553,1.434)}
\gppoint{gp mark 7}{(7.557,1.435)}
\gppoint{gp mark 7}{(7.560,1.436)}
\gppoint{gp mark 7}{(7.563,1.436)}
\gppoint{gp mark 7}{(7.567,1.437)}
\gppoint{gp mark 7}{(7.570,1.438)}
\gppoint{gp mark 7}{(7.573,1.438)}
\gppoint{gp mark 7}{(7.577,1.439)}
\gppoint{gp mark 7}{(7.580,1.440)}
\gppoint{gp mark 7}{(7.584,1.441)}
\gppoint{gp mark 7}{(7.587,1.441)}
\gppoint{gp mark 7}{(7.590,1.442)}
\gppoint{gp mark 7}{(7.594,1.443)}
\gppoint{gp mark 7}{(7.597,1.444)}
\gppoint{gp mark 7}{(7.601,1.444)}
\gppoint{gp mark 7}{(7.604,1.445)}
\gppoint{gp mark 7}{(7.607,1.446)}
\gppoint{gp mark 7}{(7.611,1.447)}
\gppoint{gp mark 7}{(7.614,1.447)}
\gppoint{gp mark 7}{(7.618,1.448)}
\gppoint{gp mark 7}{(7.621,1.449)}
\gppoint{gp mark 7}{(7.624,1.449)}
\gppoint{gp mark 7}{(7.628,1.450)}
\gppoint{gp mark 7}{(7.631,1.451)}
\gppoint{gp mark 7}{(7.634,1.452)}
\gppoint{gp mark 7}{(7.638,1.452)}
\gppoint{gp mark 7}{(7.641,1.453)}
\gppoint{gp mark 7}{(7.645,1.454)}
\gppoint{gp mark 7}{(7.648,1.455)}
\gppoint{gp mark 7}{(7.651,1.455)}
\gppoint{gp mark 7}{(7.655,1.456)}
\gppoint{gp mark 7}{(7.658,1.457)}
\gppoint{gp mark 7}{(7.662,1.458)}
\gppoint{gp mark 7}{(7.665,1.458)}
\gppoint{gp mark 7}{(7.668,1.459)}
\gppoint{gp mark 7}{(7.672,1.460)}
\gppoint{gp mark 7}{(7.675,1.460)}
\gppoint{gp mark 7}{(7.678,1.461)}
\gppoint{gp mark 7}{(7.682,1.462)}
\gppoint{gp mark 7}{(7.685,1.463)}
\gppoint{gp mark 7}{(7.689,1.463)}
\gppoint{gp mark 7}{(7.692,1.464)}
\gppoint{gp mark 7}{(7.695,1.465)}
\gppoint{gp mark 7}{(7.699,1.466)}
\gppoint{gp mark 7}{(7.702,1.466)}
\gppoint{gp mark 7}{(7.706,1.467)}
\gppoint{gp mark 7}{(7.709,1.468)}
\gppoint{gp mark 7}{(7.712,1.469)}
\gppoint{gp mark 7}{(7.716,1.469)}
\gppoint{gp mark 7}{(7.719,1.470)}
\gppoint{gp mark 7}{(7.722,1.471)}
\gppoint{gp mark 7}{(7.726,1.471)}
\gppoint{gp mark 7}{(7.729,1.472)}
\gppoint{gp mark 7}{(7.733,1.473)}
\gppoint{gp mark 7}{(7.736,1.474)}
\gppoint{gp mark 7}{(7.739,1.474)}
\gppoint{gp mark 7}{(7.743,1.475)}
\gppoint{gp mark 7}{(7.746,1.476)}
\gppoint{gp mark 7}{(7.750,1.477)}
\gppoint{gp mark 7}{(7.753,1.477)}
\gppoint{gp mark 7}{(7.756,1.478)}
\gppoint{gp mark 7}{(7.760,1.479)}
\gppoint{gp mark 7}{(7.763,1.480)}
\gppoint{gp mark 7}{(7.767,1.480)}
\gppoint{gp mark 7}{(7.770,1.481)}
\gppoint{gp mark 7}{(7.773,1.482)}
\gppoint{gp mark 7}{(7.777,1.483)}
\gppoint{gp mark 7}{(7.780,1.483)}
\gppoint{gp mark 7}{(7.783,1.484)}
\gppoint{gp mark 7}{(7.787,1.485)}
\gppoint{gp mark 7}{(7.790,1.486)}
\gppoint{gp mark 7}{(7.794,1.486)}
\gppoint{gp mark 7}{(7.797,1.487)}
\gppoint{gp mark 7}{(7.800,1.488)}
\gppoint{gp mark 7}{(7.804,1.488)}
\gppoint{gp mark 7}{(7.807,1.489)}
\gppoint{gp mark 7}{(7.811,1.490)}
\gppoint{gp mark 7}{(7.814,1.491)}
\gppoint{gp mark 7}{(7.817,1.491)}
\gppoint{gp mark 7}{(7.821,1.492)}
\gppoint{gp mark 7}{(7.824,1.493)}
\gppoint{gp mark 7}{(7.827,1.494)}
\gppoint{gp mark 7}{(7.831,1.494)}
\gppoint{gp mark 7}{(7.834,1.495)}
\gppoint{gp mark 7}{(7.838,1.496)}
\gppoint{gp mark 7}{(7.841,1.497)}
\gppoint{gp mark 7}{(7.844,1.497)}
\gppoint{gp mark 7}{(7.848,1.498)}
\gppoint{gp mark 7}{(7.851,1.499)}
\gppoint{gp mark 7}{(7.855,1.500)}
\gppoint{gp mark 7}{(7.858,1.500)}
\gppoint{gp mark 7}{(7.861,1.501)}
\gppoint{gp mark 7}{(7.865,1.502)}
\gppoint{gp mark 7}{(7.868,1.503)}
\gppoint{gp mark 7}{(7.871,1.503)}
\gppoint{gp mark 7}{(7.875,1.504)}
\gppoint{gp mark 7}{(7.878,1.505)}
\gppoint{gp mark 7}{(7.882,1.506)}
\gppoint{gp mark 7}{(7.885,1.506)}
\gppoint{gp mark 7}{(7.888,1.507)}
\gppoint{gp mark 7}{(7.892,1.508)}
\gppoint{gp mark 7}{(7.895,1.509)}
\gppoint{gp mark 7}{(7.899,1.509)}
\gppoint{gp mark 7}{(7.902,1.510)}
\gppoint{gp mark 7}{(7.905,1.511)}
\gppoint{gp mark 7}{(7.909,1.512)}
\gppoint{gp mark 7}{(7.912,1.512)}
\gppoint{gp mark 7}{(7.916,1.513)}
\gppoint{gp mark 7}{(7.919,1.514)}
\gppoint{gp mark 7}{(7.922,1.515)}
\gppoint{gp mark 7}{(7.926,1.515)}
\gppoint{gp mark 7}{(7.929,1.516)}
\gppoint{gp mark 7}{(7.932,1.517)}
\gppoint{gp mark 7}{(7.936,1.518)}
\gppoint{gp mark 7}{(7.939,1.518)}
\gppoint{gp mark 7}{(7.943,1.519)}
\gpcolor{rgb color={0.000,0.000,0.000}}
\gpsetlinetype{gp lt plot 0}
\draw[gp path] (1.531,2.562)--(1.929,2.562);
\draw[gp path] (1.929,2.562)--(3.366,2.562);
\draw[gp path] (3.366,5.536)--(3.855,5.536);
\draw[gp path] (3.855,3.568)--(4.388,3.568);
\draw[gp path] (4.388,1.318)--(6.434,1.318);
\draw[gp path] (6.434,1.318)--(6.997,1.318);
\draw[gp path] (1.013,2.853)--(1.017,2.850)--(1.021,2.848)--(1.025,2.846)--(1.028,2.843)%
  --(1.032,2.841)--(1.036,2.839)--(1.040,2.836)--(1.044,2.834)--(1.048,2.832)--(1.051,2.829)%
  --(1.055,2.827)--(1.059,2.825)--(1.063,2.822)--(1.067,2.820)--(1.070,2.818)--(1.074,2.815)%
  --(1.078,2.813)--(1.082,2.811)--(1.086,2.808)--(1.089,2.806)--(1.093,2.804)--(1.097,2.801)%
  --(1.101,2.799)--(1.105,2.797)--(1.108,2.795)--(1.112,2.792)--(1.116,2.790)--(1.120,2.788)%
  --(1.124,2.785)--(1.127,2.783)--(1.131,2.781)--(1.135,2.779)--(1.139,2.776)--(1.143,2.774)%
  --(1.147,2.772)--(1.150,2.770)--(1.154,2.767)--(1.158,2.765)--(1.162,2.763)--(1.166,2.761)%
  --(1.169,2.758)--(1.173,2.756)--(1.177,2.754)--(1.181,2.752)--(1.185,2.749)--(1.188,2.747)%
  --(1.192,2.745)--(1.196,2.743)--(1.200,2.741)--(1.204,2.738)--(1.207,2.736)--(1.211,2.734)%
  --(1.215,2.732)--(1.219,2.730)--(1.223,2.727)--(1.226,2.725)--(1.230,2.723)--(1.234,2.721)%
  --(1.238,2.719)--(1.242,2.716)--(1.246,2.714)--(1.249,2.712)--(1.253,2.710)--(1.257,2.708)%
  --(1.261,2.706)--(1.265,2.703)--(1.268,2.701)--(1.272,2.699)--(1.276,2.697)--(1.280,2.695)%
  --(1.284,2.693)--(1.287,2.691)--(1.291,2.688)--(1.295,2.686)--(1.299,2.684)--(1.303,2.682)%
  --(1.306,2.680)--(1.310,2.678)--(1.314,2.676)--(1.318,2.674)--(1.322,2.672)--(1.325,2.669)%
  --(1.329,2.667)--(1.333,2.665)--(1.337,2.663)--(1.341,2.661)--(1.345,2.659)--(1.348,2.657)%
  --(1.352,2.655)--(1.356,2.653)--(1.360,2.651)--(1.364,2.649)--(1.367,2.647)--(1.371,2.645)%
  --(1.375,2.643)--(1.379,2.640)--(1.383,2.638)--(1.386,2.636)--(1.390,2.634)--(1.394,2.632)%
  --(1.398,2.630)--(1.402,2.628)--(1.405,2.626)--(1.409,2.624)--(1.413,2.622)--(1.417,2.620)%
  --(1.421,2.618)--(1.424,2.616)--(1.428,2.614)--(1.432,2.612)--(1.436,2.610)--(1.440,2.608)%
  --(1.443,2.606)--(1.447,2.604)--(1.451,2.602)--(1.455,2.600)--(1.459,2.598)--(1.463,2.596)%
  --(1.466,2.594)--(1.470,2.593)--(1.474,2.591)--(1.478,2.589)--(1.482,2.587)--(1.485,2.585)%
  --(1.489,2.583)--(1.493,2.581)--(1.497,2.579)--(1.501,2.577)--(1.504,2.575)--(1.508,2.573)%
  --(1.512,2.571)--(1.516,2.569)--(1.520,2.567)--(1.523,2.566)--(1.527,2.564)--(1.531,2.562);
\draw[gp path] (3.366,2.562)--(3.366,5.536);
\draw[gp path] (3.855,5.536)--(3.855,3.568);
\draw[gp path] (4.388,3.568)--(4.388,1.318);
\draw[gp path] (6.997,1.318)--(7.003,1.319)--(7.008,1.319)--(7.013,1.320)--(7.018,1.321)%
  --(7.023,1.322)--(7.029,1.323)--(7.034,1.324)--(7.039,1.325)--(7.044,1.326)--(7.049,1.327)%
  --(7.055,1.328)--(7.060,1.329)--(7.065,1.330)--(7.070,1.331)--(7.075,1.332)--(7.080,1.333)%
  --(7.086,1.334)--(7.091,1.334)--(7.096,1.335)--(7.101,1.336)--(7.106,1.337)--(7.112,1.338)%
  --(7.117,1.339)--(7.122,1.340)--(7.127,1.341)--(7.132,1.342)--(7.138,1.343)--(7.143,1.344)%
  --(7.148,1.345)--(7.153,1.346)--(7.158,1.347)--(7.164,1.348)--(7.169,1.349)--(7.174,1.350)%
  --(7.179,1.351)--(7.184,1.352)--(7.189,1.353)--(7.195,1.354)--(7.200,1.355)--(7.205,1.356)%
  --(7.210,1.357)--(7.215,1.358)--(7.221,1.359)--(7.226,1.360)--(7.231,1.361)--(7.236,1.362)%
  --(7.241,1.363)--(7.247,1.364)--(7.252,1.365)--(7.257,1.366)--(7.262,1.367)--(7.267,1.368)%
  --(7.272,1.369)--(7.278,1.370)--(7.283,1.371)--(7.288,1.372)--(7.293,1.373)--(7.298,1.374)%
  --(7.304,1.375)--(7.309,1.376)--(7.314,1.377)--(7.319,1.378)--(7.324,1.379)--(7.330,1.380)%
  --(7.335,1.381)--(7.340,1.382)--(7.345,1.383)--(7.350,1.384)--(7.355,1.385)--(7.361,1.387)%
  --(7.366,1.388)--(7.371,1.389)--(7.376,1.390)--(7.381,1.391)--(7.387,1.392)--(7.392,1.393)%
  --(7.397,1.394)--(7.402,1.395)--(7.407,1.396)--(7.413,1.397)--(7.418,1.398)--(7.423,1.399)%
  --(7.428,1.400)--(7.433,1.401)--(7.439,1.402)--(7.444,1.403)--(7.449,1.405)--(7.454,1.406)%
  --(7.459,1.407)--(7.464,1.408)--(7.470,1.409)--(7.475,1.410)--(7.480,1.411)--(7.485,1.412)%
  --(7.490,1.413)--(7.496,1.414)--(7.501,1.415)--(7.506,1.416)--(7.511,1.418)--(7.516,1.419)%
  --(7.522,1.420)--(7.527,1.421)--(7.532,1.422)--(7.537,1.423)--(7.542,1.424)--(7.547,1.425)%
  --(7.553,1.426)--(7.558,1.427)--(7.563,1.429)--(7.568,1.430)--(7.573,1.431)--(7.579,1.432)%
  --(7.584,1.433)--(7.589,1.434)--(7.594,1.435)--(7.599,1.436)--(7.605,1.437)--(7.610,1.439)%
  --(7.615,1.440)--(7.620,1.441)--(7.625,1.442)--(7.631,1.443)--(7.636,1.444)--(7.641,1.445)%
  --(7.646,1.446)--(7.651,1.448)--(7.656,1.449)--(7.662,1.450)--(7.667,1.451)--(7.672,1.452)%
  --(7.677,1.453)--(7.682,1.454)--(7.688,1.456)--(7.693,1.457)--(7.698,1.458)--(7.703,1.459)%
  --(7.708,1.460)--(7.714,1.461)--(7.719,1.462)--(7.724,1.464)--(7.729,1.465)--(7.734,1.466)%
  --(7.739,1.467)--(7.745,1.468)--(7.750,1.469)--(7.755,1.470)--(7.760,1.472)--(7.765,1.473)%
  --(7.771,1.474)--(7.776,1.475)--(7.781,1.476)--(7.786,1.477)--(7.791,1.479)--(7.797,1.480)%
  --(7.802,1.481)--(7.807,1.482)--(7.812,1.483)--(7.817,1.485)--(7.823,1.486)--(7.828,1.487)%
  --(7.833,1.488)--(7.838,1.489)--(7.843,1.490)--(7.848,1.492)--(7.854,1.493)--(7.859,1.494)%
  --(7.864,1.495)--(7.869,1.496)--(7.874,1.498)--(7.880,1.499)--(7.885,1.500)--(7.890,1.501)%
  --(7.895,1.502)--(7.900,1.504)--(7.906,1.505)--(7.911,1.506)--(7.916,1.507)--(7.921,1.508)%
  --(7.926,1.510)--(7.931,1.511)--(7.937,1.512)--(7.942,1.513);
\node[gp node left,font={\fontsize{10pt}{12pt}\selectfont}] at (1.151,5.321) {\LARGE $\rho$};
\node[gp node left,font={\fontsize{10pt}{12pt}\selectfont}] at (6.005,5.321) {\large $\alpha = 2.95$};
%% coordinates of the plot area
\gpdefrectangularnode{gp plot 1}{\pgfpoint{1.012cm}{0.985cm}}{\pgfpoint{7.947cm}{5.631cm}}
\end{tikzpicture}
%% gnuplot variables
}
& 
\resizebox{0.5\linewidth}{!}{\tikzsetnextfilename{AK7_rsol_init_6} \begin{tikzpicture}[gnuplot]
%% generated with GNUPLOT 4.6p4 (Lua 5.1; terminal rev. 99, script rev. 100)
%% Tue 05 Aug 2014 02:52:57 PM EDT
\path (0.000,0.000) rectangle (8.500,6.000);
\gpfill{rgb color={1.000,1.000,1.000}} (0.828,0.985)--(7.946,0.985)--(7.946,5.630)--(0.828,5.630)--cycle;
\gpcolor{color=gp lt color border}
\gpsetlinetype{gp lt border}
\gpsetlinewidth{1.00}
\draw[gp path] (0.828,0.985)--(0.828,5.630)--(7.946,5.630)--(7.946,0.985)--cycle;
\gpcolor{color=gp lt color axes}
\gpsetlinetype{gp lt axes}
\gpsetlinewidth{2.00}
\draw[gp path] (0.828,0.985)--(7.947,0.985);
\gpcolor{color=gp lt color border}
\gpsetlinetype{gp lt border}
\draw[gp path] (0.828,0.985)--(0.900,0.985);
\draw[gp path] (7.947,0.985)--(7.875,0.985);
\gpcolor{rgb color={0.000,0.000,0.000}}
\node[gp node right,font={\fontsize{10pt}{12pt}\selectfont}] at (0.644,0.985) {-2};
\gpcolor{color=gp lt color axes}
\gpsetlinetype{gp lt axes}
\draw[gp path] (0.828,1.914)--(7.947,1.914);
\gpcolor{color=gp lt color border}
\gpsetlinetype{gp lt border}
\draw[gp path] (0.828,1.914)--(0.900,1.914);
\draw[gp path] (7.947,1.914)--(7.875,1.914);
\gpcolor{rgb color={0.000,0.000,0.000}}
\node[gp node right,font={\fontsize{10pt}{12pt}\selectfont}] at (0.644,1.914) {-1};
\gpcolor{color=gp lt color axes}
\gpsetlinetype{gp lt axes}
\draw[gp path] (0.828,2.843)--(7.947,2.843);
\gpcolor{color=gp lt color border}
\gpsetlinetype{gp lt border}
\draw[gp path] (0.828,2.843)--(0.900,2.843);
\draw[gp path] (7.947,2.843)--(7.875,2.843);
\gpcolor{rgb color={0.000,0.000,0.000}}
\node[gp node right,font={\fontsize{10pt}{12pt}\selectfont}] at (0.644,2.843) {0};
\gpcolor{color=gp lt color axes}
\gpsetlinetype{gp lt axes}
\draw[gp path] (0.828,3.773)--(7.947,3.773);
\gpcolor{color=gp lt color border}
\gpsetlinetype{gp lt border}
\draw[gp path] (0.828,3.773)--(0.900,3.773);
\draw[gp path] (7.947,3.773)--(7.875,3.773);
\gpcolor{rgb color={0.000,0.000,0.000}}
\node[gp node right,font={\fontsize{10pt}{12pt}\selectfont}] at (0.644,3.773) {1};
\gpcolor{color=gp lt color axes}
\gpsetlinetype{gp lt axes}
\draw[gp path] (0.828,4.702)--(7.947,4.702);
\gpcolor{color=gp lt color border}
\gpsetlinetype{gp lt border}
\draw[gp path] (0.828,4.702)--(0.900,4.702);
\draw[gp path] (7.947,4.702)--(7.875,4.702);
\gpcolor{rgb color={0.000,0.000,0.000}}
\node[gp node right,font={\fontsize{10pt}{12pt}\selectfont}] at (0.644,4.702) {2};
\gpcolor{color=gp lt color axes}
\gpsetlinetype{gp lt axes}
\draw[gp path] (0.828,5.631)--(7.947,5.631);
\gpcolor{color=gp lt color border}
\gpsetlinetype{gp lt border}
\draw[gp path] (0.828,5.631)--(0.900,5.631);
\draw[gp path] (7.947,5.631)--(7.875,5.631);
\gpcolor{rgb color={0.000,0.000,0.000}}
\node[gp node right,font={\fontsize{10pt}{12pt}\selectfont}] at (0.644,5.631) {3};
\gpcolor{color=gp lt color axes}
\gpsetlinetype{gp lt axes}
\draw[gp path] (0.828,0.985)--(0.828,5.631);
\gpcolor{color=gp lt color border}
\gpsetlinetype{gp lt border}
\draw[gp path] (0.828,0.985)--(0.828,1.057);
\draw[gp path] (0.828,5.631)--(0.828,5.559);
\gpcolor{rgb color={0.000,0.000,0.000}}
\node[gp node center,font={\fontsize{10pt}{12pt}\selectfont}] at (0.828,0.677) {0.3};
\gpcolor{color=gp lt color axes}
\gpsetlinetype{gp lt axes}
\draw[gp path] (2.252,0.985)--(2.252,5.631);
\gpcolor{color=gp lt color border}
\gpsetlinetype{gp lt border}
\draw[gp path] (2.252,0.985)--(2.252,1.057);
\draw[gp path] (2.252,5.631)--(2.252,5.559);
\gpcolor{rgb color={0.000,0.000,0.000}}
\node[gp node center,font={\fontsize{10pt}{12pt}\selectfont}] at (2.252,0.677) {0.4};
\gpcolor{color=gp lt color axes}
\gpsetlinetype{gp lt axes}
\draw[gp path] (3.676,0.985)--(3.676,5.631);
\gpcolor{color=gp lt color border}
\gpsetlinetype{gp lt border}
\draw[gp path] (3.676,0.985)--(3.676,1.057);
\draw[gp path] (3.676,5.631)--(3.676,5.559);
\gpcolor{rgb color={0.000,0.000,0.000}}
\node[gp node center,font={\fontsize{10pt}{12pt}\selectfont}] at (3.676,0.677) {0.5};
\gpcolor{color=gp lt color axes}
\gpsetlinetype{gp lt axes}
\draw[gp path] (5.099,0.985)--(5.099,5.631);
\gpcolor{color=gp lt color border}
\gpsetlinetype{gp lt border}
\draw[gp path] (5.099,0.985)--(5.099,1.057);
\draw[gp path] (5.099,5.631)--(5.099,5.559);
\gpcolor{rgb color={0.000,0.000,0.000}}
\node[gp node center,font={\fontsize{10pt}{12pt}\selectfont}] at (5.099,0.677) {0.6};
\gpcolor{color=gp lt color axes}
\gpsetlinetype{gp lt axes}
\draw[gp path] (6.523,0.985)--(6.523,5.631);
\gpcolor{color=gp lt color border}
\gpsetlinetype{gp lt border}
\draw[gp path] (6.523,0.985)--(6.523,1.057);
\draw[gp path] (6.523,5.631)--(6.523,5.559);
\gpcolor{rgb color={0.000,0.000,0.000}}
\node[gp node center,font={\fontsize{10pt}{12pt}\selectfont}] at (6.523,0.677) {0.7};
\gpcolor{color=gp lt color axes}
\gpsetlinetype{gp lt axes}
\draw[gp path] (7.947,0.985)--(7.947,5.631);
\gpcolor{color=gp lt color border}
\gpsetlinetype{gp lt border}
\draw[gp path] (7.947,0.985)--(7.947,1.057);
\draw[gp path] (7.947,5.631)--(7.947,5.559);
\gpcolor{rgb color={0.000,0.000,0.000}}
\node[gp node center,font={\fontsize{10pt}{12pt}\selectfont}] at (7.947,0.677) {0.8};
\gpcolor{color=gp lt color border}
\draw[gp path] (0.828,5.631)--(0.828,0.985)--(7.947,0.985)--(7.947,5.631)--cycle;
\gpcolor{rgb color={0.000,0.000,0.000}}
\node[gp node center,font={\fontsize{10pt}{12pt}\selectfont}] at (4.387,0.215) {\large $x$};
\gpcolor{rgb color={1.000,0.000,0.000}}
\gpsetlinewidth{4.00}
\gpsetpointsize{2.67}
\gppoint{gp mark 7}{(0.830,5.292)}
\gppoint{gp mark 7}{(0.834,5.288)}
\gppoint{gp mark 7}{(0.837,5.283)}
\gppoint{gp mark 7}{(0.841,5.279)}
\gppoint{gp mark 7}{(0.844,5.274)}
\gppoint{gp mark 7}{(0.848,5.270)}
\gppoint{gp mark 7}{(0.851,5.266)}
\gppoint{gp mark 7}{(0.855,5.261)}
\gppoint{gp mark 7}{(0.858,5.257)}
\gppoint{gp mark 7}{(0.862,5.252)}
\gppoint{gp mark 7}{(0.865,5.248)}
\gppoint{gp mark 7}{(0.869,5.243)}
\gppoint{gp mark 7}{(0.872,5.239)}
\gppoint{gp mark 7}{(0.876,5.235)}
\gppoint{gp mark 7}{(0.879,5.230)}
\gppoint{gp mark 7}{(0.883,5.226)}
\gppoint{gp mark 7}{(0.886,5.221)}
\gppoint{gp mark 7}{(0.890,5.217)}
\gppoint{gp mark 7}{(0.893,5.212)}
\gppoint{gp mark 7}{(0.896,5.208)}
\gppoint{gp mark 7}{(0.900,5.203)}
\gppoint{gp mark 7}{(0.903,5.199)}
\gppoint{gp mark 7}{(0.907,5.194)}
\gppoint{gp mark 7}{(0.910,5.190)}
\gppoint{gp mark 7}{(0.914,5.185)}
\gppoint{gp mark 7}{(0.917,5.181)}
\gppoint{gp mark 7}{(0.921,5.177)}
\gppoint{gp mark 7}{(0.924,5.172)}
\gppoint{gp mark 7}{(0.928,5.168)}
\gppoint{gp mark 7}{(0.931,5.163)}
\gppoint{gp mark 7}{(0.935,5.159)}
\gppoint{gp mark 7}{(0.938,5.154)}
\gppoint{gp mark 7}{(0.942,5.150)}
\gppoint{gp mark 7}{(0.945,5.145)}
\gppoint{gp mark 7}{(0.949,5.141)}
\gppoint{gp mark 7}{(0.952,5.136)}
\gppoint{gp mark 7}{(0.956,5.131)}
\gppoint{gp mark 7}{(0.959,5.127)}
\gppoint{gp mark 7}{(0.963,5.122)}
\gppoint{gp mark 7}{(0.966,5.118)}
\gppoint{gp mark 7}{(0.969,5.113)}
\gppoint{gp mark 7}{(0.973,5.109)}
\gppoint{gp mark 7}{(0.976,5.104)}
\gppoint{gp mark 7}{(0.980,5.100)}
\gppoint{gp mark 7}{(0.983,5.095)}
\gppoint{gp mark 7}{(0.987,5.091)}
\gppoint{gp mark 7}{(0.990,5.086)}
\gppoint{gp mark 7}{(0.994,5.081)}
\gppoint{gp mark 7}{(0.997,5.077)}
\gppoint{gp mark 7}{(1.001,5.072)}
\gppoint{gp mark 7}{(1.004,5.068)}
\gppoint{gp mark 7}{(1.008,5.063)}
\gppoint{gp mark 7}{(1.011,5.059)}
\gppoint{gp mark 7}{(1.015,5.054)}
\gppoint{gp mark 7}{(1.018,5.049)}
\gppoint{gp mark 7}{(1.022,5.045)}
\gppoint{gp mark 7}{(1.025,5.040)}
\gppoint{gp mark 7}{(1.029,5.036)}
\gppoint{gp mark 7}{(1.032,5.031)}
\gppoint{gp mark 7}{(1.036,5.026)}
\gppoint{gp mark 7}{(1.039,5.022)}
\gppoint{gp mark 7}{(1.042,5.017)}
\gppoint{gp mark 7}{(1.046,5.012)}
\gppoint{gp mark 7}{(1.049,5.008)}
\gppoint{gp mark 7}{(1.053,5.003)}
\gppoint{gp mark 7}{(1.056,4.998)}
\gppoint{gp mark 7}{(1.060,4.994)}
\gppoint{gp mark 7}{(1.063,4.989)}
\gppoint{gp mark 7}{(1.067,4.985)}
\gppoint{gp mark 7}{(1.070,4.980)}
\gppoint{gp mark 7}{(1.074,4.975)}
\gppoint{gp mark 7}{(1.077,4.970)}
\gppoint{gp mark 7}{(1.081,4.966)}
\gppoint{gp mark 7}{(1.084,4.961)}
\gppoint{gp mark 7}{(1.088,4.956)}
\gppoint{gp mark 7}{(1.091,4.952)}
\gppoint{gp mark 7}{(1.095,4.947)}
\gppoint{gp mark 7}{(1.098,4.942)}
\gppoint{gp mark 7}{(1.102,4.938)}
\gppoint{gp mark 7}{(1.105,4.933)}
\gppoint{gp mark 7}{(1.109,4.928)}
\gppoint{gp mark 7}{(1.112,4.923)}
\gppoint{gp mark 7}{(1.115,4.919)}
\gppoint{gp mark 7}{(1.119,4.914)}
\gppoint{gp mark 7}{(1.122,4.909)}
\gppoint{gp mark 7}{(1.126,4.904)}
\gppoint{gp mark 7}{(1.129,4.900)}
\gppoint{gp mark 7}{(1.133,4.895)}
\gppoint{gp mark 7}{(1.136,4.890)}
\gppoint{gp mark 7}{(1.140,4.885)}
\gppoint{gp mark 7}{(1.143,4.881)}
\gppoint{gp mark 7}{(1.147,4.876)}
\gppoint{gp mark 7}{(1.150,4.871)}
\gppoint{gp mark 7}{(1.154,4.866)}
\gppoint{gp mark 7}{(1.157,4.861)}
\gppoint{gp mark 7}{(1.161,4.857)}
\gppoint{gp mark 7}{(1.164,4.852)}
\gppoint{gp mark 7}{(1.168,4.847)}
\gppoint{gp mark 7}{(1.171,4.842)}
\gppoint{gp mark 7}{(1.175,4.837)}
\gppoint{gp mark 7}{(1.178,4.832)}
\gppoint{gp mark 7}{(1.182,4.828)}
\gppoint{gp mark 7}{(1.185,4.823)}
\gppoint{gp mark 7}{(1.188,4.818)}
\gppoint{gp mark 7}{(1.192,4.813)}
\gppoint{gp mark 7}{(1.195,4.808)}
\gppoint{gp mark 7}{(1.199,4.803)}
\gppoint{gp mark 7}{(1.202,4.798)}
\gppoint{gp mark 7}{(1.206,4.793)}
\gppoint{gp mark 7}{(1.209,4.788)}
\gppoint{gp mark 7}{(1.213,4.784)}
\gppoint{gp mark 7}{(1.216,4.779)}
\gppoint{gp mark 7}{(1.220,4.774)}
\gppoint{gp mark 7}{(1.223,4.769)}
\gppoint{gp mark 7}{(1.227,4.764)}
\gppoint{gp mark 7}{(1.230,4.759)}
\gppoint{gp mark 7}{(1.234,4.754)}
\gppoint{gp mark 7}{(1.237,4.749)}
\gppoint{gp mark 7}{(1.241,4.744)}
\gppoint{gp mark 7}{(1.244,4.739)}
\gppoint{gp mark 7}{(1.248,4.734)}
\gppoint{gp mark 7}{(1.251,4.729)}
\gppoint{gp mark 7}{(1.255,4.724)}
\gppoint{gp mark 7}{(1.258,4.719)}
\gppoint{gp mark 7}{(1.261,4.714)}
\gppoint{gp mark 7}{(1.265,4.709)}
\gppoint{gp mark 7}{(1.268,4.704)}
\gppoint{gp mark 7}{(1.272,4.699)}
\gppoint{gp mark 7}{(1.275,4.694)}
\gppoint{gp mark 7}{(1.279,4.689)}
\gppoint{gp mark 7}{(1.282,4.684)}
\gppoint{gp mark 7}{(1.286,4.678)}
\gppoint{gp mark 7}{(1.289,4.673)}
\gppoint{gp mark 7}{(1.293,4.668)}
\gppoint{gp mark 7}{(1.296,4.663)}
\gppoint{gp mark 7}{(1.300,4.658)}
\gppoint{gp mark 7}{(1.303,4.653)}
\gppoint{gp mark 7}{(1.307,4.648)}
\gppoint{gp mark 7}{(1.310,4.643)}
\gppoint{gp mark 7}{(1.314,4.637)}
\gppoint{gp mark 7}{(1.317,4.632)}
\gppoint{gp mark 7}{(1.321,4.627)}
\gppoint{gp mark 7}{(1.324,4.622)}
\gppoint{gp mark 7}{(1.328,4.617)}
\gppoint{gp mark 7}{(1.331,4.611)}
\gppoint{gp mark 7}{(1.334,4.606)}
\gppoint{gp mark 7}{(1.338,4.601)}
\gppoint{gp mark 7}{(1.341,4.596)}
\gppoint{gp mark 7}{(1.345,4.590)}
\gppoint{gp mark 7}{(1.348,4.585)}
\gppoint{gp mark 7}{(1.352,4.580)}
\gppoint{gp mark 7}{(1.355,4.575)}
\gppoint{gp mark 7}{(1.359,4.569)}
\gppoint{gp mark 7}{(1.362,4.565)}
\gppoint{gp mark 7}{(1.366,4.561)}
\gppoint{gp mark 7}{(1.369,4.557)}
\gppoint{gp mark 7}{(1.373,4.556)}
\gppoint{gp mark 7}{(1.376,4.555)}
\gppoint{gp mark 7}{(1.380,4.555)}
\gppoint{gp mark 7}{(1.383,4.555)}
\gppoint{gp mark 7}{(1.387,4.555)}
\gppoint{gp mark 7}{(1.390,4.555)}
\gppoint{gp mark 7}{(1.394,4.556)}
\gppoint{gp mark 7}{(1.397,4.558)}
\gppoint{gp mark 7}{(1.401,4.561)}
\gppoint{gp mark 7}{(1.404,4.564)}
\gppoint{gp mark 7}{(1.407,4.565)}
\gppoint{gp mark 7}{(1.411,4.565)}
\gppoint{gp mark 7}{(1.414,4.565)}
\gppoint{gp mark 7}{(1.418,4.565)}
\gppoint{gp mark 7}{(1.421,4.565)}
\gppoint{gp mark 7}{(1.425,4.565)}
\gppoint{gp mark 7}{(1.428,4.565)}
\gppoint{gp mark 7}{(1.432,4.565)}
\gppoint{gp mark 7}{(1.435,4.565)}
\gppoint{gp mark 7}{(1.439,4.565)}
\gppoint{gp mark 7}{(1.442,4.565)}
\gppoint{gp mark 7}{(1.446,4.565)}
\gppoint{gp mark 7}{(1.449,4.565)}
\gppoint{gp mark 7}{(1.453,4.565)}
\gppoint{gp mark 7}{(1.456,4.565)}
\gppoint{gp mark 7}{(1.460,4.565)}
\gppoint{gp mark 7}{(1.463,4.566)}
\gppoint{gp mark 7}{(1.467,4.566)}
\gppoint{gp mark 7}{(1.470,4.566)}
\gppoint{gp mark 7}{(1.474,4.566)}
\gppoint{gp mark 7}{(1.477,4.567)}
\gppoint{gp mark 7}{(1.480,4.568)}
\gppoint{gp mark 7}{(1.484,4.568)}
\gppoint{gp mark 7}{(1.487,4.569)}
\gppoint{gp mark 7}{(1.491,4.569)}
\gppoint{gp mark 7}{(1.494,4.569)}
\gppoint{gp mark 7}{(1.498,4.569)}
\gppoint{gp mark 7}{(1.501,4.569)}
\gppoint{gp mark 7}{(1.505,4.569)}
\gppoint{gp mark 7}{(1.508,4.569)}
\gppoint{gp mark 7}{(1.512,4.569)}
\gppoint{gp mark 7}{(1.515,4.569)}
\gppoint{gp mark 7}{(1.519,4.569)}
\gppoint{gp mark 7}{(1.522,4.569)}
\gppoint{gp mark 7}{(1.526,4.569)}
\gppoint{gp mark 7}{(1.529,4.569)}
\gppoint{gp mark 7}{(1.533,4.569)}
\gppoint{gp mark 7}{(1.536,4.569)}
\gppoint{gp mark 7}{(1.540,4.568)}
\gppoint{gp mark 7}{(1.543,4.568)}
\gppoint{gp mark 7}{(1.547,4.567)}
\gppoint{gp mark 7}{(1.550,4.566)}
\gppoint{gp mark 7}{(1.553,4.566)}
\gppoint{gp mark 7}{(1.557,4.566)}
\gppoint{gp mark 7}{(1.560,4.566)}
\gppoint{gp mark 7}{(1.564,4.565)}
\gppoint{gp mark 7}{(1.567,4.565)}
\gppoint{gp mark 7}{(1.571,4.564)}
\gppoint{gp mark 7}{(1.574,4.564)}
\gppoint{gp mark 7}{(1.578,4.564)}
\gppoint{gp mark 7}{(1.581,4.564)}
\gppoint{gp mark 7}{(1.585,4.564)}
\gppoint{gp mark 7}{(1.588,4.564)}
\gppoint{gp mark 7}{(1.592,4.564)}
\gppoint{gp mark 7}{(1.595,4.564)}
\gppoint{gp mark 7}{(1.599,4.564)}
\gppoint{gp mark 7}{(1.602,4.563)}
\gppoint{gp mark 7}{(1.606,4.563)}
\gppoint{gp mark 7}{(1.609,4.563)}
\gppoint{gp mark 7}{(1.613,4.563)}
\gppoint{gp mark 7}{(1.616,4.563)}
\gppoint{gp mark 7}{(1.620,4.563)}
\gppoint{gp mark 7}{(1.623,4.563)}
\gppoint{gp mark 7}{(1.626,4.563)}
\gppoint{gp mark 7}{(1.630,4.564)}
\gppoint{gp mark 7}{(1.633,4.564)}
\gppoint{gp mark 7}{(1.637,4.564)}
\gppoint{gp mark 7}{(1.640,4.564)}
\gppoint{gp mark 7}{(1.644,4.564)}
\gppoint{gp mark 7}{(1.647,4.564)}
\gppoint{gp mark 7}{(1.651,4.563)}
\gppoint{gp mark 7}{(1.654,4.563)}
\gppoint{gp mark 7}{(1.658,4.562)}
\gppoint{gp mark 7}{(1.661,4.562)}
\gppoint{gp mark 7}{(1.665,4.562)}
\gppoint{gp mark 7}{(1.668,4.562)}
\gppoint{gp mark 7}{(1.672,4.561)}
\gppoint{gp mark 7}{(1.675,4.561)}
\gppoint{gp mark 7}{(1.679,4.561)}
\gppoint{gp mark 7}{(1.682,4.561)}
\gppoint{gp mark 7}{(1.686,4.561)}
\gppoint{gp mark 7}{(1.689,4.561)}
\gppoint{gp mark 7}{(1.692,4.561)}
\gppoint{gp mark 7}{(1.696,4.561)}
\gppoint{gp mark 7}{(1.699,4.561)}
\gppoint{gp mark 7}{(1.703,4.561)}
\gppoint{gp mark 7}{(1.706,4.561)}
\gppoint{gp mark 7}{(1.710,4.561)}
\gppoint{gp mark 7}{(1.713,4.561)}
\gppoint{gp mark 7}{(1.717,4.562)}
\gppoint{gp mark 7}{(1.720,4.562)}
\gppoint{gp mark 7}{(1.724,4.562)}
\gppoint{gp mark 7}{(1.727,4.561)}
\gppoint{gp mark 7}{(1.731,4.561)}
\gppoint{gp mark 7}{(1.734,4.561)}
\gppoint{gp mark 7}{(1.738,4.560)}
\gppoint{gp mark 7}{(1.741,4.560)}
\gppoint{gp mark 7}{(1.745,4.560)}
\gppoint{gp mark 7}{(1.748,4.560)}
\gppoint{gp mark 7}{(1.752,4.559)}
\gppoint{gp mark 7}{(1.755,4.557)}
\gppoint{gp mark 7}{(1.759,4.555)}
\gppoint{gp mark 7}{(1.762,4.541)}
\gppoint{gp mark 7}{(1.765,4.388)}
\gppoint{gp mark 7}{(1.769,2.682)}
\gppoint{gp mark 7}{(1.772,1.289)}
\gppoint{gp mark 7}{(1.776,1.150)}
\gppoint{gp mark 7}{(1.779,1.133)}
\gppoint{gp mark 7}{(1.783,1.138)}
\gppoint{gp mark 7}{(1.786,1.147)}
\gppoint{gp mark 7}{(1.790,1.150)}
\gppoint{gp mark 7}{(1.793,1.151)}
\gppoint{gp mark 7}{(1.797,1.152)}
\gppoint{gp mark 7}{(1.800,1.151)}
\gppoint{gp mark 7}{(1.804,1.151)}
\gppoint{gp mark 7}{(1.807,1.151)}
\gppoint{gp mark 7}{(1.811,1.152)}
\gppoint{gp mark 7}{(1.814,1.152)}
\gppoint{gp mark 7}{(1.818,1.152)}
\gppoint{gp mark 7}{(1.821,1.152)}
\gppoint{gp mark 7}{(1.825,1.152)}
\gppoint{gp mark 7}{(1.828,1.152)}
\gppoint{gp mark 7}{(1.832,1.153)}
\gppoint{gp mark 7}{(1.835,1.154)}
\gppoint{gp mark 7}{(1.838,1.154)}
\gppoint{gp mark 7}{(1.842,1.154)}
\gppoint{gp mark 7}{(1.845,1.153)}
\gppoint{gp mark 7}{(1.849,1.153)}
\gppoint{gp mark 7}{(1.852,1.153)}
\gppoint{gp mark 7}{(1.856,1.153)}
\gppoint{gp mark 7}{(1.859,1.153)}
\gppoint{gp mark 7}{(1.863,1.152)}
\gppoint{gp mark 7}{(1.866,1.152)}
\gppoint{gp mark 7}{(1.870,1.152)}
\gppoint{gp mark 7}{(1.873,1.152)}
\gppoint{gp mark 7}{(1.877,1.152)}
\gppoint{gp mark 7}{(1.880,1.153)}
\gppoint{gp mark 7}{(1.884,1.153)}
\gppoint{gp mark 7}{(1.887,1.153)}
\gppoint{gp mark 7}{(1.891,1.154)}
\gppoint{gp mark 7}{(1.894,1.153)}
\gppoint{gp mark 7}{(1.898,1.153)}
\gppoint{gp mark 7}{(1.901,1.153)}
\gppoint{gp mark 7}{(1.905,1.153)}
\gppoint{gp mark 7}{(1.908,1.153)}
\gppoint{gp mark 7}{(1.911,1.153)}
\gppoint{gp mark 7}{(1.915,1.153)}
\gppoint{gp mark 7}{(1.918,1.153)}
\gppoint{gp mark 7}{(1.922,1.153)}
\gppoint{gp mark 7}{(1.925,1.153)}
\gppoint{gp mark 7}{(1.929,1.153)}
\gppoint{gp mark 7}{(1.932,1.153)}
\gppoint{gp mark 7}{(1.936,1.152)}
\gppoint{gp mark 7}{(1.939,1.152)}
\gppoint{gp mark 7}{(1.943,1.152)}
\gppoint{gp mark 7}{(1.946,1.152)}
\gppoint{gp mark 7}{(1.950,1.152)}
\gppoint{gp mark 7}{(1.953,1.152)}
\gppoint{gp mark 7}{(1.957,1.153)}
\gppoint{gp mark 7}{(1.960,1.153)}
\gppoint{gp mark 7}{(1.964,1.153)}
\gppoint{gp mark 7}{(1.967,1.153)}
\gppoint{gp mark 7}{(1.971,1.153)}
\gppoint{gp mark 7}{(1.974,1.153)}
\gppoint{gp mark 7}{(1.978,1.153)}
\gppoint{gp mark 7}{(1.981,1.153)}
\gppoint{gp mark 7}{(1.984,1.153)}
\gppoint{gp mark 7}{(1.988,1.153)}
\gppoint{gp mark 7}{(1.991,1.153)}
\gppoint{gp mark 7}{(1.995,1.153)}
\gppoint{gp mark 7}{(1.998,1.153)}
\gppoint{gp mark 7}{(2.002,1.153)}
\gppoint{gp mark 7}{(2.005,1.153)}
\gppoint{gp mark 7}{(2.009,1.153)}
\gppoint{gp mark 7}{(2.012,1.152)}
\gppoint{gp mark 7}{(2.016,1.152)}
\gppoint{gp mark 7}{(2.019,1.152)}
\gppoint{gp mark 7}{(2.023,1.153)}
\gppoint{gp mark 7}{(2.026,1.153)}
\gppoint{gp mark 7}{(2.030,1.154)}
\gppoint{gp mark 7}{(2.033,1.154)}
\gppoint{gp mark 7}{(2.037,1.154)}
\gppoint{gp mark 7}{(2.040,1.154)}
\gppoint{gp mark 7}{(2.044,1.154)}
\gppoint{gp mark 7}{(2.047,1.153)}
\gppoint{gp mark 7}{(2.051,1.153)}
\gppoint{gp mark 7}{(2.054,1.153)}
\gppoint{gp mark 7}{(2.057,1.153)}
\gppoint{gp mark 7}{(2.061,1.153)}
\gppoint{gp mark 7}{(2.064,1.153)}
\gppoint{gp mark 7}{(2.068,1.153)}
\gppoint{gp mark 7}{(2.071,1.153)}
\gppoint{gp mark 7}{(2.075,1.153)}
\gppoint{gp mark 7}{(2.078,1.153)}
\gppoint{gp mark 7}{(2.082,1.153)}
\gppoint{gp mark 7}{(2.085,1.153)}
\gppoint{gp mark 7}{(2.089,1.153)}
\gppoint{gp mark 7}{(2.092,1.153)}
\gppoint{gp mark 7}{(2.096,1.153)}
\gppoint{gp mark 7}{(2.099,1.153)}
\gppoint{gp mark 7}{(2.103,1.154)}
\gppoint{gp mark 7}{(2.106,1.154)}
\gppoint{gp mark 7}{(2.110,1.154)}
\gppoint{gp mark 7}{(2.113,1.153)}
\gppoint{gp mark 7}{(2.117,1.153)}
\gppoint{gp mark 7}{(2.120,1.153)}
\gppoint{gp mark 7}{(2.124,1.153)}
\gppoint{gp mark 7}{(2.127,1.153)}
\gppoint{gp mark 7}{(2.130,1.153)}
\gppoint{gp mark 7}{(2.134,1.153)}
\gppoint{gp mark 7}{(2.137,1.153)}
\gppoint{gp mark 7}{(2.141,1.153)}
\gppoint{gp mark 7}{(2.144,1.153)}
\gppoint{gp mark 7}{(2.148,1.153)}
\gppoint{gp mark 7}{(2.151,1.153)}
\gppoint{gp mark 7}{(2.155,1.153)}
\gppoint{gp mark 7}{(2.158,1.153)}
\gppoint{gp mark 7}{(2.162,1.153)}
\gppoint{gp mark 7}{(2.165,1.153)}
\gppoint{gp mark 7}{(2.169,1.153)}
\gppoint{gp mark 7}{(2.172,1.153)}
\gppoint{gp mark 7}{(2.176,1.153)}
\gppoint{gp mark 7}{(2.179,1.153)}
\gppoint{gp mark 7}{(2.183,1.153)}
\gppoint{gp mark 7}{(2.186,1.153)}
\gppoint{gp mark 7}{(2.190,1.154)}
\gppoint{gp mark 7}{(2.193,1.154)}
\gppoint{gp mark 7}{(2.197,1.153)}
\gppoint{gp mark 7}{(2.200,1.153)}
\gppoint{gp mark 7}{(2.203,1.153)}
\gppoint{gp mark 7}{(2.207,1.153)}
\gppoint{gp mark 7}{(2.210,1.152)}
\gppoint{gp mark 7}{(2.214,1.152)}
\gppoint{gp mark 7}{(2.217,1.152)}
\gppoint{gp mark 7}{(2.221,1.152)}
\gppoint{gp mark 7}{(2.224,1.153)}
\gppoint{gp mark 7}{(2.228,1.153)}
\gppoint{gp mark 7}{(2.231,1.153)}
\gppoint{gp mark 7}{(2.235,1.154)}
\gppoint{gp mark 7}{(2.238,1.154)}
\gppoint{gp mark 7}{(2.242,1.154)}
\gppoint{gp mark 7}{(2.245,1.154)}
\gppoint{gp mark 7}{(2.249,1.154)}
\gppoint{gp mark 7}{(2.252,1.153)}
\gppoint{gp mark 7}{(2.256,1.153)}
\gppoint{gp mark 7}{(2.259,1.154)}
\gppoint{gp mark 7}{(2.263,1.153)}
\gppoint{gp mark 7}{(2.266,1.153)}
\gppoint{gp mark 7}{(2.270,1.153)}
\gppoint{gp mark 7}{(2.273,1.153)}
\gppoint{gp mark 7}{(2.276,1.153)}
\gppoint{gp mark 7}{(2.280,1.153)}
\gppoint{gp mark 7}{(2.283,1.153)}
\gppoint{gp mark 7}{(2.287,1.153)}
\gppoint{gp mark 7}{(2.290,1.153)}
\gppoint{gp mark 7}{(2.294,1.153)}
\gppoint{gp mark 7}{(2.297,1.153)}
\gppoint{gp mark 7}{(2.301,1.153)}
\gppoint{gp mark 7}{(2.304,1.153)}
\gppoint{gp mark 7}{(2.308,1.153)}
\gppoint{gp mark 7}{(2.311,1.154)}
\gppoint{gp mark 7}{(2.315,1.153)}
\gppoint{gp mark 7}{(2.318,1.153)}
\gppoint{gp mark 7}{(2.322,1.153)}
\gppoint{gp mark 7}{(2.325,1.153)}
\gppoint{gp mark 7}{(2.329,1.153)}
\gppoint{gp mark 7}{(2.332,1.153)}
\gppoint{gp mark 7}{(2.336,1.153)}
\gppoint{gp mark 7}{(2.339,1.153)}
\gppoint{gp mark 7}{(2.343,1.153)}
\gppoint{gp mark 7}{(2.346,1.153)}
\gppoint{gp mark 7}{(2.349,1.153)}
\gppoint{gp mark 7}{(2.353,1.153)}
\gppoint{gp mark 7}{(2.356,1.153)}
\gppoint{gp mark 7}{(2.360,1.153)}
\gppoint{gp mark 7}{(2.363,1.153)}
\gppoint{gp mark 7}{(2.367,1.153)}
\gppoint{gp mark 7}{(2.370,1.154)}
\gppoint{gp mark 7}{(2.374,1.154)}
\gppoint{gp mark 7}{(2.377,1.154)}
\gppoint{gp mark 7}{(2.381,1.154)}
\gppoint{gp mark 7}{(2.384,1.153)}
\gppoint{gp mark 7}{(2.388,1.153)}
\gppoint{gp mark 7}{(2.391,1.153)}
\gppoint{gp mark 7}{(2.395,1.153)}
\gppoint{gp mark 7}{(2.398,1.153)}
\gppoint{gp mark 7}{(2.402,1.153)}
\gppoint{gp mark 7}{(2.405,1.153)}
\gppoint{gp mark 7}{(2.409,1.153)}
\gppoint{gp mark 7}{(2.412,1.153)}
\gppoint{gp mark 7}{(2.416,1.153)}
\gppoint{gp mark 7}{(2.419,1.153)}
\gppoint{gp mark 7}{(2.422,1.153)}
\gppoint{gp mark 7}{(2.426,1.153)}
\gppoint{gp mark 7}{(2.429,1.153)}
\gppoint{gp mark 7}{(2.433,1.153)}
\gppoint{gp mark 7}{(2.436,1.153)}
\gppoint{gp mark 7}{(2.440,1.154)}
\gppoint{gp mark 7}{(2.443,1.154)}
\gppoint{gp mark 7}{(2.447,1.154)}
\gppoint{gp mark 7}{(2.450,1.154)}
\gppoint{gp mark 7}{(2.454,1.154)}
\gppoint{gp mark 7}{(2.457,1.154)}
\gppoint{gp mark 7}{(2.461,1.154)}
\gppoint{gp mark 7}{(2.464,1.153)}
\gppoint{gp mark 7}{(2.468,1.153)}
\gppoint{gp mark 7}{(2.471,1.153)}
\gppoint{gp mark 7}{(2.475,1.153)}
\gppoint{gp mark 7}{(2.478,1.153)}
\gppoint{gp mark 7}{(2.482,1.153)}
\gppoint{gp mark 7}{(2.485,1.153)}
\gppoint{gp mark 7}{(2.489,1.153)}
\gppoint{gp mark 7}{(2.492,1.153)}
\gppoint{gp mark 7}{(2.495,1.153)}
\gppoint{gp mark 7}{(2.499,1.153)}
\gppoint{gp mark 7}{(2.502,1.153)}
\gppoint{gp mark 7}{(2.506,1.154)}
\gppoint{gp mark 7}{(2.509,1.154)}
\gppoint{gp mark 7}{(2.513,1.154)}
\gppoint{gp mark 7}{(2.516,1.154)}
\gppoint{gp mark 7}{(2.520,1.154)}
\gppoint{gp mark 7}{(2.523,1.154)}
\gppoint{gp mark 7}{(2.527,1.154)}
\gppoint{gp mark 7}{(2.530,1.154)}
\gppoint{gp mark 7}{(2.534,1.153)}
\gppoint{gp mark 7}{(2.537,1.153)}
\gppoint{gp mark 7}{(2.541,1.153)}
\gppoint{gp mark 7}{(2.544,1.153)}
\gppoint{gp mark 7}{(2.548,1.153)}
\gppoint{gp mark 7}{(2.551,1.152)}
\gppoint{gp mark 7}{(2.555,1.153)}
\gppoint{gp mark 7}{(2.558,1.153)}
\gppoint{gp mark 7}{(2.562,1.153)}
\gppoint{gp mark 7}{(2.565,1.153)}
\gppoint{gp mark 7}{(2.568,1.154)}
\gppoint{gp mark 7}{(2.572,1.154)}
\gppoint{gp mark 7}{(2.575,1.154)}
\gppoint{gp mark 7}{(2.579,1.154)}
\gppoint{gp mark 7}{(2.582,1.154)}
\gppoint{gp mark 7}{(2.586,1.154)}
\gppoint{gp mark 7}{(2.589,1.153)}
\gppoint{gp mark 7}{(2.593,1.153)}
\gppoint{gp mark 7}{(2.596,1.153)}
\gppoint{gp mark 7}{(2.600,1.153)}
\gppoint{gp mark 7}{(2.603,1.153)}
\gppoint{gp mark 7}{(2.607,1.153)}
\gppoint{gp mark 7}{(2.610,1.153)}
\gppoint{gp mark 7}{(2.614,1.153)}
\gppoint{gp mark 7}{(2.617,1.152)}
\gppoint{gp mark 7}{(2.621,1.153)}
\gppoint{gp mark 7}{(2.624,1.153)}
\gppoint{gp mark 7}{(2.628,1.153)}
\gppoint{gp mark 7}{(2.631,1.153)}
\gppoint{gp mark 7}{(2.635,1.154)}
\gppoint{gp mark 7}{(2.638,1.154)}
\gppoint{gp mark 7}{(2.641,1.154)}
\gppoint{gp mark 7}{(2.645,1.154)}
\gppoint{gp mark 7}{(2.648,1.154)}
\gppoint{gp mark 7}{(2.652,1.153)}
\gppoint{gp mark 7}{(2.655,1.153)}
\gppoint{gp mark 7}{(2.659,1.153)}
\gppoint{gp mark 7}{(2.662,1.153)}
\gppoint{gp mark 7}{(2.666,1.153)}
\gppoint{gp mark 7}{(2.669,1.153)}
\gppoint{gp mark 7}{(2.673,1.153)}
\gppoint{gp mark 7}{(2.676,1.153)}
\gppoint{gp mark 7}{(2.680,1.153)}
\gppoint{gp mark 7}{(2.683,1.153)}
\gppoint{gp mark 7}{(2.687,1.153)}
\gppoint{gp mark 7}{(2.690,1.153)}
\gppoint{gp mark 7}{(2.694,1.153)}
\gppoint{gp mark 7}{(2.697,1.153)}
\gppoint{gp mark 7}{(2.701,1.153)}
\gppoint{gp mark 7}{(2.704,1.154)}
\gppoint{gp mark 7}{(2.708,1.154)}
\gppoint{gp mark 7}{(2.711,1.154)}
\gppoint{gp mark 7}{(2.714,1.154)}
\gppoint{gp mark 7}{(2.718,1.154)}
\gppoint{gp mark 7}{(2.721,1.154)}
\gppoint{gp mark 7}{(2.725,1.154)}
\gppoint{gp mark 7}{(2.728,1.153)}
\gppoint{gp mark 7}{(2.732,1.153)}
\gppoint{gp mark 7}{(2.735,1.153)}
\gppoint{gp mark 7}{(2.739,1.153)}
\gppoint{gp mark 7}{(2.742,1.153)}
\gppoint{gp mark 7}{(2.746,1.153)}
\gppoint{gp mark 7}{(2.749,1.153)}
\gppoint{gp mark 7}{(2.753,1.153)}
\gppoint{gp mark 7}{(2.756,1.153)}
\gppoint{gp mark 7}{(2.760,1.153)}
\gppoint{gp mark 7}{(2.763,1.153)}
\gppoint{gp mark 7}{(2.767,1.153)}
\gppoint{gp mark 7}{(2.770,1.153)}
\gppoint{gp mark 7}{(2.774,1.153)}
\gppoint{gp mark 7}{(2.777,1.153)}
\gppoint{gp mark 7}{(2.781,1.153)}
\gppoint{gp mark 7}{(2.784,1.153)}
\gppoint{gp mark 7}{(2.787,1.153)}
\gppoint{gp mark 7}{(2.791,1.153)}
\gppoint{gp mark 7}{(2.794,1.153)}
\gppoint{gp mark 7}{(2.798,1.153)}
\gppoint{gp mark 7}{(2.801,1.153)}
\gppoint{gp mark 7}{(2.805,1.153)}
\gppoint{gp mark 7}{(2.808,1.152)}
\gppoint{gp mark 7}{(2.812,1.153)}
\gppoint{gp mark 7}{(2.815,1.153)}
\gppoint{gp mark 7}{(2.819,1.153)}
\gppoint{gp mark 7}{(2.822,1.153)}
\gppoint{gp mark 7}{(2.826,1.153)}
\gppoint{gp mark 7}{(2.829,1.153)}
\gppoint{gp mark 7}{(2.833,1.153)}
\gppoint{gp mark 7}{(2.836,1.154)}
\gppoint{gp mark 7}{(2.840,1.154)}
\gppoint{gp mark 7}{(2.843,1.154)}
\gppoint{gp mark 7}{(2.847,1.154)}
\gppoint{gp mark 7}{(2.850,1.154)}
\gppoint{gp mark 7}{(2.854,1.154)}
\gppoint{gp mark 7}{(2.857,1.154)}
\gppoint{gp mark 7}{(2.860,1.153)}
\gppoint{gp mark 7}{(2.864,1.153)}
\gppoint{gp mark 7}{(2.867,1.153)}
\gppoint{gp mark 7}{(2.871,1.153)}
\gppoint{gp mark 7}{(2.874,1.153)}
\gppoint{gp mark 7}{(2.878,1.153)}
\gppoint{gp mark 7}{(2.881,1.153)}
\gppoint{gp mark 7}{(2.885,1.153)}
\gppoint{gp mark 7}{(2.888,1.153)}
\gppoint{gp mark 7}{(2.892,1.153)}
\gppoint{gp mark 7}{(2.895,1.153)}
\gppoint{gp mark 7}{(2.899,1.153)}
\gppoint{gp mark 7}{(2.902,1.153)}
\gppoint{gp mark 7}{(2.906,1.153)}
\gppoint{gp mark 7}{(2.909,1.153)}
\gppoint{gp mark 7}{(2.913,1.153)}
\gppoint{gp mark 7}{(2.916,1.153)}
\gppoint{gp mark 7}{(2.920,1.153)}
\gppoint{gp mark 7}{(2.923,1.153)}
\gppoint{gp mark 7}{(2.927,1.153)}
\gppoint{gp mark 7}{(2.930,1.153)}
\gppoint{gp mark 7}{(2.933,1.153)}
\gppoint{gp mark 7}{(2.937,1.153)}
\gppoint{gp mark 7}{(2.940,1.153)}
\gppoint{gp mark 7}{(2.944,1.152)}
\gppoint{gp mark 7}{(2.947,1.152)}
\gppoint{gp mark 7}{(2.951,1.152)}
\gppoint{gp mark 7}{(2.954,1.152)}
\gppoint{gp mark 7}{(2.958,1.153)}
\gppoint{gp mark 7}{(2.961,1.153)}
\gppoint{gp mark 7}{(2.965,1.153)}
\gppoint{gp mark 7}{(2.968,1.153)}
\gppoint{gp mark 7}{(2.972,1.153)}
\gppoint{gp mark 7}{(2.975,1.153)}
\gppoint{gp mark 7}{(2.979,1.153)}
\gppoint{gp mark 7}{(2.982,1.153)}
\gppoint{gp mark 7}{(2.986,1.153)}
\gppoint{gp mark 7}{(2.989,1.153)}
\gppoint{gp mark 7}{(2.993,1.153)}
\gppoint{gp mark 7}{(2.996,1.153)}
\gppoint{gp mark 7}{(3.000,1.153)}
\gppoint{gp mark 7}{(3.003,1.153)}
\gppoint{gp mark 7}{(3.006,1.153)}
\gppoint{gp mark 7}{(3.010,1.152)}
\gppoint{gp mark 7}{(3.013,1.153)}
\gppoint{gp mark 7}{(3.017,1.153)}
\gppoint{gp mark 7}{(3.020,1.153)}
\gppoint{gp mark 7}{(3.024,1.153)}
\gppoint{gp mark 7}{(3.027,1.153)}
\gppoint{gp mark 7}{(3.031,1.153)}
\gppoint{gp mark 7}{(3.034,1.153)}
\gppoint{gp mark 7}{(3.038,1.153)}
\gppoint{gp mark 7}{(3.041,1.153)}
\gppoint{gp mark 7}{(3.045,1.153)}
\gppoint{gp mark 7}{(3.048,1.154)}
\gppoint{gp mark 7}{(3.052,1.154)}
\gppoint{gp mark 7}{(3.055,1.154)}
\gppoint{gp mark 7}{(3.059,1.153)}
\gppoint{gp mark 7}{(3.062,1.153)}
\gppoint{gp mark 7}{(3.066,1.153)}
\gppoint{gp mark 7}{(3.069,1.153)}
\gppoint{gp mark 7}{(3.073,1.153)}
\gppoint{gp mark 7}{(3.076,1.153)}
\gppoint{gp mark 7}{(3.079,1.153)}
\gppoint{gp mark 7}{(3.083,1.153)}
\gppoint{gp mark 7}{(3.086,1.153)}
\gppoint{gp mark 7}{(3.090,1.153)}
\gppoint{gp mark 7}{(3.093,1.153)}
\gppoint{gp mark 7}{(3.097,1.153)}
\gppoint{gp mark 7}{(3.100,1.153)}
\gppoint{gp mark 7}{(3.104,1.154)}
\gppoint{gp mark 7}{(3.107,1.154)}
\gppoint{gp mark 7}{(3.111,1.154)}
\gppoint{gp mark 7}{(3.114,1.153)}
\gppoint{gp mark 7}{(3.118,1.153)}
\gppoint{gp mark 7}{(3.121,1.153)}
\gppoint{gp mark 7}{(3.125,1.153)}
\gppoint{gp mark 7}{(3.128,1.154)}
\gppoint{gp mark 7}{(3.132,1.154)}
\gppoint{gp mark 7}{(3.135,1.153)}
\gppoint{gp mark 7}{(3.139,1.153)}
\gppoint{gp mark 7}{(3.142,1.152)}
\gppoint{gp mark 7}{(3.145,1.152)}
\gppoint{gp mark 7}{(3.149,1.152)}
\gppoint{gp mark 7}{(3.152,1.152)}
\gppoint{gp mark 7}{(3.156,1.153)}
\gppoint{gp mark 7}{(3.159,1.153)}
\gppoint{gp mark 7}{(3.163,1.154)}
\gppoint{gp mark 7}{(3.166,1.153)}
\gppoint{gp mark 7}{(3.170,1.153)}
\gppoint{gp mark 7}{(3.173,1.153)}
\gppoint{gp mark 7}{(3.177,1.153)}
\gppoint{gp mark 7}{(3.180,1.153)}
\gppoint{gp mark 7}{(3.184,1.154)}
\gppoint{gp mark 7}{(3.187,1.154)}
\gppoint{gp mark 7}{(3.191,1.154)}
\gppoint{gp mark 7}{(3.194,1.153)}
\gppoint{gp mark 7}{(3.198,1.152)}
\gppoint{gp mark 7}{(3.201,1.152)}
\gppoint{gp mark 7}{(3.205,1.153)}
\gppoint{gp mark 7}{(3.208,1.154)}
\gppoint{gp mark 7}{(3.212,1.157)}
\gppoint{gp mark 7}{(3.215,1.171)}
\gppoint{gp mark 7}{(3.218,1.275)}
\gppoint{gp mark 7}{(3.222,1.333)}
\gppoint{gp mark 7}{(3.225,1.334)}
\gppoint{gp mark 7}{(3.229,1.334)}
\gppoint{gp mark 7}{(3.232,1.337)}
\gppoint{gp mark 7}{(3.236,1.339)}
\gppoint{gp mark 7}{(3.239,1.338)}
\gppoint{gp mark 7}{(3.243,1.336)}
\gppoint{gp mark 7}{(3.246,1.335)}
\gppoint{gp mark 7}{(3.250,1.336)}
\gppoint{gp mark 7}{(3.253,1.337)}
\gppoint{gp mark 7}{(3.257,1.337)}
\gppoint{gp mark 7}{(3.260,1.337)}
\gppoint{gp mark 7}{(3.264,1.336)}
\gppoint{gp mark 7}{(3.267,1.336)}
\gppoint{gp mark 7}{(3.271,1.337)}
\gppoint{gp mark 7}{(3.274,1.338)}
\gppoint{gp mark 7}{(3.278,1.338)}
\gppoint{gp mark 7}{(3.281,1.337)}
\gppoint{gp mark 7}{(3.285,1.336)}
\gppoint{gp mark 7}{(3.288,1.336)}
\gppoint{gp mark 7}{(3.291,1.336)}
\gppoint{gp mark 7}{(3.295,1.336)}
\gppoint{gp mark 7}{(3.298,1.336)}
\gppoint{gp mark 7}{(3.302,1.336)}
\gppoint{gp mark 7}{(3.305,1.336)}
\gppoint{gp mark 7}{(3.309,1.336)}
\gppoint{gp mark 7}{(3.312,1.336)}
\gppoint{gp mark 7}{(3.316,1.336)}
\gppoint{gp mark 7}{(3.319,1.336)}
\gppoint{gp mark 7}{(3.323,1.336)}
\gppoint{gp mark 7}{(3.326,1.335)}
\gppoint{gp mark 7}{(3.330,1.335)}
\gppoint{gp mark 7}{(3.333,1.335)}
\gppoint{gp mark 7}{(3.337,1.336)}
\gppoint{gp mark 7}{(3.340,1.336)}
\gppoint{gp mark 7}{(3.344,1.336)}
\gppoint{gp mark 7}{(3.347,1.336)}
\gppoint{gp mark 7}{(3.351,1.336)}
\gppoint{gp mark 7}{(3.354,1.335)}
\gppoint{gp mark 7}{(3.358,1.335)}
\gppoint{gp mark 7}{(3.361,1.335)}
\gppoint{gp mark 7}{(3.364,1.335)}
\gppoint{gp mark 7}{(3.368,1.335)}
\gppoint{gp mark 7}{(3.371,1.335)}
\gppoint{gp mark 7}{(3.375,1.336)}
\gppoint{gp mark 7}{(3.378,1.336)}
\gppoint{gp mark 7}{(3.382,1.336)}
\gppoint{gp mark 7}{(3.385,1.336)}
\gppoint{gp mark 7}{(3.389,1.336)}
\gppoint{gp mark 7}{(3.392,1.336)}
\gppoint{gp mark 7}{(3.396,1.336)}
\gppoint{gp mark 7}{(3.399,1.336)}
\gppoint{gp mark 7}{(3.403,1.336)}
\gppoint{gp mark 7}{(3.406,1.335)}
\gppoint{gp mark 7}{(3.410,1.335)}
\gppoint{gp mark 7}{(3.413,1.335)}
\gppoint{gp mark 7}{(3.417,1.335)}
\gppoint{gp mark 7}{(3.420,1.336)}
\gppoint{gp mark 7}{(3.424,1.336)}
\gppoint{gp mark 7}{(3.427,1.336)}
\gppoint{gp mark 7}{(3.431,1.336)}
\gppoint{gp mark 7}{(3.434,1.336)}
\gppoint{gp mark 7}{(3.437,1.336)}
\gppoint{gp mark 7}{(3.441,1.335)}
\gppoint{gp mark 7}{(3.444,1.335)}
\gppoint{gp mark 7}{(3.448,1.335)}
\gppoint{gp mark 7}{(3.451,1.335)}
\gppoint{gp mark 7}{(3.455,1.335)}
\gppoint{gp mark 7}{(3.458,1.335)}
\gppoint{gp mark 7}{(3.462,1.335)}
\gppoint{gp mark 7}{(3.465,1.335)}
\gppoint{gp mark 7}{(3.469,1.335)}
\gppoint{gp mark 7}{(3.472,1.335)}
\gppoint{gp mark 7}{(3.476,1.335)}
\gppoint{gp mark 7}{(3.479,1.335)}
\gppoint{gp mark 7}{(3.483,1.335)}
\gppoint{gp mark 7}{(3.486,1.335)}
\gppoint{gp mark 7}{(3.490,1.335)}
\gppoint{gp mark 7}{(3.493,1.335)}
\gppoint{gp mark 7}{(3.497,1.335)}
\gppoint{gp mark 7}{(3.500,1.335)}
\gppoint{gp mark 7}{(3.504,1.335)}
\gppoint{gp mark 7}{(3.507,1.335)}
\gppoint{gp mark 7}{(3.510,1.335)}
\gppoint{gp mark 7}{(3.514,1.335)}
\gppoint{gp mark 7}{(3.517,1.335)}
\gppoint{gp mark 7}{(3.521,1.335)}
\gppoint{gp mark 7}{(3.524,1.335)}
\gppoint{gp mark 7}{(3.528,1.335)}
\gppoint{gp mark 7}{(3.531,1.335)}
\gppoint{gp mark 7}{(3.535,1.335)}
\gppoint{gp mark 7}{(3.538,1.335)}
\gppoint{gp mark 7}{(3.542,1.335)}
\gppoint{gp mark 7}{(3.545,1.335)}
\gppoint{gp mark 7}{(3.549,1.335)}
\gppoint{gp mark 7}{(3.552,1.335)}
\gppoint{gp mark 7}{(3.556,1.335)}
\gppoint{gp mark 7}{(3.559,1.335)}
\gppoint{gp mark 7}{(3.563,1.335)}
\gppoint{gp mark 7}{(3.566,1.335)}
\gppoint{gp mark 7}{(3.570,1.335)}
\gppoint{gp mark 7}{(3.573,1.335)}
\gppoint{gp mark 7}{(3.577,1.335)}
\gppoint{gp mark 7}{(3.580,1.335)}
\gppoint{gp mark 7}{(3.583,1.335)}
\gppoint{gp mark 7}{(3.587,1.335)}
\gppoint{gp mark 7}{(3.590,1.335)}
\gppoint{gp mark 7}{(3.594,1.335)}
\gppoint{gp mark 7}{(3.597,1.335)}
\gppoint{gp mark 7}{(3.601,1.335)}
\gppoint{gp mark 7}{(3.604,1.335)}
\gppoint{gp mark 7}{(3.608,1.335)}
\gppoint{gp mark 7}{(3.611,1.335)}
\gppoint{gp mark 7}{(3.615,1.335)}
\gppoint{gp mark 7}{(3.618,1.336)}
\gppoint{gp mark 7}{(3.622,1.336)}
\gppoint{gp mark 7}{(3.625,1.336)}
\gppoint{gp mark 7}{(3.629,1.336)}
\gppoint{gp mark 7}{(3.632,1.335)}
\gppoint{gp mark 7}{(3.636,1.335)}
\gppoint{gp mark 7}{(3.639,1.335)}
\gppoint{gp mark 7}{(3.643,1.335)}
\gppoint{gp mark 7}{(3.646,1.335)}
\gppoint{gp mark 7}{(3.650,1.335)}
\gppoint{gp mark 7}{(3.653,1.335)}
\gppoint{gp mark 7}{(3.656,1.335)}
\gppoint{gp mark 7}{(3.660,1.335)}
\gppoint{gp mark 7}{(3.663,1.335)}
\gppoint{gp mark 7}{(3.667,1.335)}
\gppoint{gp mark 7}{(3.670,1.335)}
\gppoint{gp mark 7}{(3.674,1.335)}
\gppoint{gp mark 7}{(3.677,1.335)}
\gppoint{gp mark 7}{(3.681,1.335)}
\gppoint{gp mark 7}{(3.684,1.336)}
\gppoint{gp mark 7}{(3.688,1.336)}
\gppoint{gp mark 7}{(3.691,1.335)}
\gppoint{gp mark 7}{(3.695,1.335)}
\gppoint{gp mark 7}{(3.698,1.335)}
\gppoint{gp mark 7}{(3.702,1.335)}
\gppoint{gp mark 7}{(3.705,1.335)}
\gppoint{gp mark 7}{(3.709,1.334)}
\gppoint{gp mark 7}{(3.712,1.334)}
\gppoint{gp mark 7}{(3.716,1.334)}
\gppoint{gp mark 7}{(3.719,1.334)}
\gppoint{gp mark 7}{(3.723,1.335)}
\gppoint{gp mark 7}{(3.726,1.335)}
\gppoint{gp mark 7}{(3.729,1.335)}
\gppoint{gp mark 7}{(3.733,1.335)}
\gppoint{gp mark 7}{(3.736,1.335)}
\gppoint{gp mark 7}{(3.740,1.335)}
\gppoint{gp mark 7}{(3.743,1.335)}
\gppoint{gp mark 7}{(3.747,1.335)}
\gppoint{gp mark 7}{(3.750,1.335)}
\gppoint{gp mark 7}{(3.754,1.335)}
\gppoint{gp mark 7}{(3.757,1.335)}
\gppoint{gp mark 7}{(3.761,1.335)}
\gppoint{gp mark 7}{(3.764,1.335)}
\gppoint{gp mark 7}{(3.768,1.335)}
\gppoint{gp mark 7}{(3.771,1.335)}
\gppoint{gp mark 7}{(3.775,1.335)}
\gppoint{gp mark 7}{(3.778,1.335)}
\gppoint{gp mark 7}{(3.782,1.335)}
\gppoint{gp mark 7}{(3.785,1.335)}
\gppoint{gp mark 7}{(3.789,1.335)}
\gppoint{gp mark 7}{(3.792,1.335)}
\gppoint{gp mark 7}{(3.796,1.335)}
\gppoint{gp mark 7}{(3.799,1.335)}
\gppoint{gp mark 7}{(3.802,1.335)}
\gppoint{gp mark 7}{(3.806,1.335)}
\gppoint{gp mark 7}{(3.809,1.335)}
\gppoint{gp mark 7}{(3.813,1.334)}
\gppoint{gp mark 7}{(3.816,1.334)}
\gppoint{gp mark 7}{(3.820,1.334)}
\gppoint{gp mark 7}{(3.823,1.335)}
\gppoint{gp mark 7}{(3.827,1.335)}
\gppoint{gp mark 7}{(3.830,1.335)}
\gppoint{gp mark 7}{(3.834,1.335)}
\gppoint{gp mark 7}{(3.837,1.335)}
\gppoint{gp mark 7}{(3.841,1.335)}
\gppoint{gp mark 7}{(3.844,1.335)}
\gppoint{gp mark 7}{(3.848,1.335)}
\gppoint{gp mark 7}{(3.851,1.335)}
\gppoint{gp mark 7}{(3.855,1.335)}
\gppoint{gp mark 7}{(3.858,1.335)}
\gppoint{gp mark 7}{(3.862,1.335)}
\gppoint{gp mark 7}{(3.865,1.335)}
\gppoint{gp mark 7}{(3.869,1.335)}
\gppoint{gp mark 7}{(3.872,1.335)}
\gppoint{gp mark 7}{(3.875,1.335)}
\gppoint{gp mark 7}{(3.879,1.335)}
\gppoint{gp mark 7}{(3.882,1.335)}
\gppoint{gp mark 7}{(3.886,1.335)}
\gppoint{gp mark 7}{(3.889,1.335)}
\gppoint{gp mark 7}{(3.893,1.335)}
\gppoint{gp mark 7}{(3.896,1.336)}
\gppoint{gp mark 7}{(3.900,1.336)}
\gppoint{gp mark 7}{(3.903,1.336)}
\gppoint{gp mark 7}{(3.907,1.336)}
\gppoint{gp mark 7}{(3.910,1.336)}
\gppoint{gp mark 7}{(3.914,1.336)}
\gppoint{gp mark 7}{(3.917,1.336)}
\gppoint{gp mark 7}{(3.921,1.335)}
\gppoint{gp mark 7}{(3.924,1.336)}
\gppoint{gp mark 7}{(3.928,1.336)}
\gppoint{gp mark 7}{(3.931,1.336)}
\gppoint{gp mark 7}{(3.935,1.336)}
\gppoint{gp mark 7}{(3.938,1.336)}
\gppoint{gp mark 7}{(3.942,1.336)}
\gppoint{gp mark 7}{(3.945,1.336)}
\gppoint{gp mark 7}{(3.948,1.336)}
\gppoint{gp mark 7}{(3.952,1.335)}
\gppoint{gp mark 7}{(3.955,1.335)}
\gppoint{gp mark 7}{(3.959,1.335)}
\gppoint{gp mark 7}{(3.962,1.335)}
\gppoint{gp mark 7}{(3.966,1.335)}
\gppoint{gp mark 7}{(3.969,1.335)}
\gppoint{gp mark 7}{(3.973,1.335)}
\gppoint{gp mark 7}{(3.976,1.336)}
\gppoint{gp mark 7}{(3.980,1.336)}
\gppoint{gp mark 7}{(3.983,1.336)}
\gppoint{gp mark 7}{(3.987,1.336)}
\gppoint{gp mark 7}{(3.990,1.335)}
\gppoint{gp mark 7}{(3.994,1.335)}
\gppoint{gp mark 7}{(3.997,1.335)}
\gppoint{gp mark 7}{(4.001,1.335)}
\gppoint{gp mark 7}{(4.004,1.336)}
\gppoint{gp mark 7}{(4.008,1.336)}
\gppoint{gp mark 7}{(4.011,1.336)}
\gppoint{gp mark 7}{(4.015,1.336)}
\gppoint{gp mark 7}{(4.018,1.336)}
\gppoint{gp mark 7}{(4.021,1.335)}
\gppoint{gp mark 7}{(4.025,1.335)}
\gppoint{gp mark 7}{(4.028,1.335)}
\gppoint{gp mark 7}{(4.032,1.335)}
\gppoint{gp mark 7}{(4.035,1.336)}
\gppoint{gp mark 7}{(4.039,1.336)}
\gppoint{gp mark 7}{(4.042,1.336)}
\gppoint{gp mark 7}{(4.046,1.336)}
\gppoint{gp mark 7}{(4.049,1.336)}
\gppoint{gp mark 7}{(4.053,1.336)}
\gppoint{gp mark 7}{(4.056,1.335)}
\gppoint{gp mark 7}{(4.060,1.335)}
\gppoint{gp mark 7}{(4.063,1.335)}
\gppoint{gp mark 7}{(4.067,1.335)}
\gppoint{gp mark 7}{(4.070,1.335)}
\gppoint{gp mark 7}{(4.074,1.335)}
\gppoint{gp mark 7}{(4.077,1.335)}
\gppoint{gp mark 7}{(4.081,1.335)}
\gppoint{gp mark 7}{(4.084,1.335)}
\gppoint{gp mark 7}{(4.088,1.335)}
\gppoint{gp mark 7}{(4.091,1.335)}
\gppoint{gp mark 7}{(4.094,1.335)}
\gppoint{gp mark 7}{(4.098,1.335)}
\gppoint{gp mark 7}{(4.101,1.335)}
\gppoint{gp mark 7}{(4.105,1.335)}
\gppoint{gp mark 7}{(4.108,1.335)}
\gppoint{gp mark 7}{(4.112,1.335)}
\gppoint{gp mark 7}{(4.115,1.335)}
\gppoint{gp mark 7}{(4.119,1.335)}
\gppoint{gp mark 7}{(4.122,1.335)}
\gppoint{gp mark 7}{(4.126,1.336)}
\gppoint{gp mark 7}{(4.129,1.336)}
\gppoint{gp mark 7}{(4.133,1.336)}
\gppoint{gp mark 7}{(4.136,1.336)}
\gppoint{gp mark 7}{(4.140,1.336)}
\gppoint{gp mark 7}{(4.143,1.335)}
\gppoint{gp mark 7}{(4.147,1.335)}
\gppoint{gp mark 7}{(4.150,1.335)}
\gppoint{gp mark 7}{(4.154,1.335)}
\gppoint{gp mark 7}{(4.157,1.335)}
\gppoint{gp mark 7}{(4.161,1.335)}
\gppoint{gp mark 7}{(4.164,1.335)}
\gppoint{gp mark 7}{(4.167,1.336)}
\gppoint{gp mark 7}{(4.171,1.336)}
\gppoint{gp mark 7}{(4.174,1.336)}
\gppoint{gp mark 7}{(4.178,1.336)}
\gppoint{gp mark 7}{(4.181,1.336)}
\gppoint{gp mark 7}{(4.185,1.336)}
\gppoint{gp mark 7}{(4.188,1.336)}
\gppoint{gp mark 7}{(4.192,1.336)}
\gppoint{gp mark 7}{(4.195,1.335)}
\gppoint{gp mark 7}{(4.199,1.335)}
\gppoint{gp mark 7}{(4.202,1.335)}
\gppoint{gp mark 7}{(4.206,1.335)}
\gppoint{gp mark 7}{(4.209,1.335)}
\gppoint{gp mark 7}{(4.213,1.335)}
\gppoint{gp mark 7}{(4.216,1.335)}
\gppoint{gp mark 7}{(4.220,1.335)}
\gppoint{gp mark 7}{(4.223,1.335)}
\gppoint{gp mark 7}{(4.227,1.335)}
\gppoint{gp mark 7}{(4.230,1.335)}
\gppoint{gp mark 7}{(4.234,1.335)}
\gppoint{gp mark 7}{(4.237,1.335)}
\gppoint{gp mark 7}{(4.240,1.335)}
\gppoint{gp mark 7}{(4.244,1.335)}
\gppoint{gp mark 7}{(4.247,1.335)}
\gppoint{gp mark 7}{(4.251,1.335)}
\gppoint{gp mark 7}{(4.254,1.335)}
\gppoint{gp mark 7}{(4.258,1.335)}
\gppoint{gp mark 7}{(4.261,1.335)}
\gppoint{gp mark 7}{(4.265,1.335)}
\gppoint{gp mark 7}{(4.268,1.335)}
\gppoint{gp mark 7}{(4.272,1.335)}
\gppoint{gp mark 7}{(4.275,1.335)}
\gppoint{gp mark 7}{(4.279,1.335)}
\gppoint{gp mark 7}{(4.282,1.335)}
\gppoint{gp mark 7}{(4.286,1.335)}
\gppoint{gp mark 7}{(4.289,1.335)}
\gppoint{gp mark 7}{(4.293,1.334)}
\gppoint{gp mark 7}{(4.296,1.334)}
\gppoint{gp mark 7}{(4.300,1.335)}
\gppoint{gp mark 7}{(4.303,1.336)}
\gppoint{gp mark 7}{(4.307,1.336)}
\gppoint{gp mark 7}{(4.310,1.334)}
\gppoint{gp mark 7}{(4.313,1.334)}
\gppoint{gp mark 7}{(4.317,1.335)}
\gppoint{gp mark 7}{(4.320,1.337)}
\gppoint{gp mark 7}{(4.324,1.336)}
\gppoint{gp mark 7}{(4.327,1.316)}
\gppoint{gp mark 7}{(4.331,1.204)}
\gppoint{gp mark 7}{(4.334,1.130)}
\gppoint{gp mark 7}{(4.338,1.127)}
\gppoint{gp mark 7}{(4.341,1.129)}
\gppoint{gp mark 7}{(4.345,1.130)}
\gppoint{gp mark 7}{(4.348,1.129)}
\gppoint{gp mark 7}{(4.352,1.128)}
\gppoint{gp mark 7}{(4.355,1.127)}
\gppoint{gp mark 7}{(4.359,1.126)}
\gppoint{gp mark 7}{(4.362,1.126)}
\gppoint{gp mark 7}{(4.366,1.127)}
\gppoint{gp mark 7}{(4.369,1.128)}
\gppoint{gp mark 7}{(4.373,1.129)}
\gppoint{gp mark 7}{(4.376,1.129)}
\gppoint{gp mark 7}{(4.380,1.129)}
\gppoint{gp mark 7}{(4.383,1.128)}
\gppoint{gp mark 7}{(4.386,1.127)}
\gppoint{gp mark 7}{(4.390,1.127)}
\gppoint{gp mark 7}{(4.393,1.127)}
\gppoint{gp mark 7}{(4.397,1.128)}
\gppoint{gp mark 7}{(4.400,1.128)}
\gppoint{gp mark 7}{(4.404,1.128)}
\gppoint{gp mark 7}{(4.407,1.128)}
\gppoint{gp mark 7}{(4.411,1.128)}
\gppoint{gp mark 7}{(4.414,1.127)}
\gppoint{gp mark 7}{(4.418,1.127)}
\gppoint{gp mark 7}{(4.421,1.127)}
\gppoint{gp mark 7}{(4.425,1.128)}
\gppoint{gp mark 7}{(4.428,1.128)}
\gppoint{gp mark 7}{(4.432,1.128)}
\gppoint{gp mark 7}{(4.435,1.128)}
\gppoint{gp mark 7}{(4.439,1.128)}
\gppoint{gp mark 7}{(4.442,1.127)}
\gppoint{gp mark 7}{(4.446,1.127)}
\gppoint{gp mark 7}{(4.449,1.127)}
\gppoint{gp mark 7}{(4.453,1.128)}
\gppoint{gp mark 7}{(4.456,1.128)}
\gppoint{gp mark 7}{(4.459,1.128)}
\gppoint{gp mark 7}{(4.463,1.128)}
\gppoint{gp mark 7}{(4.466,1.128)}
\gppoint{gp mark 7}{(4.470,1.128)}
\gppoint{gp mark 7}{(4.473,1.128)}
\gppoint{gp mark 7}{(4.477,1.128)}
\gppoint{gp mark 7}{(4.480,1.128)}
\gppoint{gp mark 7}{(4.484,1.128)}
\gppoint{gp mark 7}{(4.487,1.128)}
\gppoint{gp mark 7}{(4.491,1.128)}
\gppoint{gp mark 7}{(4.494,1.128)}
\gppoint{gp mark 7}{(4.498,1.128)}
\gppoint{gp mark 7}{(4.501,1.128)}
\gppoint{gp mark 7}{(4.505,1.128)}
\gppoint{gp mark 7}{(4.508,1.128)}
\gppoint{gp mark 7}{(4.512,1.128)}
\gppoint{gp mark 7}{(4.515,1.128)}
\gppoint{gp mark 7}{(4.519,1.128)}
\gppoint{gp mark 7}{(4.522,1.128)}
\gppoint{gp mark 7}{(4.526,1.127)}
\gppoint{gp mark 7}{(4.529,1.127)}
\gppoint{gp mark 7}{(4.532,1.127)}
\gppoint{gp mark 7}{(4.536,1.128)}
\gppoint{gp mark 7}{(4.539,1.128)}
\gppoint{gp mark 7}{(4.543,1.128)}
\gppoint{gp mark 7}{(4.546,1.128)}
\gppoint{gp mark 7}{(4.550,1.128)}
\gppoint{gp mark 7}{(4.553,1.128)}
\gppoint{gp mark 7}{(4.557,1.128)}
\gppoint{gp mark 7}{(4.560,1.128)}
\gppoint{gp mark 7}{(4.564,1.128)}
\gppoint{gp mark 7}{(4.567,1.128)}
\gppoint{gp mark 7}{(4.571,1.128)}
\gppoint{gp mark 7}{(4.574,1.128)}
\gppoint{gp mark 7}{(4.578,1.128)}
\gppoint{gp mark 7}{(4.581,1.127)}
\gppoint{gp mark 7}{(4.585,1.127)}
\gppoint{gp mark 7}{(4.588,1.128)}
\gppoint{gp mark 7}{(4.592,1.128)}
\gppoint{gp mark 7}{(4.595,1.128)}
\gppoint{gp mark 7}{(4.598,1.128)}
\gppoint{gp mark 7}{(4.602,1.128)}
\gppoint{gp mark 7}{(4.605,1.128)}
\gppoint{gp mark 7}{(4.609,1.128)}
\gppoint{gp mark 7}{(4.612,1.128)}
\gppoint{gp mark 7}{(4.616,1.128)}
\gppoint{gp mark 7}{(4.619,1.128)}
\gppoint{gp mark 7}{(4.623,1.128)}
\gppoint{gp mark 7}{(4.626,1.128)}
\gppoint{gp mark 7}{(4.630,1.128)}
\gppoint{gp mark 7}{(4.633,1.128)}
\gppoint{gp mark 7}{(4.637,1.128)}
\gppoint{gp mark 7}{(4.640,1.128)}
\gppoint{gp mark 7}{(4.644,1.128)}
\gppoint{gp mark 7}{(4.647,1.128)}
\gppoint{gp mark 7}{(4.651,1.128)}
\gppoint{gp mark 7}{(4.654,1.128)}
\gppoint{gp mark 7}{(4.658,1.128)}
\gppoint{gp mark 7}{(4.661,1.128)}
\gppoint{gp mark 7}{(4.665,1.128)}
\gppoint{gp mark 7}{(4.668,1.128)}
\gppoint{gp mark 7}{(4.671,1.127)}
\gppoint{gp mark 7}{(4.675,1.128)}
\gppoint{gp mark 7}{(4.678,1.128)}
\gppoint{gp mark 7}{(4.682,1.128)}
\gppoint{gp mark 7}{(4.685,1.128)}
\gppoint{gp mark 7}{(4.689,1.128)}
\gppoint{gp mark 7}{(4.692,1.128)}
\gppoint{gp mark 7}{(4.696,1.128)}
\gppoint{gp mark 7}{(4.699,1.128)}
\gppoint{gp mark 7}{(4.703,1.128)}
\gppoint{gp mark 7}{(4.706,1.128)}
\gppoint{gp mark 7}{(4.710,1.128)}
\gppoint{gp mark 7}{(4.713,1.128)}
\gppoint{gp mark 7}{(4.717,1.128)}
\gppoint{gp mark 7}{(4.720,1.128)}
\gppoint{gp mark 7}{(4.724,1.128)}
\gppoint{gp mark 7}{(4.727,1.128)}
\gppoint{gp mark 7}{(4.731,1.128)}
\gppoint{gp mark 7}{(4.734,1.128)}
\gppoint{gp mark 7}{(4.738,1.128)}
\gppoint{gp mark 7}{(4.741,1.128)}
\gppoint{gp mark 7}{(4.744,1.128)}
\gppoint{gp mark 7}{(4.748,1.127)}
\gppoint{gp mark 7}{(4.751,1.127)}
\gppoint{gp mark 7}{(4.755,1.127)}
\gppoint{gp mark 7}{(4.758,1.127)}
\gppoint{gp mark 7}{(4.762,1.127)}
\gppoint{gp mark 7}{(4.765,1.128)}
\gppoint{gp mark 7}{(4.769,1.128)}
\gppoint{gp mark 7}{(4.772,1.128)}
\gppoint{gp mark 7}{(4.776,1.128)}
\gppoint{gp mark 7}{(4.779,1.128)}
\gppoint{gp mark 7}{(4.783,1.128)}
\gppoint{gp mark 7}{(4.786,1.127)}
\gppoint{gp mark 7}{(4.790,1.127)}
\gppoint{gp mark 7}{(4.793,1.127)}
\gppoint{gp mark 7}{(4.797,1.127)}
\gppoint{gp mark 7}{(4.800,1.127)}
\gppoint{gp mark 7}{(4.804,1.127)}
\gppoint{gp mark 7}{(4.807,1.127)}
\gppoint{gp mark 7}{(4.811,1.127)}
\gppoint{gp mark 7}{(4.814,1.127)}
\gppoint{gp mark 7}{(4.817,1.128)}
\gppoint{gp mark 7}{(4.821,1.128)}
\gppoint{gp mark 7}{(4.824,1.128)}
\gppoint{gp mark 7}{(4.828,1.128)}
\gppoint{gp mark 7}{(4.831,1.128)}
\gppoint{gp mark 7}{(4.835,1.128)}
\gppoint{gp mark 7}{(4.838,1.128)}
\gppoint{gp mark 7}{(4.842,1.128)}
\gppoint{gp mark 7}{(4.845,1.128)}
\gppoint{gp mark 7}{(4.849,1.128)}
\gppoint{gp mark 7}{(4.852,1.128)}
\gppoint{gp mark 7}{(4.856,1.128)}
\gppoint{gp mark 7}{(4.859,1.128)}
\gppoint{gp mark 7}{(4.863,1.128)}
\gppoint{gp mark 7}{(4.866,1.127)}
\gppoint{gp mark 7}{(4.870,1.127)}
\gppoint{gp mark 7}{(4.873,1.127)}
\gppoint{gp mark 7}{(4.877,1.127)}
\gppoint{gp mark 7}{(4.880,1.127)}
\gppoint{gp mark 7}{(4.884,1.127)}
\gppoint{gp mark 7}{(4.887,1.127)}
\gppoint{gp mark 7}{(4.890,1.128)}
\gppoint{gp mark 7}{(4.894,1.128)}
\gppoint{gp mark 7}{(4.897,1.128)}
\gppoint{gp mark 7}{(4.901,1.128)}
\gppoint{gp mark 7}{(4.904,1.128)}
\gppoint{gp mark 7}{(4.908,1.128)}
\gppoint{gp mark 7}{(4.911,1.128)}
\gppoint{gp mark 7}{(4.915,1.127)}
\gppoint{gp mark 7}{(4.918,1.127)}
\gppoint{gp mark 7}{(4.922,1.127)}
\gppoint{gp mark 7}{(4.925,1.127)}
\gppoint{gp mark 7}{(4.929,1.127)}
\gppoint{gp mark 7}{(4.932,1.127)}
\gppoint{gp mark 7}{(4.936,1.127)}
\gppoint{gp mark 7}{(4.939,1.128)}
\gppoint{gp mark 7}{(4.943,1.128)}
\gppoint{gp mark 7}{(4.946,1.128)}
\gppoint{gp mark 7}{(4.950,1.128)}
\gppoint{gp mark 7}{(4.953,1.128)}
\gppoint{gp mark 7}{(4.957,1.128)}
\gppoint{gp mark 7}{(4.960,1.128)}
\gppoint{gp mark 7}{(4.963,1.128)}
\gppoint{gp mark 7}{(4.967,1.127)}
\gppoint{gp mark 7}{(4.970,1.127)}
\gppoint{gp mark 7}{(4.974,1.127)}
\gppoint{gp mark 7}{(4.977,1.127)}
\gppoint{gp mark 7}{(4.981,1.127)}
\gppoint{gp mark 7}{(4.984,1.127)}
\gppoint{gp mark 7}{(4.988,1.128)}
\gppoint{gp mark 7}{(4.991,1.128)}
\gppoint{gp mark 7}{(4.995,1.128)}
\gppoint{gp mark 7}{(4.998,1.128)}
\gppoint{gp mark 7}{(5.002,1.128)}
\gppoint{gp mark 7}{(5.005,1.128)}
\gppoint{gp mark 7}{(5.009,1.128)}
\gppoint{gp mark 7}{(5.012,1.128)}
\gppoint{gp mark 7}{(5.016,1.128)}
\gppoint{gp mark 7}{(5.019,1.128)}
\gppoint{gp mark 7}{(5.023,1.128)}
\gppoint{gp mark 7}{(5.026,1.128)}
\gppoint{gp mark 7}{(5.030,1.128)}
\gppoint{gp mark 7}{(5.033,1.128)}
\gppoint{gp mark 7}{(5.036,1.127)}
\gppoint{gp mark 7}{(5.040,1.127)}
\gppoint{gp mark 7}{(5.043,1.127)}
\gppoint{gp mark 7}{(5.047,1.127)}
\gppoint{gp mark 7}{(5.050,1.127)}
\gppoint{gp mark 7}{(5.054,1.127)}
\gppoint{gp mark 7}{(5.057,1.127)}
\gppoint{gp mark 7}{(5.061,1.127)}
\gppoint{gp mark 7}{(5.064,1.127)}
\gppoint{gp mark 7}{(5.068,1.128)}
\gppoint{gp mark 7}{(5.071,1.128)}
\gppoint{gp mark 7}{(5.075,1.128)}
\gppoint{gp mark 7}{(5.078,1.128)}
\gppoint{gp mark 7}{(5.082,1.128)}
\gppoint{gp mark 7}{(5.085,1.127)}
\gppoint{gp mark 7}{(5.089,1.127)}
\gppoint{gp mark 7}{(5.092,1.127)}
\gppoint{gp mark 7}{(5.096,1.127)}
\gppoint{gp mark 7}{(5.099,1.127)}
\gppoint{gp mark 7}{(5.103,1.127)}
\gppoint{gp mark 7}{(5.106,1.127)}
\gppoint{gp mark 7}{(5.109,1.127)}
\gppoint{gp mark 7}{(5.113,1.127)}
\gppoint{gp mark 7}{(5.116,1.128)}
\gppoint{gp mark 7}{(5.120,1.128)}
\gppoint{gp mark 7}{(5.123,1.128)}
\gppoint{gp mark 7}{(5.127,1.128)}
\gppoint{gp mark 7}{(5.130,1.128)}
\gppoint{gp mark 7}{(5.134,1.128)}
\gppoint{gp mark 7}{(5.137,1.128)}
\gppoint{gp mark 7}{(5.141,1.128)}
\gppoint{gp mark 7}{(5.144,1.127)}
\gppoint{gp mark 7}{(5.148,1.127)}
\gppoint{gp mark 7}{(5.151,1.127)}
\gppoint{gp mark 7}{(5.155,1.127)}
\gppoint{gp mark 7}{(5.158,1.127)}
\gppoint{gp mark 7}{(5.162,1.128)}
\gppoint{gp mark 7}{(5.165,1.128)}
\gppoint{gp mark 7}{(5.169,1.128)}
\gppoint{gp mark 7}{(5.172,1.128)}
\gppoint{gp mark 7}{(5.176,1.128)}
\gppoint{gp mark 7}{(5.179,1.128)}
\gppoint{gp mark 7}{(5.182,1.128)}
\gppoint{gp mark 7}{(5.186,1.127)}
\gppoint{gp mark 7}{(5.189,1.127)}
\gppoint{gp mark 7}{(5.193,1.127)}
\gppoint{gp mark 7}{(5.196,1.127)}
\gppoint{gp mark 7}{(5.200,1.128)}
\gppoint{gp mark 7}{(5.203,1.128)}
\gppoint{gp mark 7}{(5.207,1.128)}
\gppoint{gp mark 7}{(5.210,1.128)}
\gppoint{gp mark 7}{(5.214,1.128)}
\gppoint{gp mark 7}{(5.217,1.128)}
\gppoint{gp mark 7}{(5.221,1.128)}
\gppoint{gp mark 7}{(5.224,1.127)}
\gppoint{gp mark 7}{(5.228,1.127)}
\gppoint{gp mark 7}{(5.231,1.127)}
\gppoint{gp mark 7}{(5.235,1.127)}
\gppoint{gp mark 7}{(5.238,1.127)}
\gppoint{gp mark 7}{(5.242,1.127)}
\gppoint{gp mark 7}{(5.245,1.127)}
\gppoint{gp mark 7}{(5.249,1.128)}
\gppoint{gp mark 7}{(5.252,1.128)}
\gppoint{gp mark 7}{(5.255,1.128)}
\gppoint{gp mark 7}{(5.259,1.128)}
\gppoint{gp mark 7}{(5.262,1.128)}
\gppoint{gp mark 7}{(5.266,1.127)}
\gppoint{gp mark 7}{(5.269,1.127)}
\gppoint{gp mark 7}{(5.273,1.127)}
\gppoint{gp mark 7}{(5.276,1.127)}
\gppoint{gp mark 7}{(5.280,1.127)}
\gppoint{gp mark 7}{(5.283,1.127)}
\gppoint{gp mark 7}{(5.287,1.127)}
\gppoint{gp mark 7}{(5.290,1.128)}
\gppoint{gp mark 7}{(5.294,1.128)}
\gppoint{gp mark 7}{(5.297,1.128)}
\gppoint{gp mark 7}{(5.301,1.128)}
\gppoint{gp mark 7}{(5.304,1.128)}
\gppoint{gp mark 7}{(5.308,1.127)}
\gppoint{gp mark 7}{(5.311,1.127)}
\gppoint{gp mark 7}{(5.315,1.127)}
\gppoint{gp mark 7}{(5.318,1.127)}
\gppoint{gp mark 7}{(5.322,1.127)}
\gppoint{gp mark 7}{(5.325,1.127)}
\gppoint{gp mark 7}{(5.328,1.127)}
\gppoint{gp mark 7}{(5.332,1.127)}
\gppoint{gp mark 7}{(5.335,1.127)}
\gppoint{gp mark 7}{(5.339,1.128)}
\gppoint{gp mark 7}{(5.342,1.128)}
\gppoint{gp mark 7}{(5.346,1.127)}
\gppoint{gp mark 7}{(5.349,1.127)}
\gppoint{gp mark 7}{(5.353,1.127)}
\gppoint{gp mark 7}{(5.356,1.127)}
\gppoint{gp mark 7}{(5.360,1.127)}
\gppoint{gp mark 7}{(5.363,1.127)}
\gppoint{gp mark 7}{(5.367,1.127)}
\gppoint{gp mark 7}{(5.370,1.127)}
\gppoint{gp mark 7}{(5.374,1.127)}
\gppoint{gp mark 7}{(5.377,1.127)}
\gppoint{gp mark 7}{(5.381,1.128)}
\gppoint{gp mark 7}{(5.384,1.128)}
\gppoint{gp mark 7}{(5.388,1.128)}
\gppoint{gp mark 7}{(5.391,1.128)}
\gppoint{gp mark 7}{(5.395,1.128)}
\gppoint{gp mark 7}{(5.398,1.127)}
\gppoint{gp mark 7}{(5.401,1.127)}
\gppoint{gp mark 7}{(5.405,1.127)}
\gppoint{gp mark 7}{(5.408,1.127)}
\gppoint{gp mark 7}{(5.412,1.127)}
\gppoint{gp mark 7}{(5.415,1.127)}
\gppoint{gp mark 7}{(5.419,1.127)}
\gppoint{gp mark 7}{(5.422,1.127)}
\gppoint{gp mark 7}{(5.426,1.128)}
\gppoint{gp mark 7}{(5.429,1.128)}
\gppoint{gp mark 7}{(5.433,1.127)}
\gppoint{gp mark 7}{(5.436,1.127)}
\gppoint{gp mark 7}{(5.440,1.127)}
\gppoint{gp mark 7}{(5.443,1.127)}
\gppoint{gp mark 7}{(5.447,1.127)}
\gppoint{gp mark 7}{(5.450,1.127)}
\gppoint{gp mark 7}{(5.454,1.127)}
\gppoint{gp mark 7}{(5.457,1.127)}
\gppoint{gp mark 7}{(5.461,1.128)}
\gppoint{gp mark 7}{(5.464,1.128)}
\gppoint{gp mark 7}{(5.468,1.128)}
\gppoint{gp mark 7}{(5.471,1.128)}
\gppoint{gp mark 7}{(5.474,1.128)}
\gppoint{gp mark 7}{(5.478,1.128)}
\gppoint{gp mark 7}{(5.481,1.127)}
\gppoint{gp mark 7}{(5.485,1.127)}
\gppoint{gp mark 7}{(5.488,1.127)}
\gppoint{gp mark 7}{(5.492,1.127)}
\gppoint{gp mark 7}{(5.495,1.127)}
\gppoint{gp mark 7}{(5.499,1.127)}
\gppoint{gp mark 7}{(5.502,1.127)}
\gppoint{gp mark 7}{(5.506,1.127)}
\gppoint{gp mark 7}{(5.509,1.128)}
\gppoint{gp mark 7}{(5.513,1.128)}
\gppoint{gp mark 7}{(5.516,1.128)}
\gppoint{gp mark 7}{(5.520,1.128)}
\gppoint{gp mark 7}{(5.523,1.128)}
\gppoint{gp mark 7}{(5.527,1.127)}
\gppoint{gp mark 7}{(5.530,1.127)}
\gppoint{gp mark 7}{(5.534,1.127)}
\gppoint{gp mark 7}{(5.537,1.127)}
\gppoint{gp mark 7}{(5.541,1.127)}
\gppoint{gp mark 7}{(5.544,1.127)}
\gppoint{gp mark 7}{(5.547,1.127)}
\gppoint{gp mark 7}{(5.551,1.128)}
\gppoint{gp mark 7}{(5.554,1.128)}
\gppoint{gp mark 7}{(5.558,1.128)}
\gppoint{gp mark 7}{(5.561,1.128)}
\gppoint{gp mark 7}{(5.565,1.128)}
\gppoint{gp mark 7}{(5.568,1.127)}
\gppoint{gp mark 7}{(5.572,1.127)}
\gppoint{gp mark 7}{(5.575,1.127)}
\gppoint{gp mark 7}{(5.579,1.127)}
\gppoint{gp mark 7}{(5.582,1.127)}
\gppoint{gp mark 7}{(5.586,1.127)}
\gppoint{gp mark 7}{(5.589,1.127)}
\gppoint{gp mark 7}{(5.593,1.127)}
\gppoint{gp mark 7}{(5.596,1.127)}
\gppoint{gp mark 7}{(5.600,1.127)}
\gppoint{gp mark 7}{(5.603,1.127)}
\gppoint{gp mark 7}{(5.607,1.127)}
\gppoint{gp mark 7}{(5.610,1.127)}
\gppoint{gp mark 7}{(5.614,1.127)}
\gppoint{gp mark 7}{(5.617,1.127)}
\gppoint{gp mark 7}{(5.620,1.127)}
\gppoint{gp mark 7}{(5.624,1.127)}
\gppoint{gp mark 7}{(5.627,1.127)}
\gppoint{gp mark 7}{(5.631,1.127)}
\gppoint{gp mark 7}{(5.634,1.127)}
\gppoint{gp mark 7}{(5.638,1.127)}
\gppoint{gp mark 7}{(5.641,1.128)}
\gppoint{gp mark 7}{(5.645,1.128)}
\gppoint{gp mark 7}{(5.648,1.128)}
\gppoint{gp mark 7}{(5.652,1.128)}
\gppoint{gp mark 7}{(5.655,1.127)}
\gppoint{gp mark 7}{(5.659,1.127)}
\gppoint{gp mark 7}{(5.662,1.127)}
\gppoint{gp mark 7}{(5.666,1.127)}
\gppoint{gp mark 7}{(5.669,1.127)}
\gppoint{gp mark 7}{(5.673,1.127)}
\gppoint{gp mark 7}{(5.676,1.127)}
\gppoint{gp mark 7}{(5.680,1.128)}
\gppoint{gp mark 7}{(5.683,1.128)}
\gppoint{gp mark 7}{(5.687,1.128)}
\gppoint{gp mark 7}{(5.690,1.128)}
\gppoint{gp mark 7}{(5.693,1.128)}
\gppoint{gp mark 7}{(5.697,1.128)}
\gppoint{gp mark 7}{(5.700,1.127)}
\gppoint{gp mark 7}{(5.704,1.127)}
\gppoint{gp mark 7}{(5.707,1.127)}
\gppoint{gp mark 7}{(5.711,1.127)}
\gppoint{gp mark 7}{(5.714,1.127)}
\gppoint{gp mark 7}{(5.718,1.127)}
\gppoint{gp mark 7}{(5.721,1.127)}
\gppoint{gp mark 7}{(5.725,1.127)}
\gppoint{gp mark 7}{(5.728,1.127)}
\gppoint{gp mark 7}{(5.732,1.128)}
\gppoint{gp mark 7}{(5.735,1.128)}
\gppoint{gp mark 7}{(5.739,1.128)}
\gppoint{gp mark 7}{(5.742,1.128)}
\gppoint{gp mark 7}{(5.746,1.128)}
\gppoint{gp mark 7}{(5.749,1.128)}
\gppoint{gp mark 7}{(5.753,1.127)}
\gppoint{gp mark 7}{(5.756,1.127)}
\gppoint{gp mark 7}{(5.760,1.127)}
\gppoint{gp mark 7}{(5.763,1.127)}
\gppoint{gp mark 7}{(5.766,1.127)}
\gppoint{gp mark 7}{(5.770,1.127)}
\gppoint{gp mark 7}{(5.773,1.127)}
\gppoint{gp mark 7}{(5.777,1.128)}
\gppoint{gp mark 7}{(5.780,1.128)}
\gppoint{gp mark 7}{(5.784,1.128)}
\gppoint{gp mark 7}{(5.787,1.128)}
\gppoint{gp mark 7}{(5.791,1.127)}
\gppoint{gp mark 7}{(5.794,1.127)}
\gppoint{gp mark 7}{(5.798,1.127)}
\gppoint{gp mark 7}{(5.801,1.127)}
\gppoint{gp mark 7}{(5.805,1.127)}
\gppoint{gp mark 7}{(5.808,1.127)}
\gppoint{gp mark 7}{(5.812,1.127)}
\gppoint{gp mark 7}{(5.815,1.128)}
\gppoint{gp mark 7}{(5.819,1.128)}
\gppoint{gp mark 7}{(5.822,1.128)}
\gppoint{gp mark 7}{(5.826,1.128)}
\gppoint{gp mark 7}{(5.829,1.128)}
\gppoint{gp mark 7}{(5.833,1.127)}
\gppoint{gp mark 7}{(5.836,1.127)}
\gppoint{gp mark 7}{(5.839,1.127)}
\gppoint{gp mark 7}{(5.843,1.127)}
\gppoint{gp mark 7}{(5.846,1.127)}
\gppoint{gp mark 7}{(5.850,1.127)}
\gppoint{gp mark 7}{(5.853,1.127)}
\gppoint{gp mark 7}{(5.857,1.127)}
\gppoint{gp mark 7}{(5.860,1.128)}
\gppoint{gp mark 7}{(5.864,1.128)}
\gppoint{gp mark 7}{(5.867,1.128)}
\gppoint{gp mark 7}{(5.871,1.128)}
\gppoint{gp mark 7}{(5.874,1.128)}
\gppoint{gp mark 7}{(5.878,1.128)}
\gppoint{gp mark 7}{(5.881,1.127)}
\gppoint{gp mark 7}{(5.885,1.127)}
\gppoint{gp mark 7}{(5.888,1.127)}
\gppoint{gp mark 7}{(5.892,1.127)}
\gppoint{gp mark 7}{(5.895,1.127)}
\gppoint{gp mark 7}{(5.899,1.127)}
\gppoint{gp mark 7}{(5.902,1.127)}
\gppoint{gp mark 7}{(5.906,1.128)}
\gppoint{gp mark 7}{(5.909,1.127)}
\gppoint{gp mark 7}{(5.912,1.127)}
\gppoint{gp mark 7}{(5.916,1.127)}
\gppoint{gp mark 7}{(5.919,1.127)}
\gppoint{gp mark 7}{(5.923,1.127)}
\gppoint{gp mark 7}{(5.926,1.127)}
\gppoint{gp mark 7}{(5.930,1.127)}
\gppoint{gp mark 7}{(5.933,1.127)}
\gppoint{gp mark 7}{(5.937,1.127)}
\gppoint{gp mark 7}{(5.940,1.127)}
\gppoint{gp mark 7}{(5.944,1.128)}
\gppoint{gp mark 7}{(5.947,1.128)}
\gppoint{gp mark 7}{(5.951,1.128)}
\gppoint{gp mark 7}{(5.954,1.128)}
\gppoint{gp mark 7}{(5.958,1.128)}
\gppoint{gp mark 7}{(5.961,1.127)}
\gppoint{gp mark 7}{(5.965,1.127)}
\gppoint{gp mark 7}{(5.968,1.127)}
\gppoint{gp mark 7}{(5.972,1.127)}
\gppoint{gp mark 7}{(5.975,1.127)}
\gppoint{gp mark 7}{(5.978,1.127)}
\gppoint{gp mark 7}{(5.982,1.127)}
\gppoint{gp mark 7}{(5.985,1.127)}
\gppoint{gp mark 7}{(5.989,1.127)}
\gppoint{gp mark 7}{(5.992,1.128)}
\gppoint{gp mark 7}{(5.996,1.128)}
\gppoint{gp mark 7}{(5.999,1.128)}
\gppoint{gp mark 7}{(6.003,1.128)}
\gppoint{gp mark 7}{(6.006,1.127)}
\gppoint{gp mark 7}{(6.010,1.127)}
\gppoint{gp mark 7}{(6.013,1.127)}
\gppoint{gp mark 7}{(6.017,1.127)}
\gppoint{gp mark 7}{(6.020,1.127)}
\gppoint{gp mark 7}{(6.024,1.127)}
\gppoint{gp mark 7}{(6.027,1.127)}
\gppoint{gp mark 7}{(6.031,1.127)}
\gppoint{gp mark 7}{(6.034,1.127)}
\gppoint{gp mark 7}{(6.038,1.127)}
\gppoint{gp mark 7}{(6.041,1.127)}
\gppoint{gp mark 7}{(6.045,1.127)}
\gppoint{gp mark 7}{(6.048,1.127)}
\gppoint{gp mark 7}{(6.051,1.127)}
\gppoint{gp mark 7}{(6.055,1.127)}
\gppoint{gp mark 7}{(6.058,1.127)}
\gppoint{gp mark 7}{(6.062,1.127)}
\gppoint{gp mark 7}{(6.065,1.127)}
\gppoint{gp mark 7}{(6.069,1.127)}
\gppoint{gp mark 7}{(6.072,1.127)}
\gppoint{gp mark 7}{(6.076,1.127)}
\gppoint{gp mark 7}{(6.079,1.127)}
\gppoint{gp mark 7}{(6.083,1.127)}
\gppoint{gp mark 7}{(6.086,1.127)}
\gppoint{gp mark 7}{(6.090,1.127)}
\gppoint{gp mark 7}{(6.093,1.127)}
\gppoint{gp mark 7}{(6.097,1.127)}
\gppoint{gp mark 7}{(6.100,1.127)}
\gppoint{gp mark 7}{(6.104,1.127)}
\gppoint{gp mark 7}{(6.107,1.127)}
\gppoint{gp mark 7}{(6.111,1.127)}
\gppoint{gp mark 7}{(6.114,1.127)}
\gppoint{gp mark 7}{(6.118,1.127)}
\gppoint{gp mark 7}{(6.121,1.127)}
\gppoint{gp mark 7}{(6.124,1.127)}
\gppoint{gp mark 7}{(6.128,1.127)}
\gppoint{gp mark 7}{(6.131,1.127)}
\gppoint{gp mark 7}{(6.135,1.127)}
\gppoint{gp mark 7}{(6.138,1.127)}
\gppoint{gp mark 7}{(6.142,1.127)}
\gppoint{gp mark 7}{(6.145,1.127)}
\gppoint{gp mark 7}{(6.149,1.127)}
\gppoint{gp mark 7}{(6.152,1.127)}
\gppoint{gp mark 7}{(6.156,1.127)}
\gppoint{gp mark 7}{(6.159,1.127)}
\gppoint{gp mark 7}{(6.163,1.127)}
\gppoint{gp mark 7}{(6.166,1.127)}
\gppoint{gp mark 7}{(6.170,1.127)}
\gppoint{gp mark 7}{(6.173,1.127)}
\gppoint{gp mark 7}{(6.177,1.127)}
\gppoint{gp mark 7}{(6.180,1.127)}
\gppoint{gp mark 7}{(6.184,1.127)}
\gppoint{gp mark 7}{(6.187,1.127)}
\gppoint{gp mark 7}{(6.191,1.127)}
\gppoint{gp mark 7}{(6.194,1.127)}
\gppoint{gp mark 7}{(6.197,1.127)}
\gppoint{gp mark 7}{(6.201,1.127)}
\gppoint{gp mark 7}{(6.204,1.127)}
\gppoint{gp mark 7}{(6.208,1.127)}
\gppoint{gp mark 7}{(6.211,1.127)}
\gppoint{gp mark 7}{(6.215,1.127)}
\gppoint{gp mark 7}{(6.218,1.127)}
\gppoint{gp mark 7}{(6.222,1.127)}
\gppoint{gp mark 7}{(6.225,1.127)}
\gppoint{gp mark 7}{(6.229,1.127)}
\gppoint{gp mark 7}{(6.232,1.127)}
\gppoint{gp mark 7}{(6.236,1.127)}
\gppoint{gp mark 7}{(6.239,1.127)}
\gppoint{gp mark 7}{(6.243,1.127)}
\gppoint{gp mark 7}{(6.246,1.127)}
\gppoint{gp mark 7}{(6.250,1.127)}
\gppoint{gp mark 7}{(6.253,1.127)}
\gppoint{gp mark 7}{(6.257,1.127)}
\gppoint{gp mark 7}{(6.260,1.127)}
\gppoint{gp mark 7}{(6.264,1.128)}
\gppoint{gp mark 7}{(6.267,1.128)}
\gppoint{gp mark 7}{(6.270,1.127)}
\gppoint{gp mark 7}{(6.274,1.127)}
\gppoint{gp mark 7}{(6.277,1.127)}
\gppoint{gp mark 7}{(6.281,1.127)}
\gppoint{gp mark 7}{(6.284,1.127)}
\gppoint{gp mark 7}{(6.288,1.127)}
\gppoint{gp mark 7}{(6.291,1.127)}
\gppoint{gp mark 7}{(6.295,1.127)}
\gppoint{gp mark 7}{(6.298,1.127)}
\gppoint{gp mark 7}{(6.302,1.128)}
\gppoint{gp mark 7}{(6.305,1.128)}
\gppoint{gp mark 7}{(6.309,1.128)}
\gppoint{gp mark 7}{(6.312,1.127)}
\gppoint{gp mark 7}{(6.316,1.127)}
\gppoint{gp mark 7}{(6.319,1.127)}
\gppoint{gp mark 7}{(6.323,1.126)}
\gppoint{gp mark 7}{(6.326,1.127)}
\gppoint{gp mark 7}{(6.330,1.127)}
\gppoint{gp mark 7}{(6.333,1.127)}
\gppoint{gp mark 7}{(6.337,1.127)}
\gppoint{gp mark 7}{(6.340,1.127)}
\gppoint{gp mark 7}{(6.343,1.127)}
\gppoint{gp mark 7}{(6.347,1.127)}
\gppoint{gp mark 7}{(6.350,1.127)}
\gppoint{gp mark 7}{(6.354,1.126)}
\gppoint{gp mark 7}{(6.357,1.126)}
\gppoint{gp mark 7}{(6.361,1.127)}
\gppoint{gp mark 7}{(6.364,1.127)}
\gppoint{gp mark 7}{(6.368,1.127)}
\gppoint{gp mark 7}{(6.371,1.126)}
\gppoint{gp mark 7}{(6.375,1.126)}
\gppoint{gp mark 7}{(6.378,1.129)}
\gppoint{gp mark 7}{(6.382,1.132)}
\gppoint{gp mark 7}{(6.385,1.165)}
\gppoint{gp mark 7}{(6.389,1.405)}
\gppoint{gp mark 7}{(6.392,2.280)}
\gppoint{gp mark 7}{(6.396,3.215)}
\gppoint{gp mark 7}{(6.399,3.582)}
\gppoint{gp mark 7}{(6.403,3.795)}
\gppoint{gp mark 7}{(6.406,3.928)}
\gppoint{gp mark 7}{(6.410,4.003)}
\gppoint{gp mark 7}{(6.413,4.044)}
\gppoint{gp mark 7}{(6.416,4.064)}
\gppoint{gp mark 7}{(6.420,4.073)}
\gppoint{gp mark 7}{(6.423,4.077)}
\gppoint{gp mark 7}{(6.427,4.078)}
\gppoint{gp mark 7}{(6.430,4.079)}
\gppoint{gp mark 7}{(6.434,4.079)}
\gppoint{gp mark 7}{(6.437,4.079)}
\gppoint{gp mark 7}{(6.441,4.079)}
\gppoint{gp mark 7}{(6.444,4.079)}
\gppoint{gp mark 7}{(6.448,4.079)}
\gppoint{gp mark 7}{(6.451,4.079)}
\gppoint{gp mark 7}{(6.455,4.079)}
\gppoint{gp mark 7}{(6.458,4.079)}
\gppoint{gp mark 7}{(6.462,4.078)}
\gppoint{gp mark 7}{(6.465,4.078)}
\gppoint{gp mark 7}{(6.469,4.078)}
\gppoint{gp mark 7}{(6.472,4.078)}
\gppoint{gp mark 7}{(6.476,4.078)}
\gppoint{gp mark 7}{(6.479,4.078)}
\gppoint{gp mark 7}{(6.483,4.078)}
\gppoint{gp mark 7}{(6.486,4.078)}
\gppoint{gp mark 7}{(6.489,4.078)}
\gppoint{gp mark 7}{(6.493,4.078)}
\gppoint{gp mark 7}{(6.496,4.078)}
\gppoint{gp mark 7}{(6.500,4.079)}
\gppoint{gp mark 7}{(6.503,4.079)}
\gppoint{gp mark 7}{(6.507,4.079)}
\gppoint{gp mark 7}{(6.510,4.079)}
\gppoint{gp mark 7}{(6.514,4.079)}
\gppoint{gp mark 7}{(6.517,4.079)}
\gppoint{gp mark 7}{(6.521,4.079)}
\gppoint{gp mark 7}{(6.524,4.079)}
\gppoint{gp mark 7}{(6.528,4.079)}
\gppoint{gp mark 7}{(6.531,4.079)}
\gppoint{gp mark 7}{(6.535,4.079)}
\gppoint{gp mark 7}{(6.538,4.079)}
\gppoint{gp mark 7}{(6.542,4.079)}
\gppoint{gp mark 7}{(6.545,4.079)}
\gppoint{gp mark 7}{(6.549,4.079)}
\gppoint{gp mark 7}{(6.552,4.079)}
\gppoint{gp mark 7}{(6.556,4.079)}
\gppoint{gp mark 7}{(6.559,4.078)}
\gppoint{gp mark 7}{(6.562,4.078)}
\gppoint{gp mark 7}{(6.566,4.078)}
\gppoint{gp mark 7}{(6.569,4.078)}
\gppoint{gp mark 7}{(6.573,4.078)}
\gppoint{gp mark 7}{(6.576,4.078)}
\gppoint{gp mark 7}{(6.580,4.078)}
\gppoint{gp mark 7}{(6.583,4.078)}
\gppoint{gp mark 7}{(6.587,4.078)}
\gppoint{gp mark 7}{(6.590,4.078)}
\gppoint{gp mark 7}{(6.594,4.079)}
\gppoint{gp mark 7}{(6.597,4.079)}
\gppoint{gp mark 7}{(6.601,4.079)}
\gppoint{gp mark 7}{(6.604,4.079)}
\gppoint{gp mark 7}{(6.608,4.079)}
\gppoint{gp mark 7}{(6.611,4.079)}
\gppoint{gp mark 7}{(6.615,4.079)}
\gppoint{gp mark 7}{(6.618,4.079)}
\gppoint{gp mark 7}{(6.622,4.079)}
\gppoint{gp mark 7}{(6.625,4.079)}
\gppoint{gp mark 7}{(6.629,4.079)}
\gppoint{gp mark 7}{(6.632,4.078)}
\gppoint{gp mark 7}{(6.635,4.078)}
\gppoint{gp mark 7}{(6.639,4.078)}
\gppoint{gp mark 7}{(6.642,4.078)}
\gppoint{gp mark 7}{(6.646,4.078)}
\gppoint{gp mark 7}{(6.649,4.078)}
\gppoint{gp mark 7}{(6.653,4.078)}
\gppoint{gp mark 7}{(6.656,4.078)}
\gppoint{gp mark 7}{(6.660,4.078)}
\gppoint{gp mark 7}{(6.663,4.078)}
\gppoint{gp mark 7}{(6.667,4.078)}
\gppoint{gp mark 7}{(6.670,4.078)}
\gppoint{gp mark 7}{(6.674,4.078)}
\gppoint{gp mark 7}{(6.677,4.078)}
\gppoint{gp mark 7}{(6.681,4.078)}
\gppoint{gp mark 7}{(6.684,4.078)}
\gppoint{gp mark 7}{(6.688,4.078)}
\gppoint{gp mark 7}{(6.691,4.078)}
\gppoint{gp mark 7}{(6.695,4.078)}
\gppoint{gp mark 7}{(6.698,4.078)}
\gppoint{gp mark 7}{(6.702,4.078)}
\gppoint{gp mark 7}{(6.705,4.078)}
\gppoint{gp mark 7}{(6.708,4.078)}
\gppoint{gp mark 7}{(6.712,4.077)}
\gppoint{gp mark 7}{(6.715,4.077)}
\gppoint{gp mark 7}{(6.719,4.077)}
\gppoint{gp mark 7}{(6.722,4.077)}
\gppoint{gp mark 7}{(6.726,4.077)}
\gppoint{gp mark 7}{(6.729,4.077)}
\gppoint{gp mark 7}{(6.733,4.077)}
\gppoint{gp mark 7}{(6.736,4.077)}
\gppoint{gp mark 7}{(6.740,4.077)}
\gppoint{gp mark 7}{(6.743,4.077)}
\gppoint{gp mark 7}{(6.747,4.077)}
\gppoint{gp mark 7}{(6.750,4.077)}
\gppoint{gp mark 7}{(6.754,4.077)}
\gppoint{gp mark 7}{(6.757,4.077)}
\gppoint{gp mark 7}{(6.761,4.077)}
\gppoint{gp mark 7}{(6.764,4.076)}
\gppoint{gp mark 7}{(6.768,4.076)}
\gppoint{gp mark 7}{(6.771,4.076)}
\gppoint{gp mark 7}{(6.775,4.076)}
\gppoint{gp mark 7}{(6.778,4.076)}
\gppoint{gp mark 7}{(6.781,4.076)}
\gppoint{gp mark 7}{(6.785,4.076)}
\gppoint{gp mark 7}{(6.788,4.075)}
\gppoint{gp mark 7}{(6.792,4.075)}
\gppoint{gp mark 7}{(6.795,4.075)}
\gppoint{gp mark 7}{(6.799,4.075)}
\gppoint{gp mark 7}{(6.802,4.074)}
\gppoint{gp mark 7}{(6.806,4.074)}
\gppoint{gp mark 7}{(6.809,4.074)}
\gppoint{gp mark 7}{(6.813,4.074)}
\gppoint{gp mark 7}{(6.816,4.074)}
\gppoint{gp mark 7}{(6.820,4.074)}
\gppoint{gp mark 7}{(6.823,4.073)}
\gppoint{gp mark 7}{(6.827,4.073)}
\gppoint{gp mark 7}{(6.830,4.072)}
\gppoint{gp mark 7}{(6.834,4.072)}
\gppoint{gp mark 7}{(6.837,4.072)}
\gppoint{gp mark 7}{(6.841,4.072)}
\gppoint{gp mark 7}{(6.844,4.072)}
\gppoint{gp mark 7}{(6.848,4.071)}
\gppoint{gp mark 7}{(6.851,4.070)}
\gppoint{gp mark 7}{(6.854,4.069)}
\gppoint{gp mark 7}{(6.858,4.067)}
\gppoint{gp mark 7}{(6.861,4.066)}
\gppoint{gp mark 7}{(6.865,4.066)}
\gppoint{gp mark 7}{(6.868,4.066)}
\gppoint{gp mark 7}{(6.872,4.066)}
\gppoint{gp mark 7}{(6.875,4.066)}
\gppoint{gp mark 7}{(6.879,4.065)}
\gppoint{gp mark 7}{(6.882,4.059)}
\gppoint{gp mark 7}{(6.886,4.047)}
\gppoint{gp mark 7}{(6.889,4.035)}
\gppoint{gp mark 7}{(6.893,4.028)}
\gppoint{gp mark 7}{(6.896,4.026)}
\gppoint{gp mark 7}{(6.900,4.026)}
\gppoint{gp mark 7}{(6.903,4.026)}
\gppoint{gp mark 7}{(6.907,4.026)}
\gppoint{gp mark 7}{(6.910,4.027)}
\gppoint{gp mark 7}{(6.914,4.029)}
\gppoint{gp mark 7}{(6.917,4.033)}
\gppoint{gp mark 7}{(6.921,4.037)}
\gppoint{gp mark 7}{(6.924,4.041)}
\gppoint{gp mark 7}{(6.927,4.044)}
\gppoint{gp mark 7}{(6.931,4.047)}
\gppoint{gp mark 7}{(6.934,4.051)}
\gppoint{gp mark 7}{(6.938,4.054)}
\gppoint{gp mark 7}{(6.941,4.056)}
\gppoint{gp mark 7}{(6.945,4.059)}
\gppoint{gp mark 7}{(6.948,4.062)}
\gppoint{gp mark 7}{(6.952,4.065)}
\gppoint{gp mark 7}{(6.955,4.067)}
\gppoint{gp mark 7}{(6.959,4.070)}
\gppoint{gp mark 7}{(6.962,4.073)}
\gppoint{gp mark 7}{(6.966,4.076)}
\gppoint{gp mark 7}{(6.969,4.078)}
\gppoint{gp mark 7}{(6.973,4.081)}
\gppoint{gp mark 7}{(6.976,4.083)}
\gppoint{gp mark 7}{(6.980,4.086)}
\gppoint{gp mark 7}{(6.983,4.089)}
\gppoint{gp mark 7}{(6.987,4.091)}
\gppoint{gp mark 7}{(6.990,4.094)}
\gppoint{gp mark 7}{(6.994,4.097)}
\gppoint{gp mark 7}{(6.997,4.099)}
\gppoint{gp mark 7}{(7.000,4.102)}
\gppoint{gp mark 7}{(7.004,4.105)}
\gppoint{gp mark 7}{(7.007,4.107)}
\gppoint{gp mark 7}{(7.011,4.110)}
\gppoint{gp mark 7}{(7.014,4.112)}
\gppoint{gp mark 7}{(7.018,4.115)}
\gppoint{gp mark 7}{(7.021,4.118)}
\gppoint{gp mark 7}{(7.025,4.120)}
\gppoint{gp mark 7}{(7.028,4.123)}
\gppoint{gp mark 7}{(7.032,4.125)}
\gppoint{gp mark 7}{(7.035,4.128)}
\gppoint{gp mark 7}{(7.039,4.131)}
\gppoint{gp mark 7}{(7.042,4.133)}
\gppoint{gp mark 7}{(7.046,4.136)}
\gppoint{gp mark 7}{(7.049,4.138)}
\gppoint{gp mark 7}{(7.053,4.141)}
\gppoint{gp mark 7}{(7.056,4.143)}
\gppoint{gp mark 7}{(7.060,4.146)}
\gppoint{gp mark 7}{(7.063,4.149)}
\gppoint{gp mark 7}{(7.067,4.151)}
\gppoint{gp mark 7}{(7.070,4.154)}
\gppoint{gp mark 7}{(7.073,4.156)}
\gppoint{gp mark 7}{(7.077,4.159)}
\gppoint{gp mark 7}{(7.080,4.161)}
\gppoint{gp mark 7}{(7.084,4.164)}
\gppoint{gp mark 7}{(7.087,4.166)}
\gppoint{gp mark 7}{(7.091,4.169)}
\gppoint{gp mark 7}{(7.094,4.172)}
\gppoint{gp mark 7}{(7.098,4.174)}
\gppoint{gp mark 7}{(7.101,4.177)}
\gppoint{gp mark 7}{(7.105,4.179)}
\gppoint{gp mark 7}{(7.108,4.182)}
\gppoint{gp mark 7}{(7.112,4.184)}
\gppoint{gp mark 7}{(7.115,4.187)}
\gppoint{gp mark 7}{(7.119,4.189)}
\gppoint{gp mark 7}{(7.122,4.192)}
\gppoint{gp mark 7}{(7.126,4.194)}
\gppoint{gp mark 7}{(7.129,4.197)}
\gppoint{gp mark 7}{(7.133,4.199)}
\gppoint{gp mark 7}{(7.136,4.202)}
\gppoint{gp mark 7}{(7.140,4.204)}
\gppoint{gp mark 7}{(7.143,4.207)}
\gppoint{gp mark 7}{(7.146,4.209)}
\gppoint{gp mark 7}{(7.150,4.212)}
\gppoint{gp mark 7}{(7.153,4.214)}
\gppoint{gp mark 7}{(7.157,4.217)}
\gppoint{gp mark 7}{(7.160,4.220)}
\gppoint{gp mark 7}{(7.164,4.222)}
\gppoint{gp mark 7}{(7.167,4.224)}
\gppoint{gp mark 7}{(7.171,4.227)}
\gppoint{gp mark 7}{(7.174,4.229)}
\gppoint{gp mark 7}{(7.178,4.232)}
\gppoint{gp mark 7}{(7.181,4.234)}
\gppoint{gp mark 7}{(7.185,4.237)}
\gppoint{gp mark 7}{(7.188,4.239)}
\gppoint{gp mark 7}{(7.192,4.242)}
\gppoint{gp mark 7}{(7.195,4.244)}
\gppoint{gp mark 7}{(7.199,4.247)}
\gppoint{gp mark 7}{(7.202,4.249)}
\gppoint{gp mark 7}{(7.206,4.252)}
\gppoint{gp mark 7}{(7.209,4.254)}
\gppoint{gp mark 7}{(7.213,4.257)}
\gppoint{gp mark 7}{(7.216,4.259)}
\gppoint{gp mark 7}{(7.219,4.262)}
\gppoint{gp mark 7}{(7.223,4.264)}
\gppoint{gp mark 7}{(7.226,4.267)}
\gppoint{gp mark 7}{(7.230,4.269)}
\gppoint{gp mark 7}{(7.233,4.272)}
\gppoint{gp mark 7}{(7.237,4.274)}
\gppoint{gp mark 7}{(7.240,4.276)}
\gppoint{gp mark 7}{(7.244,4.279)}
\gppoint{gp mark 7}{(7.247,4.281)}
\gppoint{gp mark 7}{(7.251,4.284)}
\gppoint{gp mark 7}{(7.254,4.286)}
\gppoint{gp mark 7}{(7.258,4.289)}
\gppoint{gp mark 7}{(7.261,4.291)}
\gppoint{gp mark 7}{(7.265,4.294)}
\gppoint{gp mark 7}{(7.268,4.296)}
\gppoint{gp mark 7}{(7.272,4.298)}
\gppoint{gp mark 7}{(7.275,4.301)}
\gppoint{gp mark 7}{(7.279,4.303)}
\gppoint{gp mark 7}{(7.282,4.306)}
\gppoint{gp mark 7}{(7.286,4.308)}
\gppoint{gp mark 7}{(7.289,4.310)}
\gppoint{gp mark 7}{(7.292,4.313)}
\gppoint{gp mark 7}{(7.296,4.315)}
\gppoint{gp mark 7}{(7.299,4.318)}
\gppoint{gp mark 7}{(7.303,4.320)}
\gppoint{gp mark 7}{(7.306,4.323)}
\gppoint{gp mark 7}{(7.310,4.325)}
\gppoint{gp mark 7}{(7.313,4.327)}
\gppoint{gp mark 7}{(7.317,4.330)}
\gppoint{gp mark 7}{(7.320,4.332)}
\gppoint{gp mark 7}{(7.324,4.335)}
\gppoint{gp mark 7}{(7.327,4.337)}
\gppoint{gp mark 7}{(7.331,4.339)}
\gppoint{gp mark 7}{(7.334,4.342)}
\gppoint{gp mark 7}{(7.338,4.344)}
\gppoint{gp mark 7}{(7.341,4.347)}
\gppoint{gp mark 7}{(7.345,4.349)}
\gppoint{gp mark 7}{(7.348,4.351)}
\gppoint{gp mark 7}{(7.352,4.354)}
\gppoint{gp mark 7}{(7.355,4.356)}
\gppoint{gp mark 7}{(7.359,4.359)}
\gppoint{gp mark 7}{(7.362,4.361)}
\gppoint{gp mark 7}{(7.365,4.363)}
\gppoint{gp mark 7}{(7.369,4.366)}
\gppoint{gp mark 7}{(7.372,4.368)}
\gppoint{gp mark 7}{(7.376,4.370)}
\gppoint{gp mark 7}{(7.379,4.373)}
\gppoint{gp mark 7}{(7.383,4.375)}
\gppoint{gp mark 7}{(7.386,4.378)}
\gppoint{gp mark 7}{(7.390,4.380)}
\gppoint{gp mark 7}{(7.393,4.382)}
\gppoint{gp mark 7}{(7.397,4.385)}
\gppoint{gp mark 7}{(7.400,4.387)}
\gppoint{gp mark 7}{(7.404,4.389)}
\gppoint{gp mark 7}{(7.407,4.392)}
\gppoint{gp mark 7}{(7.411,4.394)}
\gppoint{gp mark 7}{(7.414,4.396)}
\gppoint{gp mark 7}{(7.418,4.399)}
\gppoint{gp mark 7}{(7.421,4.401)}
\gppoint{gp mark 7}{(7.425,4.403)}
\gppoint{gp mark 7}{(7.428,4.406)}
\gppoint{gp mark 7}{(7.431,4.408)}
\gppoint{gp mark 7}{(7.435,4.411)}
\gppoint{gp mark 7}{(7.438,4.413)}
\gppoint{gp mark 7}{(7.442,4.415)}
\gppoint{gp mark 7}{(7.445,4.418)}
\gppoint{gp mark 7}{(7.449,4.420)}
\gppoint{gp mark 7}{(7.452,4.422)}
\gppoint{gp mark 7}{(7.456,4.425)}
\gppoint{gp mark 7}{(7.459,4.427)}
\gppoint{gp mark 7}{(7.463,4.429)}
\gppoint{gp mark 7}{(7.466,4.432)}
\gppoint{gp mark 7}{(7.470,4.434)}
\gppoint{gp mark 7}{(7.473,4.436)}
\gppoint{gp mark 7}{(7.477,4.439)}
\gppoint{gp mark 7}{(7.480,4.441)}
\gppoint{gp mark 7}{(7.484,4.443)}
\gppoint{gp mark 7}{(7.487,4.446)}
\gppoint{gp mark 7}{(7.491,4.448)}
\gppoint{gp mark 7}{(7.494,4.450)}
\gppoint{gp mark 7}{(7.498,4.453)}
\gppoint{gp mark 7}{(7.501,4.455)}
\gppoint{gp mark 7}{(7.504,4.457)}
\gppoint{gp mark 7}{(7.508,4.459)}
\gppoint{gp mark 7}{(7.511,4.462)}
\gppoint{gp mark 7}{(7.515,4.464)}
\gppoint{gp mark 7}{(7.518,4.466)}
\gppoint{gp mark 7}{(7.522,4.469)}
\gppoint{gp mark 7}{(7.525,4.471)}
\gppoint{gp mark 7}{(7.529,4.473)}
\gppoint{gp mark 7}{(7.532,4.476)}
\gppoint{gp mark 7}{(7.536,4.478)}
\gppoint{gp mark 7}{(7.539,4.480)}
\gppoint{gp mark 7}{(7.543,4.483)}
\gppoint{gp mark 7}{(7.546,4.485)}
\gppoint{gp mark 7}{(7.550,4.487)}
\gppoint{gp mark 7}{(7.553,4.489)}
\gppoint{gp mark 7}{(7.557,4.492)}
\gppoint{gp mark 7}{(7.560,4.494)}
\gppoint{gp mark 7}{(7.564,4.496)}
\gppoint{gp mark 7}{(7.567,4.499)}
\gppoint{gp mark 7}{(7.571,4.501)}
\gppoint{gp mark 7}{(7.574,4.503)}
\gppoint{gp mark 7}{(7.577,4.505)}
\gppoint{gp mark 7}{(7.581,4.508)}
\gppoint{gp mark 7}{(7.584,4.510)}
\gppoint{gp mark 7}{(7.588,4.512)}
\gppoint{gp mark 7}{(7.591,4.515)}
\gppoint{gp mark 7}{(7.595,4.517)}
\gppoint{gp mark 7}{(7.598,4.519)}
\gppoint{gp mark 7}{(7.602,4.521)}
\gppoint{gp mark 7}{(7.605,4.524)}
\gppoint{gp mark 7}{(7.609,4.526)}
\gppoint{gp mark 7}{(7.612,4.528)}
\gppoint{gp mark 7}{(7.616,4.530)}
\gppoint{gp mark 7}{(7.619,4.533)}
\gppoint{gp mark 7}{(7.623,4.535)}
\gppoint{gp mark 7}{(7.626,4.537)}
\gppoint{gp mark 7}{(7.630,4.540)}
\gppoint{gp mark 7}{(7.633,4.542)}
\gppoint{gp mark 7}{(7.637,4.544)}
\gppoint{gp mark 7}{(7.640,4.546)}
\gppoint{gp mark 7}{(7.644,4.549)}
\gppoint{gp mark 7}{(7.647,4.551)}
\gppoint{gp mark 7}{(7.650,4.553)}
\gppoint{gp mark 7}{(7.654,4.555)}
\gppoint{gp mark 7}{(7.657,4.558)}
\gppoint{gp mark 7}{(7.661,4.560)}
\gppoint{gp mark 7}{(7.664,4.562)}
\gppoint{gp mark 7}{(7.668,4.564)}
\gppoint{gp mark 7}{(7.671,4.567)}
\gppoint{gp mark 7}{(7.675,4.569)}
\gppoint{gp mark 7}{(7.678,4.571)}
\gppoint{gp mark 7}{(7.682,4.573)}
\gppoint{gp mark 7}{(7.685,4.576)}
\gppoint{gp mark 7}{(7.689,4.578)}
\gppoint{gp mark 7}{(7.692,4.580)}
\gppoint{gp mark 7}{(7.696,4.582)}
\gppoint{gp mark 7}{(7.699,4.585)}
\gppoint{gp mark 7}{(7.703,4.587)}
\gppoint{gp mark 7}{(7.706,4.589)}
\gppoint{gp mark 7}{(7.710,4.591)}
\gppoint{gp mark 7}{(7.713,4.594)}
\gppoint{gp mark 7}{(7.717,4.596)}
\gppoint{gp mark 7}{(7.720,4.598)}
\gppoint{gp mark 7}{(7.723,4.600)}
\gppoint{gp mark 7}{(7.727,4.603)}
\gppoint{gp mark 7}{(7.730,4.605)}
\gppoint{gp mark 7}{(7.734,4.607)}
\gppoint{gp mark 7}{(7.737,4.609)}
\gppoint{gp mark 7}{(7.741,4.611)}
\gppoint{gp mark 7}{(7.744,4.614)}
\gppoint{gp mark 7}{(7.748,4.616)}
\gppoint{gp mark 7}{(7.751,4.618)}
\gppoint{gp mark 7}{(7.755,4.620)}
\gppoint{gp mark 7}{(7.758,4.623)}
\gppoint{gp mark 7}{(7.762,4.625)}
\gppoint{gp mark 7}{(7.765,4.627)}
\gppoint{gp mark 7}{(7.769,4.629)}
\gppoint{gp mark 7}{(7.772,4.631)}
\gppoint{gp mark 7}{(7.776,4.634)}
\gppoint{gp mark 7}{(7.779,4.636)}
\gppoint{gp mark 7}{(7.783,4.638)}
\gppoint{gp mark 7}{(7.786,4.640)}
\gppoint{gp mark 7}{(7.790,4.643)}
\gppoint{gp mark 7}{(7.793,4.645)}
\gppoint{gp mark 7}{(7.796,4.647)}
\gppoint{gp mark 7}{(7.800,4.649)}
\gppoint{gp mark 7}{(7.803,4.651)}
\gppoint{gp mark 7}{(7.807,4.654)}
\gppoint{gp mark 7}{(7.810,4.656)}
\gppoint{gp mark 7}{(7.814,4.658)}
\gppoint{gp mark 7}{(7.817,4.660)}
\gppoint{gp mark 7}{(7.821,4.662)}
\gppoint{gp mark 7}{(7.824,4.665)}
\gppoint{gp mark 7}{(7.828,4.667)}
\gppoint{gp mark 7}{(7.831,4.669)}
\gppoint{gp mark 7}{(7.835,4.671)}
\gppoint{gp mark 7}{(7.838,4.673)}
\gppoint{gp mark 7}{(7.842,4.676)}
\gppoint{gp mark 7}{(7.845,4.678)}
\gppoint{gp mark 7}{(7.849,4.680)}
\gppoint{gp mark 7}{(7.852,4.682)}
\gppoint{gp mark 7}{(7.856,4.684)}
\gppoint{gp mark 7}{(7.859,4.687)}
\gppoint{gp mark 7}{(7.863,4.689)}
\gppoint{gp mark 7}{(7.866,4.691)}
\gppoint{gp mark 7}{(7.869,4.693)}
\gppoint{gp mark 7}{(7.873,4.695)}
\gppoint{gp mark 7}{(7.876,4.698)}
\gppoint{gp mark 7}{(7.880,4.700)}
\gppoint{gp mark 7}{(7.883,4.702)}
\gppoint{gp mark 7}{(7.887,4.704)}
\gppoint{gp mark 7}{(7.890,4.706)}
\gppoint{gp mark 7}{(7.894,4.708)}
\gppoint{gp mark 7}{(7.897,4.711)}
\gppoint{gp mark 7}{(7.901,4.713)}
\gppoint{gp mark 7}{(7.904,4.715)}
\gppoint{gp mark 7}{(7.908,4.717)}
\gppoint{gp mark 7}{(7.911,4.719)}
\gppoint{gp mark 7}{(7.915,4.722)}
\gppoint{gp mark 7}{(7.918,4.724)}
\gppoint{gp mark 7}{(7.922,4.726)}
\gppoint{gp mark 7}{(7.925,4.728)}
\gppoint{gp mark 7}{(7.929,4.730)}
\gppoint{gp mark 7}{(7.932,4.732)}
\gppoint{gp mark 7}{(7.936,4.735)}
\gppoint{gp mark 7}{(7.939,4.737)}
\gppoint{gp mark 7}{(7.942,4.739)}
\gpcolor{rgb color={0.000,0.000,0.000}}
\gpsetlinetype{gp lt plot 0}
\draw[gp path] (1.361,4.567)--(1.769,4.567);
\draw[gp path] (1.769,1.153)--(3.245,1.153);
\draw[gp path] (3.245,1.335)--(3.746,1.335);
\draw[gp path] (3.746,1.335)--(4.294,1.335);
\draw[gp path] (4.294,1.127)--(6.394,1.127);
\draw[gp path] (6.394,4.081)--(6.972,4.081);
\draw[gp path] (0.829,5.258)--(0.833,5.253)--(0.837,5.248)--(0.841,5.243)--(0.845,5.238)%
  --(0.849,5.233)--(0.853,5.228)--(0.857,5.223)--(0.861,5.218)--(0.864,5.213)--(0.868,5.208)%
  --(0.872,5.203)--(0.876,5.197)--(0.880,5.192)--(0.884,5.187)--(0.888,5.182)--(0.892,5.177)%
  --(0.896,5.172)--(0.900,5.167)--(0.904,5.162)--(0.907,5.157)--(0.911,5.152)--(0.915,5.147)%
  --(0.919,5.142)--(0.923,5.136)--(0.927,5.131)--(0.931,5.126)--(0.935,5.121)--(0.939,5.116)%
  --(0.943,5.111)--(0.947,5.106)--(0.950,5.101)--(0.954,5.096)--(0.958,5.091)--(0.962,5.086)%
  --(0.966,5.081)--(0.970,5.075)--(0.974,5.070)--(0.978,5.065)--(0.982,5.060)--(0.986,5.055)%
  --(0.990,5.050)--(0.993,5.045)--(0.997,5.040)--(1.001,5.035)--(1.005,5.030)--(1.009,5.025)%
  --(1.013,5.020)--(1.017,5.014)--(1.021,5.009)--(1.025,5.004)--(1.029,4.999)--(1.033,4.994)%
  --(1.036,4.989)--(1.040,4.984)--(1.044,4.979)--(1.048,4.974)--(1.052,4.969)--(1.056,4.964)%
  --(1.060,4.959)--(1.064,4.953)--(1.068,4.948)--(1.072,4.943)--(1.076,4.938)--(1.079,4.933)%
  --(1.083,4.928)--(1.087,4.923)--(1.091,4.918)--(1.095,4.913)--(1.099,4.908)--(1.103,4.903)%
  --(1.107,4.898)--(1.111,4.892)--(1.115,4.887)--(1.119,4.882)--(1.122,4.877)--(1.126,4.872)%
  --(1.130,4.867)--(1.134,4.862)--(1.138,4.857)--(1.142,4.852)--(1.146,4.847)--(1.150,4.842)%
  --(1.154,4.837)--(1.158,4.831)--(1.162,4.826)--(1.165,4.821)--(1.169,4.816)--(1.173,4.811)%
  --(1.177,4.806)--(1.181,4.801)--(1.185,4.796)--(1.189,4.791)--(1.193,4.786)--(1.197,4.781)%
  --(1.201,4.776)--(1.204,4.770)--(1.208,4.765)--(1.212,4.760)--(1.216,4.755)--(1.220,4.750)%
  --(1.224,4.745)--(1.228,4.740)--(1.232,4.735)--(1.236,4.730)--(1.240,4.725)--(1.244,4.720)%
  --(1.247,4.715)--(1.251,4.709)--(1.255,4.704)--(1.259,4.699)--(1.263,4.694)--(1.267,4.689)%
  --(1.271,4.684)--(1.275,4.679)--(1.279,4.674)--(1.283,4.669)--(1.287,4.664)--(1.290,4.659)%
  --(1.294,4.654)--(1.298,4.648)--(1.302,4.643)--(1.306,4.638)--(1.310,4.633)--(1.314,4.628)%
  --(1.318,4.623)--(1.322,4.618)--(1.326,4.613)--(1.330,4.608)--(1.333,4.603)--(1.337,4.598)%
  --(1.341,4.593)--(1.345,4.587)--(1.349,4.582)--(1.353,4.577)--(1.357,4.572)--(1.361,4.567);
\draw[gp path] (1.769,4.567)--(1.769,1.153);
\draw[gp path] (3.245,1.153)--(3.245,1.335);
\draw[gp path] (4.294,1.335)--(4.294,1.127);
\draw[gp path] (6.394,1.127)--(6.394,4.081);
\draw[gp path] (6.972,4.081)--(6.978,4.084)--(6.983,4.088)--(6.988,4.091)--(6.994,4.095)%
  --(6.999,4.099)--(7.004,4.102)--(7.010,4.106)--(7.015,4.109)--(7.020,4.113)--(7.026,4.116)%
  --(7.031,4.120)--(7.036,4.123)--(7.042,4.127)--(7.047,4.130)--(7.052,4.134)--(7.057,4.137)%
  --(7.063,4.141)--(7.068,4.144)--(7.073,4.148)--(7.079,4.151)--(7.084,4.155)--(7.089,4.158)%
  --(7.095,4.162)--(7.100,4.165)--(7.105,4.169)--(7.111,4.172)--(7.116,4.176)--(7.121,4.179)%
  --(7.127,4.183)--(7.132,4.187)--(7.137,4.190)--(7.143,4.194)--(7.148,4.197)--(7.153,4.201)%
  --(7.159,4.204)--(7.164,4.208)--(7.169,4.211)--(7.175,4.215)--(7.180,4.218)--(7.185,4.222)%
  --(7.191,4.225)--(7.196,4.229)--(7.201,4.232)--(7.207,4.236)--(7.212,4.239)--(7.217,4.243)%
  --(7.223,4.246)--(7.228,4.250)--(7.233,4.253)--(7.239,4.257)--(7.244,4.260)--(7.249,4.264)%
  --(7.255,4.268)--(7.260,4.271)--(7.265,4.275)--(7.271,4.278)--(7.276,4.282)--(7.281,4.285)%
  --(7.287,4.289)--(7.292,4.292)--(7.297,4.296)--(7.303,4.299)--(7.308,4.303)--(7.313,4.306)%
  --(7.318,4.310)--(7.324,4.313)--(7.329,4.317)--(7.334,4.320)--(7.340,4.324)--(7.345,4.327)%
  --(7.350,4.331)--(7.356,4.334)--(7.361,4.338)--(7.366,4.341)--(7.372,4.345)--(7.377,4.349)%
  --(7.382,4.352)--(7.388,4.356)--(7.393,4.359)--(7.398,4.363)--(7.404,4.366)--(7.409,4.370)%
  --(7.414,4.373)--(7.420,4.377)--(7.425,4.380)--(7.430,4.384)--(7.436,4.387)--(7.441,4.391)%
  --(7.446,4.394)--(7.452,4.398)--(7.457,4.401)--(7.462,4.405)--(7.468,4.408)--(7.473,4.412)%
  --(7.478,4.415)--(7.484,4.419)--(7.489,4.422)--(7.494,4.426)--(7.500,4.429)--(7.505,4.433)%
  --(7.510,4.437)--(7.516,4.440)--(7.521,4.444)--(7.526,4.447)--(7.532,4.451)--(7.537,4.454)%
  --(7.542,4.458)--(7.548,4.461)--(7.553,4.465)--(7.558,4.468)--(7.564,4.472)--(7.569,4.475)%
  --(7.574,4.479)--(7.580,4.482)--(7.585,4.486)--(7.590,4.489)--(7.595,4.493)--(7.601,4.496)%
  --(7.606,4.500)--(7.611,4.503)--(7.617,4.507)--(7.622,4.510)--(7.627,4.514)--(7.633,4.518)%
  --(7.638,4.521)--(7.643,4.525)--(7.649,4.528)--(7.654,4.532)--(7.659,4.535)--(7.665,4.539)%
  --(7.670,4.542)--(7.675,4.546)--(7.681,4.549)--(7.686,4.553)--(7.691,4.556)--(7.697,4.560)%
  --(7.702,4.563)--(7.707,4.567)--(7.713,4.570)--(7.718,4.574)--(7.723,4.577)--(7.729,4.581)%
  --(7.734,4.584)--(7.739,4.588)--(7.745,4.591)--(7.750,4.595)--(7.755,4.599)--(7.761,4.602)%
  --(7.766,4.606)--(7.771,4.609)--(7.777,4.613)--(7.782,4.616)--(7.787,4.620)--(7.793,4.623)%
  --(7.798,4.627)--(7.803,4.630)--(7.809,4.634)--(7.814,4.637)--(7.819,4.641)--(7.825,4.644)%
  --(7.830,4.648)--(7.835,4.651)--(7.841,4.655)--(7.846,4.658)--(7.851,4.662)--(7.856,4.665)%
  --(7.862,4.669)--(7.867,4.672)--(7.872,4.676)--(7.878,4.679)--(7.883,4.683)--(7.888,4.687)%
  --(7.894,4.690)--(7.899,4.694)--(7.904,4.697)--(7.910,4.701)--(7.915,4.704)--(7.920,4.708)%
  --(7.926,4.711)--(7.931,4.715)--(7.936,4.718)--(7.942,4.722);
\draw[gp path] (2.252,3.308)--(2.679,3.308);
\gpcolor{rgb color={1.000,0.000,0.000}}
\gppoint{gp mark 7}{(2.465,2.379)}
\gpcolor{rgb color={0.000,0.000,0.000}}
\node[gp node left,font={\fontsize{10pt}{12pt}\selectfont}] at (0.970,5.166) {\LARGE $B_y$};
\node[gp node left,font={\fontsize{10pt}{12pt}\selectfont}] at (5.954,5.166) {\large $\alpha = 2.95$};
\node[gp node left,font={\fontsize{10pt}{12pt}\selectfont}] at (2.821,3.308) {\large exact};
\node[gp node left,font={\fontsize{10pt}{12pt}\selectfont}] at (2.821,2.379) {\large converging};
%% coordinates of the plot area
\gpdefrectangularnode{gp plot 1}{\pgfpoint{0.828cm}{0.985cm}}{\pgfpoint{7.947cm}{5.631cm}}
\end{tikzpicture}
%% gnuplot variables
}
\end{tabular}
\caption{The approximate solution after the first flux correction of HLLD-CWM and exact r-solution to the full Riemann problem for the near-coplanar case of Test~5 with $4096$ grid points.  The compound wave is almost completely removed, except near $x=0.366$ and $x = 0.691$ where weak intermediate shocks remain.}
\label{fig:two_fast_coplanar_b_rsol_init}
\end{figure}
  
As shown in Figure~\ref{fig:coplanar_b_rsol_init}, the transition across the rotational discontinuity is initially unresolved.  The CWM procedure removes the compound wave from the solution except in this layer and leaves a deviation from the exact solution as the rotational discontinuity is crossed.   Because the deviation occurs where there is a change in sign of the tangential magnetic field, it needs to be detected, unless an exact solver is used, in which case the location is determined from the wave speed.  

A point is considered to be within the transition region of a 180\degree\, rotation if $| \alpha_{i+2} - \alpha_{i-2}| > 2.5 \text{ rad}$.  The state variables at these points must be adjusted in order to satisfy the jump conditions of a rotational discontinuity: $[\rho] = [v_n] = [p_g] = [B_{\perp}^2] = 0$, $\pm\sqrt{\rho}[v_y] = [B_y]$, $\pm\sqrt{\rho}[v_y] = [B_y]$ $\pm\sqrt{\rho}[v_z] = [B_z]$, and $[E]$ = $\pm\sqrt{\rho}[\mbf{v}\cdot\mbf{B}]$.  A straight-forward adjustment is to have points with a rotation angle less (greater) than $\pi/2$ be assigned the value of the upstream (downstream) value outside of the transition.  Although this approach produces acceptable results for the considered one-dimensional cases, it is not conservative because mass, momentum, and energy are removed.  However, the magnetic energy in the transition can be transferred from one tangential component to the other while maintaining conservation and satisfying the jump conditions.  We also note that the goal is to produce the correct states upstream and downstream of the rotational discontinuity, not to describe how energy is stored throughout the transition across the rotational discontinuity.    

%% The stored flux in the transition region must be reduced and redistributed to the neighboring cells.  We define a target solution, $\mbf{U}_{u,d}^*$, in the transition region based on the upstream and downstream states.  Averages of the upstream and downstream states are used to calculate the continuous variables, $\rho^*$, $(\rho v_n)^*$, and $p_g^*$.  The discontinuous variables, $(\rho\mbf{v}_{\perp})^*_{u,d}$ and $\mbf{B}^*_{\perp,u,d}$, are set to the upstream or downstream state depending on if the cell has undergone more than half a rotation.  Equation~\ref{eqn:energy} is used to calculate $E_{u,d}$.  If cells with the index $i$ are in the upstream transition region and cells with the index $j$ are in the downstream transition region, the total mass, momentum and energy still stored in the compound wave is given by the sums
%% \begin{gather*}
%% \delta \mbf{U}_u = \sum_i (\mbf{U}_u^* -  \mbf{U}^n_{i}), \\
%% \delta \mbf{U}_d = \sum_j (\mbf{U}_d^* - \mbf{U}^n_{j}). 
%% \end{gather*}
%% For the solution shown in Figure~\ref{fig:coplanar_b_rsol_init}, $0.5\%$ of the mass of the system and $0.5\%$ of the energy of the system is located in the transition region.  If the flow is smooth around the transition, the stored quantities are distributed to the surrounding cells.  The region surrounding the transition is considered smooth if the first derivative is constant.  In practice, the first derivative of two cells located at $i$ and $i+1$ is considered constant if $\text{max}(|\partial_x\mbf{U}_{i+1}| - |\partial_x\mbf{U}_i|) < 10^{-3}$.  The amount distributed to each cell is determined by the maximum difference in the continuous conserved variables of the upstream and downstream states, defined by $\delta_m = \text{max}(|\rho_u - \rho_d|,|(\rho v_n)_{u} - (\rho v_n)_{d}|)$.  The number of surrounding cells to which the stored quantities are distributed is given by $n = (\delta \mbf{U}_u + \delta \mbf{U}_d)/\delta_m$.  If $n$ is greater than the number of surrounding cells where the flow is smooth, then the amount deposited to each cell is increased to $\delta_m = (\delta \mbf{U}_u + \delta \mbf{U}_d)/n$.  The shock strength of the compound wave at the transition is reduced to the difference in the upstream and downstream states outside of the transition computed with HLLD-CWM.  

%-----------------------------------------------------------------
% Coplanar flux zoomed (512)
%-----------------------------------------------------------------
\begin{figure}[htbp] 
\begin{tabular}{cc}
\resizebox{0.5\linewidth}{!}{\tikzsetnextfilename{coplanar_b_crsol_00512_1}\begin{tikzpicture}[gnuplot]
%% generated with GNUPLOT 4.6p4 (Lua 5.1; terminal rev. 99, script rev. 100)
%% Fri 22 Aug 2014 11:56:08 AM EDT
\path (0.000,0.000) rectangle (8.500,6.000);
\gpfill{rgb color={1.000,1.000,1.000}} (1.196,0.985)--(7.946,0.985)--(7.946,5.630)--(1.196,5.630)--cycle;
\gpcolor{color=gp lt color border}
\gpsetlinetype{gp lt border}
\gpsetlinewidth{1.00}
\draw[gp path] (1.196,0.985)--(1.196,5.630)--(7.946,5.630)--(7.946,0.985)--cycle;
\gpcolor{color=gp lt color axes}
\gpsetlinetype{gp lt axes}
\gpsetlinewidth{2.00}
\draw[gp path] (1.196,0.985)--(7.947,0.985);
\gpcolor{color=gp lt color border}
\gpsetlinetype{gp lt border}
\draw[gp path] (1.196,0.985)--(1.268,0.985);
\draw[gp path] (7.947,0.985)--(7.875,0.985);
\gpcolor{rgb color={0.000,0.000,0.000}}
\node[gp node right,font={\fontsize{10pt}{12pt}\selectfont}] at (1.012,0.985) {0.6};
\gpcolor{color=gp lt color axes}
\gpsetlinetype{gp lt axes}
\draw[gp path] (1.196,1.759)--(7.947,1.759);
\gpcolor{color=gp lt color border}
\gpsetlinetype{gp lt border}
\draw[gp path] (1.196,1.759)--(1.268,1.759);
\draw[gp path] (7.947,1.759)--(7.875,1.759);
\gpcolor{rgb color={0.000,0.000,0.000}}
\node[gp node right,font={\fontsize{10pt}{12pt}\selectfont}] at (1.012,1.759) {0.65};
\gpcolor{color=gp lt color axes}
\gpsetlinetype{gp lt axes}
\draw[gp path] (1.196,2.534)--(7.947,2.534);
\gpcolor{color=gp lt color border}
\gpsetlinetype{gp lt border}
\draw[gp path] (1.196,2.534)--(1.268,2.534);
\draw[gp path] (7.947,2.534)--(7.875,2.534);
\gpcolor{rgb color={0.000,0.000,0.000}}
\node[gp node right,font={\fontsize{10pt}{12pt}\selectfont}] at (1.012,2.534) {0.7};
\gpcolor{color=gp lt color axes}
\gpsetlinetype{gp lt axes}
\draw[gp path] (1.196,3.308)--(7.947,3.308);
\gpcolor{color=gp lt color border}
\gpsetlinetype{gp lt border}
\draw[gp path] (1.196,3.308)--(1.268,3.308);
\draw[gp path] (7.947,3.308)--(7.875,3.308);
\gpcolor{rgb color={0.000,0.000,0.000}}
\node[gp node right,font={\fontsize{10pt}{12pt}\selectfont}] at (1.012,3.308) {0.75};
\gpcolor{color=gp lt color axes}
\gpsetlinetype{gp lt axes}
\draw[gp path] (1.196,4.082)--(7.947,4.082);
\gpcolor{color=gp lt color border}
\gpsetlinetype{gp lt border}
\draw[gp path] (1.196,4.082)--(1.268,4.082);
\draw[gp path] (7.947,4.082)--(7.875,4.082);
\gpcolor{rgb color={0.000,0.000,0.000}}
\node[gp node right,font={\fontsize{10pt}{12pt}\selectfont}] at (1.012,4.082) {0.8};
\gpcolor{color=gp lt color axes}
\gpsetlinetype{gp lt axes}
\draw[gp path] (1.196,4.857)--(7.947,4.857);
\gpcolor{color=gp lt color border}
\gpsetlinetype{gp lt border}
\draw[gp path] (1.196,4.857)--(1.268,4.857);
\draw[gp path] (7.947,4.857)--(7.875,4.857);
\gpcolor{rgb color={0.000,0.000,0.000}}
\node[gp node right,font={\fontsize{10pt}{12pt}\selectfont}] at (1.012,4.857) {0.85};
\gpcolor{color=gp lt color axes}
\gpsetlinetype{gp lt axes}
\draw[gp path] (1.196,5.631)--(7.947,5.631);
\gpcolor{color=gp lt color border}
\gpsetlinetype{gp lt border}
\draw[gp path] (1.196,5.631)--(1.268,5.631);
\draw[gp path] (7.947,5.631)--(7.875,5.631);
\gpcolor{rgb color={0.000,0.000,0.000}}
\node[gp node right,font={\fontsize{10pt}{12pt}\selectfont}] at (1.012,5.631) {0.9};
\gpcolor{color=gp lt color axes}
\gpsetlinetype{gp lt axes}
\draw[gp path] (1.196,0.985)--(1.196,5.631);
\gpcolor{color=gp lt color border}
\gpsetlinetype{gp lt border}
\draw[gp path] (1.196,0.985)--(1.196,1.057);
\draw[gp path] (1.196,5.631)--(1.196,5.559);
\gpcolor{rgb color={0.000,0.000,0.000}}
\node[gp node center,font={\fontsize{10pt}{12pt}\selectfont}] at (1.196,0.677) {0.2};
\gpcolor{color=gp lt color axes}
\gpsetlinetype{gp lt axes}
\draw[gp path] (2.321,0.985)--(2.321,5.631);
\gpcolor{color=gp lt color border}
\gpsetlinetype{gp lt border}
\draw[gp path] (2.321,0.985)--(2.321,1.057);
\draw[gp path] (2.321,5.631)--(2.321,5.559);
\gpcolor{rgb color={0.000,0.000,0.000}}
\node[gp node center,font={\fontsize{10pt}{12pt}\selectfont}] at (2.321,0.677) {0.25};
\gpcolor{color=gp lt color axes}
\gpsetlinetype{gp lt axes}
\draw[gp path] (3.446,0.985)--(3.446,5.631);
\gpcolor{color=gp lt color border}
\gpsetlinetype{gp lt border}
\draw[gp path] (3.446,0.985)--(3.446,1.057);
\draw[gp path] (3.446,5.631)--(3.446,5.559);
\gpcolor{rgb color={0.000,0.000,0.000}}
\node[gp node center,font={\fontsize{10pt}{12pt}\selectfont}] at (3.446,0.677) {0.3};
\gpcolor{color=gp lt color axes}
\gpsetlinetype{gp lt axes}
\draw[gp path] (4.572,0.985)--(4.572,5.631);
\gpcolor{color=gp lt color border}
\gpsetlinetype{gp lt border}
\draw[gp path] (4.572,0.985)--(4.572,1.057);
\draw[gp path] (4.572,5.631)--(4.572,5.559);
\gpcolor{rgb color={0.000,0.000,0.000}}
\node[gp node center,font={\fontsize{10pt}{12pt}\selectfont}] at (4.572,0.677) {0.35};
\gpcolor{color=gp lt color axes}
\gpsetlinetype{gp lt axes}
\draw[gp path] (5.697,0.985)--(5.697,5.631);
\gpcolor{color=gp lt color border}
\gpsetlinetype{gp lt border}
\draw[gp path] (5.697,0.985)--(5.697,1.057);
\draw[gp path] (5.697,5.631)--(5.697,5.559);
\gpcolor{rgb color={0.000,0.000,0.000}}
\node[gp node center,font={\fontsize{10pt}{12pt}\selectfont}] at (5.697,0.677) {0.4};
\gpcolor{color=gp lt color axes}
\gpsetlinetype{gp lt axes}
\draw[gp path] (6.822,0.985)--(6.822,5.631);
\gpcolor{color=gp lt color border}
\gpsetlinetype{gp lt border}
\draw[gp path] (6.822,0.985)--(6.822,1.057);
\draw[gp path] (6.822,5.631)--(6.822,5.559);
\gpcolor{rgb color={0.000,0.000,0.000}}
\node[gp node center,font={\fontsize{10pt}{12pt}\selectfont}] at (6.822,0.677) {0.45};
\gpcolor{color=gp lt color axes}
\gpsetlinetype{gp lt axes}
\draw[gp path] (7.947,0.985)--(7.947,5.631);
\gpcolor{color=gp lt color border}
\gpsetlinetype{gp lt border}
\draw[gp path] (7.947,0.985)--(7.947,1.057);
\draw[gp path] (7.947,5.631)--(7.947,5.559);
\gpcolor{rgb color={0.000,0.000,0.000}}
\node[gp node center,font={\fontsize{10pt}{12pt}\selectfont}] at (7.947,0.677) {0.5};
\gpcolor{color=gp lt color border}
\draw[gp path] (1.196,5.631)--(1.196,0.985)--(7.947,0.985)--(7.947,5.631)--cycle;
\gpcolor{rgb color={0.000,0.000,0.000}}
\node[gp node center,font={\fontsize{10pt}{12pt}\selectfont}] at (4.571,0.215) {\large $x$};
\gpcolor{rgb color={0.502,0.502,0.502}}
\gpsetlinewidth{0.50}
\gpsetpointsize{2.67}
\gppoint{gp mark 7}{(1.244,4.123)}
\gppoint{gp mark 7}{(1.288,4.058)}
\gppoint{gp mark 7}{(1.332,3.993)}
\gppoint{gp mark 7}{(1.376,3.928)}
\gppoint{gp mark 7}{(1.420,3.863)}
\gppoint{gp mark 7}{(1.464,3.798)}
\gppoint{gp mark 7}{(1.508,3.734)}
\gppoint{gp mark 7}{(1.552,3.670)}
\gppoint{gp mark 7}{(1.596,3.606)}
\gppoint{gp mark 7}{(1.640,3.542)}
\gppoint{gp mark 7}{(1.684,3.479)}
\gppoint{gp mark 7}{(1.728,3.416)}
\gppoint{gp mark 7}{(1.772,3.353)}
\gppoint{gp mark 7}{(1.816,3.290)}
\gppoint{gp mark 7}{(1.860,3.228)}
\gppoint{gp mark 7}{(1.904,3.165)}
\gppoint{gp mark 7}{(1.948,3.103)}
\gppoint{gp mark 7}{(1.992,3.041)}
\gppoint{gp mark 7}{(2.035,2.980)}
\gppoint{gp mark 7}{(2.079,2.919)}
\gppoint{gp mark 7}{(2.123,2.858)}
\gppoint{gp mark 7}{(2.167,2.797)}
\gppoint{gp mark 7}{(2.211,2.736)}
\gppoint{gp mark 7}{(2.255,2.676)}
\gppoint{gp mark 7}{(2.299,2.616)}
\gppoint{gp mark 7}{(2.343,2.556)}
\gppoint{gp mark 7}{(2.387,2.497)}
\gppoint{gp mark 7}{(2.431,2.438)}
\gppoint{gp mark 7}{(2.475,2.379)}
\gppoint{gp mark 7}{(2.519,2.320)}
\gppoint{gp mark 7}{(2.563,2.262)}
\gppoint{gp mark 7}{(2.607,2.205)}
\gppoint{gp mark 7}{(2.651,2.147)}
\gppoint{gp mark 7}{(2.695,2.091)}
\gppoint{gp mark 7}{(2.739,2.035)}
\gppoint{gp mark 7}{(2.783,1.980)}
\gppoint{gp mark 7}{(2.827,1.926)}
\gppoint{gp mark 7}{(2.871,1.873)}
\gppoint{gp mark 7}{(2.915,1.821)}
\gppoint{gp mark 7}{(2.958,1.770)}
\gppoint{gp mark 7}{(3.002,1.722)}
\gppoint{gp mark 7}{(3.046,1.682)}
\gppoint{gp mark 7}{(3.090,1.664)}
\gppoint{gp mark 7}{(3.134,1.661)}
\gppoint{gp mark 7}{(3.178,1.661)}
\gppoint{gp mark 7}{(3.222,1.670)}
\gppoint{gp mark 7}{(3.266,1.706)}
\gppoint{gp mark 7}{(3.310,1.763)}
\gppoint{gp mark 7}{(3.354,1.802)}
\gppoint{gp mark 7}{(3.398,1.823)}
\gppoint{gp mark 7}{(3.442,1.880)}
\gppoint{gp mark 7}{(3.486,2.177)}
\gppoint{gp mark 7}{(3.530,3.698)}
\gppoint{gp mark 7}{(3.574,4.486)}
\gppoint{gp mark 7}{(3.618,4.491)}
\gppoint{gp mark 7}{(3.662,4.451)}
\gppoint{gp mark 7}{(3.706,4.381)}
\gppoint{gp mark 7}{(3.750,4.277)}
\gppoint{gp mark 7}{(3.794,4.161)}
\gppoint{gp mark 7}{(3.838,4.059)}
\gppoint{gp mark 7}{(3.881,3.963)}
\gppoint{gp mark 7}{(3.925,3.849)}
\gppoint{gp mark 7}{(3.969,3.739)}
\gppoint{gp mark 7}{(4.013,3.648)}
\gppoint{gp mark 7}{(4.057,3.555)}
\gppoint{gp mark 7}{(4.101,3.463)}
\gppoint{gp mark 7}{(4.145,3.396)}
\gppoint{gp mark 7}{(4.189,3.365)}
\gppoint{gp mark 7}{(4.233,3.356)}
\gppoint{gp mark 7}{(4.277,3.354)}
\gppoint{gp mark 7}{(4.321,3.355)}
\gppoint{gp mark 7}{(4.365,3.358)}
\gppoint{gp mark 7}{(4.409,3.362)}
\gppoint{gp mark 7}{(4.453,3.363)}
\gppoint{gp mark 7}{(4.497,3.362)}
\gppoint{gp mark 7}{(4.541,3.362)}
\gppoint{gp mark 7}{(4.585,3.361)}
\gppoint{gp mark 7}{(4.629,3.360)}
\gppoint{gp mark 7}{(4.673,3.360)}
\gppoint{gp mark 7}{(4.717,3.359)}
\gppoint{gp mark 7}{(4.760,3.358)}
\gppoint{gp mark 7}{(4.804,3.356)}
\gppoint{gp mark 7}{(4.848,3.356)}
\gppoint{gp mark 7}{(4.892,3.355)}
\gppoint{gp mark 7}{(4.936,3.354)}
\gppoint{gp mark 7}{(4.980,3.354)}
\gppoint{gp mark 7}{(5.024,3.353)}
\gppoint{gp mark 7}{(5.068,3.353)}
\gppoint{gp mark 7}{(5.112,3.353)}
\gppoint{gp mark 7}{(5.156,3.352)}
\gppoint{gp mark 7}{(5.200,3.352)}
\gppoint{gp mark 7}{(5.244,3.352)}
\gppoint{gp mark 7}{(5.288,3.352)}
\gppoint{gp mark 7}{(5.332,3.351)}
\gppoint{gp mark 7}{(5.376,3.350)}
\gppoint{gp mark 7}{(5.420,3.350)}
\gppoint{gp mark 7}{(5.464,3.349)}
\gppoint{gp mark 7}{(5.508,3.349)}
\gppoint{gp mark 7}{(5.552,3.349)}
\gppoint{gp mark 7}{(5.596,3.349)}
\gppoint{gp mark 7}{(5.640,3.348)}
\gppoint{gp mark 7}{(5.683,3.348)}
\gppoint{gp mark 7}{(5.727,3.347)}
\gppoint{gp mark 7}{(5.771,3.347)}
\gppoint{gp mark 7}{(5.815,3.346)}
\gppoint{gp mark 7}{(5.859,3.346)}
\gppoint{gp mark 7}{(5.903,3.346)}
\gppoint{gp mark 7}{(5.947,3.345)}
\gppoint{gp mark 7}{(5.991,3.344)}
\gppoint{gp mark 7}{(6.035,3.344)}
\gppoint{gp mark 7}{(6.079,3.343)}
\gppoint{gp mark 7}{(6.123,3.343)}
\gppoint{gp mark 7}{(6.167,3.342)}
\gppoint{gp mark 7}{(6.211,3.341)}
\gppoint{gp mark 7}{(6.255,3.340)}
\gppoint{gp mark 7}{(6.299,3.339)}
\gppoint{gp mark 7}{(6.343,3.338)}
\gppoint{gp mark 7}{(6.387,3.338)}
\gppoint{gp mark 7}{(6.431,3.337)}
\gppoint{gp mark 7}{(6.475,3.336)}
\gppoint{gp mark 7}{(6.519,3.336)}
\gppoint{gp mark 7}{(6.563,3.336)}
\gppoint{gp mark 7}{(6.606,3.336)}
\gppoint{gp mark 7}{(6.650,3.336)}
\gppoint{gp mark 7}{(6.694,3.336)}
\gppoint{gp mark 7}{(6.738,3.336)}
\gppoint{gp mark 7}{(6.782,3.336)}
\gppoint{gp mark 7}{(6.826,3.337)}
\gppoint{gp mark 7}{(6.870,3.337)}
\gppoint{gp mark 7}{(6.914,3.338)}
\gppoint{gp mark 7}{(6.958,3.341)}
\gppoint{gp mark 7}{(7.002,3.344)}
\gppoint{gp mark 7}{(7.046,3.346)}
\gppoint{gp mark 7}{(7.090,3.352)}
\gppoint{gp mark 7}{(7.134,3.362)}
\gppoint{gp mark 7}{(7.178,3.374)}
\gppoint{gp mark 7}{(7.222,3.380)}
\gppoint{gp mark 7}{(7.266,3.380)}
\gppoint{gp mark 7}{(7.310,3.380)}
\gppoint{gp mark 7}{(7.354,3.375)}
\gppoint{gp mark 7}{(7.398,3.335)}
\gppoint{gp mark 7}{(7.442,3.020)}
\gppoint{gp mark 7}{(7.486,1.731)}
\gpcolor{rgb color={1.000,0.000,0.000}}
\gpsetpointsize{4.44}
\gppoint{gp mark 7}{(1.244,4.126)}
\gppoint{gp mark 7}{(1.288,4.062)}
\gppoint{gp mark 7}{(1.332,3.997)}
\gppoint{gp mark 7}{(1.376,3.933)}
\gppoint{gp mark 7}{(1.420,3.869)}
\gppoint{gp mark 7}{(1.464,3.805)}
\gppoint{gp mark 7}{(1.508,3.742)}
\gppoint{gp mark 7}{(1.552,3.679)}
\gppoint{gp mark 7}{(1.596,3.616)}
\gppoint{gp mark 7}{(1.640,3.553)}
\gppoint{gp mark 7}{(1.684,3.491)}
\gppoint{gp mark 7}{(1.728,3.429)}
\gppoint{gp mark 7}{(1.772,3.368)}
\gppoint{gp mark 7}{(1.816,3.307)}
\gppoint{gp mark 7}{(1.860,3.247)}
\gppoint{gp mark 7}{(1.904,3.187)}
\gppoint{gp mark 7}{(1.948,3.128)}
\gppoint{gp mark 7}{(1.992,3.069)}
\gppoint{gp mark 7}{(2.035,3.012)}
\gppoint{gp mark 7}{(2.079,2.956)}
\gppoint{gp mark 7}{(2.123,2.901)}
\gppoint{gp mark 7}{(2.167,2.847)}
\gppoint{gp mark 7}{(2.211,2.796)}
\gppoint{gp mark 7}{(2.255,2.746)}
\gppoint{gp mark 7}{(2.299,2.699)}
\gppoint{gp mark 7}{(2.343,2.655)}
\gppoint{gp mark 7}{(2.387,2.614)}
\gppoint{gp mark 7}{(2.431,2.577)}
\gppoint{gp mark 7}{(2.475,2.543)}
\gppoint{gp mark 7}{(2.519,2.512)}
\gppoint{gp mark 7}{(2.563,2.485)}
\gppoint{gp mark 7}{(2.607,2.461)}
\gppoint{gp mark 7}{(2.651,2.441)}
\gppoint{gp mark 7}{(2.695,2.427)}
\gppoint{gp mark 7}{(2.739,2.421)}
\gppoint{gp mark 7}{(2.783,2.420)}
\gppoint{gp mark 7}{(2.827,2.420)}
\gppoint{gp mark 7}{(2.871,2.420)}
\gppoint{gp mark 7}{(2.915,2.422)}
\gppoint{gp mark 7}{(2.958,2.425)}
\gppoint{gp mark 7}{(3.002,2.427)}
\gppoint{gp mark 7}{(3.046,2.427)}
\gppoint{gp mark 7}{(3.090,2.428)}
\gppoint{gp mark 7}{(3.134,2.428)}
\gppoint{gp mark 7}{(3.178,2.428)}
\gppoint{gp mark 7}{(3.222,2.429)}
\gppoint{gp mark 7}{(3.266,2.429)}
\gppoint{gp mark 7}{(3.310,2.429)}
\gppoint{gp mark 7}{(3.354,2.429)}
\gppoint{gp mark 7}{(3.398,2.432)}
\gppoint{gp mark 7}{(3.442,2.441)}
\gppoint{gp mark 7}{(3.486,2.947)}
\gppoint{gp mark 7}{(3.530,3.289)}
\gppoint{gp mark 7}{(3.574,3.025)}
\gppoint{gp mark 7}{(3.618,2.491)}
\gppoint{gp mark 7}{(3.662,2.542)}
\gppoint{gp mark 7}{(3.706,2.469)}
\gppoint{gp mark 7}{(3.750,2.437)}
\gppoint{gp mark 7}{(3.794,2.439)}
\gppoint{gp mark 7}{(3.838,2.442)}
\gppoint{gp mark 7}{(3.881,2.453)}
\gppoint{gp mark 7}{(3.925,2.465)}
\gppoint{gp mark 7}{(3.969,2.463)}
\gppoint{gp mark 7}{(4.013,2.444)}
\gppoint{gp mark 7}{(4.057,2.433)}
\gppoint{gp mark 7}{(4.101,2.435)}
\gppoint{gp mark 7}{(4.145,2.439)}
\gppoint{gp mark 7}{(4.189,2.445)}
\gppoint{gp mark 7}{(4.233,2.461)}
\gppoint{gp mark 7}{(4.277,2.550)}
\gppoint{gp mark 7}{(4.321,2.918)}
\gppoint{gp mark 7}{(4.365,3.440)}
\gppoint{gp mark 7}{(4.409,3.552)}
\gppoint{gp mark 7}{(4.453,3.574)}
\gppoint{gp mark 7}{(4.497,3.586)}
\gppoint{gp mark 7}{(4.541,3.587)}
\gppoint{gp mark 7}{(4.585,3.584)}
\gppoint{gp mark 7}{(4.629,3.578)}
\gppoint{gp mark 7}{(4.673,3.579)}
\gppoint{gp mark 7}{(4.717,3.587)}
\gppoint{gp mark 7}{(4.760,3.598)}
\gppoint{gp mark 7}{(4.804,3.601)}
\gppoint{gp mark 7}{(4.848,3.599)}
\gppoint{gp mark 7}{(4.892,3.587)}
\gppoint{gp mark 7}{(4.936,3.578)}
\gppoint{gp mark 7}{(4.980,3.578)}
\gppoint{gp mark 7}{(5.024,3.582)}
\gppoint{gp mark 7}{(5.068,3.593)}
\gppoint{gp mark 7}{(5.112,3.594)}
\gppoint{gp mark 7}{(5.156,3.593)}
\gppoint{gp mark 7}{(5.200,3.585)}
\gppoint{gp mark 7}{(5.244,3.580)}
\gppoint{gp mark 7}{(5.288,3.580)}
\gppoint{gp mark 7}{(5.332,3.582)}
\gppoint{gp mark 7}{(5.376,3.584)}
\gppoint{gp mark 7}{(5.420,3.583)}
\gppoint{gp mark 7}{(5.464,3.582)}
\gppoint{gp mark 7}{(5.508,3.582)}
\gppoint{gp mark 7}{(5.552,3.583)}
\gppoint{gp mark 7}{(5.596,3.589)}
\gppoint{gp mark 7}{(5.640,3.595)}
\gppoint{gp mark 7}{(5.683,3.595)}
\gppoint{gp mark 7}{(5.727,3.593)}
\gppoint{gp mark 7}{(5.771,3.586)}
\gppoint{gp mark 7}{(5.815,3.582)}
\gppoint{gp mark 7}{(5.859,3.582)}
\gppoint{gp mark 7}{(5.903,3.586)}
\gppoint{gp mark 7}{(5.947,3.597)}
\gppoint{gp mark 7}{(5.991,3.603)}
\gppoint{gp mark 7}{(6.035,3.605)}
\gppoint{gp mark 7}{(6.079,3.604)}
\gppoint{gp mark 7}{(6.123,3.596)}
\gppoint{gp mark 7}{(6.167,3.586)}
\gppoint{gp mark 7}{(6.211,3.582)}
\gppoint{gp mark 7}{(6.255,3.581)}
\gppoint{gp mark 7}{(6.299,3.583)}
\gppoint{gp mark 7}{(6.343,3.588)}
\gppoint{gp mark 7}{(6.387,3.589)}
\gppoint{gp mark 7}{(6.431,3.588)}
\gppoint{gp mark 7}{(6.475,3.586)}
\gppoint{gp mark 7}{(6.519,3.582)}
\gppoint{gp mark 7}{(6.563,3.580)}
\gppoint{gp mark 7}{(6.606,3.580)}
\gppoint{gp mark 7}{(6.650,3.580)}
\gppoint{gp mark 7}{(6.694,3.579)}
\gppoint{gp mark 7}{(6.738,3.577)}
\gppoint{gp mark 7}{(6.782,3.574)}
\gppoint{gp mark 7}{(6.826,3.571)}
\gppoint{gp mark 7}{(6.870,3.570)}
\gppoint{gp mark 7}{(6.914,3.570)}
\gppoint{gp mark 7}{(6.958,3.569)}
\gppoint{gp mark 7}{(7.002,3.566)}
\gppoint{gp mark 7}{(7.046,3.562)}
\gppoint{gp mark 7}{(7.090,3.561)}
\gppoint{gp mark 7}{(7.134,3.561)}
\gppoint{gp mark 7}{(7.178,3.564)}
\gppoint{gp mark 7}{(7.222,3.569)}
\gppoint{gp mark 7}{(7.266,3.570)}
\gppoint{gp mark 7}{(7.310,3.569)}
\gppoint{gp mark 7}{(7.354,3.554)}
\gppoint{gp mark 7}{(7.398,3.437)}
\gppoint{gp mark 7}{(7.442,2.674)}
\gpcolor{rgb color={0.000,0.000,0.000}}
\gpsetlinetype{gp lt plot 0}
\gpsetlinewidth{4.00}
\draw[gp path] (2.411,2.442)--(3.526,2.442);
\draw[gp path] (3.526,2.442)--(4.329,2.442);
\draw[gp path] (4.329,3.580)--(7.511,3.580);
\draw[gp path] (1.202,4.128)--(1.208,4.119)--(1.214,4.109)--(1.220,4.100)--(1.226,4.091)%
  --(1.232,4.082)--(1.238,4.073)--(1.244,4.064)--(1.250,4.055)--(1.256,4.046)--(1.262,4.037)%
  --(1.268,4.027)--(1.275,4.018)--(1.281,4.009)--(1.287,4.000)--(1.293,3.991)--(1.299,3.982)%
  --(1.305,3.973)--(1.311,3.964)--(1.317,3.955)--(1.323,3.946)--(1.329,3.937)--(1.335,3.928)%
  --(1.341,3.919)--(1.347,3.910)--(1.353,3.901)--(1.359,3.892)--(1.365,3.883)--(1.371,3.874)%
  --(1.377,3.865)--(1.383,3.856)--(1.389,3.847)--(1.395,3.838)--(1.402,3.829)--(1.408,3.820)%
  --(1.414,3.811)--(1.420,3.802)--(1.426,3.793)--(1.432,3.784)--(1.438,3.775)--(1.444,3.767)%
  --(1.450,3.758)--(1.456,3.749)--(1.462,3.740)--(1.468,3.731)--(1.474,3.722)--(1.480,3.713)%
  --(1.486,3.704)--(1.492,3.695)--(1.498,3.687)--(1.504,3.678)--(1.510,3.669)--(1.516,3.660)%
  --(1.522,3.651)--(1.528,3.642)--(1.535,3.633)--(1.541,3.625)--(1.547,3.616)--(1.553,3.607)%
  --(1.559,3.598)--(1.565,3.589)--(1.571,3.581)--(1.577,3.572)--(1.583,3.563)--(1.589,3.554)%
  --(1.595,3.546)--(1.601,3.537)--(1.607,3.528)--(1.613,3.519)--(1.619,3.511)--(1.625,3.502)%
  --(1.631,3.493)--(1.637,3.484)--(1.643,3.476)--(1.649,3.467)--(1.655,3.458)--(1.662,3.450)%
  --(1.668,3.441)--(1.674,3.432)--(1.680,3.423)--(1.686,3.415)--(1.692,3.406)--(1.698,3.398)%
  --(1.704,3.389)--(1.710,3.380)--(1.716,3.372)--(1.722,3.363)--(1.728,3.354)--(1.734,3.346)%
  --(1.740,3.337)--(1.746,3.328)--(1.752,3.320)--(1.758,3.311)--(1.764,3.303)--(1.770,3.294)%
  --(1.776,3.286)--(1.782,3.277)--(1.788,3.268)--(1.795,3.260)--(1.801,3.251)--(1.807,3.243)%
  --(1.813,3.234)--(1.819,3.226)--(1.825,3.217)--(1.831,3.209)--(1.837,3.200)--(1.843,3.192)%
  --(1.849,3.183)--(1.855,3.175)--(1.861,3.166)--(1.867,3.158)--(1.873,3.149)--(1.879,3.141)%
  --(1.885,3.132)--(1.891,3.124)--(1.897,3.116)--(1.903,3.107)--(1.909,3.099)--(1.915,3.090)%
  --(1.922,3.082)--(1.928,3.074)--(1.934,3.065)--(1.940,3.057)--(1.946,3.048)--(1.952,3.040)%
  --(1.958,3.032)--(1.964,3.023)--(1.970,3.015)--(1.976,3.007)--(1.982,2.998)--(1.988,2.990)%
  --(1.994,2.982)--(2.000,2.973)--(2.006,2.965)--(2.012,2.957)--(2.018,2.948)--(2.024,2.940)%
  --(2.030,2.932)--(2.036,2.924)--(2.042,2.915)--(2.048,2.907)--(2.055,2.899)--(2.061,2.891)%
  --(2.067,2.882)--(2.073,2.874)--(2.079,2.866)--(2.085,2.858)--(2.091,2.850)--(2.097,2.841)%
  --(2.103,2.833)--(2.109,2.825)--(2.115,2.817)--(2.121,2.809)--(2.127,2.800)--(2.133,2.792)%
  --(2.139,2.784)--(2.145,2.776)--(2.151,2.768)--(2.157,2.760)--(2.163,2.752)--(2.169,2.744)%
  --(2.175,2.736)--(2.182,2.727)--(2.188,2.719)--(2.194,2.711)--(2.200,2.703)--(2.206,2.695)%
  --(2.212,2.687)--(2.218,2.679)--(2.224,2.671)--(2.230,2.663)--(2.236,2.655)--(2.242,2.647)%
  --(2.248,2.639)--(2.254,2.631)--(2.260,2.623)--(2.266,2.615)--(2.272,2.607)--(2.278,2.599)%
  --(2.284,2.591)--(2.290,2.583)--(2.296,2.575)--(2.302,2.567)--(2.308,2.560)--(2.315,2.552)%
  --(2.321,2.544)--(2.327,2.536)--(2.333,2.528)--(2.339,2.520)--(2.345,2.512)--(2.351,2.504)%
  --(2.357,2.497)--(2.363,2.489)--(2.369,2.481)--(2.375,2.473)--(2.381,2.465)--(2.387,2.457)%
  --(2.393,2.450)--(2.399,2.442)--(2.405,2.434)--(2.411,2.442);
\draw[gp path] (4.329,2.442)--(4.329,3.580);
\draw[gp path] (7.511,3.580)--(7.511,0.985);
\node[gp node left,font={\fontsize{10pt}{12pt}\selectfont}] at (1.421,5.244) {\LARGE $\rho$};
\node[gp node left,font={\fontsize{10pt}{12pt}\selectfont}] at (6.147,5.244) {\large $\alpha = 3.0$};
%% coordinates of the plot area
\gpdefrectangularnode{gp plot 1}{\pgfpoint{1.196cm}{0.985cm}}{\pgfpoint{7.947cm}{5.631cm}}
\end{tikzpicture}
%% gnuplot variables
} & 
\resizebox{0.5\linewidth}{!}{\tikzsetnextfilename{coplanar_b_crsol_00512_6}\begin{tikzpicture}[gnuplot]
%% generated with GNUPLOT 4.6p4 (Lua 5.1; terminal rev. 99, script rev. 100)
%% Fri 22 Aug 2014 11:56:08 AM EDT
\path (0.000,0.000) rectangle (8.500,6.000);
\gpfill{rgb color={1.000,1.000,1.000}} (1.196,0.985)--(7.946,0.985)--(7.946,5.630)--(1.196,5.630)--cycle;
\gpcolor{color=gp lt color border}
\gpsetlinetype{gp lt border}
\gpsetlinewidth{1.00}
\draw[gp path] (1.196,0.985)--(1.196,5.630)--(7.946,5.630)--(7.946,0.985)--cycle;
\gpcolor{color=gp lt color axes}
\gpsetlinetype{gp lt axes}
\gpsetlinewidth{2.00}
\draw[gp path] (1.196,0.985)--(7.947,0.985);
\gpcolor{color=gp lt color border}
\gpsetlinetype{gp lt border}
\draw[gp path] (1.196,0.985)--(1.268,0.985);
\draw[gp path] (7.947,0.985)--(7.875,0.985);
\gpcolor{rgb color={0.000,0.000,0.000}}
\node[gp node right,font={\fontsize{10pt}{12pt}\selectfont}] at (1.012,0.985) {-0.4};
\gpcolor{color=gp lt color axes}
\gpsetlinetype{gp lt axes}
\draw[gp path] (1.196,1.759)--(7.947,1.759);
\gpcolor{color=gp lt color border}
\gpsetlinetype{gp lt border}
\draw[gp path] (1.196,1.759)--(1.268,1.759);
\draw[gp path] (7.947,1.759)--(7.875,1.759);
\gpcolor{rgb color={0.000,0.000,0.000}}
\node[gp node right,font={\fontsize{10pt}{12pt}\selectfont}] at (1.012,1.759) {-0.2};
\gpcolor{color=gp lt color axes}
\gpsetlinetype{gp lt axes}
\draw[gp path] (1.196,2.534)--(7.947,2.534);
\gpcolor{color=gp lt color border}
\gpsetlinetype{gp lt border}
\draw[gp path] (1.196,2.534)--(1.268,2.534);
\draw[gp path] (7.947,2.534)--(7.875,2.534);
\gpcolor{rgb color={0.000,0.000,0.000}}
\node[gp node right,font={\fontsize{10pt}{12pt}\selectfont}] at (1.012,2.534) {0};
\gpcolor{color=gp lt color axes}
\gpsetlinetype{gp lt axes}
\draw[gp path] (1.196,3.308)--(7.947,3.308);
\gpcolor{color=gp lt color border}
\gpsetlinetype{gp lt border}
\draw[gp path] (1.196,3.308)--(1.268,3.308);
\draw[gp path] (7.947,3.308)--(7.875,3.308);
\gpcolor{rgb color={0.000,0.000,0.000}}
\node[gp node right,font={\fontsize{10pt}{12pt}\selectfont}] at (1.012,3.308) {0.2};
\gpcolor{color=gp lt color axes}
\gpsetlinetype{gp lt axes}
\draw[gp path] (1.196,4.082)--(7.947,4.082);
\gpcolor{color=gp lt color border}
\gpsetlinetype{gp lt border}
\draw[gp path] (1.196,4.082)--(1.268,4.082);
\draw[gp path] (7.947,4.082)--(7.875,4.082);
\gpcolor{rgb color={0.000,0.000,0.000}}
\node[gp node right,font={\fontsize{10pt}{12pt}\selectfont}] at (1.012,4.082) {0.4};
\gpcolor{color=gp lt color axes}
\gpsetlinetype{gp lt axes}
\draw[gp path] (1.196,4.857)--(7.947,4.857);
\gpcolor{color=gp lt color border}
\gpsetlinetype{gp lt border}
\draw[gp path] (1.196,4.857)--(1.268,4.857);
\draw[gp path] (7.947,4.857)--(7.875,4.857);
\gpcolor{rgb color={0.000,0.000,0.000}}
\node[gp node right,font={\fontsize{10pt}{12pt}\selectfont}] at (1.012,4.857) {0.6};
\gpcolor{color=gp lt color axes}
\gpsetlinetype{gp lt axes}
\draw[gp path] (1.196,5.631)--(7.947,5.631);
\gpcolor{color=gp lt color border}
\gpsetlinetype{gp lt border}
\draw[gp path] (1.196,5.631)--(1.268,5.631);
\draw[gp path] (7.947,5.631)--(7.875,5.631);
\gpcolor{rgb color={0.000,0.000,0.000}}
\node[gp node right,font={\fontsize{10pt}{12pt}\selectfont}] at (1.012,5.631) {0.8};
\gpcolor{color=gp lt color axes}
\gpsetlinetype{gp lt axes}
\draw[gp path] (1.196,0.985)--(1.196,5.631);
\gpcolor{color=gp lt color border}
\gpsetlinetype{gp lt border}
\draw[gp path] (1.196,0.985)--(1.196,1.057);
\draw[gp path] (1.196,5.631)--(1.196,5.559);
\gpcolor{rgb color={0.000,0.000,0.000}}
\node[gp node center,font={\fontsize{10pt}{12pt}\selectfont}] at (1.196,0.677) {0.2};
\gpcolor{color=gp lt color axes}
\gpsetlinetype{gp lt axes}
\draw[gp path] (2.321,0.985)--(2.321,5.631);
\gpcolor{color=gp lt color border}
\gpsetlinetype{gp lt border}
\draw[gp path] (2.321,0.985)--(2.321,1.057);
\draw[gp path] (2.321,5.631)--(2.321,5.559);
\gpcolor{rgb color={0.000,0.000,0.000}}
\node[gp node center,font={\fontsize{10pt}{12pt}\selectfont}] at (2.321,0.677) {0.25};
\gpcolor{color=gp lt color axes}
\gpsetlinetype{gp lt axes}
\draw[gp path] (3.446,0.985)--(3.446,5.631);
\gpcolor{color=gp lt color border}
\gpsetlinetype{gp lt border}
\draw[gp path] (3.446,0.985)--(3.446,1.057);
\draw[gp path] (3.446,5.631)--(3.446,5.559);
\gpcolor{rgb color={0.000,0.000,0.000}}
\node[gp node center,font={\fontsize{10pt}{12pt}\selectfont}] at (3.446,0.677) {0.3};
\gpcolor{color=gp lt color axes}
\gpsetlinetype{gp lt axes}
\draw[gp path] (4.572,0.985)--(4.572,5.631);
\gpcolor{color=gp lt color border}
\gpsetlinetype{gp lt border}
\draw[gp path] (4.572,0.985)--(4.572,1.057);
\draw[gp path] (4.572,5.631)--(4.572,5.559);
\gpcolor{rgb color={0.000,0.000,0.000}}
\node[gp node center,font={\fontsize{10pt}{12pt}\selectfont}] at (4.572,0.677) {0.35};
\gpcolor{color=gp lt color axes}
\gpsetlinetype{gp lt axes}
\draw[gp path] (5.697,0.985)--(5.697,5.631);
\gpcolor{color=gp lt color border}
\gpsetlinetype{gp lt border}
\draw[gp path] (5.697,0.985)--(5.697,1.057);
\draw[gp path] (5.697,5.631)--(5.697,5.559);
\gpcolor{rgb color={0.000,0.000,0.000}}
\node[gp node center,font={\fontsize{10pt}{12pt}\selectfont}] at (5.697,0.677) {0.4};
\gpcolor{color=gp lt color axes}
\gpsetlinetype{gp lt axes}
\draw[gp path] (6.822,0.985)--(6.822,5.631);
\gpcolor{color=gp lt color border}
\gpsetlinetype{gp lt border}
\draw[gp path] (6.822,0.985)--(6.822,1.057);
\draw[gp path] (6.822,5.631)--(6.822,5.559);
\gpcolor{rgb color={0.000,0.000,0.000}}
\node[gp node center,font={\fontsize{10pt}{12pt}\selectfont}] at (6.822,0.677) {0.45};
\gpcolor{color=gp lt color axes}
\gpsetlinetype{gp lt axes}
\draw[gp path] (7.947,0.985)--(7.947,5.631);
\gpcolor{color=gp lt color border}
\gpsetlinetype{gp lt border}
\draw[gp path] (7.947,0.985)--(7.947,1.057);
\draw[gp path] (7.947,5.631)--(7.947,5.559);
\gpcolor{rgb color={0.000,0.000,0.000}}
\node[gp node center,font={\fontsize{10pt}{12pt}\selectfont}] at (7.947,0.677) {0.5};
\gpcolor{color=gp lt color border}
\draw[gp path] (1.196,5.631)--(1.196,0.985)--(7.947,0.985)--(7.947,5.631)--cycle;
\gpcolor{rgb color={0.000,0.000,0.000}}
\node[gp node center,font={\fontsize{10pt}{12pt}\selectfont}] at (4.571,0.215) {\large $x$};
\gpcolor{rgb color={0.502,0.502,0.502}}
\gpsetlinewidth{0.50}
\gpsetpointsize{2.67}
\gppoint{gp mark 7}{(1.244,4.616)}
\gppoint{gp mark 7}{(1.288,4.594)}
\gppoint{gp mark 7}{(1.332,4.573)}
\gppoint{gp mark 7}{(1.376,4.552)}
\gppoint{gp mark 7}{(1.420,4.531)}
\gppoint{gp mark 7}{(1.464,4.510)}
\gppoint{gp mark 7}{(1.508,4.489)}
\gppoint{gp mark 7}{(1.552,4.467)}
\gppoint{gp mark 7}{(1.596,4.446)}
\gppoint{gp mark 7}{(1.640,4.425)}
\gppoint{gp mark 7}{(1.684,4.403)}
\gppoint{gp mark 7}{(1.728,4.382)}
\gppoint{gp mark 7}{(1.772,4.360)}
\gppoint{gp mark 7}{(1.816,4.339)}
\gppoint{gp mark 7}{(1.860,4.317)}
\gppoint{gp mark 7}{(1.904,4.296)}
\gppoint{gp mark 7}{(1.948,4.274)}
\gppoint{gp mark 7}{(1.992,4.252)}
\gppoint{gp mark 7}{(2.035,4.231)}
\gppoint{gp mark 7}{(2.079,4.209)}
\gppoint{gp mark 7}{(2.123,4.187)}
\gppoint{gp mark 7}{(2.167,4.165)}
\gppoint{gp mark 7}{(2.211,4.143)}
\gppoint{gp mark 7}{(2.255,4.121)}
\gppoint{gp mark 7}{(2.299,4.099)}
\gppoint{gp mark 7}{(2.343,4.077)}
\gppoint{gp mark 7}{(2.387,4.055)}
\gppoint{gp mark 7}{(2.431,4.033)}
\gppoint{gp mark 7}{(2.475,4.010)}
\gppoint{gp mark 7}{(2.519,3.988)}
\gppoint{gp mark 7}{(2.563,3.965)}
\gppoint{gp mark 7}{(2.607,3.943)}
\gppoint{gp mark 7}{(2.651,3.920)}
\gppoint{gp mark 7}{(2.695,3.898)}
\gppoint{gp mark 7}{(2.739,3.875)}
\gppoint{gp mark 7}{(2.783,3.853)}
\gppoint{gp mark 7}{(2.827,3.831)}
\gppoint{gp mark 7}{(2.871,3.809)}
\gppoint{gp mark 7}{(2.915,3.787)}
\gppoint{gp mark 7}{(2.958,3.766)}
\gppoint{gp mark 7}{(3.002,3.745)}
\gppoint{gp mark 7}{(3.046,3.728)}
\gppoint{gp mark 7}{(3.090,3.720)}
\gppoint{gp mark 7}{(3.134,3.718)}
\gppoint{gp mark 7}{(3.178,3.719)}
\gppoint{gp mark 7}{(3.222,3.722)}
\gppoint{gp mark 7}{(3.266,3.738)}
\gppoint{gp mark 7}{(3.310,3.762)}
\gppoint{gp mark 7}{(3.354,3.780)}
\gppoint{gp mark 7}{(3.398,3.786)}
\gppoint{gp mark 7}{(3.442,3.771)}
\gppoint{gp mark 7}{(3.486,3.571)}
\gppoint{gp mark 7}{(3.530,2.552)}
\gppoint{gp mark 7}{(3.574,1.717)}
\gppoint{gp mark 7}{(3.618,1.617)}
\gppoint{gp mark 7}{(3.662,1.600)}
\gppoint{gp mark 7}{(3.706,1.578)}
\gppoint{gp mark 7}{(3.750,1.549)}
\gppoint{gp mark 7}{(3.794,1.522)}
\gppoint{gp mark 7}{(3.838,1.491)}
\gppoint{gp mark 7}{(3.881,1.457)}
\gppoint{gp mark 7}{(3.925,1.432)}
\gppoint{gp mark 7}{(3.969,1.408)}
\gppoint{gp mark 7}{(4.013,1.381)}
\gppoint{gp mark 7}{(4.057,1.357)}
\gppoint{gp mark 7}{(4.101,1.338)}
\gppoint{gp mark 7}{(4.145,1.323)}
\gppoint{gp mark 7}{(4.189,1.314)}
\gppoint{gp mark 7}{(4.233,1.312)}
\gppoint{gp mark 7}{(4.277,1.312)}
\gppoint{gp mark 7}{(4.321,1.312)}
\gppoint{gp mark 7}{(4.365,1.313)}
\gppoint{gp mark 7}{(4.409,1.314)}
\gppoint{gp mark 7}{(4.453,1.315)}
\gppoint{gp mark 7}{(4.497,1.315)}
\gppoint{gp mark 7}{(4.541,1.315)}
\gppoint{gp mark 7}{(4.585,1.315)}
\gppoint{gp mark 7}{(4.629,1.314)}
\gppoint{gp mark 7}{(4.673,1.314)}
\gppoint{gp mark 7}{(4.717,1.314)}
\gppoint{gp mark 7}{(4.760,1.314)}
\gppoint{gp mark 7}{(4.804,1.314)}
\gppoint{gp mark 7}{(4.848,1.314)}
\gppoint{gp mark 7}{(4.892,1.314)}
\gppoint{gp mark 7}{(4.936,1.314)}
\gppoint{gp mark 7}{(4.980,1.314)}
\gppoint{gp mark 7}{(5.024,1.314)}
\gppoint{gp mark 7}{(5.068,1.314)}
\gppoint{gp mark 7}{(5.112,1.314)}
\gppoint{gp mark 7}{(5.156,1.314)}
\gppoint{gp mark 7}{(5.200,1.314)}
\gppoint{gp mark 7}{(5.244,1.314)}
\gppoint{gp mark 7}{(5.288,1.314)}
\gppoint{gp mark 7}{(5.332,1.314)}
\gppoint{gp mark 7}{(5.376,1.314)}
\gppoint{gp mark 7}{(5.420,1.314)}
\gppoint{gp mark 7}{(5.464,1.314)}
\gppoint{gp mark 7}{(5.508,1.314)}
\gppoint{gp mark 7}{(5.552,1.314)}
\gppoint{gp mark 7}{(5.596,1.314)}
\gppoint{gp mark 7}{(5.640,1.314)}
\gppoint{gp mark 7}{(5.683,1.314)}
\gppoint{gp mark 7}{(5.727,1.314)}
\gppoint{gp mark 7}{(5.771,1.314)}
\gppoint{gp mark 7}{(5.815,1.314)}
\gppoint{gp mark 7}{(5.859,1.314)}
\gppoint{gp mark 7}{(5.903,1.314)}
\gppoint{gp mark 7}{(5.947,1.314)}
\gppoint{gp mark 7}{(5.991,1.314)}
\gppoint{gp mark 7}{(6.035,1.314)}
\gppoint{gp mark 7}{(6.079,1.314)}
\gppoint{gp mark 7}{(6.123,1.314)}
\gppoint{gp mark 7}{(6.167,1.314)}
\gppoint{gp mark 7}{(6.211,1.314)}
\gppoint{gp mark 7}{(6.255,1.314)}
\gppoint{gp mark 7}{(6.299,1.314)}
\gppoint{gp mark 7}{(6.343,1.314)}
\gppoint{gp mark 7}{(6.387,1.314)}
\gppoint{gp mark 7}{(6.431,1.314)}
\gppoint{gp mark 7}{(6.475,1.314)}
\gppoint{gp mark 7}{(6.519,1.314)}
\gppoint{gp mark 7}{(6.563,1.314)}
\gppoint{gp mark 7}{(6.606,1.314)}
\gppoint{gp mark 7}{(6.650,1.314)}
\gppoint{gp mark 7}{(6.694,1.314)}
\gppoint{gp mark 7}{(6.738,1.314)}
\gppoint{gp mark 7}{(6.782,1.314)}
\gppoint{gp mark 7}{(6.826,1.314)}
\gppoint{gp mark 7}{(6.870,1.314)}
\gppoint{gp mark 7}{(6.914,1.314)}
\gppoint{gp mark 7}{(6.958,1.314)}
\gppoint{gp mark 7}{(7.002,1.314)}
\gppoint{gp mark 7}{(7.046,1.314)}
\gppoint{gp mark 7}{(7.090,1.314)}
\gppoint{gp mark 7}{(7.134,1.314)}
\gppoint{gp mark 7}{(7.178,1.314)}
\gppoint{gp mark 7}{(7.222,1.314)}
\gppoint{gp mark 7}{(7.266,1.314)}
\gppoint{gp mark 7}{(7.310,1.314)}
\gppoint{gp mark 7}{(7.354,1.314)}
\gppoint{gp mark 7}{(7.398,1.314)}
\gppoint{gp mark 7}{(7.442,1.314)}
\gppoint{gp mark 7}{(7.486,1.314)}
\gppoint{gp mark 7}{(7.529,1.314)}
\gppoint{gp mark 7}{(7.573,1.314)}
\gppoint{gp mark 7}{(7.617,1.314)}
\gppoint{gp mark 7}{(7.661,1.314)}
\gppoint{gp mark 7}{(7.705,1.314)}
\gppoint{gp mark 7}{(7.749,1.314)}
\gppoint{gp mark 7}{(7.793,1.314)}
\gppoint{gp mark 7}{(7.837,1.314)}
\gppoint{gp mark 7}{(7.881,1.314)}
\gppoint{gp mark 7}{(7.925,1.314)}
\gpcolor{rgb color={1.000,0.000,0.000}}
\gpsetpointsize{4.44}
\gppoint{gp mark 7}{(1.244,4.617)}
\gppoint{gp mark 7}{(1.288,4.596)}
\gppoint{gp mark 7}{(1.332,4.575)}
\gppoint{gp mark 7}{(1.376,4.554)}
\gppoint{gp mark 7}{(1.420,4.533)}
\gppoint{gp mark 7}{(1.464,4.512)}
\gppoint{gp mark 7}{(1.508,4.491)}
\gppoint{gp mark 7}{(1.552,4.470)}
\gppoint{gp mark 7}{(1.596,4.449)}
\gppoint{gp mark 7}{(1.640,4.428)}
\gppoint{gp mark 7}{(1.684,4.407)}
\gppoint{gp mark 7}{(1.728,4.386)}
\gppoint{gp mark 7}{(1.772,4.366)}
\gppoint{gp mark 7}{(1.816,4.345)}
\gppoint{gp mark 7}{(1.860,4.324)}
\gppoint{gp mark 7}{(1.904,4.303)}
\gppoint{gp mark 7}{(1.948,4.283)}
\gppoint{gp mark 7}{(1.992,4.262)}
\gppoint{gp mark 7}{(2.035,4.242)}
\gppoint{gp mark 7}{(2.079,4.222)}
\gppoint{gp mark 7}{(2.123,4.203)}
\gppoint{gp mark 7}{(2.167,4.183)}
\gppoint{gp mark 7}{(2.211,4.165)}
\gppoint{gp mark 7}{(2.255,4.147)}
\gppoint{gp mark 7}{(2.299,4.130)}
\gppoint{gp mark 7}{(2.343,4.114)}
\gppoint{gp mark 7}{(2.387,4.099)}
\gppoint{gp mark 7}{(2.431,4.085)}
\gppoint{gp mark 7}{(2.475,4.072)}
\gppoint{gp mark 7}{(2.519,4.061)}
\gppoint{gp mark 7}{(2.563,4.050)}
\gppoint{gp mark 7}{(2.607,4.041)}
\gppoint{gp mark 7}{(2.651,4.034)}
\gppoint{gp mark 7}{(2.695,4.029)}
\gppoint{gp mark 7}{(2.739,4.026)}
\gppoint{gp mark 7}{(2.783,4.026)}
\gppoint{gp mark 7}{(2.827,4.026)}
\gppoint{gp mark 7}{(2.871,4.026)}
\gppoint{gp mark 7}{(2.915,4.027)}
\gppoint{gp mark 7}{(2.958,4.028)}
\gppoint{gp mark 7}{(3.002,4.029)}
\gppoint{gp mark 7}{(3.046,4.029)}
\gppoint{gp mark 7}{(3.090,4.029)}
\gppoint{gp mark 7}{(3.134,4.029)}
\gppoint{gp mark 7}{(3.178,4.029)}
\gppoint{gp mark 7}{(3.222,4.029)}
\gppoint{gp mark 7}{(3.266,4.029)}
\gppoint{gp mark 7}{(3.310,4.029)}
\gppoint{gp mark 7}{(3.354,4.029)}
\gppoint{gp mark 7}{(3.398,4.028)}
\gppoint{gp mark 7}{(3.442,4.020)}
\gppoint{gp mark 7}{(3.486,3.992)}
\gppoint{gp mark 7}{(3.530,3.021)}
\gppoint{gp mark 7}{(3.574,1.091)}
\gppoint{gp mark 7}{(3.618,1.075)}
\gppoint{gp mark 7}{(3.662,1.102)}
\gppoint{gp mark 7}{(3.706,1.073)}
\gppoint{gp mark 7}{(3.750,1.077)}
\gppoint{gp mark 7}{(3.794,1.095)}
\gppoint{gp mark 7}{(3.838,1.094)}
\gppoint{gp mark 7}{(3.881,1.084)}
\gppoint{gp mark 7}{(3.925,1.084)}
\gppoint{gp mark 7}{(3.969,1.088)}
\gppoint{gp mark 7}{(4.013,1.088)}
\gppoint{gp mark 7}{(4.057,1.085)}
\gppoint{gp mark 7}{(4.101,1.082)}
\gppoint{gp mark 7}{(4.145,1.083)}
\gppoint{gp mark 7}{(4.189,1.084)}
\gppoint{gp mark 7}{(4.233,1.087)}
\gppoint{gp mark 7}{(4.277,1.103)}
\gppoint{gp mark 7}{(4.321,1.173)}
\gppoint{gp mark 7}{(4.365,1.284)}
\gppoint{gp mark 7}{(4.409,1.315)}
\gppoint{gp mark 7}{(4.453,1.316)}
\gppoint{gp mark 7}{(4.497,1.315)}
\gppoint{gp mark 7}{(4.541,1.314)}
\gppoint{gp mark 7}{(4.585,1.315)}
\gppoint{gp mark 7}{(4.629,1.316)}
\gppoint{gp mark 7}{(4.673,1.316)}
\gppoint{gp mark 7}{(4.717,1.316)}
\gppoint{gp mark 7}{(4.760,1.315)}
\gppoint{gp mark 7}{(4.804,1.314)}
\gppoint{gp mark 7}{(4.848,1.314)}
\gppoint{gp mark 7}{(4.892,1.315)}
\gppoint{gp mark 7}{(4.936,1.316)}
\gppoint{gp mark 7}{(4.980,1.316)}
\gppoint{gp mark 7}{(5.024,1.315)}
\gppoint{gp mark 7}{(5.068,1.315)}
\gppoint{gp mark 7}{(5.112,1.315)}
\gppoint{gp mark 7}{(5.156,1.315)}
\gppoint{gp mark 7}{(5.200,1.315)}
\gppoint{gp mark 7}{(5.244,1.315)}
\gppoint{gp mark 7}{(5.288,1.315)}
\gppoint{gp mark 7}{(5.332,1.315)}
\gppoint{gp mark 7}{(5.376,1.316)}
\gppoint{gp mark 7}{(5.420,1.316)}
\gppoint{gp mark 7}{(5.464,1.315)}
\gppoint{gp mark 7}{(5.508,1.315)}
\gppoint{gp mark 7}{(5.552,1.314)}
\gppoint{gp mark 7}{(5.596,1.314)}
\gppoint{gp mark 7}{(5.640,1.315)}
\gppoint{gp mark 7}{(5.683,1.315)}
\gppoint{gp mark 7}{(5.727,1.315)}
\gppoint{gp mark 7}{(5.771,1.315)}
\gppoint{gp mark 7}{(5.815,1.315)}
\gppoint{gp mark 7}{(5.859,1.315)}
\gppoint{gp mark 7}{(5.903,1.315)}
\gppoint{gp mark 7}{(5.947,1.315)}
\gppoint{gp mark 7}{(5.991,1.315)}
\gppoint{gp mark 7}{(6.035,1.315)}
\gppoint{gp mark 7}{(6.079,1.315)}
\gppoint{gp mark 7}{(6.123,1.315)}
\gppoint{gp mark 7}{(6.167,1.315)}
\gppoint{gp mark 7}{(6.211,1.315)}
\gppoint{gp mark 7}{(6.255,1.315)}
\gppoint{gp mark 7}{(6.299,1.315)}
\gppoint{gp mark 7}{(6.343,1.315)}
\gppoint{gp mark 7}{(6.387,1.315)}
\gppoint{gp mark 7}{(6.431,1.315)}
\gppoint{gp mark 7}{(6.475,1.315)}
\gppoint{gp mark 7}{(6.519,1.315)}
\gppoint{gp mark 7}{(6.563,1.315)}
\gppoint{gp mark 7}{(6.606,1.315)}
\gppoint{gp mark 7}{(6.650,1.315)}
\gppoint{gp mark 7}{(6.694,1.315)}
\gppoint{gp mark 7}{(6.738,1.315)}
\gppoint{gp mark 7}{(6.782,1.315)}
\gppoint{gp mark 7}{(6.826,1.315)}
\gppoint{gp mark 7}{(6.870,1.315)}
\gppoint{gp mark 7}{(6.914,1.315)}
\gppoint{gp mark 7}{(6.958,1.315)}
\gppoint{gp mark 7}{(7.002,1.316)}
\gppoint{gp mark 7}{(7.046,1.316)}
\gppoint{gp mark 7}{(7.090,1.316)}
\gppoint{gp mark 7}{(7.134,1.315)}
\gppoint{gp mark 7}{(7.178,1.315)}
\gppoint{gp mark 7}{(7.222,1.315)}
\gppoint{gp mark 7}{(7.266,1.315)}
\gppoint{gp mark 7}{(7.310,1.315)}
\gppoint{gp mark 7}{(7.354,1.315)}
\gppoint{gp mark 7}{(7.398,1.315)}
\gppoint{gp mark 7}{(7.442,1.315)}
\gppoint{gp mark 7}{(7.486,1.315)}
\gppoint{gp mark 7}{(7.529,1.315)}
\gppoint{gp mark 7}{(7.573,1.315)}
\gppoint{gp mark 7}{(7.617,1.315)}
\gppoint{gp mark 7}{(7.661,1.315)}
\gppoint{gp mark 7}{(7.705,1.315)}
\gppoint{gp mark 7}{(7.749,1.315)}
\gppoint{gp mark 7}{(7.793,1.315)}
\gppoint{gp mark 7}{(7.837,1.315)}
\gppoint{gp mark 7}{(7.881,1.315)}
\gppoint{gp mark 7}{(7.925,1.315)}
\gpcolor{rgb color={0.000,0.000,0.000}}
\gpsetlinetype{gp lt plot 0}
\gpsetlinewidth{4.00}
\draw[gp path] (2.411,4.028)--(3.526,4.028);
\draw[gp path] (3.526,1.086)--(4.329,1.086);
\draw[gp path] (4.329,1.315)--(7.511,1.315);
\draw[gp path] (7.511,1.315)--(7.947,1.315);
\draw[gp path] (1.202,4.617)--(1.208,4.614)--(1.214,4.611)--(1.220,4.608)--(1.226,4.605)%
  --(1.232,4.602)--(1.238,4.599)--(1.244,4.596)--(1.250,4.593)--(1.256,4.591)--(1.262,4.588)%
  --(1.268,4.585)--(1.275,4.582)--(1.281,4.579)--(1.287,4.576)--(1.293,4.573)--(1.299,4.570)%
  --(1.305,4.567)--(1.311,4.564)--(1.317,4.561)--(1.323,4.558)--(1.329,4.555)--(1.335,4.552)%
  --(1.341,4.549)--(1.347,4.546)--(1.353,4.543)--(1.359,4.540)--(1.365,4.538)--(1.371,4.535)%
  --(1.377,4.532)--(1.383,4.529)--(1.389,4.526)--(1.395,4.523)--(1.402,4.520)--(1.408,4.517)%
  --(1.414,4.514)--(1.420,4.511)--(1.426,4.508)--(1.432,4.505)--(1.438,4.502)--(1.444,4.499)%
  --(1.450,4.496)--(1.456,4.493)--(1.462,4.490)--(1.468,4.488)--(1.474,4.485)--(1.480,4.482)%
  --(1.486,4.479)--(1.492,4.476)--(1.498,4.473)--(1.504,4.470)--(1.510,4.467)--(1.516,4.464)%
  --(1.522,4.461)--(1.528,4.458)--(1.535,4.455)--(1.541,4.452)--(1.547,4.449)--(1.553,4.446)%
  --(1.559,4.443)--(1.565,4.440)--(1.571,4.437)--(1.577,4.435)--(1.583,4.432)--(1.589,4.429)%
  --(1.595,4.426)--(1.601,4.423)--(1.607,4.420)--(1.613,4.417)--(1.619,4.414)--(1.625,4.411)%
  --(1.631,4.408)--(1.637,4.405)--(1.643,4.402)--(1.649,4.399)--(1.655,4.396)--(1.662,4.393)%
  --(1.668,4.390)--(1.674,4.387)--(1.680,4.384)--(1.686,4.382)--(1.692,4.379)--(1.698,4.376)%
  --(1.704,4.373)--(1.710,4.370)--(1.716,4.367)--(1.722,4.364)--(1.728,4.361)--(1.734,4.358)%
  --(1.740,4.355)--(1.746,4.352)--(1.752,4.349)--(1.758,4.346)--(1.764,4.343)--(1.770,4.340)%
  --(1.776,4.337)--(1.782,4.334)--(1.788,4.331)--(1.795,4.329)--(1.801,4.326)--(1.807,4.323)%
  --(1.813,4.320)--(1.819,4.317)--(1.825,4.314)--(1.831,4.311)--(1.837,4.308)--(1.843,4.305)%
  --(1.849,4.302)--(1.855,4.299)--(1.861,4.296)--(1.867,4.293)--(1.873,4.290)--(1.879,4.287)%
  --(1.885,4.284)--(1.891,4.281)--(1.897,4.278)--(1.903,4.276)--(1.909,4.273)--(1.915,4.270)%
  --(1.922,4.267)--(1.928,4.264)--(1.934,4.261)--(1.940,4.258)--(1.946,4.255)--(1.952,4.252)%
  --(1.958,4.249)--(1.964,4.246)--(1.970,4.243)--(1.976,4.240)--(1.982,4.237)--(1.988,4.234)%
  --(1.994,4.231)--(2.000,4.228)--(2.006,4.225)--(2.012,4.223)--(2.018,4.220)--(2.024,4.217)%
  --(2.030,4.214)--(2.036,4.211)--(2.042,4.208)--(2.048,4.205)--(2.055,4.202)--(2.061,4.199)%
  --(2.067,4.196)--(2.073,4.193)--(2.079,4.190)--(2.085,4.187)--(2.091,4.184)--(2.097,4.181)%
  --(2.103,4.178)--(2.109,4.175)--(2.115,4.172)--(2.121,4.170)--(2.127,4.167)--(2.133,4.164)%
  --(2.139,4.161)--(2.145,4.158)--(2.151,4.155)--(2.157,4.152)--(2.163,4.149)--(2.169,4.146)%
  --(2.175,4.143)--(2.182,4.140)--(2.188,4.137)--(2.194,4.134)--(2.200,4.131)--(2.206,4.128)%
  --(2.212,4.125)--(2.218,4.122)--(2.224,4.119)--(2.230,4.117)--(2.236,4.114)--(2.242,4.111)%
  --(2.248,4.108)--(2.254,4.105)--(2.260,4.102)--(2.266,4.099)--(2.272,4.096)--(2.278,4.093)%
  --(2.284,4.090)--(2.290,4.087)--(2.296,4.084)--(2.302,4.081)--(2.308,4.078)--(2.315,4.075)%
  --(2.321,4.072)--(2.327,4.069)--(2.333,4.067)--(2.339,4.064)--(2.345,4.061)--(2.351,4.058)%
  --(2.357,4.055)--(2.363,4.052)--(2.369,4.049)--(2.375,4.046)--(2.381,4.043)--(2.387,4.040)%
  --(2.393,4.037)--(2.399,4.034)--(2.405,4.031)--(2.411,4.028);
\draw[gp path] (3.526,4.028)--(3.526,1.086);
\draw[gp path] (4.329,1.086)--(4.329,1.315);
\draw[gp path] (3.896,3.695)--(4.572,3.695);
\gpcolor{rgb color={1.000,0.000,0.000}}
\gpsetlinewidth{0.50}
\gppoint{gp mark 7}{(4.234,2.921)}
\gpcolor{rgb color={0.502,0.502,0.502}}
\gppoint{gp mark 7}{(4.234,2.147)}
\gpcolor{rgb color={0.000,0.000,0.000}}
\node[gp node left,font={\fontsize{10pt}{12pt}\selectfont}] at (1.421,5.166) {\LARGE $B_y$};
\node[gp node left,font={\fontsize{10pt}{12pt}\selectfont}] at (6.147,5.166) {\large $\alpha = 3.0$};
\node[gp node left,font={\fontsize{10pt}{12pt}\selectfont}] at (4.797,3.695) {\large exact};
\node[gp node left,font={\fontsize{10pt}{12pt}\selectfont}] at (4.797,2.921) {\large HLLD-CWM};
\node[gp node left,font={\fontsize{10pt}{12pt}\selectfont}] at (4.797,2.147) {\large HLLD};
%% coordinates of the plot area
\gpdefrectangularnode{gp plot 1}{\pgfpoint{1.196cm}{0.985cm}}{\pgfpoint{7.947cm}{5.631cm}}
\end{tikzpicture}
%% gnuplot variables
} \\
\resizebox{0.5\linewidth}{!}{\tikzsetnextfilename{coplanar_a_crsol_00512_1}\begin{tikzpicture}[gnuplot]
%% generated with GNUPLOT 4.6p4 (Lua 5.1; terminal rev. 99, script rev. 100)
%% Fri 22 Aug 2014 12:01:03 PM EDT
\path (0.000,0.000) rectangle (8.500,6.000);
\gpfill{rgb color={1.000,1.000,1.000}} (1.196,0.985)--(7.946,0.985)--(7.946,5.630)--(1.196,5.630)--cycle;
\gpcolor{color=gp lt color border}
\gpsetlinetype{gp lt border}
\gpsetlinewidth{1.00}
\draw[gp path] (1.196,0.985)--(1.196,5.630)--(7.946,5.630)--(7.946,0.985)--cycle;
\gpcolor{color=gp lt color axes}
\gpsetlinetype{gp lt axes}
\gpsetlinewidth{2.00}
\draw[gp path] (1.196,0.985)--(7.947,0.985);
\gpcolor{color=gp lt color border}
\gpsetlinetype{gp lt border}
\draw[gp path] (1.196,0.985)--(1.268,0.985);
\draw[gp path] (7.947,0.985)--(7.875,0.985);
\gpcolor{rgb color={0.000,0.000,0.000}}
\node[gp node right,font={\fontsize{10pt}{12pt}\selectfont}] at (1.012,0.985) {0.6};
\gpcolor{color=gp lt color axes}
\gpsetlinetype{gp lt axes}
\draw[gp path] (1.196,1.759)--(7.947,1.759);
\gpcolor{color=gp lt color border}
\gpsetlinetype{gp lt border}
\draw[gp path] (1.196,1.759)--(1.268,1.759);
\draw[gp path] (7.947,1.759)--(7.875,1.759);
\gpcolor{rgb color={0.000,0.000,0.000}}
\node[gp node right,font={\fontsize{10pt}{12pt}\selectfont}] at (1.012,1.759) {0.65};
\gpcolor{color=gp lt color axes}
\gpsetlinetype{gp lt axes}
\draw[gp path] (1.196,2.534)--(7.947,2.534);
\gpcolor{color=gp lt color border}
\gpsetlinetype{gp lt border}
\draw[gp path] (1.196,2.534)--(1.268,2.534);
\draw[gp path] (7.947,2.534)--(7.875,2.534);
\gpcolor{rgb color={0.000,0.000,0.000}}
\node[gp node right,font={\fontsize{10pt}{12pt}\selectfont}] at (1.012,2.534) {0.7};
\gpcolor{color=gp lt color axes}
\gpsetlinetype{gp lt axes}
\draw[gp path] (1.196,3.308)--(7.947,3.308);
\gpcolor{color=gp lt color border}
\gpsetlinetype{gp lt border}
\draw[gp path] (1.196,3.308)--(1.268,3.308);
\draw[gp path] (7.947,3.308)--(7.875,3.308);
\gpcolor{rgb color={0.000,0.000,0.000}}
\node[gp node right,font={\fontsize{10pt}{12pt}\selectfont}] at (1.012,3.308) {0.75};
\gpcolor{color=gp lt color axes}
\gpsetlinetype{gp lt axes}
\draw[gp path] (1.196,4.082)--(7.947,4.082);
\gpcolor{color=gp lt color border}
\gpsetlinetype{gp lt border}
\draw[gp path] (1.196,4.082)--(1.268,4.082);
\draw[gp path] (7.947,4.082)--(7.875,4.082);
\gpcolor{rgb color={0.000,0.000,0.000}}
\node[gp node right,font={\fontsize{10pt}{12pt}\selectfont}] at (1.012,4.082) {0.8};
\gpcolor{color=gp lt color axes}
\gpsetlinetype{gp lt axes}
\draw[gp path] (1.196,4.857)--(7.947,4.857);
\gpcolor{color=gp lt color border}
\gpsetlinetype{gp lt border}
\draw[gp path] (1.196,4.857)--(1.268,4.857);
\draw[gp path] (7.947,4.857)--(7.875,4.857);
\gpcolor{rgb color={0.000,0.000,0.000}}
\node[gp node right,font={\fontsize{10pt}{12pt}\selectfont}] at (1.012,4.857) {0.85};
\gpcolor{color=gp lt color axes}
\gpsetlinetype{gp lt axes}
\draw[gp path] (1.196,5.631)--(7.947,5.631);
\gpcolor{color=gp lt color border}
\gpsetlinetype{gp lt border}
\draw[gp path] (1.196,5.631)--(1.268,5.631);
\draw[gp path] (7.947,5.631)--(7.875,5.631);
\gpcolor{rgb color={0.000,0.000,0.000}}
\node[gp node right,font={\fontsize{10pt}{12pt}\selectfont}] at (1.012,5.631) {0.9};
\gpcolor{color=gp lt color axes}
\gpsetlinetype{gp lt axes}
\draw[gp path] (1.196,0.985)--(1.196,5.631);
\gpcolor{color=gp lt color border}
\gpsetlinetype{gp lt border}
\draw[gp path] (1.196,0.985)--(1.196,1.057);
\draw[gp path] (1.196,5.631)--(1.196,5.559);
\gpcolor{rgb color={0.000,0.000,0.000}}
\node[gp node center,font={\fontsize{10pt}{12pt}\selectfont}] at (1.196,0.677) {0.2};
\gpcolor{color=gp lt color axes}
\gpsetlinetype{gp lt axes}
\draw[gp path] (2.321,0.985)--(2.321,5.631);
\gpcolor{color=gp lt color border}
\gpsetlinetype{gp lt border}
\draw[gp path] (2.321,0.985)--(2.321,1.057);
\draw[gp path] (2.321,5.631)--(2.321,5.559);
\gpcolor{rgb color={0.000,0.000,0.000}}
\node[gp node center,font={\fontsize{10pt}{12pt}\selectfont}] at (2.321,0.677) {0.25};
\gpcolor{color=gp lt color axes}
\gpsetlinetype{gp lt axes}
\draw[gp path] (3.446,0.985)--(3.446,5.631);
\gpcolor{color=gp lt color border}
\gpsetlinetype{gp lt border}
\draw[gp path] (3.446,0.985)--(3.446,1.057);
\draw[gp path] (3.446,5.631)--(3.446,5.559);
\gpcolor{rgb color={0.000,0.000,0.000}}
\node[gp node center,font={\fontsize{10pt}{12pt}\selectfont}] at (3.446,0.677) {0.3};
\gpcolor{color=gp lt color axes}
\gpsetlinetype{gp lt axes}
\draw[gp path] (4.572,0.985)--(4.572,5.631);
\gpcolor{color=gp lt color border}
\gpsetlinetype{gp lt border}
\draw[gp path] (4.572,0.985)--(4.572,1.057);
\draw[gp path] (4.572,5.631)--(4.572,5.559);
\gpcolor{rgb color={0.000,0.000,0.000}}
\node[gp node center,font={\fontsize{10pt}{12pt}\selectfont}] at (4.572,0.677) {0.35};
\gpcolor{color=gp lt color axes}
\gpsetlinetype{gp lt axes}
\draw[gp path] (5.697,0.985)--(5.697,5.631);
\gpcolor{color=gp lt color border}
\gpsetlinetype{gp lt border}
\draw[gp path] (5.697,0.985)--(5.697,1.057);
\draw[gp path] (5.697,5.631)--(5.697,5.559);
\gpcolor{rgb color={0.000,0.000,0.000}}
\node[gp node center,font={\fontsize{10pt}{12pt}\selectfont}] at (5.697,0.677) {0.4};
\gpcolor{color=gp lt color axes}
\gpsetlinetype{gp lt axes}
\draw[gp path] (6.822,0.985)--(6.822,5.631);
\gpcolor{color=gp lt color border}
\gpsetlinetype{gp lt border}
\draw[gp path] (6.822,0.985)--(6.822,1.057);
\draw[gp path] (6.822,5.631)--(6.822,5.559);
\gpcolor{rgb color={0.000,0.000,0.000}}
\node[gp node center,font={\fontsize{10pt}{12pt}\selectfont}] at (6.822,0.677) {0.45};
\gpcolor{color=gp lt color axes}
\gpsetlinetype{gp lt axes}
\draw[gp path] (7.947,0.985)--(7.947,5.631);
\gpcolor{color=gp lt color border}
\gpsetlinetype{gp lt border}
\draw[gp path] (7.947,0.985)--(7.947,1.057);
\draw[gp path] (7.947,5.631)--(7.947,5.559);
\gpcolor{rgb color={0.000,0.000,0.000}}
\node[gp node center,font={\fontsize{10pt}{12pt}\selectfont}] at (7.947,0.677) {0.5};
\gpcolor{color=gp lt color border}
\draw[gp path] (1.196,5.631)--(1.196,0.985)--(7.947,0.985)--(7.947,5.631)--cycle;
\gpcolor{rgb color={0.000,0.000,0.000}}
\node[gp node center,font={\fontsize{10pt}{12pt}\selectfont}] at (4.571,0.215) {\large $x$};
\gpcolor{rgb color={0.502,0.502,0.502}}
\gpsetlinewidth{0.50}
\gpsetpointsize{2.67}
\gppoint{gp mark 7}{(1.244,4.123)}
\gppoint{gp mark 7}{(1.288,4.058)}
\gppoint{gp mark 7}{(1.332,3.992)}
\gppoint{gp mark 7}{(1.376,3.927)}
\gppoint{gp mark 7}{(1.420,3.863)}
\gppoint{gp mark 7}{(1.464,3.798)}
\gppoint{gp mark 7}{(1.508,3.734)}
\gppoint{gp mark 7}{(1.552,3.670)}
\gppoint{gp mark 7}{(1.596,3.606)}
\gppoint{gp mark 7}{(1.640,3.542)}
\gppoint{gp mark 7}{(1.684,3.479)}
\gppoint{gp mark 7}{(1.728,3.416)}
\gppoint{gp mark 7}{(1.772,3.353)}
\gppoint{gp mark 7}{(1.816,3.290)}
\gppoint{gp mark 7}{(1.860,3.227)}
\gppoint{gp mark 7}{(1.904,3.165)}
\gppoint{gp mark 7}{(1.948,3.103)}
\gppoint{gp mark 7}{(1.992,3.041)}
\gppoint{gp mark 7}{(2.035,2.979)}
\gppoint{gp mark 7}{(2.079,2.918)}
\gppoint{gp mark 7}{(2.123,2.857)}
\gppoint{gp mark 7}{(2.167,2.796)}
\gppoint{gp mark 7}{(2.211,2.736)}
\gppoint{gp mark 7}{(2.255,2.675)}
\gppoint{gp mark 7}{(2.299,2.615)}
\gppoint{gp mark 7}{(2.343,2.555)}
\gppoint{gp mark 7}{(2.387,2.496)}
\gppoint{gp mark 7}{(2.431,2.437)}
\gppoint{gp mark 7}{(2.475,2.378)}
\gppoint{gp mark 7}{(2.519,2.319)}
\gppoint{gp mark 7}{(2.563,2.261)}
\gppoint{gp mark 7}{(2.607,2.203)}
\gppoint{gp mark 7}{(2.651,2.145)}
\gppoint{gp mark 7}{(2.695,2.088)}
\gppoint{gp mark 7}{(2.739,2.031)}
\gppoint{gp mark 7}{(2.783,1.975)}
\gppoint{gp mark 7}{(2.827,1.920)}
\gppoint{gp mark 7}{(2.871,1.866)}
\gppoint{gp mark 7}{(2.915,1.812)}
\gppoint{gp mark 7}{(2.958,1.761)}
\gppoint{gp mark 7}{(3.002,1.712)}
\gppoint{gp mark 7}{(3.046,1.666)}
\gppoint{gp mark 7}{(3.090,1.625)}
\gppoint{gp mark 7}{(3.134,1.593)}
\gppoint{gp mark 7}{(3.178,1.574)}
\gppoint{gp mark 7}{(3.222,1.570)}
\gppoint{gp mark 7}{(3.266,1.570)}
\gppoint{gp mark 7}{(3.310,1.571)}
\gppoint{gp mark 7}{(3.354,1.582)}
\gppoint{gp mark 7}{(3.398,1.603)}
\gppoint{gp mark 7}{(3.442,1.634)}
\gppoint{gp mark 7}{(3.486,1.976)}
\gppoint{gp mark 7}{(3.530,4.153)}
\gppoint{gp mark 7}{(3.574,5.011)}
\gppoint{gp mark 7}{(3.618,4.785)}
\gppoint{gp mark 7}{(3.662,4.485)}
\gppoint{gp mark 7}{(3.706,4.355)}
\gppoint{gp mark 7}{(3.750,4.271)}
\gppoint{gp mark 7}{(3.794,4.145)}
\gppoint{gp mark 7}{(3.838,4.023)}
\gppoint{gp mark 7}{(3.881,3.922)}
\gppoint{gp mark 7}{(3.925,3.818)}
\gppoint{gp mark 7}{(3.969,3.716)}
\gppoint{gp mark 7}{(4.013,3.618)}
\gppoint{gp mark 7}{(4.057,3.522)}
\gppoint{gp mark 7}{(4.101,3.441)}
\gppoint{gp mark 7}{(4.145,3.389)}
\gppoint{gp mark 7}{(4.189,3.364)}
\gppoint{gp mark 7}{(4.233,3.360)}
\gppoint{gp mark 7}{(4.277,3.358)}
\gppoint{gp mark 7}{(4.321,3.358)}
\gppoint{gp mark 7}{(4.365,3.360)}
\gppoint{gp mark 7}{(4.409,3.363)}
\gppoint{gp mark 7}{(4.453,3.364)}
\gppoint{gp mark 7}{(4.497,3.364)}
\gppoint{gp mark 7}{(4.541,3.363)}
\gppoint{gp mark 7}{(4.585,3.363)}
\gppoint{gp mark 7}{(4.629,3.363)}
\gppoint{gp mark 7}{(4.673,3.363)}
\gppoint{gp mark 7}{(4.717,3.362)}
\gppoint{gp mark 7}{(4.760,3.362)}
\gppoint{gp mark 7}{(4.804,3.362)}
\gppoint{gp mark 7}{(4.848,3.361)}
\gppoint{gp mark 7}{(4.892,3.361)}
\gppoint{gp mark 7}{(4.936,3.360)}
\gppoint{gp mark 7}{(4.980,3.360)}
\gppoint{gp mark 7}{(5.024,3.360)}
\gppoint{gp mark 7}{(5.068,3.359)}
\gppoint{gp mark 7}{(5.112,3.359)}
\gppoint{gp mark 7}{(5.156,3.359)}
\gppoint{gp mark 7}{(5.200,3.359)}
\gppoint{gp mark 7}{(5.244,3.360)}
\gppoint{gp mark 7}{(5.288,3.360)}
\gppoint{gp mark 7}{(5.332,3.359)}
\gppoint{gp mark 7}{(5.376,3.359)}
\gppoint{gp mark 7}{(5.420,3.359)}
\gppoint{gp mark 7}{(5.464,3.359)}
\gppoint{gp mark 7}{(5.508,3.359)}
\gppoint{gp mark 7}{(5.552,3.359)}
\gppoint{gp mark 7}{(5.596,3.359)}
\gppoint{gp mark 7}{(5.640,3.359)}
\gppoint{gp mark 7}{(5.683,3.359)}
\gppoint{gp mark 7}{(5.727,3.358)}
\gppoint{gp mark 7}{(5.771,3.357)}
\gppoint{gp mark 7}{(5.815,3.357)}
\gppoint{gp mark 7}{(5.859,3.356)}
\gppoint{gp mark 7}{(5.903,3.356)}
\gppoint{gp mark 7}{(5.947,3.355)}
\gppoint{gp mark 7}{(5.991,3.355)}
\gppoint{gp mark 7}{(6.035,3.354)}
\gppoint{gp mark 7}{(6.079,3.353)}
\gppoint{gp mark 7}{(6.123,3.352)}
\gppoint{gp mark 7}{(6.167,3.351)}
\gppoint{gp mark 7}{(6.211,3.351)}
\gppoint{gp mark 7}{(6.255,3.350)}
\gppoint{gp mark 7}{(6.299,3.350)}
\gppoint{gp mark 7}{(6.343,3.349)}
\gppoint{gp mark 7}{(6.387,3.349)}
\gppoint{gp mark 7}{(6.431,3.348)}
\gppoint{gp mark 7}{(6.475,3.348)}
\gppoint{gp mark 7}{(6.519,3.348)}
\gppoint{gp mark 7}{(6.563,3.348)}
\gppoint{gp mark 7}{(6.606,3.348)}
\gppoint{gp mark 7}{(6.650,3.348)}
\gppoint{gp mark 7}{(6.694,3.348)}
\gppoint{gp mark 7}{(6.738,3.349)}
\gppoint{gp mark 7}{(6.782,3.350)}
\gppoint{gp mark 7}{(6.826,3.350)}
\gppoint{gp mark 7}{(6.870,3.351)}
\gppoint{gp mark 7}{(6.914,3.353)}
\gppoint{gp mark 7}{(6.958,3.357)}
\gppoint{gp mark 7}{(7.002,3.360)}
\gppoint{gp mark 7}{(7.046,3.362)}
\gppoint{gp mark 7}{(7.090,3.366)}
\gppoint{gp mark 7}{(7.134,3.377)}
\gppoint{gp mark 7}{(7.178,3.390)}
\gppoint{gp mark 7}{(7.222,3.396)}
\gppoint{gp mark 7}{(7.266,3.396)}
\gppoint{gp mark 7}{(7.310,3.396)}
\gppoint{gp mark 7}{(7.354,3.391)}
\gppoint{gp mark 7}{(7.398,3.350)}
\gppoint{gp mark 7}{(7.442,3.025)}
\gppoint{gp mark 7}{(7.486,1.719)}
\gpcolor{rgb color={1.000,0.000,0.000}}
\gpsetpointsize{4.44}
\gppoint{gp mark 7}{(1.244,4.126)}
\gppoint{gp mark 7}{(1.288,4.062)}
\gppoint{gp mark 7}{(1.332,3.997)}
\gppoint{gp mark 7}{(1.376,3.933)}
\gppoint{gp mark 7}{(1.420,3.869)}
\gppoint{gp mark 7}{(1.464,3.805)}
\gppoint{gp mark 7}{(1.508,3.742)}
\gppoint{gp mark 7}{(1.552,3.679)}
\gppoint{gp mark 7}{(1.596,3.616)}
\gppoint{gp mark 7}{(1.640,3.553)}
\gppoint{gp mark 7}{(1.684,3.491)}
\gppoint{gp mark 7}{(1.728,3.429)}
\gppoint{gp mark 7}{(1.772,3.368)}
\gppoint{gp mark 7}{(1.816,3.306)}
\gppoint{gp mark 7}{(1.860,3.246)}
\gppoint{gp mark 7}{(1.904,3.186)}
\gppoint{gp mark 7}{(1.948,3.127)}
\gppoint{gp mark 7}{(1.992,3.068)}
\gppoint{gp mark 7}{(2.035,3.010)}
\gppoint{gp mark 7}{(2.079,2.954)}
\gppoint{gp mark 7}{(2.123,2.898)}
\gppoint{gp mark 7}{(2.167,2.844)}
\gppoint{gp mark 7}{(2.211,2.792)}
\gppoint{gp mark 7}{(2.255,2.742)}
\gppoint{gp mark 7}{(2.299,2.694)}
\gppoint{gp mark 7}{(2.343,2.649)}
\gppoint{gp mark 7}{(2.387,2.607)}
\gppoint{gp mark 7}{(2.431,2.568)}
\gppoint{gp mark 7}{(2.475,2.532)}
\gppoint{gp mark 7}{(2.519,2.499)}
\gppoint{gp mark 7}{(2.563,2.469)}
\gppoint{gp mark 7}{(2.607,2.443)}
\gppoint{gp mark 7}{(2.651,2.422)}
\gppoint{gp mark 7}{(2.695,2.406)}
\gppoint{gp mark 7}{(2.739,2.397)}
\gppoint{gp mark 7}{(2.783,2.394)}
\gppoint{gp mark 7}{(2.827,2.394)}
\gppoint{gp mark 7}{(2.871,2.394)}
\gppoint{gp mark 7}{(2.915,2.394)}
\gppoint{gp mark 7}{(2.958,2.395)}
\gppoint{gp mark 7}{(3.002,2.396)}
\gppoint{gp mark 7}{(3.046,2.398)}
\gppoint{gp mark 7}{(3.090,2.399)}
\gppoint{gp mark 7}{(3.134,2.400)}
\gppoint{gp mark 7}{(3.178,2.400)}
\gppoint{gp mark 7}{(3.222,2.400)}
\gppoint{gp mark 7}{(3.266,2.400)}
\gppoint{gp mark 7}{(3.310,2.399)}
\gppoint{gp mark 7}{(3.354,2.398)}
\gppoint{gp mark 7}{(3.398,2.401)}
\gppoint{gp mark 7}{(3.442,2.412)}
\gppoint{gp mark 7}{(3.486,2.859)}
\gppoint{gp mark 7}{(3.530,3.193)}
\gppoint{gp mark 7}{(3.574,2.958)}
\gppoint{gp mark 7}{(3.618,2.582)}
\gppoint{gp mark 7}{(3.662,2.510)}
\gppoint{gp mark 7}{(3.706,2.408)}
\gppoint{gp mark 7}{(3.750,2.437)}
\gppoint{gp mark 7}{(3.794,2.468)}
\gppoint{gp mark 7}{(3.838,2.471)}
\gppoint{gp mark 7}{(3.881,2.471)}
\gppoint{gp mark 7}{(3.925,2.463)}
\gppoint{gp mark 7}{(3.969,2.437)}
\gppoint{gp mark 7}{(4.013,2.421)}
\gppoint{gp mark 7}{(4.057,2.424)}
\gppoint{gp mark 7}{(4.101,2.431)}
\gppoint{gp mark 7}{(4.145,2.435)}
\gppoint{gp mark 7}{(4.189,2.442)}
\gppoint{gp mark 7}{(4.233,2.465)}
\gppoint{gp mark 7}{(4.277,2.580)}
\gppoint{gp mark 7}{(4.321,3.015)}
\gppoint{gp mark 7}{(4.365,3.505)}
\gppoint{gp mark 7}{(4.409,3.592)}
\gppoint{gp mark 7}{(4.453,3.580)}
\gppoint{gp mark 7}{(4.497,3.578)}
\gppoint{gp mark 7}{(4.541,3.586)}
\gppoint{gp mark 7}{(4.585,3.603)}
\gppoint{gp mark 7}{(4.629,3.607)}
\gppoint{gp mark 7}{(4.673,3.606)}
\gppoint{gp mark 7}{(4.717,3.599)}
\gppoint{gp mark 7}{(4.760,3.589)}
\gppoint{gp mark 7}{(4.804,3.589)}
\gppoint{gp mark 7}{(4.848,3.591)}
\gppoint{gp mark 7}{(4.892,3.600)}
\gppoint{gp mark 7}{(4.936,3.608)}
\gppoint{gp mark 7}{(4.980,3.610)}
\gppoint{gp mark 7}{(5.024,3.610)}
\gppoint{gp mark 7}{(5.068,3.607)}
\gppoint{gp mark 7}{(5.112,3.597)}
\gppoint{gp mark 7}{(5.156,3.590)}
\gppoint{gp mark 7}{(5.200,3.588)}
\gppoint{gp mark 7}{(5.244,3.589)}
\gppoint{gp mark 7}{(5.288,3.599)}
\gppoint{gp mark 7}{(5.332,3.608)}
\gppoint{gp mark 7}{(5.376,3.610)}
\gppoint{gp mark 7}{(5.420,3.610)}
\gppoint{gp mark 7}{(5.464,3.606)}
\gppoint{gp mark 7}{(5.508,3.593)}
\gppoint{gp mark 7}{(5.552,3.588)}
\gppoint{gp mark 7}{(5.596,3.588)}
\gppoint{gp mark 7}{(5.640,3.592)}
\gppoint{gp mark 7}{(5.683,3.603)}
\gppoint{gp mark 7}{(5.727,3.611)}
\gppoint{gp mark 7}{(5.771,3.613)}
\gppoint{gp mark 7}{(5.815,3.611)}
\gppoint{gp mark 7}{(5.859,3.606)}
\gppoint{gp mark 7}{(5.903,3.596)}
\gppoint{gp mark 7}{(5.947,3.592)}
\gppoint{gp mark 7}{(5.991,3.592)}
\gppoint{gp mark 7}{(6.035,3.593)}
\gppoint{gp mark 7}{(6.079,3.601)}
\gppoint{gp mark 7}{(6.123,3.609)}
\gppoint{gp mark 7}{(6.167,3.611)}
\gppoint{gp mark 7}{(6.211,3.610)}
\gppoint{gp mark 7}{(6.255,3.608)}
\gppoint{gp mark 7}{(6.299,3.600)}
\gppoint{gp mark 7}{(6.343,3.596)}
\gppoint{gp mark 7}{(6.387,3.597)}
\gppoint{gp mark 7}{(6.431,3.599)}
\gppoint{gp mark 7}{(6.475,3.603)}
\gppoint{gp mark 7}{(6.519,3.609)}
\gppoint{gp mark 7}{(6.563,3.609)}
\gppoint{gp mark 7}{(6.606,3.608)}
\gppoint{gp mark 7}{(6.650,3.606)}
\gppoint{gp mark 7}{(6.694,3.597)}
\gppoint{gp mark 7}{(6.738,3.587)}
\gppoint{gp mark 7}{(6.782,3.580)}
\gppoint{gp mark 7}{(6.826,3.578)}
\gppoint{gp mark 7}{(6.870,3.578)}
\gppoint{gp mark 7}{(6.914,3.578)}
\gppoint{gp mark 7}{(6.958,3.577)}
\gppoint{gp mark 7}{(7.002,3.577)}
\gppoint{gp mark 7}{(7.046,3.576)}
\gppoint{gp mark 7}{(7.090,3.576)}
\gppoint{gp mark 7}{(7.134,3.578)}
\gppoint{gp mark 7}{(7.178,3.584)}
\gppoint{gp mark 7}{(7.222,3.589)}
\gppoint{gp mark 7}{(7.266,3.590)}
\gppoint{gp mark 7}{(7.310,3.589)}
\gppoint{gp mark 7}{(7.354,3.574)}
\gppoint{gp mark 7}{(7.398,3.446)}
\gppoint{gp mark 7}{(7.442,2.642)}
\gpcolor{rgb color={0.000,0.000,0.000}}
\gpsetlinetype{gp lt plot 0}
\gpsetlinewidth{4.00}
\draw[gp path] (2.427,2.420)--(3.533,2.420);
\draw[gp path] (3.533,2.420)--(4.326,2.420);
\draw[gp path] (4.326,3.596)--(7.510,3.596);
\draw[gp path] (1.200,4.130)--(1.206,4.121)--(1.212,4.112)--(1.218,4.102)--(1.224,4.093)%
  --(1.230,4.084)--(1.236,4.075)--(1.242,4.066)--(1.248,4.057)--(1.254,4.047)--(1.260,4.038)%
  --(1.267,4.029)--(1.273,4.020)--(1.279,4.011)--(1.285,4.002)--(1.291,3.993)--(1.297,3.984)%
  --(1.303,3.974)--(1.309,3.965)--(1.315,3.956)--(1.321,3.947)--(1.327,3.938)--(1.333,3.929)%
  --(1.339,3.920)--(1.346,3.911)--(1.352,3.902)--(1.358,3.893)--(1.364,3.884)--(1.370,3.875)%
  --(1.376,3.866)--(1.382,3.857)--(1.388,3.848)--(1.394,3.839)--(1.400,3.830)--(1.406,3.821)%
  --(1.412,3.812)--(1.418,3.803)--(1.425,3.794)--(1.431,3.785)--(1.437,3.776)--(1.443,3.767)%
  --(1.449,3.758)--(1.455,3.749)--(1.461,3.740)--(1.467,3.731)--(1.473,3.722)--(1.479,3.713)%
  --(1.485,3.704)--(1.491,3.695)--(1.497,3.686)--(1.504,3.677)--(1.510,3.668)--(1.516,3.660)%
  --(1.522,3.651)--(1.528,3.642)--(1.534,3.633)--(1.540,3.624)--(1.546,3.615)--(1.552,3.606)%
  --(1.558,3.597)--(1.564,3.589)--(1.570,3.580)--(1.576,3.571)--(1.583,3.562)--(1.589,3.553)%
  --(1.595,3.545)--(1.601,3.536)--(1.607,3.527)--(1.613,3.518)--(1.619,3.509)--(1.625,3.501)%
  --(1.631,3.492)--(1.637,3.483)--(1.643,3.474)--(1.649,3.465)--(1.656,3.457)--(1.662,3.448)%
  --(1.668,3.439)--(1.674,3.431)--(1.680,3.422)--(1.686,3.413)--(1.692,3.404)--(1.698,3.396)%
  --(1.704,3.387)--(1.710,3.378)--(1.716,3.370)--(1.722,3.361)--(1.728,3.352)--(1.735,3.344)%
  --(1.741,3.335)--(1.747,3.326)--(1.753,3.318)--(1.759,3.309)--(1.765,3.300)--(1.771,3.292)%
  --(1.777,3.283)--(1.783,3.274)--(1.789,3.266)--(1.795,3.257)--(1.801,3.249)--(1.807,3.240)%
  --(1.814,3.232)--(1.820,3.223)--(1.826,3.214)--(1.832,3.206)--(1.838,3.197)--(1.844,3.189)%
  --(1.850,3.180)--(1.856,3.172)--(1.862,3.163)--(1.868,3.155)--(1.874,3.146)--(1.880,3.138)%
  --(1.886,3.129)--(1.893,3.121)--(1.899,3.112)--(1.905,3.104)--(1.911,3.095)--(1.917,3.087)%
  --(1.923,3.078)--(1.929,3.070)--(1.935,3.061)--(1.941,3.053)--(1.947,3.045)--(1.953,3.036)%
  --(1.959,3.028)--(1.965,3.019)--(1.972,3.011)--(1.978,3.003)--(1.984,2.994)--(1.990,2.986)%
  --(1.996,2.978)--(2.002,2.969)--(2.008,2.961)--(2.014,2.952)--(2.020,2.944)--(2.026,2.936)%
  --(2.032,2.928)--(2.038,2.919)--(2.044,2.911)--(2.051,2.903)--(2.057,2.894)--(2.063,2.886)%
  --(2.069,2.878)--(2.075,2.869)--(2.081,2.861)--(2.087,2.853)--(2.093,2.845)--(2.099,2.837)%
  --(2.105,2.828)--(2.111,2.820)--(2.117,2.812)--(2.124,2.804)--(2.130,2.795)--(2.136,2.787)%
  --(2.142,2.779)--(2.148,2.771)--(2.154,2.763)--(2.160,2.755)--(2.166,2.746)--(2.172,2.738)%
  --(2.178,2.730)--(2.184,2.722)--(2.190,2.714)--(2.196,2.706)--(2.203,2.698)--(2.209,2.690)%
  --(2.215,2.682)--(2.221,2.673)--(2.227,2.665)--(2.233,2.657)--(2.239,2.649)--(2.245,2.641)%
  --(2.251,2.633)--(2.257,2.625)--(2.263,2.617)--(2.269,2.609)--(2.275,2.601)--(2.282,2.593)%
  --(2.288,2.585)--(2.294,2.577)--(2.300,2.569)--(2.306,2.561)--(2.312,2.553)--(2.318,2.545)%
  --(2.324,2.537)--(2.330,2.529)--(2.336,2.522)--(2.342,2.514)--(2.348,2.506)--(2.354,2.498)%
  --(2.361,2.490)--(2.367,2.482)--(2.373,2.474)--(2.379,2.466)--(2.385,2.459)--(2.391,2.451)%
  --(2.397,2.443)--(2.403,2.435)--(2.409,2.427)--(2.415,2.419)--(2.421,2.412)--(2.427,2.420);
\draw[gp path] (4.326,2.420)--(4.326,3.596);
\draw[gp path] (7.510,3.596)--(7.510,0.985);
\node[gp node left,font={\fontsize{10pt}{12pt}\selectfont}] at (1.421,5.244) {\LARGE $\rho$};
\node[gp node left,font={\fontsize{10pt}{12pt}\selectfont}] at (6.147,5.244) {\large $\alpha = \pi$};
%% coordinates of the plot area
\gpdefrectangularnode{gp plot 1}{\pgfpoint{1.196cm}{0.985cm}}{\pgfpoint{7.947cm}{5.631cm}}
\end{tikzpicture}
%% gnuplot variables
} & 
\resizebox{0.5\linewidth}{!}{\tikzsetnextfilename{coplanar_a_crsol_00512_6}\begin{tikzpicture}[gnuplot]
%% generated with GNUPLOT 4.6p4 (Lua 5.1; terminal rev. 99, script rev. 100)
%% Fri 22 Aug 2014 12:01:03 PM EDT
\path (0.000,0.000) rectangle (8.500,6.000);
\gpfill{rgb color={1.000,1.000,1.000}} (1.196,0.985)--(7.946,0.985)--(7.946,5.630)--(1.196,5.630)--cycle;
\gpcolor{color=gp lt color border}
\gpsetlinetype{gp lt border}
\gpsetlinewidth{1.00}
\draw[gp path] (1.196,0.985)--(1.196,5.630)--(7.946,5.630)--(7.946,0.985)--cycle;
\gpcolor{color=gp lt color axes}
\gpsetlinetype{gp lt axes}
\gpsetlinewidth{2.00}
\draw[gp path] (1.196,0.985)--(7.947,0.985);
\gpcolor{color=gp lt color border}
\gpsetlinetype{gp lt border}
\draw[gp path] (1.196,0.985)--(1.268,0.985);
\draw[gp path] (7.947,0.985)--(7.875,0.985);
\gpcolor{rgb color={0.000,0.000,0.000}}
\node[gp node right,font={\fontsize{10pt}{12pt}\selectfont}] at (1.012,0.985) {-0.4};
\gpcolor{color=gp lt color axes}
\gpsetlinetype{gp lt axes}
\draw[gp path] (1.196,1.759)--(7.947,1.759);
\gpcolor{color=gp lt color border}
\gpsetlinetype{gp lt border}
\draw[gp path] (1.196,1.759)--(1.268,1.759);
\draw[gp path] (7.947,1.759)--(7.875,1.759);
\gpcolor{rgb color={0.000,0.000,0.000}}
\node[gp node right,font={\fontsize{10pt}{12pt}\selectfont}] at (1.012,1.759) {-0.2};
\gpcolor{color=gp lt color axes}
\gpsetlinetype{gp lt axes}
\draw[gp path] (1.196,2.534)--(7.947,2.534);
\gpcolor{color=gp lt color border}
\gpsetlinetype{gp lt border}
\draw[gp path] (1.196,2.534)--(1.268,2.534);
\draw[gp path] (7.947,2.534)--(7.875,2.534);
\gpcolor{rgb color={0.000,0.000,0.000}}
\node[gp node right,font={\fontsize{10pt}{12pt}\selectfont}] at (1.012,2.534) {0};
\gpcolor{color=gp lt color axes}
\gpsetlinetype{gp lt axes}
\draw[gp path] (1.196,3.308)--(7.947,3.308);
\gpcolor{color=gp lt color border}
\gpsetlinetype{gp lt border}
\draw[gp path] (1.196,3.308)--(1.268,3.308);
\draw[gp path] (7.947,3.308)--(7.875,3.308);
\gpcolor{rgb color={0.000,0.000,0.000}}
\node[gp node right,font={\fontsize{10pt}{12pt}\selectfont}] at (1.012,3.308) {0.2};
\gpcolor{color=gp lt color axes}
\gpsetlinetype{gp lt axes}
\draw[gp path] (1.196,4.082)--(7.947,4.082);
\gpcolor{color=gp lt color border}
\gpsetlinetype{gp lt border}
\draw[gp path] (1.196,4.082)--(1.268,4.082);
\draw[gp path] (7.947,4.082)--(7.875,4.082);
\gpcolor{rgb color={0.000,0.000,0.000}}
\node[gp node right,font={\fontsize{10pt}{12pt}\selectfont}] at (1.012,4.082) {0.4};
\gpcolor{color=gp lt color axes}
\gpsetlinetype{gp lt axes}
\draw[gp path] (1.196,4.857)--(7.947,4.857);
\gpcolor{color=gp lt color border}
\gpsetlinetype{gp lt border}
\draw[gp path] (1.196,4.857)--(1.268,4.857);
\draw[gp path] (7.947,4.857)--(7.875,4.857);
\gpcolor{rgb color={0.000,0.000,0.000}}
\node[gp node right,font={\fontsize{10pt}{12pt}\selectfont}] at (1.012,4.857) {0.6};
\gpcolor{color=gp lt color axes}
\gpsetlinetype{gp lt axes}
\draw[gp path] (1.196,5.631)--(7.947,5.631);
\gpcolor{color=gp lt color border}
\gpsetlinetype{gp lt border}
\draw[gp path] (1.196,5.631)--(1.268,5.631);
\draw[gp path] (7.947,5.631)--(7.875,5.631);
\gpcolor{rgb color={0.000,0.000,0.000}}
\node[gp node right,font={\fontsize{10pt}{12pt}\selectfont}] at (1.012,5.631) {0.8};
\gpcolor{color=gp lt color axes}
\gpsetlinetype{gp lt axes}
\draw[gp path] (1.196,0.985)--(1.196,5.631);
\gpcolor{color=gp lt color border}
\gpsetlinetype{gp lt border}
\draw[gp path] (1.196,0.985)--(1.196,1.057);
\draw[gp path] (1.196,5.631)--(1.196,5.559);
\gpcolor{rgb color={0.000,0.000,0.000}}
\node[gp node center,font={\fontsize{10pt}{12pt}\selectfont}] at (1.196,0.677) {0.2};
\gpcolor{color=gp lt color axes}
\gpsetlinetype{gp lt axes}
\draw[gp path] (2.321,0.985)--(2.321,5.631);
\gpcolor{color=gp lt color border}
\gpsetlinetype{gp lt border}
\draw[gp path] (2.321,0.985)--(2.321,1.057);
\draw[gp path] (2.321,5.631)--(2.321,5.559);
\gpcolor{rgb color={0.000,0.000,0.000}}
\node[gp node center,font={\fontsize{10pt}{12pt}\selectfont}] at (2.321,0.677) {0.25};
\gpcolor{color=gp lt color axes}
\gpsetlinetype{gp lt axes}
\draw[gp path] (3.446,0.985)--(3.446,5.631);
\gpcolor{color=gp lt color border}
\gpsetlinetype{gp lt border}
\draw[gp path] (3.446,0.985)--(3.446,1.057);
\draw[gp path] (3.446,5.631)--(3.446,5.559);
\gpcolor{rgb color={0.000,0.000,0.000}}
\node[gp node center,font={\fontsize{10pt}{12pt}\selectfont}] at (3.446,0.677) {0.3};
\gpcolor{color=gp lt color axes}
\gpsetlinetype{gp lt axes}
\draw[gp path] (4.572,0.985)--(4.572,5.631);
\gpcolor{color=gp lt color border}
\gpsetlinetype{gp lt border}
\draw[gp path] (4.572,0.985)--(4.572,1.057);
\draw[gp path] (4.572,5.631)--(4.572,5.559);
\gpcolor{rgb color={0.000,0.000,0.000}}
\node[gp node center,font={\fontsize{10pt}{12pt}\selectfont}] at (4.572,0.677) {0.35};
\gpcolor{color=gp lt color axes}
\gpsetlinetype{gp lt axes}
\draw[gp path] (5.697,0.985)--(5.697,5.631);
\gpcolor{color=gp lt color border}
\gpsetlinetype{gp lt border}
\draw[gp path] (5.697,0.985)--(5.697,1.057);
\draw[gp path] (5.697,5.631)--(5.697,5.559);
\gpcolor{rgb color={0.000,0.000,0.000}}
\node[gp node center,font={\fontsize{10pt}{12pt}\selectfont}] at (5.697,0.677) {0.4};
\gpcolor{color=gp lt color axes}
\gpsetlinetype{gp lt axes}
\draw[gp path] (6.822,0.985)--(6.822,5.631);
\gpcolor{color=gp lt color border}
\gpsetlinetype{gp lt border}
\draw[gp path] (6.822,0.985)--(6.822,1.057);
\draw[gp path] (6.822,5.631)--(6.822,5.559);
\gpcolor{rgb color={0.000,0.000,0.000}}
\node[gp node center,font={\fontsize{10pt}{12pt}\selectfont}] at (6.822,0.677) {0.45};
\gpcolor{color=gp lt color axes}
\gpsetlinetype{gp lt axes}
\draw[gp path] (7.947,0.985)--(7.947,5.631);
\gpcolor{color=gp lt color border}
\gpsetlinetype{gp lt border}
\draw[gp path] (7.947,0.985)--(7.947,1.057);
\draw[gp path] (7.947,5.631)--(7.947,5.559);
\gpcolor{rgb color={0.000,0.000,0.000}}
\node[gp node center,font={\fontsize{10pt}{12pt}\selectfont}] at (7.947,0.677) {0.5};
\gpcolor{color=gp lt color border}
\draw[gp path] (1.196,5.631)--(1.196,0.985)--(7.947,0.985)--(7.947,5.631)--cycle;
\gpcolor{rgb color={0.000,0.000,0.000}}
\node[gp node center,font={\fontsize{10pt}{12pt}\selectfont}] at (4.571,0.215) {\large $x$};
\gpcolor{rgb color={0.502,0.502,0.502}}
\gpsetlinewidth{0.50}
\gpsetpointsize{2.67}
\gppoint{gp mark 7}{(1.244,4.616)}
\gppoint{gp mark 7}{(1.288,4.594)}
\gppoint{gp mark 7}{(1.332,4.573)}
\gppoint{gp mark 7}{(1.376,4.552)}
\gppoint{gp mark 7}{(1.420,4.531)}
\gppoint{gp mark 7}{(1.464,4.510)}
\gppoint{gp mark 7}{(1.508,4.488)}
\gppoint{gp mark 7}{(1.552,4.467)}
\gppoint{gp mark 7}{(1.596,4.446)}
\gppoint{gp mark 7}{(1.640,4.425)}
\gppoint{gp mark 7}{(1.684,4.403)}
\gppoint{gp mark 7}{(1.728,4.382)}
\gppoint{gp mark 7}{(1.772,4.360)}
\gppoint{gp mark 7}{(1.816,4.339)}
\gppoint{gp mark 7}{(1.860,4.317)}
\gppoint{gp mark 7}{(1.904,4.296)}
\gppoint{gp mark 7}{(1.948,4.274)}
\gppoint{gp mark 7}{(1.992,4.252)}
\gppoint{gp mark 7}{(2.035,4.231)}
\gppoint{gp mark 7}{(2.079,4.209)}
\gppoint{gp mark 7}{(2.123,4.187)}
\gppoint{gp mark 7}{(2.167,4.165)}
\gppoint{gp mark 7}{(2.211,4.143)}
\gppoint{gp mark 7}{(2.255,4.121)}
\gppoint{gp mark 7}{(2.299,4.099)}
\gppoint{gp mark 7}{(2.343,4.077)}
\gppoint{gp mark 7}{(2.387,4.054)}
\gppoint{gp mark 7}{(2.431,4.032)}
\gppoint{gp mark 7}{(2.475,4.010)}
\gppoint{gp mark 7}{(2.519,3.987)}
\gppoint{gp mark 7}{(2.563,3.965)}
\gppoint{gp mark 7}{(2.607,3.942)}
\gppoint{gp mark 7}{(2.651,3.919)}
\gppoint{gp mark 7}{(2.695,3.897)}
\gppoint{gp mark 7}{(2.739,3.874)}
\gppoint{gp mark 7}{(2.783,3.851)}
\gppoint{gp mark 7}{(2.827,3.828)}
\gppoint{gp mark 7}{(2.871,3.806)}
\gppoint{gp mark 7}{(2.915,3.783)}
\gppoint{gp mark 7}{(2.958,3.762)}
\gppoint{gp mark 7}{(3.002,3.741)}
\gppoint{gp mark 7}{(3.046,3.721)}
\gppoint{gp mark 7}{(3.090,3.703)}
\gppoint{gp mark 7}{(3.134,3.688)}
\gppoint{gp mark 7}{(3.178,3.680)}
\gppoint{gp mark 7}{(3.222,3.678)}
\gppoint{gp mark 7}{(3.266,3.678)}
\gppoint{gp mark 7}{(3.310,3.679)}
\gppoint{gp mark 7}{(3.354,3.684)}
\gppoint{gp mark 7}{(3.398,3.694)}
\gppoint{gp mark 7}{(3.442,3.700)}
\gppoint{gp mark 7}{(3.486,3.643)}
\gppoint{gp mark 7}{(3.530,2.812)}
\gppoint{gp mark 7}{(3.574,1.801)}
\gppoint{gp mark 7}{(3.618,1.651)}
\gppoint{gp mark 7}{(3.662,1.613)}
\gppoint{gp mark 7}{(3.706,1.573)}
\gppoint{gp mark 7}{(3.750,1.531)}
\gppoint{gp mark 7}{(3.794,1.499)}
\gppoint{gp mark 7}{(3.838,1.468)}
\gppoint{gp mark 7}{(3.881,1.435)}
\gppoint{gp mark 7}{(3.925,1.407)}
\gppoint{gp mark 7}{(3.969,1.380)}
\gppoint{gp mark 7}{(4.013,1.354)}
\gppoint{gp mark 7}{(4.057,1.332)}
\gppoint{gp mark 7}{(4.101,1.312)}
\gppoint{gp mark 7}{(4.145,1.297)}
\gppoint{gp mark 7}{(4.189,1.292)}
\gppoint{gp mark 7}{(4.233,1.291)}
\gppoint{gp mark 7}{(4.277,1.291)}
\gppoint{gp mark 7}{(4.321,1.291)}
\gppoint{gp mark 7}{(4.365,1.291)}
\gppoint{gp mark 7}{(4.409,1.292)}
\gppoint{gp mark 7}{(4.453,1.292)}
\gppoint{gp mark 7}{(4.497,1.292)}
\gppoint{gp mark 7}{(4.541,1.292)}
\gppoint{gp mark 7}{(4.585,1.292)}
\gppoint{gp mark 7}{(4.629,1.292)}
\gppoint{gp mark 7}{(4.673,1.292)}
\gppoint{gp mark 7}{(4.717,1.292)}
\gppoint{gp mark 7}{(4.760,1.292)}
\gppoint{gp mark 7}{(4.804,1.292)}
\gppoint{gp mark 7}{(4.848,1.292)}
\gppoint{gp mark 7}{(4.892,1.292)}
\gppoint{gp mark 7}{(4.936,1.292)}
\gppoint{gp mark 7}{(4.980,1.292)}
\gppoint{gp mark 7}{(5.024,1.292)}
\gppoint{gp mark 7}{(5.068,1.292)}
\gppoint{gp mark 7}{(5.112,1.292)}
\gppoint{gp mark 7}{(5.156,1.292)}
\gppoint{gp mark 7}{(5.200,1.292)}
\gppoint{gp mark 7}{(5.244,1.292)}
\gppoint{gp mark 7}{(5.288,1.292)}
\gppoint{gp mark 7}{(5.332,1.292)}
\gppoint{gp mark 7}{(5.376,1.292)}
\gppoint{gp mark 7}{(5.420,1.292)}
\gppoint{gp mark 7}{(5.464,1.292)}
\gppoint{gp mark 7}{(5.508,1.292)}
\gppoint{gp mark 7}{(5.552,1.292)}
\gppoint{gp mark 7}{(5.596,1.292)}
\gppoint{gp mark 7}{(5.640,1.292)}
\gppoint{gp mark 7}{(5.683,1.292)}
\gppoint{gp mark 7}{(5.727,1.292)}
\gppoint{gp mark 7}{(5.771,1.292)}
\gppoint{gp mark 7}{(5.815,1.292)}
\gppoint{gp mark 7}{(5.859,1.292)}
\gppoint{gp mark 7}{(5.903,1.292)}
\gppoint{gp mark 7}{(5.947,1.292)}
\gppoint{gp mark 7}{(5.991,1.292)}
\gppoint{gp mark 7}{(6.035,1.292)}
\gppoint{gp mark 7}{(6.079,1.292)}
\gppoint{gp mark 7}{(6.123,1.292)}
\gppoint{gp mark 7}{(6.167,1.292)}
\gppoint{gp mark 7}{(6.211,1.292)}
\gppoint{gp mark 7}{(6.255,1.292)}
\gppoint{gp mark 7}{(6.299,1.292)}
\gppoint{gp mark 7}{(6.343,1.292)}
\gppoint{gp mark 7}{(6.387,1.292)}
\gppoint{gp mark 7}{(6.431,1.292)}
\gppoint{gp mark 7}{(6.475,1.292)}
\gppoint{gp mark 7}{(6.519,1.292)}
\gppoint{gp mark 7}{(6.563,1.292)}
\gppoint{gp mark 7}{(6.606,1.292)}
\gppoint{gp mark 7}{(6.650,1.292)}
\gppoint{gp mark 7}{(6.694,1.292)}
\gppoint{gp mark 7}{(6.738,1.292)}
\gppoint{gp mark 7}{(6.782,1.292)}
\gppoint{gp mark 7}{(6.826,1.292)}
\gppoint{gp mark 7}{(6.870,1.292)}
\gppoint{gp mark 7}{(6.914,1.292)}
\gppoint{gp mark 7}{(6.958,1.292)}
\gppoint{gp mark 7}{(7.002,1.292)}
\gppoint{gp mark 7}{(7.046,1.292)}
\gppoint{gp mark 7}{(7.090,1.292)}
\gppoint{gp mark 7}{(7.134,1.292)}
\gppoint{gp mark 7}{(7.178,1.292)}
\gppoint{gp mark 7}{(7.222,1.292)}
\gppoint{gp mark 7}{(7.266,1.292)}
\gppoint{gp mark 7}{(7.310,1.292)}
\gppoint{gp mark 7}{(7.354,1.292)}
\gppoint{gp mark 7}{(7.398,1.292)}
\gppoint{gp mark 7}{(7.442,1.292)}
\gppoint{gp mark 7}{(7.486,1.292)}
\gppoint{gp mark 7}{(7.529,1.292)}
\gppoint{gp mark 7}{(7.573,1.292)}
\gppoint{gp mark 7}{(7.617,1.292)}
\gppoint{gp mark 7}{(7.661,1.292)}
\gppoint{gp mark 7}{(7.705,1.292)}
\gppoint{gp mark 7}{(7.749,1.292)}
\gppoint{gp mark 7}{(7.793,1.292)}
\gppoint{gp mark 7}{(7.837,1.292)}
\gppoint{gp mark 7}{(7.881,1.292)}
\gppoint{gp mark 7}{(7.925,1.292)}
\gpcolor{rgb color={1.000,0.000,0.000}}
\gpsetpointsize{4.44}
\gppoint{gp mark 7}{(1.244,4.617)}
\gppoint{gp mark 7}{(1.288,4.596)}
\gppoint{gp mark 7}{(1.332,4.575)}
\gppoint{gp mark 7}{(1.376,4.554)}
\gppoint{gp mark 7}{(1.420,4.533)}
\gppoint{gp mark 7}{(1.464,4.512)}
\gppoint{gp mark 7}{(1.508,4.491)}
\gppoint{gp mark 7}{(1.552,4.470)}
\gppoint{gp mark 7}{(1.596,4.449)}
\gppoint{gp mark 7}{(1.640,4.428)}
\gppoint{gp mark 7}{(1.684,4.407)}
\gppoint{gp mark 7}{(1.728,4.386)}
\gppoint{gp mark 7}{(1.772,4.365)}
\gppoint{gp mark 7}{(1.816,4.345)}
\gppoint{gp mark 7}{(1.860,4.324)}
\gppoint{gp mark 7}{(1.904,4.303)}
\gppoint{gp mark 7}{(1.948,4.282)}
\gppoint{gp mark 7}{(1.992,4.262)}
\gppoint{gp mark 7}{(2.035,4.242)}
\gppoint{gp mark 7}{(2.079,4.221)}
\gppoint{gp mark 7}{(2.123,4.202)}
\gppoint{gp mark 7}{(2.167,4.182)}
\gppoint{gp mark 7}{(2.211,4.164)}
\gppoint{gp mark 7}{(2.255,4.145)}
\gppoint{gp mark 7}{(2.299,4.128)}
\gppoint{gp mark 7}{(2.343,4.111)}
\gppoint{gp mark 7}{(2.387,4.096)}
\gppoint{gp mark 7}{(2.431,4.081)}
\gppoint{gp mark 7}{(2.475,4.068)}
\gppoint{gp mark 7}{(2.519,4.056)}
\gppoint{gp mark 7}{(2.563,4.045)}
\gppoint{gp mark 7}{(2.607,4.035)}
\gppoint{gp mark 7}{(2.651,4.027)}
\gppoint{gp mark 7}{(2.695,4.021)}
\gppoint{gp mark 7}{(2.739,4.017)}
\gppoint{gp mark 7}{(2.783,4.016)}
\gppoint{gp mark 7}{(2.827,4.016)}
\gppoint{gp mark 7}{(2.871,4.016)}
\gppoint{gp mark 7}{(2.915,4.016)}
\gppoint{gp mark 7}{(2.958,4.016)}
\gppoint{gp mark 7}{(3.002,4.017)}
\gppoint{gp mark 7}{(3.046,4.018)}
\gppoint{gp mark 7}{(3.090,4.018)}
\gppoint{gp mark 7}{(3.134,4.018)}
\gppoint{gp mark 7}{(3.178,4.018)}
\gppoint{gp mark 7}{(3.222,4.018)}
\gppoint{gp mark 7}{(3.266,4.018)}
\gppoint{gp mark 7}{(3.310,4.018)}
\gppoint{gp mark 7}{(3.354,4.018)}
\gppoint{gp mark 7}{(3.398,4.017)}
\gppoint{gp mark 7}{(3.442,4.013)}
\gppoint{gp mark 7}{(3.486,4.013)}
\gppoint{gp mark 7}{(3.530,3.231)}
\gppoint{gp mark 7}{(3.574,1.206)}
\gppoint{gp mark 7}{(3.618,1.042)}
\gppoint{gp mark 7}{(3.662,1.055)}
\gppoint{gp mark 7}{(3.706,1.036)}
\gppoint{gp mark 7}{(3.750,1.051)}
\gppoint{gp mark 7}{(3.794,1.062)}
\gppoint{gp mark 7}{(3.838,1.054)}
\gppoint{gp mark 7}{(3.881,1.047)}
\gppoint{gp mark 7}{(3.925,1.049)}
\gppoint{gp mark 7}{(3.969,1.055)}
\gppoint{gp mark 7}{(4.013,1.053)}
\gppoint{gp mark 7}{(4.057,1.049)}
\gppoint{gp mark 7}{(4.101,1.048)}
\gppoint{gp mark 7}{(4.145,1.049)}
\gppoint{gp mark 7}{(4.189,1.052)}
\gppoint{gp mark 7}{(4.233,1.056)}
\gppoint{gp mark 7}{(4.277,1.078)}
\gppoint{gp mark 7}{(4.321,1.165)}
\gppoint{gp mark 7}{(4.365,1.272)}
\gppoint{gp mark 7}{(4.409,1.292)}
\gppoint{gp mark 7}{(4.453,1.292)}
\gppoint{gp mark 7}{(4.497,1.292)}
\gppoint{gp mark 7}{(4.541,1.292)}
\gppoint{gp mark 7}{(4.585,1.293)}
\gppoint{gp mark 7}{(4.629,1.293)}
\gppoint{gp mark 7}{(4.673,1.293)}
\gppoint{gp mark 7}{(4.717,1.293)}
\gppoint{gp mark 7}{(4.760,1.292)}
\gppoint{gp mark 7}{(4.804,1.292)}
\gppoint{gp mark 7}{(4.848,1.292)}
\gppoint{gp mark 7}{(4.892,1.292)}
\gppoint{gp mark 7}{(4.936,1.293)}
\gppoint{gp mark 7}{(4.980,1.293)}
\gppoint{gp mark 7}{(5.024,1.293)}
\gppoint{gp mark 7}{(5.068,1.293)}
\gppoint{gp mark 7}{(5.112,1.293)}
\gppoint{gp mark 7}{(5.156,1.293)}
\gppoint{gp mark 7}{(5.200,1.293)}
\gppoint{gp mark 7}{(5.244,1.293)}
\gppoint{gp mark 7}{(5.288,1.293)}
\gppoint{gp mark 7}{(5.332,1.293)}
\gppoint{gp mark 7}{(5.376,1.293)}
\gppoint{gp mark 7}{(5.420,1.293)}
\gppoint{gp mark 7}{(5.464,1.293)}
\gppoint{gp mark 7}{(5.508,1.293)}
\gppoint{gp mark 7}{(5.552,1.293)}
\gppoint{gp mark 7}{(5.596,1.293)}
\gppoint{gp mark 7}{(5.640,1.293)}
\gppoint{gp mark 7}{(5.683,1.293)}
\gppoint{gp mark 7}{(5.727,1.293)}
\gppoint{gp mark 7}{(5.771,1.293)}
\gppoint{gp mark 7}{(5.815,1.293)}
\gppoint{gp mark 7}{(5.859,1.293)}
\gppoint{gp mark 7}{(5.903,1.293)}
\gppoint{gp mark 7}{(5.947,1.293)}
\gppoint{gp mark 7}{(5.991,1.293)}
\gppoint{gp mark 7}{(6.035,1.293)}
\gppoint{gp mark 7}{(6.079,1.293)}
\gppoint{gp mark 7}{(6.123,1.293)}
\gppoint{gp mark 7}{(6.167,1.293)}
\gppoint{gp mark 7}{(6.211,1.293)}
\gppoint{gp mark 7}{(6.255,1.293)}
\gppoint{gp mark 7}{(6.299,1.293)}
\gppoint{gp mark 7}{(6.343,1.293)}
\gppoint{gp mark 7}{(6.387,1.293)}
\gppoint{gp mark 7}{(6.431,1.293)}
\gppoint{gp mark 7}{(6.475,1.293)}
\gppoint{gp mark 7}{(6.519,1.293)}
\gppoint{gp mark 7}{(6.563,1.293)}
\gppoint{gp mark 7}{(6.606,1.293)}
\gppoint{gp mark 7}{(6.650,1.293)}
\gppoint{gp mark 7}{(6.694,1.293)}
\gppoint{gp mark 7}{(6.738,1.293)}
\gppoint{gp mark 7}{(6.782,1.293)}
\gppoint{gp mark 7}{(6.826,1.293)}
\gppoint{gp mark 7}{(6.870,1.293)}
\gppoint{gp mark 7}{(6.914,1.293)}
\gppoint{gp mark 7}{(6.958,1.293)}
\gppoint{gp mark 7}{(7.002,1.292)}
\gppoint{gp mark 7}{(7.046,1.292)}
\gppoint{gp mark 7}{(7.090,1.293)}
\gppoint{gp mark 7}{(7.134,1.293)}
\gppoint{gp mark 7}{(7.178,1.293)}
\gppoint{gp mark 7}{(7.222,1.293)}
\gppoint{gp mark 7}{(7.266,1.293)}
\gppoint{gp mark 7}{(7.310,1.293)}
\gppoint{gp mark 7}{(7.354,1.293)}
\gppoint{gp mark 7}{(7.398,1.293)}
\gppoint{gp mark 7}{(7.442,1.293)}
\gppoint{gp mark 7}{(7.486,1.293)}
\gppoint{gp mark 7}{(7.529,1.293)}
\gppoint{gp mark 7}{(7.573,1.293)}
\gppoint{gp mark 7}{(7.617,1.293)}
\gppoint{gp mark 7}{(7.661,1.293)}
\gppoint{gp mark 7}{(7.705,1.293)}
\gppoint{gp mark 7}{(7.749,1.293)}
\gppoint{gp mark 7}{(7.793,1.293)}
\gppoint{gp mark 7}{(7.837,1.293)}
\gppoint{gp mark 7}{(7.881,1.293)}
\gppoint{gp mark 7}{(7.925,1.293)}
\gpcolor{rgb color={0.000,0.000,0.000}}
\gpsetlinetype{gp lt plot 0}
\gpsetlinewidth{4.00}
\draw[gp path] (2.427,4.020)--(3.533,4.020);
\draw[gp path] (3.533,1.048)--(4.326,1.048);
\draw[gp path] (4.326,1.293)--(7.510,1.293);
\draw[gp path] (7.510,1.293)--(7.947,1.293);
\draw[gp path] (1.200,4.618)--(1.206,4.615)--(1.212,4.612)--(1.218,4.609)--(1.224,4.606)%
  --(1.230,4.603)--(1.236,4.600)--(1.242,4.597)--(1.248,4.594)--(1.254,4.591)--(1.260,4.588)%
  --(1.267,4.585)--(1.273,4.582)--(1.279,4.579)--(1.285,4.576)--(1.291,4.573)--(1.297,4.570)%
  --(1.303,4.567)--(1.309,4.564)--(1.315,4.562)--(1.321,4.559)--(1.327,4.556)--(1.333,4.553)%
  --(1.339,4.550)--(1.346,4.547)--(1.352,4.544)--(1.358,4.541)--(1.364,4.538)--(1.370,4.535)%
  --(1.376,4.532)--(1.382,4.529)--(1.388,4.526)--(1.394,4.523)--(1.400,4.520)--(1.406,4.517)%
  --(1.412,4.514)--(1.418,4.511)--(1.425,4.508)--(1.431,4.505)--(1.437,4.502)--(1.443,4.499)%
  --(1.449,4.496)--(1.455,4.493)--(1.461,4.490)--(1.467,4.488)--(1.473,4.485)--(1.479,4.482)%
  --(1.485,4.479)--(1.491,4.476)--(1.497,4.473)--(1.504,4.470)--(1.510,4.467)--(1.516,4.464)%
  --(1.522,4.461)--(1.528,4.458)--(1.534,4.455)--(1.540,4.452)--(1.546,4.449)--(1.552,4.446)%
  --(1.558,4.443)--(1.564,4.440)--(1.570,4.437)--(1.576,4.434)--(1.583,4.431)--(1.589,4.428)%
  --(1.595,4.425)--(1.601,4.422)--(1.607,4.419)--(1.613,4.416)--(1.619,4.413)--(1.625,4.411)%
  --(1.631,4.408)--(1.637,4.405)--(1.643,4.402)--(1.649,4.399)--(1.656,4.396)--(1.662,4.393)%
  --(1.668,4.390)--(1.674,4.387)--(1.680,4.384)--(1.686,4.381)--(1.692,4.378)--(1.698,4.375)%
  --(1.704,4.372)--(1.710,4.369)--(1.716,4.366)--(1.722,4.363)--(1.728,4.360)--(1.735,4.357)%
  --(1.741,4.354)--(1.747,4.351)--(1.753,4.348)--(1.759,4.345)--(1.765,4.342)--(1.771,4.339)%
  --(1.777,4.337)--(1.783,4.334)--(1.789,4.331)--(1.795,4.328)--(1.801,4.325)--(1.807,4.322)%
  --(1.814,4.319)--(1.820,4.316)--(1.826,4.313)--(1.832,4.310)--(1.838,4.307)--(1.844,4.304)%
  --(1.850,4.301)--(1.856,4.298)--(1.862,4.295)--(1.868,4.292)--(1.874,4.289)--(1.880,4.286)%
  --(1.886,4.283)--(1.893,4.280)--(1.899,4.277)--(1.905,4.274)--(1.911,4.271)--(1.917,4.268)%
  --(1.923,4.265)--(1.929,4.262)--(1.935,4.260)--(1.941,4.257)--(1.947,4.254)--(1.953,4.251)%
  --(1.959,4.248)--(1.965,4.245)--(1.972,4.242)--(1.978,4.239)--(1.984,4.236)--(1.990,4.233)%
  --(1.996,4.230)--(2.002,4.227)--(2.008,4.224)--(2.014,4.221)--(2.020,4.218)--(2.026,4.215)%
  --(2.032,4.212)--(2.038,4.209)--(2.044,4.206)--(2.051,4.203)--(2.057,4.200)--(2.063,4.197)%
  --(2.069,4.194)--(2.075,4.191)--(2.081,4.188)--(2.087,4.186)--(2.093,4.183)--(2.099,4.180)%
  --(2.105,4.177)--(2.111,4.174)--(2.117,4.171)--(2.124,4.168)--(2.130,4.165)--(2.136,4.162)%
  --(2.142,4.159)--(2.148,4.156)--(2.154,4.153)--(2.160,4.150)--(2.166,4.147)--(2.172,4.144)%
  --(2.178,4.141)--(2.184,4.138)--(2.190,4.135)--(2.196,4.132)--(2.203,4.129)--(2.209,4.126)%
  --(2.215,4.123)--(2.221,4.120)--(2.227,4.117)--(2.233,4.114)--(2.239,4.111)--(2.245,4.109)%
  --(2.251,4.106)--(2.257,4.103)--(2.263,4.100)--(2.269,4.097)--(2.275,4.094)--(2.282,4.091)%
  --(2.288,4.088)--(2.294,4.085)--(2.300,4.082)--(2.306,4.079)--(2.312,4.076)--(2.318,4.073)%
  --(2.324,4.070)--(2.330,4.067)--(2.336,4.064)--(2.342,4.061)--(2.348,4.058)--(2.354,4.055)%
  --(2.361,4.052)--(2.367,4.049)--(2.373,4.046)--(2.379,4.043)--(2.385,4.040)--(2.391,4.037)%
  --(2.397,4.035)--(2.403,4.032)--(2.409,4.029)--(2.415,4.026)--(2.421,4.023)--(2.427,4.020);
\draw[gp path] (3.533,4.020)--(3.533,1.048);
\draw[gp path] (4.326,1.048)--(4.326,1.293);
\draw[gp path] (3.896,3.695)--(4.572,3.695);
\gpcolor{rgb color={1.000,0.000,0.000}}
\gpsetlinewidth{0.50}
\gppoint{gp mark 7}{(4.234,2.921)}
\gpcolor{rgb color={0.502,0.502,0.502}}
\gppoint{gp mark 7}{(4.234,2.147)}
\gpcolor{rgb color={0.000,0.000,0.000}}
\node[gp node left,font={\fontsize{10pt}{12pt}\selectfont}] at (1.421,5.166) {\LARGE $B_y$};
\node[gp node left,font={\fontsize{10pt}{12pt}\selectfont}] at (6.147,5.166) {\large $\alpha = \pi$};
\node[gp node left,font={\fontsize{10pt}{12pt}\selectfont}] at (4.797,3.695) {\large exact};
\node[gp node left,font={\fontsize{10pt}{12pt}\selectfont}] at (4.797,2.921) {\large HLLD-CWM};
\node[gp node left,font={\fontsize{10pt}{12pt}\selectfont}] at (4.797,2.147) {\large HLLD};
%% coordinates of the plot area
\gpdefrectangularnode{gp plot 1}{\pgfpoint{1.196cm}{0.985cm}}{\pgfpoint{7.947cm}{5.631cm}}
\end{tikzpicture}
%% gnuplot variables
} \\
\end{tabular}
\caption{The rotational discontinuity and slow shock solution found using HLLD-CWM without the (optional) flux redistribution step, HLLD, and the exact solver using $512$ grid points for (top) a near-coplanar and pseudo-converging case and (bottom) the planar and non-converging (bottom) case.}
\label{fig:coplanar_ab_crsol_512}
\end{figure}
        
%-----------------------------------------------------------------
% Coplanar flux zoomed
%-----------------------------------------------------------------
\begin{figure}[htbp] 
\begin{tabular}{cc}
\resizebox{0.5\linewidth}{!}{\tikzsetnextfilename{coplanar_b_crsol_1}\begin{tikzpicture}[gnuplot]
%% generated with GNUPLOT 4.6p4 (Lua 5.1; terminal rev. 99, script rev. 100)
%% Mon 02 Jun 2014 11:35:14 AM EDT
\path (0.000,0.000) rectangle (8.500,6.000);
\gpfill{rgb color={1.000,1.000,1.000}} (1.196,0.985)--(7.946,0.985)--(7.946,5.630)--(1.196,5.630)--cycle;
\gpcolor{color=gp lt color border}
\gpsetlinetype{gp lt border}
\gpsetlinewidth{1.00}
\draw[gp path] (1.196,0.985)--(1.196,5.630)--(7.946,5.630)--(7.946,0.985)--cycle;
\gpcolor{color=gp lt color axes}
\gpsetlinetype{gp lt axes}
\gpsetlinewidth{2.00}
\draw[gp path] (1.196,0.985)--(7.947,0.985);
\gpcolor{color=gp lt color border}
\gpsetlinetype{gp lt border}
\draw[gp path] (1.196,0.985)--(1.268,0.985);
\draw[gp path] (7.947,0.985)--(7.875,0.985);
\gpcolor{rgb color={0.000,0.000,0.000}}
\node[gp node right,font={\fontsize{10pt}{12pt}\selectfont}] at (1.012,0.985) {0.6};
\gpcolor{color=gp lt color axes}
\gpsetlinetype{gp lt axes}
\draw[gp path] (1.196,1.759)--(7.947,1.759);
\gpcolor{color=gp lt color border}
\gpsetlinetype{gp lt border}
\draw[gp path] (1.196,1.759)--(1.268,1.759);
\draw[gp path] (7.947,1.759)--(7.875,1.759);
\gpcolor{rgb color={0.000,0.000,0.000}}
\node[gp node right,font={\fontsize{10pt}{12pt}\selectfont}] at (1.012,1.759) {0.65};
\gpcolor{color=gp lt color axes}
\gpsetlinetype{gp lt axes}
\draw[gp path] (1.196,2.534)--(7.947,2.534);
\gpcolor{color=gp lt color border}
\gpsetlinetype{gp lt border}
\draw[gp path] (1.196,2.534)--(1.268,2.534);
\draw[gp path] (7.947,2.534)--(7.875,2.534);
\gpcolor{rgb color={0.000,0.000,0.000}}
\node[gp node right,font={\fontsize{10pt}{12pt}\selectfont}] at (1.012,2.534) {0.7};
\gpcolor{color=gp lt color axes}
\gpsetlinetype{gp lt axes}
\draw[gp path] (1.196,3.308)--(7.947,3.308);
\gpcolor{color=gp lt color border}
\gpsetlinetype{gp lt border}
\draw[gp path] (1.196,3.308)--(1.268,3.308);
\draw[gp path] (7.947,3.308)--(7.875,3.308);
\gpcolor{rgb color={0.000,0.000,0.000}}
\node[gp node right,font={\fontsize{10pt}{12pt}\selectfont}] at (1.012,3.308) {0.75};
\gpcolor{color=gp lt color axes}
\gpsetlinetype{gp lt axes}
\draw[gp path] (1.196,4.082)--(7.947,4.082);
\gpcolor{color=gp lt color border}
\gpsetlinetype{gp lt border}
\draw[gp path] (1.196,4.082)--(1.268,4.082);
\draw[gp path] (7.947,4.082)--(7.875,4.082);
\gpcolor{rgb color={0.000,0.000,0.000}}
\node[gp node right,font={\fontsize{10pt}{12pt}\selectfont}] at (1.012,4.082) {0.8};
\gpcolor{color=gp lt color axes}
\gpsetlinetype{gp lt axes}
\draw[gp path] (1.196,4.857)--(7.947,4.857);
\gpcolor{color=gp lt color border}
\gpsetlinetype{gp lt border}
\draw[gp path] (1.196,4.857)--(1.268,4.857);
\draw[gp path] (7.947,4.857)--(7.875,4.857);
\gpcolor{rgb color={0.000,0.000,0.000}}
\node[gp node right,font={\fontsize{10pt}{12pt}\selectfont}] at (1.012,4.857) {0.85};
\gpcolor{color=gp lt color axes}
\gpsetlinetype{gp lt axes}
\draw[gp path] (1.196,5.631)--(7.947,5.631);
\gpcolor{color=gp lt color border}
\gpsetlinetype{gp lt border}
\draw[gp path] (1.196,5.631)--(1.268,5.631);
\draw[gp path] (7.947,5.631)--(7.875,5.631);
\gpcolor{rgb color={0.000,0.000,0.000}}
\node[gp node right,font={\fontsize{10pt}{12pt}\selectfont}] at (1.012,5.631) {0.9};
\gpcolor{color=gp lt color axes}
\gpsetlinetype{gp lt axes}
\draw[gp path] (1.196,0.985)--(1.196,5.631);
\gpcolor{color=gp lt color border}
\gpsetlinetype{gp lt border}
\draw[gp path] (1.196,0.985)--(1.196,1.057);
\draw[gp path] (1.196,5.631)--(1.196,5.559);
\gpcolor{rgb color={0.000,0.000,0.000}}
\node[gp node center,font={\fontsize{10pt}{12pt}\selectfont}] at (1.196,0.677) {0.2};
\gpcolor{color=gp lt color axes}
\gpsetlinetype{gp lt axes}
\draw[gp path] (2.321,0.985)--(2.321,5.631);
\gpcolor{color=gp lt color border}
\gpsetlinetype{gp lt border}
\draw[gp path] (2.321,0.985)--(2.321,1.057);
\draw[gp path] (2.321,5.631)--(2.321,5.559);
\gpcolor{rgb color={0.000,0.000,0.000}}
\node[gp node center,font={\fontsize{10pt}{12pt}\selectfont}] at (2.321,0.677) {0.25};
\gpcolor{color=gp lt color axes}
\gpsetlinetype{gp lt axes}
\draw[gp path] (3.446,0.985)--(3.446,5.631);
\gpcolor{color=gp lt color border}
\gpsetlinetype{gp lt border}
\draw[gp path] (3.446,0.985)--(3.446,1.057);
\draw[gp path] (3.446,5.631)--(3.446,5.559);
\gpcolor{rgb color={0.000,0.000,0.000}}
\node[gp node center,font={\fontsize{10pt}{12pt}\selectfont}] at (3.446,0.677) {0.3};
\gpcolor{color=gp lt color axes}
\gpsetlinetype{gp lt axes}
\draw[gp path] (4.572,0.985)--(4.572,5.631);
\gpcolor{color=gp lt color border}
\gpsetlinetype{gp lt border}
\draw[gp path] (4.572,0.985)--(4.572,1.057);
\draw[gp path] (4.572,5.631)--(4.572,5.559);
\gpcolor{rgb color={0.000,0.000,0.000}}
\node[gp node center,font={\fontsize{10pt}{12pt}\selectfont}] at (4.572,0.677) {0.35};
\gpcolor{color=gp lt color axes}
\gpsetlinetype{gp lt axes}
\draw[gp path] (5.697,0.985)--(5.697,5.631);
\gpcolor{color=gp lt color border}
\gpsetlinetype{gp lt border}
\draw[gp path] (5.697,0.985)--(5.697,1.057);
\draw[gp path] (5.697,5.631)--(5.697,5.559);
\gpcolor{rgb color={0.000,0.000,0.000}}
\node[gp node center,font={\fontsize{10pt}{12pt}\selectfont}] at (5.697,0.677) {0.4};
\gpcolor{color=gp lt color axes}
\gpsetlinetype{gp lt axes}
\draw[gp path] (6.822,0.985)--(6.822,5.631);
\gpcolor{color=gp lt color border}
\gpsetlinetype{gp lt border}
\draw[gp path] (6.822,0.985)--(6.822,1.057);
\draw[gp path] (6.822,5.631)--(6.822,5.559);
\gpcolor{rgb color={0.000,0.000,0.000}}
\node[gp node center,font={\fontsize{10pt}{12pt}\selectfont}] at (6.822,0.677) {0.45};
\gpcolor{color=gp lt color axes}
\gpsetlinetype{gp lt axes}
\draw[gp path] (7.947,0.985)--(7.947,5.631);
\gpcolor{color=gp lt color border}
\gpsetlinetype{gp lt border}
\draw[gp path] (7.947,0.985)--(7.947,1.057);
\draw[gp path] (7.947,5.631)--(7.947,5.559);
\gpcolor{rgb color={0.000,0.000,0.000}}
\node[gp node center,font={\fontsize{10pt}{12pt}\selectfont}] at (7.947,0.677) {0.5};
\gpcolor{color=gp lt color border}
\draw[gp path] (1.196,5.631)--(1.196,0.985)--(7.947,0.985)--(7.947,5.631)--cycle;
\gpcolor{rgb color={0.000,0.000,0.000}}
\node[gp node center,font={\fontsize{10pt}{12pt}\selectfont}] at (4.571,0.215) {\large $x$};
\gpcolor{rgb color={0.502,0.502,0.502}}
\gpsetlinewidth{0.50}
\gpsetpointsize{2.67}
\gppoint{gp mark 7}{(1.206,4.163)}
\gppoint{gp mark 7}{(1.217,4.147)}
\gppoint{gp mark 7}{(1.228,4.131)}
\gppoint{gp mark 7}{(1.239,4.114)}
\gppoint{gp mark 7}{(1.250,4.098)}
\gppoint{gp mark 7}{(1.261,4.081)}
\gppoint{gp mark 7}{(1.272,4.065)}
\gppoint{gp mark 7}{(1.283,4.049)}
\gppoint{gp mark 7}{(1.294,4.032)}
\gppoint{gp mark 7}{(1.305,4.016)}
\gppoint{gp mark 7}{(1.316,4.000)}
\gppoint{gp mark 7}{(1.327,3.983)}
\gppoint{gp mark 7}{(1.338,3.967)}
\gppoint{gp mark 7}{(1.349,3.951)}
\gppoint{gp mark 7}{(1.360,3.935)}
\gppoint{gp mark 7}{(1.371,3.918)}
\gppoint{gp mark 7}{(1.382,3.902)}
\gppoint{gp mark 7}{(1.393,3.886)}
\gppoint{gp mark 7}{(1.404,3.870)}
\gppoint{gp mark 7}{(1.415,3.853)}
\gppoint{gp mark 7}{(1.426,3.837)}
\gppoint{gp mark 7}{(1.437,3.821)}
\gppoint{gp mark 7}{(1.448,3.805)}
\gppoint{gp mark 7}{(1.459,3.789)}
\gppoint{gp mark 7}{(1.470,3.773)}
\gppoint{gp mark 7}{(1.481,3.757)}
\gppoint{gp mark 7}{(1.492,3.740)}
\gppoint{gp mark 7}{(1.503,3.724)}
\gppoint{gp mark 7}{(1.514,3.708)}
\gppoint{gp mark 7}{(1.525,3.692)}
\gppoint{gp mark 7}{(1.536,3.676)}
\gppoint{gp mark 7}{(1.547,3.660)}
\gppoint{gp mark 7}{(1.558,3.644)}
\gppoint{gp mark 7}{(1.568,3.628)}
\gppoint{gp mark 7}{(1.579,3.612)}
\gppoint{gp mark 7}{(1.590,3.596)}
\gppoint{gp mark 7}{(1.601,3.580)}
\gppoint{gp mark 7}{(1.612,3.564)}
\gppoint{gp mark 7}{(1.623,3.548)}
\gppoint{gp mark 7}{(1.634,3.532)}
\gppoint{gp mark 7}{(1.645,3.516)}
\gppoint{gp mark 7}{(1.656,3.500)}
\gppoint{gp mark 7}{(1.667,3.485)}
\gppoint{gp mark 7}{(1.678,3.469)}
\gppoint{gp mark 7}{(1.689,3.453)}
\gppoint{gp mark 7}{(1.700,3.437)}
\gppoint{gp mark 7}{(1.711,3.421)}
\gppoint{gp mark 7}{(1.722,3.405)}
\gppoint{gp mark 7}{(1.733,3.389)}
\gppoint{gp mark 7}{(1.744,3.374)}
\gppoint{gp mark 7}{(1.755,3.358)}
\gppoint{gp mark 7}{(1.766,3.342)}
\gppoint{gp mark 7}{(1.777,3.326)}
\gppoint{gp mark 7}{(1.788,3.311)}
\gppoint{gp mark 7}{(1.799,3.295)}
\gppoint{gp mark 7}{(1.810,3.279)}
\gppoint{gp mark 7}{(1.821,3.263)}
\gppoint{gp mark 7}{(1.832,3.248)}
\gppoint{gp mark 7}{(1.843,3.232)}
\gppoint{gp mark 7}{(1.854,3.216)}
\gppoint{gp mark 7}{(1.865,3.201)}
\gppoint{gp mark 7}{(1.876,3.185)}
\gppoint{gp mark 7}{(1.887,3.169)}
\gppoint{gp mark 7}{(1.898,3.154)}
\gppoint{gp mark 7}{(1.909,3.138)}
\gppoint{gp mark 7}{(1.920,3.123)}
\gppoint{gp mark 7}{(1.931,3.107)}
\gppoint{gp mark 7}{(1.942,3.092)}
\gppoint{gp mark 7}{(1.953,3.076)}
\gppoint{gp mark 7}{(1.964,3.060)}
\gppoint{gp mark 7}{(1.975,3.045)}
\gppoint{gp mark 7}{(1.986,3.029)}
\gppoint{gp mark 7}{(1.997,3.014)}
\gppoint{gp mark 7}{(2.008,2.998)}
\gppoint{gp mark 7}{(2.019,2.983)}
\gppoint{gp mark 7}{(2.030,2.968)}
\gppoint{gp mark 7}{(2.041,2.952)}
\gppoint{gp mark 7}{(2.052,2.937)}
\gppoint{gp mark 7}{(2.063,2.921)}
\gppoint{gp mark 7}{(2.074,2.906)}
\gppoint{gp mark 7}{(2.085,2.890)}
\gppoint{gp mark 7}{(2.096,2.875)}
\gppoint{gp mark 7}{(2.107,2.860)}
\gppoint{gp mark 7}{(2.118,2.844)}
\gppoint{gp mark 7}{(2.129,2.829)}
\gppoint{gp mark 7}{(2.140,2.814)}
\gppoint{gp mark 7}{(2.151,2.798)}
\gppoint{gp mark 7}{(2.162,2.783)}
\gppoint{gp mark 7}{(2.173,2.768)}
\gppoint{gp mark 7}{(2.184,2.753)}
\gppoint{gp mark 7}{(2.195,2.737)}
\gppoint{gp mark 7}{(2.206,2.722)}
\gppoint{gp mark 7}{(2.217,2.707)}
\gppoint{gp mark 7}{(2.228,2.692)}
\gppoint{gp mark 7}{(2.239,2.677)}
\gppoint{gp mark 7}{(2.250,2.661)}
\gppoint{gp mark 7}{(2.261,2.646)}
\gppoint{gp mark 7}{(2.272,2.631)}
\gppoint{gp mark 7}{(2.283,2.616)}
\gppoint{gp mark 7}{(2.294,2.601)}
\gppoint{gp mark 7}{(2.305,2.586)}
\gppoint{gp mark 7}{(2.316,2.571)}
\gppoint{gp mark 7}{(2.327,2.556)}
\gppoint{gp mark 7}{(2.338,2.540)}
\gppoint{gp mark 7}{(2.349,2.525)}
\gppoint{gp mark 7}{(2.360,2.510)}
\gppoint{gp mark 7}{(2.371,2.495)}
\gppoint{gp mark 7}{(2.382,2.480)}
\gppoint{gp mark 7}{(2.393,2.465)}
\gppoint{gp mark 7}{(2.404,2.450)}
\gppoint{gp mark 7}{(2.415,2.435)}
\gppoint{gp mark 7}{(2.426,2.420)}
\gppoint{gp mark 7}{(2.437,2.406)}
\gppoint{gp mark 7}{(2.448,2.391)}
\gppoint{gp mark 7}{(2.459,2.376)}
\gppoint{gp mark 7}{(2.470,2.361)}
\gppoint{gp mark 7}{(2.480,2.346)}
\gppoint{gp mark 7}{(2.491,2.331)}
\gppoint{gp mark 7}{(2.502,2.316)}
\gppoint{gp mark 7}{(2.513,2.301)}
\gppoint{gp mark 7}{(2.524,2.287)}
\gppoint{gp mark 7}{(2.535,2.272)}
\gppoint{gp mark 7}{(2.546,2.257)}
\gppoint{gp mark 7}{(2.557,2.242)}
\gppoint{gp mark 7}{(2.568,2.228)}
\gppoint{gp mark 7}{(2.579,2.213)}
\gppoint{gp mark 7}{(2.590,2.198)}
\gppoint{gp mark 7}{(2.601,2.183)}
\gppoint{gp mark 7}{(2.612,2.169)}
\gppoint{gp mark 7}{(2.623,2.154)}
\gppoint{gp mark 7}{(2.634,2.139)}
\gppoint{gp mark 7}{(2.645,2.125)}
\gppoint{gp mark 7}{(2.656,2.110)}
\gppoint{gp mark 7}{(2.667,2.096)}
\gppoint{gp mark 7}{(2.678,2.081)}
\gppoint{gp mark 7}{(2.689,2.067)}
\gppoint{gp mark 7}{(2.700,2.052)}
\gppoint{gp mark 7}{(2.711,2.038)}
\gppoint{gp mark 7}{(2.722,2.023)}
\gppoint{gp mark 7}{(2.733,2.009)}
\gppoint{gp mark 7}{(2.744,1.994)}
\gppoint{gp mark 7}{(2.755,1.980)}
\gppoint{gp mark 7}{(2.766,1.966)}
\gppoint{gp mark 7}{(2.777,1.951)}
\gppoint{gp mark 7}{(2.788,1.937)}
\gppoint{gp mark 7}{(2.799,1.923)}
\gppoint{gp mark 7}{(2.810,1.909)}
\gppoint{gp mark 7}{(2.821,1.894)}
\gppoint{gp mark 7}{(2.832,1.880)}
\gppoint{gp mark 7}{(2.843,1.866)}
\gppoint{gp mark 7}{(2.854,1.852)}
\gppoint{gp mark 7}{(2.865,1.838)}
\gppoint{gp mark 7}{(2.876,1.824)}
\gppoint{gp mark 7}{(2.887,1.810)}
\gppoint{gp mark 7}{(2.898,1.795)}
\gppoint{gp mark 7}{(2.909,1.780)}
\gppoint{gp mark 7}{(2.920,1.763)}
\gppoint{gp mark 7}{(2.931,1.745)}
\gppoint{gp mark 7}{(2.942,1.728)}
\gppoint{gp mark 7}{(2.953,1.716)}
\gppoint{gp mark 7}{(2.964,1.713)}
\gppoint{gp mark 7}{(2.975,1.713)}
\gppoint{gp mark 7}{(2.986,1.714)}
\gppoint{gp mark 7}{(2.997,1.730)}
\gppoint{gp mark 7}{(3.008,1.808)}
\gppoint{gp mark 7}{(3.019,1.977)}
\gppoint{gp mark 7}{(3.030,2.143)}
\gppoint{gp mark 7}{(3.041,2.207)}
\gppoint{gp mark 7}{(3.052,2.216)}
\gppoint{gp mark 7}{(3.063,2.217)}
\gppoint{gp mark 7}{(3.074,2.218)}
\gppoint{gp mark 7}{(3.085,2.218)}
\gppoint{gp mark 7}{(3.096,2.220)}
\gppoint{gp mark 7}{(3.107,2.223)}
\gppoint{gp mark 7}{(3.118,2.227)}
\gppoint{gp mark 7}{(3.129,2.231)}
\gppoint{gp mark 7}{(3.140,2.235)}
\gppoint{gp mark 7}{(3.151,2.240)}
\gppoint{gp mark 7}{(3.162,2.245)}
\gppoint{gp mark 7}{(3.173,2.249)}
\gppoint{gp mark 7}{(3.184,2.253)}
\gppoint{gp mark 7}{(3.195,2.256)}
\gppoint{gp mark 7}{(3.206,2.259)}
\gppoint{gp mark 7}{(3.217,2.262)}
\gppoint{gp mark 7}{(3.228,2.265)}
\gppoint{gp mark 7}{(3.239,2.268)}
\gppoint{gp mark 7}{(3.250,2.270)}
\gppoint{gp mark 7}{(3.261,2.272)}
\gppoint{gp mark 7}{(3.272,2.275)}
\gppoint{gp mark 7}{(3.283,2.278)}
\gppoint{gp mark 7}{(3.294,2.280)}
\gppoint{gp mark 7}{(3.305,2.282)}
\gppoint{gp mark 7}{(3.316,2.285)}
\gppoint{gp mark 7}{(3.327,2.287)}
\gppoint{gp mark 7}{(3.338,2.289)}
\gppoint{gp mark 7}{(3.349,2.290)}
\gppoint{gp mark 7}{(3.360,2.292)}
\gppoint{gp mark 7}{(3.371,2.294)}
\gppoint{gp mark 7}{(3.382,2.296)}
\gppoint{gp mark 7}{(3.392,2.298)}
\gppoint{gp mark 7}{(3.403,2.300)}
\gppoint{gp mark 7}{(3.414,2.301)}
\gppoint{gp mark 7}{(3.425,2.303)}
\gppoint{gp mark 7}{(3.436,2.304)}
\gppoint{gp mark 7}{(3.447,2.306)}
\gppoint{gp mark 7}{(3.458,2.307)}
\gppoint{gp mark 7}{(3.469,2.309)}
\gppoint{gp mark 7}{(3.480,2.311)}
\gppoint{gp mark 7}{(3.491,2.323)}
\gppoint{gp mark 7}{(3.502,2.342)}
\gppoint{gp mark 7}{(3.513,2.385)}
\gppoint{gp mark 7}{(3.524,2.668)}
\gppoint{gp mark 7}{(3.535,2.821)}
\gppoint{gp mark 7}{(3.546,2.701)}
\gppoint{gp mark 7}{(3.557,2.710)}
\gppoint{gp mark 7}{(3.568,2.711)}
\gppoint{gp mark 7}{(3.579,2.708)}
\gppoint{gp mark 7}{(3.590,2.716)}
\gppoint{gp mark 7}{(3.601,2.726)}
\gppoint{gp mark 7}{(3.612,2.730)}
\gppoint{gp mark 7}{(3.623,2.730)}
\gppoint{gp mark 7}{(3.634,2.735)}
\gppoint{gp mark 7}{(3.645,2.742)}
\gppoint{gp mark 7}{(3.656,2.749)}
\gppoint{gp mark 7}{(3.667,2.754)}
\gppoint{gp mark 7}{(3.678,2.758)}
\gppoint{gp mark 7}{(3.689,2.765)}
\gppoint{gp mark 7}{(3.700,2.773)}
\gppoint{gp mark 7}{(3.711,2.781)}
\gppoint{gp mark 7}{(3.722,2.785)}
\gppoint{gp mark 7}{(3.733,2.791)}
\gppoint{gp mark 7}{(3.744,2.800)}
\gppoint{gp mark 7}{(3.755,2.810)}
\gppoint{gp mark 7}{(3.766,2.818)}
\gppoint{gp mark 7}{(3.777,2.826)}
\gppoint{gp mark 7}{(3.788,2.835)}
\gppoint{gp mark 7}{(3.799,2.845)}
\gppoint{gp mark 7}{(3.810,2.861)}
\gppoint{gp mark 7}{(3.821,2.938)}
\gppoint{gp mark 7}{(3.832,3.305)}
\gppoint{gp mark 7}{(3.843,3.919)}
\gppoint{gp mark 7}{(3.854,4.090)}
\gppoint{gp mark 7}{(3.865,4.099)}
\gppoint{gp mark 7}{(3.876,4.076)}
\gppoint{gp mark 7}{(3.887,4.026)}
\gppoint{gp mark 7}{(3.898,3.997)}
\gppoint{gp mark 7}{(3.909,3.989)}
\gppoint{gp mark 7}{(3.920,3.975)}
\gppoint{gp mark 7}{(3.931,3.937)}
\gppoint{gp mark 7}{(3.942,3.893)}
\gppoint{gp mark 7}{(3.953,3.860)}
\gppoint{gp mark 7}{(3.964,3.836)}
\gppoint{gp mark 7}{(3.975,3.816)}
\gppoint{gp mark 7}{(3.986,3.793)}
\gppoint{gp mark 7}{(3.997,3.765)}
\gppoint{gp mark 7}{(4.008,3.733)}
\gppoint{gp mark 7}{(4.019,3.704)}
\gppoint{gp mark 7}{(4.030,3.679)}
\gppoint{gp mark 7}{(4.041,3.654)}
\gppoint{gp mark 7}{(4.052,3.628)}
\gppoint{gp mark 7}{(4.063,3.603)}
\gppoint{gp mark 7}{(4.074,3.577)}
\gppoint{gp mark 7}{(4.085,3.553)}
\gppoint{gp mark 7}{(4.096,3.531)}
\gppoint{gp mark 7}{(4.107,3.512)}
\gppoint{gp mark 7}{(4.118,3.501)}
\gppoint{gp mark 7}{(4.129,3.497)}
\gppoint{gp mark 7}{(4.140,3.496)}
\gppoint{gp mark 7}{(4.151,3.497)}
\gppoint{gp mark 7}{(4.162,3.498)}
\gppoint{gp mark 7}{(4.173,3.500)}
\gppoint{gp mark 7}{(4.184,3.503)}
\gppoint{gp mark 7}{(4.195,3.506)}
\gppoint{gp mark 7}{(4.206,3.508)}
\gppoint{gp mark 7}{(4.217,3.509)}
\gppoint{gp mark 7}{(4.228,3.508)}
\gppoint{gp mark 7}{(4.239,3.508)}
\gppoint{gp mark 7}{(4.250,3.507)}
\gppoint{gp mark 7}{(4.261,3.507)}
\gppoint{gp mark 7}{(4.272,3.507)}
\gppoint{gp mark 7}{(4.283,3.507)}
\gppoint{gp mark 7}{(4.294,3.506)}
\gppoint{gp mark 7}{(4.304,3.505)}
\gppoint{gp mark 7}{(4.315,3.505)}
\gppoint{gp mark 7}{(4.326,3.505)}
\gppoint{gp mark 7}{(4.337,3.505)}
\gppoint{gp mark 7}{(4.348,3.505)}
\gppoint{gp mark 7}{(4.359,3.504)}
\gppoint{gp mark 7}{(4.370,3.504)}
\gppoint{gp mark 7}{(4.381,3.504)}
\gppoint{gp mark 7}{(4.392,3.503)}
\gppoint{gp mark 7}{(4.403,3.503)}
\gppoint{gp mark 7}{(4.414,3.502)}
\gppoint{gp mark 7}{(4.425,3.502)}
\gppoint{gp mark 7}{(4.436,3.502)}
\gppoint{gp mark 7}{(4.447,3.502)}
\gppoint{gp mark 7}{(4.458,3.502)}
\gppoint{gp mark 7}{(4.469,3.501)}
\gppoint{gp mark 7}{(4.480,3.501)}
\gppoint{gp mark 7}{(4.491,3.500)}
\gppoint{gp mark 7}{(4.502,3.500)}
\gppoint{gp mark 7}{(4.513,3.500)}
\gppoint{gp mark 7}{(4.524,3.499)}
\gppoint{gp mark 7}{(4.535,3.499)}
\gppoint{gp mark 7}{(4.546,3.499)}
\gppoint{gp mark 7}{(4.557,3.499)}
\gppoint{gp mark 7}{(4.568,3.499)}
\gppoint{gp mark 7}{(4.579,3.498)}
\gppoint{gp mark 7}{(4.590,3.498)}
\gppoint{gp mark 7}{(4.601,3.498)}
\gppoint{gp mark 7}{(4.612,3.497)}
\gppoint{gp mark 7}{(4.623,3.497)}
\gppoint{gp mark 7}{(4.634,3.497)}
\gppoint{gp mark 7}{(4.645,3.496)}
\gppoint{gp mark 7}{(4.656,3.496)}
\gppoint{gp mark 7}{(4.667,3.496)}
\gppoint{gp mark 7}{(4.678,3.496)}
\gppoint{gp mark 7}{(4.689,3.495)}
\gppoint{gp mark 7}{(4.700,3.495)}
\gppoint{gp mark 7}{(4.711,3.494)}
\gppoint{gp mark 7}{(4.722,3.494)}
\gppoint{gp mark 7}{(4.733,3.493)}
\gppoint{gp mark 7}{(4.744,3.493)}
\gppoint{gp mark 7}{(4.755,3.493)}
\gppoint{gp mark 7}{(4.766,3.493)}
\gppoint{gp mark 7}{(4.777,3.492)}
\gppoint{gp mark 7}{(4.788,3.492)}
\gppoint{gp mark 7}{(4.799,3.491)}
\gppoint{gp mark 7}{(4.810,3.491)}
\gppoint{gp mark 7}{(4.821,3.491)}
\gppoint{gp mark 7}{(4.832,3.490)}
\gppoint{gp mark 7}{(4.843,3.490)}
\gppoint{gp mark 7}{(4.854,3.490)}
\gppoint{gp mark 7}{(4.865,3.489)}
\gppoint{gp mark 7}{(4.876,3.489)}
\gppoint{gp mark 7}{(4.887,3.489)}
\gppoint{gp mark 7}{(4.898,3.489)}
\gppoint{gp mark 7}{(4.909,3.488)}
\gppoint{gp mark 7}{(4.920,3.487)}
\gppoint{gp mark 7}{(4.931,3.487)}
\gppoint{gp mark 7}{(4.942,3.487)}
\gppoint{gp mark 7}{(4.953,3.486)}
\gppoint{gp mark 7}{(4.964,3.486)}
\gppoint{gp mark 7}{(4.975,3.486)}
\gppoint{gp mark 7}{(4.986,3.485)}
\gppoint{gp mark 7}{(4.997,3.485)}
\gppoint{gp mark 7}{(5.008,3.484)}
\gppoint{gp mark 7}{(5.019,3.484)}
\gppoint{gp mark 7}{(5.030,3.483)}
\gppoint{gp mark 7}{(5.041,3.483)}
\gppoint{gp mark 7}{(5.052,3.483)}
\gppoint{gp mark 7}{(5.063,3.482)}
\gppoint{gp mark 7}{(5.074,3.482)}
\gppoint{gp mark 7}{(5.085,3.482)}
\gppoint{gp mark 7}{(5.096,3.482)}
\gppoint{gp mark 7}{(5.107,3.481)}
\gppoint{gp mark 7}{(5.118,3.481)}
\gppoint{gp mark 7}{(5.129,3.480)}
\gppoint{gp mark 7}{(5.140,3.480)}
\gppoint{gp mark 7}{(5.151,3.479)}
\gppoint{gp mark 7}{(5.162,3.479)}
\gppoint{gp mark 7}{(5.173,3.479)}
\gppoint{gp mark 7}{(5.184,3.478)}
\gppoint{gp mark 7}{(5.195,3.478)}
\gppoint{gp mark 7}{(5.206,3.478)}
\gppoint{gp mark 7}{(5.216,3.478)}
\gppoint{gp mark 7}{(5.227,3.477)}
\gppoint{gp mark 7}{(5.238,3.476)}
\gppoint{gp mark 7}{(5.249,3.476)}
\gppoint{gp mark 7}{(5.260,3.475)}
\gppoint{gp mark 7}{(5.271,3.475)}
\gppoint{gp mark 7}{(5.282,3.475)}
\gppoint{gp mark 7}{(5.293,3.475)}
\gppoint{gp mark 7}{(5.304,3.474)}
\gppoint{gp mark 7}{(5.315,3.474)}
\gppoint{gp mark 7}{(5.326,3.474)}
\gppoint{gp mark 7}{(5.337,3.473)}
\gppoint{gp mark 7}{(5.348,3.473)}
\gppoint{gp mark 7}{(5.359,3.472)}
\gppoint{gp mark 7}{(5.370,3.472)}
\gppoint{gp mark 7}{(5.381,3.472)}
\gppoint{gp mark 7}{(5.392,3.471)}
\gppoint{gp mark 7}{(5.403,3.471)}
\gppoint{gp mark 7}{(5.414,3.471)}
\gppoint{gp mark 7}{(5.425,3.470)}
\gppoint{gp mark 7}{(5.436,3.470)}
\gppoint{gp mark 7}{(5.447,3.469)}
\gppoint{gp mark 7}{(5.458,3.469)}
\gppoint{gp mark 7}{(5.469,3.468)}
\gppoint{gp mark 7}{(5.480,3.468)}
\gppoint{gp mark 7}{(5.491,3.467)}
\gppoint{gp mark 7}{(5.502,3.467)}
\gppoint{gp mark 7}{(5.513,3.466)}
\gppoint{gp mark 7}{(5.524,3.465)}
\gppoint{gp mark 7}{(5.535,3.465)}
\gppoint{gp mark 7}{(5.546,3.464)}
\gppoint{gp mark 7}{(5.557,3.464)}
\gppoint{gp mark 7}{(5.568,3.463)}
\gppoint{gp mark 7}{(5.579,3.462)}
\gppoint{gp mark 7}{(5.590,3.461)}
\gppoint{gp mark 7}{(5.601,3.460)}
\gppoint{gp mark 7}{(5.612,3.459)}
\gppoint{gp mark 7}{(5.623,3.458)}
\gppoint{gp mark 7}{(5.634,3.457)}
\gppoint{gp mark 7}{(5.645,3.456)}
\gppoint{gp mark 7}{(5.656,3.455)}
\gppoint{gp mark 7}{(5.667,3.453)}
\gppoint{gp mark 7}{(5.678,3.452)}
\gppoint{gp mark 7}{(5.689,3.451)}
\gppoint{gp mark 7}{(5.700,3.449)}
\gppoint{gp mark 7}{(5.711,3.447)}
\gppoint{gp mark 7}{(5.722,3.446)}
\gppoint{gp mark 7}{(5.733,3.444)}
\gppoint{gp mark 7}{(5.744,3.442)}
\gppoint{gp mark 7}{(5.755,3.440)}
\gppoint{gp mark 7}{(5.766,3.438)}
\gppoint{gp mark 7}{(5.777,3.435)}
\gppoint{gp mark 7}{(5.788,3.433)}
\gppoint{gp mark 7}{(5.799,3.430)}
\gppoint{gp mark 7}{(5.810,3.428)}
\gppoint{gp mark 7}{(5.821,3.425)}
\gppoint{gp mark 7}{(5.832,3.423)}
\gppoint{gp mark 7}{(5.843,3.421)}
\gppoint{gp mark 7}{(5.854,3.418)}
\gppoint{gp mark 7}{(5.865,3.416)}
\gppoint{gp mark 7}{(5.876,3.413)}
\gppoint{gp mark 7}{(5.887,3.411)}
\gppoint{gp mark 7}{(5.898,3.409)}
\gppoint{gp mark 7}{(5.909,3.407)}
\gppoint{gp mark 7}{(5.920,3.405)}
\gppoint{gp mark 7}{(5.931,3.403)}
\gppoint{gp mark 7}{(5.942,3.401)}
\gppoint{gp mark 7}{(5.953,3.399)}
\gppoint{gp mark 7}{(5.964,3.397)}
\gppoint{gp mark 7}{(5.975,3.395)}
\gppoint{gp mark 7}{(5.986,3.394)}
\gppoint{gp mark 7}{(5.997,3.392)}
\gppoint{gp mark 7}{(6.008,3.390)}
\gppoint{gp mark 7}{(6.019,3.388)}
\gppoint{gp mark 7}{(6.030,3.387)}
\gppoint{gp mark 7}{(6.041,3.386)}
\gppoint{gp mark 7}{(6.052,3.384)}
\gppoint{gp mark 7}{(6.063,3.383)}
\gppoint{gp mark 7}{(6.074,3.382)}
\gppoint{gp mark 7}{(6.085,3.382)}
\gppoint{gp mark 7}{(6.096,3.381)}
\gppoint{gp mark 7}{(6.107,3.380)}
\gppoint{gp mark 7}{(6.118,3.378)}
\gppoint{gp mark 7}{(6.128,3.377)}
\gppoint{gp mark 7}{(6.139,3.376)}
\gppoint{gp mark 7}{(6.150,3.375)}
\gppoint{gp mark 7}{(6.161,3.374)}
\gppoint{gp mark 7}{(6.172,3.373)}
\gppoint{gp mark 7}{(6.183,3.372)}
\gppoint{gp mark 7}{(6.194,3.371)}
\gppoint{gp mark 7}{(6.205,3.370)}
\gppoint{gp mark 7}{(6.216,3.369)}
\gppoint{gp mark 7}{(6.227,3.368)}
\gppoint{gp mark 7}{(6.238,3.367)}
\gppoint{gp mark 7}{(6.249,3.366)}
\gppoint{gp mark 7}{(6.260,3.366)}
\gppoint{gp mark 7}{(6.271,3.365)}
\gppoint{gp mark 7}{(6.282,3.364)}
\gppoint{gp mark 7}{(6.293,3.363)}
\gppoint{gp mark 7}{(6.304,3.362)}
\gppoint{gp mark 7}{(6.315,3.361)}
\gppoint{gp mark 7}{(6.326,3.361)}
\gppoint{gp mark 7}{(6.337,3.360)}
\gppoint{gp mark 7}{(6.348,3.360)}
\gppoint{gp mark 7}{(6.359,3.359)}
\gppoint{gp mark 7}{(6.370,3.359)}
\gppoint{gp mark 7}{(6.381,3.359)}
\gppoint{gp mark 7}{(6.392,3.358)}
\gppoint{gp mark 7}{(6.403,3.358)}
\gppoint{gp mark 7}{(6.414,3.358)}
\gppoint{gp mark 7}{(6.425,3.357)}
\gppoint{gp mark 7}{(6.436,3.357)}
\gppoint{gp mark 7}{(6.447,3.357)}
\gppoint{gp mark 7}{(6.458,3.356)}
\gppoint{gp mark 7}{(6.469,3.356)}
\gppoint{gp mark 7}{(6.480,3.355)}
\gppoint{gp mark 7}{(6.491,3.355)}
\gppoint{gp mark 7}{(6.502,3.354)}
\gppoint{gp mark 7}{(6.513,3.354)}
\gppoint{gp mark 7}{(6.524,3.353)}
\gppoint{gp mark 7}{(6.535,3.353)}
\gppoint{gp mark 7}{(6.546,3.353)}
\gppoint{gp mark 7}{(6.557,3.352)}
\gppoint{gp mark 7}{(6.568,3.352)}
\gppoint{gp mark 7}{(6.579,3.351)}
\gppoint{gp mark 7}{(6.590,3.350)}
\gppoint{gp mark 7}{(6.601,3.350)}
\gppoint{gp mark 7}{(6.612,3.350)}
\gppoint{gp mark 7}{(6.623,3.349)}
\gppoint{gp mark 7}{(6.634,3.349)}
\gppoint{gp mark 7}{(6.645,3.349)}
\gppoint{gp mark 7}{(6.656,3.348)}
\gppoint{gp mark 7}{(6.667,3.348)}
\gppoint{gp mark 7}{(6.678,3.348)}
\gppoint{gp mark 7}{(6.689,3.347)}
\gppoint{gp mark 7}{(6.700,3.347)}
\gppoint{gp mark 7}{(6.711,3.347)}
\gppoint{gp mark 7}{(6.722,3.346)}
\gppoint{gp mark 7}{(6.733,3.346)}
\gppoint{gp mark 7}{(6.744,3.345)}
\gppoint{gp mark 7}{(6.755,3.345)}
\gppoint{gp mark 7}{(6.766,3.344)}
\gppoint{gp mark 7}{(6.777,3.344)}
\gppoint{gp mark 7}{(6.788,3.343)}
\gppoint{gp mark 7}{(6.799,3.343)}
\gppoint{gp mark 7}{(6.810,3.342)}
\gppoint{gp mark 7}{(6.821,3.342)}
\gppoint{gp mark 7}{(6.832,3.341)}
\gppoint{gp mark 7}{(6.843,3.340)}
\gppoint{gp mark 7}{(6.854,3.340)}
\gppoint{gp mark 7}{(6.865,3.339)}
\gppoint{gp mark 7}{(6.876,3.339)}
\gppoint{gp mark 7}{(6.887,3.339)}
\gppoint{gp mark 7}{(6.898,3.338)}
\gppoint{gp mark 7}{(6.909,3.337)}
\gppoint{gp mark 7}{(6.920,3.337)}
\gppoint{gp mark 7}{(6.931,3.336)}
\gppoint{gp mark 7}{(6.942,3.336)}
\gppoint{gp mark 7}{(6.953,3.336)}
\gppoint{gp mark 7}{(6.964,3.335)}
\gppoint{gp mark 7}{(6.975,3.335)}
\gppoint{gp mark 7}{(6.986,3.334)}
\gppoint{gp mark 7}{(6.997,3.334)}
\gppoint{gp mark 7}{(7.008,3.333)}
\gppoint{gp mark 7}{(7.019,3.333)}
\gppoint{gp mark 7}{(7.030,3.332)}
\gppoint{gp mark 7}{(7.040,3.332)}
\gppoint{gp mark 7}{(7.051,3.332)}
\gppoint{gp mark 7}{(7.062,3.331)}
\gppoint{gp mark 7}{(7.073,3.331)}
\gppoint{gp mark 7}{(7.084,3.330)}
\gppoint{gp mark 7}{(7.095,3.330)}
\gppoint{gp mark 7}{(7.106,3.329)}
\gppoint{gp mark 7}{(7.117,3.329)}
\gppoint{gp mark 7}{(7.128,3.329)}
\gppoint{gp mark 7}{(7.139,3.329)}
\gppoint{gp mark 7}{(7.150,3.328)}
\gppoint{gp mark 7}{(7.161,3.328)}
\gppoint{gp mark 7}{(7.172,3.327)}
\gppoint{gp mark 7}{(7.183,3.326)}
\gppoint{gp mark 7}{(7.194,3.326)}
\gppoint{gp mark 7}{(7.205,3.325)}
\gppoint{gp mark 7}{(7.216,3.324)}
\gppoint{gp mark 7}{(7.227,3.324)}
\gppoint{gp mark 7}{(7.238,3.323)}
\gppoint{gp mark 7}{(7.249,3.323)}
\gppoint{gp mark 7}{(7.260,3.322)}
\gppoint{gp mark 7}{(7.271,3.322)}
\gppoint{gp mark 7}{(7.282,3.322)}
\gppoint{gp mark 7}{(7.293,3.322)}
\gppoint{gp mark 7}{(7.304,3.322)}
\gppoint{gp mark 7}{(7.315,3.321)}
\gppoint{gp mark 7}{(7.326,3.322)}
\gppoint{gp mark 7}{(7.337,3.322)}
\gppoint{gp mark 7}{(7.348,3.323)}
\gppoint{gp mark 7}{(7.359,3.324)}
\gppoint{gp mark 7}{(7.370,3.324)}
\gppoint{gp mark 7}{(7.381,3.325)}
\gppoint{gp mark 7}{(7.392,3.326)}
\gppoint{gp mark 7}{(7.403,3.334)}
\gppoint{gp mark 7}{(7.414,3.349)}
\gppoint{gp mark 7}{(7.425,3.361)}
\gppoint{gp mark 7}{(7.436,3.366)}
\gppoint{gp mark 7}{(7.447,3.367)}
\gppoint{gp mark 7}{(7.458,3.367)}
\gppoint{gp mark 7}{(7.469,3.365)}
\gppoint{gp mark 7}{(7.480,3.347)}
\gppoint{gp mark 7}{(7.491,3.209)}
\gppoint{gp mark 7}{(7.502,2.576)}
\gppoint{gp mark 7}{(7.513,1.334)}
\gpcolor{rgb color={1.000,0.000,0.000}}
\gpsetpointsize{4.44}
\gppoint{gp mark 7}{(1.206,4.163)}
\gppoint{gp mark 7}{(1.217,4.146)}
\gppoint{gp mark 7}{(1.228,4.130)}
\gppoint{gp mark 7}{(1.239,4.114)}
\gppoint{gp mark 7}{(1.250,4.097)}
\gppoint{gp mark 7}{(1.261,4.081)}
\gppoint{gp mark 7}{(1.272,4.065)}
\gppoint{gp mark 7}{(1.283,4.048)}
\gppoint{gp mark 7}{(1.294,4.032)}
\gppoint{gp mark 7}{(1.305,4.016)}
\gppoint{gp mark 7}{(1.316,4.000)}
\gppoint{gp mark 7}{(1.327,3.983)}
\gppoint{gp mark 7}{(1.338,3.967)}
\gppoint{gp mark 7}{(1.349,3.951)}
\gppoint{gp mark 7}{(1.360,3.935)}
\gppoint{gp mark 7}{(1.371,3.919)}
\gppoint{gp mark 7}{(1.382,3.902)}
\gppoint{gp mark 7}{(1.393,3.886)}
\gppoint{gp mark 7}{(1.404,3.870)}
\gppoint{gp mark 7}{(1.415,3.854)}
\gppoint{gp mark 7}{(1.426,3.838)}
\gppoint{gp mark 7}{(1.437,3.822)}
\gppoint{gp mark 7}{(1.448,3.806)}
\gppoint{gp mark 7}{(1.459,3.789)}
\gppoint{gp mark 7}{(1.470,3.773)}
\gppoint{gp mark 7}{(1.481,3.757)}
\gppoint{gp mark 7}{(1.492,3.741)}
\gppoint{gp mark 7}{(1.503,3.725)}
\gppoint{gp mark 7}{(1.514,3.709)}
\gppoint{gp mark 7}{(1.525,3.693)}
\gppoint{gp mark 7}{(1.536,3.677)}
\gppoint{gp mark 7}{(1.547,3.661)}
\gppoint{gp mark 7}{(1.558,3.645)}
\gppoint{gp mark 7}{(1.568,3.630)}
\gppoint{gp mark 7}{(1.579,3.614)}
\gppoint{gp mark 7}{(1.590,3.598)}
\gppoint{gp mark 7}{(1.601,3.582)}
\gppoint{gp mark 7}{(1.612,3.566)}
\gppoint{gp mark 7}{(1.623,3.550)}
\gppoint{gp mark 7}{(1.634,3.534)}
\gppoint{gp mark 7}{(1.645,3.518)}
\gppoint{gp mark 7}{(1.656,3.503)}
\gppoint{gp mark 7}{(1.667,3.487)}
\gppoint{gp mark 7}{(1.678,3.471)}
\gppoint{gp mark 7}{(1.689,3.455)}
\gppoint{gp mark 7}{(1.700,3.440)}
\gppoint{gp mark 7}{(1.711,3.424)}
\gppoint{gp mark 7}{(1.722,3.408)}
\gppoint{gp mark 7}{(1.733,3.393)}
\gppoint{gp mark 7}{(1.744,3.377)}
\gppoint{gp mark 7}{(1.755,3.361)}
\gppoint{gp mark 7}{(1.766,3.346)}
\gppoint{gp mark 7}{(1.777,3.330)}
\gppoint{gp mark 7}{(1.788,3.315)}
\gppoint{gp mark 7}{(1.799,3.299)}
\gppoint{gp mark 7}{(1.810,3.283)}
\gppoint{gp mark 7}{(1.821,3.268)}
\gppoint{gp mark 7}{(1.832,3.252)}
\gppoint{gp mark 7}{(1.843,3.237)}
\gppoint{gp mark 7}{(1.854,3.222)}
\gppoint{gp mark 7}{(1.865,3.206)}
\gppoint{gp mark 7}{(1.876,3.191)}
\gppoint{gp mark 7}{(1.887,3.175)}
\gppoint{gp mark 7}{(1.898,3.160)}
\gppoint{gp mark 7}{(1.909,3.145)}
\gppoint{gp mark 7}{(1.920,3.129)}
\gppoint{gp mark 7}{(1.931,3.114)}
\gppoint{gp mark 7}{(1.942,3.099)}
\gppoint{gp mark 7}{(1.953,3.084)}
\gppoint{gp mark 7}{(1.964,3.069)}
\gppoint{gp mark 7}{(1.975,3.053)}
\gppoint{gp mark 7}{(1.986,3.038)}
\gppoint{gp mark 7}{(1.997,3.023)}
\gppoint{gp mark 7}{(2.008,3.008)}
\gppoint{gp mark 7}{(2.019,2.996)}
\gppoint{gp mark 7}{(2.030,2.988)}
\gppoint{gp mark 7}{(2.041,2.977)}
\gppoint{gp mark 7}{(2.052,2.964)}
\gppoint{gp mark 7}{(2.063,2.949)}
\gppoint{gp mark 7}{(2.074,2.934)}
\gppoint{gp mark 7}{(2.085,2.919)}
\gppoint{gp mark 7}{(2.096,2.904)}
\gppoint{gp mark 7}{(2.107,2.890)}
\gppoint{gp mark 7}{(2.118,2.875)}
\gppoint{gp mark 7}{(2.129,2.861)}
\gppoint{gp mark 7}{(2.140,2.846)}
\gppoint{gp mark 7}{(2.151,2.832)}
\gppoint{gp mark 7}{(2.162,2.817)}
\gppoint{gp mark 7}{(2.173,2.803)}
\gppoint{gp mark 7}{(2.184,2.789)}
\gppoint{gp mark 7}{(2.195,2.774)}
\gppoint{gp mark 7}{(2.206,2.760)}
\gppoint{gp mark 7}{(2.217,2.746)}
\gppoint{gp mark 7}{(2.228,2.732)}
\gppoint{gp mark 7}{(2.239,2.718)}
\gppoint{gp mark 7}{(2.250,2.704)}
\gppoint{gp mark 7}{(2.261,2.690)}
\gppoint{gp mark 7}{(2.272,2.677)}
\gppoint{gp mark 7}{(2.283,2.663)}
\gppoint{gp mark 7}{(2.294,2.650)}
\gppoint{gp mark 7}{(2.305,2.636)}
\gppoint{gp mark 7}{(2.316,2.623)}
\gppoint{gp mark 7}{(2.327,2.610)}
\gppoint{gp mark 7}{(2.338,2.596)}
\gppoint{gp mark 7}{(2.349,2.583)}
\gppoint{gp mark 7}{(2.360,2.571)}
\gppoint{gp mark 7}{(2.371,2.558)}
\gppoint{gp mark 7}{(2.382,2.545)}
\gppoint{gp mark 7}{(2.393,2.532)}
\gppoint{gp mark 7}{(2.404,2.520)}
\gppoint{gp mark 7}{(2.415,2.508)}
\gppoint{gp mark 7}{(2.426,2.496)}
\gppoint{gp mark 7}{(2.437,2.484)}
\gppoint{gp mark 7}{(2.448,2.472)}
\gppoint{gp mark 7}{(2.459,2.461)}
\gppoint{gp mark 7}{(2.470,2.451)}
\gppoint{gp mark 7}{(2.480,2.443)}
\gppoint{gp mark 7}{(2.491,2.436)}
\gppoint{gp mark 7}{(2.502,2.433)}
\gppoint{gp mark 7}{(2.513,2.432)}
\gppoint{gp mark 7}{(2.524,2.431)}
\gppoint{gp mark 7}{(2.535,2.431)}
\gppoint{gp mark 7}{(2.546,2.431)}
\gppoint{gp mark 7}{(2.557,2.432)}
\gppoint{gp mark 7}{(2.568,2.433)}
\gppoint{gp mark 7}{(2.579,2.436)}
\gppoint{gp mark 7}{(2.590,2.438)}
\gppoint{gp mark 7}{(2.601,2.440)}
\gppoint{gp mark 7}{(2.612,2.440)}
\gppoint{gp mark 7}{(2.623,2.440)}
\gppoint{gp mark 7}{(2.634,2.440)}
\gppoint{gp mark 7}{(2.645,2.440)}
\gppoint{gp mark 7}{(2.656,2.440)}
\gppoint{gp mark 7}{(2.667,2.440)}
\gppoint{gp mark 7}{(2.678,2.440)}
\gppoint{gp mark 7}{(2.689,2.440)}
\gppoint{gp mark 7}{(2.700,2.440)}
\gppoint{gp mark 7}{(2.711,2.440)}
\gppoint{gp mark 7}{(2.722,2.440)}
\gppoint{gp mark 7}{(2.733,2.440)}
\gppoint{gp mark 7}{(2.744,2.440)}
\gppoint{gp mark 7}{(2.755,2.440)}
\gppoint{gp mark 7}{(2.766,2.440)}
\gppoint{gp mark 7}{(2.777,2.440)}
\gppoint{gp mark 7}{(2.788,2.440)}
\gppoint{gp mark 7}{(2.799,2.440)}
\gppoint{gp mark 7}{(2.810,2.440)}
\gppoint{gp mark 7}{(2.821,2.440)}
\gppoint{gp mark 7}{(2.832,2.440)}
\gppoint{gp mark 7}{(2.843,2.440)}
\gppoint{gp mark 7}{(2.854,2.440)}
\gppoint{gp mark 7}{(2.865,2.440)}
\gppoint{gp mark 7}{(2.876,2.440)}
\gppoint{gp mark 7}{(2.887,2.440)}
\gppoint{gp mark 7}{(2.898,2.440)}
\gppoint{gp mark 7}{(2.909,2.440)}
\gppoint{gp mark 7}{(2.920,2.440)}
\gppoint{gp mark 7}{(2.931,2.440)}
\gppoint{gp mark 7}{(2.942,2.440)}
\gppoint{gp mark 7}{(2.953,2.440)}
\gppoint{gp mark 7}{(2.964,2.440)}
\gppoint{gp mark 7}{(2.975,2.440)}
\gppoint{gp mark 7}{(2.986,2.441)}
\gppoint{gp mark 7}{(2.997,2.441)}
\gppoint{gp mark 7}{(3.008,2.441)}
\gppoint{gp mark 7}{(3.019,2.441)}
\gppoint{gp mark 7}{(3.030,2.441)}
\gppoint{gp mark 7}{(3.041,2.441)}
\gppoint{gp mark 7}{(3.052,2.441)}
\gppoint{gp mark 7}{(3.063,2.441)}
\gppoint{gp mark 7}{(3.074,2.441)}
\gppoint{gp mark 7}{(3.085,2.441)}
\gppoint{gp mark 7}{(3.096,2.441)}
\gppoint{gp mark 7}{(3.107,2.441)}
\gppoint{gp mark 7}{(3.118,2.440)}
\gppoint{gp mark 7}{(3.129,2.440)}
\gppoint{gp mark 7}{(3.140,2.440)}
\gppoint{gp mark 7}{(3.151,2.440)}
\gppoint{gp mark 7}{(3.162,2.441)}
\gppoint{gp mark 7}{(3.173,2.441)}
\gppoint{gp mark 7}{(3.184,2.441)}
\gppoint{gp mark 7}{(3.195,2.441)}
\gppoint{gp mark 7}{(3.206,2.441)}
\gppoint{gp mark 7}{(3.217,2.441)}
\gppoint{gp mark 7}{(3.228,2.441)}
\gppoint{gp mark 7}{(3.239,2.441)}
\gppoint{gp mark 7}{(3.250,2.441)}
\gppoint{gp mark 7}{(3.261,2.441)}
\gppoint{gp mark 7}{(3.272,2.441)}
\gppoint{gp mark 7}{(3.283,2.441)}
\gppoint{gp mark 7}{(3.294,2.441)}
\gppoint{gp mark 7}{(3.305,2.441)}
\gppoint{gp mark 7}{(3.316,2.441)}
\gppoint{gp mark 7}{(3.327,2.441)}
\gppoint{gp mark 7}{(3.338,2.441)}
\gppoint{gp mark 7}{(3.349,2.441)}
\gppoint{gp mark 7}{(3.360,2.441)}
\gppoint{gp mark 7}{(3.371,2.441)}
\gppoint{gp mark 7}{(3.382,2.442)}
\gppoint{gp mark 7}{(3.392,2.442)}
\gppoint{gp mark 7}{(3.403,2.442)}
\gppoint{gp mark 7}{(3.414,2.442)}
\gppoint{gp mark 7}{(3.425,2.442)}
\gppoint{gp mark 7}{(3.436,2.442)}
\gppoint{gp mark 7}{(3.447,2.442)}
\gppoint{gp mark 7}{(3.458,2.442)}
\gppoint{gp mark 7}{(3.469,2.441)}
\gppoint{gp mark 7}{(3.480,2.441)}
\gppoint{gp mark 7}{(3.491,2.444)}
\gppoint{gp mark 7}{(3.502,2.457)}
\gppoint{gp mark 7}{(3.513,2.453)}
\gppoint{gp mark 7}{(3.524,2.447)}
\gppoint{gp mark 7}{(3.535,2.441)}
\gppoint{gp mark 7}{(3.546,2.441)}
\gppoint{gp mark 7}{(3.557,2.500)}
\gppoint{gp mark 7}{(3.568,2.445)}
\gppoint{gp mark 7}{(3.579,2.450)}
\gppoint{gp mark 7}{(3.590,2.456)}
\gppoint{gp mark 7}{(3.601,2.458)}
\gppoint{gp mark 7}{(3.612,2.462)}
\gppoint{gp mark 7}{(3.623,2.463)}
\gppoint{gp mark 7}{(3.634,2.456)}
\gppoint{gp mark 7}{(3.645,2.455)}
\gppoint{gp mark 7}{(3.656,2.458)}
\gppoint{gp mark 7}{(3.667,2.458)}
\gppoint{gp mark 7}{(3.678,2.454)}
\gppoint{gp mark 7}{(3.689,2.449)}
\gppoint{gp mark 7}{(3.700,2.449)}
\gppoint{gp mark 7}{(3.711,2.451)}
\gppoint{gp mark 7}{(3.722,2.453)}
\gppoint{gp mark 7}{(3.733,2.455)}
\gppoint{gp mark 7}{(3.744,2.457)}
\gppoint{gp mark 7}{(3.755,2.459)}
\gppoint{gp mark 7}{(3.766,2.459)}
\gppoint{gp mark 7}{(3.777,2.460)}
\gppoint{gp mark 7}{(3.788,2.459)}
\gppoint{gp mark 7}{(3.799,2.459)}
\gppoint{gp mark 7}{(3.810,2.458)}
\gppoint{gp mark 7}{(3.821,2.458)}
\gppoint{gp mark 7}{(3.832,2.455)}
\gppoint{gp mark 7}{(3.843,2.453)}
\gppoint{gp mark 7}{(3.854,2.453)}
\gppoint{gp mark 7}{(3.865,2.453)}
\gppoint{gp mark 7}{(3.876,2.455)}
\gppoint{gp mark 7}{(3.887,2.457)}
\gppoint{gp mark 7}{(3.898,2.460)}
\gppoint{gp mark 7}{(3.909,2.461)}
\gppoint{gp mark 7}{(3.920,2.461)}
\gppoint{gp mark 7}{(3.931,2.461)}
\gppoint{gp mark 7}{(3.942,2.461)}
\gppoint{gp mark 7}{(3.953,2.460)}
\gppoint{gp mark 7}{(3.964,2.458)}
\gppoint{gp mark 7}{(3.975,2.457)}
\gppoint{gp mark 7}{(3.986,2.455)}
\gppoint{gp mark 7}{(3.997,2.454)}
\gppoint{gp mark 7}{(4.008,2.454)}
\gppoint{gp mark 7}{(4.019,2.454)}
\gppoint{gp mark 7}{(4.030,2.456)}
\gppoint{gp mark 7}{(4.041,2.458)}
\gppoint{gp mark 7}{(4.052,2.459)}
\gppoint{gp mark 7}{(4.063,2.460)}
\gppoint{gp mark 7}{(4.074,2.460)}
\gppoint{gp mark 7}{(4.085,2.461)}
\gppoint{gp mark 7}{(4.096,2.461)}
\gppoint{gp mark 7}{(4.107,2.461)}
\gppoint{gp mark 7}{(4.118,2.460)}
\gppoint{gp mark 7}{(4.129,2.459)}
\gppoint{gp mark 7}{(4.140,2.457)}
\gppoint{gp mark 7}{(4.151,2.456)}
\gppoint{gp mark 7}{(4.162,2.457)}
\gppoint{gp mark 7}{(4.173,2.457)}
\gppoint{gp mark 7}{(4.184,2.459)}
\gppoint{gp mark 7}{(4.195,2.461)}
\gppoint{gp mark 7}{(4.206,2.461)}
\gppoint{gp mark 7}{(4.217,2.461)}
\gppoint{gp mark 7}{(4.228,2.460)}
\gppoint{gp mark 7}{(4.239,2.460)}
\gppoint{gp mark 7}{(4.250,2.461)}
\gppoint{gp mark 7}{(4.261,2.462)}
\gppoint{gp mark 7}{(4.272,2.463)}
\gppoint{gp mark 7}{(4.283,2.464)}
\gppoint{gp mark 7}{(4.294,2.468)}
\gppoint{gp mark 7}{(4.304,2.515)}
\gppoint{gp mark 7}{(4.315,2.784)}
\gppoint{gp mark 7}{(4.326,3.362)}
\gppoint{gp mark 7}{(4.337,3.552)}
\gppoint{gp mark 7}{(4.348,3.582)}
\gppoint{gp mark 7}{(4.359,3.589)}
\gppoint{gp mark 7}{(4.370,3.587)}
\gppoint{gp mark 7}{(4.381,3.586)}
\gppoint{gp mark 7}{(4.392,3.587)}
\gppoint{gp mark 7}{(4.403,3.588)}
\gppoint{gp mark 7}{(4.414,3.588)}
\gppoint{gp mark 7}{(4.425,3.587)}
\gppoint{gp mark 7}{(4.436,3.586)}
\gppoint{gp mark 7}{(4.447,3.585)}
\gppoint{gp mark 7}{(4.458,3.584)}
\gppoint{gp mark 7}{(4.469,3.582)}
\gppoint{gp mark 7}{(4.480,3.580)}
\gppoint{gp mark 7}{(4.491,3.581)}
\gppoint{gp mark 7}{(4.502,3.582)}
\gppoint{gp mark 7}{(4.513,3.584)}
\gppoint{gp mark 7}{(4.524,3.585)}
\gppoint{gp mark 7}{(4.535,3.587)}
\gppoint{gp mark 7}{(4.546,3.588)}
\gppoint{gp mark 7}{(4.557,3.588)}
\gppoint{gp mark 7}{(4.568,3.588)}
\gppoint{gp mark 7}{(4.579,3.588)}
\gppoint{gp mark 7}{(4.590,3.587)}
\gppoint{gp mark 7}{(4.601,3.585)}
\gppoint{gp mark 7}{(4.612,3.583)}
\gppoint{gp mark 7}{(4.623,3.583)}
\gppoint{gp mark 7}{(4.634,3.582)}
\gppoint{gp mark 7}{(4.645,3.582)}
\gppoint{gp mark 7}{(4.656,3.584)}
\gppoint{gp mark 7}{(4.667,3.586)}
\gppoint{gp mark 7}{(4.678,3.586)}
\gppoint{gp mark 7}{(4.689,3.587)}
\gppoint{gp mark 7}{(4.700,3.587)}
\gppoint{gp mark 7}{(4.711,3.587)}
\gppoint{gp mark 7}{(4.722,3.586)}
\gppoint{gp mark 7}{(4.733,3.584)}
\gppoint{gp mark 7}{(4.744,3.582)}
\gppoint{gp mark 7}{(4.755,3.582)}
\gppoint{gp mark 7}{(4.766,3.582)}
\gppoint{gp mark 7}{(4.777,3.582)}
\gppoint{gp mark 7}{(4.788,3.583)}
\gppoint{gp mark 7}{(4.799,3.585)}
\gppoint{gp mark 7}{(4.810,3.587)}
\gppoint{gp mark 7}{(4.821,3.588)}
\gppoint{gp mark 7}{(4.832,3.588)}
\gppoint{gp mark 7}{(4.843,3.588)}
\gppoint{gp mark 7}{(4.854,3.588)}
\gppoint{gp mark 7}{(4.865,3.588)}
\gppoint{gp mark 7}{(4.876,3.587)}
\gppoint{gp mark 7}{(4.887,3.585)}
\gppoint{gp mark 7}{(4.898,3.584)}
\gppoint{gp mark 7}{(4.909,3.582)}
\gppoint{gp mark 7}{(4.920,3.582)}
\gppoint{gp mark 7}{(4.931,3.582)}
\gppoint{gp mark 7}{(4.942,3.582)}
\gppoint{gp mark 7}{(4.953,3.584)}
\gppoint{gp mark 7}{(4.964,3.586)}
\gppoint{gp mark 7}{(4.975,3.587)}
\gppoint{gp mark 7}{(4.986,3.587)}
\gppoint{gp mark 7}{(4.997,3.587)}
\gppoint{gp mark 7}{(5.008,3.587)}
\gppoint{gp mark 7}{(5.019,3.587)}
\gppoint{gp mark 7}{(5.030,3.587)}
\gppoint{gp mark 7}{(5.041,3.586)}
\gppoint{gp mark 7}{(5.052,3.584)}
\gppoint{gp mark 7}{(5.063,3.582)}
\gppoint{gp mark 7}{(5.074,3.581)}
\gppoint{gp mark 7}{(5.085,3.581)}
\gppoint{gp mark 7}{(5.096,3.581)}
\gppoint{gp mark 7}{(5.107,3.582)}
\gppoint{gp mark 7}{(5.118,3.584)}
\gppoint{gp mark 7}{(5.129,3.585)}
\gppoint{gp mark 7}{(5.140,3.586)}
\gppoint{gp mark 7}{(5.151,3.587)}
\gppoint{gp mark 7}{(5.162,3.587)}
\gppoint{gp mark 7}{(5.173,3.587)}
\gppoint{gp mark 7}{(5.184,3.585)}
\gppoint{gp mark 7}{(5.195,3.583)}
\gppoint{gp mark 7}{(5.206,3.582)}
\gppoint{gp mark 7}{(5.216,3.582)}
\gppoint{gp mark 7}{(5.227,3.583)}
\gppoint{gp mark 7}{(5.238,3.583)}
\gppoint{gp mark 7}{(5.249,3.584)}
\gppoint{gp mark 7}{(5.260,3.586)}
\gppoint{gp mark 7}{(5.271,3.587)}
\gppoint{gp mark 7}{(5.282,3.588)}
\gppoint{gp mark 7}{(5.293,3.588)}
\gppoint{gp mark 7}{(5.304,3.588)}
\gppoint{gp mark 7}{(5.315,3.588)}
\gppoint{gp mark 7}{(5.326,3.587)}
\gppoint{gp mark 7}{(5.337,3.585)}
\gppoint{gp mark 7}{(5.348,3.583)}
\gppoint{gp mark 7}{(5.359,3.582)}
\gppoint{gp mark 7}{(5.370,3.581)}
\gppoint{gp mark 7}{(5.381,3.581)}
\gppoint{gp mark 7}{(5.392,3.581)}
\gppoint{gp mark 7}{(5.403,3.582)}
\gppoint{gp mark 7}{(5.414,3.584)}
\gppoint{gp mark 7}{(5.425,3.585)}
\gppoint{gp mark 7}{(5.436,3.586)}
\gppoint{gp mark 7}{(5.447,3.587)}
\gppoint{gp mark 7}{(5.458,3.587)}
\gppoint{gp mark 7}{(5.469,3.587)}
\gppoint{gp mark 7}{(5.480,3.585)}
\gppoint{gp mark 7}{(5.491,3.584)}
\gppoint{gp mark 7}{(5.502,3.583)}
\gppoint{gp mark 7}{(5.513,3.583)}
\gppoint{gp mark 7}{(5.524,3.583)}
\gppoint{gp mark 7}{(5.535,3.583)}
\gppoint{gp mark 7}{(5.546,3.584)}
\gppoint{gp mark 7}{(5.557,3.586)}
\gppoint{gp mark 7}{(5.568,3.587)}
\gppoint{gp mark 7}{(5.579,3.587)}
\gppoint{gp mark 7}{(5.590,3.587)}
\gppoint{gp mark 7}{(5.601,3.586)}
\gppoint{gp mark 7}{(5.612,3.586)}
\gppoint{gp mark 7}{(5.623,3.586)}
\gppoint{gp mark 7}{(5.634,3.586)}
\gppoint{gp mark 7}{(5.645,3.586)}
\gppoint{gp mark 7}{(5.656,3.585)}
\gppoint{gp mark 7}{(5.667,3.583)}
\gppoint{gp mark 7}{(5.678,3.583)}
\gppoint{gp mark 7}{(5.689,3.582)}
\gppoint{gp mark 7}{(5.700,3.582)}
\gppoint{gp mark 7}{(5.711,3.583)}
\gppoint{gp mark 7}{(5.722,3.584)}
\gppoint{gp mark 7}{(5.733,3.585)}
\gppoint{gp mark 7}{(5.744,3.586)}
\gppoint{gp mark 7}{(5.755,3.587)}
\gppoint{gp mark 7}{(5.766,3.587)}
\gppoint{gp mark 7}{(5.777,3.587)}
\gppoint{gp mark 7}{(5.788,3.586)}
\gppoint{gp mark 7}{(5.799,3.586)}
\gppoint{gp mark 7}{(5.810,3.585)}
\gppoint{gp mark 7}{(5.821,3.583)}
\gppoint{gp mark 7}{(5.832,3.581)}
\gppoint{gp mark 7}{(5.843,3.581)}
\gppoint{gp mark 7}{(5.854,3.581)}
\gppoint{gp mark 7}{(5.865,3.581)}
\gppoint{gp mark 7}{(5.876,3.582)}
\gppoint{gp mark 7}{(5.887,3.583)}
\gppoint{gp mark 7}{(5.898,3.585)}
\gppoint{gp mark 7}{(5.909,3.586)}
\gppoint{gp mark 7}{(5.920,3.587)}
\gppoint{gp mark 7}{(5.931,3.587)}
\gppoint{gp mark 7}{(5.942,3.587)}
\gppoint{gp mark 7}{(5.953,3.587)}
\gppoint{gp mark 7}{(5.964,3.587)}
\gppoint{gp mark 7}{(5.975,3.585)}
\gppoint{gp mark 7}{(5.986,3.583)}
\gppoint{gp mark 7}{(5.997,3.582)}
\gppoint{gp mark 7}{(6.008,3.582)}
\gppoint{gp mark 7}{(6.019,3.582)}
\gppoint{gp mark 7}{(6.030,3.582)}
\gppoint{gp mark 7}{(6.041,3.583)}
\gppoint{gp mark 7}{(6.052,3.584)}
\gppoint{gp mark 7}{(6.063,3.585)}
\gppoint{gp mark 7}{(6.074,3.586)}
\gppoint{gp mark 7}{(6.085,3.586)}
\gppoint{gp mark 7}{(6.096,3.586)}
\gppoint{gp mark 7}{(6.107,3.586)}
\gppoint{gp mark 7}{(6.118,3.585)}
\gppoint{gp mark 7}{(6.128,3.584)}
\gppoint{gp mark 7}{(6.139,3.582)}
\gppoint{gp mark 7}{(6.150,3.581)}
\gppoint{gp mark 7}{(6.161,3.581)}
\gppoint{gp mark 7}{(6.172,3.581)}
\gppoint{gp mark 7}{(6.183,3.581)}
\gppoint{gp mark 7}{(6.194,3.583)}
\gppoint{gp mark 7}{(6.205,3.585)}
\gppoint{gp mark 7}{(6.216,3.586)}
\gppoint{gp mark 7}{(6.227,3.587)}
\gppoint{gp mark 7}{(6.238,3.587)}
\gppoint{gp mark 7}{(6.249,3.587)}
\gppoint{gp mark 7}{(6.260,3.587)}
\gppoint{gp mark 7}{(6.271,3.587)}
\gppoint{gp mark 7}{(6.282,3.587)}
\gppoint{gp mark 7}{(6.293,3.586)}
\gppoint{gp mark 7}{(6.304,3.584)}
\gppoint{gp mark 7}{(6.315,3.583)}
\gppoint{gp mark 7}{(6.326,3.581)}
\gppoint{gp mark 7}{(6.337,3.581)}
\gppoint{gp mark 7}{(6.348,3.581)}
\gppoint{gp mark 7}{(6.359,3.581)}
\gppoint{gp mark 7}{(6.370,3.582)}
\gppoint{gp mark 7}{(6.381,3.583)}
\gppoint{gp mark 7}{(6.392,3.585)}
\gppoint{gp mark 7}{(6.403,3.586)}
\gppoint{gp mark 7}{(6.414,3.587)}
\gppoint{gp mark 7}{(6.425,3.587)}
\gppoint{gp mark 7}{(6.436,3.587)}
\gppoint{gp mark 7}{(6.447,3.587)}
\gppoint{gp mark 7}{(6.458,3.586)}
\gppoint{gp mark 7}{(6.469,3.586)}
\gppoint{gp mark 7}{(6.480,3.584)}
\gppoint{gp mark 7}{(6.491,3.583)}
\gppoint{gp mark 7}{(6.502,3.581)}
\gppoint{gp mark 7}{(6.513,3.581)}
\gppoint{gp mark 7}{(6.524,3.581)}
\gppoint{gp mark 7}{(6.535,3.581)}
\gppoint{gp mark 7}{(6.546,3.582)}
\gppoint{gp mark 7}{(6.557,3.583)}
\gppoint{gp mark 7}{(6.568,3.585)}
\gppoint{gp mark 7}{(6.579,3.586)}
\gppoint{gp mark 7}{(6.590,3.587)}
\gppoint{gp mark 7}{(6.601,3.587)}
\gppoint{gp mark 7}{(6.612,3.587)}
\gppoint{gp mark 7}{(6.623,3.587)}
\gppoint{gp mark 7}{(6.634,3.586)}
\gppoint{gp mark 7}{(6.645,3.584)}
\gppoint{gp mark 7}{(6.656,3.583)}
\gppoint{gp mark 7}{(6.667,3.581)}
\gppoint{gp mark 7}{(6.678,3.581)}
\gppoint{gp mark 7}{(6.689,3.581)}
\gppoint{gp mark 7}{(6.700,3.581)}
\gppoint{gp mark 7}{(6.711,3.582)}
\gppoint{gp mark 7}{(6.722,3.583)}
\gppoint{gp mark 7}{(6.733,3.585)}
\gppoint{gp mark 7}{(6.744,3.586)}
\gppoint{gp mark 7}{(6.755,3.586)}
\gppoint{gp mark 7}{(6.766,3.586)}
\gppoint{gp mark 7}{(6.777,3.586)}
\gppoint{gp mark 7}{(6.788,3.586)}
\gppoint{gp mark 7}{(6.799,3.586)}
\gppoint{gp mark 7}{(6.810,3.586)}
\gppoint{gp mark 7}{(6.821,3.586)}
\gppoint{gp mark 7}{(6.832,3.586)}
\gppoint{gp mark 7}{(6.843,3.585)}
\gppoint{gp mark 7}{(6.854,3.583)}
\gppoint{gp mark 7}{(6.865,3.582)}
\gppoint{gp mark 7}{(6.876,3.582)}
\gppoint{gp mark 7}{(6.887,3.582)}
\gppoint{gp mark 7}{(6.898,3.582)}
\gppoint{gp mark 7}{(6.909,3.583)}
\gppoint{gp mark 7}{(6.920,3.584)}
\gppoint{gp mark 7}{(6.931,3.585)}
\gppoint{gp mark 7}{(6.942,3.586)}
\gppoint{gp mark 7}{(6.953,3.586)}
\gppoint{gp mark 7}{(6.964,3.586)}
\gppoint{gp mark 7}{(6.975,3.586)}
\gppoint{gp mark 7}{(6.986,3.585)}
\gppoint{gp mark 7}{(6.997,3.585)}
\gppoint{gp mark 7}{(7.008,3.585)}
\gppoint{gp mark 7}{(7.019,3.585)}
\gppoint{gp mark 7}{(7.030,3.586)}
\gppoint{gp mark 7}{(7.040,3.586)}
\gppoint{gp mark 7}{(7.051,3.587)}
\gppoint{gp mark 7}{(7.062,3.588)}
\gppoint{gp mark 7}{(7.073,3.588)}
\gppoint{gp mark 7}{(7.084,3.588)}
\gppoint{gp mark 7}{(7.095,3.588)}
\gppoint{gp mark 7}{(7.106,3.588)}
\gppoint{gp mark 7}{(7.117,3.588)}
\gppoint{gp mark 7}{(7.128,3.588)}
\gppoint{gp mark 7}{(7.139,3.588)}
\gppoint{gp mark 7}{(7.150,3.588)}
\gppoint{gp mark 7}{(7.161,3.588)}
\gppoint{gp mark 7}{(7.172,3.588)}
\gppoint{gp mark 7}{(7.183,3.588)}
\gppoint{gp mark 7}{(7.194,3.587)}
\gppoint{gp mark 7}{(7.205,3.587)}
\gppoint{gp mark 7}{(7.216,3.586)}
\gppoint{gp mark 7}{(7.227,3.586)}
\gppoint{gp mark 7}{(7.238,3.586)}
\gppoint{gp mark 7}{(7.249,3.586)}
\gppoint{gp mark 7}{(7.260,3.584)}
\gppoint{gp mark 7}{(7.271,3.583)}
\gppoint{gp mark 7}{(7.282,3.581)}
\gppoint{gp mark 7}{(7.293,3.579)}
\gppoint{gp mark 7}{(7.304,3.577)}
\gppoint{gp mark 7}{(7.315,3.573)}
\gppoint{gp mark 7}{(7.326,3.569)}
\gppoint{gp mark 7}{(7.337,3.563)}
\gppoint{gp mark 7}{(7.348,3.559)}
\gppoint{gp mark 7}{(7.359,3.557)}
\gppoint{gp mark 7}{(7.370,3.557)}
\gppoint{gp mark 7}{(7.381,3.556)}
\gppoint{gp mark 7}{(7.392,3.556)}
\gppoint{gp mark 7}{(7.403,3.556)}
\gppoint{gp mark 7}{(7.414,3.557)}
\gppoint{gp mark 7}{(7.425,3.557)}
\gppoint{gp mark 7}{(7.436,3.556)}
\gppoint{gp mark 7}{(7.447,3.555)}
\gppoint{gp mark 7}{(7.458,3.543)}
\gppoint{gp mark 7}{(7.469,3.459)}
\gppoint{gp mark 7}{(7.480,2.983)}
\gppoint{gp mark 7}{(7.491,1.895)}
\gpcolor{rgb color={0.000,0.000,0.000}}
\gpsetlinetype{gp lt plot 0}
\gpsetlinewidth{4.00}
\draw[gp path] (2.411,2.442)--(3.526,2.442);
\draw[gp path] (3.526,2.442)--(4.329,2.442);
\draw[gp path] (4.329,3.580)--(7.511,3.580);
\draw[gp path] (1.202,4.128)--(1.208,4.119)--(1.214,4.109)--(1.220,4.100)--(1.226,4.091)%
  --(1.232,4.082)--(1.238,4.073)--(1.244,4.064)--(1.250,4.055)--(1.256,4.046)--(1.262,4.037)%
  --(1.268,4.027)--(1.275,4.018)--(1.281,4.009)--(1.287,4.000)--(1.293,3.991)--(1.299,3.982)%
  --(1.305,3.973)--(1.311,3.964)--(1.317,3.955)--(1.323,3.946)--(1.329,3.937)--(1.335,3.928)%
  --(1.341,3.919)--(1.347,3.910)--(1.353,3.901)--(1.359,3.892)--(1.365,3.883)--(1.371,3.874)%
  --(1.377,3.865)--(1.383,3.856)--(1.389,3.847)--(1.395,3.838)--(1.402,3.829)--(1.408,3.820)%
  --(1.414,3.811)--(1.420,3.802)--(1.426,3.793)--(1.432,3.784)--(1.438,3.775)--(1.444,3.767)%
  --(1.450,3.758)--(1.456,3.749)--(1.462,3.740)--(1.468,3.731)--(1.474,3.722)--(1.480,3.713)%
  --(1.486,3.704)--(1.492,3.695)--(1.498,3.687)--(1.504,3.678)--(1.510,3.669)--(1.516,3.660)%
  --(1.522,3.651)--(1.528,3.642)--(1.535,3.633)--(1.541,3.625)--(1.547,3.616)--(1.553,3.607)%
  --(1.559,3.598)--(1.565,3.589)--(1.571,3.581)--(1.577,3.572)--(1.583,3.563)--(1.589,3.554)%
  --(1.595,3.546)--(1.601,3.537)--(1.607,3.528)--(1.613,3.519)--(1.619,3.511)--(1.625,3.502)%
  --(1.631,3.493)--(1.637,3.484)--(1.643,3.476)--(1.649,3.467)--(1.655,3.458)--(1.662,3.450)%
  --(1.668,3.441)--(1.674,3.432)--(1.680,3.423)--(1.686,3.415)--(1.692,3.406)--(1.698,3.398)%
  --(1.704,3.389)--(1.710,3.380)--(1.716,3.372)--(1.722,3.363)--(1.728,3.354)--(1.734,3.346)%
  --(1.740,3.337)--(1.746,3.328)--(1.752,3.320)--(1.758,3.311)--(1.764,3.303)--(1.770,3.294)%
  --(1.776,3.286)--(1.782,3.277)--(1.788,3.268)--(1.795,3.260)--(1.801,3.251)--(1.807,3.243)%
  --(1.813,3.234)--(1.819,3.226)--(1.825,3.217)--(1.831,3.209)--(1.837,3.200)--(1.843,3.192)%
  --(1.849,3.183)--(1.855,3.175)--(1.861,3.166)--(1.867,3.158)--(1.873,3.149)--(1.879,3.141)%
  --(1.885,3.132)--(1.891,3.124)--(1.897,3.116)--(1.903,3.107)--(1.909,3.099)--(1.915,3.090)%
  --(1.922,3.082)--(1.928,3.074)--(1.934,3.065)--(1.940,3.057)--(1.946,3.048)--(1.952,3.040)%
  --(1.958,3.032)--(1.964,3.023)--(1.970,3.015)--(1.976,3.007)--(1.982,2.998)--(1.988,2.990)%
  --(1.994,2.982)--(2.000,2.973)--(2.006,2.965)--(2.012,2.957)--(2.018,2.948)--(2.024,2.940)%
  --(2.030,2.932)--(2.036,2.924)--(2.042,2.915)--(2.048,2.907)--(2.055,2.899)--(2.061,2.891)%
  --(2.067,2.882)--(2.073,2.874)--(2.079,2.866)--(2.085,2.858)--(2.091,2.850)--(2.097,2.841)%
  --(2.103,2.833)--(2.109,2.825)--(2.115,2.817)--(2.121,2.809)--(2.127,2.800)--(2.133,2.792)%
  --(2.139,2.784)--(2.145,2.776)--(2.151,2.768)--(2.157,2.760)--(2.163,2.752)--(2.169,2.744)%
  --(2.175,2.736)--(2.182,2.727)--(2.188,2.719)--(2.194,2.711)--(2.200,2.703)--(2.206,2.695)%
  --(2.212,2.687)--(2.218,2.679)--(2.224,2.671)--(2.230,2.663)--(2.236,2.655)--(2.242,2.647)%
  --(2.248,2.639)--(2.254,2.631)--(2.260,2.623)--(2.266,2.615)--(2.272,2.607)--(2.278,2.599)%
  --(2.284,2.591)--(2.290,2.583)--(2.296,2.575)--(2.302,2.567)--(2.308,2.560)--(2.315,2.552)%
  --(2.321,2.544)--(2.327,2.536)--(2.333,2.528)--(2.339,2.520)--(2.345,2.512)--(2.351,2.504)%
  --(2.357,2.497)--(2.363,2.489)--(2.369,2.481)--(2.375,2.473)--(2.381,2.465)--(2.387,2.457)%
  --(2.393,2.450)--(2.399,2.442)--(2.405,2.434)--(2.411,2.442);
\draw[gp path] (4.329,2.442)--(4.329,3.580);
\draw[gp path] (7.511,3.580)--(7.511,0.985);
\node[gp node left,font={\fontsize{10pt}{12pt}\selectfont}] at (1.421,5.244) {\LARGE $\rho$};
\node[gp node left,font={\fontsize{10pt}{12pt}\selectfont}] at (6.147,5.244) {\large $\alpha = 3.0$};
%% coordinates of the plot area
\gpdefrectangularnode{gp plot 1}{\pgfpoint{1.196cm}{0.985cm}}{\pgfpoint{7.947cm}{5.631cm}}
\end{tikzpicture}
%% gnuplot variables
} & 
\resizebox{0.5\linewidth}{!}{\tikzsetnextfilename{coplanar_b_crsol_6}\begin{tikzpicture}[gnuplot]
%% generated with GNUPLOT 4.6p4 (Lua 5.1; terminal rev. 99, script rev. 100)
%% Mon 02 Jun 2014 11:35:14 AM EDT
\path (0.000,0.000) rectangle (8.500,6.000);
\gpfill{rgb color={1.000,1.000,1.000}} (1.196,0.985)--(7.946,0.985)--(7.946,5.630)--(1.196,5.630)--cycle;
\gpcolor{color=gp lt color border}
\gpsetlinetype{gp lt border}
\gpsetlinewidth{1.00}
\draw[gp path] (1.196,0.985)--(1.196,5.630)--(7.946,5.630)--(7.946,0.985)--cycle;
\gpcolor{color=gp lt color axes}
\gpsetlinetype{gp lt axes}
\gpsetlinewidth{2.00}
\draw[gp path] (1.196,0.985)--(7.947,0.985);
\gpcolor{color=gp lt color border}
\gpsetlinetype{gp lt border}
\draw[gp path] (1.196,0.985)--(1.268,0.985);
\draw[gp path] (7.947,0.985)--(7.875,0.985);
\gpcolor{rgb color={0.000,0.000,0.000}}
\node[gp node right,font={\fontsize{10pt}{12pt}\selectfont}] at (1.012,0.985) {-0.4};
\gpcolor{color=gp lt color axes}
\gpsetlinetype{gp lt axes}
\draw[gp path] (1.196,1.759)--(7.947,1.759);
\gpcolor{color=gp lt color border}
\gpsetlinetype{gp lt border}
\draw[gp path] (1.196,1.759)--(1.268,1.759);
\draw[gp path] (7.947,1.759)--(7.875,1.759);
\gpcolor{rgb color={0.000,0.000,0.000}}
\node[gp node right,font={\fontsize{10pt}{12pt}\selectfont}] at (1.012,1.759) {-0.2};
\gpcolor{color=gp lt color axes}
\gpsetlinetype{gp lt axes}
\draw[gp path] (1.196,2.534)--(7.947,2.534);
\gpcolor{color=gp lt color border}
\gpsetlinetype{gp lt border}
\draw[gp path] (1.196,2.534)--(1.268,2.534);
\draw[gp path] (7.947,2.534)--(7.875,2.534);
\gpcolor{rgb color={0.000,0.000,0.000}}
\node[gp node right,font={\fontsize{10pt}{12pt}\selectfont}] at (1.012,2.534) {0};
\gpcolor{color=gp lt color axes}
\gpsetlinetype{gp lt axes}
\draw[gp path] (1.196,3.308)--(7.947,3.308);
\gpcolor{color=gp lt color border}
\gpsetlinetype{gp lt border}
\draw[gp path] (1.196,3.308)--(1.268,3.308);
\draw[gp path] (7.947,3.308)--(7.875,3.308);
\gpcolor{rgb color={0.000,0.000,0.000}}
\node[gp node right,font={\fontsize{10pt}{12pt}\selectfont}] at (1.012,3.308) {0.2};
\gpcolor{color=gp lt color axes}
\gpsetlinetype{gp lt axes}
\draw[gp path] (1.196,4.082)--(7.947,4.082);
\gpcolor{color=gp lt color border}
\gpsetlinetype{gp lt border}
\draw[gp path] (1.196,4.082)--(1.268,4.082);
\draw[gp path] (7.947,4.082)--(7.875,4.082);
\gpcolor{rgb color={0.000,0.000,0.000}}
\node[gp node right,font={\fontsize{10pt}{12pt}\selectfont}] at (1.012,4.082) {0.4};
\gpcolor{color=gp lt color axes}
\gpsetlinetype{gp lt axes}
\draw[gp path] (1.196,4.857)--(7.947,4.857);
\gpcolor{color=gp lt color border}
\gpsetlinetype{gp lt border}
\draw[gp path] (1.196,4.857)--(1.268,4.857);
\draw[gp path] (7.947,4.857)--(7.875,4.857);
\gpcolor{rgb color={0.000,0.000,0.000}}
\node[gp node right,font={\fontsize{10pt}{12pt}\selectfont}] at (1.012,4.857) {0.6};
\gpcolor{color=gp lt color axes}
\gpsetlinetype{gp lt axes}
\draw[gp path] (1.196,5.631)--(7.947,5.631);
\gpcolor{color=gp lt color border}
\gpsetlinetype{gp lt border}
\draw[gp path] (1.196,5.631)--(1.268,5.631);
\draw[gp path] (7.947,5.631)--(7.875,5.631);
\gpcolor{rgb color={0.000,0.000,0.000}}
\node[gp node right,font={\fontsize{10pt}{12pt}\selectfont}] at (1.012,5.631) {0.8};
\gpcolor{color=gp lt color axes}
\gpsetlinetype{gp lt axes}
\draw[gp path] (1.196,0.985)--(1.196,5.631);
\gpcolor{color=gp lt color border}
\gpsetlinetype{gp lt border}
\draw[gp path] (1.196,0.985)--(1.196,1.057);
\draw[gp path] (1.196,5.631)--(1.196,5.559);
\gpcolor{rgb color={0.000,0.000,0.000}}
\node[gp node center,font={\fontsize{10pt}{12pt}\selectfont}] at (1.196,0.677) {0.2};
\gpcolor{color=gp lt color axes}
\gpsetlinetype{gp lt axes}
\draw[gp path] (2.321,0.985)--(2.321,5.631);
\gpcolor{color=gp lt color border}
\gpsetlinetype{gp lt border}
\draw[gp path] (2.321,0.985)--(2.321,1.057);
\draw[gp path] (2.321,5.631)--(2.321,5.559);
\gpcolor{rgb color={0.000,0.000,0.000}}
\node[gp node center,font={\fontsize{10pt}{12pt}\selectfont}] at (2.321,0.677) {0.25};
\gpcolor{color=gp lt color axes}
\gpsetlinetype{gp lt axes}
\draw[gp path] (3.446,0.985)--(3.446,5.631);
\gpcolor{color=gp lt color border}
\gpsetlinetype{gp lt border}
\draw[gp path] (3.446,0.985)--(3.446,1.057);
\draw[gp path] (3.446,5.631)--(3.446,5.559);
\gpcolor{rgb color={0.000,0.000,0.000}}
\node[gp node center,font={\fontsize{10pt}{12pt}\selectfont}] at (3.446,0.677) {0.3};
\gpcolor{color=gp lt color axes}
\gpsetlinetype{gp lt axes}
\draw[gp path] (4.572,0.985)--(4.572,5.631);
\gpcolor{color=gp lt color border}
\gpsetlinetype{gp lt border}
\draw[gp path] (4.572,0.985)--(4.572,1.057);
\draw[gp path] (4.572,5.631)--(4.572,5.559);
\gpcolor{rgb color={0.000,0.000,0.000}}
\node[gp node center,font={\fontsize{10pt}{12pt}\selectfont}] at (4.572,0.677) {0.35};
\gpcolor{color=gp lt color axes}
\gpsetlinetype{gp lt axes}
\draw[gp path] (5.697,0.985)--(5.697,5.631);
\gpcolor{color=gp lt color border}
\gpsetlinetype{gp lt border}
\draw[gp path] (5.697,0.985)--(5.697,1.057);
\draw[gp path] (5.697,5.631)--(5.697,5.559);
\gpcolor{rgb color={0.000,0.000,0.000}}
\node[gp node center,font={\fontsize{10pt}{12pt}\selectfont}] at (5.697,0.677) {0.4};
\gpcolor{color=gp lt color axes}
\gpsetlinetype{gp lt axes}
\draw[gp path] (6.822,0.985)--(6.822,5.631);
\gpcolor{color=gp lt color border}
\gpsetlinetype{gp lt border}
\draw[gp path] (6.822,0.985)--(6.822,1.057);
\draw[gp path] (6.822,5.631)--(6.822,5.559);
\gpcolor{rgb color={0.000,0.000,0.000}}
\node[gp node center,font={\fontsize{10pt}{12pt}\selectfont}] at (6.822,0.677) {0.45};
\gpcolor{color=gp lt color axes}
\gpsetlinetype{gp lt axes}
\draw[gp path] (7.947,0.985)--(7.947,5.631);
\gpcolor{color=gp lt color border}
\gpsetlinetype{gp lt border}
\draw[gp path] (7.947,0.985)--(7.947,1.057);
\draw[gp path] (7.947,5.631)--(7.947,5.559);
\gpcolor{rgb color={0.000,0.000,0.000}}
\node[gp node center,font={\fontsize{10pt}{12pt}\selectfont}] at (7.947,0.677) {0.5};
\gpcolor{color=gp lt color border}
\draw[gp path] (1.196,5.631)--(1.196,0.985)--(7.947,0.985)--(7.947,5.631)--cycle;
\gpcolor{rgb color={0.000,0.000,0.000}}
\node[gp node center,font={\fontsize{10pt}{12pt}\selectfont}] at (4.571,0.215) {\large $x$};
\gpcolor{rgb color={0.502,0.502,0.502}}
\gpsetlinewidth{0.50}
\gpsetpointsize{2.67}
\gppoint{gp mark 7}{(1.206,4.629)}
\gppoint{gp mark 7}{(1.217,4.623)}
\gppoint{gp mark 7}{(1.228,4.618)}
\gppoint{gp mark 7}{(1.239,4.613)}
\gppoint{gp mark 7}{(1.250,4.607)}
\gppoint{gp mark 7}{(1.261,4.602)}
\gppoint{gp mark 7}{(1.272,4.597)}
\gppoint{gp mark 7}{(1.283,4.592)}
\gppoint{gp mark 7}{(1.294,4.586)}
\gppoint{gp mark 7}{(1.305,4.581)}
\gppoint{gp mark 7}{(1.316,4.576)}
\gppoint{gp mark 7}{(1.327,4.570)}
\gppoint{gp mark 7}{(1.338,4.565)}
\gppoint{gp mark 7}{(1.349,4.560)}
\gppoint{gp mark 7}{(1.360,4.554)}
\gppoint{gp mark 7}{(1.371,4.549)}
\gppoint{gp mark 7}{(1.382,4.544)}
\gppoint{gp mark 7}{(1.393,4.539)}
\gppoint{gp mark 7}{(1.404,4.533)}
\gppoint{gp mark 7}{(1.415,4.528)}
\gppoint{gp mark 7}{(1.426,4.523)}
\gppoint{gp mark 7}{(1.437,4.517)}
\gppoint{gp mark 7}{(1.448,4.512)}
\gppoint{gp mark 7}{(1.459,4.507)}
\gppoint{gp mark 7}{(1.470,4.501)}
\gppoint{gp mark 7}{(1.481,4.496)}
\gppoint{gp mark 7}{(1.492,4.491)}
\gppoint{gp mark 7}{(1.503,4.485)}
\gppoint{gp mark 7}{(1.514,4.480)}
\gppoint{gp mark 7}{(1.525,4.475)}
\gppoint{gp mark 7}{(1.536,4.469)}
\gppoint{gp mark 7}{(1.547,4.464)}
\gppoint{gp mark 7}{(1.558,4.459)}
\gppoint{gp mark 7}{(1.568,4.453)}
\gppoint{gp mark 7}{(1.579,4.448)}
\gppoint{gp mark 7}{(1.590,4.443)}
\gppoint{gp mark 7}{(1.601,4.437)}
\gppoint{gp mark 7}{(1.612,4.432)}
\gppoint{gp mark 7}{(1.623,4.427)}
\gppoint{gp mark 7}{(1.634,4.421)}
\gppoint{gp mark 7}{(1.645,4.416)}
\gppoint{gp mark 7}{(1.656,4.410)}
\gppoint{gp mark 7}{(1.667,4.405)}
\gppoint{gp mark 7}{(1.678,4.400)}
\gppoint{gp mark 7}{(1.689,4.394)}
\gppoint{gp mark 7}{(1.700,4.389)}
\gppoint{gp mark 7}{(1.711,4.384)}
\gppoint{gp mark 7}{(1.722,4.378)}
\gppoint{gp mark 7}{(1.733,4.373)}
\gppoint{gp mark 7}{(1.744,4.368)}
\gppoint{gp mark 7}{(1.755,4.362)}
\gppoint{gp mark 7}{(1.766,4.357)}
\gppoint{gp mark 7}{(1.777,4.351)}
\gppoint{gp mark 7}{(1.788,4.346)}
\gppoint{gp mark 7}{(1.799,4.341)}
\gppoint{gp mark 7}{(1.810,4.335)}
\gppoint{gp mark 7}{(1.821,4.330)}
\gppoint{gp mark 7}{(1.832,4.324)}
\gppoint{gp mark 7}{(1.843,4.319)}
\gppoint{gp mark 7}{(1.854,4.314)}
\gppoint{gp mark 7}{(1.865,4.308)}
\gppoint{gp mark 7}{(1.876,4.303)}
\gppoint{gp mark 7}{(1.887,4.297)}
\gppoint{gp mark 7}{(1.898,4.292)}
\gppoint{gp mark 7}{(1.909,4.286)}
\gppoint{gp mark 7}{(1.920,4.281)}
\gppoint{gp mark 7}{(1.931,4.276)}
\gppoint{gp mark 7}{(1.942,4.270)}
\gppoint{gp mark 7}{(1.953,4.265)}
\gppoint{gp mark 7}{(1.964,4.259)}
\gppoint{gp mark 7}{(1.975,4.254)}
\gppoint{gp mark 7}{(1.986,4.248)}
\gppoint{gp mark 7}{(1.997,4.243)}
\gppoint{gp mark 7}{(2.008,4.237)}
\gppoint{gp mark 7}{(2.019,4.232)}
\gppoint{gp mark 7}{(2.030,4.226)}
\gppoint{gp mark 7}{(2.041,4.221)}
\gppoint{gp mark 7}{(2.052,4.215)}
\gppoint{gp mark 7}{(2.063,4.210)}
\gppoint{gp mark 7}{(2.074,4.204)}
\gppoint{gp mark 7}{(2.085,4.199)}
\gppoint{gp mark 7}{(2.096,4.193)}
\gppoint{gp mark 7}{(2.107,4.188)}
\gppoint{gp mark 7}{(2.118,4.182)}
\gppoint{gp mark 7}{(2.129,4.177)}
\gppoint{gp mark 7}{(2.140,4.171)}
\gppoint{gp mark 7}{(2.151,4.166)}
\gppoint{gp mark 7}{(2.162,4.160)}
\gppoint{gp mark 7}{(2.173,4.155)}
\gppoint{gp mark 7}{(2.184,4.149)}
\gppoint{gp mark 7}{(2.195,4.144)}
\gppoint{gp mark 7}{(2.206,4.138)}
\gppoint{gp mark 7}{(2.217,4.133)}
\gppoint{gp mark 7}{(2.228,4.127)}
\gppoint{gp mark 7}{(2.239,4.122)}
\gppoint{gp mark 7}{(2.250,4.116)}
\gppoint{gp mark 7}{(2.261,4.110)}
\gppoint{gp mark 7}{(2.272,4.105)}
\gppoint{gp mark 7}{(2.283,4.099)}
\gppoint{gp mark 7}{(2.294,4.094)}
\gppoint{gp mark 7}{(2.305,4.088)}
\gppoint{gp mark 7}{(2.316,4.082)}
\gppoint{gp mark 7}{(2.327,4.077)}
\gppoint{gp mark 7}{(2.338,4.071)}
\gppoint{gp mark 7}{(2.349,4.066)}
\gppoint{gp mark 7}{(2.360,4.060)}
\gppoint{gp mark 7}{(2.371,4.054)}
\gppoint{gp mark 7}{(2.382,4.049)}
\gppoint{gp mark 7}{(2.393,4.043)}
\gppoint{gp mark 7}{(2.404,4.037)}
\gppoint{gp mark 7}{(2.415,4.032)}
\gppoint{gp mark 7}{(2.426,4.026)}
\gppoint{gp mark 7}{(2.437,4.020)}
\gppoint{gp mark 7}{(2.448,4.015)}
\gppoint{gp mark 7}{(2.459,4.009)}
\gppoint{gp mark 7}{(2.470,4.003)}
\gppoint{gp mark 7}{(2.480,3.998)}
\gppoint{gp mark 7}{(2.491,3.992)}
\gppoint{gp mark 7}{(2.502,3.986)}
\gppoint{gp mark 7}{(2.513,3.980)}
\gppoint{gp mark 7}{(2.524,3.975)}
\gppoint{gp mark 7}{(2.535,3.969)}
\gppoint{gp mark 7}{(2.546,3.963)}
\gppoint{gp mark 7}{(2.557,3.958)}
\gppoint{gp mark 7}{(2.568,3.952)}
\gppoint{gp mark 7}{(2.579,3.946)}
\gppoint{gp mark 7}{(2.590,3.940)}
\gppoint{gp mark 7}{(2.601,3.934)}
\gppoint{gp mark 7}{(2.612,3.929)}
\gppoint{gp mark 7}{(2.623,3.923)}
\gppoint{gp mark 7}{(2.634,3.917)}
\gppoint{gp mark 7}{(2.645,3.911)}
\gppoint{gp mark 7}{(2.656,3.905)}
\gppoint{gp mark 7}{(2.667,3.900)}
\gppoint{gp mark 7}{(2.678,3.894)}
\gppoint{gp mark 7}{(2.689,3.888)}
\gppoint{gp mark 7}{(2.700,3.882)}
\gppoint{gp mark 7}{(2.711,3.876)}
\gppoint{gp mark 7}{(2.722,3.871)}
\gppoint{gp mark 7}{(2.733,3.865)}
\gppoint{gp mark 7}{(2.744,3.859)}
\gppoint{gp mark 7}{(2.755,3.853)}
\gppoint{gp mark 7}{(2.766,3.847)}
\gppoint{gp mark 7}{(2.777,3.841)}
\gppoint{gp mark 7}{(2.788,3.835)}
\gppoint{gp mark 7}{(2.799,3.830)}
\gppoint{gp mark 7}{(2.810,3.824)}
\gppoint{gp mark 7}{(2.821,3.818)}
\gppoint{gp mark 7}{(2.832,3.812)}
\gppoint{gp mark 7}{(2.843,3.806)}
\gppoint{gp mark 7}{(2.854,3.800)}
\gppoint{gp mark 7}{(2.865,3.794)}
\gppoint{gp mark 7}{(2.876,3.788)}
\gppoint{gp mark 7}{(2.887,3.782)}
\gppoint{gp mark 7}{(2.898,3.776)}
\gppoint{gp mark 7}{(2.909,3.770)}
\gppoint{gp mark 7}{(2.920,3.763)}
\gppoint{gp mark 7}{(2.931,3.755)}
\gppoint{gp mark 7}{(2.942,3.747)}
\gppoint{gp mark 7}{(2.953,3.742)}
\gppoint{gp mark 7}{(2.964,3.741)}
\gppoint{gp mark 7}{(2.975,3.741)}
\gppoint{gp mark 7}{(2.986,3.741)}
\gppoint{gp mark 7}{(2.997,3.748)}
\gppoint{gp mark 7}{(3.008,3.782)}
\gppoint{gp mark 7}{(3.019,3.851)}
\gppoint{gp mark 7}{(3.030,3.918)}
\gppoint{gp mark 7}{(3.041,3.943)}
\gppoint{gp mark 7}{(3.052,3.947)}
\gppoint{gp mark 7}{(3.063,3.948)}
\gppoint{gp mark 7}{(3.074,3.948)}
\gppoint{gp mark 7}{(3.085,3.948)}
\gppoint{gp mark 7}{(3.096,3.949)}
\gppoint{gp mark 7}{(3.107,3.950)}
\gppoint{gp mark 7}{(3.118,3.951)}
\gppoint{gp mark 7}{(3.129,3.953)}
\gppoint{gp mark 7}{(3.140,3.955)}
\gppoint{gp mark 7}{(3.151,3.956)}
\gppoint{gp mark 7}{(3.162,3.958)}
\gppoint{gp mark 7}{(3.173,3.960)}
\gppoint{gp mark 7}{(3.184,3.962)}
\gppoint{gp mark 7}{(3.195,3.963)}
\gppoint{gp mark 7}{(3.206,3.964)}
\gppoint{gp mark 7}{(3.217,3.965)}
\gppoint{gp mark 7}{(3.228,3.966)}
\gppoint{gp mark 7}{(3.239,3.967)}
\gppoint{gp mark 7}{(3.250,3.968)}
\gppoint{gp mark 7}{(3.261,3.969)}
\gppoint{gp mark 7}{(3.272,3.970)}
\gppoint{gp mark 7}{(3.283,3.971)}
\gppoint{gp mark 7}{(3.294,3.972)}
\gppoint{gp mark 7}{(3.305,3.973)}
\gppoint{gp mark 7}{(3.316,3.974)}
\gppoint{gp mark 7}{(3.327,3.975)}
\gppoint{gp mark 7}{(3.338,3.975)}
\gppoint{gp mark 7}{(3.349,3.976)}
\gppoint{gp mark 7}{(3.360,3.977)}
\gppoint{gp mark 7}{(3.371,3.978)}
\gppoint{gp mark 7}{(3.382,3.978)}
\gppoint{gp mark 7}{(3.392,3.979)}
\gppoint{gp mark 7}{(3.403,3.980)}
\gppoint{gp mark 7}{(3.414,3.980)}
\gppoint{gp mark 7}{(3.425,3.981)}
\gppoint{gp mark 7}{(3.436,3.981)}
\gppoint{gp mark 7}{(3.447,3.982)}
\gppoint{gp mark 7}{(3.458,3.983)}
\gppoint{gp mark 7}{(3.469,3.983)}
\gppoint{gp mark 7}{(3.480,3.983)}
\gppoint{gp mark 7}{(3.491,3.975)}
\gppoint{gp mark 7}{(3.502,3.880)}
\gppoint{gp mark 7}{(3.513,3.466)}
\gppoint{gp mark 7}{(3.524,2.508)}
\gppoint{gp mark 7}{(3.535,1.525)}
\gppoint{gp mark 7}{(3.546,1.203)}
\gppoint{gp mark 7}{(3.557,1.152)}
\gppoint{gp mark 7}{(3.568,1.147)}
\gppoint{gp mark 7}{(3.579,1.148)}
\gppoint{gp mark 7}{(3.590,1.149)}
\gppoint{gp mark 7}{(3.601,1.149)}
\gppoint{gp mark 7}{(3.612,1.149)}
\gppoint{gp mark 7}{(3.623,1.151)}
\gppoint{gp mark 7}{(3.634,1.153)}
\gppoint{gp mark 7}{(3.645,1.153)}
\gppoint{gp mark 7}{(3.656,1.153)}
\gppoint{gp mark 7}{(3.667,1.154)}
\gppoint{gp mark 7}{(3.678,1.156)}
\gppoint{gp mark 7}{(3.689,1.157)}
\gppoint{gp mark 7}{(3.700,1.158)}
\gppoint{gp mark 7}{(3.711,1.159)}
\gppoint{gp mark 7}{(3.722,1.162)}
\gppoint{gp mark 7}{(3.733,1.163)}
\gppoint{gp mark 7}{(3.744,1.164)}
\gppoint{gp mark 7}{(3.755,1.166)}
\gppoint{gp mark 7}{(3.766,1.168)}
\gppoint{gp mark 7}{(3.777,1.170)}
\gppoint{gp mark 7}{(3.788,1.171)}
\gppoint{gp mark 7}{(3.799,1.172)}
\gppoint{gp mark 7}{(3.810,1.176)}
\gppoint{gp mark 7}{(3.821,1.194)}
\gppoint{gp mark 7}{(3.832,1.277)}
\gppoint{gp mark 7}{(3.843,1.414)}
\gppoint{gp mark 7}{(3.854,1.454)}
\gppoint{gp mark 7}{(3.865,1.456)}
\gppoint{gp mark 7}{(3.876,1.451)}
\gppoint{gp mark 7}{(3.887,1.437)}
\gppoint{gp mark 7}{(3.898,1.431)}
\gppoint{gp mark 7}{(3.909,1.429)}
\gppoint{gp mark 7}{(3.920,1.425)}
\gppoint{gp mark 7}{(3.931,1.415)}
\gppoint{gp mark 7}{(3.942,1.405)}
\gppoint{gp mark 7}{(3.953,1.397)}
\gppoint{gp mark 7}{(3.964,1.391)}
\gppoint{gp mark 7}{(3.975,1.386)}
\gppoint{gp mark 7}{(3.986,1.380)}
\gppoint{gp mark 7}{(3.997,1.373)}
\gppoint{gp mark 7}{(4.008,1.367)}
\gppoint{gp mark 7}{(4.019,1.360)}
\gppoint{gp mark 7}{(4.030,1.354)}
\gppoint{gp mark 7}{(4.041,1.348)}
\gppoint{gp mark 7}{(4.052,1.342)}
\gppoint{gp mark 7}{(4.063,1.336)}
\gppoint{gp mark 7}{(4.074,1.330)}
\gppoint{gp mark 7}{(4.085,1.325)}
\gppoint{gp mark 7}{(4.096,1.319)}
\gppoint{gp mark 7}{(4.107,1.315)}
\gppoint{gp mark 7}{(4.118,1.313)}
\gppoint{gp mark 7}{(4.129,1.312)}
\gppoint{gp mark 7}{(4.140,1.312)}
\gppoint{gp mark 7}{(4.151,1.312)}
\gppoint{gp mark 7}{(4.162,1.312)}
\gppoint{gp mark 7}{(4.173,1.313)}
\gppoint{gp mark 7}{(4.184,1.314)}
\gppoint{gp mark 7}{(4.195,1.314)}
\gppoint{gp mark 7}{(4.206,1.315)}
\gppoint{gp mark 7}{(4.217,1.315)}
\gppoint{gp mark 7}{(4.228,1.315)}
\gppoint{gp mark 7}{(4.239,1.315)}
\gppoint{gp mark 7}{(4.250,1.315)}
\gppoint{gp mark 7}{(4.261,1.315)}
\gppoint{gp mark 7}{(4.272,1.315)}
\gppoint{gp mark 7}{(4.283,1.315)}
\gppoint{gp mark 7}{(4.294,1.315)}
\gppoint{gp mark 7}{(4.304,1.315)}
\gppoint{gp mark 7}{(4.315,1.315)}
\gppoint{gp mark 7}{(4.326,1.315)}
\gppoint{gp mark 7}{(4.337,1.315)}
\gppoint{gp mark 7}{(4.348,1.315)}
\gppoint{gp mark 7}{(4.359,1.315)}
\gppoint{gp mark 7}{(4.370,1.315)}
\gppoint{gp mark 7}{(4.381,1.315)}
\gppoint{gp mark 7}{(4.392,1.315)}
\gppoint{gp mark 7}{(4.403,1.315)}
\gppoint{gp mark 7}{(4.414,1.315)}
\gppoint{gp mark 7}{(4.425,1.315)}
\gppoint{gp mark 7}{(4.436,1.315)}
\gppoint{gp mark 7}{(4.447,1.315)}
\gppoint{gp mark 7}{(4.458,1.315)}
\gppoint{gp mark 7}{(4.469,1.315)}
\gppoint{gp mark 7}{(4.480,1.315)}
\gppoint{gp mark 7}{(4.491,1.315)}
\gppoint{gp mark 7}{(4.502,1.315)}
\gppoint{gp mark 7}{(4.513,1.315)}
\gppoint{gp mark 7}{(4.524,1.315)}
\gppoint{gp mark 7}{(4.535,1.315)}
\gppoint{gp mark 7}{(4.546,1.315)}
\gppoint{gp mark 7}{(4.557,1.315)}
\gppoint{gp mark 7}{(4.568,1.315)}
\gppoint{gp mark 7}{(4.579,1.315)}
\gppoint{gp mark 7}{(4.590,1.315)}
\gppoint{gp mark 7}{(4.601,1.315)}
\gppoint{gp mark 7}{(4.612,1.315)}
\gppoint{gp mark 7}{(4.623,1.315)}
\gppoint{gp mark 7}{(4.634,1.315)}
\gppoint{gp mark 7}{(4.645,1.315)}
\gppoint{gp mark 7}{(4.656,1.315)}
\gppoint{gp mark 7}{(4.667,1.315)}
\gppoint{gp mark 7}{(4.678,1.315)}
\gppoint{gp mark 7}{(4.689,1.315)}
\gppoint{gp mark 7}{(4.700,1.315)}
\gppoint{gp mark 7}{(4.711,1.315)}
\gppoint{gp mark 7}{(4.722,1.315)}
\gppoint{gp mark 7}{(4.733,1.315)}
\gppoint{gp mark 7}{(4.744,1.315)}
\gppoint{gp mark 7}{(4.755,1.315)}
\gppoint{gp mark 7}{(4.766,1.315)}
\gppoint{gp mark 7}{(4.777,1.315)}
\gppoint{gp mark 7}{(4.788,1.315)}
\gppoint{gp mark 7}{(4.799,1.315)}
\gppoint{gp mark 7}{(4.810,1.315)}
\gppoint{gp mark 7}{(4.821,1.315)}
\gppoint{gp mark 7}{(4.832,1.315)}
\gppoint{gp mark 7}{(4.843,1.315)}
\gppoint{gp mark 7}{(4.854,1.315)}
\gppoint{gp mark 7}{(4.865,1.315)}
\gppoint{gp mark 7}{(4.876,1.315)}
\gppoint{gp mark 7}{(4.887,1.315)}
\gppoint{gp mark 7}{(4.898,1.315)}
\gppoint{gp mark 7}{(4.909,1.315)}
\gppoint{gp mark 7}{(4.920,1.315)}
\gppoint{gp mark 7}{(4.931,1.315)}
\gppoint{gp mark 7}{(4.942,1.315)}
\gppoint{gp mark 7}{(4.953,1.315)}
\gppoint{gp mark 7}{(4.964,1.315)}
\gppoint{gp mark 7}{(4.975,1.315)}
\gppoint{gp mark 7}{(4.986,1.315)}
\gppoint{gp mark 7}{(4.997,1.315)}
\gppoint{gp mark 7}{(5.008,1.315)}
\gppoint{gp mark 7}{(5.019,1.315)}
\gppoint{gp mark 7}{(5.030,1.315)}
\gppoint{gp mark 7}{(5.041,1.315)}
\gppoint{gp mark 7}{(5.052,1.315)}
\gppoint{gp mark 7}{(5.063,1.315)}
\gppoint{gp mark 7}{(5.074,1.315)}
\gppoint{gp mark 7}{(5.085,1.315)}
\gppoint{gp mark 7}{(5.096,1.315)}
\gppoint{gp mark 7}{(5.107,1.315)}
\gppoint{gp mark 7}{(5.118,1.315)}
\gppoint{gp mark 7}{(5.129,1.315)}
\gppoint{gp mark 7}{(5.140,1.315)}
\gppoint{gp mark 7}{(5.151,1.315)}
\gppoint{gp mark 7}{(5.162,1.315)}
\gppoint{gp mark 7}{(5.173,1.315)}
\gppoint{gp mark 7}{(5.184,1.315)}
\gppoint{gp mark 7}{(5.195,1.315)}
\gppoint{gp mark 7}{(5.206,1.315)}
\gppoint{gp mark 7}{(5.216,1.315)}
\gppoint{gp mark 7}{(5.227,1.315)}
\gppoint{gp mark 7}{(5.238,1.315)}
\gppoint{gp mark 7}{(5.249,1.315)}
\gppoint{gp mark 7}{(5.260,1.315)}
\gppoint{gp mark 7}{(5.271,1.315)}
\gppoint{gp mark 7}{(5.282,1.315)}
\gppoint{gp mark 7}{(5.293,1.315)}
\gppoint{gp mark 7}{(5.304,1.315)}
\gppoint{gp mark 7}{(5.315,1.315)}
\gppoint{gp mark 7}{(5.326,1.315)}
\gppoint{gp mark 7}{(5.337,1.315)}
\gppoint{gp mark 7}{(5.348,1.315)}
\gppoint{gp mark 7}{(5.359,1.315)}
\gppoint{gp mark 7}{(5.370,1.315)}
\gppoint{gp mark 7}{(5.381,1.315)}
\gppoint{gp mark 7}{(5.392,1.315)}
\gppoint{gp mark 7}{(5.403,1.315)}
\gppoint{gp mark 7}{(5.414,1.315)}
\gppoint{gp mark 7}{(5.425,1.315)}
\gppoint{gp mark 7}{(5.436,1.315)}
\gppoint{gp mark 7}{(5.447,1.315)}
\gppoint{gp mark 7}{(5.458,1.315)}
\gppoint{gp mark 7}{(5.469,1.315)}
\gppoint{gp mark 7}{(5.480,1.315)}
\gppoint{gp mark 7}{(5.491,1.315)}
\gppoint{gp mark 7}{(5.502,1.315)}
\gppoint{gp mark 7}{(5.513,1.315)}
\gppoint{gp mark 7}{(5.524,1.315)}
\gppoint{gp mark 7}{(5.535,1.315)}
\gppoint{gp mark 7}{(5.546,1.315)}
\gppoint{gp mark 7}{(5.557,1.315)}
\gppoint{gp mark 7}{(5.568,1.315)}
\gppoint{gp mark 7}{(5.579,1.315)}
\gppoint{gp mark 7}{(5.590,1.315)}
\gppoint{gp mark 7}{(5.601,1.315)}
\gppoint{gp mark 7}{(5.612,1.315)}
\gppoint{gp mark 7}{(5.623,1.315)}
\gppoint{gp mark 7}{(5.634,1.315)}
\gppoint{gp mark 7}{(5.645,1.315)}
\gppoint{gp mark 7}{(5.656,1.315)}
\gppoint{gp mark 7}{(5.667,1.315)}
\gppoint{gp mark 7}{(5.678,1.315)}
\gppoint{gp mark 7}{(5.689,1.315)}
\gppoint{gp mark 7}{(5.700,1.315)}
\gppoint{gp mark 7}{(5.711,1.315)}
\gppoint{gp mark 7}{(5.722,1.315)}
\gppoint{gp mark 7}{(5.733,1.315)}
\gppoint{gp mark 7}{(5.744,1.315)}
\gppoint{gp mark 7}{(5.755,1.315)}
\gppoint{gp mark 7}{(5.766,1.315)}
\gppoint{gp mark 7}{(5.777,1.315)}
\gppoint{gp mark 7}{(5.788,1.315)}
\gppoint{gp mark 7}{(5.799,1.315)}
\gppoint{gp mark 7}{(5.810,1.315)}
\gppoint{gp mark 7}{(5.821,1.315)}
\gppoint{gp mark 7}{(5.832,1.315)}
\gppoint{gp mark 7}{(5.843,1.315)}
\gppoint{gp mark 7}{(5.854,1.315)}
\gppoint{gp mark 7}{(5.865,1.315)}
\gppoint{gp mark 7}{(5.876,1.315)}
\gppoint{gp mark 7}{(5.887,1.315)}
\gppoint{gp mark 7}{(5.898,1.315)}
\gppoint{gp mark 7}{(5.909,1.315)}
\gppoint{gp mark 7}{(5.920,1.315)}
\gppoint{gp mark 7}{(5.931,1.315)}
\gppoint{gp mark 7}{(5.942,1.315)}
\gppoint{gp mark 7}{(5.953,1.315)}
\gppoint{gp mark 7}{(5.964,1.315)}
\gppoint{gp mark 7}{(5.975,1.315)}
\gppoint{gp mark 7}{(5.986,1.315)}
\gppoint{gp mark 7}{(5.997,1.315)}
\gppoint{gp mark 7}{(6.008,1.315)}
\gppoint{gp mark 7}{(6.019,1.315)}
\gppoint{gp mark 7}{(6.030,1.315)}
\gppoint{gp mark 7}{(6.041,1.315)}
\gppoint{gp mark 7}{(6.052,1.315)}
\gppoint{gp mark 7}{(6.063,1.315)}
\gppoint{gp mark 7}{(6.074,1.315)}
\gppoint{gp mark 7}{(6.085,1.315)}
\gppoint{gp mark 7}{(6.096,1.315)}
\gppoint{gp mark 7}{(6.107,1.315)}
\gppoint{gp mark 7}{(6.118,1.315)}
\gppoint{gp mark 7}{(6.128,1.315)}
\gppoint{gp mark 7}{(6.139,1.315)}
\gppoint{gp mark 7}{(6.150,1.315)}
\gppoint{gp mark 7}{(6.161,1.315)}
\gppoint{gp mark 7}{(6.172,1.315)}
\gppoint{gp mark 7}{(6.183,1.315)}
\gppoint{gp mark 7}{(6.194,1.315)}
\gppoint{gp mark 7}{(6.205,1.314)}
\gppoint{gp mark 7}{(6.216,1.314)}
\gppoint{gp mark 7}{(6.227,1.314)}
\gppoint{gp mark 7}{(6.238,1.314)}
\gppoint{gp mark 7}{(6.249,1.314)}
\gppoint{gp mark 7}{(6.260,1.315)}
\gppoint{gp mark 7}{(6.271,1.315)}
\gppoint{gp mark 7}{(6.282,1.315)}
\gppoint{gp mark 7}{(6.293,1.315)}
\gppoint{gp mark 7}{(6.304,1.315)}
\gppoint{gp mark 7}{(6.315,1.315)}
\gppoint{gp mark 7}{(6.326,1.315)}
\gppoint{gp mark 7}{(6.337,1.315)}
\gppoint{gp mark 7}{(6.348,1.315)}
\gppoint{gp mark 7}{(6.359,1.315)}
\gppoint{gp mark 7}{(6.370,1.315)}
\gppoint{gp mark 7}{(6.381,1.315)}
\gppoint{gp mark 7}{(6.392,1.314)}
\gppoint{gp mark 7}{(6.403,1.314)}
\gppoint{gp mark 7}{(6.414,1.314)}
\gppoint{gp mark 7}{(6.425,1.314)}
\gppoint{gp mark 7}{(6.436,1.314)}
\gppoint{gp mark 7}{(6.447,1.314)}
\gppoint{gp mark 7}{(6.458,1.314)}
\gppoint{gp mark 7}{(6.469,1.314)}
\gppoint{gp mark 7}{(6.480,1.314)}
\gppoint{gp mark 7}{(6.491,1.314)}
\gppoint{gp mark 7}{(6.502,1.314)}
\gppoint{gp mark 7}{(6.513,1.314)}
\gppoint{gp mark 7}{(6.524,1.315)}
\gppoint{gp mark 7}{(6.535,1.315)}
\gppoint{gp mark 7}{(6.546,1.315)}
\gppoint{gp mark 7}{(6.557,1.314)}
\gppoint{gp mark 7}{(6.568,1.314)}
\gppoint{gp mark 7}{(6.579,1.315)}
\gppoint{gp mark 7}{(6.590,1.315)}
\gppoint{gp mark 7}{(6.601,1.315)}
\gppoint{gp mark 7}{(6.612,1.315)}
\gppoint{gp mark 7}{(6.623,1.314)}
\gppoint{gp mark 7}{(6.634,1.314)}
\gppoint{gp mark 7}{(6.645,1.314)}
\gppoint{gp mark 7}{(6.656,1.314)}
\gppoint{gp mark 7}{(6.667,1.314)}
\gppoint{gp mark 7}{(6.678,1.314)}
\gppoint{gp mark 7}{(6.689,1.315)}
\gppoint{gp mark 7}{(6.700,1.314)}
\gppoint{gp mark 7}{(6.711,1.314)}
\gppoint{gp mark 7}{(6.722,1.314)}
\gppoint{gp mark 7}{(6.733,1.314)}
\gppoint{gp mark 7}{(6.744,1.314)}
\gppoint{gp mark 7}{(6.755,1.314)}
\gppoint{gp mark 7}{(6.766,1.314)}
\gppoint{gp mark 7}{(6.777,1.315)}
\gppoint{gp mark 7}{(6.788,1.314)}
\gppoint{gp mark 7}{(6.799,1.314)}
\gppoint{gp mark 7}{(6.810,1.314)}
\gppoint{gp mark 7}{(6.821,1.315)}
\gppoint{gp mark 7}{(6.832,1.315)}
\gppoint{gp mark 7}{(6.843,1.315)}
\gppoint{gp mark 7}{(6.854,1.315)}
\gppoint{gp mark 7}{(6.865,1.314)}
\gppoint{gp mark 7}{(6.876,1.314)}
\gppoint{gp mark 7}{(6.887,1.314)}
\gppoint{gp mark 7}{(6.898,1.314)}
\gppoint{gp mark 7}{(6.909,1.314)}
\gppoint{gp mark 7}{(6.920,1.315)}
\gppoint{gp mark 7}{(6.931,1.315)}
\gppoint{gp mark 7}{(6.942,1.314)}
\gppoint{gp mark 7}{(6.953,1.314)}
\gppoint{gp mark 7}{(6.964,1.314)}
\gppoint{gp mark 7}{(6.975,1.314)}
\gppoint{gp mark 7}{(6.986,1.314)}
\gppoint{gp mark 7}{(6.997,1.314)}
\gppoint{gp mark 7}{(7.008,1.314)}
\gppoint{gp mark 7}{(7.019,1.314)}
\gppoint{gp mark 7}{(7.030,1.314)}
\gppoint{gp mark 7}{(7.040,1.314)}
\gppoint{gp mark 7}{(7.051,1.314)}
\gppoint{gp mark 7}{(7.062,1.315)}
\gppoint{gp mark 7}{(7.073,1.315)}
\gppoint{gp mark 7}{(7.084,1.315)}
\gppoint{gp mark 7}{(7.095,1.315)}
\gppoint{gp mark 7}{(7.106,1.315)}
\gppoint{gp mark 7}{(7.117,1.315)}
\gppoint{gp mark 7}{(7.128,1.315)}
\gppoint{gp mark 7}{(7.139,1.315)}
\gppoint{gp mark 7}{(7.150,1.315)}
\gppoint{gp mark 7}{(7.161,1.315)}
\gppoint{gp mark 7}{(7.172,1.315)}
\gppoint{gp mark 7}{(7.183,1.315)}
\gppoint{gp mark 7}{(7.194,1.315)}
\gppoint{gp mark 7}{(7.205,1.315)}
\gppoint{gp mark 7}{(7.216,1.315)}
\gppoint{gp mark 7}{(7.227,1.315)}
\gppoint{gp mark 7}{(7.238,1.315)}
\gppoint{gp mark 7}{(7.249,1.315)}
\gppoint{gp mark 7}{(7.260,1.315)}
\gppoint{gp mark 7}{(7.271,1.315)}
\gppoint{gp mark 7}{(7.282,1.315)}
\gppoint{gp mark 7}{(7.293,1.315)}
\gppoint{gp mark 7}{(7.304,1.315)}
\gppoint{gp mark 7}{(7.315,1.315)}
\gppoint{gp mark 7}{(7.326,1.315)}
\gppoint{gp mark 7}{(7.337,1.315)}
\gppoint{gp mark 7}{(7.348,1.315)}
\gppoint{gp mark 7}{(7.359,1.315)}
\gppoint{gp mark 7}{(7.370,1.315)}
\gppoint{gp mark 7}{(7.381,1.315)}
\gppoint{gp mark 7}{(7.392,1.315)}
\gppoint{gp mark 7}{(7.403,1.315)}
\gppoint{gp mark 7}{(7.414,1.315)}
\gppoint{gp mark 7}{(7.425,1.315)}
\gppoint{gp mark 7}{(7.436,1.315)}
\gppoint{gp mark 7}{(7.447,1.315)}
\gppoint{gp mark 7}{(7.458,1.315)}
\gppoint{gp mark 7}{(7.469,1.315)}
\gppoint{gp mark 7}{(7.480,1.315)}
\gppoint{gp mark 7}{(7.491,1.315)}
\gppoint{gp mark 7}{(7.502,1.315)}
\gppoint{gp mark 7}{(7.513,1.315)}
\gppoint{gp mark 7}{(7.524,1.315)}
\gppoint{gp mark 7}{(7.535,1.315)}
\gppoint{gp mark 7}{(7.546,1.315)}
\gppoint{gp mark 7}{(7.557,1.315)}
\gppoint{gp mark 7}{(7.568,1.315)}
\gppoint{gp mark 7}{(7.579,1.315)}
\gppoint{gp mark 7}{(7.590,1.315)}
\gppoint{gp mark 7}{(7.601,1.315)}
\gppoint{gp mark 7}{(7.612,1.315)}
\gppoint{gp mark 7}{(7.623,1.315)}
\gppoint{gp mark 7}{(7.634,1.315)}
\gppoint{gp mark 7}{(7.645,1.315)}
\gppoint{gp mark 7}{(7.656,1.315)}
\gppoint{gp mark 7}{(7.667,1.315)}
\gppoint{gp mark 7}{(7.678,1.315)}
\gppoint{gp mark 7}{(7.689,1.315)}
\gppoint{gp mark 7}{(7.700,1.315)}
\gppoint{gp mark 7}{(7.711,1.315)}
\gppoint{gp mark 7}{(7.722,1.315)}
\gppoint{gp mark 7}{(7.733,1.315)}
\gppoint{gp mark 7}{(7.744,1.315)}
\gppoint{gp mark 7}{(7.755,1.315)}
\gppoint{gp mark 7}{(7.766,1.315)}
\gppoint{gp mark 7}{(7.777,1.315)}
\gppoint{gp mark 7}{(7.788,1.315)}
\gppoint{gp mark 7}{(7.799,1.315)}
\gppoint{gp mark 7}{(7.810,1.315)}
\gppoint{gp mark 7}{(7.821,1.315)}
\gppoint{gp mark 7}{(7.832,1.315)}
\gppoint{gp mark 7}{(7.843,1.315)}
\gppoint{gp mark 7}{(7.854,1.315)}
\gppoint{gp mark 7}{(7.865,1.315)}
\gppoint{gp mark 7}{(7.876,1.315)}
\gppoint{gp mark 7}{(7.887,1.315)}
\gppoint{gp mark 7}{(7.898,1.315)}
\gppoint{gp mark 7}{(7.909,1.315)}
\gppoint{gp mark 7}{(7.920,1.315)}
\gppoint{gp mark 7}{(7.931,1.315)}
\gppoint{gp mark 7}{(7.942,1.315)}
\gpcolor{rgb color={1.000,0.000,0.000}}
\gpsetpointsize{4.44}
\gppoint{gp mark 7}{(1.206,4.628)}
\gppoint{gp mark 7}{(1.217,4.623)}
\gppoint{gp mark 7}{(1.228,4.618)}
\gppoint{gp mark 7}{(1.239,4.613)}
\gppoint{gp mark 7}{(1.250,4.607)}
\gppoint{gp mark 7}{(1.261,4.602)}
\gppoint{gp mark 7}{(1.272,4.597)}
\gppoint{gp mark 7}{(1.283,4.591)}
\gppoint{gp mark 7}{(1.294,4.586)}
\gppoint{gp mark 7}{(1.305,4.581)}
\gppoint{gp mark 7}{(1.316,4.576)}
\gppoint{gp mark 7}{(1.327,4.570)}
\gppoint{gp mark 7}{(1.338,4.565)}
\gppoint{gp mark 7}{(1.349,4.560)}
\gppoint{gp mark 7}{(1.360,4.554)}
\gppoint{gp mark 7}{(1.371,4.549)}
\gppoint{gp mark 7}{(1.382,4.544)}
\gppoint{gp mark 7}{(1.393,4.539)}
\gppoint{gp mark 7}{(1.404,4.533)}
\gppoint{gp mark 7}{(1.415,4.528)}
\gppoint{gp mark 7}{(1.426,4.523)}
\gppoint{gp mark 7}{(1.437,4.517)}
\gppoint{gp mark 7}{(1.448,4.512)}
\gppoint{gp mark 7}{(1.459,4.507)}
\gppoint{gp mark 7}{(1.470,4.502)}
\gppoint{gp mark 7}{(1.481,4.496)}
\gppoint{gp mark 7}{(1.492,4.491)}
\gppoint{gp mark 7}{(1.503,4.486)}
\gppoint{gp mark 7}{(1.514,4.480)}
\gppoint{gp mark 7}{(1.525,4.475)}
\gppoint{gp mark 7}{(1.536,4.470)}
\gppoint{gp mark 7}{(1.547,4.464)}
\gppoint{gp mark 7}{(1.558,4.459)}
\gppoint{gp mark 7}{(1.568,4.454)}
\gppoint{gp mark 7}{(1.579,4.448)}
\gppoint{gp mark 7}{(1.590,4.443)}
\gppoint{gp mark 7}{(1.601,4.438)}
\gppoint{gp mark 7}{(1.612,4.433)}
\gppoint{gp mark 7}{(1.623,4.427)}
\gppoint{gp mark 7}{(1.634,4.422)}
\gppoint{gp mark 7}{(1.645,4.417)}
\gppoint{gp mark 7}{(1.656,4.411)}
\gppoint{gp mark 7}{(1.667,4.406)}
\gppoint{gp mark 7}{(1.678,4.401)}
\gppoint{gp mark 7}{(1.689,4.395)}
\gppoint{gp mark 7}{(1.700,4.390)}
\gppoint{gp mark 7}{(1.711,4.385)}
\gppoint{gp mark 7}{(1.722,4.379)}
\gppoint{gp mark 7}{(1.733,4.374)}
\gppoint{gp mark 7}{(1.744,4.369)}
\gppoint{gp mark 7}{(1.755,4.363)}
\gppoint{gp mark 7}{(1.766,4.358)}
\gppoint{gp mark 7}{(1.777,4.353)}
\gppoint{gp mark 7}{(1.788,4.347)}
\gppoint{gp mark 7}{(1.799,4.342)}
\gppoint{gp mark 7}{(1.810,4.337)}
\gppoint{gp mark 7}{(1.821,4.331)}
\gppoint{gp mark 7}{(1.832,4.326)}
\gppoint{gp mark 7}{(1.843,4.321)}
\gppoint{gp mark 7}{(1.854,4.315)}
\gppoint{gp mark 7}{(1.865,4.310)}
\gppoint{gp mark 7}{(1.876,4.305)}
\gppoint{gp mark 7}{(1.887,4.299)}
\gppoint{gp mark 7}{(1.898,4.294)}
\gppoint{gp mark 7}{(1.909,4.289)}
\gppoint{gp mark 7}{(1.920,4.283)}
\gppoint{gp mark 7}{(1.931,4.278)}
\gppoint{gp mark 7}{(1.942,4.273)}
\gppoint{gp mark 7}{(1.953,4.267)}
\gppoint{gp mark 7}{(1.964,4.262)}
\gppoint{gp mark 7}{(1.975,4.257)}
\gppoint{gp mark 7}{(1.986,4.251)}
\gppoint{gp mark 7}{(1.997,4.246)}
\gppoint{gp mark 7}{(2.008,4.241)}
\gppoint{gp mark 7}{(2.019,4.235)}
\gppoint{gp mark 7}{(2.030,4.225)}
\gppoint{gp mark 7}{(2.041,4.219)}
\gppoint{gp mark 7}{(2.052,4.214)}
\gppoint{gp mark 7}{(2.063,4.209)}
\gppoint{gp mark 7}{(2.074,4.204)}
\gppoint{gp mark 7}{(2.085,4.198)}
\gppoint{gp mark 7}{(2.096,4.193)}
\gppoint{gp mark 7}{(2.107,4.188)}
\gppoint{gp mark 7}{(2.118,4.183)}
\gppoint{gp mark 7}{(2.129,4.177)}
\gppoint{gp mark 7}{(2.140,4.172)}
\gppoint{gp mark 7}{(2.151,4.167)}
\gppoint{gp mark 7}{(2.162,4.162)}
\gppoint{gp mark 7}{(2.173,4.157)}
\gppoint{gp mark 7}{(2.184,4.151)}
\gppoint{gp mark 7}{(2.195,4.146)}
\gppoint{gp mark 7}{(2.206,4.141)}
\gppoint{gp mark 7}{(2.217,4.136)}
\gppoint{gp mark 7}{(2.228,4.131)}
\gppoint{gp mark 7}{(2.239,4.126)}
\gppoint{gp mark 7}{(2.250,4.121)}
\gppoint{gp mark 7}{(2.261,4.116)}
\gppoint{gp mark 7}{(2.272,4.111)}
\gppoint{gp mark 7}{(2.283,4.106)}
\gppoint{gp mark 7}{(2.294,4.101)}
\gppoint{gp mark 7}{(2.305,4.096)}
\gppoint{gp mark 7}{(2.316,4.091)}
\gppoint{gp mark 7}{(2.327,4.086)}
\gppoint{gp mark 7}{(2.338,4.081)}
\gppoint{gp mark 7}{(2.349,4.076)}
\gppoint{gp mark 7}{(2.360,4.071)}
\gppoint{gp mark 7}{(2.371,4.066)}
\gppoint{gp mark 7}{(2.382,4.062)}
\gppoint{gp mark 7}{(2.393,4.057)}
\gppoint{gp mark 7}{(2.404,4.052)}
\gppoint{gp mark 7}{(2.415,4.048)}
\gppoint{gp mark 7}{(2.426,4.043)}
\gppoint{gp mark 7}{(2.437,4.039)}
\gppoint{gp mark 7}{(2.448,4.034)}
\gppoint{gp mark 7}{(2.459,4.030)}
\gppoint{gp mark 7}{(2.470,4.027)}
\gppoint{gp mark 7}{(2.480,4.023)}
\gppoint{gp mark 7}{(2.491,4.021)}
\gppoint{gp mark 7}{(2.502,4.019)}
\gppoint{gp mark 7}{(2.513,4.019)}
\gppoint{gp mark 7}{(2.524,4.019)}
\gppoint{gp mark 7}{(2.535,4.019)}
\gppoint{gp mark 7}{(2.546,4.019)}
\gppoint{gp mark 7}{(2.557,4.019)}
\gppoint{gp mark 7}{(2.568,4.020)}
\gppoint{gp mark 7}{(2.579,4.021)}
\gppoint{gp mark 7}{(2.590,4.022)}
\gppoint{gp mark 7}{(2.601,4.022)}
\gppoint{gp mark 7}{(2.612,4.022)}
\gppoint{gp mark 7}{(2.623,4.022)}
\gppoint{gp mark 7}{(2.634,4.022)}
\gppoint{gp mark 7}{(2.645,4.022)}
\gppoint{gp mark 7}{(2.656,4.022)}
\gppoint{gp mark 7}{(2.667,4.022)}
\gppoint{gp mark 7}{(2.678,4.022)}
\gppoint{gp mark 7}{(2.689,4.022)}
\gppoint{gp mark 7}{(2.700,4.022)}
\gppoint{gp mark 7}{(2.711,4.022)}
\gppoint{gp mark 7}{(2.722,4.022)}
\gppoint{gp mark 7}{(2.733,4.022)}
\gppoint{gp mark 7}{(2.744,4.022)}
\gppoint{gp mark 7}{(2.755,4.022)}
\gppoint{gp mark 7}{(2.766,4.022)}
\gppoint{gp mark 7}{(2.777,4.022)}
\gppoint{gp mark 7}{(2.788,4.022)}
\gppoint{gp mark 7}{(2.799,4.022)}
\gppoint{gp mark 7}{(2.810,4.022)}
\gppoint{gp mark 7}{(2.821,4.022)}
\gppoint{gp mark 7}{(2.832,4.022)}
\gppoint{gp mark 7}{(2.843,4.022)}
\gppoint{gp mark 7}{(2.854,4.022)}
\gppoint{gp mark 7}{(2.865,4.022)}
\gppoint{gp mark 7}{(2.876,4.022)}
\gppoint{gp mark 7}{(2.887,4.022)}
\gppoint{gp mark 7}{(2.898,4.022)}
\gppoint{gp mark 7}{(2.909,4.022)}
\gppoint{gp mark 7}{(2.920,4.022)}
\gppoint{gp mark 7}{(2.931,4.022)}
\gppoint{gp mark 7}{(2.942,4.022)}
\gppoint{gp mark 7}{(2.953,4.022)}
\gppoint{gp mark 7}{(2.964,4.022)}
\gppoint{gp mark 7}{(2.975,4.022)}
\gppoint{gp mark 7}{(2.986,4.023)}
\gppoint{gp mark 7}{(2.997,4.023)}
\gppoint{gp mark 7}{(3.008,4.023)}
\gppoint{gp mark 7}{(3.019,4.023)}
\gppoint{gp mark 7}{(3.030,4.023)}
\gppoint{gp mark 7}{(3.041,4.023)}
\gppoint{gp mark 7}{(3.052,4.023)}
\gppoint{gp mark 7}{(3.063,4.023)}
\gppoint{gp mark 7}{(3.074,4.022)}
\gppoint{gp mark 7}{(3.085,4.022)}
\gppoint{gp mark 7}{(3.096,4.022)}
\gppoint{gp mark 7}{(3.107,4.022)}
\gppoint{gp mark 7}{(3.118,4.022)}
\gppoint{gp mark 7}{(3.129,4.022)}
\gppoint{gp mark 7}{(3.140,4.022)}
\gppoint{gp mark 7}{(3.151,4.022)}
\gppoint{gp mark 7}{(3.162,4.022)}
\gppoint{gp mark 7}{(3.173,4.022)}
\gppoint{gp mark 7}{(3.184,4.023)}
\gppoint{gp mark 7}{(3.195,4.023)}
\gppoint{gp mark 7}{(3.206,4.023)}
\gppoint{gp mark 7}{(3.217,4.023)}
\gppoint{gp mark 7}{(3.228,4.023)}
\gppoint{gp mark 7}{(3.239,4.023)}
\gppoint{gp mark 7}{(3.250,4.023)}
\gppoint{gp mark 7}{(3.261,4.023)}
\gppoint{gp mark 7}{(3.272,4.023)}
\gppoint{gp mark 7}{(3.283,4.023)}
\gppoint{gp mark 7}{(3.294,4.023)}
\gppoint{gp mark 7}{(3.305,4.023)}
\gppoint{gp mark 7}{(3.316,4.023)}
\gppoint{gp mark 7}{(3.327,4.023)}
\gppoint{gp mark 7}{(3.338,4.023)}
\gppoint{gp mark 7}{(3.349,4.023)}
\gppoint{gp mark 7}{(3.360,4.023)}
\gppoint{gp mark 7}{(3.371,4.023)}
\gppoint{gp mark 7}{(3.382,4.023)}
\gppoint{gp mark 7}{(3.392,4.023)}
\gppoint{gp mark 7}{(3.403,4.023)}
\gppoint{gp mark 7}{(3.414,4.023)}
\gppoint{gp mark 7}{(3.425,4.023)}
\gppoint{gp mark 7}{(3.436,4.023)}
\gppoint{gp mark 7}{(3.447,4.023)}
\gppoint{gp mark 7}{(3.458,4.023)}
\gppoint{gp mark 7}{(3.469,4.023)}
\gppoint{gp mark 7}{(3.480,4.023)}
\gppoint{gp mark 7}{(3.491,4.022)}
\gppoint{gp mark 7}{(3.502,4.016)}
\gppoint{gp mark 7}{(3.513,4.016)}
\gppoint{gp mark 7}{(3.524,4.021)}
\gppoint{gp mark 7}{(3.535,1.066)}
\gppoint{gp mark 7}{(3.546,1.066)}
\gppoint{gp mark 7}{(3.557,1.084)}
\gppoint{gp mark 7}{(3.568,1.074)}
\gppoint{gp mark 7}{(3.579,1.089)}
\gppoint{gp mark 7}{(3.590,1.097)}
\gppoint{gp mark 7}{(3.601,1.090)}
\gppoint{gp mark 7}{(3.612,1.085)}
\gppoint{gp mark 7}{(3.623,1.087)}
\gppoint{gp mark 7}{(3.634,1.090)}
\gppoint{gp mark 7}{(3.645,1.090)}
\gppoint{gp mark 7}{(3.656,1.088)}
\gppoint{gp mark 7}{(3.667,1.087)}
\gppoint{gp mark 7}{(3.678,1.088)}
\gppoint{gp mark 7}{(3.689,1.089)}
\gppoint{gp mark 7}{(3.700,1.088)}
\gppoint{gp mark 7}{(3.711,1.088)}
\gppoint{gp mark 7}{(3.722,1.088)}
\gppoint{gp mark 7}{(3.733,1.088)}
\gppoint{gp mark 7}{(3.744,1.089)}
\gppoint{gp mark 7}{(3.755,1.089)}
\gppoint{gp mark 7}{(3.766,1.089)}
\gppoint{gp mark 7}{(3.777,1.089)}
\gppoint{gp mark 7}{(3.788,1.089)}
\gppoint{gp mark 7}{(3.799,1.089)}
\gppoint{gp mark 7}{(3.810,1.089)}
\gppoint{gp mark 7}{(3.821,1.089)}
\gppoint{gp mark 7}{(3.832,1.089)}
\gppoint{gp mark 7}{(3.843,1.089)}
\gppoint{gp mark 7}{(3.854,1.089)}
\gppoint{gp mark 7}{(3.865,1.088)}
\gppoint{gp mark 7}{(3.876,1.089)}
\gppoint{gp mark 7}{(3.887,1.089)}
\gppoint{gp mark 7}{(3.898,1.089)}
\gppoint{gp mark 7}{(3.909,1.089)}
\gppoint{gp mark 7}{(3.920,1.089)}
\gppoint{gp mark 7}{(3.931,1.089)}
\gppoint{gp mark 7}{(3.942,1.089)}
\gppoint{gp mark 7}{(3.953,1.089)}
\gppoint{gp mark 7}{(3.964,1.089)}
\gppoint{gp mark 7}{(3.975,1.089)}
\gppoint{gp mark 7}{(3.986,1.089)}
\gppoint{gp mark 7}{(3.997,1.089)}
\gppoint{gp mark 7}{(4.008,1.089)}
\gppoint{gp mark 7}{(4.019,1.089)}
\gppoint{gp mark 7}{(4.030,1.089)}
\gppoint{gp mark 7}{(4.041,1.089)}
\gppoint{gp mark 7}{(4.052,1.089)}
\gppoint{gp mark 7}{(4.063,1.089)}
\gppoint{gp mark 7}{(4.074,1.089)}
\gppoint{gp mark 7}{(4.085,1.089)}
\gppoint{gp mark 7}{(4.096,1.089)}
\gppoint{gp mark 7}{(4.107,1.089)}
\gppoint{gp mark 7}{(4.118,1.089)}
\gppoint{gp mark 7}{(4.129,1.089)}
\gppoint{gp mark 7}{(4.140,1.089)}
\gppoint{gp mark 7}{(4.151,1.089)}
\gppoint{gp mark 7}{(4.162,1.089)}
\gppoint{gp mark 7}{(4.173,1.089)}
\gppoint{gp mark 7}{(4.184,1.089)}
\gppoint{gp mark 7}{(4.195,1.089)}
\gppoint{gp mark 7}{(4.206,1.089)}
\gppoint{gp mark 7}{(4.217,1.089)}
\gppoint{gp mark 7}{(4.228,1.090)}
\gppoint{gp mark 7}{(4.239,1.090)}
\gppoint{gp mark 7}{(4.250,1.089)}
\gppoint{gp mark 7}{(4.261,1.089)}
\gppoint{gp mark 7}{(4.272,1.089)}
\gppoint{gp mark 7}{(4.283,1.089)}
\gppoint{gp mark 7}{(4.294,1.091)}
\gppoint{gp mark 7}{(4.304,1.101)}
\gppoint{gp mark 7}{(4.315,1.153)}
\gppoint{gp mark 7}{(4.326,1.269)}
\gppoint{gp mark 7}{(4.337,1.313)}
\gppoint{gp mark 7}{(4.348,1.317)}
\gppoint{gp mark 7}{(4.359,1.316)}
\gppoint{gp mark 7}{(4.370,1.316)}
\gppoint{gp mark 7}{(4.381,1.316)}
\gppoint{gp mark 7}{(4.392,1.317)}
\gppoint{gp mark 7}{(4.403,1.317)}
\gppoint{gp mark 7}{(4.414,1.317)}
\gppoint{gp mark 7}{(4.425,1.316)}
\gppoint{gp mark 7}{(4.436,1.317)}
\gppoint{gp mark 7}{(4.447,1.317)}
\gppoint{gp mark 7}{(4.458,1.317)}
\gppoint{gp mark 7}{(4.469,1.317)}
\gppoint{gp mark 7}{(4.480,1.317)}
\gppoint{gp mark 7}{(4.491,1.317)}
\gppoint{gp mark 7}{(4.502,1.317)}
\gppoint{gp mark 7}{(4.513,1.317)}
\gppoint{gp mark 7}{(4.524,1.317)}
\gppoint{gp mark 7}{(4.535,1.317)}
\gppoint{gp mark 7}{(4.546,1.317)}
\gppoint{gp mark 7}{(4.557,1.317)}
\gppoint{gp mark 7}{(4.568,1.316)}
\gppoint{gp mark 7}{(4.579,1.317)}
\gppoint{gp mark 7}{(4.590,1.317)}
\gppoint{gp mark 7}{(4.601,1.317)}
\gppoint{gp mark 7}{(4.612,1.317)}
\gppoint{gp mark 7}{(4.623,1.317)}
\gppoint{gp mark 7}{(4.634,1.317)}
\gppoint{gp mark 7}{(4.645,1.317)}
\gppoint{gp mark 7}{(4.656,1.317)}
\gppoint{gp mark 7}{(4.667,1.317)}
\gppoint{gp mark 7}{(4.678,1.317)}
\gppoint{gp mark 7}{(4.689,1.317)}
\gppoint{gp mark 7}{(4.700,1.317)}
\gppoint{gp mark 7}{(4.711,1.317)}
\gppoint{gp mark 7}{(4.722,1.317)}
\gppoint{gp mark 7}{(4.733,1.317)}
\gppoint{gp mark 7}{(4.744,1.317)}
\gppoint{gp mark 7}{(4.755,1.317)}
\gppoint{gp mark 7}{(4.766,1.317)}
\gppoint{gp mark 7}{(4.777,1.317)}
\gppoint{gp mark 7}{(4.788,1.317)}
\gppoint{gp mark 7}{(4.799,1.317)}
\gppoint{gp mark 7}{(4.810,1.317)}
\gppoint{gp mark 7}{(4.821,1.317)}
\gppoint{gp mark 7}{(4.832,1.317)}
\gppoint{gp mark 7}{(4.843,1.317)}
\gppoint{gp mark 7}{(4.854,1.317)}
\gppoint{gp mark 7}{(4.865,1.317)}
\gppoint{gp mark 7}{(4.876,1.317)}
\gppoint{gp mark 7}{(4.887,1.317)}
\gppoint{gp mark 7}{(4.898,1.317)}
\gppoint{gp mark 7}{(4.909,1.317)}
\gppoint{gp mark 7}{(4.920,1.317)}
\gppoint{gp mark 7}{(4.931,1.317)}
\gppoint{gp mark 7}{(4.942,1.317)}
\gppoint{gp mark 7}{(4.953,1.317)}
\gppoint{gp mark 7}{(4.964,1.317)}
\gppoint{gp mark 7}{(4.975,1.317)}
\gppoint{gp mark 7}{(4.986,1.317)}
\gppoint{gp mark 7}{(4.997,1.317)}
\gppoint{gp mark 7}{(5.008,1.317)}
\gppoint{gp mark 7}{(5.019,1.317)}
\gppoint{gp mark 7}{(5.030,1.317)}
\gppoint{gp mark 7}{(5.041,1.315)}
\gppoint{gp mark 7}{(5.052,1.315)}
\gppoint{gp mark 7}{(5.063,1.315)}
\gppoint{gp mark 7}{(5.074,1.315)}
\gppoint{gp mark 7}{(5.085,1.315)}
\gppoint{gp mark 7}{(5.096,1.315)}
\gppoint{gp mark 7}{(5.107,1.315)}
\gppoint{gp mark 7}{(5.118,1.315)}
\gppoint{gp mark 7}{(5.129,1.315)}
\gppoint{gp mark 7}{(5.140,1.315)}
\gppoint{gp mark 7}{(5.151,1.315)}
\gppoint{gp mark 7}{(5.162,1.315)}
\gppoint{gp mark 7}{(5.173,1.315)}
\gppoint{gp mark 7}{(5.184,1.315)}
\gppoint{gp mark 7}{(5.195,1.315)}
\gppoint{gp mark 7}{(5.206,1.315)}
\gppoint{gp mark 7}{(5.216,1.315)}
\gppoint{gp mark 7}{(5.227,1.315)}
\gppoint{gp mark 7}{(5.238,1.315)}
\gppoint{gp mark 7}{(5.249,1.315)}
\gppoint{gp mark 7}{(5.260,1.315)}
\gppoint{gp mark 7}{(5.271,1.315)}
\gppoint{gp mark 7}{(5.282,1.315)}
\gppoint{gp mark 7}{(5.293,1.315)}
\gppoint{gp mark 7}{(5.304,1.315)}
\gppoint{gp mark 7}{(5.315,1.315)}
\gppoint{gp mark 7}{(5.326,1.315)}
\gppoint{gp mark 7}{(5.337,1.315)}
\gppoint{gp mark 7}{(5.348,1.315)}
\gppoint{gp mark 7}{(5.359,1.315)}
\gppoint{gp mark 7}{(5.370,1.315)}
\gppoint{gp mark 7}{(5.381,1.315)}
\gppoint{gp mark 7}{(5.392,1.315)}
\gppoint{gp mark 7}{(5.403,1.315)}
\gppoint{gp mark 7}{(5.414,1.315)}
\gppoint{gp mark 7}{(5.425,1.315)}
\gppoint{gp mark 7}{(5.436,1.315)}
\gppoint{gp mark 7}{(5.447,1.315)}
\gppoint{gp mark 7}{(5.458,1.315)}
\gppoint{gp mark 7}{(5.469,1.315)}
\gppoint{gp mark 7}{(5.480,1.315)}
\gppoint{gp mark 7}{(5.491,1.315)}
\gppoint{gp mark 7}{(5.502,1.315)}
\gppoint{gp mark 7}{(5.513,1.315)}
\gppoint{gp mark 7}{(5.524,1.315)}
\gppoint{gp mark 7}{(5.535,1.315)}
\gppoint{gp mark 7}{(5.546,1.315)}
\gppoint{gp mark 7}{(5.557,1.315)}
\gppoint{gp mark 7}{(5.568,1.315)}
\gppoint{gp mark 7}{(5.579,1.315)}
\gppoint{gp mark 7}{(5.590,1.315)}
\gppoint{gp mark 7}{(5.601,1.315)}
\gppoint{gp mark 7}{(5.612,1.315)}
\gppoint{gp mark 7}{(5.623,1.315)}
\gppoint{gp mark 7}{(5.634,1.315)}
\gppoint{gp mark 7}{(5.645,1.315)}
\gppoint{gp mark 7}{(5.656,1.315)}
\gppoint{gp mark 7}{(5.667,1.315)}
\gppoint{gp mark 7}{(5.678,1.315)}
\gppoint{gp mark 7}{(5.689,1.315)}
\gppoint{gp mark 7}{(5.700,1.315)}
\gppoint{gp mark 7}{(5.711,1.315)}
\gppoint{gp mark 7}{(5.722,1.315)}
\gppoint{gp mark 7}{(5.733,1.315)}
\gppoint{gp mark 7}{(5.744,1.315)}
\gppoint{gp mark 7}{(5.755,1.315)}
\gppoint{gp mark 7}{(5.766,1.315)}
\gppoint{gp mark 7}{(5.777,1.315)}
\gppoint{gp mark 7}{(5.788,1.315)}
\gppoint{gp mark 7}{(5.799,1.315)}
\gppoint{gp mark 7}{(5.810,1.315)}
\gppoint{gp mark 7}{(5.821,1.315)}
\gppoint{gp mark 7}{(5.832,1.315)}
\gppoint{gp mark 7}{(5.843,1.315)}
\gppoint{gp mark 7}{(5.854,1.315)}
\gppoint{gp mark 7}{(5.865,1.315)}
\gppoint{gp mark 7}{(5.876,1.315)}
\gppoint{gp mark 7}{(5.887,1.315)}
\gppoint{gp mark 7}{(5.898,1.315)}
\gppoint{gp mark 7}{(5.909,1.315)}
\gppoint{gp mark 7}{(5.920,1.315)}
\gppoint{gp mark 7}{(5.931,1.315)}
\gppoint{gp mark 7}{(5.942,1.315)}
\gppoint{gp mark 7}{(5.953,1.315)}
\gppoint{gp mark 7}{(5.964,1.315)}
\gppoint{gp mark 7}{(5.975,1.315)}
\gppoint{gp mark 7}{(5.986,1.315)}
\gppoint{gp mark 7}{(5.997,1.315)}
\gppoint{gp mark 7}{(6.008,1.315)}
\gppoint{gp mark 7}{(6.019,1.315)}
\gppoint{gp mark 7}{(6.030,1.315)}
\gppoint{gp mark 7}{(6.041,1.315)}
\gppoint{gp mark 7}{(6.052,1.315)}
\gppoint{gp mark 7}{(6.063,1.315)}
\gppoint{gp mark 7}{(6.074,1.315)}
\gppoint{gp mark 7}{(6.085,1.315)}
\gppoint{gp mark 7}{(6.096,1.315)}
\gppoint{gp mark 7}{(6.107,1.315)}
\gppoint{gp mark 7}{(6.118,1.315)}
\gppoint{gp mark 7}{(6.128,1.315)}
\gppoint{gp mark 7}{(6.139,1.315)}
\gppoint{gp mark 7}{(6.150,1.315)}
\gppoint{gp mark 7}{(6.161,1.315)}
\gppoint{gp mark 7}{(6.172,1.315)}
\gppoint{gp mark 7}{(6.183,1.315)}
\gppoint{gp mark 7}{(6.194,1.315)}
\gppoint{gp mark 7}{(6.205,1.315)}
\gppoint{gp mark 7}{(6.216,1.315)}
\gppoint{gp mark 7}{(6.227,1.315)}
\gppoint{gp mark 7}{(6.238,1.315)}
\gppoint{gp mark 7}{(6.249,1.315)}
\gppoint{gp mark 7}{(6.260,1.315)}
\gppoint{gp mark 7}{(6.271,1.315)}
\gppoint{gp mark 7}{(6.282,1.315)}
\gppoint{gp mark 7}{(6.293,1.315)}
\gppoint{gp mark 7}{(6.304,1.315)}
\gppoint{gp mark 7}{(6.315,1.315)}
\gppoint{gp mark 7}{(6.326,1.315)}
\gppoint{gp mark 7}{(6.337,1.315)}
\gppoint{gp mark 7}{(6.348,1.315)}
\gppoint{gp mark 7}{(6.359,1.315)}
\gppoint{gp mark 7}{(6.370,1.315)}
\gppoint{gp mark 7}{(6.381,1.315)}
\gppoint{gp mark 7}{(6.392,1.315)}
\gppoint{gp mark 7}{(6.403,1.315)}
\gppoint{gp mark 7}{(6.414,1.315)}
\gppoint{gp mark 7}{(6.425,1.315)}
\gppoint{gp mark 7}{(6.436,1.315)}
\gppoint{gp mark 7}{(6.447,1.315)}
\gppoint{gp mark 7}{(6.458,1.315)}
\gppoint{gp mark 7}{(6.469,1.315)}
\gppoint{gp mark 7}{(6.480,1.315)}
\gppoint{gp mark 7}{(6.491,1.315)}
\gppoint{gp mark 7}{(6.502,1.315)}
\gppoint{gp mark 7}{(6.513,1.315)}
\gppoint{gp mark 7}{(6.524,1.315)}
\gppoint{gp mark 7}{(6.535,1.315)}
\gppoint{gp mark 7}{(6.546,1.315)}
\gppoint{gp mark 7}{(6.557,1.315)}
\gppoint{gp mark 7}{(6.568,1.315)}
\gppoint{gp mark 7}{(6.579,1.315)}
\gppoint{gp mark 7}{(6.590,1.315)}
\gppoint{gp mark 7}{(6.601,1.315)}
\gppoint{gp mark 7}{(6.612,1.315)}
\gppoint{gp mark 7}{(6.623,1.315)}
\gppoint{gp mark 7}{(6.634,1.315)}
\gppoint{gp mark 7}{(6.645,1.315)}
\gppoint{gp mark 7}{(6.656,1.315)}
\gppoint{gp mark 7}{(6.667,1.315)}
\gppoint{gp mark 7}{(6.678,1.315)}
\gppoint{gp mark 7}{(6.689,1.315)}
\gppoint{gp mark 7}{(6.700,1.315)}
\gppoint{gp mark 7}{(6.711,1.315)}
\gppoint{gp mark 7}{(6.722,1.315)}
\gppoint{gp mark 7}{(6.733,1.315)}
\gppoint{gp mark 7}{(6.744,1.315)}
\gppoint{gp mark 7}{(6.755,1.315)}
\gppoint{gp mark 7}{(6.766,1.315)}
\gppoint{gp mark 7}{(6.777,1.315)}
\gppoint{gp mark 7}{(6.788,1.315)}
\gppoint{gp mark 7}{(6.799,1.315)}
\gppoint{gp mark 7}{(6.810,1.315)}
\gppoint{gp mark 7}{(6.821,1.315)}
\gppoint{gp mark 7}{(6.832,1.315)}
\gppoint{gp mark 7}{(6.843,1.315)}
\gppoint{gp mark 7}{(6.854,1.315)}
\gppoint{gp mark 7}{(6.865,1.315)}
\gppoint{gp mark 7}{(6.876,1.315)}
\gppoint{gp mark 7}{(6.887,1.315)}
\gppoint{gp mark 7}{(6.898,1.315)}
\gppoint{gp mark 7}{(6.909,1.315)}
\gppoint{gp mark 7}{(6.920,1.315)}
\gppoint{gp mark 7}{(6.931,1.315)}
\gppoint{gp mark 7}{(6.942,1.315)}
\gppoint{gp mark 7}{(6.953,1.315)}
\gppoint{gp mark 7}{(6.964,1.315)}
\gppoint{gp mark 7}{(6.975,1.315)}
\gppoint{gp mark 7}{(6.986,1.315)}
\gppoint{gp mark 7}{(6.997,1.315)}
\gppoint{gp mark 7}{(7.008,1.315)}
\gppoint{gp mark 7}{(7.019,1.315)}
\gppoint{gp mark 7}{(7.030,1.315)}
\gppoint{gp mark 7}{(7.040,1.315)}
\gppoint{gp mark 7}{(7.051,1.315)}
\gppoint{gp mark 7}{(7.062,1.315)}
\gppoint{gp mark 7}{(7.073,1.315)}
\gppoint{gp mark 7}{(7.084,1.315)}
\gppoint{gp mark 7}{(7.095,1.315)}
\gppoint{gp mark 7}{(7.106,1.315)}
\gppoint{gp mark 7}{(7.117,1.315)}
\gppoint{gp mark 7}{(7.128,1.315)}
\gppoint{gp mark 7}{(7.139,1.315)}
\gppoint{gp mark 7}{(7.150,1.315)}
\gppoint{gp mark 7}{(7.161,1.315)}
\gppoint{gp mark 7}{(7.172,1.315)}
\gppoint{gp mark 7}{(7.183,1.315)}
\gppoint{gp mark 7}{(7.194,1.315)}
\gppoint{gp mark 7}{(7.205,1.315)}
\gppoint{gp mark 7}{(7.216,1.315)}
\gppoint{gp mark 7}{(7.227,1.315)}
\gppoint{gp mark 7}{(7.238,1.315)}
\gppoint{gp mark 7}{(7.249,1.315)}
\gppoint{gp mark 7}{(7.260,1.315)}
\gppoint{gp mark 7}{(7.271,1.315)}
\gppoint{gp mark 7}{(7.282,1.315)}
\gppoint{gp mark 7}{(7.293,1.315)}
\gppoint{gp mark 7}{(7.304,1.315)}
\gppoint{gp mark 7}{(7.315,1.315)}
\gppoint{gp mark 7}{(7.326,1.315)}
\gppoint{gp mark 7}{(7.337,1.315)}
\gppoint{gp mark 7}{(7.348,1.315)}
\gppoint{gp mark 7}{(7.359,1.315)}
\gppoint{gp mark 7}{(7.370,1.315)}
\gppoint{gp mark 7}{(7.381,1.315)}
\gppoint{gp mark 7}{(7.392,1.315)}
\gppoint{gp mark 7}{(7.403,1.315)}
\gppoint{gp mark 7}{(7.414,1.315)}
\gppoint{gp mark 7}{(7.425,1.315)}
\gppoint{gp mark 7}{(7.436,1.315)}
\gppoint{gp mark 7}{(7.447,1.315)}
\gppoint{gp mark 7}{(7.458,1.315)}
\gppoint{gp mark 7}{(7.469,1.315)}
\gppoint{gp mark 7}{(7.480,1.315)}
\gppoint{gp mark 7}{(7.491,1.315)}
\gppoint{gp mark 7}{(7.502,1.315)}
\gppoint{gp mark 7}{(7.513,1.315)}
\gppoint{gp mark 7}{(7.524,1.315)}
\gppoint{gp mark 7}{(7.535,1.315)}
\gppoint{gp mark 7}{(7.546,1.315)}
\gppoint{gp mark 7}{(7.557,1.315)}
\gppoint{gp mark 7}{(7.568,1.315)}
\gppoint{gp mark 7}{(7.579,1.315)}
\gppoint{gp mark 7}{(7.590,1.315)}
\gppoint{gp mark 7}{(7.601,1.315)}
\gppoint{gp mark 7}{(7.612,1.315)}
\gppoint{gp mark 7}{(7.623,1.315)}
\gppoint{gp mark 7}{(7.634,1.315)}
\gppoint{gp mark 7}{(7.645,1.315)}
\gppoint{gp mark 7}{(7.656,1.315)}
\gppoint{gp mark 7}{(7.667,1.315)}
\gppoint{gp mark 7}{(7.678,1.315)}
\gppoint{gp mark 7}{(7.689,1.315)}
\gppoint{gp mark 7}{(7.700,1.315)}
\gppoint{gp mark 7}{(7.711,1.315)}
\gppoint{gp mark 7}{(7.722,1.315)}
\gppoint{gp mark 7}{(7.733,1.315)}
\gppoint{gp mark 7}{(7.744,1.315)}
\gppoint{gp mark 7}{(7.755,1.315)}
\gppoint{gp mark 7}{(7.766,1.315)}
\gppoint{gp mark 7}{(7.777,1.315)}
\gppoint{gp mark 7}{(7.788,1.315)}
\gppoint{gp mark 7}{(7.799,1.315)}
\gppoint{gp mark 7}{(7.810,1.315)}
\gppoint{gp mark 7}{(7.821,1.315)}
\gppoint{gp mark 7}{(7.832,1.315)}
\gppoint{gp mark 7}{(7.843,1.315)}
\gppoint{gp mark 7}{(7.854,1.315)}
\gppoint{gp mark 7}{(7.865,1.315)}
\gppoint{gp mark 7}{(7.876,1.315)}
\gppoint{gp mark 7}{(7.887,1.315)}
\gppoint{gp mark 7}{(7.898,1.315)}
\gppoint{gp mark 7}{(7.909,1.315)}
\gppoint{gp mark 7}{(7.920,1.315)}
\gppoint{gp mark 7}{(7.931,1.315)}
\gppoint{gp mark 7}{(7.942,1.315)}
\gpcolor{rgb color={0.000,0.000,0.000}}
\gpsetlinetype{gp lt plot 0}
\gpsetlinewidth{4.00}
\draw[gp path] (2.411,4.028)--(3.526,4.028);
\draw[gp path] (3.526,1.086)--(4.329,1.086);
\draw[gp path] (4.329,1.315)--(7.511,1.315);
\draw[gp path] (7.511,1.315)--(7.947,1.315);
\draw[gp path] (1.202,4.617)--(1.208,4.614)--(1.214,4.611)--(1.220,4.608)--(1.226,4.605)%
  --(1.232,4.602)--(1.238,4.599)--(1.244,4.596)--(1.250,4.593)--(1.256,4.591)--(1.262,4.588)%
  --(1.268,4.585)--(1.275,4.582)--(1.281,4.579)--(1.287,4.576)--(1.293,4.573)--(1.299,4.570)%
  --(1.305,4.567)--(1.311,4.564)--(1.317,4.561)--(1.323,4.558)--(1.329,4.555)--(1.335,4.552)%
  --(1.341,4.549)--(1.347,4.546)--(1.353,4.543)--(1.359,4.540)--(1.365,4.538)--(1.371,4.535)%
  --(1.377,4.532)--(1.383,4.529)--(1.389,4.526)--(1.395,4.523)--(1.402,4.520)--(1.408,4.517)%
  --(1.414,4.514)--(1.420,4.511)--(1.426,4.508)--(1.432,4.505)--(1.438,4.502)--(1.444,4.499)%
  --(1.450,4.496)--(1.456,4.493)--(1.462,4.490)--(1.468,4.488)--(1.474,4.485)--(1.480,4.482)%
  --(1.486,4.479)--(1.492,4.476)--(1.498,4.473)--(1.504,4.470)--(1.510,4.467)--(1.516,4.464)%
  --(1.522,4.461)--(1.528,4.458)--(1.535,4.455)--(1.541,4.452)--(1.547,4.449)--(1.553,4.446)%
  --(1.559,4.443)--(1.565,4.440)--(1.571,4.437)--(1.577,4.435)--(1.583,4.432)--(1.589,4.429)%
  --(1.595,4.426)--(1.601,4.423)--(1.607,4.420)--(1.613,4.417)--(1.619,4.414)--(1.625,4.411)%
  --(1.631,4.408)--(1.637,4.405)--(1.643,4.402)--(1.649,4.399)--(1.655,4.396)--(1.662,4.393)%
  --(1.668,4.390)--(1.674,4.387)--(1.680,4.384)--(1.686,4.382)--(1.692,4.379)--(1.698,4.376)%
  --(1.704,4.373)--(1.710,4.370)--(1.716,4.367)--(1.722,4.364)--(1.728,4.361)--(1.734,4.358)%
  --(1.740,4.355)--(1.746,4.352)--(1.752,4.349)--(1.758,4.346)--(1.764,4.343)--(1.770,4.340)%
  --(1.776,4.337)--(1.782,4.334)--(1.788,4.331)--(1.795,4.329)--(1.801,4.326)--(1.807,4.323)%
  --(1.813,4.320)--(1.819,4.317)--(1.825,4.314)--(1.831,4.311)--(1.837,4.308)--(1.843,4.305)%
  --(1.849,4.302)--(1.855,4.299)--(1.861,4.296)--(1.867,4.293)--(1.873,4.290)--(1.879,4.287)%
  --(1.885,4.284)--(1.891,4.281)--(1.897,4.278)--(1.903,4.276)--(1.909,4.273)--(1.915,4.270)%
  --(1.922,4.267)--(1.928,4.264)--(1.934,4.261)--(1.940,4.258)--(1.946,4.255)--(1.952,4.252)%
  --(1.958,4.249)--(1.964,4.246)--(1.970,4.243)--(1.976,4.240)--(1.982,4.237)--(1.988,4.234)%
  --(1.994,4.231)--(2.000,4.228)--(2.006,4.225)--(2.012,4.223)--(2.018,4.220)--(2.024,4.217)%
  --(2.030,4.214)--(2.036,4.211)--(2.042,4.208)--(2.048,4.205)--(2.055,4.202)--(2.061,4.199)%
  --(2.067,4.196)--(2.073,4.193)--(2.079,4.190)--(2.085,4.187)--(2.091,4.184)--(2.097,4.181)%
  --(2.103,4.178)--(2.109,4.175)--(2.115,4.172)--(2.121,4.170)--(2.127,4.167)--(2.133,4.164)%
  --(2.139,4.161)--(2.145,4.158)--(2.151,4.155)--(2.157,4.152)--(2.163,4.149)--(2.169,4.146)%
  --(2.175,4.143)--(2.182,4.140)--(2.188,4.137)--(2.194,4.134)--(2.200,4.131)--(2.206,4.128)%
  --(2.212,4.125)--(2.218,4.122)--(2.224,4.119)--(2.230,4.117)--(2.236,4.114)--(2.242,4.111)%
  --(2.248,4.108)--(2.254,4.105)--(2.260,4.102)--(2.266,4.099)--(2.272,4.096)--(2.278,4.093)%
  --(2.284,4.090)--(2.290,4.087)--(2.296,4.084)--(2.302,4.081)--(2.308,4.078)--(2.315,4.075)%
  --(2.321,4.072)--(2.327,4.069)--(2.333,4.067)--(2.339,4.064)--(2.345,4.061)--(2.351,4.058)%
  --(2.357,4.055)--(2.363,4.052)--(2.369,4.049)--(2.375,4.046)--(2.381,4.043)--(2.387,4.040)%
  --(2.393,4.037)--(2.399,4.034)--(2.405,4.031)--(2.411,4.028);
\draw[gp path] (3.526,4.028)--(3.526,1.086);
\draw[gp path] (4.329,1.086)--(4.329,1.315);
\draw[gp path] (3.896,3.695)--(4.572,3.695);
\gpcolor{rgb color={1.000,0.000,0.000}}
\gpsetlinewidth{0.50}
\gppoint{gp mark 7}{(4.234,2.921)}
\gpcolor{rgb color={0.502,0.502,0.502}}
\gppoint{gp mark 7}{(4.234,2.147)}
\gpcolor{rgb color={0.000,0.000,0.000}}
\node[gp node left,font={\fontsize{10pt}{12pt}\selectfont}] at (1.421,5.166) {\LARGE $B_y$};
\node[gp node left,font={\fontsize{10pt}{12pt}\selectfont}] at (6.147,5.166) {\large $\alpha = 3.0$};
\node[gp node left,font={\fontsize{10pt}{12pt}\selectfont}] at (4.797,3.695) {\large exact};
\node[gp node left,font={\fontsize{10pt}{12pt}\selectfont}] at (4.797,2.921) {\large HLLD-CWM};
\node[gp node left,font={\fontsize{10pt}{12pt}\selectfont}] at (4.797,2.147) {\large HLLD};
%% coordinates of the plot area
\gpdefrectangularnode{gp plot 1}{\pgfpoint{1.196cm}{0.985cm}}{\pgfpoint{7.947cm}{5.631cm}}
\end{tikzpicture}
%% gnuplot variables
} \\
\resizebox{0.5\linewidth}{!}{\tikzsetnextfilename{coplanar_a_crsol_1}\begin{tikzpicture}[gnuplot]
%% generated with GNUPLOT 4.6p4 (Lua 5.1; terminal rev. 99, script rev. 100)
%% Mon 02 Jun 2014 11:38:52 AM EDT
\path (0.000,0.000) rectangle (8.500,6.000);
\gpfill{rgb color={1.000,1.000,1.000}} (1.196,0.985)--(7.946,0.985)--(7.946,5.630)--(1.196,5.630)--cycle;
\gpcolor{color=gp lt color border}
\gpsetlinetype{gp lt border}
\gpsetlinewidth{1.00}
\draw[gp path] (1.196,0.985)--(1.196,5.630)--(7.946,5.630)--(7.946,0.985)--cycle;
\gpcolor{color=gp lt color axes}
\gpsetlinetype{gp lt axes}
\gpsetlinewidth{2.00}
\draw[gp path] (1.196,0.985)--(7.947,0.985);
\gpcolor{color=gp lt color border}
\gpsetlinetype{gp lt border}
\draw[gp path] (1.196,0.985)--(1.268,0.985);
\draw[gp path] (7.947,0.985)--(7.875,0.985);
\gpcolor{rgb color={0.000,0.000,0.000}}
\node[gp node right,font={\fontsize{10pt}{12pt}\selectfont}] at (1.012,0.985) {0.6};
\gpcolor{color=gp lt color axes}
\gpsetlinetype{gp lt axes}
\draw[gp path] (1.196,1.759)--(7.947,1.759);
\gpcolor{color=gp lt color border}
\gpsetlinetype{gp lt border}
\draw[gp path] (1.196,1.759)--(1.268,1.759);
\draw[gp path] (7.947,1.759)--(7.875,1.759);
\gpcolor{rgb color={0.000,0.000,0.000}}
\node[gp node right,font={\fontsize{10pt}{12pt}\selectfont}] at (1.012,1.759) {0.65};
\gpcolor{color=gp lt color axes}
\gpsetlinetype{gp lt axes}
\draw[gp path] (1.196,2.534)--(7.947,2.534);
\gpcolor{color=gp lt color border}
\gpsetlinetype{gp lt border}
\draw[gp path] (1.196,2.534)--(1.268,2.534);
\draw[gp path] (7.947,2.534)--(7.875,2.534);
\gpcolor{rgb color={0.000,0.000,0.000}}
\node[gp node right,font={\fontsize{10pt}{12pt}\selectfont}] at (1.012,2.534) {0.7};
\gpcolor{color=gp lt color axes}
\gpsetlinetype{gp lt axes}
\draw[gp path] (1.196,3.308)--(7.947,3.308);
\gpcolor{color=gp lt color border}
\gpsetlinetype{gp lt border}
\draw[gp path] (1.196,3.308)--(1.268,3.308);
\draw[gp path] (7.947,3.308)--(7.875,3.308);
\gpcolor{rgb color={0.000,0.000,0.000}}
\node[gp node right,font={\fontsize{10pt}{12pt}\selectfont}] at (1.012,3.308) {0.75};
\gpcolor{color=gp lt color axes}
\gpsetlinetype{gp lt axes}
\draw[gp path] (1.196,4.082)--(7.947,4.082);
\gpcolor{color=gp lt color border}
\gpsetlinetype{gp lt border}
\draw[gp path] (1.196,4.082)--(1.268,4.082);
\draw[gp path] (7.947,4.082)--(7.875,4.082);
\gpcolor{rgb color={0.000,0.000,0.000}}
\node[gp node right,font={\fontsize{10pt}{12pt}\selectfont}] at (1.012,4.082) {0.8};
\gpcolor{color=gp lt color axes}
\gpsetlinetype{gp lt axes}
\draw[gp path] (1.196,4.857)--(7.947,4.857);
\gpcolor{color=gp lt color border}
\gpsetlinetype{gp lt border}
\draw[gp path] (1.196,4.857)--(1.268,4.857);
\draw[gp path] (7.947,4.857)--(7.875,4.857);
\gpcolor{rgb color={0.000,0.000,0.000}}
\node[gp node right,font={\fontsize{10pt}{12pt}\selectfont}] at (1.012,4.857) {0.85};
\gpcolor{color=gp lt color axes}
\gpsetlinetype{gp lt axes}
\draw[gp path] (1.196,5.631)--(7.947,5.631);
\gpcolor{color=gp lt color border}
\gpsetlinetype{gp lt border}
\draw[gp path] (1.196,5.631)--(1.268,5.631);
\draw[gp path] (7.947,5.631)--(7.875,5.631);
\gpcolor{rgb color={0.000,0.000,0.000}}
\node[gp node right,font={\fontsize{10pt}{12pt}\selectfont}] at (1.012,5.631) {0.9};
\gpcolor{color=gp lt color axes}
\gpsetlinetype{gp lt axes}
\draw[gp path] (1.196,0.985)--(1.196,5.631);
\gpcolor{color=gp lt color border}
\gpsetlinetype{gp lt border}
\draw[gp path] (1.196,0.985)--(1.196,1.057);
\draw[gp path] (1.196,5.631)--(1.196,5.559);
\gpcolor{rgb color={0.000,0.000,0.000}}
\node[gp node center,font={\fontsize{10pt}{12pt}\selectfont}] at (1.196,0.677) {0.2};
\gpcolor{color=gp lt color axes}
\gpsetlinetype{gp lt axes}
\draw[gp path] (2.321,0.985)--(2.321,5.631);
\gpcolor{color=gp lt color border}
\gpsetlinetype{gp lt border}
\draw[gp path] (2.321,0.985)--(2.321,1.057);
\draw[gp path] (2.321,5.631)--(2.321,5.559);
\gpcolor{rgb color={0.000,0.000,0.000}}
\node[gp node center,font={\fontsize{10pt}{12pt}\selectfont}] at (2.321,0.677) {0.25};
\gpcolor{color=gp lt color axes}
\gpsetlinetype{gp lt axes}
\draw[gp path] (3.446,0.985)--(3.446,5.631);
\gpcolor{color=gp lt color border}
\gpsetlinetype{gp lt border}
\draw[gp path] (3.446,0.985)--(3.446,1.057);
\draw[gp path] (3.446,5.631)--(3.446,5.559);
\gpcolor{rgb color={0.000,0.000,0.000}}
\node[gp node center,font={\fontsize{10pt}{12pt}\selectfont}] at (3.446,0.677) {0.3};
\gpcolor{color=gp lt color axes}
\gpsetlinetype{gp lt axes}
\draw[gp path] (4.572,0.985)--(4.572,5.631);
\gpcolor{color=gp lt color border}
\gpsetlinetype{gp lt border}
\draw[gp path] (4.572,0.985)--(4.572,1.057);
\draw[gp path] (4.572,5.631)--(4.572,5.559);
\gpcolor{rgb color={0.000,0.000,0.000}}
\node[gp node center,font={\fontsize{10pt}{12pt}\selectfont}] at (4.572,0.677) {0.35};
\gpcolor{color=gp lt color axes}
\gpsetlinetype{gp lt axes}
\draw[gp path] (5.697,0.985)--(5.697,5.631);
\gpcolor{color=gp lt color border}
\gpsetlinetype{gp lt border}
\draw[gp path] (5.697,0.985)--(5.697,1.057);
\draw[gp path] (5.697,5.631)--(5.697,5.559);
\gpcolor{rgb color={0.000,0.000,0.000}}
\node[gp node center,font={\fontsize{10pt}{12pt}\selectfont}] at (5.697,0.677) {0.4};
\gpcolor{color=gp lt color axes}
\gpsetlinetype{gp lt axes}
\draw[gp path] (6.822,0.985)--(6.822,5.631);
\gpcolor{color=gp lt color border}
\gpsetlinetype{gp lt border}
\draw[gp path] (6.822,0.985)--(6.822,1.057);
\draw[gp path] (6.822,5.631)--(6.822,5.559);
\gpcolor{rgb color={0.000,0.000,0.000}}
\node[gp node center,font={\fontsize{10pt}{12pt}\selectfont}] at (6.822,0.677) {0.45};
\gpcolor{color=gp lt color axes}
\gpsetlinetype{gp lt axes}
\draw[gp path] (7.947,0.985)--(7.947,5.631);
\gpcolor{color=gp lt color border}
\gpsetlinetype{gp lt border}
\draw[gp path] (7.947,0.985)--(7.947,1.057);
\draw[gp path] (7.947,5.631)--(7.947,5.559);
\gpcolor{rgb color={0.000,0.000,0.000}}
\node[gp node center,font={\fontsize{10pt}{12pt}\selectfont}] at (7.947,0.677) {0.5};
\gpcolor{color=gp lt color border}
\draw[gp path] (1.196,5.631)--(1.196,0.985)--(7.947,0.985)--(7.947,5.631)--cycle;
\gpcolor{rgb color={0.000,0.000,0.000}}
\node[gp node center,font={\fontsize{10pt}{12pt}\selectfont}] at (4.571,0.215) {\large $x$};
\gpcolor{rgb color={0.502,0.502,0.502}}
\gpsetlinewidth{0.50}
\gpsetpointsize{2.67}
\gppoint{gp mark 7}{(1.206,4.163)}
\gppoint{gp mark 7}{(1.217,4.147)}
\gppoint{gp mark 7}{(1.228,4.131)}
\gppoint{gp mark 7}{(1.239,4.114)}
\gppoint{gp mark 7}{(1.250,4.098)}
\gppoint{gp mark 7}{(1.261,4.081)}
\gppoint{gp mark 7}{(1.272,4.065)}
\gppoint{gp mark 7}{(1.283,4.049)}
\gppoint{gp mark 7}{(1.294,4.032)}
\gppoint{gp mark 7}{(1.305,4.016)}
\gppoint{gp mark 7}{(1.316,4.000)}
\gppoint{gp mark 7}{(1.327,3.983)}
\gppoint{gp mark 7}{(1.338,3.967)}
\gppoint{gp mark 7}{(1.349,3.951)}
\gppoint{gp mark 7}{(1.360,3.935)}
\gppoint{gp mark 7}{(1.371,3.918)}
\gppoint{gp mark 7}{(1.382,3.902)}
\gppoint{gp mark 7}{(1.393,3.886)}
\gppoint{gp mark 7}{(1.404,3.870)}
\gppoint{gp mark 7}{(1.415,3.853)}
\gppoint{gp mark 7}{(1.426,3.837)}
\gppoint{gp mark 7}{(1.437,3.821)}
\gppoint{gp mark 7}{(1.448,3.805)}
\gppoint{gp mark 7}{(1.459,3.789)}
\gppoint{gp mark 7}{(1.470,3.773)}
\gppoint{gp mark 7}{(1.481,3.757)}
\gppoint{gp mark 7}{(1.492,3.740)}
\gppoint{gp mark 7}{(1.503,3.724)}
\gppoint{gp mark 7}{(1.514,3.708)}
\gppoint{gp mark 7}{(1.525,3.692)}
\gppoint{gp mark 7}{(1.536,3.676)}
\gppoint{gp mark 7}{(1.547,3.660)}
\gppoint{gp mark 7}{(1.558,3.644)}
\gppoint{gp mark 7}{(1.568,3.628)}
\gppoint{gp mark 7}{(1.579,3.612)}
\gppoint{gp mark 7}{(1.590,3.596)}
\gppoint{gp mark 7}{(1.601,3.580)}
\gppoint{gp mark 7}{(1.612,3.564)}
\gppoint{gp mark 7}{(1.623,3.548)}
\gppoint{gp mark 7}{(1.634,3.532)}
\gppoint{gp mark 7}{(1.645,3.516)}
\gppoint{gp mark 7}{(1.656,3.500)}
\gppoint{gp mark 7}{(1.667,3.484)}
\gppoint{gp mark 7}{(1.678,3.469)}
\gppoint{gp mark 7}{(1.689,3.453)}
\gppoint{gp mark 7}{(1.700,3.437)}
\gppoint{gp mark 7}{(1.711,3.421)}
\gppoint{gp mark 7}{(1.722,3.405)}
\gppoint{gp mark 7}{(1.733,3.389)}
\gppoint{gp mark 7}{(1.744,3.374)}
\gppoint{gp mark 7}{(1.755,3.358)}
\gppoint{gp mark 7}{(1.766,3.342)}
\gppoint{gp mark 7}{(1.777,3.326)}
\gppoint{gp mark 7}{(1.788,3.311)}
\gppoint{gp mark 7}{(1.799,3.295)}
\gppoint{gp mark 7}{(1.810,3.279)}
\gppoint{gp mark 7}{(1.821,3.263)}
\gppoint{gp mark 7}{(1.832,3.248)}
\gppoint{gp mark 7}{(1.843,3.232)}
\gppoint{gp mark 7}{(1.854,3.216)}
\gppoint{gp mark 7}{(1.865,3.201)}
\gppoint{gp mark 7}{(1.876,3.185)}
\gppoint{gp mark 7}{(1.887,3.169)}
\gppoint{gp mark 7}{(1.898,3.154)}
\gppoint{gp mark 7}{(1.909,3.138)}
\gppoint{gp mark 7}{(1.920,3.123)}
\gppoint{gp mark 7}{(1.931,3.107)}
\gppoint{gp mark 7}{(1.942,3.091)}
\gppoint{gp mark 7}{(1.953,3.076)}
\gppoint{gp mark 7}{(1.964,3.060)}
\gppoint{gp mark 7}{(1.975,3.045)}
\gppoint{gp mark 7}{(1.986,3.029)}
\gppoint{gp mark 7}{(1.997,3.014)}
\gppoint{gp mark 7}{(2.008,2.998)}
\gppoint{gp mark 7}{(2.019,2.983)}
\gppoint{gp mark 7}{(2.030,2.967)}
\gppoint{gp mark 7}{(2.041,2.952)}
\gppoint{gp mark 7}{(2.052,2.937)}
\gppoint{gp mark 7}{(2.063,2.921)}
\gppoint{gp mark 7}{(2.074,2.906)}
\gppoint{gp mark 7}{(2.085,2.890)}
\gppoint{gp mark 7}{(2.096,2.875)}
\gppoint{gp mark 7}{(2.107,2.860)}
\gppoint{gp mark 7}{(2.118,2.844)}
\gppoint{gp mark 7}{(2.129,2.829)}
\gppoint{gp mark 7}{(2.140,2.814)}
\gppoint{gp mark 7}{(2.151,2.798)}
\gppoint{gp mark 7}{(2.162,2.783)}
\gppoint{gp mark 7}{(2.173,2.768)}
\gppoint{gp mark 7}{(2.184,2.753)}
\gppoint{gp mark 7}{(2.195,2.737)}
\gppoint{gp mark 7}{(2.206,2.722)}
\gppoint{gp mark 7}{(2.217,2.707)}
\gppoint{gp mark 7}{(2.228,2.692)}
\gppoint{gp mark 7}{(2.239,2.676)}
\gppoint{gp mark 7}{(2.250,2.661)}
\gppoint{gp mark 7}{(2.261,2.646)}
\gppoint{gp mark 7}{(2.272,2.631)}
\gppoint{gp mark 7}{(2.283,2.616)}
\gppoint{gp mark 7}{(2.294,2.601)}
\gppoint{gp mark 7}{(2.305,2.586)}
\gppoint{gp mark 7}{(2.316,2.570)}
\gppoint{gp mark 7}{(2.327,2.555)}
\gppoint{gp mark 7}{(2.338,2.540)}
\gppoint{gp mark 7}{(2.349,2.525)}
\gppoint{gp mark 7}{(2.360,2.510)}
\gppoint{gp mark 7}{(2.371,2.495)}
\gppoint{gp mark 7}{(2.382,2.480)}
\gppoint{gp mark 7}{(2.393,2.465)}
\gppoint{gp mark 7}{(2.404,2.450)}
\gppoint{gp mark 7}{(2.415,2.435)}
\gppoint{gp mark 7}{(2.426,2.420)}
\gppoint{gp mark 7}{(2.437,2.405)}
\gppoint{gp mark 7}{(2.448,2.390)}
\gppoint{gp mark 7}{(2.459,2.375)}
\gppoint{gp mark 7}{(2.470,2.361)}
\gppoint{gp mark 7}{(2.480,2.346)}
\gppoint{gp mark 7}{(2.491,2.331)}
\gppoint{gp mark 7}{(2.502,2.316)}
\gppoint{gp mark 7}{(2.513,2.301)}
\gppoint{gp mark 7}{(2.524,2.286)}
\gppoint{gp mark 7}{(2.535,2.271)}
\gppoint{gp mark 7}{(2.546,2.257)}
\gppoint{gp mark 7}{(2.557,2.242)}
\gppoint{gp mark 7}{(2.568,2.227)}
\gppoint{gp mark 7}{(2.579,2.212)}
\gppoint{gp mark 7}{(2.590,2.198)}
\gppoint{gp mark 7}{(2.601,2.183)}
\gppoint{gp mark 7}{(2.612,2.168)}
\gppoint{gp mark 7}{(2.623,2.154)}
\gppoint{gp mark 7}{(2.634,2.139)}
\gppoint{gp mark 7}{(2.645,2.124)}
\gppoint{gp mark 7}{(2.656,2.110)}
\gppoint{gp mark 7}{(2.667,2.095)}
\gppoint{gp mark 7}{(2.678,2.080)}
\gppoint{gp mark 7}{(2.689,2.066)}
\gppoint{gp mark 7}{(2.700,2.051)}
\gppoint{gp mark 7}{(2.711,2.037)}
\gppoint{gp mark 7}{(2.722,2.022)}
\gppoint{gp mark 7}{(2.733,2.008)}
\gppoint{gp mark 7}{(2.744,1.993)}
\gppoint{gp mark 7}{(2.755,1.979)}
\gppoint{gp mark 7}{(2.766,1.964)}
\gppoint{gp mark 7}{(2.777,1.950)}
\gppoint{gp mark 7}{(2.788,1.935)}
\gppoint{gp mark 7}{(2.799,1.921)}
\gppoint{gp mark 7}{(2.810,1.907)}
\gppoint{gp mark 7}{(2.821,1.892)}
\gppoint{gp mark 7}{(2.832,1.878)}
\gppoint{gp mark 7}{(2.843,1.864)}
\gppoint{gp mark 7}{(2.854,1.850)}
\gppoint{gp mark 7}{(2.865,1.836)}
\gppoint{gp mark 7}{(2.876,1.822)}
\gppoint{gp mark 7}{(2.887,1.807)}
\gppoint{gp mark 7}{(2.898,1.793)}
\gppoint{gp mark 7}{(2.909,1.779)}
\gppoint{gp mark 7}{(2.920,1.766)}
\gppoint{gp mark 7}{(2.931,1.752)}
\gppoint{gp mark 7}{(2.942,1.738)}
\gppoint{gp mark 7}{(2.953,1.724)}
\gppoint{gp mark 7}{(2.964,1.710)}
\gppoint{gp mark 7}{(2.975,1.697)}
\gppoint{gp mark 7}{(2.986,1.683)}
\gppoint{gp mark 7}{(2.997,1.669)}
\gppoint{gp mark 7}{(3.008,1.655)}
\gppoint{gp mark 7}{(3.019,1.640)}
\gppoint{gp mark 7}{(3.030,1.626)}
\gppoint{gp mark 7}{(3.041,1.613)}
\gppoint{gp mark 7}{(3.052,1.605)}
\gppoint{gp mark 7}{(3.063,1.602)}
\gppoint{gp mark 7}{(3.074,1.601)}
\gppoint{gp mark 7}{(3.085,1.602)}
\gppoint{gp mark 7}{(3.096,1.606)}
\gppoint{gp mark 7}{(3.107,1.626)}
\gppoint{gp mark 7}{(3.118,1.664)}
\gppoint{gp mark 7}{(3.129,1.695)}
\gppoint{gp mark 7}{(3.140,1.703)}
\gppoint{gp mark 7}{(3.151,1.704)}
\gppoint{gp mark 7}{(3.162,1.704)}
\gppoint{gp mark 7}{(3.173,1.704)}
\gppoint{gp mark 7}{(3.184,1.704)}
\gppoint{gp mark 7}{(3.195,1.704)}
\gppoint{gp mark 7}{(3.206,1.703)}
\gppoint{gp mark 7}{(3.217,1.701)}
\gppoint{gp mark 7}{(3.228,1.699)}
\gppoint{gp mark 7}{(3.239,1.696)}
\gppoint{gp mark 7}{(3.250,1.692)}
\gppoint{gp mark 7}{(3.261,1.690)}
\gppoint{gp mark 7}{(3.272,1.687)}
\gppoint{gp mark 7}{(3.283,1.685)}
\gppoint{gp mark 7}{(3.294,1.684)}
\gppoint{gp mark 7}{(3.305,1.682)}
\gppoint{gp mark 7}{(3.316,1.681)}
\gppoint{gp mark 7}{(3.327,1.680)}
\gppoint{gp mark 7}{(3.338,1.680)}
\gppoint{gp mark 7}{(3.349,1.679)}
\gppoint{gp mark 7}{(3.360,1.679)}
\gppoint{gp mark 7}{(3.371,1.679)}
\gppoint{gp mark 7}{(3.382,1.679)}
\gppoint{gp mark 7}{(3.392,1.680)}
\gppoint{gp mark 7}{(3.403,1.680)}
\gppoint{gp mark 7}{(3.414,1.681)}
\gppoint{gp mark 7}{(3.425,1.682)}
\gppoint{gp mark 7}{(3.436,1.683)}
\gppoint{gp mark 7}{(3.447,1.684)}
\gppoint{gp mark 7}{(3.458,1.685)}
\gppoint{gp mark 7}{(3.469,1.685)}
\gppoint{gp mark 7}{(3.480,1.686)}
\gppoint{gp mark 7}{(3.491,1.687)}
\gppoint{gp mark 7}{(3.502,1.706)}
\gppoint{gp mark 7}{(3.513,2.151)}
\gppoint{gp mark 7}{(3.524,4.379)}
\gppoint{gp mark 7}{(3.535,4.982)}
\gppoint{gp mark 7}{(3.546,4.738)}
\gppoint{gp mark 7}{(3.557,4.642)}
\gppoint{gp mark 7}{(3.568,4.603)}
\gppoint{gp mark 7}{(3.579,4.567)}
\gppoint{gp mark 7}{(3.590,4.551)}
\gppoint{gp mark 7}{(3.601,4.544)}
\gppoint{gp mark 7}{(3.612,4.509)}
\gppoint{gp mark 7}{(3.623,4.469)}
\gppoint{gp mark 7}{(3.634,4.449)}
\gppoint{gp mark 7}{(3.645,4.430)}
\gppoint{gp mark 7}{(3.656,4.398)}
\gppoint{gp mark 7}{(3.667,4.361)}
\gppoint{gp mark 7}{(3.678,4.336)}
\gppoint{gp mark 7}{(3.689,4.317)}
\gppoint{gp mark 7}{(3.700,4.291)}
\gppoint{gp mark 7}{(3.711,4.259)}
\gppoint{gp mark 7}{(3.722,4.230)}
\gppoint{gp mark 7}{(3.733,4.205)}
\gppoint{gp mark 7}{(3.744,4.180)}
\gppoint{gp mark 7}{(3.755,4.152)}
\gppoint{gp mark 7}{(3.766,4.125)}
\gppoint{gp mark 7}{(3.777,4.098)}
\gppoint{gp mark 7}{(3.788,4.072)}
\gppoint{gp mark 7}{(3.799,4.045)}
\gppoint{gp mark 7}{(3.810,4.018)}
\gppoint{gp mark 7}{(3.821,3.990)}
\gppoint{gp mark 7}{(3.832,3.963)}
\gppoint{gp mark 7}{(3.843,3.937)}
\gppoint{gp mark 7}{(3.854,3.912)}
\gppoint{gp mark 7}{(3.865,3.885)}
\gppoint{gp mark 7}{(3.876,3.856)}
\gppoint{gp mark 7}{(3.887,3.830)}
\gppoint{gp mark 7}{(3.898,3.805)}
\gppoint{gp mark 7}{(3.909,3.780)}
\gppoint{gp mark 7}{(3.920,3.753)}
\gppoint{gp mark 7}{(3.931,3.724)}
\gppoint{gp mark 7}{(3.942,3.696)}
\gppoint{gp mark 7}{(3.953,3.671)}
\gppoint{gp mark 7}{(3.964,3.647)}
\gppoint{gp mark 7}{(3.975,3.621)}
\gppoint{gp mark 7}{(3.986,3.593)}
\gppoint{gp mark 7}{(3.997,3.567)}
\gppoint{gp mark 7}{(4.008,3.542)}
\gppoint{gp mark 7}{(4.019,3.518)}
\gppoint{gp mark 7}{(4.030,3.492)}
\gppoint{gp mark 7}{(4.041,3.465)}
\gppoint{gp mark 7}{(4.052,3.440)}
\gppoint{gp mark 7}{(4.063,3.417)}
\gppoint{gp mark 7}{(4.074,3.395)}
\gppoint{gp mark 7}{(4.085,3.376)}
\gppoint{gp mark 7}{(4.096,3.363)}
\gppoint{gp mark 7}{(4.107,3.357)}
\gppoint{gp mark 7}{(4.118,3.358)}
\gppoint{gp mark 7}{(4.129,3.358)}
\gppoint{gp mark 7}{(4.140,3.358)}
\gppoint{gp mark 7}{(4.151,3.359)}
\gppoint{gp mark 7}{(4.162,3.361)}
\gppoint{gp mark 7}{(4.173,3.365)}
\gppoint{gp mark 7}{(4.184,3.368)}
\gppoint{gp mark 7}{(4.195,3.369)}
\gppoint{gp mark 7}{(4.206,3.368)}
\gppoint{gp mark 7}{(4.217,3.368)}
\gppoint{gp mark 7}{(4.228,3.368)}
\gppoint{gp mark 7}{(4.239,3.369)}
\gppoint{gp mark 7}{(4.250,3.369)}
\gppoint{gp mark 7}{(4.261,3.369)}
\gppoint{gp mark 7}{(4.272,3.368)}
\gppoint{gp mark 7}{(4.283,3.368)}
\gppoint{gp mark 7}{(4.294,3.368)}
\gppoint{gp mark 7}{(4.304,3.369)}
\gppoint{gp mark 7}{(4.315,3.369)}
\gppoint{gp mark 7}{(4.326,3.368)}
\gppoint{gp mark 7}{(4.337,3.368)}
\gppoint{gp mark 7}{(4.348,3.368)}
\gppoint{gp mark 7}{(4.359,3.368)}
\gppoint{gp mark 7}{(4.370,3.368)}
\gppoint{gp mark 7}{(4.381,3.367)}
\gppoint{gp mark 7}{(4.392,3.367)}
\gppoint{gp mark 7}{(4.403,3.368)}
\gppoint{gp mark 7}{(4.414,3.369)}
\gppoint{gp mark 7}{(4.425,3.369)}
\gppoint{gp mark 7}{(4.436,3.368)}
\gppoint{gp mark 7}{(4.447,3.368)}
\gppoint{gp mark 7}{(4.458,3.368)}
\gppoint{gp mark 7}{(4.469,3.368)}
\gppoint{gp mark 7}{(4.480,3.368)}
\gppoint{gp mark 7}{(4.491,3.368)}
\gppoint{gp mark 7}{(4.502,3.367)}
\gppoint{gp mark 7}{(4.513,3.368)}
\gppoint{gp mark 7}{(4.524,3.368)}
\gppoint{gp mark 7}{(4.535,3.368)}
\gppoint{gp mark 7}{(4.546,3.369)}
\gppoint{gp mark 7}{(4.557,3.369)}
\gppoint{gp mark 7}{(4.568,3.368)}
\gppoint{gp mark 7}{(4.579,3.368)}
\gppoint{gp mark 7}{(4.590,3.367)}
\gppoint{gp mark 7}{(4.601,3.367)}
\gppoint{gp mark 7}{(4.612,3.367)}
\gppoint{gp mark 7}{(4.623,3.367)}
\gppoint{gp mark 7}{(4.634,3.368)}
\gppoint{gp mark 7}{(4.645,3.368)}
\gppoint{gp mark 7}{(4.656,3.368)}
\gppoint{gp mark 7}{(4.667,3.368)}
\gppoint{gp mark 7}{(4.678,3.368)}
\gppoint{gp mark 7}{(4.689,3.368)}
\gppoint{gp mark 7}{(4.700,3.368)}
\gppoint{gp mark 7}{(4.711,3.368)}
\gppoint{gp mark 7}{(4.722,3.367)}
\gppoint{gp mark 7}{(4.733,3.367)}
\gppoint{gp mark 7}{(4.744,3.368)}
\gppoint{gp mark 7}{(4.755,3.368)}
\gppoint{gp mark 7}{(4.766,3.368)}
\gppoint{gp mark 7}{(4.777,3.368)}
\gppoint{gp mark 7}{(4.788,3.368)}
\gppoint{gp mark 7}{(4.799,3.368)}
\gppoint{gp mark 7}{(4.810,3.368)}
\gppoint{gp mark 7}{(4.821,3.368)}
\gppoint{gp mark 7}{(4.832,3.368)}
\gppoint{gp mark 7}{(4.843,3.368)}
\gppoint{gp mark 7}{(4.854,3.368)}
\gppoint{gp mark 7}{(4.865,3.368)}
\gppoint{gp mark 7}{(4.876,3.368)}
\gppoint{gp mark 7}{(4.887,3.368)}
\gppoint{gp mark 7}{(4.898,3.368)}
\gppoint{gp mark 7}{(4.909,3.368)}
\gppoint{gp mark 7}{(4.920,3.368)}
\gppoint{gp mark 7}{(4.931,3.368)}
\gppoint{gp mark 7}{(4.942,3.368)}
\gppoint{gp mark 7}{(4.953,3.368)}
\gppoint{gp mark 7}{(4.964,3.369)}
\gppoint{gp mark 7}{(4.975,3.369)}
\gppoint{gp mark 7}{(4.986,3.368)}
\gppoint{gp mark 7}{(4.997,3.368)}
\gppoint{gp mark 7}{(5.008,3.367)}
\gppoint{gp mark 7}{(5.019,3.367)}
\gppoint{gp mark 7}{(5.030,3.367)}
\gppoint{gp mark 7}{(5.041,3.368)}
\gppoint{gp mark 7}{(5.052,3.368)}
\gppoint{gp mark 7}{(5.063,3.369)}
\gppoint{gp mark 7}{(5.074,3.369)}
\gppoint{gp mark 7}{(5.085,3.369)}
\gppoint{gp mark 7}{(5.096,3.368)}
\gppoint{gp mark 7}{(5.107,3.368)}
\gppoint{gp mark 7}{(5.118,3.368)}
\gppoint{gp mark 7}{(5.129,3.368)}
\gppoint{gp mark 7}{(5.140,3.368)}
\gppoint{gp mark 7}{(5.151,3.368)}
\gppoint{gp mark 7}{(5.162,3.368)}
\gppoint{gp mark 7}{(5.173,3.368)}
\gppoint{gp mark 7}{(5.184,3.369)}
\gppoint{gp mark 7}{(5.195,3.369)}
\gppoint{gp mark 7}{(5.206,3.369)}
\gppoint{gp mark 7}{(5.216,3.369)}
\gppoint{gp mark 7}{(5.227,3.368)}
\gppoint{gp mark 7}{(5.238,3.368)}
\gppoint{gp mark 7}{(5.249,3.368)}
\gppoint{gp mark 7}{(5.260,3.368)}
\gppoint{gp mark 7}{(5.271,3.368)}
\gppoint{gp mark 7}{(5.282,3.368)}
\gppoint{gp mark 7}{(5.293,3.368)}
\gppoint{gp mark 7}{(5.304,3.369)}
\gppoint{gp mark 7}{(5.315,3.369)}
\gppoint{gp mark 7}{(5.326,3.369)}
\gppoint{gp mark 7}{(5.337,3.369)}
\gppoint{gp mark 7}{(5.348,3.369)}
\gppoint{gp mark 7}{(5.359,3.368)}
\gppoint{gp mark 7}{(5.370,3.368)}
\gppoint{gp mark 7}{(5.381,3.368)}
\gppoint{gp mark 7}{(5.392,3.369)}
\gppoint{gp mark 7}{(5.403,3.369)}
\gppoint{gp mark 7}{(5.414,3.369)}
\gppoint{gp mark 7}{(5.425,3.369)}
\gppoint{gp mark 7}{(5.436,3.369)}
\gppoint{gp mark 7}{(5.447,3.369)}
\gppoint{gp mark 7}{(5.458,3.369)}
\gppoint{gp mark 7}{(5.469,3.369)}
\gppoint{gp mark 7}{(5.480,3.369)}
\gppoint{gp mark 7}{(5.491,3.369)}
\gppoint{gp mark 7}{(5.502,3.369)}
\gppoint{gp mark 7}{(5.513,3.369)}
\gppoint{gp mark 7}{(5.524,3.369)}
\gppoint{gp mark 7}{(5.535,3.369)}
\gppoint{gp mark 7}{(5.546,3.368)}
\gppoint{gp mark 7}{(5.557,3.368)}
\gppoint{gp mark 7}{(5.568,3.369)}
\gppoint{gp mark 7}{(5.579,3.369)}
\gppoint{gp mark 7}{(5.590,3.369)}
\gppoint{gp mark 7}{(5.601,3.369)}
\gppoint{gp mark 7}{(5.612,3.369)}
\gppoint{gp mark 7}{(5.623,3.369)}
\gppoint{gp mark 7}{(5.634,3.369)}
\gppoint{gp mark 7}{(5.645,3.369)}
\gppoint{gp mark 7}{(5.656,3.368)}
\gppoint{gp mark 7}{(5.667,3.368)}
\gppoint{gp mark 7}{(5.678,3.368)}
\gppoint{gp mark 7}{(5.689,3.369)}
\gppoint{gp mark 7}{(5.700,3.369)}
\gppoint{gp mark 7}{(5.711,3.369)}
\gppoint{gp mark 7}{(5.722,3.369)}
\gppoint{gp mark 7}{(5.733,3.369)}
\gppoint{gp mark 7}{(5.744,3.369)}
\gppoint{gp mark 7}{(5.755,3.369)}
\gppoint{gp mark 7}{(5.766,3.369)}
\gppoint{gp mark 7}{(5.777,3.369)}
\gppoint{gp mark 7}{(5.788,3.369)}
\gppoint{gp mark 7}{(5.799,3.369)}
\gppoint{gp mark 7}{(5.810,3.369)}
\gppoint{gp mark 7}{(5.821,3.369)}
\gppoint{gp mark 7}{(5.832,3.369)}
\gppoint{gp mark 7}{(5.843,3.369)}
\gppoint{gp mark 7}{(5.854,3.370)}
\gppoint{gp mark 7}{(5.865,3.369)}
\gppoint{gp mark 7}{(5.876,3.369)}
\gppoint{gp mark 7}{(5.887,3.369)}
\gppoint{gp mark 7}{(5.898,3.369)}
\gppoint{gp mark 7}{(5.909,3.369)}
\gppoint{gp mark 7}{(5.920,3.369)}
\gppoint{gp mark 7}{(5.931,3.369)}
\gppoint{gp mark 7}{(5.942,3.369)}
\gppoint{gp mark 7}{(5.953,3.369)}
\gppoint{gp mark 7}{(5.964,3.369)}
\gppoint{gp mark 7}{(5.975,3.369)}
\gppoint{gp mark 7}{(5.986,3.369)}
\gppoint{gp mark 7}{(5.997,3.369)}
\gppoint{gp mark 7}{(6.008,3.369)}
\gppoint{gp mark 7}{(6.019,3.369)}
\gppoint{gp mark 7}{(6.030,3.369)}
\gppoint{gp mark 7}{(6.041,3.369)}
\gppoint{gp mark 7}{(6.052,3.369)}
\gppoint{gp mark 7}{(6.063,3.369)}
\gppoint{gp mark 7}{(6.074,3.369)}
\gppoint{gp mark 7}{(6.085,3.369)}
\gppoint{gp mark 7}{(6.096,3.369)}
\gppoint{gp mark 7}{(6.107,3.369)}
\gppoint{gp mark 7}{(6.118,3.369)}
\gppoint{gp mark 7}{(6.128,3.369)}
\gppoint{gp mark 7}{(6.139,3.369)}
\gppoint{gp mark 7}{(6.150,3.369)}
\gppoint{gp mark 7}{(6.161,3.369)}
\gppoint{gp mark 7}{(6.172,3.369)}
\gppoint{gp mark 7}{(6.183,3.369)}
\gppoint{gp mark 7}{(6.194,3.369)}
\gppoint{gp mark 7}{(6.205,3.369)}
\gppoint{gp mark 7}{(6.216,3.369)}
\gppoint{gp mark 7}{(6.227,3.369)}
\gppoint{gp mark 7}{(6.238,3.369)}
\gppoint{gp mark 7}{(6.249,3.369)}
\gppoint{gp mark 7}{(6.260,3.369)}
\gppoint{gp mark 7}{(6.271,3.369)}
\gppoint{gp mark 7}{(6.282,3.369)}
\gppoint{gp mark 7}{(6.293,3.369)}
\gppoint{gp mark 7}{(6.304,3.369)}
\gppoint{gp mark 7}{(6.315,3.369)}
\gppoint{gp mark 7}{(6.326,3.369)}
\gppoint{gp mark 7}{(6.337,3.369)}
\gppoint{gp mark 7}{(6.348,3.369)}
\gppoint{gp mark 7}{(6.359,3.369)}
\gppoint{gp mark 7}{(6.370,3.369)}
\gppoint{gp mark 7}{(6.381,3.369)}
\gppoint{gp mark 7}{(6.392,3.369)}
\gppoint{gp mark 7}{(6.403,3.369)}
\gppoint{gp mark 7}{(6.414,3.368)}
\gppoint{gp mark 7}{(6.425,3.368)}
\gppoint{gp mark 7}{(6.436,3.368)}
\gppoint{gp mark 7}{(6.447,3.368)}
\gppoint{gp mark 7}{(6.458,3.368)}
\gppoint{gp mark 7}{(6.469,3.368)}
\gppoint{gp mark 7}{(6.480,3.368)}
\gppoint{gp mark 7}{(6.491,3.368)}
\gppoint{gp mark 7}{(6.502,3.368)}
\gppoint{gp mark 7}{(6.513,3.368)}
\gppoint{gp mark 7}{(6.524,3.368)}
\gppoint{gp mark 7}{(6.535,3.368)}
\gppoint{gp mark 7}{(6.546,3.367)}
\gppoint{gp mark 7}{(6.557,3.367)}
\gppoint{gp mark 7}{(6.568,3.367)}
\gppoint{gp mark 7}{(6.579,3.367)}
\gppoint{gp mark 7}{(6.590,3.367)}
\gppoint{gp mark 7}{(6.601,3.367)}
\gppoint{gp mark 7}{(6.612,3.366)}
\gppoint{gp mark 7}{(6.623,3.366)}
\gppoint{gp mark 7}{(6.634,3.366)}
\gppoint{gp mark 7}{(6.645,3.366)}
\gppoint{gp mark 7}{(6.656,3.366)}
\gppoint{gp mark 7}{(6.667,3.366)}
\gppoint{gp mark 7}{(6.678,3.366)}
\gppoint{gp mark 7}{(6.689,3.366)}
\gppoint{gp mark 7}{(6.700,3.365)}
\gppoint{gp mark 7}{(6.711,3.365)}
\gppoint{gp mark 7}{(6.722,3.365)}
\gppoint{gp mark 7}{(6.733,3.365)}
\gppoint{gp mark 7}{(6.744,3.365)}
\gppoint{gp mark 7}{(6.755,3.364)}
\gppoint{gp mark 7}{(6.766,3.364)}
\gppoint{gp mark 7}{(6.777,3.364)}
\gppoint{gp mark 7}{(6.788,3.364)}
\gppoint{gp mark 7}{(6.799,3.364)}
\gppoint{gp mark 7}{(6.810,3.364)}
\gppoint{gp mark 7}{(6.821,3.364)}
\gppoint{gp mark 7}{(6.832,3.364)}
\gppoint{gp mark 7}{(6.843,3.363)}
\gppoint{gp mark 7}{(6.854,3.363)}
\gppoint{gp mark 7}{(6.865,3.363)}
\gppoint{gp mark 7}{(6.876,3.363)}
\gppoint{gp mark 7}{(6.887,3.363)}
\gppoint{gp mark 7}{(6.898,3.363)}
\gppoint{gp mark 7}{(6.909,3.363)}
\gppoint{gp mark 7}{(6.920,3.362)}
\gppoint{gp mark 7}{(6.931,3.362)}
\gppoint{gp mark 7}{(6.942,3.362)}
\gppoint{gp mark 7}{(6.953,3.361)}
\gppoint{gp mark 7}{(6.964,3.361)}
\gppoint{gp mark 7}{(6.975,3.361)}
\gppoint{gp mark 7}{(6.986,3.361)}
\gppoint{gp mark 7}{(6.997,3.361)}
\gppoint{gp mark 7}{(7.008,3.361)}
\gppoint{gp mark 7}{(7.019,3.361)}
\gppoint{gp mark 7}{(7.030,3.360)}
\gppoint{gp mark 7}{(7.040,3.360)}
\gppoint{gp mark 7}{(7.051,3.360)}
\gppoint{gp mark 7}{(7.062,3.360)}
\gppoint{gp mark 7}{(7.073,3.360)}
\gppoint{gp mark 7}{(7.084,3.359)}
\gppoint{gp mark 7}{(7.095,3.359)}
\gppoint{gp mark 7}{(7.106,3.359)}
\gppoint{gp mark 7}{(7.117,3.358)}
\gppoint{gp mark 7}{(7.128,3.358)}
\gppoint{gp mark 7}{(7.139,3.357)}
\gppoint{gp mark 7}{(7.150,3.357)}
\gppoint{gp mark 7}{(7.161,3.356)}
\gppoint{gp mark 7}{(7.172,3.355)}
\gppoint{gp mark 7}{(7.183,3.354)}
\gppoint{gp mark 7}{(7.194,3.354)}
\gppoint{gp mark 7}{(7.205,3.353)}
\gppoint{gp mark 7}{(7.216,3.353)}
\gppoint{gp mark 7}{(7.227,3.352)}
\gppoint{gp mark 7}{(7.238,3.352)}
\gppoint{gp mark 7}{(7.249,3.352)}
\gppoint{gp mark 7}{(7.260,3.352)}
\gppoint{gp mark 7}{(7.271,3.352)}
\gppoint{gp mark 7}{(7.282,3.352)}
\gppoint{gp mark 7}{(7.293,3.352)}
\gppoint{gp mark 7}{(7.304,3.353)}
\gppoint{gp mark 7}{(7.315,3.353)}
\gppoint{gp mark 7}{(7.326,3.354)}
\gppoint{gp mark 7}{(7.337,3.354)}
\gppoint{gp mark 7}{(7.348,3.356)}
\gppoint{gp mark 7}{(7.359,3.356)}
\gppoint{gp mark 7}{(7.370,3.356)}
\gppoint{gp mark 7}{(7.381,3.357)}
\gppoint{gp mark 7}{(7.392,3.358)}
\gppoint{gp mark 7}{(7.403,3.365)}
\gppoint{gp mark 7}{(7.414,3.379)}
\gppoint{gp mark 7}{(7.425,3.393)}
\gppoint{gp mark 7}{(7.436,3.399)}
\gppoint{gp mark 7}{(7.447,3.400)}
\gppoint{gp mark 7}{(7.458,3.401)}
\gppoint{gp mark 7}{(7.469,3.400)}
\gppoint{gp mark 7}{(7.480,3.389)}
\gppoint{gp mark 7}{(7.491,3.301)}
\gppoint{gp mark 7}{(7.502,2.807)}
\gppoint{gp mark 7}{(7.513,1.676)}
\gpcolor{rgb color={1.000,0.000,0.000}}
\gpsetpointsize{4.44}
\gppoint{gp mark 7}{(1.206,4.163)}
\gppoint{gp mark 7}{(1.217,4.146)}
\gppoint{gp mark 7}{(1.228,4.130)}
\gppoint{gp mark 7}{(1.239,4.114)}
\gppoint{gp mark 7}{(1.250,4.097)}
\gppoint{gp mark 7}{(1.261,4.081)}
\gppoint{gp mark 7}{(1.272,4.065)}
\gppoint{gp mark 7}{(1.283,4.048)}
\gppoint{gp mark 7}{(1.294,4.032)}
\gppoint{gp mark 7}{(1.305,4.016)}
\gppoint{gp mark 7}{(1.316,4.000)}
\gppoint{gp mark 7}{(1.327,3.983)}
\gppoint{gp mark 7}{(1.338,3.967)}
\gppoint{gp mark 7}{(1.349,3.951)}
\gppoint{gp mark 7}{(1.360,3.935)}
\gppoint{gp mark 7}{(1.371,3.918)}
\gppoint{gp mark 7}{(1.382,3.902)}
\gppoint{gp mark 7}{(1.393,3.886)}
\gppoint{gp mark 7}{(1.404,3.870)}
\gppoint{gp mark 7}{(1.415,3.854)}
\gppoint{gp mark 7}{(1.426,3.838)}
\gppoint{gp mark 7}{(1.437,3.822)}
\gppoint{gp mark 7}{(1.448,3.806)}
\gppoint{gp mark 7}{(1.459,3.789)}
\gppoint{gp mark 7}{(1.470,3.773)}
\gppoint{gp mark 7}{(1.481,3.757)}
\gppoint{gp mark 7}{(1.492,3.741)}
\gppoint{gp mark 7}{(1.503,3.725)}
\gppoint{gp mark 7}{(1.514,3.709)}
\gppoint{gp mark 7}{(1.525,3.693)}
\gppoint{gp mark 7}{(1.536,3.677)}
\gppoint{gp mark 7}{(1.547,3.661)}
\gppoint{gp mark 7}{(1.558,3.645)}
\gppoint{gp mark 7}{(1.568,3.630)}
\gppoint{gp mark 7}{(1.579,3.614)}
\gppoint{gp mark 7}{(1.590,3.598)}
\gppoint{gp mark 7}{(1.601,3.582)}
\gppoint{gp mark 7}{(1.612,3.566)}
\gppoint{gp mark 7}{(1.623,3.550)}
\gppoint{gp mark 7}{(1.634,3.534)}
\gppoint{gp mark 7}{(1.645,3.518)}
\gppoint{gp mark 7}{(1.656,3.503)}
\gppoint{gp mark 7}{(1.667,3.487)}
\gppoint{gp mark 7}{(1.678,3.471)}
\gppoint{gp mark 7}{(1.689,3.455)}
\gppoint{gp mark 7}{(1.700,3.440)}
\gppoint{gp mark 7}{(1.711,3.424)}
\gppoint{gp mark 7}{(1.722,3.408)}
\gppoint{gp mark 7}{(1.733,3.393)}
\gppoint{gp mark 7}{(1.744,3.377)}
\gppoint{gp mark 7}{(1.755,3.361)}
\gppoint{gp mark 7}{(1.766,3.346)}
\gppoint{gp mark 7}{(1.777,3.330)}
\gppoint{gp mark 7}{(1.788,3.315)}
\gppoint{gp mark 7}{(1.799,3.299)}
\gppoint{gp mark 7}{(1.810,3.283)}
\gppoint{gp mark 7}{(1.821,3.268)}
\gppoint{gp mark 7}{(1.832,3.252)}
\gppoint{gp mark 7}{(1.843,3.237)}
\gppoint{gp mark 7}{(1.854,3.222)}
\gppoint{gp mark 7}{(1.865,3.206)}
\gppoint{gp mark 7}{(1.876,3.191)}
\gppoint{gp mark 7}{(1.887,3.175)}
\gppoint{gp mark 7}{(1.898,3.160)}
\gppoint{gp mark 7}{(1.909,3.145)}
\gppoint{gp mark 7}{(1.920,3.129)}
\gppoint{gp mark 7}{(1.931,3.114)}
\gppoint{gp mark 7}{(1.942,3.099)}
\gppoint{gp mark 7}{(1.953,3.084)}
\gppoint{gp mark 7}{(1.964,3.069)}
\gppoint{gp mark 7}{(1.975,3.053)}
\gppoint{gp mark 7}{(1.986,3.038)}
\gppoint{gp mark 7}{(1.997,3.023)}
\gppoint{gp mark 7}{(2.008,3.008)}
\gppoint{gp mark 7}{(2.019,2.993)}
\gppoint{gp mark 7}{(2.030,2.983)}
\gppoint{gp mark 7}{(2.041,2.975)}
\gppoint{gp mark 7}{(2.052,2.964)}
\gppoint{gp mark 7}{(2.063,2.948)}
\gppoint{gp mark 7}{(2.074,2.933)}
\gppoint{gp mark 7}{(2.085,2.918)}
\gppoint{gp mark 7}{(2.096,2.904)}
\gppoint{gp mark 7}{(2.107,2.889)}
\gppoint{gp mark 7}{(2.118,2.874)}
\gppoint{gp mark 7}{(2.129,2.860)}
\gppoint{gp mark 7}{(2.140,2.845)}
\gppoint{gp mark 7}{(2.151,2.831)}
\gppoint{gp mark 7}{(2.162,2.816)}
\gppoint{gp mark 7}{(2.173,2.802)}
\gppoint{gp mark 7}{(2.184,2.787)}
\gppoint{gp mark 7}{(2.195,2.773)}
\gppoint{gp mark 7}{(2.206,2.759)}
\gppoint{gp mark 7}{(2.217,2.745)}
\gppoint{gp mark 7}{(2.228,2.731)}
\gppoint{gp mark 7}{(2.239,2.717)}
\gppoint{gp mark 7}{(2.250,2.703)}
\gppoint{gp mark 7}{(2.261,2.689)}
\gppoint{gp mark 7}{(2.272,2.675)}
\gppoint{gp mark 7}{(2.283,2.661)}
\gppoint{gp mark 7}{(2.294,2.648)}
\gppoint{gp mark 7}{(2.305,2.634)}
\gppoint{gp mark 7}{(2.316,2.621)}
\gppoint{gp mark 7}{(2.327,2.607)}
\gppoint{gp mark 7}{(2.338,2.594)}
\gppoint{gp mark 7}{(2.349,2.581)}
\gppoint{gp mark 7}{(2.360,2.568)}
\gppoint{gp mark 7}{(2.371,2.555)}
\gppoint{gp mark 7}{(2.382,2.542)}
\gppoint{gp mark 7}{(2.393,2.529)}
\gppoint{gp mark 7}{(2.404,2.516)}
\gppoint{gp mark 7}{(2.415,2.504)}
\gppoint{gp mark 7}{(2.426,2.492)}
\gppoint{gp mark 7}{(2.437,2.479)}
\gppoint{gp mark 7}{(2.448,2.468)}
\gppoint{gp mark 7}{(2.459,2.456)}
\gppoint{gp mark 7}{(2.470,2.445)}
\gppoint{gp mark 7}{(2.480,2.434)}
\gppoint{gp mark 7}{(2.491,2.425)}
\gppoint{gp mark 7}{(2.502,2.418)}
\gppoint{gp mark 7}{(2.513,2.413)}
\gppoint{gp mark 7}{(2.524,2.411)}
\gppoint{gp mark 7}{(2.535,2.411)}
\gppoint{gp mark 7}{(2.546,2.411)}
\gppoint{gp mark 7}{(2.557,2.411)}
\gppoint{gp mark 7}{(2.568,2.411)}
\gppoint{gp mark 7}{(2.579,2.412)}
\gppoint{gp mark 7}{(2.590,2.414)}
\gppoint{gp mark 7}{(2.601,2.417)}
\gppoint{gp mark 7}{(2.612,2.418)}
\gppoint{gp mark 7}{(2.623,2.418)}
\gppoint{gp mark 7}{(2.634,2.418)}
\gppoint{gp mark 7}{(2.645,2.418)}
\gppoint{gp mark 7}{(2.656,2.418)}
\gppoint{gp mark 7}{(2.667,2.418)}
\gppoint{gp mark 7}{(2.678,2.418)}
\gppoint{gp mark 7}{(2.689,2.418)}
\gppoint{gp mark 7}{(2.700,2.417)}
\gppoint{gp mark 7}{(2.711,2.417)}
\gppoint{gp mark 7}{(2.722,2.417)}
\gppoint{gp mark 7}{(2.733,2.417)}
\gppoint{gp mark 7}{(2.744,2.417)}
\gppoint{gp mark 7}{(2.755,2.417)}
\gppoint{gp mark 7}{(2.766,2.417)}
\gppoint{gp mark 7}{(2.777,2.417)}
\gppoint{gp mark 7}{(2.788,2.417)}
\gppoint{gp mark 7}{(2.799,2.418)}
\gppoint{gp mark 7}{(2.810,2.418)}
\gppoint{gp mark 7}{(2.821,2.418)}
\gppoint{gp mark 7}{(2.832,2.418)}
\gppoint{gp mark 7}{(2.843,2.417)}
\gppoint{gp mark 7}{(2.854,2.417)}
\gppoint{gp mark 7}{(2.865,2.417)}
\gppoint{gp mark 7}{(2.876,2.417)}
\gppoint{gp mark 7}{(2.887,2.417)}
\gppoint{gp mark 7}{(2.898,2.417)}
\gppoint{gp mark 7}{(2.909,2.417)}
\gppoint{gp mark 7}{(2.920,2.417)}
\gppoint{gp mark 7}{(2.931,2.417)}
\gppoint{gp mark 7}{(2.942,2.417)}
\gppoint{gp mark 7}{(2.953,2.417)}
\gppoint{gp mark 7}{(2.964,2.417)}
\gppoint{gp mark 7}{(2.975,2.417)}
\gppoint{gp mark 7}{(2.986,2.417)}
\gppoint{gp mark 7}{(2.997,2.417)}
\gppoint{gp mark 7}{(3.008,2.417)}
\gppoint{gp mark 7}{(3.019,2.417)}
\gppoint{gp mark 7}{(3.030,2.417)}
\gppoint{gp mark 7}{(3.041,2.417)}
\gppoint{gp mark 7}{(3.052,2.417)}
\gppoint{gp mark 7}{(3.063,2.417)}
\gppoint{gp mark 7}{(3.074,2.417)}
\gppoint{gp mark 7}{(3.085,2.417)}
\gppoint{gp mark 7}{(3.096,2.417)}
\gppoint{gp mark 7}{(3.107,2.417)}
\gppoint{gp mark 7}{(3.118,2.417)}
\gppoint{gp mark 7}{(3.129,2.417)}
\gppoint{gp mark 7}{(3.140,2.417)}
\gppoint{gp mark 7}{(3.151,2.417)}
\gppoint{gp mark 7}{(3.162,2.417)}
\gppoint{gp mark 7}{(3.173,2.417)}
\gppoint{gp mark 7}{(3.184,2.417)}
\gppoint{gp mark 7}{(3.195,2.417)}
\gppoint{gp mark 7}{(3.206,2.417)}
\gppoint{gp mark 7}{(3.217,2.417)}
\gppoint{gp mark 7}{(3.228,2.417)}
\gppoint{gp mark 7}{(3.239,2.417)}
\gppoint{gp mark 7}{(3.250,2.417)}
\gppoint{gp mark 7}{(3.261,2.417)}
\gppoint{gp mark 7}{(3.272,2.417)}
\gppoint{gp mark 7}{(3.283,2.417)}
\gppoint{gp mark 7}{(3.294,2.417)}
\gppoint{gp mark 7}{(3.305,2.417)}
\gppoint{gp mark 7}{(3.316,2.417)}
\gppoint{gp mark 7}{(3.327,2.417)}
\gppoint{gp mark 7}{(3.338,2.417)}
\gppoint{gp mark 7}{(3.349,2.417)}
\gppoint{gp mark 7}{(3.360,2.417)}
\gppoint{gp mark 7}{(3.371,2.417)}
\gppoint{gp mark 7}{(3.382,2.417)}
\gppoint{gp mark 7}{(3.392,2.417)}
\gppoint{gp mark 7}{(3.403,2.417)}
\gppoint{gp mark 7}{(3.414,2.417)}
\gppoint{gp mark 7}{(3.425,2.417)}
\gppoint{gp mark 7}{(3.436,2.417)}
\gppoint{gp mark 7}{(3.447,2.417)}
\gppoint{gp mark 7}{(3.458,2.417)}
\gppoint{gp mark 7}{(3.469,2.417)}
\gppoint{gp mark 7}{(3.480,2.418)}
\gppoint{gp mark 7}{(3.491,2.418)}
\gppoint{gp mark 7}{(3.502,2.425)}
\gppoint{gp mark 7}{(3.513,2.439)}
\gppoint{gp mark 7}{(3.524,2.434)}
\gppoint{gp mark 7}{(3.535,2.427)}
\gppoint{gp mark 7}{(3.546,2.424)}
\gppoint{gp mark 7}{(3.557,2.424)}
\gppoint{gp mark 7}{(3.568,2.436)}
\gppoint{gp mark 7}{(3.579,2.466)}
\gppoint{gp mark 7}{(3.590,2.440)}
\gppoint{gp mark 7}{(3.601,2.432)}
\gppoint{gp mark 7}{(3.612,2.427)}
\gppoint{gp mark 7}{(3.623,2.427)}
\gppoint{gp mark 7}{(3.634,2.433)}
\gppoint{gp mark 7}{(3.645,2.438)}
\gppoint{gp mark 7}{(3.656,2.437)}
\gppoint{gp mark 7}{(3.667,2.435)}
\gppoint{gp mark 7}{(3.678,2.436)}
\gppoint{gp mark 7}{(3.689,2.438)}
\gppoint{gp mark 7}{(3.700,2.437)}
\gppoint{gp mark 7}{(3.711,2.435)}
\gppoint{gp mark 7}{(3.722,2.435)}
\gppoint{gp mark 7}{(3.733,2.436)}
\gppoint{gp mark 7}{(3.744,2.437)}
\gppoint{gp mark 7}{(3.755,2.438)}
\gppoint{gp mark 7}{(3.766,2.440)}
\gppoint{gp mark 7}{(3.777,2.441)}
\gppoint{gp mark 7}{(3.788,2.442)}
\gppoint{gp mark 7}{(3.799,2.441)}
\gppoint{gp mark 7}{(3.810,2.438)}
\gppoint{gp mark 7}{(3.821,2.435)}
\gppoint{gp mark 7}{(3.832,2.434)}
\gppoint{gp mark 7}{(3.843,2.434)}
\gppoint{gp mark 7}{(3.854,2.434)}
\gppoint{gp mark 7}{(3.865,2.434)}
\gppoint{gp mark 7}{(3.876,2.434)}
\gppoint{gp mark 7}{(3.887,2.435)}
\gppoint{gp mark 7}{(3.898,2.435)}
\gppoint{gp mark 7}{(3.909,2.435)}
\gppoint{gp mark 7}{(3.920,2.436)}
\gppoint{gp mark 7}{(3.931,2.436)}
\gppoint{gp mark 7}{(3.942,2.437)}
\gppoint{gp mark 7}{(3.953,2.438)}
\gppoint{gp mark 7}{(3.964,2.440)}
\gppoint{gp mark 7}{(3.975,2.440)}
\gppoint{gp mark 7}{(3.986,2.440)}
\gppoint{gp mark 7}{(3.997,2.439)}
\gppoint{gp mark 7}{(4.008,2.436)}
\gppoint{gp mark 7}{(4.019,2.434)}
\gppoint{gp mark 7}{(4.030,2.434)}
\gppoint{gp mark 7}{(4.041,2.434)}
\gppoint{gp mark 7}{(4.052,2.434)}
\gppoint{gp mark 7}{(4.063,2.433)}
\gppoint{gp mark 7}{(4.074,2.433)}
\gppoint{gp mark 7}{(4.085,2.433)}
\gppoint{gp mark 7}{(4.096,2.433)}
\gppoint{gp mark 7}{(4.107,2.434)}
\gppoint{gp mark 7}{(4.118,2.435)}
\gppoint{gp mark 7}{(4.129,2.436)}
\gppoint{gp mark 7}{(4.140,2.437)}
\gppoint{gp mark 7}{(4.151,2.437)}
\gppoint{gp mark 7}{(4.162,2.438)}
\gppoint{gp mark 7}{(4.173,2.439)}
\gppoint{gp mark 7}{(4.184,2.441)}
\gppoint{gp mark 7}{(4.195,2.442)}
\gppoint{gp mark 7}{(4.206,2.442)}
\gppoint{gp mark 7}{(4.217,2.441)}
\gppoint{gp mark 7}{(4.228,2.438)}
\gppoint{gp mark 7}{(4.239,2.435)}
\gppoint{gp mark 7}{(4.250,2.432)}
\gppoint{gp mark 7}{(4.261,2.431)}
\gppoint{gp mark 7}{(4.272,2.431)}
\gppoint{gp mark 7}{(4.283,2.434)}
\gppoint{gp mark 7}{(4.294,2.445)}
\gppoint{gp mark 7}{(4.304,2.514)}
\gppoint{gp mark 7}{(4.315,2.862)}
\gppoint{gp mark 7}{(4.326,3.442)}
\gppoint{gp mark 7}{(4.337,3.585)}
\gppoint{gp mark 7}{(4.348,3.601)}
\gppoint{gp mark 7}{(4.359,3.604)}
\gppoint{gp mark 7}{(4.370,3.604)}
\gppoint{gp mark 7}{(4.381,3.607)}
\gppoint{gp mark 7}{(4.392,3.606)}
\gppoint{gp mark 7}{(4.403,3.605)}
\gppoint{gp mark 7}{(4.414,3.601)}
\gppoint{gp mark 7}{(4.425,3.601)}
\gppoint{gp mark 7}{(4.436,3.601)}
\gppoint{gp mark 7}{(4.447,3.600)}
\gppoint{gp mark 7}{(4.458,3.599)}
\gppoint{gp mark 7}{(4.469,3.598)}
\gppoint{gp mark 7}{(4.480,3.599)}
\gppoint{gp mark 7}{(4.491,3.600)}
\gppoint{gp mark 7}{(4.502,3.601)}
\gppoint{gp mark 7}{(4.513,3.601)}
\gppoint{gp mark 7}{(4.524,3.601)}
\gppoint{gp mark 7}{(4.535,3.602)}
\gppoint{gp mark 7}{(4.546,3.604)}
\gppoint{gp mark 7}{(4.557,3.605)}
\gppoint{gp mark 7}{(4.568,3.606)}
\gppoint{gp mark 7}{(4.579,3.606)}
\gppoint{gp mark 7}{(4.590,3.604)}
\gppoint{gp mark 7}{(4.601,3.601)}
\gppoint{gp mark 7}{(4.612,3.599)}
\gppoint{gp mark 7}{(4.623,3.599)}
\gppoint{gp mark 7}{(4.634,3.599)}
\gppoint{gp mark 7}{(4.645,3.599)}
\gppoint{gp mark 7}{(4.656,3.599)}
\gppoint{gp mark 7}{(4.667,3.599)}
\gppoint{gp mark 7}{(4.678,3.599)}
\gppoint{gp mark 7}{(4.689,3.600)}
\gppoint{gp mark 7}{(4.700,3.600)}
\gppoint{gp mark 7}{(4.711,3.600)}
\gppoint{gp mark 7}{(4.722,3.601)}
\gppoint{gp mark 7}{(4.733,3.602)}
\gppoint{gp mark 7}{(4.744,3.604)}
\gppoint{gp mark 7}{(4.755,3.605)}
\gppoint{gp mark 7}{(4.766,3.605)}
\gppoint{gp mark 7}{(4.777,3.605)}
\gppoint{gp mark 7}{(4.788,3.604)}
\gppoint{gp mark 7}{(4.799,3.601)}
\gppoint{gp mark 7}{(4.810,3.600)}
\gppoint{gp mark 7}{(4.821,3.600)}
\gppoint{gp mark 7}{(4.832,3.600)}
\gppoint{gp mark 7}{(4.843,3.600)}
\gppoint{gp mark 7}{(4.854,3.600)}
\gppoint{gp mark 7}{(4.865,3.601)}
\gppoint{gp mark 7}{(4.876,3.601)}
\gppoint{gp mark 7}{(4.887,3.601)}
\gppoint{gp mark 7}{(4.898,3.601)}
\gppoint{gp mark 7}{(4.909,3.603)}
\gppoint{gp mark 7}{(4.920,3.604)}
\gppoint{gp mark 7}{(4.931,3.604)}
\gppoint{gp mark 7}{(4.942,3.604)}
\gppoint{gp mark 7}{(4.953,3.604)}
\gppoint{gp mark 7}{(4.964,3.603)}
\gppoint{gp mark 7}{(4.975,3.600)}
\gppoint{gp mark 7}{(4.986,3.598)}
\gppoint{gp mark 7}{(4.997,3.598)}
\gppoint{gp mark 7}{(5.008,3.598)}
\gppoint{gp mark 7}{(5.019,3.598)}
\gppoint{gp mark 7}{(5.030,3.599)}
\gppoint{gp mark 7}{(5.041,3.599)}
\gppoint{gp mark 7}{(5.052,3.600)}
\gppoint{gp mark 7}{(5.063,3.600)}
\gppoint{gp mark 7}{(5.074,3.601)}
\gppoint{gp mark 7}{(5.085,3.603)}
\gppoint{gp mark 7}{(5.096,3.604)}
\gppoint{gp mark 7}{(5.107,3.605)}
\gppoint{gp mark 7}{(5.118,3.605)}
\gppoint{gp mark 7}{(5.129,3.605)}
\gppoint{gp mark 7}{(5.140,3.604)}
\gppoint{gp mark 7}{(5.151,3.602)}
\gppoint{gp mark 7}{(5.162,3.600)}
\gppoint{gp mark 7}{(5.173,3.599)}
\gppoint{gp mark 7}{(5.184,3.599)}
\gppoint{gp mark 7}{(5.195,3.599)}
\gppoint{gp mark 7}{(5.206,3.599)}
\gppoint{gp mark 7}{(5.216,3.599)}
\gppoint{gp mark 7}{(5.227,3.600)}
\gppoint{gp mark 7}{(5.238,3.600)}
\gppoint{gp mark 7}{(5.249,3.601)}
\gppoint{gp mark 7}{(5.260,3.602)}
\gppoint{gp mark 7}{(5.271,3.604)}
\gppoint{gp mark 7}{(5.282,3.604)}
\gppoint{gp mark 7}{(5.293,3.605)}
\gppoint{gp mark 7}{(5.304,3.605)}
\gppoint{gp mark 7}{(5.315,3.604)}
\gppoint{gp mark 7}{(5.326,3.602)}
\gppoint{gp mark 7}{(5.337,3.600)}
\gppoint{gp mark 7}{(5.348,3.599)}
\gppoint{gp mark 7}{(5.359,3.598)}
\gppoint{gp mark 7}{(5.370,3.598)}
\gppoint{gp mark 7}{(5.381,3.598)}
\gppoint{gp mark 7}{(5.392,3.598)}
\gppoint{gp mark 7}{(5.403,3.598)}
\gppoint{gp mark 7}{(5.414,3.598)}
\gppoint{gp mark 7}{(5.425,3.599)}
\gppoint{gp mark 7}{(5.436,3.601)}
\gppoint{gp mark 7}{(5.447,3.602)}
\gppoint{gp mark 7}{(5.458,3.604)}
\gppoint{gp mark 7}{(5.469,3.604)}
\gppoint{gp mark 7}{(5.480,3.605)}
\gppoint{gp mark 7}{(5.491,3.605)}
\gppoint{gp mark 7}{(5.502,3.604)}
\gppoint{gp mark 7}{(5.513,3.602)}
\gppoint{gp mark 7}{(5.524,3.600)}
\gppoint{gp mark 7}{(5.535,3.600)}
\gppoint{gp mark 7}{(5.546,3.600)}
\gppoint{gp mark 7}{(5.557,3.600)}
\gppoint{gp mark 7}{(5.568,3.600)}
\gppoint{gp mark 7}{(5.579,3.600)}
\gppoint{gp mark 7}{(5.590,3.600)}
\gppoint{gp mark 7}{(5.601,3.600)}
\gppoint{gp mark 7}{(5.612,3.602)}
\gppoint{gp mark 7}{(5.623,3.603)}
\gppoint{gp mark 7}{(5.634,3.603)}
\gppoint{gp mark 7}{(5.645,3.604)}
\gppoint{gp mark 7}{(5.656,3.604)}
\gppoint{gp mark 7}{(5.667,3.604)}
\gppoint{gp mark 7}{(5.678,3.603)}
\gppoint{gp mark 7}{(5.689,3.601)}
\gppoint{gp mark 7}{(5.700,3.599)}
\gppoint{gp mark 7}{(5.711,3.599)}
\gppoint{gp mark 7}{(5.722,3.599)}
\gppoint{gp mark 7}{(5.733,3.599)}
\gppoint{gp mark 7}{(5.744,3.599)}
\gppoint{gp mark 7}{(5.755,3.599)}
\gppoint{gp mark 7}{(5.766,3.600)}
\gppoint{gp mark 7}{(5.777,3.602)}
\gppoint{gp mark 7}{(5.788,3.603)}
\gppoint{gp mark 7}{(5.799,3.604)}
\gppoint{gp mark 7}{(5.810,3.604)}
\gppoint{gp mark 7}{(5.821,3.604)}
\gppoint{gp mark 7}{(5.832,3.603)}
\gppoint{gp mark 7}{(5.843,3.602)}
\gppoint{gp mark 7}{(5.854,3.600)}
\gppoint{gp mark 7}{(5.865,3.600)}
\gppoint{gp mark 7}{(5.876,3.599)}
\gppoint{gp mark 7}{(5.887,3.599)}
\gppoint{gp mark 7}{(5.898,3.599)}
\gppoint{gp mark 7}{(5.909,3.599)}
\gppoint{gp mark 7}{(5.920,3.600)}
\gppoint{gp mark 7}{(5.931,3.602)}
\gppoint{gp mark 7}{(5.942,3.603)}
\gppoint{gp mark 7}{(5.953,3.604)}
\gppoint{gp mark 7}{(5.964,3.604)}
\gppoint{gp mark 7}{(5.975,3.604)}
\gppoint{gp mark 7}{(5.986,3.604)}
\gppoint{gp mark 7}{(5.997,3.603)}
\gppoint{gp mark 7}{(6.008,3.601)}
\gppoint{gp mark 7}{(6.019,3.600)}
\gppoint{gp mark 7}{(6.030,3.599)}
\gppoint{gp mark 7}{(6.041,3.599)}
\gppoint{gp mark 7}{(6.052,3.599)}
\gppoint{gp mark 7}{(6.063,3.598)}
\gppoint{gp mark 7}{(6.074,3.598)}
\gppoint{gp mark 7}{(6.085,3.598)}
\gppoint{gp mark 7}{(6.096,3.599)}
\gppoint{gp mark 7}{(6.107,3.601)}
\gppoint{gp mark 7}{(6.118,3.602)}
\gppoint{gp mark 7}{(6.128,3.603)}
\gppoint{gp mark 7}{(6.139,3.603)}
\gppoint{gp mark 7}{(6.150,3.603)}
\gppoint{gp mark 7}{(6.161,3.603)}
\gppoint{gp mark 7}{(6.172,3.602)}
\gppoint{gp mark 7}{(6.183,3.601)}
\gppoint{gp mark 7}{(6.194,3.600)}
\gppoint{gp mark 7}{(6.205,3.600)}
\gppoint{gp mark 7}{(6.216,3.599)}
\gppoint{gp mark 7}{(6.227,3.599)}
\gppoint{gp mark 7}{(6.238,3.599)}
\gppoint{gp mark 7}{(6.249,3.600)}
\gppoint{gp mark 7}{(6.260,3.600)}
\gppoint{gp mark 7}{(6.271,3.602)}
\gppoint{gp mark 7}{(6.282,3.603)}
\gppoint{gp mark 7}{(6.293,3.603)}
\gppoint{gp mark 7}{(6.304,3.603)}
\gppoint{gp mark 7}{(6.315,3.603)}
\gppoint{gp mark 7}{(6.326,3.602)}
\gppoint{gp mark 7}{(6.337,3.601)}
\gppoint{gp mark 7}{(6.348,3.600)}
\gppoint{gp mark 7}{(6.359,3.598)}
\gppoint{gp mark 7}{(6.370,3.598)}
\gppoint{gp mark 7}{(6.381,3.598)}
\gppoint{gp mark 7}{(6.392,3.598)}
\gppoint{gp mark 7}{(6.403,3.599)}
\gppoint{gp mark 7}{(6.414,3.600)}
\gppoint{gp mark 7}{(6.425,3.601)}
\gppoint{gp mark 7}{(6.436,3.603)}
\gppoint{gp mark 7}{(6.447,3.604)}
\gppoint{gp mark 7}{(6.458,3.605)}
\gppoint{gp mark 7}{(6.469,3.605)}
\gppoint{gp mark 7}{(6.480,3.605)}
\gppoint{gp mark 7}{(6.491,3.604)}
\gppoint{gp mark 7}{(6.502,3.602)}
\gppoint{gp mark 7}{(6.513,3.601)}
\gppoint{gp mark 7}{(6.524,3.600)}
\gppoint{gp mark 7}{(6.535,3.600)}
\gppoint{gp mark 7}{(6.546,3.600)}
\gppoint{gp mark 7}{(6.557,3.600)}
\gppoint{gp mark 7}{(6.568,3.602)}
\gppoint{gp mark 7}{(6.579,3.603)}
\gppoint{gp mark 7}{(6.590,3.603)}
\gppoint{gp mark 7}{(6.601,3.603)}
\gppoint{gp mark 7}{(6.612,3.603)}
\gppoint{gp mark 7}{(6.623,3.603)}
\gppoint{gp mark 7}{(6.634,3.601)}
\gppoint{gp mark 7}{(6.645,3.599)}
\gppoint{gp mark 7}{(6.656,3.598)}
\gppoint{gp mark 7}{(6.667,3.598)}
\gppoint{gp mark 7}{(6.678,3.597)}
\gppoint{gp mark 7}{(6.689,3.597)}
\gppoint{gp mark 7}{(6.700,3.597)}
\gppoint{gp mark 7}{(6.711,3.597)}
\gppoint{gp mark 7}{(6.722,3.598)}
\gppoint{gp mark 7}{(6.733,3.600)}
\gppoint{gp mark 7}{(6.744,3.602)}
\gppoint{gp mark 7}{(6.755,3.604)}
\gppoint{gp mark 7}{(6.766,3.604)}
\gppoint{gp mark 7}{(6.777,3.604)}
\gppoint{gp mark 7}{(6.788,3.604)}
\gppoint{gp mark 7}{(6.799,3.604)}
\gppoint{gp mark 7}{(6.810,3.603)}
\gppoint{gp mark 7}{(6.821,3.602)}
\gppoint{gp mark 7}{(6.832,3.601)}
\gppoint{gp mark 7}{(6.843,3.600)}
\gppoint{gp mark 7}{(6.854,3.600)}
\gppoint{gp mark 7}{(6.865,3.600)}
\gppoint{gp mark 7}{(6.876,3.600)}
\gppoint{gp mark 7}{(6.887,3.601)}
\gppoint{gp mark 7}{(6.898,3.602)}
\gppoint{gp mark 7}{(6.909,3.603)}
\gppoint{gp mark 7}{(6.920,3.605)}
\gppoint{gp mark 7}{(6.931,3.605)}
\gppoint{gp mark 7}{(6.942,3.605)}
\gppoint{gp mark 7}{(6.953,3.604)}
\gppoint{gp mark 7}{(6.964,3.603)}
\gppoint{gp mark 7}{(6.975,3.602)}
\gppoint{gp mark 7}{(6.986,3.602)}
\gppoint{gp mark 7}{(6.997,3.601)}
\gppoint{gp mark 7}{(7.008,3.601)}
\gppoint{gp mark 7}{(7.019,3.601)}
\gppoint{gp mark 7}{(7.030,3.602)}
\gppoint{gp mark 7}{(7.040,3.604)}
\gppoint{gp mark 7}{(7.051,3.606)}
\gppoint{gp mark 7}{(7.062,3.607)}
\gppoint{gp mark 7}{(7.073,3.607)}
\gppoint{gp mark 7}{(7.084,3.608)}
\gppoint{gp mark 7}{(7.095,3.608)}
\gppoint{gp mark 7}{(7.106,3.608)}
\gppoint{gp mark 7}{(7.117,3.608)}
\gppoint{gp mark 7}{(7.128,3.608)}
\gppoint{gp mark 7}{(7.139,3.608)}
\gppoint{gp mark 7}{(7.150,3.608)}
\gppoint{gp mark 7}{(7.161,3.608)}
\gppoint{gp mark 7}{(7.172,3.608)}
\gppoint{gp mark 7}{(7.183,3.608)}
\gppoint{gp mark 7}{(7.194,3.608)}
\gppoint{gp mark 7}{(7.205,3.608)}
\gppoint{gp mark 7}{(7.216,3.607)}
\gppoint{gp mark 7}{(7.227,3.607)}
\gppoint{gp mark 7}{(7.238,3.605)}
\gppoint{gp mark 7}{(7.249,3.603)}
\gppoint{gp mark 7}{(7.260,3.600)}
\gppoint{gp mark 7}{(7.271,3.599)}
\gppoint{gp mark 7}{(7.282,3.598)}
\gppoint{gp mark 7}{(7.293,3.597)}
\gppoint{gp mark 7}{(7.304,3.596)}
\gppoint{gp mark 7}{(7.315,3.593)}
\gppoint{gp mark 7}{(7.326,3.587)}
\gppoint{gp mark 7}{(7.337,3.582)}
\gppoint{gp mark 7}{(7.348,3.577)}
\gppoint{gp mark 7}{(7.359,3.575)}
\gppoint{gp mark 7}{(7.370,3.574)}
\gppoint{gp mark 7}{(7.381,3.573)}
\gppoint{gp mark 7}{(7.392,3.573)}
\gppoint{gp mark 7}{(7.403,3.573)}
\gppoint{gp mark 7}{(7.414,3.573)}
\gppoint{gp mark 7}{(7.425,3.573)}
\gppoint{gp mark 7}{(7.436,3.573)}
\gppoint{gp mark 7}{(7.447,3.571)}
\gppoint{gp mark 7}{(7.458,3.557)}
\gppoint{gp mark 7}{(7.469,3.451)}
\gppoint{gp mark 7}{(7.480,2.908)}
\gppoint{gp mark 7}{(7.491,1.761)}
\gpcolor{rgb color={0.000,0.000,0.000}}
\gpsetlinetype{gp lt plot 0}
\gpsetlinewidth{4.00}
\draw[gp path] (2.427,2.420)--(3.533,2.420);
\draw[gp path] (3.533,2.420)--(4.326,2.420);
\draw[gp path] (4.326,3.596)--(7.510,3.596);
\draw[gp path] (1.200,4.130)--(1.206,4.121)--(1.212,4.112)--(1.218,4.102)--(1.224,4.093)%
  --(1.230,4.084)--(1.236,4.075)--(1.242,4.066)--(1.248,4.057)--(1.254,4.047)--(1.260,4.038)%
  --(1.267,4.029)--(1.273,4.020)--(1.279,4.011)--(1.285,4.002)--(1.291,3.993)--(1.297,3.984)%
  --(1.303,3.974)--(1.309,3.965)--(1.315,3.956)--(1.321,3.947)--(1.327,3.938)--(1.333,3.929)%
  --(1.339,3.920)--(1.346,3.911)--(1.352,3.902)--(1.358,3.893)--(1.364,3.884)--(1.370,3.875)%
  --(1.376,3.866)--(1.382,3.857)--(1.388,3.848)--(1.394,3.839)--(1.400,3.830)--(1.406,3.821)%
  --(1.412,3.812)--(1.418,3.803)--(1.425,3.794)--(1.431,3.785)--(1.437,3.776)--(1.443,3.767)%
  --(1.449,3.758)--(1.455,3.749)--(1.461,3.740)--(1.467,3.731)--(1.473,3.722)--(1.479,3.713)%
  --(1.485,3.704)--(1.491,3.695)--(1.497,3.686)--(1.504,3.677)--(1.510,3.668)--(1.516,3.660)%
  --(1.522,3.651)--(1.528,3.642)--(1.534,3.633)--(1.540,3.624)--(1.546,3.615)--(1.552,3.606)%
  --(1.558,3.597)--(1.564,3.589)--(1.570,3.580)--(1.576,3.571)--(1.583,3.562)--(1.589,3.553)%
  --(1.595,3.545)--(1.601,3.536)--(1.607,3.527)--(1.613,3.518)--(1.619,3.509)--(1.625,3.501)%
  --(1.631,3.492)--(1.637,3.483)--(1.643,3.474)--(1.649,3.465)--(1.656,3.457)--(1.662,3.448)%
  --(1.668,3.439)--(1.674,3.431)--(1.680,3.422)--(1.686,3.413)--(1.692,3.404)--(1.698,3.396)%
  --(1.704,3.387)--(1.710,3.378)--(1.716,3.370)--(1.722,3.361)--(1.728,3.352)--(1.735,3.344)%
  --(1.741,3.335)--(1.747,3.326)--(1.753,3.318)--(1.759,3.309)--(1.765,3.300)--(1.771,3.292)%
  --(1.777,3.283)--(1.783,3.274)--(1.789,3.266)--(1.795,3.257)--(1.801,3.249)--(1.807,3.240)%
  --(1.814,3.232)--(1.820,3.223)--(1.826,3.214)--(1.832,3.206)--(1.838,3.197)--(1.844,3.189)%
  --(1.850,3.180)--(1.856,3.172)--(1.862,3.163)--(1.868,3.155)--(1.874,3.146)--(1.880,3.138)%
  --(1.886,3.129)--(1.893,3.121)--(1.899,3.112)--(1.905,3.104)--(1.911,3.095)--(1.917,3.087)%
  --(1.923,3.078)--(1.929,3.070)--(1.935,3.061)--(1.941,3.053)--(1.947,3.045)--(1.953,3.036)%
  --(1.959,3.028)--(1.965,3.019)--(1.972,3.011)--(1.978,3.003)--(1.984,2.994)--(1.990,2.986)%
  --(1.996,2.978)--(2.002,2.969)--(2.008,2.961)--(2.014,2.952)--(2.020,2.944)--(2.026,2.936)%
  --(2.032,2.928)--(2.038,2.919)--(2.044,2.911)--(2.051,2.903)--(2.057,2.894)--(2.063,2.886)%
  --(2.069,2.878)--(2.075,2.869)--(2.081,2.861)--(2.087,2.853)--(2.093,2.845)--(2.099,2.837)%
  --(2.105,2.828)--(2.111,2.820)--(2.117,2.812)--(2.124,2.804)--(2.130,2.795)--(2.136,2.787)%
  --(2.142,2.779)--(2.148,2.771)--(2.154,2.763)--(2.160,2.755)--(2.166,2.746)--(2.172,2.738)%
  --(2.178,2.730)--(2.184,2.722)--(2.190,2.714)--(2.196,2.706)--(2.203,2.698)--(2.209,2.690)%
  --(2.215,2.682)--(2.221,2.673)--(2.227,2.665)--(2.233,2.657)--(2.239,2.649)--(2.245,2.641)%
  --(2.251,2.633)--(2.257,2.625)--(2.263,2.617)--(2.269,2.609)--(2.275,2.601)--(2.282,2.593)%
  --(2.288,2.585)--(2.294,2.577)--(2.300,2.569)--(2.306,2.561)--(2.312,2.553)--(2.318,2.545)%
  --(2.324,2.537)--(2.330,2.529)--(2.336,2.522)--(2.342,2.514)--(2.348,2.506)--(2.354,2.498)%
  --(2.361,2.490)--(2.367,2.482)--(2.373,2.474)--(2.379,2.466)--(2.385,2.459)--(2.391,2.451)%
  --(2.397,2.443)--(2.403,2.435)--(2.409,2.427)--(2.415,2.419)--(2.421,2.412)--(2.427,2.420);
\draw[gp path] (4.326,2.420)--(4.326,3.596);
\draw[gp path] (7.510,3.596)--(7.510,0.985);
\node[gp node left,font={\fontsize{10pt}{12pt}\selectfont}] at (1.421,5.244) {\LARGE $\rho$};
\node[gp node left,font={\fontsize{10pt}{12pt}\selectfont}] at (6.147,5.244) {\large $\alpha = \pi$};
%% coordinates of the plot area
\gpdefrectangularnode{gp plot 1}{\pgfpoint{1.196cm}{0.985cm}}{\pgfpoint{7.947cm}{5.631cm}}
\end{tikzpicture}
%% gnuplot variables
} & 
\resizebox{0.5\linewidth}{!}{\tikzsetnextfilename{coplanar_a_crsol_6}\begin{tikzpicture}[gnuplot]
%% generated with GNUPLOT 4.6p4 (Lua 5.1; terminal rev. 99, script rev. 100)
%% Mon 02 Jun 2014 11:38:52 AM EDT
\path (0.000,0.000) rectangle (8.500,6.000);
\gpfill{rgb color={1.000,1.000,1.000}} (1.196,0.985)--(7.946,0.985)--(7.946,5.630)--(1.196,5.630)--cycle;
\gpcolor{color=gp lt color border}
\gpsetlinetype{gp lt border}
\gpsetlinewidth{1.00}
\draw[gp path] (1.196,0.985)--(1.196,5.630)--(7.946,5.630)--(7.946,0.985)--cycle;
\gpcolor{color=gp lt color axes}
\gpsetlinetype{gp lt axes}
\gpsetlinewidth{2.00}
\draw[gp path] (1.196,0.985)--(7.947,0.985);
\gpcolor{color=gp lt color border}
\gpsetlinetype{gp lt border}
\draw[gp path] (1.196,0.985)--(1.268,0.985);
\draw[gp path] (7.947,0.985)--(7.875,0.985);
\gpcolor{rgb color={0.000,0.000,0.000}}
\node[gp node right,font={\fontsize{10pt}{12pt}\selectfont}] at (1.012,0.985) {-0.4};
\gpcolor{color=gp lt color axes}
\gpsetlinetype{gp lt axes}
\draw[gp path] (1.196,1.759)--(7.947,1.759);
\gpcolor{color=gp lt color border}
\gpsetlinetype{gp lt border}
\draw[gp path] (1.196,1.759)--(1.268,1.759);
\draw[gp path] (7.947,1.759)--(7.875,1.759);
\gpcolor{rgb color={0.000,0.000,0.000}}
\node[gp node right,font={\fontsize{10pt}{12pt}\selectfont}] at (1.012,1.759) {-0.2};
\gpcolor{color=gp lt color axes}
\gpsetlinetype{gp lt axes}
\draw[gp path] (1.196,2.534)--(7.947,2.534);
\gpcolor{color=gp lt color border}
\gpsetlinetype{gp lt border}
\draw[gp path] (1.196,2.534)--(1.268,2.534);
\draw[gp path] (7.947,2.534)--(7.875,2.534);
\gpcolor{rgb color={0.000,0.000,0.000}}
\node[gp node right,font={\fontsize{10pt}{12pt}\selectfont}] at (1.012,2.534) {0};
\gpcolor{color=gp lt color axes}
\gpsetlinetype{gp lt axes}
\draw[gp path] (1.196,3.308)--(7.947,3.308);
\gpcolor{color=gp lt color border}
\gpsetlinetype{gp lt border}
\draw[gp path] (1.196,3.308)--(1.268,3.308);
\draw[gp path] (7.947,3.308)--(7.875,3.308);
\gpcolor{rgb color={0.000,0.000,0.000}}
\node[gp node right,font={\fontsize{10pt}{12pt}\selectfont}] at (1.012,3.308) {0.2};
\gpcolor{color=gp lt color axes}
\gpsetlinetype{gp lt axes}
\draw[gp path] (1.196,4.082)--(7.947,4.082);
\gpcolor{color=gp lt color border}
\gpsetlinetype{gp lt border}
\draw[gp path] (1.196,4.082)--(1.268,4.082);
\draw[gp path] (7.947,4.082)--(7.875,4.082);
\gpcolor{rgb color={0.000,0.000,0.000}}
\node[gp node right,font={\fontsize{10pt}{12pt}\selectfont}] at (1.012,4.082) {0.4};
\gpcolor{color=gp lt color axes}
\gpsetlinetype{gp lt axes}
\draw[gp path] (1.196,4.857)--(7.947,4.857);
\gpcolor{color=gp lt color border}
\gpsetlinetype{gp lt border}
\draw[gp path] (1.196,4.857)--(1.268,4.857);
\draw[gp path] (7.947,4.857)--(7.875,4.857);
\gpcolor{rgb color={0.000,0.000,0.000}}
\node[gp node right,font={\fontsize{10pt}{12pt}\selectfont}] at (1.012,4.857) {0.6};
\gpcolor{color=gp lt color axes}
\gpsetlinetype{gp lt axes}
\draw[gp path] (1.196,5.631)--(7.947,5.631);
\gpcolor{color=gp lt color border}
\gpsetlinetype{gp lt border}
\draw[gp path] (1.196,5.631)--(1.268,5.631);
\draw[gp path] (7.947,5.631)--(7.875,5.631);
\gpcolor{rgb color={0.000,0.000,0.000}}
\node[gp node right,font={\fontsize{10pt}{12pt}\selectfont}] at (1.012,5.631) {0.8};
\gpcolor{color=gp lt color axes}
\gpsetlinetype{gp lt axes}
\draw[gp path] (1.196,0.985)--(1.196,5.631);
\gpcolor{color=gp lt color border}
\gpsetlinetype{gp lt border}
\draw[gp path] (1.196,0.985)--(1.196,1.057);
\draw[gp path] (1.196,5.631)--(1.196,5.559);
\gpcolor{rgb color={0.000,0.000,0.000}}
\node[gp node center,font={\fontsize{10pt}{12pt}\selectfont}] at (1.196,0.677) {0.2};
\gpcolor{color=gp lt color axes}
\gpsetlinetype{gp lt axes}
\draw[gp path] (2.321,0.985)--(2.321,5.631);
\gpcolor{color=gp lt color border}
\gpsetlinetype{gp lt border}
\draw[gp path] (2.321,0.985)--(2.321,1.057);
\draw[gp path] (2.321,5.631)--(2.321,5.559);
\gpcolor{rgb color={0.000,0.000,0.000}}
\node[gp node center,font={\fontsize{10pt}{12pt}\selectfont}] at (2.321,0.677) {0.25};
\gpcolor{color=gp lt color axes}
\gpsetlinetype{gp lt axes}
\draw[gp path] (3.446,0.985)--(3.446,5.631);
\gpcolor{color=gp lt color border}
\gpsetlinetype{gp lt border}
\draw[gp path] (3.446,0.985)--(3.446,1.057);
\draw[gp path] (3.446,5.631)--(3.446,5.559);
\gpcolor{rgb color={0.000,0.000,0.000}}
\node[gp node center,font={\fontsize{10pt}{12pt}\selectfont}] at (3.446,0.677) {0.3};
\gpcolor{color=gp lt color axes}
\gpsetlinetype{gp lt axes}
\draw[gp path] (4.572,0.985)--(4.572,5.631);
\gpcolor{color=gp lt color border}
\gpsetlinetype{gp lt border}
\draw[gp path] (4.572,0.985)--(4.572,1.057);
\draw[gp path] (4.572,5.631)--(4.572,5.559);
\gpcolor{rgb color={0.000,0.000,0.000}}
\node[gp node center,font={\fontsize{10pt}{12pt}\selectfont}] at (4.572,0.677) {0.35};
\gpcolor{color=gp lt color axes}
\gpsetlinetype{gp lt axes}
\draw[gp path] (5.697,0.985)--(5.697,5.631);
\gpcolor{color=gp lt color border}
\gpsetlinetype{gp lt border}
\draw[gp path] (5.697,0.985)--(5.697,1.057);
\draw[gp path] (5.697,5.631)--(5.697,5.559);
\gpcolor{rgb color={0.000,0.000,0.000}}
\node[gp node center,font={\fontsize{10pt}{12pt}\selectfont}] at (5.697,0.677) {0.4};
\gpcolor{color=gp lt color axes}
\gpsetlinetype{gp lt axes}
\draw[gp path] (6.822,0.985)--(6.822,5.631);
\gpcolor{color=gp lt color border}
\gpsetlinetype{gp lt border}
\draw[gp path] (6.822,0.985)--(6.822,1.057);
\draw[gp path] (6.822,5.631)--(6.822,5.559);
\gpcolor{rgb color={0.000,0.000,0.000}}
\node[gp node center,font={\fontsize{10pt}{12pt}\selectfont}] at (6.822,0.677) {0.45};
\gpcolor{color=gp lt color axes}
\gpsetlinetype{gp lt axes}
\draw[gp path] (7.947,0.985)--(7.947,5.631);
\gpcolor{color=gp lt color border}
\gpsetlinetype{gp lt border}
\draw[gp path] (7.947,0.985)--(7.947,1.057);
\draw[gp path] (7.947,5.631)--(7.947,5.559);
\gpcolor{rgb color={0.000,0.000,0.000}}
\node[gp node center,font={\fontsize{10pt}{12pt}\selectfont}] at (7.947,0.677) {0.5};
\gpcolor{color=gp lt color border}
\draw[gp path] (1.196,5.631)--(1.196,0.985)--(7.947,0.985)--(7.947,5.631)--cycle;
\gpcolor{rgb color={0.000,0.000,0.000}}
\node[gp node center,font={\fontsize{10pt}{12pt}\selectfont}] at (4.571,0.215) {\large $x$};
\gpcolor{rgb color={0.502,0.502,0.502}}
\gpsetlinewidth{0.50}
\gpsetpointsize{2.67}
\gppoint{gp mark 7}{(1.206,4.629)}
\gppoint{gp mark 7}{(1.217,4.623)}
\gppoint{gp mark 7}{(1.228,4.618)}
\gppoint{gp mark 7}{(1.239,4.613)}
\gppoint{gp mark 7}{(1.250,4.607)}
\gppoint{gp mark 7}{(1.261,4.602)}
\gppoint{gp mark 7}{(1.272,4.597)}
\gppoint{gp mark 7}{(1.283,4.592)}
\gppoint{gp mark 7}{(1.294,4.586)}
\gppoint{gp mark 7}{(1.305,4.581)}
\gppoint{gp mark 7}{(1.316,4.576)}
\gppoint{gp mark 7}{(1.327,4.570)}
\gppoint{gp mark 7}{(1.338,4.565)}
\gppoint{gp mark 7}{(1.349,4.560)}
\gppoint{gp mark 7}{(1.360,4.554)}
\gppoint{gp mark 7}{(1.371,4.549)}
\gppoint{gp mark 7}{(1.382,4.544)}
\gppoint{gp mark 7}{(1.393,4.539)}
\gppoint{gp mark 7}{(1.404,4.533)}
\gppoint{gp mark 7}{(1.415,4.528)}
\gppoint{gp mark 7}{(1.426,4.523)}
\gppoint{gp mark 7}{(1.437,4.517)}
\gppoint{gp mark 7}{(1.448,4.512)}
\gppoint{gp mark 7}{(1.459,4.507)}
\gppoint{gp mark 7}{(1.470,4.501)}
\gppoint{gp mark 7}{(1.481,4.496)}
\gppoint{gp mark 7}{(1.492,4.491)}
\gppoint{gp mark 7}{(1.503,4.485)}
\gppoint{gp mark 7}{(1.514,4.480)}
\gppoint{gp mark 7}{(1.525,4.475)}
\gppoint{gp mark 7}{(1.536,4.469)}
\gppoint{gp mark 7}{(1.547,4.464)}
\gppoint{gp mark 7}{(1.558,4.459)}
\gppoint{gp mark 7}{(1.568,4.453)}
\gppoint{gp mark 7}{(1.579,4.448)}
\gppoint{gp mark 7}{(1.590,4.443)}
\gppoint{gp mark 7}{(1.601,4.437)}
\gppoint{gp mark 7}{(1.612,4.432)}
\gppoint{gp mark 7}{(1.623,4.427)}
\gppoint{gp mark 7}{(1.634,4.421)}
\gppoint{gp mark 7}{(1.645,4.416)}
\gppoint{gp mark 7}{(1.656,4.410)}
\gppoint{gp mark 7}{(1.667,4.405)}
\gppoint{gp mark 7}{(1.678,4.400)}
\gppoint{gp mark 7}{(1.689,4.394)}
\gppoint{gp mark 7}{(1.700,4.389)}
\gppoint{gp mark 7}{(1.711,4.384)}
\gppoint{gp mark 7}{(1.722,4.378)}
\gppoint{gp mark 7}{(1.733,4.373)}
\gppoint{gp mark 7}{(1.744,4.367)}
\gppoint{gp mark 7}{(1.755,4.362)}
\gppoint{gp mark 7}{(1.766,4.357)}
\gppoint{gp mark 7}{(1.777,4.351)}
\gppoint{gp mark 7}{(1.788,4.346)}
\gppoint{gp mark 7}{(1.799,4.341)}
\gppoint{gp mark 7}{(1.810,4.335)}
\gppoint{gp mark 7}{(1.821,4.330)}
\gppoint{gp mark 7}{(1.832,4.324)}
\gppoint{gp mark 7}{(1.843,4.319)}
\gppoint{gp mark 7}{(1.854,4.313)}
\gppoint{gp mark 7}{(1.865,4.308)}
\gppoint{gp mark 7}{(1.876,4.303)}
\gppoint{gp mark 7}{(1.887,4.297)}
\gppoint{gp mark 7}{(1.898,4.292)}
\gppoint{gp mark 7}{(1.909,4.286)}
\gppoint{gp mark 7}{(1.920,4.281)}
\gppoint{gp mark 7}{(1.931,4.275)}
\gppoint{gp mark 7}{(1.942,4.270)}
\gppoint{gp mark 7}{(1.953,4.265)}
\gppoint{gp mark 7}{(1.964,4.259)}
\gppoint{gp mark 7}{(1.975,4.254)}
\gppoint{gp mark 7}{(1.986,4.248)}
\gppoint{gp mark 7}{(1.997,4.243)}
\gppoint{gp mark 7}{(2.008,4.237)}
\gppoint{gp mark 7}{(2.019,4.232)}
\gppoint{gp mark 7}{(2.030,4.226)}
\gppoint{gp mark 7}{(2.041,4.221)}
\gppoint{gp mark 7}{(2.052,4.215)}
\gppoint{gp mark 7}{(2.063,4.210)}
\gppoint{gp mark 7}{(2.074,4.204)}
\gppoint{gp mark 7}{(2.085,4.199)}
\gppoint{gp mark 7}{(2.096,4.193)}
\gppoint{gp mark 7}{(2.107,4.188)}
\gppoint{gp mark 7}{(2.118,4.182)}
\gppoint{gp mark 7}{(2.129,4.177)}
\gppoint{gp mark 7}{(2.140,4.171)}
\gppoint{gp mark 7}{(2.151,4.166)}
\gppoint{gp mark 7}{(2.162,4.160)}
\gppoint{gp mark 7}{(2.173,4.155)}
\gppoint{gp mark 7}{(2.184,4.149)}
\gppoint{gp mark 7}{(2.195,4.144)}
\gppoint{gp mark 7}{(2.206,4.138)}
\gppoint{gp mark 7}{(2.217,4.133)}
\gppoint{gp mark 7}{(2.228,4.127)}
\gppoint{gp mark 7}{(2.239,4.121)}
\gppoint{gp mark 7}{(2.250,4.116)}
\gppoint{gp mark 7}{(2.261,4.110)}
\gppoint{gp mark 7}{(2.272,4.105)}
\gppoint{gp mark 7}{(2.283,4.099)}
\gppoint{gp mark 7}{(2.294,4.094)}
\gppoint{gp mark 7}{(2.305,4.088)}
\gppoint{gp mark 7}{(2.316,4.082)}
\gppoint{gp mark 7}{(2.327,4.077)}
\gppoint{gp mark 7}{(2.338,4.071)}
\gppoint{gp mark 7}{(2.349,4.065)}
\gppoint{gp mark 7}{(2.360,4.060)}
\gppoint{gp mark 7}{(2.371,4.054)}
\gppoint{gp mark 7}{(2.382,4.049)}
\gppoint{gp mark 7}{(2.393,4.043)}
\gppoint{gp mark 7}{(2.404,4.037)}
\gppoint{gp mark 7}{(2.415,4.032)}
\gppoint{gp mark 7}{(2.426,4.026)}
\gppoint{gp mark 7}{(2.437,4.020)}
\gppoint{gp mark 7}{(2.448,4.015)}
\gppoint{gp mark 7}{(2.459,4.009)}
\gppoint{gp mark 7}{(2.470,4.003)}
\gppoint{gp mark 7}{(2.480,3.997)}
\gppoint{gp mark 7}{(2.491,3.992)}
\gppoint{gp mark 7}{(2.502,3.986)}
\gppoint{gp mark 7}{(2.513,3.980)}
\gppoint{gp mark 7}{(2.524,3.975)}
\gppoint{gp mark 7}{(2.535,3.969)}
\gppoint{gp mark 7}{(2.546,3.963)}
\gppoint{gp mark 7}{(2.557,3.957)}
\gppoint{gp mark 7}{(2.568,3.952)}
\gppoint{gp mark 7}{(2.579,3.946)}
\gppoint{gp mark 7}{(2.590,3.940)}
\gppoint{gp mark 7}{(2.601,3.934)}
\gppoint{gp mark 7}{(2.612,3.928)}
\gppoint{gp mark 7}{(2.623,3.923)}
\gppoint{gp mark 7}{(2.634,3.917)}
\gppoint{gp mark 7}{(2.645,3.911)}
\gppoint{gp mark 7}{(2.656,3.905)}
\gppoint{gp mark 7}{(2.667,3.899)}
\gppoint{gp mark 7}{(2.678,3.894)}
\gppoint{gp mark 7}{(2.689,3.888)}
\gppoint{gp mark 7}{(2.700,3.882)}
\gppoint{gp mark 7}{(2.711,3.876)}
\gppoint{gp mark 7}{(2.722,3.870)}
\gppoint{gp mark 7}{(2.733,3.864)}
\gppoint{gp mark 7}{(2.744,3.858)}
\gppoint{gp mark 7}{(2.755,3.852)}
\gppoint{gp mark 7}{(2.766,3.847)}
\gppoint{gp mark 7}{(2.777,3.841)}
\gppoint{gp mark 7}{(2.788,3.835)}
\gppoint{gp mark 7}{(2.799,3.829)}
\gppoint{gp mark 7}{(2.810,3.823)}
\gppoint{gp mark 7}{(2.821,3.817)}
\gppoint{gp mark 7}{(2.832,3.811)}
\gppoint{gp mark 7}{(2.843,3.805)}
\gppoint{gp mark 7}{(2.854,3.799)}
\gppoint{gp mark 7}{(2.865,3.793)}
\gppoint{gp mark 7}{(2.876,3.787)}
\gppoint{gp mark 7}{(2.887,3.781)}
\gppoint{gp mark 7}{(2.898,3.775)}
\gppoint{gp mark 7}{(2.909,3.770)}
\gppoint{gp mark 7}{(2.920,3.764)}
\gppoint{gp mark 7}{(2.931,3.758)}
\gppoint{gp mark 7}{(2.942,3.752)}
\gppoint{gp mark 7}{(2.953,3.746)}
\gppoint{gp mark 7}{(2.964,3.740)}
\gppoint{gp mark 7}{(2.975,3.734)}
\gppoint{gp mark 7}{(2.986,3.728)}
\gppoint{gp mark 7}{(2.997,3.722)}
\gppoint{gp mark 7}{(3.008,3.716)}
\gppoint{gp mark 7}{(3.019,3.709)}
\gppoint{gp mark 7}{(3.030,3.703)}
\gppoint{gp mark 7}{(3.041,3.698)}
\gppoint{gp mark 7}{(3.052,3.694)}
\gppoint{gp mark 7}{(3.063,3.692)}
\gppoint{gp mark 7}{(3.074,3.692)}
\gppoint{gp mark 7}{(3.085,3.692)}
\gppoint{gp mark 7}{(3.096,3.694)}
\gppoint{gp mark 7}{(3.107,3.703)}
\gppoint{gp mark 7}{(3.118,3.720)}
\gppoint{gp mark 7}{(3.129,3.733)}
\gppoint{gp mark 7}{(3.140,3.737)}
\gppoint{gp mark 7}{(3.151,3.737)}
\gppoint{gp mark 7}{(3.162,3.737)}
\gppoint{gp mark 7}{(3.173,3.737)}
\gppoint{gp mark 7}{(3.184,3.737)}
\gppoint{gp mark 7}{(3.195,3.737)}
\gppoint{gp mark 7}{(3.206,3.737)}
\gppoint{gp mark 7}{(3.217,3.736)}
\gppoint{gp mark 7}{(3.228,3.735)}
\gppoint{gp mark 7}{(3.239,3.734)}
\gppoint{gp mark 7}{(3.250,3.732)}
\gppoint{gp mark 7}{(3.261,3.731)}
\gppoint{gp mark 7}{(3.272,3.730)}
\gppoint{gp mark 7}{(3.283,3.729)}
\gppoint{gp mark 7}{(3.294,3.728)}
\gppoint{gp mark 7}{(3.305,3.728)}
\gppoint{gp mark 7}{(3.316,3.727)}
\gppoint{gp mark 7}{(3.327,3.727)}
\gppoint{gp mark 7}{(3.338,3.727)}
\gppoint{gp mark 7}{(3.349,3.726)}
\gppoint{gp mark 7}{(3.360,3.726)}
\gppoint{gp mark 7}{(3.371,3.726)}
\gppoint{gp mark 7}{(3.382,3.727)}
\gppoint{gp mark 7}{(3.392,3.727)}
\gppoint{gp mark 7}{(3.403,3.727)}
\gppoint{gp mark 7}{(3.414,3.727)}
\gppoint{gp mark 7}{(3.425,3.728)}
\gppoint{gp mark 7}{(3.436,3.728)}
\gppoint{gp mark 7}{(3.447,3.728)}
\gppoint{gp mark 7}{(3.458,3.729)}
\gppoint{gp mark 7}{(3.469,3.729)}
\gppoint{gp mark 7}{(3.480,3.729)}
\gppoint{gp mark 7}{(3.491,3.730)}
\gppoint{gp mark 7}{(3.502,3.726)}
\gppoint{gp mark 7}{(3.513,3.593)}
\gppoint{gp mark 7}{(3.524,2.513)}
\gppoint{gp mark 7}{(3.535,1.766)}
\gppoint{gp mark 7}{(3.546,1.677)}
\gppoint{gp mark 7}{(3.557,1.654)}
\gppoint{gp mark 7}{(3.568,1.643)}
\gppoint{gp mark 7}{(3.579,1.635)}
\gppoint{gp mark 7}{(3.590,1.629)}
\gppoint{gp mark 7}{(3.601,1.621)}
\gppoint{gp mark 7}{(3.612,1.612)}
\gppoint{gp mark 7}{(3.623,1.604)}
\gppoint{gp mark 7}{(3.634,1.594)}
\gppoint{gp mark 7}{(3.645,1.583)}
\gppoint{gp mark 7}{(3.656,1.575)}
\gppoint{gp mark 7}{(3.667,1.567)}
\gppoint{gp mark 7}{(3.678,1.559)}
\gppoint{gp mark 7}{(3.689,1.549)}
\gppoint{gp mark 7}{(3.700,1.541)}
\gppoint{gp mark 7}{(3.711,1.533)}
\gppoint{gp mark 7}{(3.722,1.525)}
\gppoint{gp mark 7}{(3.733,1.516)}
\gppoint{gp mark 7}{(3.744,1.508)}
\gppoint{gp mark 7}{(3.755,1.501)}
\gppoint{gp mark 7}{(3.766,1.493)}
\gppoint{gp mark 7}{(3.777,1.484)}
\gppoint{gp mark 7}{(3.788,1.477)}
\gppoint{gp mark 7}{(3.799,1.470)}
\gppoint{gp mark 7}{(3.810,1.462)}
\gppoint{gp mark 7}{(3.821,1.454)}
\gppoint{gp mark 7}{(3.832,1.446)}
\gppoint{gp mark 7}{(3.843,1.439)}
\gppoint{gp mark 7}{(3.854,1.431)}
\gppoint{gp mark 7}{(3.865,1.424)}
\gppoint{gp mark 7}{(3.876,1.417)}
\gppoint{gp mark 7}{(3.887,1.410)}
\gppoint{gp mark 7}{(3.898,1.403)}
\gppoint{gp mark 7}{(3.909,1.395)}
\gppoint{gp mark 7}{(3.920,1.389)}
\gppoint{gp mark 7}{(3.931,1.382)}
\gppoint{gp mark 7}{(3.942,1.375)}
\gppoint{gp mark 7}{(3.953,1.368)}
\gppoint{gp mark 7}{(3.964,1.361)}
\gppoint{gp mark 7}{(3.975,1.354)}
\gppoint{gp mark 7}{(3.986,1.348)}
\gppoint{gp mark 7}{(3.997,1.341)}
\gppoint{gp mark 7}{(4.008,1.335)}
\gppoint{gp mark 7}{(4.019,1.328)}
\gppoint{gp mark 7}{(4.030,1.322)}
\gppoint{gp mark 7}{(4.041,1.316)}
\gppoint{gp mark 7}{(4.052,1.309)}
\gppoint{gp mark 7}{(4.063,1.304)}
\gppoint{gp mark 7}{(4.074,1.299)}
\gppoint{gp mark 7}{(4.085,1.294)}
\gppoint{gp mark 7}{(4.096,1.291)}
\gppoint{gp mark 7}{(4.107,1.289)}
\gppoint{gp mark 7}{(4.118,1.289)}
\gppoint{gp mark 7}{(4.129,1.289)}
\gppoint{gp mark 7}{(4.140,1.289)}
\gppoint{gp mark 7}{(4.151,1.290)}
\gppoint{gp mark 7}{(4.162,1.290)}
\gppoint{gp mark 7}{(4.173,1.291)}
\gppoint{gp mark 7}{(4.184,1.291)}
\gppoint{gp mark 7}{(4.195,1.292)}
\gppoint{gp mark 7}{(4.206,1.292)}
\gppoint{gp mark 7}{(4.217,1.292)}
\gppoint{gp mark 7}{(4.228,1.292)}
\gppoint{gp mark 7}{(4.239,1.292)}
\gppoint{gp mark 7}{(4.250,1.292)}
\gppoint{gp mark 7}{(4.261,1.292)}
\gppoint{gp mark 7}{(4.272,1.292)}
\gppoint{gp mark 7}{(4.283,1.292)}
\gppoint{gp mark 7}{(4.294,1.292)}
\gppoint{gp mark 7}{(4.304,1.292)}
\gppoint{gp mark 7}{(4.315,1.292)}
\gppoint{gp mark 7}{(4.326,1.292)}
\gppoint{gp mark 7}{(4.337,1.292)}
\gppoint{gp mark 7}{(4.348,1.292)}
\gppoint{gp mark 7}{(4.359,1.292)}
\gppoint{gp mark 7}{(4.370,1.292)}
\gppoint{gp mark 7}{(4.381,1.292)}
\gppoint{gp mark 7}{(4.392,1.292)}
\gppoint{gp mark 7}{(4.403,1.292)}
\gppoint{gp mark 7}{(4.414,1.292)}
\gppoint{gp mark 7}{(4.425,1.292)}
\gppoint{gp mark 7}{(4.436,1.292)}
\gppoint{gp mark 7}{(4.447,1.292)}
\gppoint{gp mark 7}{(4.458,1.292)}
\gppoint{gp mark 7}{(4.469,1.292)}
\gppoint{gp mark 7}{(4.480,1.292)}
\gppoint{gp mark 7}{(4.491,1.292)}
\gppoint{gp mark 7}{(4.502,1.292)}
\gppoint{gp mark 7}{(4.513,1.292)}
\gppoint{gp mark 7}{(4.524,1.292)}
\gppoint{gp mark 7}{(4.535,1.292)}
\gppoint{gp mark 7}{(4.546,1.292)}
\gppoint{gp mark 7}{(4.557,1.292)}
\gppoint{gp mark 7}{(4.568,1.292)}
\gppoint{gp mark 7}{(4.579,1.292)}
\gppoint{gp mark 7}{(4.590,1.292)}
\gppoint{gp mark 7}{(4.601,1.292)}
\gppoint{gp mark 7}{(4.612,1.292)}
\gppoint{gp mark 7}{(4.623,1.292)}
\gppoint{gp mark 7}{(4.634,1.292)}
\gppoint{gp mark 7}{(4.645,1.292)}
\gppoint{gp mark 7}{(4.656,1.292)}
\gppoint{gp mark 7}{(4.667,1.292)}
\gppoint{gp mark 7}{(4.678,1.292)}
\gppoint{gp mark 7}{(4.689,1.292)}
\gppoint{gp mark 7}{(4.700,1.292)}
\gppoint{gp mark 7}{(4.711,1.292)}
\gppoint{gp mark 7}{(4.722,1.292)}
\gppoint{gp mark 7}{(4.733,1.292)}
\gppoint{gp mark 7}{(4.744,1.292)}
\gppoint{gp mark 7}{(4.755,1.292)}
\gppoint{gp mark 7}{(4.766,1.292)}
\gppoint{gp mark 7}{(4.777,1.292)}
\gppoint{gp mark 7}{(4.788,1.292)}
\gppoint{gp mark 7}{(4.799,1.292)}
\gppoint{gp mark 7}{(4.810,1.292)}
\gppoint{gp mark 7}{(4.821,1.292)}
\gppoint{gp mark 7}{(4.832,1.292)}
\gppoint{gp mark 7}{(4.843,1.292)}
\gppoint{gp mark 7}{(4.854,1.292)}
\gppoint{gp mark 7}{(4.865,1.292)}
\gppoint{gp mark 7}{(4.876,1.292)}
\gppoint{gp mark 7}{(4.887,1.292)}
\gppoint{gp mark 7}{(4.898,1.292)}
\gppoint{gp mark 7}{(4.909,1.292)}
\gppoint{gp mark 7}{(4.920,1.292)}
\gppoint{gp mark 7}{(4.931,1.292)}
\gppoint{gp mark 7}{(4.942,1.292)}
\gppoint{gp mark 7}{(4.953,1.292)}
\gppoint{gp mark 7}{(4.964,1.292)}
\gppoint{gp mark 7}{(4.975,1.292)}
\gppoint{gp mark 7}{(4.986,1.292)}
\gppoint{gp mark 7}{(4.997,1.292)}
\gppoint{gp mark 7}{(5.008,1.292)}
\gppoint{gp mark 7}{(5.019,1.292)}
\gppoint{gp mark 7}{(5.030,1.292)}
\gppoint{gp mark 7}{(5.041,1.292)}
\gppoint{gp mark 7}{(5.052,1.292)}
\gppoint{gp mark 7}{(5.063,1.292)}
\gppoint{gp mark 7}{(5.074,1.292)}
\gppoint{gp mark 7}{(5.085,1.292)}
\gppoint{gp mark 7}{(5.096,1.292)}
\gppoint{gp mark 7}{(5.107,1.292)}
\gppoint{gp mark 7}{(5.118,1.292)}
\gppoint{gp mark 7}{(5.129,1.292)}
\gppoint{gp mark 7}{(5.140,1.292)}
\gppoint{gp mark 7}{(5.151,1.292)}
\gppoint{gp mark 7}{(5.162,1.292)}
\gppoint{gp mark 7}{(5.173,1.292)}
\gppoint{gp mark 7}{(5.184,1.292)}
\gppoint{gp mark 7}{(5.195,1.292)}
\gppoint{gp mark 7}{(5.206,1.292)}
\gppoint{gp mark 7}{(5.216,1.292)}
\gppoint{gp mark 7}{(5.227,1.292)}
\gppoint{gp mark 7}{(5.238,1.292)}
\gppoint{gp mark 7}{(5.249,1.292)}
\gppoint{gp mark 7}{(5.260,1.292)}
\gppoint{gp mark 7}{(5.271,1.292)}
\gppoint{gp mark 7}{(5.282,1.292)}
\gppoint{gp mark 7}{(5.293,1.292)}
\gppoint{gp mark 7}{(5.304,1.292)}
\gppoint{gp mark 7}{(5.315,1.292)}
\gppoint{gp mark 7}{(5.326,1.292)}
\gppoint{gp mark 7}{(5.337,1.292)}
\gppoint{gp mark 7}{(5.348,1.292)}
\gppoint{gp mark 7}{(5.359,1.292)}
\gppoint{gp mark 7}{(5.370,1.292)}
\gppoint{gp mark 7}{(5.381,1.292)}
\gppoint{gp mark 7}{(5.392,1.292)}
\gppoint{gp mark 7}{(5.403,1.292)}
\gppoint{gp mark 7}{(5.414,1.292)}
\gppoint{gp mark 7}{(5.425,1.292)}
\gppoint{gp mark 7}{(5.436,1.292)}
\gppoint{gp mark 7}{(5.447,1.292)}
\gppoint{gp mark 7}{(5.458,1.292)}
\gppoint{gp mark 7}{(5.469,1.292)}
\gppoint{gp mark 7}{(5.480,1.292)}
\gppoint{gp mark 7}{(5.491,1.292)}
\gppoint{gp mark 7}{(5.502,1.292)}
\gppoint{gp mark 7}{(5.513,1.292)}
\gppoint{gp mark 7}{(5.524,1.292)}
\gppoint{gp mark 7}{(5.535,1.292)}
\gppoint{gp mark 7}{(5.546,1.292)}
\gppoint{gp mark 7}{(5.557,1.292)}
\gppoint{gp mark 7}{(5.568,1.292)}
\gppoint{gp mark 7}{(5.579,1.292)}
\gppoint{gp mark 7}{(5.590,1.292)}
\gppoint{gp mark 7}{(5.601,1.292)}
\gppoint{gp mark 7}{(5.612,1.292)}
\gppoint{gp mark 7}{(5.623,1.292)}
\gppoint{gp mark 7}{(5.634,1.292)}
\gppoint{gp mark 7}{(5.645,1.292)}
\gppoint{gp mark 7}{(5.656,1.292)}
\gppoint{gp mark 7}{(5.667,1.292)}
\gppoint{gp mark 7}{(5.678,1.292)}
\gppoint{gp mark 7}{(5.689,1.292)}
\gppoint{gp mark 7}{(5.700,1.292)}
\gppoint{gp mark 7}{(5.711,1.292)}
\gppoint{gp mark 7}{(5.722,1.292)}
\gppoint{gp mark 7}{(5.733,1.292)}
\gppoint{gp mark 7}{(5.744,1.292)}
\gppoint{gp mark 7}{(5.755,1.292)}
\gppoint{gp mark 7}{(5.766,1.292)}
\gppoint{gp mark 7}{(5.777,1.292)}
\gppoint{gp mark 7}{(5.788,1.292)}
\gppoint{gp mark 7}{(5.799,1.292)}
\gppoint{gp mark 7}{(5.810,1.292)}
\gppoint{gp mark 7}{(5.821,1.292)}
\gppoint{gp mark 7}{(5.832,1.292)}
\gppoint{gp mark 7}{(5.843,1.292)}
\gppoint{gp mark 7}{(5.854,1.292)}
\gppoint{gp mark 7}{(5.865,1.292)}
\gppoint{gp mark 7}{(5.876,1.292)}
\gppoint{gp mark 7}{(5.887,1.292)}
\gppoint{gp mark 7}{(5.898,1.292)}
\gppoint{gp mark 7}{(5.909,1.292)}
\gppoint{gp mark 7}{(5.920,1.292)}
\gppoint{gp mark 7}{(5.931,1.292)}
\gppoint{gp mark 7}{(5.942,1.292)}
\gppoint{gp mark 7}{(5.953,1.292)}
\gppoint{gp mark 7}{(5.964,1.292)}
\gppoint{gp mark 7}{(5.975,1.292)}
\gppoint{gp mark 7}{(5.986,1.292)}
\gppoint{gp mark 7}{(5.997,1.292)}
\gppoint{gp mark 7}{(6.008,1.292)}
\gppoint{gp mark 7}{(6.019,1.292)}
\gppoint{gp mark 7}{(6.030,1.292)}
\gppoint{gp mark 7}{(6.041,1.292)}
\gppoint{gp mark 7}{(6.052,1.292)}
\gppoint{gp mark 7}{(6.063,1.292)}
\gppoint{gp mark 7}{(6.074,1.292)}
\gppoint{gp mark 7}{(6.085,1.292)}
\gppoint{gp mark 7}{(6.096,1.292)}
\gppoint{gp mark 7}{(6.107,1.292)}
\gppoint{gp mark 7}{(6.118,1.292)}
\gppoint{gp mark 7}{(6.128,1.292)}
\gppoint{gp mark 7}{(6.139,1.292)}
\gppoint{gp mark 7}{(6.150,1.292)}
\gppoint{gp mark 7}{(6.161,1.292)}
\gppoint{gp mark 7}{(6.172,1.292)}
\gppoint{gp mark 7}{(6.183,1.292)}
\gppoint{gp mark 7}{(6.194,1.292)}
\gppoint{gp mark 7}{(6.205,1.292)}
\gppoint{gp mark 7}{(6.216,1.292)}
\gppoint{gp mark 7}{(6.227,1.292)}
\gppoint{gp mark 7}{(6.238,1.292)}
\gppoint{gp mark 7}{(6.249,1.292)}
\gppoint{gp mark 7}{(6.260,1.292)}
\gppoint{gp mark 7}{(6.271,1.292)}
\gppoint{gp mark 7}{(6.282,1.292)}
\gppoint{gp mark 7}{(6.293,1.292)}
\gppoint{gp mark 7}{(6.304,1.292)}
\gppoint{gp mark 7}{(6.315,1.292)}
\gppoint{gp mark 7}{(6.326,1.292)}
\gppoint{gp mark 7}{(6.337,1.292)}
\gppoint{gp mark 7}{(6.348,1.292)}
\gppoint{gp mark 7}{(6.359,1.292)}
\gppoint{gp mark 7}{(6.370,1.292)}
\gppoint{gp mark 7}{(6.381,1.292)}
\gppoint{gp mark 7}{(6.392,1.292)}
\gppoint{gp mark 7}{(6.403,1.292)}
\gppoint{gp mark 7}{(6.414,1.292)}
\gppoint{gp mark 7}{(6.425,1.292)}
\gppoint{gp mark 7}{(6.436,1.292)}
\gppoint{gp mark 7}{(6.447,1.292)}
\gppoint{gp mark 7}{(6.458,1.292)}
\gppoint{gp mark 7}{(6.469,1.292)}
\gppoint{gp mark 7}{(6.480,1.292)}
\gppoint{gp mark 7}{(6.491,1.292)}
\gppoint{gp mark 7}{(6.502,1.292)}
\gppoint{gp mark 7}{(6.513,1.292)}
\gppoint{gp mark 7}{(6.524,1.292)}
\gppoint{gp mark 7}{(6.535,1.292)}
\gppoint{gp mark 7}{(6.546,1.292)}
\gppoint{gp mark 7}{(6.557,1.292)}
\gppoint{gp mark 7}{(6.568,1.292)}
\gppoint{gp mark 7}{(6.579,1.292)}
\gppoint{gp mark 7}{(6.590,1.292)}
\gppoint{gp mark 7}{(6.601,1.292)}
\gppoint{gp mark 7}{(6.612,1.292)}
\gppoint{gp mark 7}{(6.623,1.292)}
\gppoint{gp mark 7}{(6.634,1.292)}
\gppoint{gp mark 7}{(6.645,1.292)}
\gppoint{gp mark 7}{(6.656,1.292)}
\gppoint{gp mark 7}{(6.667,1.292)}
\gppoint{gp mark 7}{(6.678,1.292)}
\gppoint{gp mark 7}{(6.689,1.292)}
\gppoint{gp mark 7}{(6.700,1.292)}
\gppoint{gp mark 7}{(6.711,1.292)}
\gppoint{gp mark 7}{(6.722,1.292)}
\gppoint{gp mark 7}{(6.733,1.292)}
\gppoint{gp mark 7}{(6.744,1.292)}
\gppoint{gp mark 7}{(6.755,1.292)}
\gppoint{gp mark 7}{(6.766,1.292)}
\gppoint{gp mark 7}{(6.777,1.292)}
\gppoint{gp mark 7}{(6.788,1.292)}
\gppoint{gp mark 7}{(6.799,1.292)}
\gppoint{gp mark 7}{(6.810,1.292)}
\gppoint{gp mark 7}{(6.821,1.292)}
\gppoint{gp mark 7}{(6.832,1.292)}
\gppoint{gp mark 7}{(6.843,1.292)}
\gppoint{gp mark 7}{(6.854,1.292)}
\gppoint{gp mark 7}{(6.865,1.292)}
\gppoint{gp mark 7}{(6.876,1.292)}
\gppoint{gp mark 7}{(6.887,1.292)}
\gppoint{gp mark 7}{(6.898,1.292)}
\gppoint{gp mark 7}{(6.909,1.292)}
\gppoint{gp mark 7}{(6.920,1.292)}
\gppoint{gp mark 7}{(6.931,1.292)}
\gppoint{gp mark 7}{(6.942,1.292)}
\gppoint{gp mark 7}{(6.953,1.292)}
\gppoint{gp mark 7}{(6.964,1.292)}
\gppoint{gp mark 7}{(6.975,1.292)}
\gppoint{gp mark 7}{(6.986,1.292)}
\gppoint{gp mark 7}{(6.997,1.292)}
\gppoint{gp mark 7}{(7.008,1.292)}
\gppoint{gp mark 7}{(7.019,1.292)}
\gppoint{gp mark 7}{(7.030,1.292)}
\gppoint{gp mark 7}{(7.040,1.292)}
\gppoint{gp mark 7}{(7.051,1.292)}
\gppoint{gp mark 7}{(7.062,1.292)}
\gppoint{gp mark 7}{(7.073,1.292)}
\gppoint{gp mark 7}{(7.084,1.292)}
\gppoint{gp mark 7}{(7.095,1.292)}
\gppoint{gp mark 7}{(7.106,1.292)}
\gppoint{gp mark 7}{(7.117,1.292)}
\gppoint{gp mark 7}{(7.128,1.292)}
\gppoint{gp mark 7}{(7.139,1.292)}
\gppoint{gp mark 7}{(7.150,1.292)}
\gppoint{gp mark 7}{(7.161,1.292)}
\gppoint{gp mark 7}{(7.172,1.292)}
\gppoint{gp mark 7}{(7.183,1.292)}
\gppoint{gp mark 7}{(7.194,1.292)}
\gppoint{gp mark 7}{(7.205,1.292)}
\gppoint{gp mark 7}{(7.216,1.292)}
\gppoint{gp mark 7}{(7.227,1.292)}
\gppoint{gp mark 7}{(7.238,1.292)}
\gppoint{gp mark 7}{(7.249,1.292)}
\gppoint{gp mark 7}{(7.260,1.292)}
\gppoint{gp mark 7}{(7.271,1.292)}
\gppoint{gp mark 7}{(7.282,1.292)}
\gppoint{gp mark 7}{(7.293,1.292)}
\gppoint{gp mark 7}{(7.304,1.292)}
\gppoint{gp mark 7}{(7.315,1.292)}
\gppoint{gp mark 7}{(7.326,1.292)}
\gppoint{gp mark 7}{(7.337,1.292)}
\gppoint{gp mark 7}{(7.348,1.292)}
\gppoint{gp mark 7}{(7.359,1.292)}
\gppoint{gp mark 7}{(7.370,1.292)}
\gppoint{gp mark 7}{(7.381,1.292)}
\gppoint{gp mark 7}{(7.392,1.292)}
\gppoint{gp mark 7}{(7.403,1.292)}
\gppoint{gp mark 7}{(7.414,1.292)}
\gppoint{gp mark 7}{(7.425,1.292)}
\gppoint{gp mark 7}{(7.436,1.292)}
\gppoint{gp mark 7}{(7.447,1.292)}
\gppoint{gp mark 7}{(7.458,1.292)}
\gppoint{gp mark 7}{(7.469,1.292)}
\gppoint{gp mark 7}{(7.480,1.292)}
\gppoint{gp mark 7}{(7.491,1.292)}
\gppoint{gp mark 7}{(7.502,1.292)}
\gppoint{gp mark 7}{(7.513,1.292)}
\gppoint{gp mark 7}{(7.524,1.292)}
\gppoint{gp mark 7}{(7.535,1.292)}
\gppoint{gp mark 7}{(7.546,1.292)}
\gppoint{gp mark 7}{(7.557,1.292)}
\gppoint{gp mark 7}{(7.568,1.292)}
\gppoint{gp mark 7}{(7.579,1.292)}
\gppoint{gp mark 7}{(7.590,1.292)}
\gppoint{gp mark 7}{(7.601,1.292)}
\gppoint{gp mark 7}{(7.612,1.292)}
\gppoint{gp mark 7}{(7.623,1.292)}
\gppoint{gp mark 7}{(7.634,1.292)}
\gppoint{gp mark 7}{(7.645,1.292)}
\gppoint{gp mark 7}{(7.656,1.292)}
\gppoint{gp mark 7}{(7.667,1.292)}
\gppoint{gp mark 7}{(7.678,1.292)}
\gppoint{gp mark 7}{(7.689,1.292)}
\gppoint{gp mark 7}{(7.700,1.292)}
\gppoint{gp mark 7}{(7.711,1.292)}
\gppoint{gp mark 7}{(7.722,1.292)}
\gppoint{gp mark 7}{(7.733,1.292)}
\gppoint{gp mark 7}{(7.744,1.292)}
\gppoint{gp mark 7}{(7.755,1.292)}
\gppoint{gp mark 7}{(7.766,1.292)}
\gppoint{gp mark 7}{(7.777,1.292)}
\gppoint{gp mark 7}{(7.788,1.292)}
\gppoint{gp mark 7}{(7.799,1.292)}
\gppoint{gp mark 7}{(7.810,1.292)}
\gppoint{gp mark 7}{(7.821,1.292)}
\gppoint{gp mark 7}{(7.832,1.292)}
\gppoint{gp mark 7}{(7.843,1.292)}
\gppoint{gp mark 7}{(7.854,1.292)}
\gppoint{gp mark 7}{(7.865,1.292)}
\gppoint{gp mark 7}{(7.876,1.292)}
\gppoint{gp mark 7}{(7.887,1.292)}
\gppoint{gp mark 7}{(7.898,1.292)}
\gppoint{gp mark 7}{(7.909,1.292)}
\gppoint{gp mark 7}{(7.920,1.292)}
\gppoint{gp mark 7}{(7.931,1.292)}
\gppoint{gp mark 7}{(7.942,1.292)}
\gpcolor{rgb color={1.000,0.000,0.000}}
\gpsetpointsize{4.44}
\gppoint{gp mark 7}{(1.206,4.628)}
\gppoint{gp mark 7}{(1.217,4.623)}
\gppoint{gp mark 7}{(1.228,4.618)}
\gppoint{gp mark 7}{(1.239,4.613)}
\gppoint{gp mark 7}{(1.250,4.607)}
\gppoint{gp mark 7}{(1.261,4.602)}
\gppoint{gp mark 7}{(1.272,4.597)}
\gppoint{gp mark 7}{(1.283,4.591)}
\gppoint{gp mark 7}{(1.294,4.586)}
\gppoint{gp mark 7}{(1.305,4.581)}
\gppoint{gp mark 7}{(1.316,4.576)}
\gppoint{gp mark 7}{(1.327,4.570)}
\gppoint{gp mark 7}{(1.338,4.565)}
\gppoint{gp mark 7}{(1.349,4.560)}
\gppoint{gp mark 7}{(1.360,4.554)}
\gppoint{gp mark 7}{(1.371,4.549)}
\gppoint{gp mark 7}{(1.382,4.544)}
\gppoint{gp mark 7}{(1.393,4.539)}
\gppoint{gp mark 7}{(1.404,4.533)}
\gppoint{gp mark 7}{(1.415,4.528)}
\gppoint{gp mark 7}{(1.426,4.523)}
\gppoint{gp mark 7}{(1.437,4.517)}
\gppoint{gp mark 7}{(1.448,4.512)}
\gppoint{gp mark 7}{(1.459,4.507)}
\gppoint{gp mark 7}{(1.470,4.502)}
\gppoint{gp mark 7}{(1.481,4.496)}
\gppoint{gp mark 7}{(1.492,4.491)}
\gppoint{gp mark 7}{(1.503,4.486)}
\gppoint{gp mark 7}{(1.514,4.480)}
\gppoint{gp mark 7}{(1.525,4.475)}
\gppoint{gp mark 7}{(1.536,4.470)}
\gppoint{gp mark 7}{(1.547,4.464)}
\gppoint{gp mark 7}{(1.558,4.459)}
\gppoint{gp mark 7}{(1.568,4.454)}
\gppoint{gp mark 7}{(1.579,4.448)}
\gppoint{gp mark 7}{(1.590,4.443)}
\gppoint{gp mark 7}{(1.601,4.438)}
\gppoint{gp mark 7}{(1.612,4.433)}
\gppoint{gp mark 7}{(1.623,4.427)}
\gppoint{gp mark 7}{(1.634,4.422)}
\gppoint{gp mark 7}{(1.645,4.417)}
\gppoint{gp mark 7}{(1.656,4.411)}
\gppoint{gp mark 7}{(1.667,4.406)}
\gppoint{gp mark 7}{(1.678,4.401)}
\gppoint{gp mark 7}{(1.689,4.395)}
\gppoint{gp mark 7}{(1.700,4.390)}
\gppoint{gp mark 7}{(1.711,4.385)}
\gppoint{gp mark 7}{(1.722,4.379)}
\gppoint{gp mark 7}{(1.733,4.374)}
\gppoint{gp mark 7}{(1.744,4.369)}
\gppoint{gp mark 7}{(1.755,4.363)}
\gppoint{gp mark 7}{(1.766,4.358)}
\gppoint{gp mark 7}{(1.777,4.353)}
\gppoint{gp mark 7}{(1.788,4.347)}
\gppoint{gp mark 7}{(1.799,4.342)}
\gppoint{gp mark 7}{(1.810,4.337)}
\gppoint{gp mark 7}{(1.821,4.331)}
\gppoint{gp mark 7}{(1.832,4.326)}
\gppoint{gp mark 7}{(1.843,4.321)}
\gppoint{gp mark 7}{(1.854,4.315)}
\gppoint{gp mark 7}{(1.865,4.310)}
\gppoint{gp mark 7}{(1.876,4.305)}
\gppoint{gp mark 7}{(1.887,4.299)}
\gppoint{gp mark 7}{(1.898,4.294)}
\gppoint{gp mark 7}{(1.909,4.289)}
\gppoint{gp mark 7}{(1.920,4.283)}
\gppoint{gp mark 7}{(1.931,4.278)}
\gppoint{gp mark 7}{(1.942,4.273)}
\gppoint{gp mark 7}{(1.953,4.267)}
\gppoint{gp mark 7}{(1.964,4.262)}
\gppoint{gp mark 7}{(1.975,4.257)}
\gppoint{gp mark 7}{(1.986,4.251)}
\gppoint{gp mark 7}{(1.997,4.246)}
\gppoint{gp mark 7}{(2.008,4.241)}
\gppoint{gp mark 7}{(2.019,4.235)}
\gppoint{gp mark 7}{(2.030,4.229)}
\gppoint{gp mark 7}{(2.041,4.213)}
\gppoint{gp mark 7}{(2.052,4.208)}
\gppoint{gp mark 7}{(2.063,4.203)}
\gppoint{gp mark 7}{(2.074,4.197)}
\gppoint{gp mark 7}{(2.085,4.192)}
\gppoint{gp mark 7}{(2.096,4.187)}
\gppoint{gp mark 7}{(2.107,4.182)}
\gppoint{gp mark 7}{(2.118,4.176)}
\gppoint{gp mark 7}{(2.129,4.171)}
\gppoint{gp mark 7}{(2.140,4.166)}
\gppoint{gp mark 7}{(2.151,4.161)}
\gppoint{gp mark 7}{(2.162,4.155)}
\gppoint{gp mark 7}{(2.173,4.150)}
\gppoint{gp mark 7}{(2.184,4.145)}
\gppoint{gp mark 7}{(2.195,4.140)}
\gppoint{gp mark 7}{(2.206,4.135)}
\gppoint{gp mark 7}{(2.217,4.130)}
\gppoint{gp mark 7}{(2.228,4.124)}
\gppoint{gp mark 7}{(2.239,4.119)}
\gppoint{gp mark 7}{(2.250,4.114)}
\gppoint{gp mark 7}{(2.261,4.109)}
\gppoint{gp mark 7}{(2.272,4.104)}
\gppoint{gp mark 7}{(2.283,4.099)}
\gppoint{gp mark 7}{(2.294,4.094)}
\gppoint{gp mark 7}{(2.305,4.089)}
\gppoint{gp mark 7}{(2.316,4.084)}
\gppoint{gp mark 7}{(2.327,4.079)}
\gppoint{gp mark 7}{(2.338,4.074)}
\gppoint{gp mark 7}{(2.349,4.069)}
\gppoint{gp mark 7}{(2.360,4.064)}
\gppoint{gp mark 7}{(2.371,4.060)}
\gppoint{gp mark 7}{(2.382,4.055)}
\gppoint{gp mark 7}{(2.393,4.050)}
\gppoint{gp mark 7}{(2.404,4.045)}
\gppoint{gp mark 7}{(2.415,4.041)}
\gppoint{gp mark 7}{(2.426,4.036)}
\gppoint{gp mark 7}{(2.437,4.031)}
\gppoint{gp mark 7}{(2.448,4.027)}
\gppoint{gp mark 7}{(2.459,4.022)}
\gppoint{gp mark 7}{(2.470,4.018)}
\gppoint{gp mark 7}{(2.480,4.014)}
\gppoint{gp mark 7}{(2.491,4.011)}
\gppoint{gp mark 7}{(2.502,4.008)}
\gppoint{gp mark 7}{(2.513,4.006)}
\gppoint{gp mark 7}{(2.524,4.005)}
\gppoint{gp mark 7}{(2.535,4.005)}
\gppoint{gp mark 7}{(2.546,4.005)}
\gppoint{gp mark 7}{(2.557,4.005)}
\gppoint{gp mark 7}{(2.568,4.005)}
\gppoint{gp mark 7}{(2.579,4.006)}
\gppoint{gp mark 7}{(2.590,4.007)}
\gppoint{gp mark 7}{(2.601,4.008)}
\gppoint{gp mark 7}{(2.612,4.008)}
\gppoint{gp mark 7}{(2.623,4.008)}
\gppoint{gp mark 7}{(2.634,4.008)}
\gppoint{gp mark 7}{(2.645,4.008)}
\gppoint{gp mark 7}{(2.656,4.008)}
\gppoint{gp mark 7}{(2.667,4.008)}
\gppoint{gp mark 7}{(2.678,4.008)}
\gppoint{gp mark 7}{(2.689,4.008)}
\gppoint{gp mark 7}{(2.700,4.008)}
\gppoint{gp mark 7}{(2.711,4.008)}
\gppoint{gp mark 7}{(2.722,4.008)}
\gppoint{gp mark 7}{(2.733,4.008)}
\gppoint{gp mark 7}{(2.744,4.008)}
\gppoint{gp mark 7}{(2.755,4.008)}
\gppoint{gp mark 7}{(2.766,4.008)}
\gppoint{gp mark 7}{(2.777,4.008)}
\gppoint{gp mark 7}{(2.788,4.008)}
\gppoint{gp mark 7}{(2.799,4.008)}
\gppoint{gp mark 7}{(2.810,4.008)}
\gppoint{gp mark 7}{(2.821,4.008)}
\gppoint{gp mark 7}{(2.832,4.008)}
\gppoint{gp mark 7}{(2.843,4.008)}
\gppoint{gp mark 7}{(2.854,4.008)}
\gppoint{gp mark 7}{(2.865,4.008)}
\gppoint{gp mark 7}{(2.876,4.008)}
\gppoint{gp mark 7}{(2.887,4.008)}
\gppoint{gp mark 7}{(2.898,4.008)}
\gppoint{gp mark 7}{(2.909,4.008)}
\gppoint{gp mark 7}{(2.920,4.008)}
\gppoint{gp mark 7}{(2.931,4.008)}
\gppoint{gp mark 7}{(2.942,4.008)}
\gppoint{gp mark 7}{(2.953,4.008)}
\gppoint{gp mark 7}{(2.964,4.008)}
\gppoint{gp mark 7}{(2.975,4.008)}
\gppoint{gp mark 7}{(2.986,4.008)}
\gppoint{gp mark 7}{(2.997,4.008)}
\gppoint{gp mark 7}{(3.008,4.008)}
\gppoint{gp mark 7}{(3.019,4.008)}
\gppoint{gp mark 7}{(3.030,4.008)}
\gppoint{gp mark 7}{(3.041,4.008)}
\gppoint{gp mark 7}{(3.052,4.008)}
\gppoint{gp mark 7}{(3.063,4.008)}
\gppoint{gp mark 7}{(3.074,4.008)}
\gppoint{gp mark 7}{(3.085,4.008)}
\gppoint{gp mark 7}{(3.096,4.008)}
\gppoint{gp mark 7}{(3.107,4.008)}
\gppoint{gp mark 7}{(3.118,4.008)}
\gppoint{gp mark 7}{(3.129,4.008)}
\gppoint{gp mark 7}{(3.140,4.008)}
\gppoint{gp mark 7}{(3.151,4.008)}
\gppoint{gp mark 7}{(3.162,4.008)}
\gppoint{gp mark 7}{(3.173,4.008)}
\gppoint{gp mark 7}{(3.184,4.008)}
\gppoint{gp mark 7}{(3.195,4.008)}
\gppoint{gp mark 7}{(3.206,4.008)}
\gppoint{gp mark 7}{(3.217,4.008)}
\gppoint{gp mark 7}{(3.228,4.008)}
\gppoint{gp mark 7}{(3.239,4.008)}
\gppoint{gp mark 7}{(3.250,4.008)}
\gppoint{gp mark 7}{(3.261,4.008)}
\gppoint{gp mark 7}{(3.272,4.008)}
\gppoint{gp mark 7}{(3.283,4.008)}
\gppoint{gp mark 7}{(3.294,4.008)}
\gppoint{gp mark 7}{(3.305,4.008)}
\gppoint{gp mark 7}{(3.316,4.008)}
\gppoint{gp mark 7}{(3.327,4.008)}
\gppoint{gp mark 7}{(3.338,4.008)}
\gppoint{gp mark 7}{(3.349,4.008)}
\gppoint{gp mark 7}{(3.360,4.008)}
\gppoint{gp mark 7}{(3.371,4.008)}
\gppoint{gp mark 7}{(3.382,4.008)}
\gppoint{gp mark 7}{(3.392,4.008)}
\gppoint{gp mark 7}{(3.403,4.008)}
\gppoint{gp mark 7}{(3.414,4.008)}
\gppoint{gp mark 7}{(3.425,4.008)}
\gppoint{gp mark 7}{(3.436,4.008)}
\gppoint{gp mark 7}{(3.447,4.008)}
\gppoint{gp mark 7}{(3.458,4.008)}
\gppoint{gp mark 7}{(3.469,4.008)}
\gppoint{gp mark 7}{(3.480,4.008)}
\gppoint{gp mark 7}{(3.491,4.008)}
\gppoint{gp mark 7}{(3.502,4.006)}
\gppoint{gp mark 7}{(3.513,3.988)}
\gppoint{gp mark 7}{(3.524,3.989)}
\gppoint{gp mark 7}{(3.535,4.000)}
\gppoint{gp mark 7}{(3.546,1.029)}
\gppoint{gp mark 7}{(3.557,1.029)}
\gppoint{gp mark 7}{(3.568,1.074)}
\gppoint{gp mark 7}{(3.579,1.052)}
\gppoint{gp mark 7}{(3.590,1.039)}
\gppoint{gp mark 7}{(3.601,1.044)}
\gppoint{gp mark 7}{(3.612,1.052)}
\gppoint{gp mark 7}{(3.623,1.051)}
\gppoint{gp mark 7}{(3.634,1.047)}
\gppoint{gp mark 7}{(3.645,1.047)}
\gppoint{gp mark 7}{(3.656,1.049)}
\gppoint{gp mark 7}{(3.667,1.049)}
\gppoint{gp mark 7}{(3.678,1.049)}
\gppoint{gp mark 7}{(3.689,1.048)}
\gppoint{gp mark 7}{(3.700,1.048)}
\gppoint{gp mark 7}{(3.711,1.048)}
\gppoint{gp mark 7}{(3.722,1.048)}
\gppoint{gp mark 7}{(3.733,1.048)}
\gppoint{gp mark 7}{(3.744,1.048)}
\gppoint{gp mark 7}{(3.755,1.048)}
\gppoint{gp mark 7}{(3.766,1.048)}
\gppoint{gp mark 7}{(3.777,1.048)}
\gppoint{gp mark 7}{(3.788,1.048)}
\gppoint{gp mark 7}{(3.799,1.049)}
\gppoint{gp mark 7}{(3.810,1.049)}
\gppoint{gp mark 7}{(3.821,1.049)}
\gppoint{gp mark 7}{(3.832,1.049)}
\gppoint{gp mark 7}{(3.843,1.048)}
\gppoint{gp mark 7}{(3.854,1.048)}
\gppoint{gp mark 7}{(3.865,1.048)}
\gppoint{gp mark 7}{(3.876,1.048)}
\gppoint{gp mark 7}{(3.887,1.048)}
\gppoint{gp mark 7}{(3.898,1.048)}
\gppoint{gp mark 7}{(3.909,1.048)}
\gppoint{gp mark 7}{(3.920,1.048)}
\gppoint{gp mark 7}{(3.931,1.048)}
\gppoint{gp mark 7}{(3.942,1.048)}
\gppoint{gp mark 7}{(3.953,1.048)}
\gppoint{gp mark 7}{(3.964,1.048)}
\gppoint{gp mark 7}{(3.975,1.048)}
\gppoint{gp mark 7}{(3.986,1.048)}
\gppoint{gp mark 7}{(3.997,1.049)}
\gppoint{gp mark 7}{(4.008,1.049)}
\gppoint{gp mark 7}{(4.019,1.049)}
\gppoint{gp mark 7}{(4.030,1.048)}
\gppoint{gp mark 7}{(4.041,1.048)}
\gppoint{gp mark 7}{(4.052,1.048)}
\gppoint{gp mark 7}{(4.063,1.048)}
\gppoint{gp mark 7}{(4.074,1.049)}
\gppoint{gp mark 7}{(4.085,1.049)}
\gppoint{gp mark 7}{(4.096,1.048)}
\gppoint{gp mark 7}{(4.107,1.048)}
\gppoint{gp mark 7}{(4.118,1.048)}
\gppoint{gp mark 7}{(4.129,1.048)}
\gppoint{gp mark 7}{(4.140,1.048)}
\gppoint{gp mark 7}{(4.151,1.048)}
\gppoint{gp mark 7}{(4.162,1.048)}
\gppoint{gp mark 7}{(4.173,1.049)}
\gppoint{gp mark 7}{(4.184,1.049)}
\gppoint{gp mark 7}{(4.195,1.048)}
\gppoint{gp mark 7}{(4.206,1.048)}
\gppoint{gp mark 7}{(4.217,1.048)}
\gppoint{gp mark 7}{(4.228,1.048)}
\gppoint{gp mark 7}{(4.239,1.048)}
\gppoint{gp mark 7}{(4.250,1.049)}
\gppoint{gp mark 7}{(4.261,1.049)}
\gppoint{gp mark 7}{(4.272,1.049)}
\gppoint{gp mark 7}{(4.283,1.048)}
\gppoint{gp mark 7}{(4.294,1.049)}
\gppoint{gp mark 7}{(4.304,1.064)}
\gppoint{gp mark 7}{(4.315,1.135)}
\gppoint{gp mark 7}{(4.326,1.257)}
\gppoint{gp mark 7}{(4.337,1.288)}
\gppoint{gp mark 7}{(4.348,1.292)}
\gppoint{gp mark 7}{(4.359,1.292)}
\gppoint{gp mark 7}{(4.370,1.291)}
\gppoint{gp mark 7}{(4.381,1.291)}
\gppoint{gp mark 7}{(4.392,1.292)}
\gppoint{gp mark 7}{(4.403,1.292)}
\gppoint{gp mark 7}{(4.414,1.292)}
\gppoint{gp mark 7}{(4.425,1.292)}
\gppoint{gp mark 7}{(4.436,1.292)}
\gppoint{gp mark 7}{(4.447,1.292)}
\gppoint{gp mark 7}{(4.458,1.292)}
\gppoint{gp mark 7}{(4.469,1.292)}
\gppoint{gp mark 7}{(4.480,1.292)}
\gppoint{gp mark 7}{(4.491,1.292)}
\gppoint{gp mark 7}{(4.502,1.292)}
\gppoint{gp mark 7}{(4.513,1.292)}
\gppoint{gp mark 7}{(4.524,1.292)}
\gppoint{gp mark 7}{(4.535,1.292)}
\gppoint{gp mark 7}{(4.546,1.292)}
\gppoint{gp mark 7}{(4.557,1.292)}
\gppoint{gp mark 7}{(4.568,1.292)}
\gppoint{gp mark 7}{(4.579,1.292)}
\gppoint{gp mark 7}{(4.590,1.292)}
\gppoint{gp mark 7}{(4.601,1.292)}
\gppoint{gp mark 7}{(4.612,1.292)}
\gppoint{gp mark 7}{(4.623,1.292)}
\gppoint{gp mark 7}{(4.634,1.291)}
\gppoint{gp mark 7}{(4.645,1.291)}
\gppoint{gp mark 7}{(4.656,1.292)}
\gppoint{gp mark 7}{(4.667,1.292)}
\gppoint{gp mark 7}{(4.678,1.292)}
\gppoint{gp mark 7}{(4.689,1.292)}
\gppoint{gp mark 7}{(4.700,1.292)}
\gppoint{gp mark 7}{(4.711,1.292)}
\gppoint{gp mark 7}{(4.722,1.292)}
\gppoint{gp mark 7}{(4.733,1.292)}
\gppoint{gp mark 7}{(4.744,1.292)}
\gppoint{gp mark 7}{(4.755,1.292)}
\gppoint{gp mark 7}{(4.766,1.292)}
\gppoint{gp mark 7}{(4.777,1.292)}
\gppoint{gp mark 7}{(4.788,1.292)}
\gppoint{gp mark 7}{(4.799,1.292)}
\gppoint{gp mark 7}{(4.810,1.292)}
\gppoint{gp mark 7}{(4.821,1.292)}
\gppoint{gp mark 7}{(4.832,1.292)}
\gppoint{gp mark 7}{(4.843,1.292)}
\gppoint{gp mark 7}{(4.854,1.292)}
\gppoint{gp mark 7}{(4.865,1.292)}
\gppoint{gp mark 7}{(4.876,1.292)}
\gppoint{gp mark 7}{(4.887,1.292)}
\gppoint{gp mark 7}{(4.898,1.292)}
\gppoint{gp mark 7}{(4.909,1.292)}
\gppoint{gp mark 7}{(4.920,1.292)}
\gppoint{gp mark 7}{(4.931,1.292)}
\gppoint{gp mark 7}{(4.942,1.292)}
\gppoint{gp mark 7}{(4.953,1.292)}
\gppoint{gp mark 7}{(4.964,1.292)}
\gppoint{gp mark 7}{(4.975,1.292)}
\gppoint{gp mark 7}{(4.986,1.292)}
\gppoint{gp mark 7}{(4.997,1.292)}
\gppoint{gp mark 7}{(5.008,1.292)}
\gppoint{gp mark 7}{(5.019,1.292)}
\gppoint{gp mark 7}{(5.030,1.292)}
\gppoint{gp mark 7}{(5.041,1.292)}
\gppoint{gp mark 7}{(5.052,1.293)}
\gppoint{gp mark 7}{(5.063,1.293)}
\gppoint{gp mark 7}{(5.074,1.293)}
\gppoint{gp mark 7}{(5.085,1.293)}
\gppoint{gp mark 7}{(5.096,1.293)}
\gppoint{gp mark 7}{(5.107,1.293)}
\gppoint{gp mark 7}{(5.118,1.293)}
\gppoint{gp mark 7}{(5.129,1.293)}
\gppoint{gp mark 7}{(5.140,1.293)}
\gppoint{gp mark 7}{(5.151,1.293)}
\gppoint{gp mark 7}{(5.162,1.293)}
\gppoint{gp mark 7}{(5.173,1.293)}
\gppoint{gp mark 7}{(5.184,1.293)}
\gppoint{gp mark 7}{(5.195,1.293)}
\gppoint{gp mark 7}{(5.206,1.293)}
\gppoint{gp mark 7}{(5.216,1.293)}
\gppoint{gp mark 7}{(5.227,1.293)}
\gppoint{gp mark 7}{(5.238,1.293)}
\gppoint{gp mark 7}{(5.249,1.293)}
\gppoint{gp mark 7}{(5.260,1.293)}
\gppoint{gp mark 7}{(5.271,1.293)}
\gppoint{gp mark 7}{(5.282,1.293)}
\gppoint{gp mark 7}{(5.293,1.293)}
\gppoint{gp mark 7}{(5.304,1.293)}
\gppoint{gp mark 7}{(5.315,1.293)}
\gppoint{gp mark 7}{(5.326,1.293)}
\gppoint{gp mark 7}{(5.337,1.293)}
\gppoint{gp mark 7}{(5.348,1.293)}
\gppoint{gp mark 7}{(5.359,1.293)}
\gppoint{gp mark 7}{(5.370,1.293)}
\gppoint{gp mark 7}{(5.381,1.293)}
\gppoint{gp mark 7}{(5.392,1.293)}
\gppoint{gp mark 7}{(5.403,1.293)}
\gppoint{gp mark 7}{(5.414,1.293)}
\gppoint{gp mark 7}{(5.425,1.293)}
\gppoint{gp mark 7}{(5.436,1.293)}
\gppoint{gp mark 7}{(5.447,1.293)}
\gppoint{gp mark 7}{(5.458,1.293)}
\gppoint{gp mark 7}{(5.469,1.293)}
\gppoint{gp mark 7}{(5.480,1.293)}
\gppoint{gp mark 7}{(5.491,1.293)}
\gppoint{gp mark 7}{(5.502,1.293)}
\gppoint{gp mark 7}{(5.513,1.293)}
\gppoint{gp mark 7}{(5.524,1.293)}
\gppoint{gp mark 7}{(5.535,1.293)}
\gppoint{gp mark 7}{(5.546,1.293)}
\gppoint{gp mark 7}{(5.557,1.293)}
\gppoint{gp mark 7}{(5.568,1.293)}
\gppoint{gp mark 7}{(5.579,1.293)}
\gppoint{gp mark 7}{(5.590,1.293)}
\gppoint{gp mark 7}{(5.601,1.293)}
\gppoint{gp mark 7}{(5.612,1.293)}
\gppoint{gp mark 7}{(5.623,1.293)}
\gppoint{gp mark 7}{(5.634,1.293)}
\gppoint{gp mark 7}{(5.645,1.293)}
\gppoint{gp mark 7}{(5.656,1.293)}
\gppoint{gp mark 7}{(5.667,1.293)}
\gppoint{gp mark 7}{(5.678,1.293)}
\gppoint{gp mark 7}{(5.689,1.293)}
\gppoint{gp mark 7}{(5.700,1.293)}
\gppoint{gp mark 7}{(5.711,1.293)}
\gppoint{gp mark 7}{(5.722,1.293)}
\gppoint{gp mark 7}{(5.733,1.293)}
\gppoint{gp mark 7}{(5.744,1.293)}
\gppoint{gp mark 7}{(5.755,1.293)}
\gppoint{gp mark 7}{(5.766,1.293)}
\gppoint{gp mark 7}{(5.777,1.293)}
\gppoint{gp mark 7}{(5.788,1.293)}
\gppoint{gp mark 7}{(5.799,1.293)}
\gppoint{gp mark 7}{(5.810,1.293)}
\gppoint{gp mark 7}{(5.821,1.293)}
\gppoint{gp mark 7}{(5.832,1.293)}
\gppoint{gp mark 7}{(5.843,1.293)}
\gppoint{gp mark 7}{(5.854,1.293)}
\gppoint{gp mark 7}{(5.865,1.293)}
\gppoint{gp mark 7}{(5.876,1.293)}
\gppoint{gp mark 7}{(5.887,1.293)}
\gppoint{gp mark 7}{(5.898,1.293)}
\gppoint{gp mark 7}{(5.909,1.293)}
\gppoint{gp mark 7}{(5.920,1.293)}
\gppoint{gp mark 7}{(5.931,1.293)}
\gppoint{gp mark 7}{(5.942,1.293)}
\gppoint{gp mark 7}{(5.953,1.293)}
\gppoint{gp mark 7}{(5.964,1.293)}
\gppoint{gp mark 7}{(5.975,1.293)}
\gppoint{gp mark 7}{(5.986,1.293)}
\gppoint{gp mark 7}{(5.997,1.293)}
\gppoint{gp mark 7}{(6.008,1.293)}
\gppoint{gp mark 7}{(6.019,1.293)}
\gppoint{gp mark 7}{(6.030,1.293)}
\gppoint{gp mark 7}{(6.041,1.293)}
\gppoint{gp mark 7}{(6.052,1.293)}
\gppoint{gp mark 7}{(6.063,1.293)}
\gppoint{gp mark 7}{(6.074,1.293)}
\gppoint{gp mark 7}{(6.085,1.293)}
\gppoint{gp mark 7}{(6.096,1.293)}
\gppoint{gp mark 7}{(6.107,1.293)}
\gppoint{gp mark 7}{(6.118,1.293)}
\gppoint{gp mark 7}{(6.128,1.293)}
\gppoint{gp mark 7}{(6.139,1.293)}
\gppoint{gp mark 7}{(6.150,1.293)}
\gppoint{gp mark 7}{(6.161,1.293)}
\gppoint{gp mark 7}{(6.172,1.293)}
\gppoint{gp mark 7}{(6.183,1.293)}
\gppoint{gp mark 7}{(6.194,1.293)}
\gppoint{gp mark 7}{(6.205,1.293)}
\gppoint{gp mark 7}{(6.216,1.293)}
\gppoint{gp mark 7}{(6.227,1.293)}
\gppoint{gp mark 7}{(6.238,1.293)}
\gppoint{gp mark 7}{(6.249,1.293)}
\gppoint{gp mark 7}{(6.260,1.293)}
\gppoint{gp mark 7}{(6.271,1.293)}
\gppoint{gp mark 7}{(6.282,1.293)}
\gppoint{gp mark 7}{(6.293,1.293)}
\gppoint{gp mark 7}{(6.304,1.293)}
\gppoint{gp mark 7}{(6.315,1.293)}
\gppoint{gp mark 7}{(6.326,1.293)}
\gppoint{gp mark 7}{(6.337,1.293)}
\gppoint{gp mark 7}{(6.348,1.293)}
\gppoint{gp mark 7}{(6.359,1.293)}
\gppoint{gp mark 7}{(6.370,1.293)}
\gppoint{gp mark 7}{(6.381,1.293)}
\gppoint{gp mark 7}{(6.392,1.293)}
\gppoint{gp mark 7}{(6.403,1.293)}
\gppoint{gp mark 7}{(6.414,1.293)}
\gppoint{gp mark 7}{(6.425,1.293)}
\gppoint{gp mark 7}{(6.436,1.293)}
\gppoint{gp mark 7}{(6.447,1.293)}
\gppoint{gp mark 7}{(6.458,1.293)}
\gppoint{gp mark 7}{(6.469,1.293)}
\gppoint{gp mark 7}{(6.480,1.293)}
\gppoint{gp mark 7}{(6.491,1.293)}
\gppoint{gp mark 7}{(6.502,1.293)}
\gppoint{gp mark 7}{(6.513,1.293)}
\gppoint{gp mark 7}{(6.524,1.293)}
\gppoint{gp mark 7}{(6.535,1.293)}
\gppoint{gp mark 7}{(6.546,1.293)}
\gppoint{gp mark 7}{(6.557,1.293)}
\gppoint{gp mark 7}{(6.568,1.293)}
\gppoint{gp mark 7}{(6.579,1.293)}
\gppoint{gp mark 7}{(6.590,1.293)}
\gppoint{gp mark 7}{(6.601,1.293)}
\gppoint{gp mark 7}{(6.612,1.293)}
\gppoint{gp mark 7}{(6.623,1.293)}
\gppoint{gp mark 7}{(6.634,1.293)}
\gppoint{gp mark 7}{(6.645,1.293)}
\gppoint{gp mark 7}{(6.656,1.293)}
\gppoint{gp mark 7}{(6.667,1.293)}
\gppoint{gp mark 7}{(6.678,1.293)}
\gppoint{gp mark 7}{(6.689,1.293)}
\gppoint{gp mark 7}{(6.700,1.293)}
\gppoint{gp mark 7}{(6.711,1.293)}
\gppoint{gp mark 7}{(6.722,1.293)}
\gppoint{gp mark 7}{(6.733,1.293)}
\gppoint{gp mark 7}{(6.744,1.293)}
\gppoint{gp mark 7}{(6.755,1.293)}
\gppoint{gp mark 7}{(6.766,1.293)}
\gppoint{gp mark 7}{(6.777,1.293)}
\gppoint{gp mark 7}{(6.788,1.293)}
\gppoint{gp mark 7}{(6.799,1.293)}
\gppoint{gp mark 7}{(6.810,1.293)}
\gppoint{gp mark 7}{(6.821,1.293)}
\gppoint{gp mark 7}{(6.832,1.293)}
\gppoint{gp mark 7}{(6.843,1.293)}
\gppoint{gp mark 7}{(6.854,1.293)}
\gppoint{gp mark 7}{(6.865,1.293)}
\gppoint{gp mark 7}{(6.876,1.293)}
\gppoint{gp mark 7}{(6.887,1.293)}
\gppoint{gp mark 7}{(6.898,1.293)}
\gppoint{gp mark 7}{(6.909,1.293)}
\gppoint{gp mark 7}{(6.920,1.293)}
\gppoint{gp mark 7}{(6.931,1.293)}
\gppoint{gp mark 7}{(6.942,1.293)}
\gppoint{gp mark 7}{(6.953,1.293)}
\gppoint{gp mark 7}{(6.964,1.293)}
\gppoint{gp mark 7}{(6.975,1.293)}
\gppoint{gp mark 7}{(6.986,1.293)}
\gppoint{gp mark 7}{(6.997,1.293)}
\gppoint{gp mark 7}{(7.008,1.293)}
\gppoint{gp mark 7}{(7.019,1.293)}
\gppoint{gp mark 7}{(7.030,1.293)}
\gppoint{gp mark 7}{(7.040,1.293)}
\gppoint{gp mark 7}{(7.051,1.293)}
\gppoint{gp mark 7}{(7.062,1.293)}
\gppoint{gp mark 7}{(7.073,1.293)}
\gppoint{gp mark 7}{(7.084,1.293)}
\gppoint{gp mark 7}{(7.095,1.293)}
\gppoint{gp mark 7}{(7.106,1.293)}
\gppoint{gp mark 7}{(7.117,1.293)}
\gppoint{gp mark 7}{(7.128,1.293)}
\gppoint{gp mark 7}{(7.139,1.293)}
\gppoint{gp mark 7}{(7.150,1.293)}
\gppoint{gp mark 7}{(7.161,1.293)}
\gppoint{gp mark 7}{(7.172,1.293)}
\gppoint{gp mark 7}{(7.183,1.293)}
\gppoint{gp mark 7}{(7.194,1.293)}
\gppoint{gp mark 7}{(7.205,1.293)}
\gppoint{gp mark 7}{(7.216,1.293)}
\gppoint{gp mark 7}{(7.227,1.293)}
\gppoint{gp mark 7}{(7.238,1.293)}
\gppoint{gp mark 7}{(7.249,1.293)}
\gppoint{gp mark 7}{(7.260,1.293)}
\gppoint{gp mark 7}{(7.271,1.293)}
\gppoint{gp mark 7}{(7.282,1.293)}
\gppoint{gp mark 7}{(7.293,1.293)}
\gppoint{gp mark 7}{(7.304,1.293)}
\gppoint{gp mark 7}{(7.315,1.293)}
\gppoint{gp mark 7}{(7.326,1.293)}
\gppoint{gp mark 7}{(7.337,1.293)}
\gppoint{gp mark 7}{(7.348,1.293)}
\gppoint{gp mark 7}{(7.359,1.293)}
\gppoint{gp mark 7}{(7.370,1.293)}
\gppoint{gp mark 7}{(7.381,1.293)}
\gppoint{gp mark 7}{(7.392,1.293)}
\gppoint{gp mark 7}{(7.403,1.293)}
\gppoint{gp mark 7}{(7.414,1.293)}
\gppoint{gp mark 7}{(7.425,1.293)}
\gppoint{gp mark 7}{(7.436,1.293)}
\gppoint{gp mark 7}{(7.447,1.293)}
\gppoint{gp mark 7}{(7.458,1.293)}
\gppoint{gp mark 7}{(7.469,1.293)}
\gppoint{gp mark 7}{(7.480,1.293)}
\gppoint{gp mark 7}{(7.491,1.293)}
\gppoint{gp mark 7}{(7.502,1.293)}
\gppoint{gp mark 7}{(7.513,1.293)}
\gppoint{gp mark 7}{(7.524,1.293)}
\gppoint{gp mark 7}{(7.535,1.293)}
\gppoint{gp mark 7}{(7.546,1.293)}
\gppoint{gp mark 7}{(7.557,1.293)}
\gppoint{gp mark 7}{(7.568,1.293)}
\gppoint{gp mark 7}{(7.579,1.293)}
\gppoint{gp mark 7}{(7.590,1.293)}
\gppoint{gp mark 7}{(7.601,1.293)}
\gppoint{gp mark 7}{(7.612,1.293)}
\gppoint{gp mark 7}{(7.623,1.293)}
\gppoint{gp mark 7}{(7.634,1.293)}
\gppoint{gp mark 7}{(7.645,1.293)}
\gppoint{gp mark 7}{(7.656,1.293)}
\gppoint{gp mark 7}{(7.667,1.293)}
\gppoint{gp mark 7}{(7.678,1.293)}
\gppoint{gp mark 7}{(7.689,1.293)}
\gppoint{gp mark 7}{(7.700,1.293)}
\gppoint{gp mark 7}{(7.711,1.293)}
\gppoint{gp mark 7}{(7.722,1.293)}
\gppoint{gp mark 7}{(7.733,1.293)}
\gppoint{gp mark 7}{(7.744,1.293)}
\gppoint{gp mark 7}{(7.755,1.293)}
\gppoint{gp mark 7}{(7.766,1.293)}
\gppoint{gp mark 7}{(7.777,1.293)}
\gppoint{gp mark 7}{(7.788,1.293)}
\gppoint{gp mark 7}{(7.799,1.293)}
\gppoint{gp mark 7}{(7.810,1.293)}
\gppoint{gp mark 7}{(7.821,1.293)}
\gppoint{gp mark 7}{(7.832,1.293)}
\gppoint{gp mark 7}{(7.843,1.293)}
\gppoint{gp mark 7}{(7.854,1.293)}
\gppoint{gp mark 7}{(7.865,1.293)}
\gppoint{gp mark 7}{(7.876,1.293)}
\gppoint{gp mark 7}{(7.887,1.293)}
\gppoint{gp mark 7}{(7.898,1.293)}
\gppoint{gp mark 7}{(7.909,1.293)}
\gppoint{gp mark 7}{(7.920,1.293)}
\gppoint{gp mark 7}{(7.931,1.293)}
\gppoint{gp mark 7}{(7.942,1.293)}
\gpcolor{rgb color={0.000,0.000,0.000}}
\gpsetlinetype{gp lt plot 0}
\gpsetlinewidth{4.00}
\draw[gp path] (2.427,4.020)--(3.533,4.020);
\draw[gp path] (3.533,1.048)--(4.326,1.048);
\draw[gp path] (4.326,1.293)--(7.510,1.293);
\draw[gp path] (7.510,1.293)--(7.947,1.293);
\draw[gp path] (1.200,4.618)--(1.206,4.615)--(1.212,4.612)--(1.218,4.609)--(1.224,4.606)%
  --(1.230,4.603)--(1.236,4.600)--(1.242,4.597)--(1.248,4.594)--(1.254,4.591)--(1.260,4.588)%
  --(1.267,4.585)--(1.273,4.582)--(1.279,4.579)--(1.285,4.576)--(1.291,4.573)--(1.297,4.570)%
  --(1.303,4.567)--(1.309,4.564)--(1.315,4.562)--(1.321,4.559)--(1.327,4.556)--(1.333,4.553)%
  --(1.339,4.550)--(1.346,4.547)--(1.352,4.544)--(1.358,4.541)--(1.364,4.538)--(1.370,4.535)%
  --(1.376,4.532)--(1.382,4.529)--(1.388,4.526)--(1.394,4.523)--(1.400,4.520)--(1.406,4.517)%
  --(1.412,4.514)--(1.418,4.511)--(1.425,4.508)--(1.431,4.505)--(1.437,4.502)--(1.443,4.499)%
  --(1.449,4.496)--(1.455,4.493)--(1.461,4.490)--(1.467,4.488)--(1.473,4.485)--(1.479,4.482)%
  --(1.485,4.479)--(1.491,4.476)--(1.497,4.473)--(1.504,4.470)--(1.510,4.467)--(1.516,4.464)%
  --(1.522,4.461)--(1.528,4.458)--(1.534,4.455)--(1.540,4.452)--(1.546,4.449)--(1.552,4.446)%
  --(1.558,4.443)--(1.564,4.440)--(1.570,4.437)--(1.576,4.434)--(1.583,4.431)--(1.589,4.428)%
  --(1.595,4.425)--(1.601,4.422)--(1.607,4.419)--(1.613,4.416)--(1.619,4.413)--(1.625,4.411)%
  --(1.631,4.408)--(1.637,4.405)--(1.643,4.402)--(1.649,4.399)--(1.656,4.396)--(1.662,4.393)%
  --(1.668,4.390)--(1.674,4.387)--(1.680,4.384)--(1.686,4.381)--(1.692,4.378)--(1.698,4.375)%
  --(1.704,4.372)--(1.710,4.369)--(1.716,4.366)--(1.722,4.363)--(1.728,4.360)--(1.735,4.357)%
  --(1.741,4.354)--(1.747,4.351)--(1.753,4.348)--(1.759,4.345)--(1.765,4.342)--(1.771,4.339)%
  --(1.777,4.337)--(1.783,4.334)--(1.789,4.331)--(1.795,4.328)--(1.801,4.325)--(1.807,4.322)%
  --(1.814,4.319)--(1.820,4.316)--(1.826,4.313)--(1.832,4.310)--(1.838,4.307)--(1.844,4.304)%
  --(1.850,4.301)--(1.856,4.298)--(1.862,4.295)--(1.868,4.292)--(1.874,4.289)--(1.880,4.286)%
  --(1.886,4.283)--(1.893,4.280)--(1.899,4.277)--(1.905,4.274)--(1.911,4.271)--(1.917,4.268)%
  --(1.923,4.265)--(1.929,4.262)--(1.935,4.260)--(1.941,4.257)--(1.947,4.254)--(1.953,4.251)%
  --(1.959,4.248)--(1.965,4.245)--(1.972,4.242)--(1.978,4.239)--(1.984,4.236)--(1.990,4.233)%
  --(1.996,4.230)--(2.002,4.227)--(2.008,4.224)--(2.014,4.221)--(2.020,4.218)--(2.026,4.215)%
  --(2.032,4.212)--(2.038,4.209)--(2.044,4.206)--(2.051,4.203)--(2.057,4.200)--(2.063,4.197)%
  --(2.069,4.194)--(2.075,4.191)--(2.081,4.188)--(2.087,4.186)--(2.093,4.183)--(2.099,4.180)%
  --(2.105,4.177)--(2.111,4.174)--(2.117,4.171)--(2.124,4.168)--(2.130,4.165)--(2.136,4.162)%
  --(2.142,4.159)--(2.148,4.156)--(2.154,4.153)--(2.160,4.150)--(2.166,4.147)--(2.172,4.144)%
  --(2.178,4.141)--(2.184,4.138)--(2.190,4.135)--(2.196,4.132)--(2.203,4.129)--(2.209,4.126)%
  --(2.215,4.123)--(2.221,4.120)--(2.227,4.117)--(2.233,4.114)--(2.239,4.111)--(2.245,4.109)%
  --(2.251,4.106)--(2.257,4.103)--(2.263,4.100)--(2.269,4.097)--(2.275,4.094)--(2.282,4.091)%
  --(2.288,4.088)--(2.294,4.085)--(2.300,4.082)--(2.306,4.079)--(2.312,4.076)--(2.318,4.073)%
  --(2.324,4.070)--(2.330,4.067)--(2.336,4.064)--(2.342,4.061)--(2.348,4.058)--(2.354,4.055)%
  --(2.361,4.052)--(2.367,4.049)--(2.373,4.046)--(2.379,4.043)--(2.385,4.040)--(2.391,4.037)%
  --(2.397,4.035)--(2.403,4.032)--(2.409,4.029)--(2.415,4.026)--(2.421,4.023)--(2.427,4.020);
\draw[gp path] (3.533,4.020)--(3.533,1.048);
\draw[gp path] (4.326,1.048)--(4.326,1.293);
\draw[gp path] (3.896,3.695)--(4.572,3.695);
\gpcolor{rgb color={1.000,0.000,0.000}}
\gpsetlinewidth{0.50}
\gppoint{gp mark 7}{(4.234,2.921)}
\gpcolor{rgb color={0.502,0.502,0.502}}
\gppoint{gp mark 7}{(4.234,2.147)}
\gpcolor{rgb color={0.000,0.000,0.000}}
\node[gp node left,font={\fontsize{10pt}{12pt}\selectfont}] at (1.421,5.166) {\LARGE $B_y$};
\node[gp node left,font={\fontsize{10pt}{12pt}\selectfont}] at (6.147,5.166) {\large $\alpha = \pi$};
\node[gp node left,font={\fontsize{10pt}{12pt}\selectfont}] at (4.797,3.695) {\large exact};
\node[gp node left,font={\fontsize{10pt}{12pt}\selectfont}] at (4.797,2.921) {\large HLLD-CWM};
\node[gp node left,font={\fontsize{10pt}{12pt}\selectfont}] at (4.797,2.147) {\large HLLD};
%% coordinates of the plot area
\gpdefrectangularnode{gp plot 1}{\pgfpoint{1.196cm}{0.985cm}}{\pgfpoint{7.947cm}{5.631cm}}
\end{tikzpicture}
%% gnuplot variables
} \\
\end{tabular}
\caption{The rotational discontinuity and slow shock solution found using HLLD-CWM after the (optional) flux redistribution step, HLLD, and the exact solver using $2048$ grid points for (top) a near-coplanar and pseudo-converging case and (bottom) the planar and non-converging (bottom) case.}
\label{fig:coplanar_ab_crsol}
\end{figure}

Solutions obtained using HLLD-CWM, without any additional artificial viscosity, and HLLD for a near-coplanar case and the coplanar case of Test~5 with $A = 0.1$ for 512 grid points are shown in Figure~\ref{fig:coplanar_ab_crsol_512}, and for 2048 grid points are shown in Figure~\ref{fig:coplanar_ab_crsol}.  The solutions now include a left-going rotational discontinuity and slow shock.  This is the correct solution for ideal MHD.  Pseudo-convergence does not occur in the coplanar case -- successive give refinement only reduces the error w.r.t. the r-solution.  For the near-coplanar case, pseudo-convergence occurs -- the compound wave structure is lost as the grid is refined from 1024 to 2048 points. 

In the unmodified solution for the near-coplanar case of Test~5, the state between the left-going slow shock and the contact discontinuity slowly converges to the exact solution, but at different rates.  The state should be constant throughout the region, but directly behind the left-going slow shock, the solution differs from the exact by $\approx 0.7\%$ and the difference behind the contact discontinuity is $\approx 2\%$.  For the coplanar case, the solution in this region remains constant, but differs from the exact solution by $\approx 3\%$.  These issues are eliminated with HLLD-CWM.  The appearance of the rotational discontinuity and slow shock is independent of the value of $\alpha$ and the regular structures are present and have correct values for the state variables in both cases.

%-----------------------------------------------------------------
% Fast coplanar flux zoomed (512)
%-----------------------------------------------------------------
\begin{figure}[htbp] 
\begin{tabular}{cc}
\resizebox{0.5\linewidth}{!}{\tikzsetnextfilename{fast_coplanar_b_crsol_00512_1}\begin{tikzpicture}[gnuplot]
%% generated with GNUPLOT 4.6p4 (Lua 5.1; terminal rev. 99, script rev. 100)
%% Fri 22 Aug 2014 11:36:07 AM EDT
\path (0.000,0.000) rectangle (8.500,6.000);
\gpfill{rgb color={1.000,1.000,1.000}} (1.196,0.985)--(7.946,0.985)--(7.946,5.630)--(1.196,5.630)--cycle;
\gpcolor{color=gp lt color border}
\gpsetlinetype{gp lt border}
\gpsetlinewidth{1.00}
\draw[gp path] (1.196,0.985)--(1.196,5.630)--(7.946,5.630)--(7.946,0.985)--cycle;
\gpcolor{color=gp lt color axes}
\gpsetlinetype{gp lt axes}
\gpsetlinewidth{2.00}
\draw[gp path] (1.196,1.275)--(7.947,1.275);
\gpcolor{color=gp lt color border}
\gpsetlinetype{gp lt border}
\draw[gp path] (1.196,1.275)--(1.268,1.275);
\draw[gp path] (7.947,1.275)--(7.875,1.275);
\gpcolor{rgb color={0.000,0.000,0.000}}
\node[gp node right,font={\fontsize{10pt}{12pt}\selectfont}] at (1.012,1.275) {0.65};
\gpcolor{color=gp lt color axes}
\gpsetlinetype{gp lt axes}
\draw[gp path] (1.196,2.001)--(7.947,2.001);
\gpcolor{color=gp lt color border}
\gpsetlinetype{gp lt border}
\draw[gp path] (1.196,2.001)--(1.268,2.001);
\draw[gp path] (7.947,2.001)--(7.875,2.001);
\gpcolor{rgb color={0.000,0.000,0.000}}
\node[gp node right,font={\fontsize{10pt}{12pt}\selectfont}] at (1.012,2.001) {0.7};
\gpcolor{color=gp lt color axes}
\gpsetlinetype{gp lt axes}
\draw[gp path] (1.196,2.727)--(7.947,2.727);
\gpcolor{color=gp lt color border}
\gpsetlinetype{gp lt border}
\draw[gp path] (1.196,2.727)--(1.268,2.727);
\draw[gp path] (7.947,2.727)--(7.875,2.727);
\gpcolor{rgb color={0.000,0.000,0.000}}
\node[gp node right,font={\fontsize{10pt}{12pt}\selectfont}] at (1.012,2.727) {0.75};
\gpcolor{color=gp lt color axes}
\gpsetlinetype{gp lt axes}
\draw[gp path] (1.196,3.453)--(7.947,3.453);
\gpcolor{color=gp lt color border}
\gpsetlinetype{gp lt border}
\draw[gp path] (1.196,3.453)--(1.268,3.453);
\draw[gp path] (7.947,3.453)--(7.875,3.453);
\gpcolor{rgb color={0.000,0.000,0.000}}
\node[gp node right,font={\fontsize{10pt}{12pt}\selectfont}] at (1.012,3.453) {0.8};
\gpcolor{color=gp lt color axes}
\gpsetlinetype{gp lt axes}
\draw[gp path] (1.196,4.179)--(7.947,4.179);
\gpcolor{color=gp lt color border}
\gpsetlinetype{gp lt border}
\draw[gp path] (1.196,4.179)--(1.268,4.179);
\draw[gp path] (7.947,4.179)--(7.875,4.179);
\gpcolor{rgb color={0.000,0.000,0.000}}
\node[gp node right,font={\fontsize{10pt}{12pt}\selectfont}] at (1.012,4.179) {0.85};
\gpcolor{color=gp lt color axes}
\gpsetlinetype{gp lt axes}
\draw[gp path] (1.196,4.905)--(7.947,4.905);
\gpcolor{color=gp lt color border}
\gpsetlinetype{gp lt border}
\draw[gp path] (1.196,4.905)--(1.268,4.905);
\draw[gp path] (7.947,4.905)--(7.875,4.905);
\gpcolor{rgb color={0.000,0.000,0.000}}
\node[gp node right,font={\fontsize{10pt}{12pt}\selectfont}] at (1.012,4.905) {0.9};
\gpcolor{color=gp lt color axes}
\gpsetlinetype{gp lt axes}
\draw[gp path] (1.196,5.631)--(7.947,5.631);
\gpcolor{color=gp lt color border}
\gpsetlinetype{gp lt border}
\draw[gp path] (1.196,5.631)--(1.268,5.631);
\draw[gp path] (7.947,5.631)--(7.875,5.631);
\gpcolor{rgb color={0.000,0.000,0.000}}
\node[gp node right,font={\fontsize{10pt}{12pt}\selectfont}] at (1.012,5.631) {0.95};
\gpcolor{color=gp lt color axes}
\gpsetlinetype{gp lt axes}
\draw[gp path] (1.196,0.985)--(1.196,5.631);
\gpcolor{color=gp lt color border}
\gpsetlinetype{gp lt border}
\draw[gp path] (1.196,0.985)--(1.196,1.057);
\draw[gp path] (1.196,5.631)--(1.196,5.559);
\gpcolor{rgb color={0.000,0.000,0.000}}
\node[gp node center,font={\fontsize{10pt}{12pt}\selectfont}] at (1.196,0.677) {0.3};
\gpcolor{color=gp lt color axes}
\gpsetlinetype{gp lt axes}
\draw[gp path] (2.494,0.985)--(2.494,5.631);
\gpcolor{color=gp lt color border}
\gpsetlinetype{gp lt border}
\draw[gp path] (2.494,0.985)--(2.494,1.057);
\draw[gp path] (2.494,5.631)--(2.494,5.559);
\gpcolor{rgb color={0.000,0.000,0.000}}
\node[gp node center,font={\fontsize{10pt}{12pt}\selectfont}] at (2.494,0.677) {0.35};
\gpcolor{color=gp lt color axes}
\gpsetlinetype{gp lt axes}
\draw[gp path] (3.793,0.985)--(3.793,5.631);
\gpcolor{color=gp lt color border}
\gpsetlinetype{gp lt border}
\draw[gp path] (3.793,0.985)--(3.793,1.057);
\draw[gp path] (3.793,5.631)--(3.793,5.559);
\gpcolor{rgb color={0.000,0.000,0.000}}
\node[gp node center,font={\fontsize{10pt}{12pt}\selectfont}] at (3.793,0.677) {0.4};
\gpcolor{color=gp lt color axes}
\gpsetlinetype{gp lt axes}
\draw[gp path] (5.091,0.985)--(5.091,5.631);
\gpcolor{color=gp lt color border}
\gpsetlinetype{gp lt border}
\draw[gp path] (5.091,0.985)--(5.091,1.057);
\draw[gp path] (5.091,5.631)--(5.091,5.559);
\gpcolor{rgb color={0.000,0.000,0.000}}
\node[gp node center,font={\fontsize{10pt}{12pt}\selectfont}] at (5.091,0.677) {0.45};
\gpcolor{color=gp lt color axes}
\gpsetlinetype{gp lt axes}
\draw[gp path] (6.389,0.985)--(6.389,5.631);
\gpcolor{color=gp lt color border}
\gpsetlinetype{gp lt border}
\draw[gp path] (6.389,0.985)--(6.389,1.057);
\draw[gp path] (6.389,5.631)--(6.389,5.559);
\gpcolor{rgb color={0.000,0.000,0.000}}
\node[gp node center,font={\fontsize{10pt}{12pt}\selectfont}] at (6.389,0.677) {0.5};
\gpcolor{color=gp lt color axes}
\gpsetlinetype{gp lt axes}
\draw[gp path] (7.687,0.985)--(7.687,5.631);
\gpcolor{color=gp lt color border}
\gpsetlinetype{gp lt border}
\draw[gp path] (7.687,0.985)--(7.687,1.057);
\draw[gp path] (7.687,5.631)--(7.687,5.559);
\gpcolor{rgb color={0.000,0.000,0.000}}
\node[gp node center,font={\fontsize{10pt}{12pt}\selectfont}] at (7.687,0.677) {0.55};
\gpcolor{color=gp lt color border}
\draw[gp path] (1.196,5.631)--(1.196,0.985)--(7.947,0.985)--(7.947,5.631)--cycle;
\gpcolor{rgb color={0.000,0.000,0.000}}
\node[gp node center,font={\fontsize{10pt}{12pt}\selectfont}] at (4.571,0.215) {\large $x$};
\gpcolor{rgb color={0.502,0.502,0.502}}
\gpsetlinewidth{0.50}
\gpsetpointsize{2.67}
\gppoint{gp mark 7}{(1.242,3.594)}
\gppoint{gp mark 7}{(1.292,3.532)}
\gppoint{gp mark 7}{(1.343,3.471)}
\gppoint{gp mark 7}{(1.394,3.409)}
\gppoint{gp mark 7}{(1.444,3.348)}
\gppoint{gp mark 7}{(1.495,3.287)}
\gppoint{gp mark 7}{(1.546,3.226)}
\gppoint{gp mark 7}{(1.597,3.166)}
\gppoint{gp mark 7}{(1.647,3.105)}
\gppoint{gp mark 7}{(1.698,3.045)}
\gppoint{gp mark 7}{(1.749,2.984)}
\gppoint{gp mark 7}{(1.799,2.924)}
\gppoint{gp mark 7}{(1.850,2.864)}
\gppoint{gp mark 7}{(1.901,2.805)}
\gppoint{gp mark 7}{(1.952,2.745)}
\gppoint{gp mark 7}{(2.002,2.685)}
\gppoint{gp mark 7}{(2.053,2.626)}
\gppoint{gp mark 7}{(2.104,2.566)}
\gppoint{gp mark 7}{(2.154,2.507)}
\gppoint{gp mark 7}{(2.205,2.447)}
\gppoint{gp mark 7}{(2.256,2.388)}
\gppoint{gp mark 7}{(2.307,2.329)}
\gppoint{gp mark 7}{(2.357,2.269)}
\gppoint{gp mark 7}{(2.408,2.210)}
\gppoint{gp mark 7}{(2.459,2.151)}
\gppoint{gp mark 7}{(2.509,2.091)}
\gppoint{gp mark 7}{(2.560,2.031)}
\gppoint{gp mark 7}{(2.611,1.971)}
\gppoint{gp mark 7}{(2.662,1.911)}
\gppoint{gp mark 7}{(2.712,1.849)}
\gppoint{gp mark 7}{(2.763,1.782)}
\gppoint{gp mark 7}{(2.814,1.693)}
\gppoint{gp mark 7}{(2.864,1.603)}
\gppoint{gp mark 7}{(2.915,1.643)}
\gppoint{gp mark 7}{(2.966,2.105)}
\gppoint{gp mark 7}{(3.017,2.414)}
\gppoint{gp mark 7}{(3.067,2.413)}
\gppoint{gp mark 7}{(3.118,2.392)}
\gppoint{gp mark 7}{(3.169,2.393)}
\gppoint{gp mark 7}{(3.219,2.398)}
\gppoint{gp mark 7}{(3.270,2.400)}
\gppoint{gp mark 7}{(3.321,2.402)}
\gppoint{gp mark 7}{(3.372,2.404)}
\gppoint{gp mark 7}{(3.422,2.406)}
\gppoint{gp mark 7}{(3.473,2.407)}
\gppoint{gp mark 7}{(3.524,2.409)}
\gppoint{gp mark 7}{(3.574,2.432)}
\gppoint{gp mark 7}{(3.625,2.673)}
\gppoint{gp mark 7}{(3.676,3.706)}
\gppoint{gp mark 7}{(3.727,4.554)}
\gppoint{gp mark 7}{(3.777,4.633)}
\gppoint{gp mark 7}{(3.828,4.639)}
\gppoint{gp mark 7}{(3.879,4.638)}
\gppoint{gp mark 7}{(3.929,4.636)}
\gppoint{gp mark 7}{(3.980,4.636)}
\gppoint{gp mark 7}{(4.031,4.635)}
\gppoint{gp mark 7}{(4.082,4.635)}
\gppoint{gp mark 7}{(4.132,4.635)}
\gppoint{gp mark 7}{(4.183,4.635)}
\gppoint{gp mark 7}{(4.234,4.634)}
\gppoint{gp mark 7}{(4.284,4.634)}
\gppoint{gp mark 7}{(4.335,4.634)}
\gppoint{gp mark 7}{(4.386,4.635)}
\gppoint{gp mark 7}{(4.437,4.635)}
\gppoint{gp mark 7}{(4.487,4.634)}
\gppoint{gp mark 7}{(4.538,4.634)}
\gppoint{gp mark 7}{(4.589,4.634)}
\gppoint{gp mark 7}{(4.639,4.634)}
\gppoint{gp mark 7}{(4.690,4.635)}
\gppoint{gp mark 7}{(4.741,4.635)}
\gppoint{gp mark 7}{(4.792,4.635)}
\gppoint{gp mark 7}{(4.842,4.635)}
\gppoint{gp mark 7}{(4.893,4.635)}
\gppoint{gp mark 7}{(4.944,4.635)}
\gppoint{gp mark 7}{(4.994,4.635)}
\gppoint{gp mark 7}{(5.045,4.635)}
\gppoint{gp mark 7}{(5.096,4.635)}
\gppoint{gp mark 7}{(5.147,4.635)}
\gppoint{gp mark 7}{(5.197,4.635)}
\gppoint{gp mark 7}{(5.248,4.636)}
\gppoint{gp mark 7}{(5.299,4.636)}
\gppoint{gp mark 7}{(5.349,4.636)}
\gppoint{gp mark 7}{(5.400,4.636)}
\gppoint{gp mark 7}{(5.451,4.636)}
\gppoint{gp mark 7}{(5.502,4.636)}
\gppoint{gp mark 7}{(5.552,4.636)}
\gppoint{gp mark 7}{(5.603,4.637)}
\gppoint{gp mark 7}{(5.654,4.637)}
\gppoint{gp mark 7}{(5.704,4.637)}
\gppoint{gp mark 7}{(5.755,4.638)}
\gppoint{gp mark 7}{(5.806,4.639)}
\gppoint{gp mark 7}{(5.857,4.641)}
\gppoint{gp mark 7}{(5.907,4.642)}
\gppoint{gp mark 7}{(5.958,4.643)}
\gppoint{gp mark 7}{(6.009,4.644)}
\gppoint{gp mark 7}{(6.059,4.645)}
\gppoint{gp mark 7}{(6.110,4.646)}
\gppoint{gp mark 7}{(6.161,4.646)}
\gppoint{gp mark 7}{(6.212,4.646)}
\gppoint{gp mark 7}{(6.262,4.646)}
\gppoint{gp mark 7}{(6.313,4.646)}
\gppoint{gp mark 7}{(6.364,4.645)}
\gppoint{gp mark 7}{(6.414,4.644)}
\gppoint{gp mark 7}{(6.465,4.643)}
\gppoint{gp mark 7}{(6.516,4.641)}
\gppoint{gp mark 7}{(6.567,4.638)}
\gppoint{gp mark 7}{(6.617,4.634)}
\gppoint{gp mark 7}{(6.668,4.629)}
\gppoint{gp mark 7}{(6.719,4.623)}
\gppoint{gp mark 7}{(6.769,4.617)}
\gppoint{gp mark 7}{(6.820,4.610)}
\gppoint{gp mark 7}{(6.871,4.600)}
\gppoint{gp mark 7}{(6.922,4.590)}
\gppoint{gp mark 7}{(6.972,4.583)}
\gppoint{gp mark 7}{(7.023,4.577)}
\gppoint{gp mark 7}{(7.074,4.569)}
\gppoint{gp mark 7}{(7.124,4.562)}
\gppoint{gp mark 7}{(7.175,4.558)}
\gppoint{gp mark 7}{(7.226,4.553)}
\gppoint{gp mark 7}{(7.277,4.545)}
\gppoint{gp mark 7}{(7.327,4.530)}
\gppoint{gp mark 7}{(7.378,4.519)}
\gppoint{gp mark 7}{(7.429,4.515)}
\gppoint{gp mark 7}{(7.479,4.512)}
\gppoint{gp mark 7}{(7.530,4.501)}
\gppoint{gp mark 7}{(7.581,4.424)}
\gppoint{gp mark 7}{(7.632,3.928)}
\gppoint{gp mark 7}{(7.682,2.823)}
\gppoint{gp mark 7}{(7.733,1.841)}
\gppoint{gp mark 7}{(7.784,1.375)}
\gppoint{gp mark 7}{(7.834,1.225)}
\gppoint{gp mark 7}{(7.885,1.179)}
\gpcolor{rgb color={1.000,0.000,0.000}}
\gpsetpointsize{4.44}
\gppoint{gp mark 7}{(1.242,3.603)}
\gppoint{gp mark 7}{(1.292,3.542)}
\gppoint{gp mark 7}{(1.343,3.481)}
\gppoint{gp mark 7}{(1.394,3.420)}
\gppoint{gp mark 7}{(1.444,3.360)}
\gppoint{gp mark 7}{(1.495,3.300)}
\gppoint{gp mark 7}{(1.546,3.240)}
\gppoint{gp mark 7}{(1.597,3.181)}
\gppoint{gp mark 7}{(1.647,3.122)}
\gppoint{gp mark 7}{(1.698,3.063)}
\gppoint{gp mark 7}{(1.749,3.004)}
\gppoint{gp mark 7}{(1.799,2.946)}
\gppoint{gp mark 7}{(1.850,2.888)}
\gppoint{gp mark 7}{(1.901,2.831)}
\gppoint{gp mark 7}{(1.952,2.774)}
\gppoint{gp mark 7}{(2.002,2.717)}
\gppoint{gp mark 7}{(2.053,2.662)}
\gppoint{gp mark 7}{(2.104,2.607)}
\gppoint{gp mark 7}{(2.154,2.552)}
\gppoint{gp mark 7}{(2.205,2.499)}
\gppoint{gp mark 7}{(2.256,2.447)}
\gppoint{gp mark 7}{(2.307,2.397)}
\gppoint{gp mark 7}{(2.357,2.350)}
\gppoint{gp mark 7}{(2.408,2.308)}
\gppoint{gp mark 7}{(2.459,2.274)}
\gppoint{gp mark 7}{(2.509,2.250)}
\gppoint{gp mark 7}{(2.560,2.235)}
\gppoint{gp mark 7}{(2.611,2.229)}
\gppoint{gp mark 7}{(2.662,2.227)}
\gppoint{gp mark 7}{(2.712,2.228)}
\gppoint{gp mark 7}{(2.763,2.246)}
\gppoint{gp mark 7}{(2.814,2.364)}
\gppoint{gp mark 7}{(2.864,2.304)}
\gppoint{gp mark 7}{(2.915,2.219)}
\gppoint{gp mark 7}{(2.966,2.255)}
\gppoint{gp mark 7}{(3.017,2.222)}
\gppoint{gp mark 7}{(3.067,2.218)}
\gppoint{gp mark 7}{(3.118,2.233)}
\gppoint{gp mark 7}{(3.169,2.233)}
\gppoint{gp mark 7}{(3.219,2.233)}
\gppoint{gp mark 7}{(3.270,2.235)}
\gppoint{gp mark 7}{(3.321,2.235)}
\gppoint{gp mark 7}{(3.372,2.232)}
\gppoint{gp mark 7}{(3.422,2.226)}
\gppoint{gp mark 7}{(3.473,2.222)}
\gppoint{gp mark 7}{(3.524,2.215)}
\gppoint{gp mark 7}{(3.574,2.203)}
\gppoint{gp mark 7}{(3.625,2.198)}
\gppoint{gp mark 7}{(3.676,2.207)}
\gppoint{gp mark 7}{(3.727,2.360)}
\gppoint{gp mark 7}{(3.777,3.329)}
\gppoint{gp mark 7}{(3.828,4.518)}
\gppoint{gp mark 7}{(3.879,4.675)}
\gppoint{gp mark 7}{(3.929,4.678)}
\gppoint{gp mark 7}{(3.980,4.672)}
\gppoint{gp mark 7}{(4.031,4.670)}
\gppoint{gp mark 7}{(4.082,4.671)}
\gppoint{gp mark 7}{(4.132,4.671)}
\gppoint{gp mark 7}{(4.183,4.670)}
\gppoint{gp mark 7}{(4.234,4.672)}
\gppoint{gp mark 7}{(4.284,4.677)}
\gppoint{gp mark 7}{(4.335,4.679)}
\gppoint{gp mark 7}{(4.386,4.677)}
\gppoint{gp mark 7}{(4.437,4.673)}
\gppoint{gp mark 7}{(4.487,4.671)}
\gppoint{gp mark 7}{(4.538,4.670)}
\gppoint{gp mark 7}{(4.589,4.669)}
\gppoint{gp mark 7}{(4.639,4.669)}
\gppoint{gp mark 7}{(4.690,4.670)}
\gppoint{gp mark 7}{(4.741,4.673)}
\gppoint{gp mark 7}{(4.792,4.678)}
\gppoint{gp mark 7}{(4.842,4.681)}
\gppoint{gp mark 7}{(4.893,4.681)}
\gppoint{gp mark 7}{(4.944,4.679)}
\gppoint{gp mark 7}{(4.994,4.677)}
\gppoint{gp mark 7}{(5.045,4.676)}
\gppoint{gp mark 7}{(5.096,4.676)}
\gppoint{gp mark 7}{(5.147,4.676)}
\gppoint{gp mark 7}{(5.197,4.678)}
\gppoint{gp mark 7}{(5.248,4.683)}
\gppoint{gp mark 7}{(5.299,4.685)}
\gppoint{gp mark 7}{(5.349,4.685)}
\gppoint{gp mark 7}{(5.400,4.683)}
\gppoint{gp mark 7}{(5.451,4.680)}
\gppoint{gp mark 7}{(5.502,4.678)}
\gppoint{gp mark 7}{(5.552,4.678)}
\gppoint{gp mark 7}{(5.603,4.679)}
\gppoint{gp mark 7}{(5.654,4.680)}
\gppoint{gp mark 7}{(5.704,4.683)}
\gppoint{gp mark 7}{(5.755,4.687)}
\gppoint{gp mark 7}{(5.806,4.691)}
\gppoint{gp mark 7}{(5.857,4.693)}
\gppoint{gp mark 7}{(5.907,4.693)}
\gppoint{gp mark 7}{(5.958,4.693)}
\gppoint{gp mark 7}{(6.009,4.693)}
\gppoint{gp mark 7}{(6.059,4.693)}
\gppoint{gp mark 7}{(6.110,4.694)}
\gppoint{gp mark 7}{(6.161,4.697)}
\gppoint{gp mark 7}{(6.212,4.704)}
\gppoint{gp mark 7}{(6.262,4.710)}
\gppoint{gp mark 7}{(6.313,4.712)}
\gppoint{gp mark 7}{(6.364,4.712)}
\gppoint{gp mark 7}{(6.414,4.712)}
\gppoint{gp mark 7}{(6.465,4.710)}
\gppoint{gp mark 7}{(6.516,4.707)}
\gppoint{gp mark 7}{(6.567,4.706)}
\gppoint{gp mark 7}{(6.617,4.706)}
\gppoint{gp mark 7}{(6.668,4.707)}
\gppoint{gp mark 7}{(6.719,4.708)}
\gppoint{gp mark 7}{(6.769,4.708)}
\gppoint{gp mark 7}{(6.820,4.707)}
\gppoint{gp mark 7}{(6.871,4.702)}
\gppoint{gp mark 7}{(6.922,4.694)}
\gppoint{gp mark 7}{(6.972,4.688)}
\gppoint{gp mark 7}{(7.023,4.682)}
\gppoint{gp mark 7}{(7.074,4.669)}
\gppoint{gp mark 7}{(7.124,4.648)}
\gppoint{gp mark 7}{(7.175,4.627)}
\gppoint{gp mark 7}{(7.226,4.611)}
\gppoint{gp mark 7}{(7.277,4.599)}
\gppoint{gp mark 7}{(7.327,4.591)}
\gppoint{gp mark 7}{(7.378,4.587)}
\gppoint{gp mark 7}{(7.429,4.585)}
\gppoint{gp mark 7}{(7.479,4.582)}
\gppoint{gp mark 7}{(7.530,4.568)}
\gppoint{gp mark 7}{(7.581,4.473)}
\gppoint{gp mark 7}{(7.632,3.897)}
\gppoint{gp mark 7}{(7.682,2.741)}
\gppoint{gp mark 7}{(7.733,1.789)}
\gppoint{gp mark 7}{(7.784,1.359)}
\gppoint{gp mark 7}{(7.834,1.225)}
\gppoint{gp mark 7}{(7.885,1.181)}
\gpcolor{rgb color={0.000,0.000,0.000}}
\gpsetlinetype{gp lt plot 0}
\gpsetlinewidth{4.00}
\draw[gp path] (2.404,2.243)--(2.886,2.243);
\draw[gp path] (2.886,2.243)--(3.811,2.243);
\draw[gp path] (3.811,4.679)--(7.722,4.679);
\draw[gp path] (7.722,1.152)--(7.947,1.152);
\draw[gp path] (1.207,3.449)--(1.220,3.434)--(1.233,3.418)--(1.246,3.403)--(1.259,3.388)%
  --(1.271,3.373)--(1.284,3.358)--(1.297,3.343)--(1.310,3.328)--(1.323,3.313)--(1.336,3.298)%
  --(1.349,3.283)--(1.361,3.268)--(1.374,3.253)--(1.387,3.238)--(1.400,3.224)--(1.413,3.209)%
  --(1.426,3.194)--(1.439,3.180)--(1.452,3.165)--(1.464,3.151)--(1.477,3.136)--(1.490,3.122)%
  --(1.503,3.108)--(1.516,3.093)--(1.529,3.079)--(1.542,3.065)--(1.555,3.051)--(1.567,3.036)%
  --(1.580,3.022)--(1.593,3.008)--(1.606,2.994)--(1.619,2.980)--(1.632,2.967)--(1.645,2.953)%
  --(1.657,2.939)--(1.670,2.925)--(1.683,2.911)--(1.696,2.898)--(1.709,2.884)--(1.722,2.871)%
  --(1.735,2.857)--(1.748,2.844)--(1.760,2.830)--(1.773,2.817)--(1.786,2.803)--(1.799,2.790)%
  --(1.812,2.777)--(1.825,2.764)--(1.838,2.751)--(1.850,2.738)--(1.863,2.725)--(1.876,2.712)%
  --(1.889,2.699)--(1.902,2.686)--(1.915,2.673)--(1.928,2.660)--(1.941,2.647)--(1.953,2.635)%
  --(1.966,2.622)--(1.979,2.610)--(1.992,2.597)--(2.005,2.585)--(2.018,2.572)--(2.031,2.560)%
  --(2.044,2.548)--(2.056,2.535)--(2.069,2.523)--(2.082,2.511)--(2.095,2.499)--(2.108,2.487)%
  --(2.121,2.475)--(2.134,2.463)--(2.146,2.451)--(2.159,2.439)--(2.172,2.428)--(2.185,2.416)%
  --(2.198,2.404)--(2.211,2.393)--(2.224,2.381)--(2.237,2.370)--(2.249,2.358)--(2.262,2.347)%
  --(2.275,2.336)--(2.288,2.325)--(2.301,2.313)--(2.314,2.302)--(2.327,2.291)--(2.340,2.280)%
  --(2.352,2.269)--(2.365,2.258)--(2.378,2.248)--(2.391,2.237)--(2.404,2.243);
\draw[gp path] (3.811,2.243)--(3.811,4.679);
\draw[gp path] (7.722,4.679)--(7.722,1.152);
\node[gp node left,font={\fontsize{10pt}{12pt}\selectfont}] at (1.456,5.268) {\LARGE $\rho$};
\node[gp node left,font={\fontsize{10pt}{12pt}\selectfont}] at (5.740,5.268) {\large $\alpha = 3.0$};
%% coordinates of the plot area
\gpdefrectangularnode{gp plot 1}{\pgfpoint{1.196cm}{0.985cm}}{\pgfpoint{7.947cm}{5.631cm}}
\end{tikzpicture}
%% gnuplot variables
} & 
\resizebox{0.5\linewidth}{!}{\tikzsetnextfilename{fast_coplanar_b_crsol_00512_6}\begin{tikzpicture}[gnuplot]
%% generated with GNUPLOT 4.6p4 (Lua 5.1; terminal rev. 99, script rev. 100)
%% Fri 22 Aug 2014 11:36:07 AM EDT
\path (0.000,0.000) rectangle (8.500,6.000);
\gpfill{rgb color={1.000,1.000,1.000}} (1.196,0.985)--(7.946,0.985)--(7.946,5.630)--(1.196,5.630)--cycle;
\gpcolor{color=gp lt color border}
\gpsetlinetype{gp lt border}
\gpsetlinewidth{1.00}
\draw[gp path] (1.196,0.985)--(1.196,5.630)--(7.946,5.630)--(7.946,0.985)--cycle;
\gpcolor{color=gp lt color axes}
\gpsetlinetype{gp lt axes}
\gpsetlinewidth{2.00}
\draw[gp path] (1.196,0.985)--(7.947,0.985);
\gpcolor{color=gp lt color border}
\gpsetlinetype{gp lt border}
\draw[gp path] (1.196,0.985)--(1.268,0.985);
\draw[gp path] (7.947,0.985)--(7.875,0.985);
\gpcolor{rgb color={0.000,0.000,0.000}}
\node[gp node right,font={\fontsize{10pt}{12pt}\selectfont}] at (1.012,0.985) {-0.4};
\gpcolor{color=gp lt color axes}
\gpsetlinetype{gp lt axes}
\draw[gp path] (1.196,1.759)--(7.947,1.759);
\gpcolor{color=gp lt color border}
\gpsetlinetype{gp lt border}
\draw[gp path] (1.196,1.759)--(1.268,1.759);
\draw[gp path] (7.947,1.759)--(7.875,1.759);
\gpcolor{rgb color={0.000,0.000,0.000}}
\node[gp node right,font={\fontsize{10pt}{12pt}\selectfont}] at (1.012,1.759) {-0.2};
\gpcolor{color=gp lt color axes}
\gpsetlinetype{gp lt axes}
\draw[gp path] (1.196,2.534)--(7.947,2.534);
\gpcolor{color=gp lt color border}
\gpsetlinetype{gp lt border}
\draw[gp path] (1.196,2.534)--(1.268,2.534);
\draw[gp path] (7.947,2.534)--(7.875,2.534);
\gpcolor{rgb color={0.000,0.000,0.000}}
\node[gp node right,font={\fontsize{10pt}{12pt}\selectfont}] at (1.012,2.534) {0};
\gpcolor{color=gp lt color axes}
\gpsetlinetype{gp lt axes}
\draw[gp path] (1.196,3.308)--(7.947,3.308);
\gpcolor{color=gp lt color border}
\gpsetlinetype{gp lt border}
\draw[gp path] (1.196,3.308)--(1.268,3.308);
\draw[gp path] (7.947,3.308)--(7.875,3.308);
\gpcolor{rgb color={0.000,0.000,0.000}}
\node[gp node right,font={\fontsize{10pt}{12pt}\selectfont}] at (1.012,3.308) {0.2};
\gpcolor{color=gp lt color axes}
\gpsetlinetype{gp lt axes}
\draw[gp path] (1.196,4.082)--(7.947,4.082);
\gpcolor{color=gp lt color border}
\gpsetlinetype{gp lt border}
\draw[gp path] (1.196,4.082)--(1.268,4.082);
\draw[gp path] (7.947,4.082)--(7.875,4.082);
\gpcolor{rgb color={0.000,0.000,0.000}}
\node[gp node right,font={\fontsize{10pt}{12pt}\selectfont}] at (1.012,4.082) {0.4};
\gpcolor{color=gp lt color axes}
\gpsetlinetype{gp lt axes}
\draw[gp path] (1.196,4.857)--(7.947,4.857);
\gpcolor{color=gp lt color border}
\gpsetlinetype{gp lt border}
\draw[gp path] (1.196,4.857)--(1.268,4.857);
\draw[gp path] (7.947,4.857)--(7.875,4.857);
\gpcolor{rgb color={0.000,0.000,0.000}}
\node[gp node right,font={\fontsize{10pt}{12pt}\selectfont}] at (1.012,4.857) {0.6};
\gpcolor{color=gp lt color axes}
\gpsetlinetype{gp lt axes}
\draw[gp path] (1.196,5.631)--(7.947,5.631);
\gpcolor{color=gp lt color border}
\gpsetlinetype{gp lt border}
\draw[gp path] (1.196,5.631)--(1.268,5.631);
\draw[gp path] (7.947,5.631)--(7.875,5.631);
\gpcolor{rgb color={0.000,0.000,0.000}}
\node[gp node right,font={\fontsize{10pt}{12pt}\selectfont}] at (1.012,5.631) {0.8};
\gpcolor{color=gp lt color axes}
\gpsetlinetype{gp lt axes}
\draw[gp path] (1.196,0.985)--(1.196,5.631);
\gpcolor{color=gp lt color border}
\gpsetlinetype{gp lt border}
\draw[gp path] (1.196,0.985)--(1.196,1.057);
\draw[gp path] (1.196,5.631)--(1.196,5.559);
\gpcolor{rgb color={0.000,0.000,0.000}}
\node[gp node center,font={\fontsize{10pt}{12pt}\selectfont}] at (1.196,0.677) {0.3};
\gpcolor{color=gp lt color axes}
\gpsetlinetype{gp lt axes}
\draw[gp path] (2.494,0.985)--(2.494,5.631);
\gpcolor{color=gp lt color border}
\gpsetlinetype{gp lt border}
\draw[gp path] (2.494,0.985)--(2.494,1.057);
\draw[gp path] (2.494,5.631)--(2.494,5.559);
\gpcolor{rgb color={0.000,0.000,0.000}}
\node[gp node center,font={\fontsize{10pt}{12pt}\selectfont}] at (2.494,0.677) {0.35};
\gpcolor{color=gp lt color axes}
\gpsetlinetype{gp lt axes}
\draw[gp path] (3.793,0.985)--(3.793,5.631);
\gpcolor{color=gp lt color border}
\gpsetlinetype{gp lt border}
\draw[gp path] (3.793,0.985)--(3.793,1.057);
\draw[gp path] (3.793,5.631)--(3.793,5.559);
\gpcolor{rgb color={0.000,0.000,0.000}}
\node[gp node center,font={\fontsize{10pt}{12pt}\selectfont}] at (3.793,0.677) {0.4};
\gpcolor{color=gp lt color axes}
\gpsetlinetype{gp lt axes}
\draw[gp path] (5.091,0.985)--(5.091,5.631);
\gpcolor{color=gp lt color border}
\gpsetlinetype{gp lt border}
\draw[gp path] (5.091,0.985)--(5.091,1.057);
\draw[gp path] (5.091,5.631)--(5.091,5.559);
\gpcolor{rgb color={0.000,0.000,0.000}}
\node[gp node center,font={\fontsize{10pt}{12pt}\selectfont}] at (5.091,0.677) {0.45};
\gpcolor{color=gp lt color axes}
\gpsetlinetype{gp lt axes}
\draw[gp path] (6.389,0.985)--(6.389,5.631);
\gpcolor{color=gp lt color border}
\gpsetlinetype{gp lt border}
\draw[gp path] (6.389,0.985)--(6.389,1.057);
\draw[gp path] (6.389,5.631)--(6.389,5.559);
\gpcolor{rgb color={0.000,0.000,0.000}}
\node[gp node center,font={\fontsize{10pt}{12pt}\selectfont}] at (6.389,0.677) {0.5};
\gpcolor{color=gp lt color axes}
\gpsetlinetype{gp lt axes}
\draw[gp path] (7.687,0.985)--(7.687,5.631);
\gpcolor{color=gp lt color border}
\gpsetlinetype{gp lt border}
\draw[gp path] (7.687,0.985)--(7.687,1.057);
\draw[gp path] (7.687,5.631)--(7.687,5.559);
\gpcolor{rgb color={0.000,0.000,0.000}}
\node[gp node center,font={\fontsize{10pt}{12pt}\selectfont}] at (7.687,0.677) {0.55};
\gpcolor{color=gp lt color border}
\draw[gp path] (1.196,5.631)--(1.196,0.985)--(7.947,0.985)--(7.947,5.631)--cycle;
\gpcolor{rgb color={0.000,0.000,0.000}}
\node[gp node center,font={\fontsize{10pt}{12pt}\selectfont}] at (4.571,0.215) {\large $x$};
\gpcolor{rgb color={0.502,0.502,0.502}}
\gpsetlinewidth{0.50}
\gpsetpointsize{2.67}
\gppoint{gp mark 7}{(1.242,4.926)}
\gppoint{gp mark 7}{(1.292,4.888)}
\gppoint{gp mark 7}{(1.343,4.849)}
\gppoint{gp mark 7}{(1.394,4.811)}
\gppoint{gp mark 7}{(1.444,4.771)}
\gppoint{gp mark 7}{(1.495,4.732)}
\gppoint{gp mark 7}{(1.546,4.692)}
\gppoint{gp mark 7}{(1.597,4.652)}
\gppoint{gp mark 7}{(1.647,4.611)}
\gppoint{gp mark 7}{(1.698,4.570)}
\gppoint{gp mark 7}{(1.749,4.528)}
\gppoint{gp mark 7}{(1.799,4.486)}
\gppoint{gp mark 7}{(1.850,4.444)}
\gppoint{gp mark 7}{(1.901,4.400)}
\gppoint{gp mark 7}{(1.952,4.357)}
\gppoint{gp mark 7}{(2.002,4.312)}
\gppoint{gp mark 7}{(2.053,4.267)}
\gppoint{gp mark 7}{(2.104,4.220)}
\gppoint{gp mark 7}{(2.154,4.173)}
\gppoint{gp mark 7}{(2.205,4.125)}
\gppoint{gp mark 7}{(2.256,4.076)}
\gppoint{gp mark 7}{(2.307,4.025)}
\gppoint{gp mark 7}{(2.357,3.973)}
\gppoint{gp mark 7}{(2.408,3.920)}
\gppoint{gp mark 7}{(2.459,3.864)}
\gppoint{gp mark 7}{(2.509,3.807)}
\gppoint{gp mark 7}{(2.560,3.747)}
\gppoint{gp mark 7}{(2.611,3.684)}
\gppoint{gp mark 7}{(2.662,3.618)}
\gppoint{gp mark 7}{(2.712,3.545)}
\gppoint{gp mark 7}{(2.763,3.463)}
\gppoint{gp mark 7}{(2.814,3.357)}
\gppoint{gp mark 7}{(2.864,3.186)}
\gppoint{gp mark 7}{(2.915,2.756)}
\gppoint{gp mark 7}{(2.966,1.908)}
\gppoint{gp mark 7}{(3.017,1.369)}
\gppoint{gp mark 7}{(3.067,1.269)}
\gppoint{gp mark 7}{(3.118,1.255)}
\gppoint{gp mark 7}{(3.169,1.253)}
\gppoint{gp mark 7}{(3.219,1.251)}
\gppoint{gp mark 7}{(3.270,1.251)}
\gppoint{gp mark 7}{(3.321,1.252)}
\gppoint{gp mark 7}{(3.372,1.254)}
\gppoint{gp mark 7}{(3.422,1.254)}
\gppoint{gp mark 7}{(3.473,1.254)}
\gppoint{gp mark 7}{(3.524,1.254)}
\gppoint{gp mark 7}{(3.574,1.257)}
\gppoint{gp mark 7}{(3.625,1.297)}
\gppoint{gp mark 7}{(3.676,1.475)}
\gppoint{gp mark 7}{(3.727,1.618)}
\gppoint{gp mark 7}{(3.777,1.634)}
\gppoint{gp mark 7}{(3.828,1.635)}
\gppoint{gp mark 7}{(3.879,1.634)}
\gppoint{gp mark 7}{(3.929,1.634)}
\gppoint{gp mark 7}{(3.980,1.634)}
\gppoint{gp mark 7}{(4.031,1.634)}
\gppoint{gp mark 7}{(4.082,1.634)}
\gppoint{gp mark 7}{(4.132,1.634)}
\gppoint{gp mark 7}{(4.183,1.634)}
\gppoint{gp mark 7}{(4.234,1.634)}
\gppoint{gp mark 7}{(4.284,1.634)}
\gppoint{gp mark 7}{(4.335,1.634)}
\gppoint{gp mark 7}{(4.386,1.634)}
\gppoint{gp mark 7}{(4.437,1.634)}
\gppoint{gp mark 7}{(4.487,1.634)}
\gppoint{gp mark 7}{(4.538,1.634)}
\gppoint{gp mark 7}{(4.589,1.634)}
\gppoint{gp mark 7}{(4.639,1.634)}
\gppoint{gp mark 7}{(4.690,1.634)}
\gppoint{gp mark 7}{(4.741,1.634)}
\gppoint{gp mark 7}{(4.792,1.634)}
\gppoint{gp mark 7}{(4.842,1.634)}
\gppoint{gp mark 7}{(4.893,1.634)}
\gppoint{gp mark 7}{(4.944,1.634)}
\gppoint{gp mark 7}{(4.994,1.634)}
\gppoint{gp mark 7}{(5.045,1.634)}
\gppoint{gp mark 7}{(5.096,1.634)}
\gppoint{gp mark 7}{(5.147,1.634)}
\gppoint{gp mark 7}{(5.197,1.634)}
\gppoint{gp mark 7}{(5.248,1.634)}
\gppoint{gp mark 7}{(5.299,1.634)}
\gppoint{gp mark 7}{(5.349,1.634)}
\gppoint{gp mark 7}{(5.400,1.634)}
\gppoint{gp mark 7}{(5.451,1.634)}
\gppoint{gp mark 7}{(5.502,1.634)}
\gppoint{gp mark 7}{(5.552,1.634)}
\gppoint{gp mark 7}{(5.603,1.634)}
\gppoint{gp mark 7}{(5.654,1.634)}
\gppoint{gp mark 7}{(5.704,1.634)}
\gppoint{gp mark 7}{(5.755,1.634)}
\gppoint{gp mark 7}{(5.806,1.634)}
\gppoint{gp mark 7}{(5.857,1.634)}
\gppoint{gp mark 7}{(5.907,1.634)}
\gppoint{gp mark 7}{(5.958,1.634)}
\gppoint{gp mark 7}{(6.009,1.634)}
\gppoint{gp mark 7}{(6.059,1.634)}
\gppoint{gp mark 7}{(6.110,1.634)}
\gppoint{gp mark 7}{(6.161,1.634)}
\gppoint{gp mark 7}{(6.212,1.634)}
\gppoint{gp mark 7}{(6.262,1.634)}
\gppoint{gp mark 7}{(6.313,1.634)}
\gppoint{gp mark 7}{(6.364,1.634)}
\gppoint{gp mark 7}{(6.414,1.634)}
\gppoint{gp mark 7}{(6.465,1.634)}
\gppoint{gp mark 7}{(6.516,1.634)}
\gppoint{gp mark 7}{(6.567,1.634)}
\gppoint{gp mark 7}{(6.617,1.634)}
\gppoint{gp mark 7}{(6.668,1.634)}
\gppoint{gp mark 7}{(6.719,1.634)}
\gppoint{gp mark 7}{(6.769,1.634)}
\gppoint{gp mark 7}{(6.820,1.634)}
\gppoint{gp mark 7}{(6.871,1.634)}
\gppoint{gp mark 7}{(6.922,1.634)}
\gppoint{gp mark 7}{(6.972,1.634)}
\gppoint{gp mark 7}{(7.023,1.634)}
\gppoint{gp mark 7}{(7.074,1.634)}
\gppoint{gp mark 7}{(7.124,1.634)}
\gppoint{gp mark 7}{(7.175,1.634)}
\gppoint{gp mark 7}{(7.226,1.634)}
\gppoint{gp mark 7}{(7.277,1.634)}
\gppoint{gp mark 7}{(7.327,1.634)}
\gppoint{gp mark 7}{(7.378,1.634)}
\gppoint{gp mark 7}{(7.429,1.634)}
\gppoint{gp mark 7}{(7.479,1.634)}
\gppoint{gp mark 7}{(7.530,1.634)}
\gppoint{gp mark 7}{(7.581,1.634)}
\gppoint{gp mark 7}{(7.632,1.634)}
\gppoint{gp mark 7}{(7.682,1.634)}
\gppoint{gp mark 7}{(7.733,1.634)}
\gppoint{gp mark 7}{(7.784,1.634)}
\gppoint{gp mark 7}{(7.834,1.634)}
\gppoint{gp mark 7}{(7.885,1.634)}
\gpcolor{rgb color={1.000,0.000,0.000}}
\gpsetpointsize{4.44}
\gppoint{gp mark 7}{(1.242,4.932)}
\gppoint{gp mark 7}{(1.292,4.894)}
\gppoint{gp mark 7}{(1.343,4.856)}
\gppoint{gp mark 7}{(1.394,4.818)}
\gppoint{gp mark 7}{(1.444,4.779)}
\gppoint{gp mark 7}{(1.495,4.740)}
\gppoint{gp mark 7}{(1.546,4.701)}
\gppoint{gp mark 7}{(1.597,4.662)}
\gppoint{gp mark 7}{(1.647,4.622)}
\gppoint{gp mark 7}{(1.698,4.582)}
\gppoint{gp mark 7}{(1.749,4.542)}
\gppoint{gp mark 7}{(1.799,4.502)}
\gppoint{gp mark 7}{(1.850,4.461)}
\gppoint{gp mark 7}{(1.901,4.419)}
\gppoint{gp mark 7}{(1.952,4.378)}
\gppoint{gp mark 7}{(2.002,4.336)}
\gppoint{gp mark 7}{(2.053,4.294)}
\gppoint{gp mark 7}{(2.104,4.252)}
\gppoint{gp mark 7}{(2.154,4.209)}
\gppoint{gp mark 7}{(2.205,4.167)}
\gppoint{gp mark 7}{(2.256,4.125)}
\gppoint{gp mark 7}{(2.307,4.084)}
\gppoint{gp mark 7}{(2.357,4.044)}
\gppoint{gp mark 7}{(2.408,4.008)}
\gppoint{gp mark 7}{(2.459,3.977)}
\gppoint{gp mark 7}{(2.509,3.956)}
\gppoint{gp mark 7}{(2.560,3.943)}
\gppoint{gp mark 7}{(2.611,3.938)}
\gppoint{gp mark 7}{(2.662,3.936)}
\gppoint{gp mark 7}{(2.712,3.935)}
\gppoint{gp mark 7}{(2.763,3.930)}
\gppoint{gp mark 7}{(2.814,3.909)}
\gppoint{gp mark 7}{(2.864,3.209)}
\gppoint{gp mark 7}{(2.915,1.611)}
\gppoint{gp mark 7}{(2.966,1.254)}
\gppoint{gp mark 7}{(3.017,1.230)}
\gppoint{gp mark 7}{(3.067,1.231)}
\gppoint{gp mark 7}{(3.118,1.231)}
\gppoint{gp mark 7}{(3.169,1.229)}
\gppoint{gp mark 7}{(3.219,1.228)}
\gppoint{gp mark 7}{(3.270,1.229)}
\gppoint{gp mark 7}{(3.321,1.227)}
\gppoint{gp mark 7}{(3.372,1.225)}
\gppoint{gp mark 7}{(3.422,1.225)}
\gppoint{gp mark 7}{(3.473,1.224)}
\gppoint{gp mark 7}{(3.524,1.223)}
\gppoint{gp mark 7}{(3.574,1.223)}
\gppoint{gp mark 7}{(3.625,1.223)}
\gppoint{gp mark 7}{(3.676,1.225)}
\gppoint{gp mark 7}{(3.727,1.254)}
\gppoint{gp mark 7}{(3.777,1.408)}
\gppoint{gp mark 7}{(3.828,1.604)}
\gppoint{gp mark 7}{(3.879,1.633)}
\gppoint{gp mark 7}{(3.929,1.635)}
\gppoint{gp mark 7}{(3.980,1.635)}
\gppoint{gp mark 7}{(4.031,1.635)}
\gppoint{gp mark 7}{(4.082,1.635)}
\gppoint{gp mark 7}{(4.132,1.635)}
\gppoint{gp mark 7}{(4.183,1.634)}
\gppoint{gp mark 7}{(4.234,1.634)}
\gppoint{gp mark 7}{(4.284,1.634)}
\gppoint{gp mark 7}{(4.335,1.635)}
\gppoint{gp mark 7}{(4.386,1.635)}
\gppoint{gp mark 7}{(4.437,1.635)}
\gppoint{gp mark 7}{(4.487,1.634)}
\gppoint{gp mark 7}{(4.538,1.634)}
\gppoint{gp mark 7}{(4.589,1.635)}
\gppoint{gp mark 7}{(4.639,1.635)}
\gppoint{gp mark 7}{(4.690,1.635)}
\gppoint{gp mark 7}{(4.741,1.635)}
\gppoint{gp mark 7}{(4.792,1.635)}
\gppoint{gp mark 7}{(4.842,1.635)}
\gppoint{gp mark 7}{(4.893,1.635)}
\gppoint{gp mark 7}{(4.944,1.635)}
\gppoint{gp mark 7}{(4.994,1.635)}
\gppoint{gp mark 7}{(5.045,1.635)}
\gppoint{gp mark 7}{(5.096,1.635)}
\gppoint{gp mark 7}{(5.147,1.635)}
\gppoint{gp mark 7}{(5.197,1.634)}
\gppoint{gp mark 7}{(5.248,1.634)}
\gppoint{gp mark 7}{(5.299,1.635)}
\gppoint{gp mark 7}{(5.349,1.635)}
\gppoint{gp mark 7}{(5.400,1.635)}
\gppoint{gp mark 7}{(5.451,1.634)}
\gppoint{gp mark 7}{(5.502,1.634)}
\gppoint{gp mark 7}{(5.552,1.634)}
\gppoint{gp mark 7}{(5.603,1.634)}
\gppoint{gp mark 7}{(5.654,1.635)}
\gppoint{gp mark 7}{(5.704,1.635)}
\gppoint{gp mark 7}{(5.755,1.635)}
\gppoint{gp mark 7}{(5.806,1.635)}
\gppoint{gp mark 7}{(5.857,1.635)}
\gppoint{gp mark 7}{(5.907,1.635)}
\gppoint{gp mark 7}{(5.958,1.635)}
\gppoint{gp mark 7}{(6.009,1.635)}
\gppoint{gp mark 7}{(6.059,1.635)}
\gppoint{gp mark 7}{(6.110,1.635)}
\gppoint{gp mark 7}{(6.161,1.635)}
\gppoint{gp mark 7}{(6.212,1.635)}
\gppoint{gp mark 7}{(6.262,1.635)}
\gppoint{gp mark 7}{(6.313,1.635)}
\gppoint{gp mark 7}{(6.364,1.634)}
\gppoint{gp mark 7}{(6.414,1.634)}
\gppoint{gp mark 7}{(6.465,1.634)}
\gppoint{gp mark 7}{(6.516,1.634)}
\gppoint{gp mark 7}{(6.567,1.634)}
\gppoint{gp mark 7}{(6.617,1.634)}
\gppoint{gp mark 7}{(6.668,1.634)}
\gppoint{gp mark 7}{(6.719,1.634)}
\gppoint{gp mark 7}{(6.769,1.635)}
\gppoint{gp mark 7}{(6.820,1.635)}
\gppoint{gp mark 7}{(6.871,1.635)}
\gppoint{gp mark 7}{(6.922,1.635)}
\gppoint{gp mark 7}{(6.972,1.635)}
\gppoint{gp mark 7}{(7.023,1.635)}
\gppoint{gp mark 7}{(7.074,1.635)}
\gppoint{gp mark 7}{(7.124,1.635)}
\gppoint{gp mark 7}{(7.175,1.635)}
\gppoint{gp mark 7}{(7.226,1.635)}
\gppoint{gp mark 7}{(7.277,1.635)}
\gppoint{gp mark 7}{(7.327,1.635)}
\gppoint{gp mark 7}{(7.378,1.635)}
\gppoint{gp mark 7}{(7.429,1.634)}
\gppoint{gp mark 7}{(7.479,1.634)}
\gppoint{gp mark 7}{(7.530,1.634)}
\gppoint{gp mark 7}{(7.581,1.634)}
\gppoint{gp mark 7}{(7.632,1.634)}
\gppoint{gp mark 7}{(7.682,1.634)}
\gppoint{gp mark 7}{(7.733,1.634)}
\gppoint{gp mark 7}{(7.784,1.634)}
\gppoint{gp mark 7}{(7.834,1.635)}
\gppoint{gp mark 7}{(7.885,1.635)}
\gpcolor{rgb color={0.000,0.000,0.000}}
\gpsetlinetype{gp lt plot 0}
\gpsetlinewidth{4.00}
\draw[gp path] (2.404,3.934)--(2.886,3.934);
\draw[gp path] (2.886,1.228)--(3.811,1.228);
\draw[gp path] (3.811,1.635)--(7.722,1.635);
\draw[gp path] (7.722,1.635)--(7.947,1.635);
\draw[gp path] (1.207,4.836)--(1.220,4.826)--(1.233,4.816)--(1.246,4.807)--(1.259,4.797)%
  --(1.271,4.787)--(1.284,4.777)--(1.297,4.768)--(1.310,4.758)--(1.323,4.748)--(1.336,4.739)%
  --(1.349,4.729)--(1.361,4.719)--(1.374,4.710)--(1.387,4.700)--(1.400,4.690)--(1.413,4.681)%
  --(1.426,4.671)--(1.439,4.661)--(1.452,4.652)--(1.464,4.642)--(1.477,4.632)--(1.490,4.622)%
  --(1.503,4.613)--(1.516,4.603)--(1.529,4.593)--(1.542,4.584)--(1.555,4.574)--(1.567,4.564)%
  --(1.580,4.555)--(1.593,4.545)--(1.606,4.535)--(1.619,4.526)--(1.632,4.516)--(1.645,4.506)%
  --(1.657,4.496)--(1.670,4.487)--(1.683,4.477)--(1.696,4.467)--(1.709,4.458)--(1.722,4.448)%
  --(1.735,4.438)--(1.748,4.429)--(1.760,4.419)--(1.773,4.409)--(1.786,4.400)--(1.799,4.390)%
  --(1.812,4.380)--(1.825,4.371)--(1.838,4.361)--(1.850,4.351)--(1.863,4.341)--(1.876,4.332)%
  --(1.889,4.322)--(1.902,4.312)--(1.915,4.303)--(1.928,4.293)--(1.941,4.283)--(1.953,4.274)%
  --(1.966,4.264)--(1.979,4.254)--(1.992,4.245)--(2.005,4.235)--(2.018,4.225)--(2.031,4.215)%
  --(2.044,4.206)--(2.056,4.196)--(2.069,4.186)--(2.082,4.177)--(2.095,4.167)--(2.108,4.157)%
  --(2.121,4.148)--(2.134,4.138)--(2.146,4.128)--(2.159,4.119)--(2.172,4.109)--(2.185,4.099)%
  --(2.198,4.090)--(2.211,4.080)--(2.224,4.070)--(2.237,4.060)--(2.249,4.051)--(2.262,4.041)%
  --(2.275,4.031)--(2.288,4.022)--(2.301,4.012)--(2.314,4.002)--(2.327,3.993)--(2.340,3.983)%
  --(2.352,3.973)--(2.365,3.964)--(2.378,3.954)--(2.391,3.944)--(2.404,3.934);
\draw[gp path] (2.886,3.934)--(2.886,1.228);
\draw[gp path] (3.811,1.228)--(3.811,1.635);
\draw[gp path] (3.793,3.695)--(4.572,3.695);
\gpcolor{rgb color={1.000,0.000,0.000}}
\gpsetlinewidth{0.50}
\gppoint{gp mark 7}{(4.182,2.921)}
\gpcolor{rgb color={0.502,0.502,0.502}}
\gppoint{gp mark 7}{(4.182,2.147)}
\gpcolor{rgb color={0.000,0.000,0.000}}
\node[gp node left,font={\fontsize{10pt}{12pt}\selectfont}] at (1.456,5.166) {\LARGE $B_y$};
\node[gp node left,font={\fontsize{10pt}{12pt}\selectfont}] at (5.740,5.166) {\large $\alpha = 3.0$};
\node[gp node left,font={\fontsize{10pt}{12pt}\selectfont}] at (4.831,3.695) {\large exact};
\node[gp node left,font={\fontsize{10pt}{12pt}\selectfont}] at (4.831,2.921) {\large HLLD-CWM};
\node[gp node left,font={\fontsize{10pt}{12pt}\selectfont}] at (4.831,2.147) {\large HLLD};
%% coordinates of the plot area
\gpdefrectangularnode{gp plot 1}{\pgfpoint{1.196cm}{0.985cm}}{\pgfpoint{7.947cm}{5.631cm}}
\end{tikzpicture}
%% gnuplot variables
} \\
\resizebox{0.5\linewidth}{!}{\tikzsetnextfilename{fast_coplanar_a_crsol_00512_1}\begin{tikzpicture}[gnuplot]
%% generated with GNUPLOT 4.6p4 (Lua 5.1; terminal rev. 99, script rev. 100)
%% Fri 22 Aug 2014 11:40:26 AM EDT
\path (0.000,0.000) rectangle (8.500,6.000);
\gpfill{rgb color={1.000,1.000,1.000}} (1.196,0.985)--(7.946,0.985)--(7.946,5.630)--(1.196,5.630)--cycle;
\gpcolor{color=gp lt color border}
\gpsetlinetype{gp lt border}
\gpsetlinewidth{1.00}
\draw[gp path] (1.196,0.985)--(1.196,5.630)--(7.946,5.630)--(7.946,0.985)--cycle;
\gpcolor{color=gp lt color axes}
\gpsetlinetype{gp lt axes}
\gpsetlinewidth{2.00}
\draw[gp path] (1.196,1.275)--(7.947,1.275);
\gpcolor{color=gp lt color border}
\gpsetlinetype{gp lt border}
\draw[gp path] (1.196,1.275)--(1.268,1.275);
\draw[gp path] (7.947,1.275)--(7.875,1.275);
\gpcolor{rgb color={0.000,0.000,0.000}}
\node[gp node right,font={\fontsize{10pt}{12pt}\selectfont}] at (1.012,1.275) {0.65};
\gpcolor{color=gp lt color axes}
\gpsetlinetype{gp lt axes}
\draw[gp path] (1.196,2.001)--(7.947,2.001);
\gpcolor{color=gp lt color border}
\gpsetlinetype{gp lt border}
\draw[gp path] (1.196,2.001)--(1.268,2.001);
\draw[gp path] (7.947,2.001)--(7.875,2.001);
\gpcolor{rgb color={0.000,0.000,0.000}}
\node[gp node right,font={\fontsize{10pt}{12pt}\selectfont}] at (1.012,2.001) {0.7};
\gpcolor{color=gp lt color axes}
\gpsetlinetype{gp lt axes}
\draw[gp path] (1.196,2.727)--(7.947,2.727);
\gpcolor{color=gp lt color border}
\gpsetlinetype{gp lt border}
\draw[gp path] (1.196,2.727)--(1.268,2.727);
\draw[gp path] (7.947,2.727)--(7.875,2.727);
\gpcolor{rgb color={0.000,0.000,0.000}}
\node[gp node right,font={\fontsize{10pt}{12pt}\selectfont}] at (1.012,2.727) {0.75};
\gpcolor{color=gp lt color axes}
\gpsetlinetype{gp lt axes}
\draw[gp path] (1.196,3.453)--(7.947,3.453);
\gpcolor{color=gp lt color border}
\gpsetlinetype{gp lt border}
\draw[gp path] (1.196,3.453)--(1.268,3.453);
\draw[gp path] (7.947,3.453)--(7.875,3.453);
\gpcolor{rgb color={0.000,0.000,0.000}}
\node[gp node right,font={\fontsize{10pt}{12pt}\selectfont}] at (1.012,3.453) {0.8};
\gpcolor{color=gp lt color axes}
\gpsetlinetype{gp lt axes}
\draw[gp path] (1.196,4.179)--(7.947,4.179);
\gpcolor{color=gp lt color border}
\gpsetlinetype{gp lt border}
\draw[gp path] (1.196,4.179)--(1.268,4.179);
\draw[gp path] (7.947,4.179)--(7.875,4.179);
\gpcolor{rgb color={0.000,0.000,0.000}}
\node[gp node right,font={\fontsize{10pt}{12pt}\selectfont}] at (1.012,4.179) {0.85};
\gpcolor{color=gp lt color axes}
\gpsetlinetype{gp lt axes}
\draw[gp path] (1.196,4.905)--(7.947,4.905);
\gpcolor{color=gp lt color border}
\gpsetlinetype{gp lt border}
\draw[gp path] (1.196,4.905)--(1.268,4.905);
\draw[gp path] (7.947,4.905)--(7.875,4.905);
\gpcolor{rgb color={0.000,0.000,0.000}}
\node[gp node right,font={\fontsize{10pt}{12pt}\selectfont}] at (1.012,4.905) {0.9};
\gpcolor{color=gp lt color axes}
\gpsetlinetype{gp lt axes}
\draw[gp path] (1.196,5.631)--(7.947,5.631);
\gpcolor{color=gp lt color border}
\gpsetlinetype{gp lt border}
\draw[gp path] (1.196,5.631)--(1.268,5.631);
\draw[gp path] (7.947,5.631)--(7.875,5.631);
\gpcolor{rgb color={0.000,0.000,0.000}}
\node[gp node right,font={\fontsize{10pt}{12pt}\selectfont}] at (1.012,5.631) {0.95};
\gpcolor{color=gp lt color axes}
\gpsetlinetype{gp lt axes}
\draw[gp path] (1.196,0.985)--(1.196,5.631);
\gpcolor{color=gp lt color border}
\gpsetlinetype{gp lt border}
\draw[gp path] (1.196,0.985)--(1.196,1.057);
\draw[gp path] (1.196,5.631)--(1.196,5.559);
\gpcolor{rgb color={0.000,0.000,0.000}}
\node[gp node center,font={\fontsize{10pt}{12pt}\selectfont}] at (1.196,0.677) {0.3};
\gpcolor{color=gp lt color axes}
\gpsetlinetype{gp lt axes}
\draw[gp path] (2.494,0.985)--(2.494,5.631);
\gpcolor{color=gp lt color border}
\gpsetlinetype{gp lt border}
\draw[gp path] (2.494,0.985)--(2.494,1.057);
\draw[gp path] (2.494,5.631)--(2.494,5.559);
\gpcolor{rgb color={0.000,0.000,0.000}}
\node[gp node center,font={\fontsize{10pt}{12pt}\selectfont}] at (2.494,0.677) {0.35};
\gpcolor{color=gp lt color axes}
\gpsetlinetype{gp lt axes}
\draw[gp path] (3.793,0.985)--(3.793,5.631);
\gpcolor{color=gp lt color border}
\gpsetlinetype{gp lt border}
\draw[gp path] (3.793,0.985)--(3.793,1.057);
\draw[gp path] (3.793,5.631)--(3.793,5.559);
\gpcolor{rgb color={0.000,0.000,0.000}}
\node[gp node center,font={\fontsize{10pt}{12pt}\selectfont}] at (3.793,0.677) {0.4};
\gpcolor{color=gp lt color axes}
\gpsetlinetype{gp lt axes}
\draw[gp path] (5.091,0.985)--(5.091,5.631);
\gpcolor{color=gp lt color border}
\gpsetlinetype{gp lt border}
\draw[gp path] (5.091,0.985)--(5.091,1.057);
\draw[gp path] (5.091,5.631)--(5.091,5.559);
\gpcolor{rgb color={0.000,0.000,0.000}}
\node[gp node center,font={\fontsize{10pt}{12pt}\selectfont}] at (5.091,0.677) {0.45};
\gpcolor{color=gp lt color axes}
\gpsetlinetype{gp lt axes}
\draw[gp path] (6.389,0.985)--(6.389,5.631);
\gpcolor{color=gp lt color border}
\gpsetlinetype{gp lt border}
\draw[gp path] (6.389,0.985)--(6.389,1.057);
\draw[gp path] (6.389,5.631)--(6.389,5.559);
\gpcolor{rgb color={0.000,0.000,0.000}}
\node[gp node center,font={\fontsize{10pt}{12pt}\selectfont}] at (6.389,0.677) {0.5};
\gpcolor{color=gp lt color axes}
\gpsetlinetype{gp lt axes}
\draw[gp path] (7.687,0.985)--(7.687,5.631);
\gpcolor{color=gp lt color border}
\gpsetlinetype{gp lt border}
\draw[gp path] (7.687,0.985)--(7.687,1.057);
\draw[gp path] (7.687,5.631)--(7.687,5.559);
\gpcolor{rgb color={0.000,0.000,0.000}}
\node[gp node center,font={\fontsize{10pt}{12pt}\selectfont}] at (7.687,0.677) {0.55};
\gpcolor{color=gp lt color border}
\draw[gp path] (1.196,5.631)--(1.196,0.985)--(7.947,0.985)--(7.947,5.631)--cycle;
\gpcolor{rgb color={0.000,0.000,0.000}}
\node[gp node center,font={\fontsize{10pt}{12pt}\selectfont}] at (4.571,0.215) {\large $x$};
\gpcolor{rgb color={0.502,0.502,0.502}}
\gpsetlinewidth{0.50}
\gpsetpointsize{2.67}
\gppoint{gp mark 7}{(1.242,3.594)}
\gppoint{gp mark 7}{(1.292,3.532)}
\gppoint{gp mark 7}{(1.343,3.470)}
\gppoint{gp mark 7}{(1.394,3.409)}
\gppoint{gp mark 7}{(1.444,3.348)}
\gppoint{gp mark 7}{(1.495,3.287)}
\gppoint{gp mark 7}{(1.546,3.226)}
\gppoint{gp mark 7}{(1.597,3.165)}
\gppoint{gp mark 7}{(1.647,3.105)}
\gppoint{gp mark 7}{(1.698,3.044)}
\gppoint{gp mark 7}{(1.749,2.984)}
\gppoint{gp mark 7}{(1.799,2.924)}
\gppoint{gp mark 7}{(1.850,2.864)}
\gppoint{gp mark 7}{(1.901,2.804)}
\gppoint{gp mark 7}{(1.952,2.744)}
\gppoint{gp mark 7}{(2.002,2.685)}
\gppoint{gp mark 7}{(2.053,2.625)}
\gppoint{gp mark 7}{(2.104,2.566)}
\gppoint{gp mark 7}{(2.154,2.506)}
\gppoint{gp mark 7}{(2.205,2.447)}
\gppoint{gp mark 7}{(2.256,2.387)}
\gppoint{gp mark 7}{(2.307,2.328)}
\gppoint{gp mark 7}{(2.357,2.269)}
\gppoint{gp mark 7}{(2.408,2.209)}
\gppoint{gp mark 7}{(2.459,2.150)}
\gppoint{gp mark 7}{(2.509,2.090)}
\gppoint{gp mark 7}{(2.560,2.031)}
\gppoint{gp mark 7}{(2.611,1.972)}
\gppoint{gp mark 7}{(2.662,1.912)}
\gppoint{gp mark 7}{(2.712,1.852)}
\gppoint{gp mark 7}{(2.763,1.788)}
\gppoint{gp mark 7}{(2.814,1.707)}
\gppoint{gp mark 7}{(2.864,1.580)}
\gppoint{gp mark 7}{(2.915,1.505)}
\gppoint{gp mark 7}{(2.966,1.728)}
\gppoint{gp mark 7}{(3.017,2.229)}
\gppoint{gp mark 7}{(3.067,2.337)}
\gppoint{gp mark 7}{(3.118,2.354)}
\gppoint{gp mark 7}{(3.169,2.359)}
\gppoint{gp mark 7}{(3.219,2.367)}
\gppoint{gp mark 7}{(3.270,2.373)}
\gppoint{gp mark 7}{(3.321,2.379)}
\gppoint{gp mark 7}{(3.372,2.381)}
\gppoint{gp mark 7}{(3.422,2.379)}
\gppoint{gp mark 7}{(3.473,2.374)}
\gppoint{gp mark 7}{(3.524,2.375)}
\gppoint{gp mark 7}{(3.574,2.399)}
\gppoint{gp mark 7}{(3.625,2.688)}
\gppoint{gp mark 7}{(3.676,3.819)}
\gppoint{gp mark 7}{(3.727,4.590)}
\gppoint{gp mark 7}{(3.777,4.649)}
\gppoint{gp mark 7}{(3.828,4.654)}
\gppoint{gp mark 7}{(3.879,4.656)}
\gppoint{gp mark 7}{(3.929,4.653)}
\gppoint{gp mark 7}{(3.980,4.648)}
\gppoint{gp mark 7}{(4.031,4.648)}
\gppoint{gp mark 7}{(4.082,4.649)}
\gppoint{gp mark 7}{(4.132,4.652)}
\gppoint{gp mark 7}{(4.183,4.652)}
\gppoint{gp mark 7}{(4.234,4.650)}
\gppoint{gp mark 7}{(4.284,4.648)}
\gppoint{gp mark 7}{(4.335,4.648)}
\gppoint{gp mark 7}{(4.386,4.649)}
\gppoint{gp mark 7}{(4.437,4.650)}
\gppoint{gp mark 7}{(4.487,4.649)}
\gppoint{gp mark 7}{(4.538,4.649)}
\gppoint{gp mark 7}{(4.589,4.649)}
\gppoint{gp mark 7}{(4.639,4.650)}
\gppoint{gp mark 7}{(4.690,4.650)}
\gppoint{gp mark 7}{(4.741,4.650)}
\gppoint{gp mark 7}{(4.792,4.650)}
\gppoint{gp mark 7}{(4.842,4.649)}
\gppoint{gp mark 7}{(4.893,4.650)}
\gppoint{gp mark 7}{(4.944,4.650)}
\gppoint{gp mark 7}{(4.994,4.650)}
\gppoint{gp mark 7}{(5.045,4.650)}
\gppoint{gp mark 7}{(5.096,4.649)}
\gppoint{gp mark 7}{(5.147,4.649)}
\gppoint{gp mark 7}{(5.197,4.649)}
\gppoint{gp mark 7}{(5.248,4.650)}
\gppoint{gp mark 7}{(5.299,4.650)}
\gppoint{gp mark 7}{(5.349,4.650)}
\gppoint{gp mark 7}{(5.400,4.650)}
\gppoint{gp mark 7}{(5.451,4.650)}
\gppoint{gp mark 7}{(5.502,4.651)}
\gppoint{gp mark 7}{(5.552,4.652)}
\gppoint{gp mark 7}{(5.603,4.653)}
\gppoint{gp mark 7}{(5.654,4.654)}
\gppoint{gp mark 7}{(5.704,4.655)}
\gppoint{gp mark 7}{(5.755,4.656)}
\gppoint{gp mark 7}{(5.806,4.658)}
\gppoint{gp mark 7}{(5.857,4.659)}
\gppoint{gp mark 7}{(5.907,4.659)}
\gppoint{gp mark 7}{(5.958,4.660)}
\gppoint{gp mark 7}{(6.009,4.662)}
\gppoint{gp mark 7}{(6.059,4.663)}
\gppoint{gp mark 7}{(6.110,4.663)}
\gppoint{gp mark 7}{(6.161,4.663)}
\gppoint{gp mark 7}{(6.212,4.663)}
\gppoint{gp mark 7}{(6.262,4.663)}
\gppoint{gp mark 7}{(6.313,4.663)}
\gppoint{gp mark 7}{(6.364,4.662)}
\gppoint{gp mark 7}{(6.414,4.661)}
\gppoint{gp mark 7}{(6.465,4.659)}
\gppoint{gp mark 7}{(6.516,4.656)}
\gppoint{gp mark 7}{(6.567,4.654)}
\gppoint{gp mark 7}{(6.617,4.649)}
\gppoint{gp mark 7}{(6.668,4.643)}
\gppoint{gp mark 7}{(6.719,4.637)}
\gppoint{gp mark 7}{(6.769,4.631)}
\gppoint{gp mark 7}{(6.820,4.624)}
\gppoint{gp mark 7}{(6.871,4.614)}
\gppoint{gp mark 7}{(6.922,4.604)}
\gppoint{gp mark 7}{(6.972,4.597)}
\gppoint{gp mark 7}{(7.023,4.591)}
\gppoint{gp mark 7}{(7.074,4.583)}
\gppoint{gp mark 7}{(7.124,4.576)}
\gppoint{gp mark 7}{(7.175,4.572)}
\gppoint{gp mark 7}{(7.226,4.568)}
\gppoint{gp mark 7}{(7.277,4.559)}
\gppoint{gp mark 7}{(7.327,4.545)}
\gppoint{gp mark 7}{(7.378,4.534)}
\gppoint{gp mark 7}{(7.429,4.529)}
\gppoint{gp mark 7}{(7.479,4.526)}
\gppoint{gp mark 7}{(7.530,4.514)}
\gppoint{gp mark 7}{(7.581,4.433)}
\gppoint{gp mark 7}{(7.632,3.922)}
\gppoint{gp mark 7}{(7.682,2.807)}
\gppoint{gp mark 7}{(7.733,1.835)}
\gppoint{gp mark 7}{(7.784,1.378)}
\gppoint{gp mark 7}{(7.834,1.231)}
\gppoint{gp mark 7}{(7.885,1.184)}
\gpcolor{rgb color={1.000,0.000,0.000}}
\gpsetpointsize{4.44}
\gppoint{gp mark 7}{(1.242,3.602)}
\gppoint{gp mark 7}{(1.292,3.541)}
\gppoint{gp mark 7}{(1.343,3.480)}
\gppoint{gp mark 7}{(1.394,3.419)}
\gppoint{gp mark 7}{(1.444,3.359)}
\gppoint{gp mark 7}{(1.495,3.299)}
\gppoint{gp mark 7}{(1.546,3.239)}
\gppoint{gp mark 7}{(1.597,3.179)}
\gppoint{gp mark 7}{(1.647,3.120)}
\gppoint{gp mark 7}{(1.698,3.061)}
\gppoint{gp mark 7}{(1.749,3.002)}
\gppoint{gp mark 7}{(1.799,2.944)}
\gppoint{gp mark 7}{(1.850,2.885)}
\gppoint{gp mark 7}{(1.901,2.828)}
\gppoint{gp mark 7}{(1.952,2.770)}
\gppoint{gp mark 7}{(2.002,2.713)}
\gppoint{gp mark 7}{(2.053,2.657)}
\gppoint{gp mark 7}{(2.104,2.601)}
\gppoint{gp mark 7}{(2.154,2.547)}
\gppoint{gp mark 7}{(2.205,2.493)}
\gppoint{gp mark 7}{(2.256,2.440)}
\gppoint{gp mark 7}{(2.307,2.389)}
\gppoint{gp mark 7}{(2.357,2.340)}
\gppoint{gp mark 7}{(2.408,2.295)}
\gppoint{gp mark 7}{(2.459,2.254)}
\gppoint{gp mark 7}{(2.509,2.222)}
\gppoint{gp mark 7}{(2.560,2.200)}
\gppoint{gp mark 7}{(2.611,2.187)}
\gppoint{gp mark 7}{(2.662,2.182)}
\gppoint{gp mark 7}{(2.712,2.182)}
\gppoint{gp mark 7}{(2.763,2.192)}
\gppoint{gp mark 7}{(2.814,2.268)}
\gppoint{gp mark 7}{(2.864,2.253)}
\gppoint{gp mark 7}{(2.915,2.186)}
\gppoint{gp mark 7}{(2.966,2.189)}
\gppoint{gp mark 7}{(3.017,2.178)}
\gppoint{gp mark 7}{(3.067,2.207)}
\gppoint{gp mark 7}{(3.118,2.197)}
\gppoint{gp mark 7}{(3.169,2.181)}
\gppoint{gp mark 7}{(3.219,2.183)}
\gppoint{gp mark 7}{(3.270,2.186)}
\gppoint{gp mark 7}{(3.321,2.184)}
\gppoint{gp mark 7}{(3.372,2.183)}
\gppoint{gp mark 7}{(3.422,2.181)}
\gppoint{gp mark 7}{(3.473,2.180)}
\gppoint{gp mark 7}{(3.524,2.177)}
\gppoint{gp mark 7}{(3.574,2.172)}
\gppoint{gp mark 7}{(3.625,2.166)}
\gppoint{gp mark 7}{(3.676,2.178)}
\gppoint{gp mark 7}{(3.727,2.405)}
\gppoint{gp mark 7}{(3.777,3.571)}
\gppoint{gp mark 7}{(3.828,4.586)}
\gppoint{gp mark 7}{(3.879,4.691)}
\gppoint{gp mark 7}{(3.929,4.698)}
\gppoint{gp mark 7}{(3.980,4.689)}
\gppoint{gp mark 7}{(4.031,4.679)}
\gppoint{gp mark 7}{(4.082,4.678)}
\gppoint{gp mark 7}{(4.132,4.680)}
\gppoint{gp mark 7}{(4.183,4.684)}
\gppoint{gp mark 7}{(4.234,4.688)}
\gppoint{gp mark 7}{(4.284,4.694)}
\gppoint{gp mark 7}{(4.335,4.693)}
\gppoint{gp mark 7}{(4.386,4.691)}
\gppoint{gp mark 7}{(4.437,4.686)}
\gppoint{gp mark 7}{(4.487,4.685)}
\gppoint{gp mark 7}{(4.538,4.687)}
\gppoint{gp mark 7}{(4.589,4.690)}
\gppoint{gp mark 7}{(4.639,4.692)}
\gppoint{gp mark 7}{(4.690,4.690)}
\gppoint{gp mark 7}{(4.741,4.688)}
\gppoint{gp mark 7}{(4.792,4.686)}
\gppoint{gp mark 7}{(4.842,4.685)}
\gppoint{gp mark 7}{(4.893,4.687)}
\gppoint{gp mark 7}{(4.944,4.690)}
\gppoint{gp mark 7}{(4.994,4.696)}
\gppoint{gp mark 7}{(5.045,4.698)}
\gppoint{gp mark 7}{(5.096,4.698)}
\gppoint{gp mark 7}{(5.147,4.696)}
\gppoint{gp mark 7}{(5.197,4.693)}
\gppoint{gp mark 7}{(5.248,4.692)}
\gppoint{gp mark 7}{(5.299,4.692)}
\gppoint{gp mark 7}{(5.349,4.694)}
\gppoint{gp mark 7}{(5.400,4.699)}
\gppoint{gp mark 7}{(5.451,4.699)}
\gppoint{gp mark 7}{(5.502,4.698)}
\gppoint{gp mark 7}{(5.552,4.696)}
\gppoint{gp mark 7}{(5.603,4.696)}
\gppoint{gp mark 7}{(5.654,4.696)}
\gppoint{gp mark 7}{(5.704,4.699)}
\gppoint{gp mark 7}{(5.755,4.703)}
\gppoint{gp mark 7}{(5.806,4.706)}
\gppoint{gp mark 7}{(5.857,4.707)}
\gppoint{gp mark 7}{(5.907,4.707)}
\gppoint{gp mark 7}{(5.958,4.707)}
\gppoint{gp mark 7}{(6.009,4.708)}
\gppoint{gp mark 7}{(6.059,4.710)}
\gppoint{gp mark 7}{(6.110,4.714)}
\gppoint{gp mark 7}{(6.161,4.718)}
\gppoint{gp mark 7}{(6.212,4.719)}
\gppoint{gp mark 7}{(6.262,4.719)}
\gppoint{gp mark 7}{(6.313,4.719)}
\gppoint{gp mark 7}{(6.364,4.719)}
\gppoint{gp mark 7}{(6.414,4.723)}
\gppoint{gp mark 7}{(6.465,4.728)}
\gppoint{gp mark 7}{(6.516,4.730)}
\gppoint{gp mark 7}{(6.567,4.731)}
\gppoint{gp mark 7}{(6.617,4.731)}
\gppoint{gp mark 7}{(6.668,4.727)}
\gppoint{gp mark 7}{(6.719,4.721)}
\gppoint{gp mark 7}{(6.769,4.719)}
\gppoint{gp mark 7}{(6.820,4.717)}
\gppoint{gp mark 7}{(6.871,4.713)}
\gppoint{gp mark 7}{(6.922,4.707)}
\gppoint{gp mark 7}{(6.972,4.701)}
\gppoint{gp mark 7}{(7.023,4.691)}
\gppoint{gp mark 7}{(7.074,4.674)}
\gppoint{gp mark 7}{(7.124,4.653)}
\gppoint{gp mark 7}{(7.175,4.636)}
\gppoint{gp mark 7}{(7.226,4.622)}
\gppoint{gp mark 7}{(7.277,4.607)}
\gppoint{gp mark 7}{(7.327,4.595)}
\gppoint{gp mark 7}{(7.378,4.589)}
\gppoint{gp mark 7}{(7.429,4.587)}
\gppoint{gp mark 7}{(7.479,4.585)}
\gppoint{gp mark 7}{(7.530,4.572)}
\gppoint{gp mark 7}{(7.581,4.474)}
\gppoint{gp mark 7}{(7.632,3.891)}
\gppoint{gp mark 7}{(7.682,2.733)}
\gppoint{gp mark 7}{(7.733,1.788)}
\gppoint{gp mark 7}{(7.784,1.363)}
\gppoint{gp mark 7}{(7.834,1.229)}
\gppoint{gp mark 7}{(7.885,1.184)}
\gpcolor{rgb color={0.000,0.000,0.000}}
\gpsetlinetype{gp lt plot 0}
\gpsetlinewidth{4.00}
\draw[gp path] (2.440,2.203)--(2.899,2.203);
\draw[gp path] (2.899,2.203)--(3.805,2.203);
\draw[gp path] (3.805,4.694)--(7.720,4.694);
\draw[gp path] (7.720,1.153)--(7.947,1.153);
\draw[gp path] (1.204,3.442)--(1.217,3.426)--(1.230,3.411)--(1.243,3.396)--(1.256,3.380)%
  --(1.269,3.365)--(1.282,3.350)--(1.295,3.334)--(1.308,3.319)--(1.321,3.304)--(1.334,3.289)%
  --(1.347,3.274)--(1.360,3.259)--(1.373,3.244)--(1.386,3.229)--(1.399,3.214)--(1.412,3.199)%
  --(1.425,3.184)--(1.438,3.169)--(1.451,3.155)--(1.464,3.140)--(1.477,3.125)--(1.490,3.111)%
  --(1.503,3.096)--(1.516,3.082)--(1.529,3.067)--(1.542,3.053)--(1.555,3.038)--(1.568,3.024)%
  --(1.581,3.010)--(1.594,2.996)--(1.607,2.982)--(1.620,2.967)--(1.633,2.953)--(1.646,2.939)%
  --(1.659,2.925)--(1.672,2.911)--(1.686,2.898)--(1.699,2.884)--(1.712,2.870)--(1.725,2.856)%
  --(1.738,2.843)--(1.751,2.829)--(1.764,2.815)--(1.777,2.802)--(1.790,2.788)--(1.803,2.775)%
  --(1.816,2.762)--(1.829,2.748)--(1.842,2.735)--(1.855,2.722)--(1.868,2.709)--(1.881,2.696)%
  --(1.894,2.683)--(1.907,2.670)--(1.920,2.657)--(1.933,2.644)--(1.946,2.631)--(1.959,2.618)%
  --(1.972,2.605)--(1.985,2.593)--(1.998,2.580)--(2.011,2.567)--(2.024,2.555)--(2.037,2.542)%
  --(2.050,2.530)--(2.063,2.518)--(2.076,2.505)--(2.089,2.493)--(2.102,2.481)--(2.115,2.469)%
  --(2.128,2.457)--(2.141,2.445)--(2.154,2.433)--(2.167,2.421)--(2.180,2.409)--(2.193,2.397)%
  --(2.206,2.386)--(2.219,2.374)--(2.232,2.362)--(2.245,2.351)--(2.258,2.339)--(2.271,2.328)%
  --(2.284,2.317)--(2.297,2.305)--(2.310,2.294)--(2.323,2.283)--(2.336,2.272)--(2.349,2.261)%
  --(2.362,2.250)--(2.375,2.239)--(2.388,2.228)--(2.401,2.217)--(2.414,2.206)--(2.427,2.196)%
  --(2.440,2.203);
\draw[gp path] (3.805,2.203)--(3.805,4.694);
\draw[gp path] (7.720,4.694)--(7.720,1.153);
\node[gp node left,font={\fontsize{10pt}{12pt}\selectfont}] at (1.456,5.268) {\LARGE $\rho$};
\node[gp node left,font={\fontsize{10pt}{12pt}\selectfont}] at (5.740,5.268) {\large $\alpha = \pi$};
%% coordinates of the plot area
\gpdefrectangularnode{gp plot 1}{\pgfpoint{1.196cm}{0.985cm}}{\pgfpoint{7.947cm}{5.631cm}}
\end{tikzpicture}
%% gnuplot variables
} & 
\resizebox{0.5\linewidth}{!}{\tikzsetnextfilename{fast_coplanar_a_crsol_00512_6}\begin{tikzpicture}[gnuplot]
%% generated with GNUPLOT 4.6p4 (Lua 5.1; terminal rev. 99, script rev. 100)
%% Fri 22 Aug 2014 11:40:26 AM EDT
\path (0.000,0.000) rectangle (8.500,6.000);
\gpfill{rgb color={1.000,1.000,1.000}} (1.196,0.985)--(7.946,0.985)--(7.946,5.630)--(1.196,5.630)--cycle;
\gpcolor{color=gp lt color border}
\gpsetlinetype{gp lt border}
\gpsetlinewidth{1.00}
\draw[gp path] (1.196,0.985)--(1.196,5.630)--(7.946,5.630)--(7.946,0.985)--cycle;
\gpcolor{color=gp lt color axes}
\gpsetlinetype{gp lt axes}
\gpsetlinewidth{2.00}
\draw[gp path] (1.196,0.985)--(7.947,0.985);
\gpcolor{color=gp lt color border}
\gpsetlinetype{gp lt border}
\draw[gp path] (1.196,0.985)--(1.268,0.985);
\draw[gp path] (7.947,0.985)--(7.875,0.985);
\gpcolor{rgb color={0.000,0.000,0.000}}
\node[gp node right,font={\fontsize{10pt}{12pt}\selectfont}] at (1.012,0.985) {-0.4};
\gpcolor{color=gp lt color axes}
\gpsetlinetype{gp lt axes}
\draw[gp path] (1.196,1.759)--(7.947,1.759);
\gpcolor{color=gp lt color border}
\gpsetlinetype{gp lt border}
\draw[gp path] (1.196,1.759)--(1.268,1.759);
\draw[gp path] (7.947,1.759)--(7.875,1.759);
\gpcolor{rgb color={0.000,0.000,0.000}}
\node[gp node right,font={\fontsize{10pt}{12pt}\selectfont}] at (1.012,1.759) {-0.2};
\gpcolor{color=gp lt color axes}
\gpsetlinetype{gp lt axes}
\draw[gp path] (1.196,2.534)--(7.947,2.534);
\gpcolor{color=gp lt color border}
\gpsetlinetype{gp lt border}
\draw[gp path] (1.196,2.534)--(1.268,2.534);
\draw[gp path] (7.947,2.534)--(7.875,2.534);
\gpcolor{rgb color={0.000,0.000,0.000}}
\node[gp node right,font={\fontsize{10pt}{12pt}\selectfont}] at (1.012,2.534) {0};
\gpcolor{color=gp lt color axes}
\gpsetlinetype{gp lt axes}
\draw[gp path] (1.196,3.308)--(7.947,3.308);
\gpcolor{color=gp lt color border}
\gpsetlinetype{gp lt border}
\draw[gp path] (1.196,3.308)--(1.268,3.308);
\draw[gp path] (7.947,3.308)--(7.875,3.308);
\gpcolor{rgb color={0.000,0.000,0.000}}
\node[gp node right,font={\fontsize{10pt}{12pt}\selectfont}] at (1.012,3.308) {0.2};
\gpcolor{color=gp lt color axes}
\gpsetlinetype{gp lt axes}
\draw[gp path] (1.196,4.082)--(7.947,4.082);
\gpcolor{color=gp lt color border}
\gpsetlinetype{gp lt border}
\draw[gp path] (1.196,4.082)--(1.268,4.082);
\draw[gp path] (7.947,4.082)--(7.875,4.082);
\gpcolor{rgb color={0.000,0.000,0.000}}
\node[gp node right,font={\fontsize{10pt}{12pt}\selectfont}] at (1.012,4.082) {0.4};
\gpcolor{color=gp lt color axes}
\gpsetlinetype{gp lt axes}
\draw[gp path] (1.196,4.857)--(7.947,4.857);
\gpcolor{color=gp lt color border}
\gpsetlinetype{gp lt border}
\draw[gp path] (1.196,4.857)--(1.268,4.857);
\draw[gp path] (7.947,4.857)--(7.875,4.857);
\gpcolor{rgb color={0.000,0.000,0.000}}
\node[gp node right,font={\fontsize{10pt}{12pt}\selectfont}] at (1.012,4.857) {0.6};
\gpcolor{color=gp lt color axes}
\gpsetlinetype{gp lt axes}
\draw[gp path] (1.196,5.631)--(7.947,5.631);
\gpcolor{color=gp lt color border}
\gpsetlinetype{gp lt border}
\draw[gp path] (1.196,5.631)--(1.268,5.631);
\draw[gp path] (7.947,5.631)--(7.875,5.631);
\gpcolor{rgb color={0.000,0.000,0.000}}
\node[gp node right,font={\fontsize{10pt}{12pt}\selectfont}] at (1.012,5.631) {0.8};
\gpcolor{color=gp lt color axes}
\gpsetlinetype{gp lt axes}
\draw[gp path] (1.196,0.985)--(1.196,5.631);
\gpcolor{color=gp lt color border}
\gpsetlinetype{gp lt border}
\draw[gp path] (1.196,0.985)--(1.196,1.057);
\draw[gp path] (1.196,5.631)--(1.196,5.559);
\gpcolor{rgb color={0.000,0.000,0.000}}
\node[gp node center,font={\fontsize{10pt}{12pt}\selectfont}] at (1.196,0.677) {0.3};
\gpcolor{color=gp lt color axes}
\gpsetlinetype{gp lt axes}
\draw[gp path] (2.494,0.985)--(2.494,5.631);
\gpcolor{color=gp lt color border}
\gpsetlinetype{gp lt border}
\draw[gp path] (2.494,0.985)--(2.494,1.057);
\draw[gp path] (2.494,5.631)--(2.494,5.559);
\gpcolor{rgb color={0.000,0.000,0.000}}
\node[gp node center,font={\fontsize{10pt}{12pt}\selectfont}] at (2.494,0.677) {0.35};
\gpcolor{color=gp lt color axes}
\gpsetlinetype{gp lt axes}
\draw[gp path] (3.793,0.985)--(3.793,5.631);
\gpcolor{color=gp lt color border}
\gpsetlinetype{gp lt border}
\draw[gp path] (3.793,0.985)--(3.793,1.057);
\draw[gp path] (3.793,5.631)--(3.793,5.559);
\gpcolor{rgb color={0.000,0.000,0.000}}
\node[gp node center,font={\fontsize{10pt}{12pt}\selectfont}] at (3.793,0.677) {0.4};
\gpcolor{color=gp lt color axes}
\gpsetlinetype{gp lt axes}
\draw[gp path] (5.091,0.985)--(5.091,5.631);
\gpcolor{color=gp lt color border}
\gpsetlinetype{gp lt border}
\draw[gp path] (5.091,0.985)--(5.091,1.057);
\draw[gp path] (5.091,5.631)--(5.091,5.559);
\gpcolor{rgb color={0.000,0.000,0.000}}
\node[gp node center,font={\fontsize{10pt}{12pt}\selectfont}] at (5.091,0.677) {0.45};
\gpcolor{color=gp lt color axes}
\gpsetlinetype{gp lt axes}
\draw[gp path] (6.389,0.985)--(6.389,5.631);
\gpcolor{color=gp lt color border}
\gpsetlinetype{gp lt border}
\draw[gp path] (6.389,0.985)--(6.389,1.057);
\draw[gp path] (6.389,5.631)--(6.389,5.559);
\gpcolor{rgb color={0.000,0.000,0.000}}
\node[gp node center,font={\fontsize{10pt}{12pt}\selectfont}] at (6.389,0.677) {0.5};
\gpcolor{color=gp lt color axes}
\gpsetlinetype{gp lt axes}
\draw[gp path] (7.687,0.985)--(7.687,5.631);
\gpcolor{color=gp lt color border}
\gpsetlinetype{gp lt border}
\draw[gp path] (7.687,0.985)--(7.687,1.057);
\draw[gp path] (7.687,5.631)--(7.687,5.559);
\gpcolor{rgb color={0.000,0.000,0.000}}
\node[gp node center,font={\fontsize{10pt}{12pt}\selectfont}] at (7.687,0.677) {0.55};
\gpcolor{color=gp lt color border}
\draw[gp path] (1.196,5.631)--(1.196,0.985)--(7.947,0.985)--(7.947,5.631)--cycle;
\gpcolor{rgb color={0.000,0.000,0.000}}
\node[gp node center,font={\fontsize{10pt}{12pt}\selectfont}] at (4.571,0.215) {\large $x$};
\gpcolor{rgb color={0.502,0.502,0.502}}
\gpsetlinewidth{0.50}
\gpsetpointsize{2.67}
\gppoint{gp mark 7}{(1.242,4.926)}
\gppoint{gp mark 7}{(1.292,4.888)}
\gppoint{gp mark 7}{(1.343,4.849)}
\gppoint{gp mark 7}{(1.394,4.810)}
\gppoint{gp mark 7}{(1.444,4.771)}
\gppoint{gp mark 7}{(1.495,4.732)}
\gppoint{gp mark 7}{(1.546,4.692)}
\gppoint{gp mark 7}{(1.597,4.651)}
\gppoint{gp mark 7}{(1.647,4.611)}
\gppoint{gp mark 7}{(1.698,4.570)}
\gppoint{gp mark 7}{(1.749,4.528)}
\gppoint{gp mark 7}{(1.799,4.486)}
\gppoint{gp mark 7}{(1.850,4.443)}
\gppoint{gp mark 7}{(1.901,4.400)}
\gppoint{gp mark 7}{(1.952,4.356)}
\gppoint{gp mark 7}{(2.002,4.311)}
\gppoint{gp mark 7}{(2.053,4.266)}
\gppoint{gp mark 7}{(2.104,4.220)}
\gppoint{gp mark 7}{(2.154,4.173)}
\gppoint{gp mark 7}{(2.205,4.124)}
\gppoint{gp mark 7}{(2.256,4.075)}
\gppoint{gp mark 7}{(2.307,4.025)}
\gppoint{gp mark 7}{(2.357,3.973)}
\gppoint{gp mark 7}{(2.408,3.919)}
\gppoint{gp mark 7}{(2.459,3.864)}
\gppoint{gp mark 7}{(2.509,3.806)}
\gppoint{gp mark 7}{(2.560,3.747)}
\gppoint{gp mark 7}{(2.611,3.685)}
\gppoint{gp mark 7}{(2.662,3.619)}
\gppoint{gp mark 7}{(2.712,3.549)}
\gppoint{gp mark 7}{(2.763,3.469)}
\gppoint{gp mark 7}{(2.814,3.366)}
\gppoint{gp mark 7}{(2.864,3.196)}
\gppoint{gp mark 7}{(2.915,2.817)}
\gppoint{gp mark 7}{(2.966,1.989)}
\gppoint{gp mark 7}{(3.017,1.302)}
\gppoint{gp mark 7}{(3.067,1.204)}
\gppoint{gp mark 7}{(3.118,1.198)}
\gppoint{gp mark 7}{(3.169,1.198)}
\gppoint{gp mark 7}{(3.219,1.198)}
\gppoint{gp mark 7}{(3.270,1.198)}
\gppoint{gp mark 7}{(3.321,1.198)}
\gppoint{gp mark 7}{(3.372,1.197)}
\gppoint{gp mark 7}{(3.422,1.196)}
\gppoint{gp mark 7}{(3.473,1.196)}
\gppoint{gp mark 7}{(3.524,1.196)}
\gppoint{gp mark 7}{(3.574,1.203)}
\gppoint{gp mark 7}{(3.625,1.258)}
\gppoint{gp mark 7}{(3.676,1.458)}
\gppoint{gp mark 7}{(3.727,1.597)}
\gppoint{gp mark 7}{(3.777,1.608)}
\gppoint{gp mark 7}{(3.828,1.610)}
\gppoint{gp mark 7}{(3.879,1.610)}
\gppoint{gp mark 7}{(3.929,1.610)}
\gppoint{gp mark 7}{(3.980,1.609)}
\gppoint{gp mark 7}{(4.031,1.609)}
\gppoint{gp mark 7}{(4.082,1.609)}
\gppoint{gp mark 7}{(4.132,1.610)}
\gppoint{gp mark 7}{(4.183,1.610)}
\gppoint{gp mark 7}{(4.234,1.610)}
\gppoint{gp mark 7}{(4.284,1.609)}
\gppoint{gp mark 7}{(4.335,1.609)}
\gppoint{gp mark 7}{(4.386,1.610)}
\gppoint{gp mark 7}{(4.437,1.610)}
\gppoint{gp mark 7}{(4.487,1.610)}
\gppoint{gp mark 7}{(4.538,1.610)}
\gppoint{gp mark 7}{(4.589,1.610)}
\gppoint{gp mark 7}{(4.639,1.610)}
\gppoint{gp mark 7}{(4.690,1.610)}
\gppoint{gp mark 7}{(4.741,1.610)}
\gppoint{gp mark 7}{(4.792,1.610)}
\gppoint{gp mark 7}{(4.842,1.609)}
\gppoint{gp mark 7}{(4.893,1.609)}
\gppoint{gp mark 7}{(4.944,1.610)}
\gppoint{gp mark 7}{(4.994,1.610)}
\gppoint{gp mark 7}{(5.045,1.609)}
\gppoint{gp mark 7}{(5.096,1.609)}
\gppoint{gp mark 7}{(5.147,1.609)}
\gppoint{gp mark 7}{(5.197,1.609)}
\gppoint{gp mark 7}{(5.248,1.609)}
\gppoint{gp mark 7}{(5.299,1.609)}
\gppoint{gp mark 7}{(5.349,1.609)}
\gppoint{gp mark 7}{(5.400,1.609)}
\gppoint{gp mark 7}{(5.451,1.609)}
\gppoint{gp mark 7}{(5.502,1.610)}
\gppoint{gp mark 7}{(5.552,1.610)}
\gppoint{gp mark 7}{(5.603,1.610)}
\gppoint{gp mark 7}{(5.654,1.610)}
\gppoint{gp mark 7}{(5.704,1.610)}
\gppoint{gp mark 7}{(5.755,1.610)}
\gppoint{gp mark 7}{(5.806,1.610)}
\gppoint{gp mark 7}{(5.857,1.609)}
\gppoint{gp mark 7}{(5.907,1.609)}
\gppoint{gp mark 7}{(5.958,1.609)}
\gppoint{gp mark 7}{(6.009,1.609)}
\gppoint{gp mark 7}{(6.059,1.609)}
\gppoint{gp mark 7}{(6.110,1.609)}
\gppoint{gp mark 7}{(6.161,1.609)}
\gppoint{gp mark 7}{(6.212,1.609)}
\gppoint{gp mark 7}{(6.262,1.609)}
\gppoint{gp mark 7}{(6.313,1.609)}
\gppoint{gp mark 7}{(6.364,1.609)}
\gppoint{gp mark 7}{(6.414,1.610)}
\gppoint{gp mark 7}{(6.465,1.610)}
\gppoint{gp mark 7}{(6.516,1.610)}
\gppoint{gp mark 7}{(6.567,1.610)}
\gppoint{gp mark 7}{(6.617,1.610)}
\gppoint{gp mark 7}{(6.668,1.610)}
\gppoint{gp mark 7}{(6.719,1.610)}
\gppoint{gp mark 7}{(6.769,1.610)}
\gppoint{gp mark 7}{(6.820,1.610)}
\gppoint{gp mark 7}{(6.871,1.610)}
\gppoint{gp mark 7}{(6.922,1.610)}
\gppoint{gp mark 7}{(6.972,1.609)}
\gppoint{gp mark 7}{(7.023,1.609)}
\gppoint{gp mark 7}{(7.074,1.609)}
\gppoint{gp mark 7}{(7.124,1.609)}
\gppoint{gp mark 7}{(7.175,1.609)}
\gppoint{gp mark 7}{(7.226,1.609)}
\gppoint{gp mark 7}{(7.277,1.609)}
\gppoint{gp mark 7}{(7.327,1.609)}
\gppoint{gp mark 7}{(7.378,1.609)}
\gppoint{gp mark 7}{(7.429,1.610)}
\gppoint{gp mark 7}{(7.479,1.610)}
\gppoint{gp mark 7}{(7.530,1.610)}
\gppoint{gp mark 7}{(7.581,1.610)}
\gppoint{gp mark 7}{(7.632,1.610)}
\gppoint{gp mark 7}{(7.682,1.610)}
\gppoint{gp mark 7}{(7.733,1.610)}
\gppoint{gp mark 7}{(7.784,1.610)}
\gppoint{gp mark 7}{(7.834,1.610)}
\gppoint{gp mark 7}{(7.885,1.610)}
\gpcolor{rgb color={1.000,0.000,0.000}}
\gpsetpointsize{4.44}
\gppoint{gp mark 7}{(1.242,4.931)}
\gppoint{gp mark 7}{(1.292,4.893)}
\gppoint{gp mark 7}{(1.343,4.855)}
\gppoint{gp mark 7}{(1.394,4.817)}
\gppoint{gp mark 7}{(1.444,4.778)}
\gppoint{gp mark 7}{(1.495,4.739)}
\gppoint{gp mark 7}{(1.546,4.700)}
\gppoint{gp mark 7}{(1.597,4.661)}
\gppoint{gp mark 7}{(1.647,4.621)}
\gppoint{gp mark 7}{(1.698,4.581)}
\gppoint{gp mark 7}{(1.749,4.541)}
\gppoint{gp mark 7}{(1.799,4.500)}
\gppoint{gp mark 7}{(1.850,4.459)}
\gppoint{gp mark 7}{(1.901,4.417)}
\gppoint{gp mark 7}{(1.952,4.375)}
\gppoint{gp mark 7}{(2.002,4.333)}
\gppoint{gp mark 7}{(2.053,4.291)}
\gppoint{gp mark 7}{(2.104,4.248)}
\gppoint{gp mark 7}{(2.154,4.205)}
\gppoint{gp mark 7}{(2.205,4.162)}
\gppoint{gp mark 7}{(2.256,4.119)}
\gppoint{gp mark 7}{(2.307,4.077)}
\gppoint{gp mark 7}{(2.357,4.035)}
\gppoint{gp mark 7}{(2.408,3.996)}
\gppoint{gp mark 7}{(2.459,3.960)}
\gppoint{gp mark 7}{(2.509,3.931)}
\gppoint{gp mark 7}{(2.560,3.911)}
\gppoint{gp mark 7}{(2.611,3.899)}
\gppoint{gp mark 7}{(2.662,3.895)}
\gppoint{gp mark 7}{(2.712,3.893)}
\gppoint{gp mark 7}{(2.763,3.891)}
\gppoint{gp mark 7}{(2.814,3.888)}
\gppoint{gp mark 7}{(2.864,3.448)}
\gppoint{gp mark 7}{(2.915,1.956)}
\gppoint{gp mark 7}{(2.966,1.205)}
\gppoint{gp mark 7}{(3.017,1.176)}
\gppoint{gp mark 7}{(3.067,1.179)}
\gppoint{gp mark 7}{(3.118,1.171)}
\gppoint{gp mark 7}{(3.169,1.168)}
\gppoint{gp mark 7}{(3.219,1.170)}
\gppoint{gp mark 7}{(3.270,1.171)}
\gppoint{gp mark 7}{(3.321,1.169)}
\gppoint{gp mark 7}{(3.372,1.167)}
\gppoint{gp mark 7}{(3.422,1.166)}
\gppoint{gp mark 7}{(3.473,1.165)}
\gppoint{gp mark 7}{(3.524,1.164)}
\gppoint{gp mark 7}{(3.574,1.164)}
\gppoint{gp mark 7}{(3.625,1.164)}
\gppoint{gp mark 7}{(3.676,1.168)}
\gppoint{gp mark 7}{(3.727,1.213)}
\gppoint{gp mark 7}{(3.777,1.413)}
\gppoint{gp mark 7}{(3.828,1.591)}
\gppoint{gp mark 7}{(3.879,1.609)}
\gppoint{gp mark 7}{(3.929,1.610)}
\gppoint{gp mark 7}{(3.980,1.610)}
\gppoint{gp mark 7}{(4.031,1.610)}
\gppoint{gp mark 7}{(4.082,1.610)}
\gppoint{gp mark 7}{(4.132,1.610)}
\gppoint{gp mark 7}{(4.183,1.610)}
\gppoint{gp mark 7}{(4.234,1.610)}
\gppoint{gp mark 7}{(4.284,1.610)}
\gppoint{gp mark 7}{(4.335,1.610)}
\gppoint{gp mark 7}{(4.386,1.610)}
\gppoint{gp mark 7}{(4.437,1.610)}
\gppoint{gp mark 7}{(4.487,1.610)}
\gppoint{gp mark 7}{(4.538,1.610)}
\gppoint{gp mark 7}{(4.589,1.610)}
\gppoint{gp mark 7}{(4.639,1.609)}
\gppoint{gp mark 7}{(4.690,1.609)}
\gppoint{gp mark 7}{(4.741,1.610)}
\gppoint{gp mark 7}{(4.792,1.610)}
\gppoint{gp mark 7}{(4.842,1.611)}
\gppoint{gp mark 7}{(4.893,1.610)}
\gppoint{gp mark 7}{(4.944,1.610)}
\gppoint{gp mark 7}{(4.994,1.610)}
\gppoint{gp mark 7}{(5.045,1.610)}
\gppoint{gp mark 7}{(5.096,1.610)}
\gppoint{gp mark 7}{(5.147,1.610)}
\gppoint{gp mark 7}{(5.197,1.610)}
\gppoint{gp mark 7}{(5.248,1.610)}
\gppoint{gp mark 7}{(5.299,1.610)}
\gppoint{gp mark 7}{(5.349,1.610)}
\gppoint{gp mark 7}{(5.400,1.610)}
\gppoint{gp mark 7}{(5.451,1.609)}
\gppoint{gp mark 7}{(5.502,1.609)}
\gppoint{gp mark 7}{(5.552,1.609)}
\gppoint{gp mark 7}{(5.603,1.610)}
\gppoint{gp mark 7}{(5.654,1.610)}
\gppoint{gp mark 7}{(5.704,1.611)}
\gppoint{gp mark 7}{(5.755,1.611)}
\gppoint{gp mark 7}{(5.806,1.610)}
\gppoint{gp mark 7}{(5.857,1.610)}
\gppoint{gp mark 7}{(5.907,1.610)}
\gppoint{gp mark 7}{(5.958,1.610)}
\gppoint{gp mark 7}{(6.009,1.610)}
\gppoint{gp mark 7}{(6.059,1.610)}
\gppoint{gp mark 7}{(6.110,1.610)}
\gppoint{gp mark 7}{(6.161,1.609)}
\gppoint{gp mark 7}{(6.212,1.609)}
\gppoint{gp mark 7}{(6.262,1.609)}
\gppoint{gp mark 7}{(6.313,1.609)}
\gppoint{gp mark 7}{(6.364,1.610)}
\gppoint{gp mark 7}{(6.414,1.610)}
\gppoint{gp mark 7}{(6.465,1.610)}
\gppoint{gp mark 7}{(6.516,1.610)}
\gppoint{gp mark 7}{(6.567,1.610)}
\gppoint{gp mark 7}{(6.617,1.610)}
\gppoint{gp mark 7}{(6.668,1.610)}
\gppoint{gp mark 7}{(6.719,1.610)}
\gppoint{gp mark 7}{(6.769,1.610)}
\gppoint{gp mark 7}{(6.820,1.610)}
\gppoint{gp mark 7}{(6.871,1.610)}
\gppoint{gp mark 7}{(6.922,1.610)}
\gppoint{gp mark 7}{(6.972,1.610)}
\gppoint{gp mark 7}{(7.023,1.609)}
\gppoint{gp mark 7}{(7.074,1.609)}
\gppoint{gp mark 7}{(7.124,1.610)}
\gppoint{gp mark 7}{(7.175,1.610)}
\gppoint{gp mark 7}{(7.226,1.610)}
\gppoint{gp mark 7}{(7.277,1.610)}
\gppoint{gp mark 7}{(7.327,1.610)}
\gppoint{gp mark 7}{(7.378,1.610)}
\gppoint{gp mark 7}{(7.429,1.610)}
\gppoint{gp mark 7}{(7.479,1.610)}
\gppoint{gp mark 7}{(7.530,1.610)}
\gppoint{gp mark 7}{(7.581,1.610)}
\gppoint{gp mark 7}{(7.632,1.610)}
\gppoint{gp mark 7}{(7.682,1.610)}
\gppoint{gp mark 7}{(7.733,1.610)}
\gppoint{gp mark 7}{(7.784,1.610)}
\gppoint{gp mark 7}{(7.834,1.609)}
\gppoint{gp mark 7}{(7.885,1.609)}
\gpcolor{rgb color={0.000,0.000,0.000}}
\gpsetlinetype{gp lt plot 0}
\gpsetlinewidth{4.00}
\draw[gp path] (2.440,3.897)--(2.899,3.897);
\draw[gp path] (2.899,1.171)--(3.805,1.171);
\draw[gp path] (3.805,1.610)--(7.720,1.610);
\draw[gp path] (7.720,1.610)--(7.947,1.610);
\draw[gp path] (1.204,4.831)--(1.217,4.821)--(1.230,4.812)--(1.243,4.802)--(1.256,4.792)%
  --(1.269,4.782)--(1.282,4.772)--(1.295,4.762)--(1.308,4.753)--(1.321,4.743)--(1.334,4.733)%
  --(1.347,4.723)--(1.360,4.713)--(1.373,4.703)--(1.386,4.694)--(1.399,4.684)--(1.412,4.674)%
  --(1.425,4.664)--(1.438,4.654)--(1.451,4.644)--(1.464,4.635)--(1.477,4.625)--(1.490,4.615)%
  --(1.503,4.605)--(1.516,4.595)--(1.529,4.585)--(1.542,4.576)--(1.555,4.566)--(1.568,4.556)%
  --(1.581,4.546)--(1.594,4.536)--(1.607,4.526)--(1.620,4.517)--(1.633,4.507)--(1.646,4.497)%
  --(1.659,4.487)--(1.672,4.477)--(1.686,4.467)--(1.699,4.457)--(1.712,4.448)--(1.725,4.438)%
  --(1.738,4.428)--(1.751,4.418)--(1.764,4.408)--(1.777,4.398)--(1.790,4.389)--(1.803,4.379)%
  --(1.816,4.369)--(1.829,4.359)--(1.842,4.349)--(1.855,4.339)--(1.868,4.330)--(1.881,4.320)%
  --(1.894,4.310)--(1.907,4.300)--(1.920,4.290)--(1.933,4.280)--(1.946,4.271)--(1.959,4.261)%
  --(1.972,4.251)--(1.985,4.241)--(1.998,4.231)--(2.011,4.221)--(2.024,4.212)--(2.037,4.202)%
  --(2.050,4.192)--(2.063,4.182)--(2.076,4.172)--(2.089,4.162)--(2.102,4.153)--(2.115,4.143)%
  --(2.128,4.133)--(2.141,4.123)--(2.154,4.113)--(2.167,4.103)--(2.180,4.093)--(2.193,4.084)%
  --(2.206,4.074)--(2.219,4.064)--(2.232,4.054)--(2.245,4.044)--(2.258,4.034)--(2.271,4.025)%
  --(2.284,4.015)--(2.297,4.005)--(2.310,3.995)--(2.323,3.985)--(2.336,3.975)--(2.349,3.966)%
  --(2.362,3.956)--(2.375,3.946)--(2.388,3.936)--(2.401,3.926)--(2.414,3.916)--(2.427,3.907)%
  --(2.440,3.897);
\draw[gp path] (2.899,3.897)--(2.899,1.171);
\draw[gp path] (3.805,1.171)--(3.805,1.610);
\draw[gp path] (3.793,3.695)--(4.572,3.695);
\gpcolor{rgb color={1.000,0.000,0.000}}
\gpsetlinewidth{0.50}
\gppoint{gp mark 7}{(4.182,2.921)}
\gpcolor{rgb color={0.502,0.502,0.502}}
\gppoint{gp mark 7}{(4.182,2.147)}
\gpcolor{rgb color={0.000,0.000,0.000}}
\node[gp node left,font={\fontsize{10pt}{12pt}\selectfont}] at (1.456,5.166) {\LARGE $B_y$};
\node[gp node left,font={\fontsize{10pt}{12pt}\selectfont}] at (5.740,5.166) {\large $\alpha = \pi$};
\node[gp node left,font={\fontsize{10pt}{12pt}\selectfont}] at (4.831,3.695) {\large exact};
\node[gp node left,font={\fontsize{10pt}{12pt}\selectfont}] at (4.831,2.921) {\large HLLD-CWM};
\node[gp node left,font={\fontsize{10pt}{12pt}\selectfont}] at (4.831,2.147) {\large HLLD};
%% coordinates of the plot area
\gpdefrectangularnode{gp plot 1}{\pgfpoint{1.196cm}{0.985cm}}{\pgfpoint{7.947cm}{5.631cm}}
\end{tikzpicture}
%% gnuplot variables
} \\
\end{tabular}
\caption{The fast rarefaction and rotational discontinuity solution found using HLLD-CWM without the (optional) flux redistribution step, HLLD, and the exact solver using $512$ grid points for (top) a near-coplanar and pseudo-converging case and (bottom) the planar and non-converging (bottom) case.}
\label{fig:fast_coplanar_ab_crsol_512}
\end{figure}

%-----------------------------------------------------------------
% Fast coplanar flux zoomed
%-----------------------------------------------------------------
\begin{figure}[htbp] 
\begin{tabular}{cc}
\resizebox{0.5\linewidth}{!}{\tikzsetnextfilename{fast_coplanar_b_crsol_1}\begin{tikzpicture}[gnuplot]
%% generated with GNUPLOT 4.6p4 (Lua 5.1; terminal rev. 99, script rev. 100)
%% Sat 02 Aug 2014 10:10:41 AM EDT
\path (0.000,0.000) rectangle (8.500,6.000);
\gpfill{rgb color={1.000,1.000,1.000}} (1.196,0.985)--(7.946,0.985)--(7.946,5.630)--(1.196,5.630)--cycle;
\gpcolor{color=gp lt color border}
\gpsetlinetype{gp lt border}
\gpsetlinewidth{1.00}
\draw[gp path] (1.196,0.985)--(1.196,5.630)--(7.946,5.630)--(7.946,0.985)--cycle;
\gpcolor{color=gp lt color axes}
\gpsetlinetype{gp lt axes}
\gpsetlinewidth{2.00}
\draw[gp path] (1.196,1.275)--(7.947,1.275);
\gpcolor{color=gp lt color border}
\gpsetlinetype{gp lt border}
\draw[gp path] (1.196,1.275)--(1.268,1.275);
\draw[gp path] (7.947,1.275)--(7.875,1.275);
\gpcolor{rgb color={0.000,0.000,0.000}}
\node[gp node right,font={\fontsize{10pt}{12pt}\selectfont}] at (1.012,1.275) {0.65};
\gpcolor{color=gp lt color axes}
\gpsetlinetype{gp lt axes}
\draw[gp path] (1.196,2.001)--(7.947,2.001);
\gpcolor{color=gp lt color border}
\gpsetlinetype{gp lt border}
\draw[gp path] (1.196,2.001)--(1.268,2.001);
\draw[gp path] (7.947,2.001)--(7.875,2.001);
\gpcolor{rgb color={0.000,0.000,0.000}}
\node[gp node right,font={\fontsize{10pt}{12pt}\selectfont}] at (1.012,2.001) {0.7};
\gpcolor{color=gp lt color axes}
\gpsetlinetype{gp lt axes}
\draw[gp path] (1.196,2.727)--(7.947,2.727);
\gpcolor{color=gp lt color border}
\gpsetlinetype{gp lt border}
\draw[gp path] (1.196,2.727)--(1.268,2.727);
\draw[gp path] (7.947,2.727)--(7.875,2.727);
\gpcolor{rgb color={0.000,0.000,0.000}}
\node[gp node right,font={\fontsize{10pt}{12pt}\selectfont}] at (1.012,2.727) {0.75};
\gpcolor{color=gp lt color axes}
\gpsetlinetype{gp lt axes}
\draw[gp path] (1.196,3.453)--(7.947,3.453);
\gpcolor{color=gp lt color border}
\gpsetlinetype{gp lt border}
\draw[gp path] (1.196,3.453)--(1.268,3.453);
\draw[gp path] (7.947,3.453)--(7.875,3.453);
\gpcolor{rgb color={0.000,0.000,0.000}}
\node[gp node right,font={\fontsize{10pt}{12pt}\selectfont}] at (1.012,3.453) {0.8};
\gpcolor{color=gp lt color axes}
\gpsetlinetype{gp lt axes}
\draw[gp path] (1.196,4.179)--(7.947,4.179);
\gpcolor{color=gp lt color border}
\gpsetlinetype{gp lt border}
\draw[gp path] (1.196,4.179)--(1.268,4.179);
\draw[gp path] (7.947,4.179)--(7.875,4.179);
\gpcolor{rgb color={0.000,0.000,0.000}}
\node[gp node right,font={\fontsize{10pt}{12pt}\selectfont}] at (1.012,4.179) {0.85};
\gpcolor{color=gp lt color axes}
\gpsetlinetype{gp lt axes}
\draw[gp path] (1.196,4.905)--(7.947,4.905);
\gpcolor{color=gp lt color border}
\gpsetlinetype{gp lt border}
\draw[gp path] (1.196,4.905)--(1.268,4.905);
\draw[gp path] (7.947,4.905)--(7.875,4.905);
\gpcolor{rgb color={0.000,0.000,0.000}}
\node[gp node right,font={\fontsize{10pt}{12pt}\selectfont}] at (1.012,4.905) {0.9};
\gpcolor{color=gp lt color axes}
\gpsetlinetype{gp lt axes}
\draw[gp path] (1.196,5.631)--(7.947,5.631);
\gpcolor{color=gp lt color border}
\gpsetlinetype{gp lt border}
\draw[gp path] (1.196,5.631)--(1.268,5.631);
\draw[gp path] (7.947,5.631)--(7.875,5.631);
\gpcolor{rgb color={0.000,0.000,0.000}}
\node[gp node right,font={\fontsize{10pt}{12pt}\selectfont}] at (1.012,5.631) {0.95};
\gpcolor{color=gp lt color axes}
\gpsetlinetype{gp lt axes}
\draw[gp path] (1.196,0.985)--(1.196,5.631);
\gpcolor{color=gp lt color border}
\gpsetlinetype{gp lt border}
\draw[gp path] (1.196,0.985)--(1.196,1.057);
\draw[gp path] (1.196,5.631)--(1.196,5.559);
\gpcolor{rgb color={0.000,0.000,0.000}}
\node[gp node center,font={\fontsize{10pt}{12pt}\selectfont}] at (1.196,0.677) {0.3};
\gpcolor{color=gp lt color axes}
\gpsetlinetype{gp lt axes}
\draw[gp path] (2.494,0.985)--(2.494,5.631);
\gpcolor{color=gp lt color border}
\gpsetlinetype{gp lt border}
\draw[gp path] (2.494,0.985)--(2.494,1.057);
\draw[gp path] (2.494,5.631)--(2.494,5.559);
\gpcolor{rgb color={0.000,0.000,0.000}}
\node[gp node center,font={\fontsize{10pt}{12pt}\selectfont}] at (2.494,0.677) {0.35};
\gpcolor{color=gp lt color axes}
\gpsetlinetype{gp lt axes}
\draw[gp path] (3.793,0.985)--(3.793,5.631);
\gpcolor{color=gp lt color border}
\gpsetlinetype{gp lt border}
\draw[gp path] (3.793,0.985)--(3.793,1.057);
\draw[gp path] (3.793,5.631)--(3.793,5.559);
\gpcolor{rgb color={0.000,0.000,0.000}}
\node[gp node center,font={\fontsize{10pt}{12pt}\selectfont}] at (3.793,0.677) {0.4};
\gpcolor{color=gp lt color axes}
\gpsetlinetype{gp lt axes}
\draw[gp path] (5.091,0.985)--(5.091,5.631);
\gpcolor{color=gp lt color border}
\gpsetlinetype{gp lt border}
\draw[gp path] (5.091,0.985)--(5.091,1.057);
\draw[gp path] (5.091,5.631)--(5.091,5.559);
\gpcolor{rgb color={0.000,0.000,0.000}}
\node[gp node center,font={\fontsize{10pt}{12pt}\selectfont}] at (5.091,0.677) {0.45};
\gpcolor{color=gp lt color axes}
\gpsetlinetype{gp lt axes}
\draw[gp path] (6.389,0.985)--(6.389,5.631);
\gpcolor{color=gp lt color border}
\gpsetlinetype{gp lt border}
\draw[gp path] (6.389,0.985)--(6.389,1.057);
\draw[gp path] (6.389,5.631)--(6.389,5.559);
\gpcolor{rgb color={0.000,0.000,0.000}}
\node[gp node center,font={\fontsize{10pt}{12pt}\selectfont}] at (6.389,0.677) {0.5};
\gpcolor{color=gp lt color axes}
\gpsetlinetype{gp lt axes}
\draw[gp path] (7.687,0.985)--(7.687,5.631);
\gpcolor{color=gp lt color border}
\gpsetlinetype{gp lt border}
\draw[gp path] (7.687,0.985)--(7.687,1.057);
\draw[gp path] (7.687,5.631)--(7.687,5.559);
\gpcolor{rgb color={0.000,0.000,0.000}}
\node[gp node center,font={\fontsize{10pt}{12pt}\selectfont}] at (7.687,0.677) {0.55};
\gpcolor{color=gp lt color border}
\draw[gp path] (1.196,5.631)--(1.196,0.985)--(7.947,0.985)--(7.947,5.631)--cycle;
\gpcolor{rgb color={0.000,0.000,0.000}}
\node[gp node center,font={\fontsize{10pt}{12pt}\selectfont}] at (4.571,0.215) {\large $x$};
\gpcolor{rgb color={0.502,0.502,0.502}}
\gpsetlinewidth{0.50}
\gpsetpointsize{2.67}
\gppoint{gp mark 7}{(1.210,3.636)}
\gppoint{gp mark 7}{(1.223,3.621)}
\gppoint{gp mark 7}{(1.235,3.605)}
\gppoint{gp mark 7}{(1.248,3.590)}
\gppoint{gp mark 7}{(1.261,3.574)}
\gppoint{gp mark 7}{(1.273,3.559)}
\gppoint{gp mark 7}{(1.286,3.543)}
\gppoint{gp mark 7}{(1.299,3.528)}
\gppoint{gp mark 7}{(1.311,3.513)}
\gppoint{gp mark 7}{(1.324,3.497)}
\gppoint{gp mark 7}{(1.337,3.482)}
\gppoint{gp mark 7}{(1.349,3.466)}
\gppoint{gp mark 7}{(1.362,3.451)}
\gppoint{gp mark 7}{(1.375,3.436)}
\gppoint{gp mark 7}{(1.387,3.420)}
\gppoint{gp mark 7}{(1.400,3.405)}
\gppoint{gp mark 7}{(1.413,3.390)}
\gppoint{gp mark 7}{(1.425,3.375)}
\gppoint{gp mark 7}{(1.438,3.359)}
\gppoint{gp mark 7}{(1.451,3.344)}
\gppoint{gp mark 7}{(1.464,3.329)}
\gppoint{gp mark 7}{(1.476,3.313)}
\gppoint{gp mark 7}{(1.489,3.298)}
\gppoint{gp mark 7}{(1.502,3.283)}
\gppoint{gp mark 7}{(1.514,3.268)}
\gppoint{gp mark 7}{(1.527,3.253)}
\gppoint{gp mark 7}{(1.540,3.237)}
\gppoint{gp mark 7}{(1.552,3.222)}
\gppoint{gp mark 7}{(1.565,3.207)}
\gppoint{gp mark 7}{(1.578,3.192)}
\gppoint{gp mark 7}{(1.590,3.177)}
\gppoint{gp mark 7}{(1.603,3.162)}
\gppoint{gp mark 7}{(1.616,3.147)}
\gppoint{gp mark 7}{(1.628,3.131)}
\gppoint{gp mark 7}{(1.641,3.116)}
\gppoint{gp mark 7}{(1.654,3.101)}
\gppoint{gp mark 7}{(1.666,3.086)}
\gppoint{gp mark 7}{(1.679,3.071)}
\gppoint{gp mark 7}{(1.692,3.056)}
\gppoint{gp mark 7}{(1.704,3.041)}
\gppoint{gp mark 7}{(1.717,3.026)}
\gppoint{gp mark 7}{(1.730,3.011)}
\gppoint{gp mark 7}{(1.742,2.996)}
\gppoint{gp mark 7}{(1.755,2.981)}
\gppoint{gp mark 7}{(1.768,2.966)}
\gppoint{gp mark 7}{(1.780,2.951)}
\gppoint{gp mark 7}{(1.793,2.936)}
\gppoint{gp mark 7}{(1.806,2.921)}
\gppoint{gp mark 7}{(1.819,2.906)}
\gppoint{gp mark 7}{(1.831,2.891)}
\gppoint{gp mark 7}{(1.844,2.876)}
\gppoint{gp mark 7}{(1.857,2.862)}
\gppoint{gp mark 7}{(1.869,2.847)}
\gppoint{gp mark 7}{(1.882,2.832)}
\gppoint{gp mark 7}{(1.895,2.817)}
\gppoint{gp mark 7}{(1.907,2.802)}
\gppoint{gp mark 7}{(1.920,2.787)}
\gppoint{gp mark 7}{(1.933,2.773)}
\gppoint{gp mark 7}{(1.945,2.758)}
\gppoint{gp mark 7}{(1.958,2.743)}
\gppoint{gp mark 7}{(1.971,2.728)}
\gppoint{gp mark 7}{(1.983,2.713)}
\gppoint{gp mark 7}{(1.996,2.699)}
\gppoint{gp mark 7}{(2.009,2.684)}
\gppoint{gp mark 7}{(2.021,2.669)}
\gppoint{gp mark 7}{(2.034,2.654)}
\gppoint{gp mark 7}{(2.047,2.640)}
\gppoint{gp mark 7}{(2.059,2.625)}
\gppoint{gp mark 7}{(2.072,2.610)}
\gppoint{gp mark 7}{(2.085,2.596)}
\gppoint{gp mark 7}{(2.097,2.581)}
\gppoint{gp mark 7}{(2.110,2.566)}
\gppoint{gp mark 7}{(2.123,2.552)}
\gppoint{gp mark 7}{(2.135,2.537)}
\gppoint{gp mark 7}{(2.148,2.522)}
\gppoint{gp mark 7}{(2.161,2.508)}
\gppoint{gp mark 7}{(2.174,2.493)}
\gppoint{gp mark 7}{(2.186,2.479)}
\gppoint{gp mark 7}{(2.199,2.464)}
\gppoint{gp mark 7}{(2.212,2.449)}
\gppoint{gp mark 7}{(2.224,2.435)}
\gppoint{gp mark 7}{(2.237,2.420)}
\gppoint{gp mark 7}{(2.250,2.406)}
\gppoint{gp mark 7}{(2.262,2.391)}
\gppoint{gp mark 7}{(2.275,2.377)}
\gppoint{gp mark 7}{(2.288,2.362)}
\gppoint{gp mark 7}{(2.300,2.348)}
\gppoint{gp mark 7}{(2.313,2.333)}
\gppoint{gp mark 7}{(2.326,2.319)}
\gppoint{gp mark 7}{(2.338,2.304)}
\gppoint{gp mark 7}{(2.351,2.290)}
\gppoint{gp mark 7}{(2.364,2.275)}
\gppoint{gp mark 7}{(2.376,2.261)}
\gppoint{gp mark 7}{(2.389,2.246)}
\gppoint{gp mark 7}{(2.402,2.232)}
\gppoint{gp mark 7}{(2.414,2.218)}
\gppoint{gp mark 7}{(2.427,2.203)}
\gppoint{gp mark 7}{(2.440,2.189)}
\gppoint{gp mark 7}{(2.452,2.175)}
\gppoint{gp mark 7}{(2.465,2.160)}
\gppoint{gp mark 7}{(2.478,2.146)}
\gppoint{gp mark 7}{(2.490,2.131)}
\gppoint{gp mark 7}{(2.503,2.117)}
\gppoint{gp mark 7}{(2.516,2.103)}
\gppoint{gp mark 7}{(2.529,2.089)}
\gppoint{gp mark 7}{(2.541,2.074)}
\gppoint{gp mark 7}{(2.554,2.060)}
\gppoint{gp mark 7}{(2.567,2.046)}
\gppoint{gp mark 7}{(2.579,2.032)}
\gppoint{gp mark 7}{(2.592,2.017)}
\gppoint{gp mark 7}{(2.605,2.003)}
\gppoint{gp mark 7}{(2.617,1.989)}
\gppoint{gp mark 7}{(2.630,1.975)}
\gppoint{gp mark 7}{(2.643,1.961)}
\gppoint{gp mark 7}{(2.655,1.947)}
\gppoint{gp mark 7}{(2.668,1.933)}
\gppoint{gp mark 7}{(2.681,1.919)}
\gppoint{gp mark 7}{(2.693,1.905)}
\gppoint{gp mark 7}{(2.706,1.891)}
\gppoint{gp mark 7}{(2.719,1.877)}
\gppoint{gp mark 7}{(2.731,1.863)}
\gppoint{gp mark 7}{(2.744,1.850)}
\gppoint{gp mark 7}{(2.757,1.836)}
\gppoint{gp mark 7}{(2.769,1.823)}
\gppoint{gp mark 7}{(2.782,1.809)}
\gppoint{gp mark 7}{(2.795,1.794)}
\gppoint{gp mark 7}{(2.807,1.779)}
\gppoint{gp mark 7}{(2.820,1.770)}
\gppoint{gp mark 7}{(2.833,1.768)}
\gppoint{gp mark 7}{(2.845,1.767)}
\gppoint{gp mark 7}{(2.858,1.769)}
\gppoint{gp mark 7}{(2.871,1.781)}
\gppoint{gp mark 7}{(2.883,1.811)}
\gppoint{gp mark 7}{(2.896,1.850)}
\gppoint{gp mark 7}{(2.909,1.924)}
\gppoint{gp mark 7}{(2.922,2.041)}
\gppoint{gp mark 7}{(2.934,2.281)}
\gppoint{gp mark 7}{(2.947,2.432)}
\gppoint{gp mark 7}{(2.960,2.393)}
\gppoint{gp mark 7}{(2.972,2.348)}
\gppoint{gp mark 7}{(2.985,2.349)}
\gppoint{gp mark 7}{(2.998,2.356)}
\gppoint{gp mark 7}{(3.010,2.357)}
\gppoint{gp mark 7}{(3.023,2.357)}
\gppoint{gp mark 7}{(3.036,2.359)}
\gppoint{gp mark 7}{(3.048,2.361)}
\gppoint{gp mark 7}{(3.061,2.362)}
\gppoint{gp mark 7}{(3.074,2.363)}
\gppoint{gp mark 7}{(3.086,2.365)}
\gppoint{gp mark 7}{(3.099,2.366)}
\gppoint{gp mark 7}{(3.112,2.367)}
\gppoint{gp mark 7}{(3.124,2.368)}
\gppoint{gp mark 7}{(3.137,2.369)}
\gppoint{gp mark 7}{(3.150,2.369)}
\gppoint{gp mark 7}{(3.162,2.369)}
\gppoint{gp mark 7}{(3.175,2.369)}
\gppoint{gp mark 7}{(3.188,2.369)}
\gppoint{gp mark 7}{(3.200,2.369)}
\gppoint{gp mark 7}{(3.213,2.370)}
\gppoint{gp mark 7}{(3.226,2.371)}
\gppoint{gp mark 7}{(3.238,2.371)}
\gppoint{gp mark 7}{(3.251,2.372)}
\gppoint{gp mark 7}{(3.264,2.372)}
\gppoint{gp mark 7}{(3.277,2.372)}
\gppoint{gp mark 7}{(3.289,2.372)}
\gppoint{gp mark 7}{(3.302,2.372)}
\gppoint{gp mark 7}{(3.315,2.371)}
\gppoint{gp mark 7}{(3.327,2.371)}
\gppoint{gp mark 7}{(3.340,2.370)}
\gppoint{gp mark 7}{(3.353,2.370)}
\gppoint{gp mark 7}{(3.365,2.371)}
\gppoint{gp mark 7}{(3.378,2.371)}
\gppoint{gp mark 7}{(3.391,2.372)}
\gppoint{gp mark 7}{(3.403,2.373)}
\gppoint{gp mark 7}{(3.416,2.374)}
\gppoint{gp mark 7}{(3.429,2.375)}
\gppoint{gp mark 7}{(3.441,2.375)}
\gppoint{gp mark 7}{(3.454,2.375)}
\gppoint{gp mark 7}{(3.467,2.375)}
\gppoint{gp mark 7}{(3.479,2.375)}
\gppoint{gp mark 7}{(3.492,2.374)}
\gppoint{gp mark 7}{(3.505,2.373)}
\gppoint{gp mark 7}{(3.517,2.373)}
\gppoint{gp mark 7}{(3.530,2.373)}
\gppoint{gp mark 7}{(3.543,2.373)}
\gppoint{gp mark 7}{(3.555,2.374)}
\gppoint{gp mark 7}{(3.568,2.374)}
\gppoint{gp mark 7}{(3.581,2.375)}
\gppoint{gp mark 7}{(3.593,2.376)}
\gppoint{gp mark 7}{(3.606,2.376)}
\gppoint{gp mark 7}{(3.619,2.376)}
\gppoint{gp mark 7}{(3.632,2.376)}
\gppoint{gp mark 7}{(3.644,2.376)}
\gppoint{gp mark 7}{(3.657,2.375)}
\gppoint{gp mark 7}{(3.670,2.374)}
\gppoint{gp mark 7}{(3.682,2.374)}
\gppoint{gp mark 7}{(3.695,2.374)}
\gppoint{gp mark 7}{(3.708,2.375)}
\gppoint{gp mark 7}{(3.720,2.383)}
\gppoint{gp mark 7}{(3.733,2.473)}
\gppoint{gp mark 7}{(3.746,3.140)}
\gppoint{gp mark 7}{(3.758,4.347)}
\gppoint{gp mark 7}{(3.771,4.639)}
\gppoint{gp mark 7}{(3.784,4.652)}
\gppoint{gp mark 7}{(3.796,4.646)}
\gppoint{gp mark 7}{(3.809,4.644)}
\gppoint{gp mark 7}{(3.822,4.646)}
\gppoint{gp mark 7}{(3.834,4.646)}
\gppoint{gp mark 7}{(3.847,4.645)}
\gppoint{gp mark 7}{(3.860,4.645)}
\gppoint{gp mark 7}{(3.872,4.645)}
\gppoint{gp mark 7}{(3.885,4.645)}
\gppoint{gp mark 7}{(3.898,4.644)}
\gppoint{gp mark 7}{(3.910,4.644)}
\gppoint{gp mark 7}{(3.923,4.644)}
\gppoint{gp mark 7}{(3.936,4.644)}
\gppoint{gp mark 7}{(3.948,4.644)}
\gppoint{gp mark 7}{(3.961,4.644)}
\gppoint{gp mark 7}{(3.974,4.644)}
\gppoint{gp mark 7}{(3.987,4.644)}
\gppoint{gp mark 7}{(3.999,4.644)}
\gppoint{gp mark 7}{(4.012,4.644)}
\gppoint{gp mark 7}{(4.025,4.645)}
\gppoint{gp mark 7}{(4.037,4.645)}
\gppoint{gp mark 7}{(4.050,4.645)}
\gppoint{gp mark 7}{(4.063,4.645)}
\gppoint{gp mark 7}{(4.075,4.645)}
\gppoint{gp mark 7}{(4.088,4.645)}
\gppoint{gp mark 7}{(4.101,4.645)}
\gppoint{gp mark 7}{(4.113,4.645)}
\gppoint{gp mark 7}{(4.126,4.645)}
\gppoint{gp mark 7}{(4.139,4.645)}
\gppoint{gp mark 7}{(4.151,4.645)}
\gppoint{gp mark 7}{(4.164,4.645)}
\gppoint{gp mark 7}{(4.177,4.645)}
\gppoint{gp mark 7}{(4.189,4.645)}
\gppoint{gp mark 7}{(4.202,4.645)}
\gppoint{gp mark 7}{(4.215,4.645)}
\gppoint{gp mark 7}{(4.227,4.645)}
\gppoint{gp mark 7}{(4.240,4.645)}
\gppoint{gp mark 7}{(4.253,4.644)}
\gppoint{gp mark 7}{(4.265,4.644)}
\gppoint{gp mark 7}{(4.278,4.644)}
\gppoint{gp mark 7}{(4.291,4.644)}
\gppoint{gp mark 7}{(4.303,4.645)}
\gppoint{gp mark 7}{(4.316,4.645)}
\gppoint{gp mark 7}{(4.329,4.645)}
\gppoint{gp mark 7}{(4.342,4.645)}
\gppoint{gp mark 7}{(4.354,4.645)}
\gppoint{gp mark 7}{(4.367,4.644)}
\gppoint{gp mark 7}{(4.380,4.644)}
\gppoint{gp mark 7}{(4.392,4.644)}
\gppoint{gp mark 7}{(4.405,4.644)}
\gppoint{gp mark 7}{(4.418,4.644)}
\gppoint{gp mark 7}{(4.430,4.644)}
\gppoint{gp mark 7}{(4.443,4.644)}
\gppoint{gp mark 7}{(4.456,4.644)}
\gppoint{gp mark 7}{(4.468,4.644)}
\gppoint{gp mark 7}{(4.481,4.644)}
\gppoint{gp mark 7}{(4.494,4.644)}
\gppoint{gp mark 7}{(4.506,4.644)}
\gppoint{gp mark 7}{(4.519,4.644)}
\gppoint{gp mark 7}{(4.532,4.644)}
\gppoint{gp mark 7}{(4.544,4.644)}
\gppoint{gp mark 7}{(4.557,4.644)}
\gppoint{gp mark 7}{(4.570,4.644)}
\gppoint{gp mark 7}{(4.582,4.644)}
\gppoint{gp mark 7}{(4.595,4.644)}
\gppoint{gp mark 7}{(4.608,4.644)}
\gppoint{gp mark 7}{(4.620,4.644)}
\gppoint{gp mark 7}{(4.633,4.644)}
\gppoint{gp mark 7}{(4.646,4.644)}
\gppoint{gp mark 7}{(4.658,4.644)}
\gppoint{gp mark 7}{(4.671,4.644)}
\gppoint{gp mark 7}{(4.684,4.644)}
\gppoint{gp mark 7}{(4.697,4.644)}
\gppoint{gp mark 7}{(4.709,4.644)}
\gppoint{gp mark 7}{(4.722,4.644)}
\gppoint{gp mark 7}{(4.735,4.644)}
\gppoint{gp mark 7}{(4.747,4.644)}
\gppoint{gp mark 7}{(4.760,4.644)}
\gppoint{gp mark 7}{(4.773,4.644)}
\gppoint{gp mark 7}{(4.785,4.644)}
\gppoint{gp mark 7}{(4.798,4.644)}
\gppoint{gp mark 7}{(4.811,4.643)}
\gppoint{gp mark 7}{(4.823,4.643)}
\gppoint{gp mark 7}{(4.836,4.644)}
\gppoint{gp mark 7}{(4.849,4.644)}
\gppoint{gp mark 7}{(4.861,4.644)}
\gppoint{gp mark 7}{(4.874,4.644)}
\gppoint{gp mark 7}{(4.887,4.644)}
\gppoint{gp mark 7}{(4.899,4.644)}
\gppoint{gp mark 7}{(4.912,4.644)}
\gppoint{gp mark 7}{(4.925,4.643)}
\gppoint{gp mark 7}{(4.937,4.643)}
\gppoint{gp mark 7}{(4.950,4.643)}
\gppoint{gp mark 7}{(4.963,4.643)}
\gppoint{gp mark 7}{(4.975,4.643)}
\gppoint{gp mark 7}{(4.988,4.643)}
\gppoint{gp mark 7}{(5.001,4.644)}
\gppoint{gp mark 7}{(5.013,4.644)}
\gppoint{gp mark 7}{(5.026,4.644)}
\gppoint{gp mark 7}{(5.039,4.643)}
\gppoint{gp mark 7}{(5.052,4.643)}
\gppoint{gp mark 7}{(5.064,4.643)}
\gppoint{gp mark 7}{(5.077,4.643)}
\gppoint{gp mark 7}{(5.090,4.643)}
\gppoint{gp mark 7}{(5.102,4.643)}
\gppoint{gp mark 7}{(5.115,4.643)}
\gppoint{gp mark 7}{(5.128,4.644)}
\gppoint{gp mark 7}{(5.140,4.644)}
\gppoint{gp mark 7}{(5.153,4.644)}
\gppoint{gp mark 7}{(5.166,4.644)}
\gppoint{gp mark 7}{(5.178,4.644)}
\gppoint{gp mark 7}{(5.191,4.644)}
\gppoint{gp mark 7}{(5.204,4.644)}
\gppoint{gp mark 7}{(5.216,4.644)}
\gppoint{gp mark 7}{(5.229,4.644)}
\gppoint{gp mark 7}{(5.242,4.644)}
\gppoint{gp mark 7}{(5.254,4.644)}
\gppoint{gp mark 7}{(5.267,4.644)}
\gppoint{gp mark 7}{(5.280,4.644)}
\gppoint{gp mark 7}{(5.292,4.644)}
\gppoint{gp mark 7}{(5.305,4.644)}
\gppoint{gp mark 7}{(5.318,4.644)}
\gppoint{gp mark 7}{(5.330,4.644)}
\gppoint{gp mark 7}{(5.343,4.643)}
\gppoint{gp mark 7}{(5.356,4.643)}
\gppoint{gp mark 7}{(5.368,4.643)}
\gppoint{gp mark 7}{(5.381,4.643)}
\gppoint{gp mark 7}{(5.394,4.643)}
\gppoint{gp mark 7}{(5.407,4.643)}
\gppoint{gp mark 7}{(5.419,4.643)}
\gppoint{gp mark 7}{(5.432,4.643)}
\gppoint{gp mark 7}{(5.445,4.643)}
\gppoint{gp mark 7}{(5.457,4.643)}
\gppoint{gp mark 7}{(5.470,4.643)}
\gppoint{gp mark 7}{(5.483,4.643)}
\gppoint{gp mark 7}{(5.495,4.643)}
\gppoint{gp mark 7}{(5.508,4.643)}
\gppoint{gp mark 7}{(5.521,4.643)}
\gppoint{gp mark 7}{(5.533,4.643)}
\gppoint{gp mark 7}{(5.546,4.643)}
\gppoint{gp mark 7}{(5.559,4.643)}
\gppoint{gp mark 7}{(5.571,4.643)}
\gppoint{gp mark 7}{(5.584,4.643)}
\gppoint{gp mark 7}{(5.597,4.643)}
\gppoint{gp mark 7}{(5.609,4.643)}
\gppoint{gp mark 7}{(5.622,4.643)}
\gppoint{gp mark 7}{(5.635,4.643)}
\gppoint{gp mark 7}{(5.647,4.643)}
\gppoint{gp mark 7}{(5.660,4.643)}
\gppoint{gp mark 7}{(5.673,4.643)}
\gppoint{gp mark 7}{(5.685,4.643)}
\gppoint{gp mark 7}{(5.698,4.643)}
\gppoint{gp mark 7}{(5.711,4.643)}
\gppoint{gp mark 7}{(5.723,4.643)}
\gppoint{gp mark 7}{(5.736,4.643)}
\gppoint{gp mark 7}{(5.749,4.643)}
\gppoint{gp mark 7}{(5.761,4.643)}
\gppoint{gp mark 7}{(5.774,4.643)}
\gppoint{gp mark 7}{(5.787,4.643)}
\gppoint{gp mark 7}{(5.800,4.643)}
\gppoint{gp mark 7}{(5.812,4.643)}
\gppoint{gp mark 7}{(5.825,4.643)}
\gppoint{gp mark 7}{(5.838,4.643)}
\gppoint{gp mark 7}{(5.850,4.643)}
\gppoint{gp mark 7}{(5.863,4.643)}
\gppoint{gp mark 7}{(5.876,4.643)}
\gppoint{gp mark 7}{(5.888,4.643)}
\gppoint{gp mark 7}{(5.901,4.643)}
\gppoint{gp mark 7}{(5.914,4.643)}
\gppoint{gp mark 7}{(5.926,4.643)}
\gppoint{gp mark 7}{(5.939,4.643)}
\gppoint{gp mark 7}{(5.952,4.642)}
\gppoint{gp mark 7}{(5.964,4.642)}
\gppoint{gp mark 7}{(5.977,4.642)}
\gppoint{gp mark 7}{(5.990,4.642)}
\gppoint{gp mark 7}{(6.002,4.642)}
\gppoint{gp mark 7}{(6.015,4.642)}
\gppoint{gp mark 7}{(6.028,4.642)}
\gppoint{gp mark 7}{(6.040,4.642)}
\gppoint{gp mark 7}{(6.053,4.642)}
\gppoint{gp mark 7}{(6.066,4.642)}
\gppoint{gp mark 7}{(6.078,4.642)}
\gppoint{gp mark 7}{(6.091,4.642)}
\gppoint{gp mark 7}{(6.104,4.642)}
\gppoint{gp mark 7}{(6.116,4.642)}
\gppoint{gp mark 7}{(6.129,4.643)}
\gppoint{gp mark 7}{(6.142,4.643)}
\gppoint{gp mark 7}{(6.155,4.643)}
\gppoint{gp mark 7}{(6.167,4.643)}
\gppoint{gp mark 7}{(6.180,4.643)}
\gppoint{gp mark 7}{(6.193,4.643)}
\gppoint{gp mark 7}{(6.205,4.642)}
\gppoint{gp mark 7}{(6.218,4.642)}
\gppoint{gp mark 7}{(6.231,4.642)}
\gppoint{gp mark 7}{(6.243,4.642)}
\gppoint{gp mark 7}{(6.256,4.642)}
\gppoint{gp mark 7}{(6.269,4.642)}
\gppoint{gp mark 7}{(6.281,4.642)}
\gppoint{gp mark 7}{(6.294,4.642)}
\gppoint{gp mark 7}{(6.307,4.642)}
\gppoint{gp mark 7}{(6.319,4.642)}
\gppoint{gp mark 7}{(6.332,4.642)}
\gppoint{gp mark 7}{(6.345,4.642)}
\gppoint{gp mark 7}{(6.357,4.642)}
\gppoint{gp mark 7}{(6.370,4.642)}
\gppoint{gp mark 7}{(6.383,4.642)}
\gppoint{gp mark 7}{(6.395,4.642)}
\gppoint{gp mark 7}{(6.408,4.642)}
\gppoint{gp mark 7}{(6.421,4.642)}
\gppoint{gp mark 7}{(6.433,4.642)}
\gppoint{gp mark 7}{(6.446,4.642)}
\gppoint{gp mark 7}{(6.459,4.642)}
\gppoint{gp mark 7}{(6.471,4.641)}
\gppoint{gp mark 7}{(6.484,4.641)}
\gppoint{gp mark 7}{(6.497,4.641)}
\gppoint{gp mark 7}{(6.510,4.641)}
\gppoint{gp mark 7}{(6.522,4.641)}
\gppoint{gp mark 7}{(6.535,4.641)}
\gppoint{gp mark 7}{(6.548,4.641)}
\gppoint{gp mark 7}{(6.560,4.641)}
\gppoint{gp mark 7}{(6.573,4.641)}
\gppoint{gp mark 7}{(6.586,4.641)}
\gppoint{gp mark 7}{(6.598,4.641)}
\gppoint{gp mark 7}{(6.611,4.641)}
\gppoint{gp mark 7}{(6.624,4.641)}
\gppoint{gp mark 7}{(6.636,4.641)}
\gppoint{gp mark 7}{(6.649,4.641)}
\gppoint{gp mark 7}{(6.662,4.641)}
\gppoint{gp mark 7}{(6.674,4.641)}
\gppoint{gp mark 7}{(6.687,4.641)}
\gppoint{gp mark 7}{(6.700,4.641)}
\gppoint{gp mark 7}{(6.712,4.641)}
\gppoint{gp mark 7}{(6.725,4.641)}
\gppoint{gp mark 7}{(6.738,4.640)}
\gppoint{gp mark 7}{(6.750,4.640)}
\gppoint{gp mark 7}{(6.763,4.640)}
\gppoint{gp mark 7}{(6.776,4.640)}
\gppoint{gp mark 7}{(6.788,4.639)}
\gppoint{gp mark 7}{(6.801,4.639)}
\gppoint{gp mark 7}{(6.814,4.639)}
\gppoint{gp mark 7}{(6.826,4.639)}
\gppoint{gp mark 7}{(6.839,4.639)}
\gppoint{gp mark 7}{(6.852,4.639)}
\gppoint{gp mark 7}{(6.865,4.639)}
\gppoint{gp mark 7}{(6.877,4.639)}
\gppoint{gp mark 7}{(6.890,4.639)}
\gppoint{gp mark 7}{(6.903,4.639)}
\gppoint{gp mark 7}{(6.915,4.638)}
\gppoint{gp mark 7}{(6.928,4.638)}
\gppoint{gp mark 7}{(6.941,4.637)}
\gppoint{gp mark 7}{(6.953,4.637)}
\gppoint{gp mark 7}{(6.966,4.637)}
\gppoint{gp mark 7}{(6.979,4.636)}
\gppoint{gp mark 7}{(6.991,4.636)}
\gppoint{gp mark 7}{(7.004,4.637)}
\gppoint{gp mark 7}{(7.017,4.637)}
\gppoint{gp mark 7}{(7.029,4.637)}
\gppoint{gp mark 7}{(7.042,4.638)}
\gppoint{gp mark 7}{(7.055,4.638)}
\gppoint{gp mark 7}{(7.067,4.638)}
\gppoint{gp mark 7}{(7.080,4.638)}
\gppoint{gp mark 7}{(7.093,4.638)}
\gppoint{gp mark 7}{(7.105,4.637)}
\gppoint{gp mark 7}{(7.118,4.637)}
\gppoint{gp mark 7}{(7.131,4.637)}
\gppoint{gp mark 7}{(7.143,4.638)}
\gppoint{gp mark 7}{(7.156,4.638)}
\gppoint{gp mark 7}{(7.169,4.638)}
\gppoint{gp mark 7}{(7.181,4.638)}
\gppoint{gp mark 7}{(7.194,4.639)}
\gppoint{gp mark 7}{(7.207,4.639)}
\gppoint{gp mark 7}{(7.220,4.639)}
\gppoint{gp mark 7}{(7.232,4.640)}
\gppoint{gp mark 7}{(7.245,4.640)}
\gppoint{gp mark 7}{(7.258,4.641)}
\gppoint{gp mark 7}{(7.270,4.643)}
\gppoint{gp mark 7}{(7.283,4.644)}
\gppoint{gp mark 7}{(7.296,4.645)}
\gppoint{gp mark 7}{(7.308,4.646)}
\gppoint{gp mark 7}{(7.321,4.647)}
\gppoint{gp mark 7}{(7.334,4.648)}
\gppoint{gp mark 7}{(7.346,4.649)}
\gppoint{gp mark 7}{(7.359,4.650)}
\gppoint{gp mark 7}{(7.372,4.650)}
\gppoint{gp mark 7}{(7.384,4.650)}
\gppoint{gp mark 7}{(7.397,4.650)}
\gppoint{gp mark 7}{(7.410,4.650)}
\gppoint{gp mark 7}{(7.422,4.650)}
\gppoint{gp mark 7}{(7.435,4.649)}
\gppoint{gp mark 7}{(7.448,4.646)}
\gppoint{gp mark 7}{(7.460,4.642)}
\gppoint{gp mark 7}{(7.473,4.639)}
\gppoint{gp mark 7}{(7.486,4.634)}
\gppoint{gp mark 7}{(7.498,4.629)}
\gppoint{gp mark 7}{(7.511,4.621)}
\gppoint{gp mark 7}{(7.524,4.612)}
\gppoint{gp mark 7}{(7.536,4.606)}
\gppoint{gp mark 7}{(7.549,4.601)}
\gppoint{gp mark 7}{(7.562,4.596)}
\gppoint{gp mark 7}{(7.575,4.589)}
\gppoint{gp mark 7}{(7.587,4.580)}
\gppoint{gp mark 7}{(7.600,4.570)}
\gppoint{gp mark 7}{(7.613,4.562)}
\gppoint{gp mark 7}{(7.625,4.559)}
\gppoint{gp mark 7}{(7.638,4.558)}
\gppoint{gp mark 7}{(7.651,4.558)}
\gppoint{gp mark 7}{(7.663,4.555)}
\gppoint{gp mark 7}{(7.676,4.535)}
\gppoint{gp mark 7}{(7.689,4.392)}
\gppoint{gp mark 7}{(7.701,3.879)}
\gppoint{gp mark 7}{(7.714,3.039)}
\gppoint{gp mark 7}{(7.727,2.189)}
\gppoint{gp mark 7}{(7.739,1.597)}
\gppoint{gp mark 7}{(7.752,1.297)}
\gppoint{gp mark 7}{(7.765,1.185)}
\gppoint{gp mark 7}{(7.777,1.154)}
\gppoint{gp mark 7}{(7.790,1.149)}
\gppoint{gp mark 7}{(7.803,1.148)}
\gppoint{gp mark 7}{(7.815,1.148)}
\gppoint{gp mark 7}{(7.828,1.149)}
\gppoint{gp mark 7}{(7.841,1.153)}
\gppoint{gp mark 7}{(7.853,1.158)}
\gppoint{gp mark 7}{(7.866,1.161)}
\gppoint{gp mark 7}{(7.879,1.162)}
\gppoint{gp mark 7}{(7.891,1.162)}
\gppoint{gp mark 7}{(7.904,1.162)}
\gppoint{gp mark 7}{(7.917,1.162)}
\gppoint{gp mark 7}{(7.930,1.162)}
\gpcolor{rgb color={1.000,0.000,0.000}}
\gpsetpointsize{4.44}
\gppoint{gp mark 7}{(1.210,3.636)}
\gppoint{gp mark 7}{(1.223,3.621)}
\gppoint{gp mark 7}{(1.235,3.605)}
\gppoint{gp mark 7}{(1.248,3.590)}
\gppoint{gp mark 7}{(1.261,3.574)}
\gppoint{gp mark 7}{(1.273,3.559)}
\gppoint{gp mark 7}{(1.286,3.544)}
\gppoint{gp mark 7}{(1.299,3.528)}
\gppoint{gp mark 7}{(1.311,3.513)}
\gppoint{gp mark 7}{(1.324,3.498)}
\gppoint{gp mark 7}{(1.337,3.482)}
\gppoint{gp mark 7}{(1.349,3.467)}
\gppoint{gp mark 7}{(1.362,3.452)}
\gppoint{gp mark 7}{(1.375,3.436)}
\gppoint{gp mark 7}{(1.387,3.421)}
\gppoint{gp mark 7}{(1.400,3.406)}
\gppoint{gp mark 7}{(1.413,3.391)}
\gppoint{gp mark 7}{(1.425,3.375)}
\gppoint{gp mark 7}{(1.438,3.360)}
\gppoint{gp mark 7}{(1.451,3.345)}
\gppoint{gp mark 7}{(1.464,3.330)}
\gppoint{gp mark 7}{(1.476,3.314)}
\gppoint{gp mark 7}{(1.489,3.299)}
\gppoint{gp mark 7}{(1.502,3.284)}
\gppoint{gp mark 7}{(1.514,3.269)}
\gppoint{gp mark 7}{(1.527,3.254)}
\gppoint{gp mark 7}{(1.540,3.239)}
\gppoint{gp mark 7}{(1.552,3.224)}
\gppoint{gp mark 7}{(1.565,3.209)}
\gppoint{gp mark 7}{(1.578,3.193)}
\gppoint{gp mark 7}{(1.590,3.178)}
\gppoint{gp mark 7}{(1.603,3.163)}
\gppoint{gp mark 7}{(1.616,3.148)}
\gppoint{gp mark 7}{(1.628,3.133)}
\gppoint{gp mark 7}{(1.641,3.118)}
\gppoint{gp mark 7}{(1.654,3.103)}
\gppoint{gp mark 7}{(1.666,3.088)}
\gppoint{gp mark 7}{(1.679,3.073)}
\gppoint{gp mark 7}{(1.692,3.058)}
\gppoint{gp mark 7}{(1.704,3.043)}
\gppoint{gp mark 7}{(1.717,3.029)}
\gppoint{gp mark 7}{(1.730,3.014)}
\gppoint{gp mark 7}{(1.742,2.999)}
\gppoint{gp mark 7}{(1.755,2.984)}
\gppoint{gp mark 7}{(1.768,2.969)}
\gppoint{gp mark 7}{(1.780,2.954)}
\gppoint{gp mark 7}{(1.793,2.939)}
\gppoint{gp mark 7}{(1.806,2.924)}
\gppoint{gp mark 7}{(1.819,2.910)}
\gppoint{gp mark 7}{(1.831,2.895)}
\gppoint{gp mark 7}{(1.844,2.880)}
\gppoint{gp mark 7}{(1.857,2.865)}
\gppoint{gp mark 7}{(1.869,2.851)}
\gppoint{gp mark 7}{(1.882,2.836)}
\gppoint{gp mark 7}{(1.895,2.821)}
\gppoint{gp mark 7}{(1.907,2.806)}
\gppoint{gp mark 7}{(1.920,2.792)}
\gppoint{gp mark 7}{(1.933,2.777)}
\gppoint{gp mark 7}{(1.945,2.762)}
\gppoint{gp mark 7}{(1.958,2.748)}
\gppoint{gp mark 7}{(1.971,2.733)}
\gppoint{gp mark 7}{(1.983,2.719)}
\gppoint{gp mark 7}{(1.996,2.704)}
\gppoint{gp mark 7}{(2.009,2.689)}
\gppoint{gp mark 7}{(2.021,2.675)}
\gppoint{gp mark 7}{(2.034,2.660)}
\gppoint{gp mark 7}{(2.047,2.646)}
\gppoint{gp mark 7}{(2.059,2.631)}
\gppoint{gp mark 7}{(2.072,2.617)}
\gppoint{gp mark 7}{(2.085,2.602)}
\gppoint{gp mark 7}{(2.097,2.588)}
\gppoint{gp mark 7}{(2.110,2.573)}
\gppoint{gp mark 7}{(2.123,2.559)}
\gppoint{gp mark 7}{(2.135,2.545)}
\gppoint{gp mark 7}{(2.148,2.530)}
\gppoint{gp mark 7}{(2.161,2.516)}
\gppoint{gp mark 7}{(2.174,2.502)}
\gppoint{gp mark 7}{(2.186,2.487)}
\gppoint{gp mark 7}{(2.199,2.473)}
\gppoint{gp mark 7}{(2.212,2.459)}
\gppoint{gp mark 7}{(2.224,2.445)}
\gppoint{gp mark 7}{(2.237,2.430)}
\gppoint{gp mark 7}{(2.250,2.416)}
\gppoint{gp mark 7}{(2.262,2.402)}
\gppoint{gp mark 7}{(2.275,2.388)}
\gppoint{gp mark 7}{(2.288,2.374)}
\gppoint{gp mark 7}{(2.300,2.360)}
\gppoint{gp mark 7}{(2.313,2.346)}
\gppoint{gp mark 7}{(2.326,2.332)}
\gppoint{gp mark 7}{(2.338,2.319)}
\gppoint{gp mark 7}{(2.351,2.305)}
\gppoint{gp mark 7}{(2.364,2.292)}
\gppoint{gp mark 7}{(2.376,2.278)}
\gppoint{gp mark 7}{(2.389,2.266)}
\gppoint{gp mark 7}{(2.402,2.254)}
\gppoint{gp mark 7}{(2.414,2.243)}
\gppoint{gp mark 7}{(2.427,2.234)}
\gppoint{gp mark 7}{(2.440,2.228)}
\gppoint{gp mark 7}{(2.452,2.224)}
\gppoint{gp mark 7}{(2.465,2.223)}
\gppoint{gp mark 7}{(2.478,2.223)}
\gppoint{gp mark 7}{(2.490,2.223)}
\gppoint{gp mark 7}{(2.503,2.223)}
\gppoint{gp mark 7}{(2.516,2.223)}
\gppoint{gp mark 7}{(2.529,2.224)}
\gppoint{gp mark 7}{(2.541,2.227)}
\gppoint{gp mark 7}{(2.554,2.229)}
\gppoint{gp mark 7}{(2.567,2.231)}
\gppoint{gp mark 7}{(2.579,2.232)}
\gppoint{gp mark 7}{(2.592,2.232)}
\gppoint{gp mark 7}{(2.605,2.232)}
\gppoint{gp mark 7}{(2.617,2.232)}
\gppoint{gp mark 7}{(2.630,2.233)}
\gppoint{gp mark 7}{(2.643,2.233)}
\gppoint{gp mark 7}{(2.655,2.233)}
\gppoint{gp mark 7}{(2.668,2.234)}
\gppoint{gp mark 7}{(2.681,2.234)}
\gppoint{gp mark 7}{(2.693,2.234)}
\gppoint{gp mark 7}{(2.706,2.234)}
\gppoint{gp mark 7}{(2.719,2.234)}
\gppoint{gp mark 7}{(2.731,2.234)}
\gppoint{gp mark 7}{(2.744,2.234)}
\gppoint{gp mark 7}{(2.757,2.233)}
\gppoint{gp mark 7}{(2.769,2.233)}
\gppoint{gp mark 7}{(2.782,2.233)}
\gppoint{gp mark 7}{(2.795,2.233)}
\gppoint{gp mark 7}{(2.807,2.233)}
\gppoint{gp mark 7}{(2.820,2.233)}
\gppoint{gp mark 7}{(2.833,2.233)}
\gppoint{gp mark 7}{(2.845,2.235)}
\gppoint{gp mark 7}{(2.858,2.256)}
\gppoint{gp mark 7}{(2.871,2.373)}
\gppoint{gp mark 7}{(2.883,2.283)}
\gppoint{gp mark 7}{(2.896,2.167)}
\gppoint{gp mark 7}{(2.909,2.232)}
\gppoint{gp mark 7}{(2.922,2.213)}
\gppoint{gp mark 7}{(2.934,2.219)}
\gppoint{gp mark 7}{(2.947,2.227)}
\gppoint{gp mark 7}{(2.960,2.227)}
\gppoint{gp mark 7}{(2.972,2.227)}
\gppoint{gp mark 7}{(2.985,2.228)}
\gppoint{gp mark 7}{(2.998,2.229)}
\gppoint{gp mark 7}{(3.010,2.227)}
\gppoint{gp mark 7}{(3.023,2.227)}
\gppoint{gp mark 7}{(3.036,2.227)}
\gppoint{gp mark 7}{(3.048,2.227)}
\gppoint{gp mark 7}{(3.061,2.226)}
\gppoint{gp mark 7}{(3.074,2.225)}
\gppoint{gp mark 7}{(3.086,2.226)}
\gppoint{gp mark 7}{(3.099,2.227)}
\gppoint{gp mark 7}{(3.112,2.228)}
\gppoint{gp mark 7}{(3.124,2.227)}
\gppoint{gp mark 7}{(3.137,2.227)}
\gppoint{gp mark 7}{(3.150,2.227)}
\gppoint{gp mark 7}{(3.162,2.227)}
\gppoint{gp mark 7}{(3.175,2.228)}
\gppoint{gp mark 7}{(3.188,2.228)}
\gppoint{gp mark 7}{(3.200,2.227)}
\gppoint{gp mark 7}{(3.213,2.225)}
\gppoint{gp mark 7}{(3.226,2.225)}
\gppoint{gp mark 7}{(3.238,2.225)}
\gppoint{gp mark 7}{(3.251,2.225)}
\gppoint{gp mark 7}{(3.264,2.226)}
\gppoint{gp mark 7}{(3.277,2.226)}
\gppoint{gp mark 7}{(3.289,2.226)}
\gppoint{gp mark 7}{(3.302,2.227)}
\gppoint{gp mark 7}{(3.315,2.227)}
\gppoint{gp mark 7}{(3.327,2.227)}
\gppoint{gp mark 7}{(3.340,2.227)}
\gppoint{gp mark 7}{(3.353,2.227)}
\gppoint{gp mark 7}{(3.365,2.226)}
\gppoint{gp mark 7}{(3.378,2.225)}
\gppoint{gp mark 7}{(3.391,2.225)}
\gppoint{gp mark 7}{(3.403,2.224)}
\gppoint{gp mark 7}{(3.416,2.224)}
\gppoint{gp mark 7}{(3.429,2.224)}
\gppoint{gp mark 7}{(3.441,2.226)}
\gppoint{gp mark 7}{(3.454,2.228)}
\gppoint{gp mark 7}{(3.467,2.229)}
\gppoint{gp mark 7}{(3.479,2.229)}
\gppoint{gp mark 7}{(3.492,2.229)}
\gppoint{gp mark 7}{(3.505,2.228)}
\gppoint{gp mark 7}{(3.517,2.227)}
\gppoint{gp mark 7}{(3.530,2.226)}
\gppoint{gp mark 7}{(3.543,2.225)}
\gppoint{gp mark 7}{(3.555,2.225)}
\gppoint{gp mark 7}{(3.568,2.225)}
\gppoint{gp mark 7}{(3.581,2.225)}
\gppoint{gp mark 7}{(3.593,2.226)}
\gppoint{gp mark 7}{(3.606,2.228)}
\gppoint{gp mark 7}{(3.619,2.229)}
\gppoint{gp mark 7}{(3.632,2.229)}
\gppoint{gp mark 7}{(3.644,2.230)}
\gppoint{gp mark 7}{(3.657,2.229)}
\gppoint{gp mark 7}{(3.670,2.229)}
\gppoint{gp mark 7}{(3.682,2.227)}
\gppoint{gp mark 7}{(3.695,2.226)}
\gppoint{gp mark 7}{(3.708,2.225)}
\gppoint{gp mark 7}{(3.720,2.225)}
\gppoint{gp mark 7}{(3.733,2.226)}
\gppoint{gp mark 7}{(3.746,2.227)}
\gppoint{gp mark 7}{(3.758,2.230)}
\gppoint{gp mark 7}{(3.771,2.242)}
\gppoint{gp mark 7}{(3.784,2.362)}
\gppoint{gp mark 7}{(3.796,3.145)}
\gppoint{gp mark 7}{(3.809,4.421)}
\gppoint{gp mark 7}{(3.822,4.660)}
\gppoint{gp mark 7}{(3.834,4.681)}
\gppoint{gp mark 7}{(3.847,4.679)}
\gppoint{gp mark 7}{(3.860,4.677)}
\gppoint{gp mark 7}{(3.872,4.678)}
\gppoint{gp mark 7}{(3.885,4.679)}
\gppoint{gp mark 7}{(3.898,4.679)}
\gppoint{gp mark 7}{(3.910,4.678)}
\gppoint{gp mark 7}{(3.923,4.677)}
\gppoint{gp mark 7}{(3.936,4.677)}
\gppoint{gp mark 7}{(3.948,4.677)}
\gppoint{gp mark 7}{(3.961,4.678)}
\gppoint{gp mark 7}{(3.974,4.678)}
\gppoint{gp mark 7}{(3.987,4.678)}
\gppoint{gp mark 7}{(3.999,4.678)}
\gppoint{gp mark 7}{(4.012,4.678)}
\gppoint{gp mark 7}{(4.025,4.678)}
\gppoint{gp mark 7}{(4.037,4.678)}
\gppoint{gp mark 7}{(4.050,4.678)}
\gppoint{gp mark 7}{(4.063,4.678)}
\gppoint{gp mark 7}{(4.075,4.678)}
\gppoint{gp mark 7}{(4.088,4.678)}
\gppoint{gp mark 7}{(4.101,4.678)}
\gppoint{gp mark 7}{(4.113,4.678)}
\gppoint{gp mark 7}{(4.126,4.677)}
\gppoint{gp mark 7}{(4.139,4.678)}
\gppoint{gp mark 7}{(4.151,4.678)}
\gppoint{gp mark 7}{(4.164,4.679)}
\gppoint{gp mark 7}{(4.177,4.679)}
\gppoint{gp mark 7}{(4.189,4.678)}
\gppoint{gp mark 7}{(4.202,4.678)}
\gppoint{gp mark 7}{(4.215,4.678)}
\gppoint{gp mark 7}{(4.227,4.678)}
\gppoint{gp mark 7}{(4.240,4.678)}
\gppoint{gp mark 7}{(4.253,4.679)}
\gppoint{gp mark 7}{(4.265,4.679)}
\gppoint{gp mark 7}{(4.278,4.679)}
\gppoint{gp mark 7}{(4.291,4.679)}
\gppoint{gp mark 7}{(4.303,4.679)}
\gppoint{gp mark 7}{(4.316,4.679)}
\gppoint{gp mark 7}{(4.329,4.678)}
\gppoint{gp mark 7}{(4.342,4.678)}
\gppoint{gp mark 7}{(4.354,4.678)}
\gppoint{gp mark 7}{(4.367,4.679)}
\gppoint{gp mark 7}{(4.380,4.679)}
\gppoint{gp mark 7}{(4.392,4.679)}
\gppoint{gp mark 7}{(4.405,4.679)}
\gppoint{gp mark 7}{(4.418,4.678)}
\gppoint{gp mark 7}{(4.430,4.678)}
\gppoint{gp mark 7}{(4.443,4.678)}
\gppoint{gp mark 7}{(4.456,4.678)}
\gppoint{gp mark 7}{(4.468,4.678)}
\gppoint{gp mark 7}{(4.481,4.678)}
\gppoint{gp mark 7}{(4.494,4.678)}
\gppoint{gp mark 7}{(4.506,4.678)}
\gppoint{gp mark 7}{(4.519,4.677)}
\gppoint{gp mark 7}{(4.532,4.677)}
\gppoint{gp mark 7}{(4.544,4.678)}
\gppoint{gp mark 7}{(4.557,4.678)}
\gppoint{gp mark 7}{(4.570,4.679)}
\gppoint{gp mark 7}{(4.582,4.679)}
\gppoint{gp mark 7}{(4.595,4.679)}
\gppoint{gp mark 7}{(4.608,4.678)}
\gppoint{gp mark 7}{(4.620,4.678)}
\gppoint{gp mark 7}{(4.633,4.678)}
\gppoint{gp mark 7}{(4.646,4.678)}
\gppoint{gp mark 7}{(4.658,4.678)}
\gppoint{gp mark 7}{(4.671,4.678)}
\gppoint{gp mark 7}{(4.684,4.678)}
\gppoint{gp mark 7}{(4.697,4.678)}
\gppoint{gp mark 7}{(4.709,4.678)}
\gppoint{gp mark 7}{(4.722,4.678)}
\gppoint{gp mark 7}{(4.735,4.679)}
\gppoint{gp mark 7}{(4.747,4.679)}
\gppoint{gp mark 7}{(4.760,4.678)}
\gppoint{gp mark 7}{(4.773,4.678)}
\gppoint{gp mark 7}{(4.785,4.678)}
\gppoint{gp mark 7}{(4.798,4.678)}
\gppoint{gp mark 7}{(4.811,4.678)}
\gppoint{gp mark 7}{(4.823,4.678)}
\gppoint{gp mark 7}{(4.836,4.678)}
\gppoint{gp mark 7}{(4.849,4.678)}
\gppoint{gp mark 7}{(4.861,4.678)}
\gppoint{gp mark 7}{(4.874,4.678)}
\gppoint{gp mark 7}{(4.887,4.678)}
\gppoint{gp mark 7}{(4.899,4.678)}
\gppoint{gp mark 7}{(4.912,4.678)}
\gppoint{gp mark 7}{(4.925,4.678)}
\gppoint{gp mark 7}{(4.937,4.678)}
\gppoint{gp mark 7}{(4.950,4.678)}
\gppoint{gp mark 7}{(4.963,4.678)}
\gppoint{gp mark 7}{(4.975,4.678)}
\gppoint{gp mark 7}{(4.988,4.678)}
\gppoint{gp mark 7}{(5.001,4.678)}
\gppoint{gp mark 7}{(5.013,4.678)}
\gppoint{gp mark 7}{(5.026,4.678)}
\gppoint{gp mark 7}{(5.039,4.678)}
\gppoint{gp mark 7}{(5.052,4.678)}
\gppoint{gp mark 7}{(5.064,4.678)}
\gppoint{gp mark 7}{(5.077,4.678)}
\gppoint{gp mark 7}{(5.090,4.678)}
\gppoint{gp mark 7}{(5.102,4.678)}
\gppoint{gp mark 7}{(5.115,4.678)}
\gppoint{gp mark 7}{(5.128,4.678)}
\gppoint{gp mark 7}{(5.140,4.678)}
\gppoint{gp mark 7}{(5.153,4.678)}
\gppoint{gp mark 7}{(5.166,4.678)}
\gppoint{gp mark 7}{(5.178,4.678)}
\gppoint{gp mark 7}{(5.191,4.678)}
\gppoint{gp mark 7}{(5.204,4.678)}
\gppoint{gp mark 7}{(5.216,4.678)}
\gppoint{gp mark 7}{(5.229,4.678)}
\gppoint{gp mark 7}{(5.242,4.678)}
\gppoint{gp mark 7}{(5.254,4.678)}
\gppoint{gp mark 7}{(5.267,4.678)}
\gppoint{gp mark 7}{(5.280,4.678)}
\gppoint{gp mark 7}{(5.292,4.678)}
\gppoint{gp mark 7}{(5.305,4.678)}
\gppoint{gp mark 7}{(5.318,4.678)}
\gppoint{gp mark 7}{(5.330,4.678)}
\gppoint{gp mark 7}{(5.343,4.678)}
\gppoint{gp mark 7}{(5.356,4.678)}
\gppoint{gp mark 7}{(5.368,4.678)}
\gppoint{gp mark 7}{(5.381,4.678)}
\gppoint{gp mark 7}{(5.394,4.678)}
\gppoint{gp mark 7}{(5.407,4.678)}
\gppoint{gp mark 7}{(5.419,4.678)}
\gppoint{gp mark 7}{(5.432,4.678)}
\gppoint{gp mark 7}{(5.445,4.678)}
\gppoint{gp mark 7}{(5.457,4.678)}
\gppoint{gp mark 7}{(5.470,4.678)}
\gppoint{gp mark 7}{(5.483,4.678)}
\gppoint{gp mark 7}{(5.495,4.678)}
\gppoint{gp mark 7}{(5.508,4.678)}
\gppoint{gp mark 7}{(5.521,4.678)}
\gppoint{gp mark 7}{(5.533,4.678)}
\gppoint{gp mark 7}{(5.546,4.678)}
\gppoint{gp mark 7}{(5.559,4.679)}
\gppoint{gp mark 7}{(5.571,4.679)}
\gppoint{gp mark 7}{(5.584,4.679)}
\gppoint{gp mark 7}{(5.597,4.678)}
\gppoint{gp mark 7}{(5.609,4.678)}
\gppoint{gp mark 7}{(5.622,4.678)}
\gppoint{gp mark 7}{(5.635,4.678)}
\gppoint{gp mark 7}{(5.647,4.678)}
\gppoint{gp mark 7}{(5.660,4.678)}
\gppoint{gp mark 7}{(5.673,4.678)}
\gppoint{gp mark 7}{(5.685,4.678)}
\gppoint{gp mark 7}{(5.698,4.678)}
\gppoint{gp mark 7}{(5.711,4.678)}
\gppoint{gp mark 7}{(5.723,4.678)}
\gppoint{gp mark 7}{(5.736,4.678)}
\gppoint{gp mark 7}{(5.749,4.678)}
\gppoint{gp mark 7}{(5.761,4.678)}
\gppoint{gp mark 7}{(5.774,4.678)}
\gppoint{gp mark 7}{(5.787,4.679)}
\gppoint{gp mark 7}{(5.800,4.679)}
\gppoint{gp mark 7}{(5.812,4.679)}
\gppoint{gp mark 7}{(5.825,4.679)}
\gppoint{gp mark 7}{(5.838,4.678)}
\gppoint{gp mark 7}{(5.850,4.678)}
\gppoint{gp mark 7}{(5.863,4.678)}
\gppoint{gp mark 7}{(5.876,4.678)}
\gppoint{gp mark 7}{(5.888,4.678)}
\gppoint{gp mark 7}{(5.901,4.678)}
\gppoint{gp mark 7}{(5.914,4.678)}
\gppoint{gp mark 7}{(5.926,4.679)}
\gppoint{gp mark 7}{(5.939,4.678)}
\gppoint{gp mark 7}{(5.952,4.678)}
\gppoint{gp mark 7}{(5.964,4.678)}
\gppoint{gp mark 7}{(5.977,4.678)}
\gppoint{gp mark 7}{(5.990,4.678)}
\gppoint{gp mark 7}{(6.002,4.678)}
\gppoint{gp mark 7}{(6.015,4.678)}
\gppoint{gp mark 7}{(6.028,4.678)}
\gppoint{gp mark 7}{(6.040,4.678)}
\gppoint{gp mark 7}{(6.053,4.678)}
\gppoint{gp mark 7}{(6.066,4.678)}
\gppoint{gp mark 7}{(6.078,4.678)}
\gppoint{gp mark 7}{(6.091,4.678)}
\gppoint{gp mark 7}{(6.104,4.678)}
\gppoint{gp mark 7}{(6.116,4.678)}
\gppoint{gp mark 7}{(6.129,4.678)}
\gppoint{gp mark 7}{(6.142,4.678)}
\gppoint{gp mark 7}{(6.155,4.678)}
\gppoint{gp mark 7}{(6.167,4.678)}
\gppoint{gp mark 7}{(6.180,4.678)}
\gppoint{gp mark 7}{(6.193,4.678)}
\gppoint{gp mark 7}{(6.205,4.678)}
\gppoint{gp mark 7}{(6.218,4.678)}
\gppoint{gp mark 7}{(6.231,4.678)}
\gppoint{gp mark 7}{(6.243,4.678)}
\gppoint{gp mark 7}{(6.256,4.678)}
\gppoint{gp mark 7}{(6.269,4.678)}
\gppoint{gp mark 7}{(6.281,4.678)}
\gppoint{gp mark 7}{(6.294,4.678)}
\gppoint{gp mark 7}{(6.307,4.678)}
\gppoint{gp mark 7}{(6.319,4.678)}
\gppoint{gp mark 7}{(6.332,4.678)}
\gppoint{gp mark 7}{(6.345,4.678)}
\gppoint{gp mark 7}{(6.357,4.677)}
\gppoint{gp mark 7}{(6.370,4.677)}
\gppoint{gp mark 7}{(6.383,4.677)}
\gppoint{gp mark 7}{(6.395,4.677)}
\gppoint{gp mark 7}{(6.408,4.677)}
\gppoint{gp mark 7}{(6.421,4.677)}
\gppoint{gp mark 7}{(6.433,4.677)}
\gppoint{gp mark 7}{(6.446,4.677)}
\gppoint{gp mark 7}{(6.459,4.677)}
\gppoint{gp mark 7}{(6.471,4.677)}
\gppoint{gp mark 7}{(6.484,4.677)}
\gppoint{gp mark 7}{(6.497,4.676)}
\gppoint{gp mark 7}{(6.510,4.676)}
\gppoint{gp mark 7}{(6.522,4.676)}
\gppoint{gp mark 7}{(6.535,4.676)}
\gppoint{gp mark 7}{(6.548,4.676)}
\gppoint{gp mark 7}{(6.560,4.676)}
\gppoint{gp mark 7}{(6.573,4.675)}
\gppoint{gp mark 7}{(6.586,4.675)}
\gppoint{gp mark 7}{(6.598,4.675)}
\gppoint{gp mark 7}{(6.611,4.675)}
\gppoint{gp mark 7}{(6.624,4.675)}
\gppoint{gp mark 7}{(6.636,4.675)}
\gppoint{gp mark 7}{(6.649,4.675)}
\gppoint{gp mark 7}{(6.662,4.675)}
\gppoint{gp mark 7}{(6.674,4.675)}
\gppoint{gp mark 7}{(6.687,4.675)}
\gppoint{gp mark 7}{(6.700,4.675)}
\gppoint{gp mark 7}{(6.712,4.674)}
\gppoint{gp mark 7}{(6.725,4.674)}
\gppoint{gp mark 7}{(6.738,4.674)}
\gppoint{gp mark 7}{(6.750,4.673)}
\gppoint{gp mark 7}{(6.763,4.673)}
\gppoint{gp mark 7}{(6.776,4.673)}
\gppoint{gp mark 7}{(6.788,4.673)}
\gppoint{gp mark 7}{(6.801,4.673)}
\gppoint{gp mark 7}{(6.814,4.673)}
\gppoint{gp mark 7}{(6.826,4.673)}
\gppoint{gp mark 7}{(6.839,4.673)}
\gppoint{gp mark 7}{(6.852,4.673)}
\gppoint{gp mark 7}{(6.865,4.673)}
\gppoint{gp mark 7}{(6.877,4.673)}
\gppoint{gp mark 7}{(6.890,4.673)}
\gppoint{gp mark 7}{(6.903,4.673)}
\gppoint{gp mark 7}{(6.915,4.673)}
\gppoint{gp mark 7}{(6.928,4.673)}
\gppoint{gp mark 7}{(6.941,4.673)}
\gppoint{gp mark 7}{(6.953,4.673)}
\gppoint{gp mark 7}{(6.966,4.673)}
\gppoint{gp mark 7}{(6.979,4.673)}
\gppoint{gp mark 7}{(6.991,4.673)}
\gppoint{gp mark 7}{(7.004,4.673)}
\gppoint{gp mark 7}{(7.017,4.674)}
\gppoint{gp mark 7}{(7.029,4.674)}
\gppoint{gp mark 7}{(7.042,4.674)}
\gppoint{gp mark 7}{(7.055,4.674)}
\gppoint{gp mark 7}{(7.067,4.674)}
\gppoint{gp mark 7}{(7.080,4.674)}
\gppoint{gp mark 7}{(7.093,4.675)}
\gppoint{gp mark 7}{(7.105,4.675)}
\gppoint{gp mark 7}{(7.118,4.675)}
\gppoint{gp mark 7}{(7.131,4.676)}
\gppoint{gp mark 7}{(7.143,4.676)}
\gppoint{gp mark 7}{(7.156,4.677)}
\gppoint{gp mark 7}{(7.169,4.678)}
\gppoint{gp mark 7}{(7.181,4.679)}
\gppoint{gp mark 7}{(7.194,4.680)}
\gppoint{gp mark 7}{(7.207,4.680)}
\gppoint{gp mark 7}{(7.220,4.680)}
\gppoint{gp mark 7}{(7.232,4.682)}
\gppoint{gp mark 7}{(7.245,4.685)}
\gppoint{gp mark 7}{(7.258,4.688)}
\gppoint{gp mark 7}{(7.270,4.690)}
\gppoint{gp mark 7}{(7.283,4.691)}
\gppoint{gp mark 7}{(7.296,4.692)}
\gppoint{gp mark 7}{(7.308,4.693)}
\gppoint{gp mark 7}{(7.321,4.695)}
\gppoint{gp mark 7}{(7.334,4.698)}
\gppoint{gp mark 7}{(7.346,4.700)}
\gppoint{gp mark 7}{(7.359,4.702)}
\gppoint{gp mark 7}{(7.372,4.703)}
\gppoint{gp mark 7}{(7.384,4.703)}
\gppoint{gp mark 7}{(7.397,4.703)}
\gppoint{gp mark 7}{(7.410,4.703)}
\gppoint{gp mark 7}{(7.422,4.703)}
\gppoint{gp mark 7}{(7.435,4.701)}
\gppoint{gp mark 7}{(7.448,4.698)}
\gppoint{gp mark 7}{(7.460,4.693)}
\gppoint{gp mark 7}{(7.473,4.689)}
\gppoint{gp mark 7}{(7.486,4.685)}
\gppoint{gp mark 7}{(7.498,4.680)}
\gppoint{gp mark 7}{(7.511,4.676)}
\gppoint{gp mark 7}{(7.524,4.672)}
\gppoint{gp mark 7}{(7.536,4.666)}
\gppoint{gp mark 7}{(7.549,4.657)}
\gppoint{gp mark 7}{(7.562,4.640)}
\gppoint{gp mark 7}{(7.575,4.617)}
\gppoint{gp mark 7}{(7.587,4.596)}
\gppoint{gp mark 7}{(7.600,4.582)}
\gppoint{gp mark 7}{(7.613,4.577)}
\gppoint{gp mark 7}{(7.625,4.575)}
\gppoint{gp mark 7}{(7.638,4.575)}
\gppoint{gp mark 7}{(7.651,4.573)}
\gppoint{gp mark 7}{(7.663,4.563)}
\gppoint{gp mark 7}{(7.676,4.490)}
\gppoint{gp mark 7}{(7.689,4.127)}
\gppoint{gp mark 7}{(7.701,3.401)}
\gppoint{gp mark 7}{(7.714,2.560)}
\gppoint{gp mark 7}{(7.727,1.885)}
\gppoint{gp mark 7}{(7.739,1.473)}
\gppoint{gp mark 7}{(7.752,1.269)}
\gppoint{gp mark 7}{(7.765,1.184)}
\gppoint{gp mark 7}{(7.777,1.154)}
\gppoint{gp mark 7}{(7.790,1.145)}
\gppoint{gp mark 7}{(7.803,1.144)}
\gppoint{gp mark 7}{(7.815,1.144)}
\gppoint{gp mark 7}{(7.828,1.144)}
\gppoint{gp mark 7}{(7.841,1.146)}
\gppoint{gp mark 7}{(7.853,1.149)}
\gppoint{gp mark 7}{(7.866,1.152)}
\gppoint{gp mark 7}{(7.879,1.153)}
\gppoint{gp mark 7}{(7.891,1.154)}
\gppoint{gp mark 7}{(7.904,1.154)}
\gppoint{gp mark 7}{(7.917,1.154)}
\gppoint{gp mark 7}{(7.930,1.154)}
\gpcolor{rgb color={0.000,0.000,0.000}}
\gpsetlinetype{gp lt plot 0}
\gpsetlinewidth{4.00}
\draw[gp path] (2.404,2.243)--(2.886,2.243);
\draw[gp path] (2.886,2.243)--(3.811,2.243);
\draw[gp path] (3.811,4.679)--(7.722,4.679);
\draw[gp path] (7.722,1.152)--(7.947,1.152);
\draw[gp path] (1.207,3.449)--(1.220,3.434)--(1.233,3.418)--(1.246,3.403)--(1.259,3.388)%
  --(1.271,3.373)--(1.284,3.358)--(1.297,3.343)--(1.310,3.328)--(1.323,3.313)--(1.336,3.298)%
  --(1.349,3.283)--(1.361,3.268)--(1.374,3.253)--(1.387,3.238)--(1.400,3.224)--(1.413,3.209)%
  --(1.426,3.194)--(1.439,3.180)--(1.452,3.165)--(1.464,3.151)--(1.477,3.136)--(1.490,3.122)%
  --(1.503,3.108)--(1.516,3.093)--(1.529,3.079)--(1.542,3.065)--(1.555,3.051)--(1.567,3.036)%
  --(1.580,3.022)--(1.593,3.008)--(1.606,2.994)--(1.619,2.980)--(1.632,2.967)--(1.645,2.953)%
  --(1.657,2.939)--(1.670,2.925)--(1.683,2.911)--(1.696,2.898)--(1.709,2.884)--(1.722,2.871)%
  --(1.735,2.857)--(1.748,2.844)--(1.760,2.830)--(1.773,2.817)--(1.786,2.803)--(1.799,2.790)%
  --(1.812,2.777)--(1.825,2.764)--(1.838,2.751)--(1.850,2.738)--(1.863,2.725)--(1.876,2.712)%
  --(1.889,2.699)--(1.902,2.686)--(1.915,2.673)--(1.928,2.660)--(1.941,2.647)--(1.953,2.635)%
  --(1.966,2.622)--(1.979,2.610)--(1.992,2.597)--(2.005,2.585)--(2.018,2.572)--(2.031,2.560)%
  --(2.044,2.548)--(2.056,2.535)--(2.069,2.523)--(2.082,2.511)--(2.095,2.499)--(2.108,2.487)%
  --(2.121,2.475)--(2.134,2.463)--(2.146,2.451)--(2.159,2.439)--(2.172,2.428)--(2.185,2.416)%
  --(2.198,2.404)--(2.211,2.393)--(2.224,2.381)--(2.237,2.370)--(2.249,2.358)--(2.262,2.347)%
  --(2.275,2.336)--(2.288,2.325)--(2.301,2.313)--(2.314,2.302)--(2.327,2.291)--(2.340,2.280)%
  --(2.352,2.269)--(2.365,2.258)--(2.378,2.248)--(2.391,2.237)--(2.404,2.243);
\draw[gp path] (3.811,2.243)--(3.811,4.679);
\draw[gp path] (7.722,4.679)--(7.722,1.152);
\node[gp node left,font={\fontsize{10pt}{12pt}\selectfont}] at (1.456,5.268) {\LARGE $\rho$};
\node[gp node left,font={\fontsize{10pt}{12pt}\selectfont}] at (5.740,5.268) {\large $\alpha = 3.0$};
%% coordinates of the plot area
\gpdefrectangularnode{gp plot 1}{\pgfpoint{1.196cm}{0.985cm}}{\pgfpoint{7.947cm}{5.631cm}}
\end{tikzpicture}
%% gnuplot variables
} & 
\resizebox{0.5\linewidth}{!}{\tikzsetnextfilename{fast_coplanar_b_crsol_6}\begin{tikzpicture}[gnuplot]
%% generated with GNUPLOT 4.6p4 (Lua 5.1; terminal rev. 99, script rev. 100)
%% Sat 02 Aug 2014 10:10:42 AM EDT
\path (0.000,0.000) rectangle (8.500,6.000);
\gpfill{rgb color={1.000,1.000,1.000}} (1.196,0.985)--(7.946,0.985)--(7.946,5.630)--(1.196,5.630)--cycle;
\gpcolor{color=gp lt color border}
\gpsetlinetype{gp lt border}
\gpsetlinewidth{1.00}
\draw[gp path] (1.196,0.985)--(1.196,5.630)--(7.946,5.630)--(7.946,0.985)--cycle;
\gpcolor{color=gp lt color axes}
\gpsetlinetype{gp lt axes}
\gpsetlinewidth{2.00}
\draw[gp path] (1.196,0.985)--(7.947,0.985);
\gpcolor{color=gp lt color border}
\gpsetlinetype{gp lt border}
\draw[gp path] (1.196,0.985)--(1.268,0.985);
\draw[gp path] (7.947,0.985)--(7.875,0.985);
\gpcolor{rgb color={0.000,0.000,0.000}}
\node[gp node right,font={\fontsize{10pt}{12pt}\selectfont}] at (1.012,0.985) {-0.4};
\gpcolor{color=gp lt color axes}
\gpsetlinetype{gp lt axes}
\draw[gp path] (1.196,1.759)--(7.947,1.759);
\gpcolor{color=gp lt color border}
\gpsetlinetype{gp lt border}
\draw[gp path] (1.196,1.759)--(1.268,1.759);
\draw[gp path] (7.947,1.759)--(7.875,1.759);
\gpcolor{rgb color={0.000,0.000,0.000}}
\node[gp node right,font={\fontsize{10pt}{12pt}\selectfont}] at (1.012,1.759) {-0.2};
\gpcolor{color=gp lt color axes}
\gpsetlinetype{gp lt axes}
\draw[gp path] (1.196,2.534)--(7.947,2.534);
\gpcolor{color=gp lt color border}
\gpsetlinetype{gp lt border}
\draw[gp path] (1.196,2.534)--(1.268,2.534);
\draw[gp path] (7.947,2.534)--(7.875,2.534);
\gpcolor{rgb color={0.000,0.000,0.000}}
\node[gp node right,font={\fontsize{10pt}{12pt}\selectfont}] at (1.012,2.534) {0};
\gpcolor{color=gp lt color axes}
\gpsetlinetype{gp lt axes}
\draw[gp path] (1.196,3.308)--(7.947,3.308);
\gpcolor{color=gp lt color border}
\gpsetlinetype{gp lt border}
\draw[gp path] (1.196,3.308)--(1.268,3.308);
\draw[gp path] (7.947,3.308)--(7.875,3.308);
\gpcolor{rgb color={0.000,0.000,0.000}}
\node[gp node right,font={\fontsize{10pt}{12pt}\selectfont}] at (1.012,3.308) {0.2};
\gpcolor{color=gp lt color axes}
\gpsetlinetype{gp lt axes}
\draw[gp path] (1.196,4.082)--(7.947,4.082);
\gpcolor{color=gp lt color border}
\gpsetlinetype{gp lt border}
\draw[gp path] (1.196,4.082)--(1.268,4.082);
\draw[gp path] (7.947,4.082)--(7.875,4.082);
\gpcolor{rgb color={0.000,0.000,0.000}}
\node[gp node right,font={\fontsize{10pt}{12pt}\selectfont}] at (1.012,4.082) {0.4};
\gpcolor{color=gp lt color axes}
\gpsetlinetype{gp lt axes}
\draw[gp path] (1.196,4.857)--(7.947,4.857);
\gpcolor{color=gp lt color border}
\gpsetlinetype{gp lt border}
\draw[gp path] (1.196,4.857)--(1.268,4.857);
\draw[gp path] (7.947,4.857)--(7.875,4.857);
\gpcolor{rgb color={0.000,0.000,0.000}}
\node[gp node right,font={\fontsize{10pt}{12pt}\selectfont}] at (1.012,4.857) {0.6};
\gpcolor{color=gp lt color axes}
\gpsetlinetype{gp lt axes}
\draw[gp path] (1.196,5.631)--(7.947,5.631);
\gpcolor{color=gp lt color border}
\gpsetlinetype{gp lt border}
\draw[gp path] (1.196,5.631)--(1.268,5.631);
\draw[gp path] (7.947,5.631)--(7.875,5.631);
\gpcolor{rgb color={0.000,0.000,0.000}}
\node[gp node right,font={\fontsize{10pt}{12pt}\selectfont}] at (1.012,5.631) {0.8};
\gpcolor{color=gp lt color axes}
\gpsetlinetype{gp lt axes}
\draw[gp path] (1.196,0.985)--(1.196,5.631);
\gpcolor{color=gp lt color border}
\gpsetlinetype{gp lt border}
\draw[gp path] (1.196,0.985)--(1.196,1.057);
\draw[gp path] (1.196,5.631)--(1.196,5.559);
\gpcolor{rgb color={0.000,0.000,0.000}}
\node[gp node center,font={\fontsize{10pt}{12pt}\selectfont}] at (1.196,0.677) {0.3};
\gpcolor{color=gp lt color axes}
\gpsetlinetype{gp lt axes}
\draw[gp path] (2.494,0.985)--(2.494,5.631);
\gpcolor{color=gp lt color border}
\gpsetlinetype{gp lt border}
\draw[gp path] (2.494,0.985)--(2.494,1.057);
\draw[gp path] (2.494,5.631)--(2.494,5.559);
\gpcolor{rgb color={0.000,0.000,0.000}}
\node[gp node center,font={\fontsize{10pt}{12pt}\selectfont}] at (2.494,0.677) {0.35};
\gpcolor{color=gp lt color axes}
\gpsetlinetype{gp lt axes}
\draw[gp path] (3.793,0.985)--(3.793,5.631);
\gpcolor{color=gp lt color border}
\gpsetlinetype{gp lt border}
\draw[gp path] (3.793,0.985)--(3.793,1.057);
\draw[gp path] (3.793,5.631)--(3.793,5.559);
\gpcolor{rgb color={0.000,0.000,0.000}}
\node[gp node center,font={\fontsize{10pt}{12pt}\selectfont}] at (3.793,0.677) {0.4};
\gpcolor{color=gp lt color axes}
\gpsetlinetype{gp lt axes}
\draw[gp path] (5.091,0.985)--(5.091,5.631);
\gpcolor{color=gp lt color border}
\gpsetlinetype{gp lt border}
\draw[gp path] (5.091,0.985)--(5.091,1.057);
\draw[gp path] (5.091,5.631)--(5.091,5.559);
\gpcolor{rgb color={0.000,0.000,0.000}}
\node[gp node center,font={\fontsize{10pt}{12pt}\selectfont}] at (5.091,0.677) {0.45};
\gpcolor{color=gp lt color axes}
\gpsetlinetype{gp lt axes}
\draw[gp path] (6.389,0.985)--(6.389,5.631);
\gpcolor{color=gp lt color border}
\gpsetlinetype{gp lt border}
\draw[gp path] (6.389,0.985)--(6.389,1.057);
\draw[gp path] (6.389,5.631)--(6.389,5.559);
\gpcolor{rgb color={0.000,0.000,0.000}}
\node[gp node center,font={\fontsize{10pt}{12pt}\selectfont}] at (6.389,0.677) {0.5};
\gpcolor{color=gp lt color axes}
\gpsetlinetype{gp lt axes}
\draw[gp path] (7.687,0.985)--(7.687,5.631);
\gpcolor{color=gp lt color border}
\gpsetlinetype{gp lt border}
\draw[gp path] (7.687,0.985)--(7.687,1.057);
\draw[gp path] (7.687,5.631)--(7.687,5.559);
\gpcolor{rgb color={0.000,0.000,0.000}}
\node[gp node center,font={\fontsize{10pt}{12pt}\selectfont}] at (7.687,0.677) {0.55};
\gpcolor{color=gp lt color border}
\draw[gp path] (1.196,5.631)--(1.196,0.985)--(7.947,0.985)--(7.947,5.631)--cycle;
\gpcolor{rgb color={0.000,0.000,0.000}}
\node[gp node center,font={\fontsize{10pt}{12pt}\selectfont}] at (4.571,0.215) {\large $x$};
\gpcolor{rgb color={0.502,0.502,0.502}}
\gpsetlinewidth{0.50}
\gpsetpointsize{2.67}
\gppoint{gp mark 7}{(1.210,4.952)}
\gppoint{gp mark 7}{(1.223,4.943)}
\gppoint{gp mark 7}{(1.235,4.933)}
\gppoint{gp mark 7}{(1.248,4.924)}
\gppoint{gp mark 7}{(1.261,4.914)}
\gppoint{gp mark 7}{(1.273,4.905)}
\gppoint{gp mark 7}{(1.286,4.895)}
\gppoint{gp mark 7}{(1.299,4.885)}
\gppoint{gp mark 7}{(1.311,4.876)}
\gppoint{gp mark 7}{(1.324,4.866)}
\gppoint{gp mark 7}{(1.337,4.856)}
\gppoint{gp mark 7}{(1.349,4.847)}
\gppoint{gp mark 7}{(1.362,4.837)}
\gppoint{gp mark 7}{(1.375,4.827)}
\gppoint{gp mark 7}{(1.387,4.818)}
\gppoint{gp mark 7}{(1.400,4.808)}
\gppoint{gp mark 7}{(1.413,4.798)}
\gppoint{gp mark 7}{(1.425,4.788)}
\gppoint{gp mark 7}{(1.438,4.779)}
\gppoint{gp mark 7}{(1.451,4.769)}
\gppoint{gp mark 7}{(1.464,4.759)}
\gppoint{gp mark 7}{(1.476,4.749)}
\gppoint{gp mark 7}{(1.489,4.739)}
\gppoint{gp mark 7}{(1.502,4.729)}
\gppoint{gp mark 7}{(1.514,4.719)}
\gppoint{gp mark 7}{(1.527,4.709)}
\gppoint{gp mark 7}{(1.540,4.699)}
\gppoint{gp mark 7}{(1.552,4.689)}
\gppoint{gp mark 7}{(1.565,4.679)}
\gppoint{gp mark 7}{(1.578,4.669)}
\gppoint{gp mark 7}{(1.590,4.659)}
\gppoint{gp mark 7}{(1.603,4.649)}
\gppoint{gp mark 7}{(1.616,4.639)}
\gppoint{gp mark 7}{(1.628,4.629)}
\gppoint{gp mark 7}{(1.641,4.619)}
\gppoint{gp mark 7}{(1.654,4.608)}
\gppoint{gp mark 7}{(1.666,4.598)}
\gppoint{gp mark 7}{(1.679,4.588)}
\gppoint{gp mark 7}{(1.692,4.578)}
\gppoint{gp mark 7}{(1.704,4.567)}
\gppoint{gp mark 7}{(1.717,4.557)}
\gppoint{gp mark 7}{(1.730,4.547)}
\gppoint{gp mark 7}{(1.742,4.536)}
\gppoint{gp mark 7}{(1.755,4.526)}
\gppoint{gp mark 7}{(1.768,4.516)}
\gppoint{gp mark 7}{(1.780,4.505)}
\gppoint{gp mark 7}{(1.793,4.495)}
\gppoint{gp mark 7}{(1.806,4.484)}
\gppoint{gp mark 7}{(1.819,4.474)}
\gppoint{gp mark 7}{(1.831,4.463)}
\gppoint{gp mark 7}{(1.844,4.452)}
\gppoint{gp mark 7}{(1.857,4.442)}
\gppoint{gp mark 7}{(1.869,4.431)}
\gppoint{gp mark 7}{(1.882,4.420)}
\gppoint{gp mark 7}{(1.895,4.409)}
\gppoint{gp mark 7}{(1.907,4.399)}
\gppoint{gp mark 7}{(1.920,4.388)}
\gppoint{gp mark 7}{(1.933,4.377)}
\gppoint{gp mark 7}{(1.945,4.366)}
\gppoint{gp mark 7}{(1.958,4.355)}
\gppoint{gp mark 7}{(1.971,4.344)}
\gppoint{gp mark 7}{(1.983,4.333)}
\gppoint{gp mark 7}{(1.996,4.322)}
\gppoint{gp mark 7}{(2.009,4.311)}
\gppoint{gp mark 7}{(2.021,4.300)}
\gppoint{gp mark 7}{(2.034,4.289)}
\gppoint{gp mark 7}{(2.047,4.277)}
\gppoint{gp mark 7}{(2.059,4.266)}
\gppoint{gp mark 7}{(2.072,4.255)}
\gppoint{gp mark 7}{(2.085,4.243)}
\gppoint{gp mark 7}{(2.097,4.232)}
\gppoint{gp mark 7}{(2.110,4.220)}
\gppoint{gp mark 7}{(2.123,4.209)}
\gppoint{gp mark 7}{(2.135,4.197)}
\gppoint{gp mark 7}{(2.148,4.186)}
\gppoint{gp mark 7}{(2.161,4.174)}
\gppoint{gp mark 7}{(2.174,4.162)}
\gppoint{gp mark 7}{(2.186,4.150)}
\gppoint{gp mark 7}{(2.199,4.139)}
\gppoint{gp mark 7}{(2.212,4.127)}
\gppoint{gp mark 7}{(2.224,4.115)}
\gppoint{gp mark 7}{(2.237,4.103)}
\gppoint{gp mark 7}{(2.250,4.091)}
\gppoint{gp mark 7}{(2.262,4.078)}
\gppoint{gp mark 7}{(2.275,4.066)}
\gppoint{gp mark 7}{(2.288,4.054)}
\gppoint{gp mark 7}{(2.300,4.042)}
\gppoint{gp mark 7}{(2.313,4.029)}
\gppoint{gp mark 7}{(2.326,4.017)}
\gppoint{gp mark 7}{(2.338,4.004)}
\gppoint{gp mark 7}{(2.351,3.991)}
\gppoint{gp mark 7}{(2.364,3.979)}
\gppoint{gp mark 7}{(2.376,3.966)}
\gppoint{gp mark 7}{(2.389,3.953)}
\gppoint{gp mark 7}{(2.402,3.940)}
\gppoint{gp mark 7}{(2.414,3.927)}
\gppoint{gp mark 7}{(2.427,3.914)}
\gppoint{gp mark 7}{(2.440,3.900)}
\gppoint{gp mark 7}{(2.452,3.887)}
\gppoint{gp mark 7}{(2.465,3.874)}
\gppoint{gp mark 7}{(2.478,3.860)}
\gppoint{gp mark 7}{(2.490,3.846)}
\gppoint{gp mark 7}{(2.503,3.833)}
\gppoint{gp mark 7}{(2.516,3.819)}
\gppoint{gp mark 7}{(2.529,3.805)}
\gppoint{gp mark 7}{(2.541,3.791)}
\gppoint{gp mark 7}{(2.554,3.776)}
\gppoint{gp mark 7}{(2.567,3.762)}
\gppoint{gp mark 7}{(2.579,3.747)}
\gppoint{gp mark 7}{(2.592,3.733)}
\gppoint{gp mark 7}{(2.605,3.718)}
\gppoint{gp mark 7}{(2.617,3.703)}
\gppoint{gp mark 7}{(2.630,3.688)}
\gppoint{gp mark 7}{(2.643,3.673)}
\gppoint{gp mark 7}{(2.655,3.658)}
\gppoint{gp mark 7}{(2.668,3.642)}
\gppoint{gp mark 7}{(2.681,3.627)}
\gppoint{gp mark 7}{(2.693,3.611)}
\gppoint{gp mark 7}{(2.706,3.595)}
\gppoint{gp mark 7}{(2.719,3.579)}
\gppoint{gp mark 7}{(2.731,3.563)}
\gppoint{gp mark 7}{(2.744,3.547)}
\gppoint{gp mark 7}{(2.757,3.531)}
\gppoint{gp mark 7}{(2.769,3.515)}
\gppoint{gp mark 7}{(2.782,3.497)}
\gppoint{gp mark 7}{(2.795,3.478)}
\gppoint{gp mark 7}{(2.807,3.460)}
\gppoint{gp mark 7}{(2.820,3.447)}
\gppoint{gp mark 7}{(2.833,3.444)}
\gppoint{gp mark 7}{(2.845,3.444)}
\gppoint{gp mark 7}{(2.858,3.446)}
\gppoint{gp mark 7}{(2.871,3.462)}
\gppoint{gp mark 7}{(2.883,3.501)}
\gppoint{gp mark 7}{(2.896,3.528)}
\gppoint{gp mark 7}{(2.909,3.486)}
\gppoint{gp mark 7}{(2.922,3.105)}
\gppoint{gp mark 7}{(2.934,2.248)}
\gppoint{gp mark 7}{(2.947,1.534)}
\gppoint{gp mark 7}{(2.960,1.290)}
\gppoint{gp mark 7}{(2.972,1.258)}
\gppoint{gp mark 7}{(2.985,1.252)}
\gppoint{gp mark 7}{(2.998,1.247)}
\gppoint{gp mark 7}{(3.010,1.247)}
\gppoint{gp mark 7}{(3.023,1.247)}
\gppoint{gp mark 7}{(3.036,1.247)}
\gppoint{gp mark 7}{(3.048,1.246)}
\gppoint{gp mark 7}{(3.061,1.246)}
\gppoint{gp mark 7}{(3.074,1.246)}
\gppoint{gp mark 7}{(3.086,1.246)}
\gppoint{gp mark 7}{(3.099,1.246)}
\gppoint{gp mark 7}{(3.112,1.246)}
\gppoint{gp mark 7}{(3.124,1.246)}
\gppoint{gp mark 7}{(3.137,1.246)}
\gppoint{gp mark 7}{(3.150,1.247)}
\gppoint{gp mark 7}{(3.162,1.248)}
\gppoint{gp mark 7}{(3.175,1.248)}
\gppoint{gp mark 7}{(3.188,1.248)}
\gppoint{gp mark 7}{(3.200,1.248)}
\gppoint{gp mark 7}{(3.213,1.248)}
\gppoint{gp mark 7}{(3.226,1.248)}
\gppoint{gp mark 7}{(3.238,1.247)}
\gppoint{gp mark 7}{(3.251,1.247)}
\gppoint{gp mark 7}{(3.264,1.246)}
\gppoint{gp mark 7}{(3.277,1.246)}
\gppoint{gp mark 7}{(3.289,1.246)}
\gppoint{gp mark 7}{(3.302,1.247)}
\gppoint{gp mark 7}{(3.315,1.247)}
\gppoint{gp mark 7}{(3.327,1.248)}
\gppoint{gp mark 7}{(3.340,1.249)}
\gppoint{gp mark 7}{(3.353,1.249)}
\gppoint{gp mark 7}{(3.365,1.249)}
\gppoint{gp mark 7}{(3.378,1.249)}
\gppoint{gp mark 7}{(3.391,1.248)}
\gppoint{gp mark 7}{(3.403,1.248)}
\gppoint{gp mark 7}{(3.416,1.247)}
\gppoint{gp mark 7}{(3.429,1.247)}
\gppoint{gp mark 7}{(3.441,1.247)}
\gppoint{gp mark 7}{(3.454,1.247)}
\gppoint{gp mark 7}{(3.467,1.247)}
\gppoint{gp mark 7}{(3.479,1.247)}
\gppoint{gp mark 7}{(3.492,1.248)}
\gppoint{gp mark 7}{(3.505,1.249)}
\gppoint{gp mark 7}{(3.517,1.249)}
\gppoint{gp mark 7}{(3.530,1.249)}
\gppoint{gp mark 7}{(3.543,1.249)}
\gppoint{gp mark 7}{(3.555,1.249)}
\gppoint{gp mark 7}{(3.568,1.248)}
\gppoint{gp mark 7}{(3.581,1.247)}
\gppoint{gp mark 7}{(3.593,1.247)}
\gppoint{gp mark 7}{(3.606,1.247)}
\gppoint{gp mark 7}{(3.619,1.247)}
\gppoint{gp mark 7}{(3.632,1.247)}
\gppoint{gp mark 7}{(3.644,1.247)}
\gppoint{gp mark 7}{(3.657,1.248)}
\gppoint{gp mark 7}{(3.670,1.249)}
\gppoint{gp mark 7}{(3.682,1.250)}
\gppoint{gp mark 7}{(3.695,1.250)}
\gppoint{gp mark 7}{(3.708,1.250)}
\gppoint{gp mark 7}{(3.720,1.250)}
\gppoint{gp mark 7}{(3.733,1.263)}
\gppoint{gp mark 7}{(3.746,1.378)}
\gppoint{gp mark 7}{(3.758,1.583)}
\gppoint{gp mark 7}{(3.771,1.632)}
\gppoint{gp mark 7}{(3.784,1.634)}
\gppoint{gp mark 7}{(3.796,1.634)}
\gppoint{gp mark 7}{(3.809,1.634)}
\gppoint{gp mark 7}{(3.822,1.634)}
\gppoint{gp mark 7}{(3.834,1.634)}
\gppoint{gp mark 7}{(3.847,1.634)}
\gppoint{gp mark 7}{(3.860,1.634)}
\gppoint{gp mark 7}{(3.872,1.634)}
\gppoint{gp mark 7}{(3.885,1.634)}
\gppoint{gp mark 7}{(3.898,1.634)}
\gppoint{gp mark 7}{(3.910,1.634)}
\gppoint{gp mark 7}{(3.923,1.634)}
\gppoint{gp mark 7}{(3.936,1.634)}
\gppoint{gp mark 7}{(3.948,1.634)}
\gppoint{gp mark 7}{(3.961,1.634)}
\gppoint{gp mark 7}{(3.974,1.634)}
\gppoint{gp mark 7}{(3.987,1.634)}
\gppoint{gp mark 7}{(3.999,1.634)}
\gppoint{gp mark 7}{(4.012,1.634)}
\gppoint{gp mark 7}{(4.025,1.634)}
\gppoint{gp mark 7}{(4.037,1.634)}
\gppoint{gp mark 7}{(4.050,1.634)}
\gppoint{gp mark 7}{(4.063,1.634)}
\gppoint{gp mark 7}{(4.075,1.634)}
\gppoint{gp mark 7}{(4.088,1.634)}
\gppoint{gp mark 7}{(4.101,1.634)}
\gppoint{gp mark 7}{(4.113,1.634)}
\gppoint{gp mark 7}{(4.126,1.634)}
\gppoint{gp mark 7}{(4.139,1.634)}
\gppoint{gp mark 7}{(4.151,1.634)}
\gppoint{gp mark 7}{(4.164,1.634)}
\gppoint{gp mark 7}{(4.177,1.634)}
\gppoint{gp mark 7}{(4.189,1.634)}
\gppoint{gp mark 7}{(4.202,1.634)}
\gppoint{gp mark 7}{(4.215,1.634)}
\gppoint{gp mark 7}{(4.227,1.634)}
\gppoint{gp mark 7}{(4.240,1.634)}
\gppoint{gp mark 7}{(4.253,1.634)}
\gppoint{gp mark 7}{(4.265,1.634)}
\gppoint{gp mark 7}{(4.278,1.634)}
\gppoint{gp mark 7}{(4.291,1.634)}
\gppoint{gp mark 7}{(4.303,1.634)}
\gppoint{gp mark 7}{(4.316,1.634)}
\gppoint{gp mark 7}{(4.329,1.634)}
\gppoint{gp mark 7}{(4.342,1.634)}
\gppoint{gp mark 7}{(4.354,1.634)}
\gppoint{gp mark 7}{(4.367,1.634)}
\gppoint{gp mark 7}{(4.380,1.634)}
\gppoint{gp mark 7}{(4.392,1.634)}
\gppoint{gp mark 7}{(4.405,1.634)}
\gppoint{gp mark 7}{(4.418,1.634)}
\gppoint{gp mark 7}{(4.430,1.634)}
\gppoint{gp mark 7}{(4.443,1.634)}
\gppoint{gp mark 7}{(4.456,1.634)}
\gppoint{gp mark 7}{(4.468,1.634)}
\gppoint{gp mark 7}{(4.481,1.634)}
\gppoint{gp mark 7}{(4.494,1.634)}
\gppoint{gp mark 7}{(4.506,1.634)}
\gppoint{gp mark 7}{(4.519,1.634)}
\gppoint{gp mark 7}{(4.532,1.634)}
\gppoint{gp mark 7}{(4.544,1.634)}
\gppoint{gp mark 7}{(4.557,1.634)}
\gppoint{gp mark 7}{(4.570,1.634)}
\gppoint{gp mark 7}{(4.582,1.634)}
\gppoint{gp mark 7}{(4.595,1.634)}
\gppoint{gp mark 7}{(4.608,1.634)}
\gppoint{gp mark 7}{(4.620,1.634)}
\gppoint{gp mark 7}{(4.633,1.634)}
\gppoint{gp mark 7}{(4.646,1.634)}
\gppoint{gp mark 7}{(4.658,1.634)}
\gppoint{gp mark 7}{(4.671,1.634)}
\gppoint{gp mark 7}{(4.684,1.634)}
\gppoint{gp mark 7}{(4.697,1.634)}
\gppoint{gp mark 7}{(4.709,1.634)}
\gppoint{gp mark 7}{(4.722,1.634)}
\gppoint{gp mark 7}{(4.735,1.634)}
\gppoint{gp mark 7}{(4.747,1.634)}
\gppoint{gp mark 7}{(4.760,1.634)}
\gppoint{gp mark 7}{(4.773,1.634)}
\gppoint{gp mark 7}{(4.785,1.634)}
\gppoint{gp mark 7}{(4.798,1.634)}
\gppoint{gp mark 7}{(4.811,1.634)}
\gppoint{gp mark 7}{(4.823,1.634)}
\gppoint{gp mark 7}{(4.836,1.634)}
\gppoint{gp mark 7}{(4.849,1.634)}
\gppoint{gp mark 7}{(4.861,1.634)}
\gppoint{gp mark 7}{(4.874,1.634)}
\gppoint{gp mark 7}{(4.887,1.634)}
\gppoint{gp mark 7}{(4.899,1.634)}
\gppoint{gp mark 7}{(4.912,1.634)}
\gppoint{gp mark 7}{(4.925,1.634)}
\gppoint{gp mark 7}{(4.937,1.634)}
\gppoint{gp mark 7}{(4.950,1.634)}
\gppoint{gp mark 7}{(4.963,1.634)}
\gppoint{gp mark 7}{(4.975,1.634)}
\gppoint{gp mark 7}{(4.988,1.634)}
\gppoint{gp mark 7}{(5.001,1.634)}
\gppoint{gp mark 7}{(5.013,1.634)}
\gppoint{gp mark 7}{(5.026,1.634)}
\gppoint{gp mark 7}{(5.039,1.634)}
\gppoint{gp mark 7}{(5.052,1.634)}
\gppoint{gp mark 7}{(5.064,1.634)}
\gppoint{gp mark 7}{(5.077,1.634)}
\gppoint{gp mark 7}{(5.090,1.634)}
\gppoint{gp mark 7}{(5.102,1.634)}
\gppoint{gp mark 7}{(5.115,1.634)}
\gppoint{gp mark 7}{(5.128,1.634)}
\gppoint{gp mark 7}{(5.140,1.634)}
\gppoint{gp mark 7}{(5.153,1.634)}
\gppoint{gp mark 7}{(5.166,1.634)}
\gppoint{gp mark 7}{(5.178,1.634)}
\gppoint{gp mark 7}{(5.191,1.634)}
\gppoint{gp mark 7}{(5.204,1.634)}
\gppoint{gp mark 7}{(5.216,1.634)}
\gppoint{gp mark 7}{(5.229,1.634)}
\gppoint{gp mark 7}{(5.242,1.634)}
\gppoint{gp mark 7}{(5.254,1.634)}
\gppoint{gp mark 7}{(5.267,1.634)}
\gppoint{gp mark 7}{(5.280,1.634)}
\gppoint{gp mark 7}{(5.292,1.634)}
\gppoint{gp mark 7}{(5.305,1.634)}
\gppoint{gp mark 7}{(5.318,1.634)}
\gppoint{gp mark 7}{(5.330,1.634)}
\gppoint{gp mark 7}{(5.343,1.634)}
\gppoint{gp mark 7}{(5.356,1.634)}
\gppoint{gp mark 7}{(5.368,1.634)}
\gppoint{gp mark 7}{(5.381,1.634)}
\gppoint{gp mark 7}{(5.394,1.634)}
\gppoint{gp mark 7}{(5.407,1.634)}
\gppoint{gp mark 7}{(5.419,1.634)}
\gppoint{gp mark 7}{(5.432,1.634)}
\gppoint{gp mark 7}{(5.445,1.634)}
\gppoint{gp mark 7}{(5.457,1.634)}
\gppoint{gp mark 7}{(5.470,1.634)}
\gppoint{gp mark 7}{(5.483,1.634)}
\gppoint{gp mark 7}{(5.495,1.634)}
\gppoint{gp mark 7}{(5.508,1.634)}
\gppoint{gp mark 7}{(5.521,1.634)}
\gppoint{gp mark 7}{(5.533,1.634)}
\gppoint{gp mark 7}{(5.546,1.634)}
\gppoint{gp mark 7}{(5.559,1.634)}
\gppoint{gp mark 7}{(5.571,1.634)}
\gppoint{gp mark 7}{(5.584,1.634)}
\gppoint{gp mark 7}{(5.597,1.634)}
\gppoint{gp mark 7}{(5.609,1.634)}
\gppoint{gp mark 7}{(5.622,1.634)}
\gppoint{gp mark 7}{(5.635,1.634)}
\gppoint{gp mark 7}{(5.647,1.634)}
\gppoint{gp mark 7}{(5.660,1.634)}
\gppoint{gp mark 7}{(5.673,1.634)}
\gppoint{gp mark 7}{(5.685,1.634)}
\gppoint{gp mark 7}{(5.698,1.634)}
\gppoint{gp mark 7}{(5.711,1.634)}
\gppoint{gp mark 7}{(5.723,1.634)}
\gppoint{gp mark 7}{(5.736,1.634)}
\gppoint{gp mark 7}{(5.749,1.634)}
\gppoint{gp mark 7}{(5.761,1.634)}
\gppoint{gp mark 7}{(5.774,1.634)}
\gppoint{gp mark 7}{(5.787,1.634)}
\gppoint{gp mark 7}{(5.800,1.634)}
\gppoint{gp mark 7}{(5.812,1.634)}
\gppoint{gp mark 7}{(5.825,1.634)}
\gppoint{gp mark 7}{(5.838,1.634)}
\gppoint{gp mark 7}{(5.850,1.634)}
\gppoint{gp mark 7}{(5.863,1.634)}
\gppoint{gp mark 7}{(5.876,1.634)}
\gppoint{gp mark 7}{(5.888,1.634)}
\gppoint{gp mark 7}{(5.901,1.634)}
\gppoint{gp mark 7}{(5.914,1.634)}
\gppoint{gp mark 7}{(5.926,1.634)}
\gppoint{gp mark 7}{(5.939,1.634)}
\gppoint{gp mark 7}{(5.952,1.634)}
\gppoint{gp mark 7}{(5.964,1.634)}
\gppoint{gp mark 7}{(5.977,1.634)}
\gppoint{gp mark 7}{(5.990,1.634)}
\gppoint{gp mark 7}{(6.002,1.634)}
\gppoint{gp mark 7}{(6.015,1.634)}
\gppoint{gp mark 7}{(6.028,1.634)}
\gppoint{gp mark 7}{(6.040,1.634)}
\gppoint{gp mark 7}{(6.053,1.634)}
\gppoint{gp mark 7}{(6.066,1.634)}
\gppoint{gp mark 7}{(6.078,1.634)}
\gppoint{gp mark 7}{(6.091,1.634)}
\gppoint{gp mark 7}{(6.104,1.634)}
\gppoint{gp mark 7}{(6.116,1.634)}
\gppoint{gp mark 7}{(6.129,1.634)}
\gppoint{gp mark 7}{(6.142,1.634)}
\gppoint{gp mark 7}{(6.155,1.634)}
\gppoint{gp mark 7}{(6.167,1.634)}
\gppoint{gp mark 7}{(6.180,1.634)}
\gppoint{gp mark 7}{(6.193,1.634)}
\gppoint{gp mark 7}{(6.205,1.634)}
\gppoint{gp mark 7}{(6.218,1.634)}
\gppoint{gp mark 7}{(6.231,1.634)}
\gppoint{gp mark 7}{(6.243,1.634)}
\gppoint{gp mark 7}{(6.256,1.634)}
\gppoint{gp mark 7}{(6.269,1.634)}
\gppoint{gp mark 7}{(6.281,1.634)}
\gppoint{gp mark 7}{(6.294,1.634)}
\gppoint{gp mark 7}{(6.307,1.634)}
\gppoint{gp mark 7}{(6.319,1.634)}
\gppoint{gp mark 7}{(6.332,1.634)}
\gppoint{gp mark 7}{(6.345,1.634)}
\gppoint{gp mark 7}{(6.357,1.634)}
\gppoint{gp mark 7}{(6.370,1.634)}
\gppoint{gp mark 7}{(6.383,1.634)}
\gppoint{gp mark 7}{(6.395,1.634)}
\gppoint{gp mark 7}{(6.408,1.634)}
\gppoint{gp mark 7}{(6.421,1.634)}
\gppoint{gp mark 7}{(6.433,1.634)}
\gppoint{gp mark 7}{(6.446,1.634)}
\gppoint{gp mark 7}{(6.459,1.634)}
\gppoint{gp mark 7}{(6.471,1.634)}
\gppoint{gp mark 7}{(6.484,1.634)}
\gppoint{gp mark 7}{(6.497,1.634)}
\gppoint{gp mark 7}{(6.510,1.634)}
\gppoint{gp mark 7}{(6.522,1.634)}
\gppoint{gp mark 7}{(6.535,1.634)}
\gppoint{gp mark 7}{(6.548,1.634)}
\gppoint{gp mark 7}{(6.560,1.634)}
\gppoint{gp mark 7}{(6.573,1.634)}
\gppoint{gp mark 7}{(6.586,1.634)}
\gppoint{gp mark 7}{(6.598,1.634)}
\gppoint{gp mark 7}{(6.611,1.634)}
\gppoint{gp mark 7}{(6.624,1.634)}
\gppoint{gp mark 7}{(6.636,1.634)}
\gppoint{gp mark 7}{(6.649,1.634)}
\gppoint{gp mark 7}{(6.662,1.634)}
\gppoint{gp mark 7}{(6.674,1.634)}
\gppoint{gp mark 7}{(6.687,1.634)}
\gppoint{gp mark 7}{(6.700,1.634)}
\gppoint{gp mark 7}{(6.712,1.634)}
\gppoint{gp mark 7}{(6.725,1.634)}
\gppoint{gp mark 7}{(6.738,1.634)}
\gppoint{gp mark 7}{(6.750,1.634)}
\gppoint{gp mark 7}{(6.763,1.634)}
\gppoint{gp mark 7}{(6.776,1.634)}
\gppoint{gp mark 7}{(6.788,1.634)}
\gppoint{gp mark 7}{(6.801,1.634)}
\gppoint{gp mark 7}{(6.814,1.634)}
\gppoint{gp mark 7}{(6.826,1.634)}
\gppoint{gp mark 7}{(6.839,1.634)}
\gppoint{gp mark 7}{(6.852,1.634)}
\gppoint{gp mark 7}{(6.865,1.634)}
\gppoint{gp mark 7}{(6.877,1.634)}
\gppoint{gp mark 7}{(6.890,1.634)}
\gppoint{gp mark 7}{(6.903,1.634)}
\gppoint{gp mark 7}{(6.915,1.634)}
\gppoint{gp mark 7}{(6.928,1.634)}
\gppoint{gp mark 7}{(6.941,1.634)}
\gppoint{gp mark 7}{(6.953,1.634)}
\gppoint{gp mark 7}{(6.966,1.634)}
\gppoint{gp mark 7}{(6.979,1.634)}
\gppoint{gp mark 7}{(6.991,1.634)}
\gppoint{gp mark 7}{(7.004,1.634)}
\gppoint{gp mark 7}{(7.017,1.634)}
\gppoint{gp mark 7}{(7.029,1.634)}
\gppoint{gp mark 7}{(7.042,1.634)}
\gppoint{gp mark 7}{(7.055,1.634)}
\gppoint{gp mark 7}{(7.067,1.634)}
\gppoint{gp mark 7}{(7.080,1.634)}
\gppoint{gp mark 7}{(7.093,1.634)}
\gppoint{gp mark 7}{(7.105,1.634)}
\gppoint{gp mark 7}{(7.118,1.634)}
\gppoint{gp mark 7}{(7.131,1.634)}
\gppoint{gp mark 7}{(7.143,1.634)}
\gppoint{gp mark 7}{(7.156,1.634)}
\gppoint{gp mark 7}{(7.169,1.634)}
\gppoint{gp mark 7}{(7.181,1.634)}
\gppoint{gp mark 7}{(7.194,1.634)}
\gppoint{gp mark 7}{(7.207,1.634)}
\gppoint{gp mark 7}{(7.220,1.634)}
\gppoint{gp mark 7}{(7.232,1.634)}
\gppoint{gp mark 7}{(7.245,1.634)}
\gppoint{gp mark 7}{(7.258,1.634)}
\gppoint{gp mark 7}{(7.270,1.634)}
\gppoint{gp mark 7}{(7.283,1.634)}
\gppoint{gp mark 7}{(7.296,1.634)}
\gppoint{gp mark 7}{(7.308,1.634)}
\gppoint{gp mark 7}{(7.321,1.634)}
\gppoint{gp mark 7}{(7.334,1.634)}
\gppoint{gp mark 7}{(7.346,1.634)}
\gppoint{gp mark 7}{(7.359,1.634)}
\gppoint{gp mark 7}{(7.372,1.634)}
\gppoint{gp mark 7}{(7.384,1.634)}
\gppoint{gp mark 7}{(7.397,1.634)}
\gppoint{gp mark 7}{(7.410,1.634)}
\gppoint{gp mark 7}{(7.422,1.634)}
\gppoint{gp mark 7}{(7.435,1.634)}
\gppoint{gp mark 7}{(7.448,1.634)}
\gppoint{gp mark 7}{(7.460,1.634)}
\gppoint{gp mark 7}{(7.473,1.634)}
\gppoint{gp mark 7}{(7.486,1.634)}
\gppoint{gp mark 7}{(7.498,1.634)}
\gppoint{gp mark 7}{(7.511,1.634)}
\gppoint{gp mark 7}{(7.524,1.634)}
\gppoint{gp mark 7}{(7.536,1.634)}
\gppoint{gp mark 7}{(7.549,1.634)}
\gppoint{gp mark 7}{(7.562,1.634)}
\gppoint{gp mark 7}{(7.575,1.634)}
\gppoint{gp mark 7}{(7.587,1.634)}
\gppoint{gp mark 7}{(7.600,1.634)}
\gppoint{gp mark 7}{(7.613,1.634)}
\gppoint{gp mark 7}{(7.625,1.634)}
\gppoint{gp mark 7}{(7.638,1.634)}
\gppoint{gp mark 7}{(7.651,1.634)}
\gppoint{gp mark 7}{(7.663,1.634)}
\gppoint{gp mark 7}{(7.676,1.634)}
\gppoint{gp mark 7}{(7.689,1.634)}
\gppoint{gp mark 7}{(7.701,1.634)}
\gppoint{gp mark 7}{(7.714,1.634)}
\gppoint{gp mark 7}{(7.727,1.634)}
\gppoint{gp mark 7}{(7.739,1.634)}
\gppoint{gp mark 7}{(7.752,1.634)}
\gppoint{gp mark 7}{(7.765,1.634)}
\gppoint{gp mark 7}{(7.777,1.634)}
\gppoint{gp mark 7}{(7.790,1.634)}
\gppoint{gp mark 7}{(7.803,1.634)}
\gppoint{gp mark 7}{(7.815,1.634)}
\gppoint{gp mark 7}{(7.828,1.634)}
\gppoint{gp mark 7}{(7.841,1.634)}
\gppoint{gp mark 7}{(7.853,1.634)}
\gppoint{gp mark 7}{(7.866,1.634)}
\gppoint{gp mark 7}{(7.879,1.634)}
\gppoint{gp mark 7}{(7.891,1.634)}
\gppoint{gp mark 7}{(7.904,1.634)}
\gppoint{gp mark 7}{(7.917,1.634)}
\gppoint{gp mark 7}{(7.930,1.634)}
\gpcolor{rgb color={1.000,0.000,0.000}}
\gpsetpointsize{4.44}
\gppoint{gp mark 7}{(1.210,4.952)}
\gppoint{gp mark 7}{(1.223,4.943)}
\gppoint{gp mark 7}{(1.235,4.933)}
\gppoint{gp mark 7}{(1.248,4.924)}
\gppoint{gp mark 7}{(1.261,4.914)}
\gppoint{gp mark 7}{(1.273,4.905)}
\gppoint{gp mark 7}{(1.286,4.895)}
\gppoint{gp mark 7}{(1.299,4.886)}
\gppoint{gp mark 7}{(1.311,4.876)}
\gppoint{gp mark 7}{(1.324,4.866)}
\gppoint{gp mark 7}{(1.337,4.857)}
\gppoint{gp mark 7}{(1.349,4.847)}
\gppoint{gp mark 7}{(1.362,4.837)}
\gppoint{gp mark 7}{(1.375,4.828)}
\gppoint{gp mark 7}{(1.387,4.818)}
\gppoint{gp mark 7}{(1.400,4.808)}
\gppoint{gp mark 7}{(1.413,4.799)}
\gppoint{gp mark 7}{(1.425,4.789)}
\gppoint{gp mark 7}{(1.438,4.779)}
\gppoint{gp mark 7}{(1.451,4.769)}
\gppoint{gp mark 7}{(1.464,4.759)}
\gppoint{gp mark 7}{(1.476,4.750)}
\gppoint{gp mark 7}{(1.489,4.740)}
\gppoint{gp mark 7}{(1.502,4.730)}
\gppoint{gp mark 7}{(1.514,4.720)}
\gppoint{gp mark 7}{(1.527,4.710)}
\gppoint{gp mark 7}{(1.540,4.700)}
\gppoint{gp mark 7}{(1.552,4.690)}
\gppoint{gp mark 7}{(1.565,4.680)}
\gppoint{gp mark 7}{(1.578,4.670)}
\gppoint{gp mark 7}{(1.590,4.660)}
\gppoint{gp mark 7}{(1.603,4.650)}
\gppoint{gp mark 7}{(1.616,4.640)}
\gppoint{gp mark 7}{(1.628,4.630)}
\gppoint{gp mark 7}{(1.641,4.620)}
\gppoint{gp mark 7}{(1.654,4.610)}
\gppoint{gp mark 7}{(1.666,4.600)}
\gppoint{gp mark 7}{(1.679,4.590)}
\gppoint{gp mark 7}{(1.692,4.579)}
\gppoint{gp mark 7}{(1.704,4.569)}
\gppoint{gp mark 7}{(1.717,4.559)}
\gppoint{gp mark 7}{(1.730,4.549)}
\gppoint{gp mark 7}{(1.742,4.538)}
\gppoint{gp mark 7}{(1.755,4.528)}
\gppoint{gp mark 7}{(1.768,4.518)}
\gppoint{gp mark 7}{(1.780,4.507)}
\gppoint{gp mark 7}{(1.793,4.497)}
\gppoint{gp mark 7}{(1.806,4.486)}
\gppoint{gp mark 7}{(1.819,4.476)}
\gppoint{gp mark 7}{(1.831,4.465)}
\gppoint{gp mark 7}{(1.844,4.455)}
\gppoint{gp mark 7}{(1.857,4.444)}
\gppoint{gp mark 7}{(1.869,4.434)}
\gppoint{gp mark 7}{(1.882,4.423)}
\gppoint{gp mark 7}{(1.895,4.412)}
\gppoint{gp mark 7}{(1.907,4.402)}
\gppoint{gp mark 7}{(1.920,4.391)}
\gppoint{gp mark 7}{(1.933,4.380)}
\gppoint{gp mark 7}{(1.945,4.370)}
\gppoint{gp mark 7}{(1.958,4.359)}
\gppoint{gp mark 7}{(1.971,4.348)}
\gppoint{gp mark 7}{(1.983,4.337)}
\gppoint{gp mark 7}{(1.996,4.326)}
\gppoint{gp mark 7}{(2.009,4.315)}
\gppoint{gp mark 7}{(2.021,4.304)}
\gppoint{gp mark 7}{(2.034,4.293)}
\gppoint{gp mark 7}{(2.047,4.282)}
\gppoint{gp mark 7}{(2.059,4.271)}
\gppoint{gp mark 7}{(2.072,4.260)}
\gppoint{gp mark 7}{(2.085,4.249)}
\gppoint{gp mark 7}{(2.097,4.237)}
\gppoint{gp mark 7}{(2.110,4.226)}
\gppoint{gp mark 7}{(2.123,4.215)}
\gppoint{gp mark 7}{(2.135,4.203)}
\gppoint{gp mark 7}{(2.148,4.192)}
\gppoint{gp mark 7}{(2.161,4.181)}
\gppoint{gp mark 7}{(2.174,4.169)}
\gppoint{gp mark 7}{(2.186,4.158)}
\gppoint{gp mark 7}{(2.199,4.146)}
\gppoint{gp mark 7}{(2.212,4.134)}
\gppoint{gp mark 7}{(2.224,4.123)}
\gppoint{gp mark 7}{(2.237,4.111)}
\gppoint{gp mark 7}{(2.250,4.099)}
\gppoint{gp mark 7}{(2.262,4.088)}
\gppoint{gp mark 7}{(2.275,4.076)}
\gppoint{gp mark 7}{(2.288,4.064)}
\gppoint{gp mark 7}{(2.300,4.052)}
\gppoint{gp mark 7}{(2.313,4.040)}
\gppoint{gp mark 7}{(2.326,4.028)}
\gppoint{gp mark 7}{(2.338,4.017)}
\gppoint{gp mark 7}{(2.351,4.005)}
\gppoint{gp mark 7}{(2.364,3.993)}
\gppoint{gp mark 7}{(2.376,3.981)}
\gppoint{gp mark 7}{(2.389,3.970)}
\gppoint{gp mark 7}{(2.402,3.959)}
\gppoint{gp mark 7}{(2.414,3.950)}
\gppoint{gp mark 7}{(2.427,3.942)}
\gppoint{gp mark 7}{(2.440,3.936)}
\gppoint{gp mark 7}{(2.452,3.933)}
\gppoint{gp mark 7}{(2.465,3.932)}
\gppoint{gp mark 7}{(2.478,3.932)}
\gppoint{gp mark 7}{(2.490,3.932)}
\gppoint{gp mark 7}{(2.503,3.932)}
\gppoint{gp mark 7}{(2.516,3.932)}
\gppoint{gp mark 7}{(2.529,3.933)}
\gppoint{gp mark 7}{(2.541,3.935)}
\gppoint{gp mark 7}{(2.554,3.937)}
\gppoint{gp mark 7}{(2.567,3.939)}
\gppoint{gp mark 7}{(2.579,3.940)}
\gppoint{gp mark 7}{(2.592,3.940)}
\gppoint{gp mark 7}{(2.605,3.940)}
\gppoint{gp mark 7}{(2.617,3.940)}
\gppoint{gp mark 7}{(2.630,3.940)}
\gppoint{gp mark 7}{(2.643,3.941)}
\gppoint{gp mark 7}{(2.655,3.941)}
\gppoint{gp mark 7}{(2.668,3.941)}
\gppoint{gp mark 7}{(2.681,3.941)}
\gppoint{gp mark 7}{(2.693,3.941)}
\gppoint{gp mark 7}{(2.706,3.941)}
\gppoint{gp mark 7}{(2.719,3.941)}
\gppoint{gp mark 7}{(2.731,3.941)}
\gppoint{gp mark 7}{(2.744,3.941)}
\gppoint{gp mark 7}{(2.757,3.941)}
\gppoint{gp mark 7}{(2.769,3.941)}
\gppoint{gp mark 7}{(2.782,3.941)}
\gppoint{gp mark 7}{(2.795,3.941)}
\gppoint{gp mark 7}{(2.807,3.941)}
\gppoint{gp mark 7}{(2.820,3.941)}
\gppoint{gp mark 7}{(2.833,3.941)}
\gppoint{gp mark 7}{(2.845,3.940)}
\gppoint{gp mark 7}{(2.858,3.936)}
\gppoint{gp mark 7}{(2.871,3.915)}
\gppoint{gp mark 7}{(2.883,3.319)}
\gppoint{gp mark 7}{(2.896,1.748)}
\gppoint{gp mark 7}{(2.909,1.263)}
\gppoint{gp mark 7}{(2.922,1.237)}
\gppoint{gp mark 7}{(2.934,1.237)}
\gppoint{gp mark 7}{(2.947,1.234)}
\gppoint{gp mark 7}{(2.960,1.230)}
\gppoint{gp mark 7}{(2.972,1.228)}
\gppoint{gp mark 7}{(2.985,1.227)}
\gppoint{gp mark 7}{(2.998,1.226)}
\gppoint{gp mark 7}{(3.010,1.226)}
\gppoint{gp mark 7}{(3.023,1.226)}
\gppoint{gp mark 7}{(3.036,1.225)}
\gppoint{gp mark 7}{(3.048,1.225)}
\gppoint{gp mark 7}{(3.061,1.226)}
\gppoint{gp mark 7}{(3.074,1.226)}
\gppoint{gp mark 7}{(3.086,1.226)}
\gppoint{gp mark 7}{(3.099,1.226)}
\gppoint{gp mark 7}{(3.112,1.226)}
\gppoint{gp mark 7}{(3.124,1.225)}
\gppoint{gp mark 7}{(3.137,1.225)}
\gppoint{gp mark 7}{(3.150,1.225)}
\gppoint{gp mark 7}{(3.162,1.225)}
\gppoint{gp mark 7}{(3.175,1.225)}
\gppoint{gp mark 7}{(3.188,1.224)}
\gppoint{gp mark 7}{(3.200,1.225)}
\gppoint{gp mark 7}{(3.213,1.225)}
\gppoint{gp mark 7}{(3.226,1.226)}
\gppoint{gp mark 7}{(3.238,1.227)}
\gppoint{gp mark 7}{(3.251,1.227)}
\gppoint{gp mark 7}{(3.264,1.226)}
\gppoint{gp mark 7}{(3.277,1.226)}
\gppoint{gp mark 7}{(3.289,1.225)}
\gppoint{gp mark 7}{(3.302,1.225)}
\gppoint{gp mark 7}{(3.315,1.225)}
\gppoint{gp mark 7}{(3.327,1.225)}
\gppoint{gp mark 7}{(3.340,1.225)}
\gppoint{gp mark 7}{(3.353,1.225)}
\gppoint{gp mark 7}{(3.365,1.225)}
\gppoint{gp mark 7}{(3.378,1.226)}
\gppoint{gp mark 7}{(3.391,1.226)}
\gppoint{gp mark 7}{(3.403,1.227)}
\gppoint{gp mark 7}{(3.416,1.227)}
\gppoint{gp mark 7}{(3.429,1.226)}
\gppoint{gp mark 7}{(3.441,1.226)}
\gppoint{gp mark 7}{(3.454,1.225)}
\gppoint{gp mark 7}{(3.467,1.224)}
\gppoint{gp mark 7}{(3.479,1.224)}
\gppoint{gp mark 7}{(3.492,1.225)}
\gppoint{gp mark 7}{(3.505,1.225)}
\gppoint{gp mark 7}{(3.517,1.225)}
\gppoint{gp mark 7}{(3.530,1.226)}
\gppoint{gp mark 7}{(3.543,1.226)}
\gppoint{gp mark 7}{(3.555,1.227)}
\gppoint{gp mark 7}{(3.568,1.227)}
\gppoint{gp mark 7}{(3.581,1.227)}
\gppoint{gp mark 7}{(3.593,1.226)}
\gppoint{gp mark 7}{(3.606,1.225)}
\gppoint{gp mark 7}{(3.619,1.224)}
\gppoint{gp mark 7}{(3.632,1.224)}
\gppoint{gp mark 7}{(3.644,1.224)}
\gppoint{gp mark 7}{(3.657,1.224)}
\gppoint{gp mark 7}{(3.670,1.224)}
\gppoint{gp mark 7}{(3.682,1.225)}
\gppoint{gp mark 7}{(3.695,1.227)}
\gppoint{gp mark 7}{(3.708,1.227)}
\gppoint{gp mark 7}{(3.720,1.227)}
\gppoint{gp mark 7}{(3.733,1.227)}
\gppoint{gp mark 7}{(3.746,1.226)}
\gppoint{gp mark 7}{(3.758,1.225)}
\gppoint{gp mark 7}{(3.771,1.226)}
\gppoint{gp mark 7}{(3.784,1.244)}
\gppoint{gp mark 7}{(3.796,1.373)}
\gppoint{gp mark 7}{(3.809,1.590)}
\gppoint{gp mark 7}{(3.822,1.631)}
\gppoint{gp mark 7}{(3.834,1.635)}
\gppoint{gp mark 7}{(3.847,1.635)}
\gppoint{gp mark 7}{(3.860,1.635)}
\gppoint{gp mark 7}{(3.872,1.635)}
\gppoint{gp mark 7}{(3.885,1.635)}
\gppoint{gp mark 7}{(3.898,1.635)}
\gppoint{gp mark 7}{(3.910,1.635)}
\gppoint{gp mark 7}{(3.923,1.635)}
\gppoint{gp mark 7}{(3.936,1.635)}
\gppoint{gp mark 7}{(3.948,1.635)}
\gppoint{gp mark 7}{(3.961,1.635)}
\gppoint{gp mark 7}{(3.974,1.635)}
\gppoint{gp mark 7}{(3.987,1.635)}
\gppoint{gp mark 7}{(3.999,1.635)}
\gppoint{gp mark 7}{(4.012,1.635)}
\gppoint{gp mark 7}{(4.025,1.635)}
\gppoint{gp mark 7}{(4.037,1.635)}
\gppoint{gp mark 7}{(4.050,1.635)}
\gppoint{gp mark 7}{(4.063,1.635)}
\gppoint{gp mark 7}{(4.075,1.635)}
\gppoint{gp mark 7}{(4.088,1.635)}
\gppoint{gp mark 7}{(4.101,1.635)}
\gppoint{gp mark 7}{(4.113,1.635)}
\gppoint{gp mark 7}{(4.126,1.635)}
\gppoint{gp mark 7}{(4.139,1.635)}
\gppoint{gp mark 7}{(4.151,1.635)}
\gppoint{gp mark 7}{(4.164,1.635)}
\gppoint{gp mark 7}{(4.177,1.635)}
\gppoint{gp mark 7}{(4.189,1.635)}
\gppoint{gp mark 7}{(4.202,1.635)}
\gppoint{gp mark 7}{(4.215,1.635)}
\gppoint{gp mark 7}{(4.227,1.635)}
\gppoint{gp mark 7}{(4.240,1.635)}
\gppoint{gp mark 7}{(4.253,1.635)}
\gppoint{gp mark 7}{(4.265,1.635)}
\gppoint{gp mark 7}{(4.278,1.635)}
\gppoint{gp mark 7}{(4.291,1.635)}
\gppoint{gp mark 7}{(4.303,1.635)}
\gppoint{gp mark 7}{(4.316,1.635)}
\gppoint{gp mark 7}{(4.329,1.634)}
\gppoint{gp mark 7}{(4.342,1.634)}
\gppoint{gp mark 7}{(4.354,1.634)}
\gppoint{gp mark 7}{(4.367,1.635)}
\gppoint{gp mark 7}{(4.380,1.635)}
\gppoint{gp mark 7}{(4.392,1.635)}
\gppoint{gp mark 7}{(4.405,1.635)}
\gppoint{gp mark 7}{(4.418,1.635)}
\gppoint{gp mark 7}{(4.430,1.635)}
\gppoint{gp mark 7}{(4.443,1.635)}
\gppoint{gp mark 7}{(4.456,1.635)}
\gppoint{gp mark 7}{(4.468,1.635)}
\gppoint{gp mark 7}{(4.481,1.635)}
\gppoint{gp mark 7}{(4.494,1.635)}
\gppoint{gp mark 7}{(4.506,1.635)}
\gppoint{gp mark 7}{(4.519,1.635)}
\gppoint{gp mark 7}{(4.532,1.635)}
\gppoint{gp mark 7}{(4.544,1.635)}
\gppoint{gp mark 7}{(4.557,1.635)}
\gppoint{gp mark 7}{(4.570,1.635)}
\gppoint{gp mark 7}{(4.582,1.635)}
\gppoint{gp mark 7}{(4.595,1.635)}
\gppoint{gp mark 7}{(4.608,1.635)}
\gppoint{gp mark 7}{(4.620,1.635)}
\gppoint{gp mark 7}{(4.633,1.635)}
\gppoint{gp mark 7}{(4.646,1.635)}
\gppoint{gp mark 7}{(4.658,1.635)}
\gppoint{gp mark 7}{(4.671,1.635)}
\gppoint{gp mark 7}{(4.684,1.635)}
\gppoint{gp mark 7}{(4.697,1.635)}
\gppoint{gp mark 7}{(4.709,1.635)}
\gppoint{gp mark 7}{(4.722,1.635)}
\gppoint{gp mark 7}{(4.735,1.635)}
\gppoint{gp mark 7}{(4.747,1.635)}
\gppoint{gp mark 7}{(4.760,1.635)}
\gppoint{gp mark 7}{(4.773,1.635)}
\gppoint{gp mark 7}{(4.785,1.635)}
\gppoint{gp mark 7}{(4.798,1.635)}
\gppoint{gp mark 7}{(4.811,1.635)}
\gppoint{gp mark 7}{(4.823,1.635)}
\gppoint{gp mark 7}{(4.836,1.635)}
\gppoint{gp mark 7}{(4.849,1.635)}
\gppoint{gp mark 7}{(4.861,1.635)}
\gppoint{gp mark 7}{(4.874,1.635)}
\gppoint{gp mark 7}{(4.887,1.635)}
\gppoint{gp mark 7}{(4.899,1.635)}
\gppoint{gp mark 7}{(4.912,1.635)}
\gppoint{gp mark 7}{(4.925,1.635)}
\gppoint{gp mark 7}{(4.937,1.635)}
\gppoint{gp mark 7}{(4.950,1.635)}
\gppoint{gp mark 7}{(4.963,1.635)}
\gppoint{gp mark 7}{(4.975,1.635)}
\gppoint{gp mark 7}{(4.988,1.635)}
\gppoint{gp mark 7}{(5.001,1.635)}
\gppoint{gp mark 7}{(5.013,1.635)}
\gppoint{gp mark 7}{(5.026,1.635)}
\gppoint{gp mark 7}{(5.039,1.635)}
\gppoint{gp mark 7}{(5.052,1.635)}
\gppoint{gp mark 7}{(5.064,1.635)}
\gppoint{gp mark 7}{(5.077,1.635)}
\gppoint{gp mark 7}{(5.090,1.635)}
\gppoint{gp mark 7}{(5.102,1.635)}
\gppoint{gp mark 7}{(5.115,1.635)}
\gppoint{gp mark 7}{(5.128,1.635)}
\gppoint{gp mark 7}{(5.140,1.635)}
\gppoint{gp mark 7}{(5.153,1.635)}
\gppoint{gp mark 7}{(5.166,1.635)}
\gppoint{gp mark 7}{(5.178,1.635)}
\gppoint{gp mark 7}{(5.191,1.635)}
\gppoint{gp mark 7}{(5.204,1.635)}
\gppoint{gp mark 7}{(5.216,1.635)}
\gppoint{gp mark 7}{(5.229,1.635)}
\gppoint{gp mark 7}{(5.242,1.635)}
\gppoint{gp mark 7}{(5.254,1.635)}
\gppoint{gp mark 7}{(5.267,1.635)}
\gppoint{gp mark 7}{(5.280,1.635)}
\gppoint{gp mark 7}{(5.292,1.635)}
\gppoint{gp mark 7}{(5.305,1.635)}
\gppoint{gp mark 7}{(5.318,1.635)}
\gppoint{gp mark 7}{(5.330,1.635)}
\gppoint{gp mark 7}{(5.343,1.635)}
\gppoint{gp mark 7}{(5.356,1.635)}
\gppoint{gp mark 7}{(5.368,1.635)}
\gppoint{gp mark 7}{(5.381,1.635)}
\gppoint{gp mark 7}{(5.394,1.635)}
\gppoint{gp mark 7}{(5.407,1.635)}
\gppoint{gp mark 7}{(5.419,1.635)}
\gppoint{gp mark 7}{(5.432,1.635)}
\gppoint{gp mark 7}{(5.445,1.635)}
\gppoint{gp mark 7}{(5.457,1.635)}
\gppoint{gp mark 7}{(5.470,1.635)}
\gppoint{gp mark 7}{(5.483,1.635)}
\gppoint{gp mark 7}{(5.495,1.635)}
\gppoint{gp mark 7}{(5.508,1.635)}
\gppoint{gp mark 7}{(5.521,1.635)}
\gppoint{gp mark 7}{(5.533,1.635)}
\gppoint{gp mark 7}{(5.546,1.635)}
\gppoint{gp mark 7}{(5.559,1.635)}
\gppoint{gp mark 7}{(5.571,1.635)}
\gppoint{gp mark 7}{(5.584,1.635)}
\gppoint{gp mark 7}{(5.597,1.635)}
\gppoint{gp mark 7}{(5.609,1.635)}
\gppoint{gp mark 7}{(5.622,1.635)}
\gppoint{gp mark 7}{(5.635,1.635)}
\gppoint{gp mark 7}{(5.647,1.635)}
\gppoint{gp mark 7}{(5.660,1.635)}
\gppoint{gp mark 7}{(5.673,1.635)}
\gppoint{gp mark 7}{(5.685,1.635)}
\gppoint{gp mark 7}{(5.698,1.635)}
\gppoint{gp mark 7}{(5.711,1.635)}
\gppoint{gp mark 7}{(5.723,1.634)}
\gppoint{gp mark 7}{(5.736,1.634)}
\gppoint{gp mark 7}{(5.749,1.634)}
\gppoint{gp mark 7}{(5.761,1.635)}
\gppoint{gp mark 7}{(5.774,1.635)}
\gppoint{gp mark 7}{(5.787,1.635)}
\gppoint{gp mark 7}{(5.800,1.635)}
\gppoint{gp mark 7}{(5.812,1.635)}
\gppoint{gp mark 7}{(5.825,1.635)}
\gppoint{gp mark 7}{(5.838,1.635)}
\gppoint{gp mark 7}{(5.850,1.635)}
\gppoint{gp mark 7}{(5.863,1.635)}
\gppoint{gp mark 7}{(5.876,1.635)}
\gppoint{gp mark 7}{(5.888,1.635)}
\gppoint{gp mark 7}{(5.901,1.635)}
\gppoint{gp mark 7}{(5.914,1.635)}
\gppoint{gp mark 7}{(5.926,1.635)}
\gppoint{gp mark 7}{(5.939,1.635)}
\gppoint{gp mark 7}{(5.952,1.635)}
\gppoint{gp mark 7}{(5.964,1.635)}
\gppoint{gp mark 7}{(5.977,1.635)}
\gppoint{gp mark 7}{(5.990,1.635)}
\gppoint{gp mark 7}{(6.002,1.635)}
\gppoint{gp mark 7}{(6.015,1.635)}
\gppoint{gp mark 7}{(6.028,1.635)}
\gppoint{gp mark 7}{(6.040,1.635)}
\gppoint{gp mark 7}{(6.053,1.635)}
\gppoint{gp mark 7}{(6.066,1.635)}
\gppoint{gp mark 7}{(6.078,1.635)}
\gppoint{gp mark 7}{(6.091,1.635)}
\gppoint{gp mark 7}{(6.104,1.635)}
\gppoint{gp mark 7}{(6.116,1.635)}
\gppoint{gp mark 7}{(6.129,1.635)}
\gppoint{gp mark 7}{(6.142,1.635)}
\gppoint{gp mark 7}{(6.155,1.635)}
\gppoint{gp mark 7}{(6.167,1.635)}
\gppoint{gp mark 7}{(6.180,1.635)}
\gppoint{gp mark 7}{(6.193,1.635)}
\gppoint{gp mark 7}{(6.205,1.635)}
\gppoint{gp mark 7}{(6.218,1.635)}
\gppoint{gp mark 7}{(6.231,1.635)}
\gppoint{gp mark 7}{(6.243,1.635)}
\gppoint{gp mark 7}{(6.256,1.635)}
\gppoint{gp mark 7}{(6.269,1.635)}
\gppoint{gp mark 7}{(6.281,1.635)}
\gppoint{gp mark 7}{(6.294,1.635)}
\gppoint{gp mark 7}{(6.307,1.635)}
\gppoint{gp mark 7}{(6.319,1.635)}
\gppoint{gp mark 7}{(6.332,1.635)}
\gppoint{gp mark 7}{(6.345,1.635)}
\gppoint{gp mark 7}{(6.357,1.635)}
\gppoint{gp mark 7}{(6.370,1.635)}
\gppoint{gp mark 7}{(6.383,1.635)}
\gppoint{gp mark 7}{(6.395,1.635)}
\gppoint{gp mark 7}{(6.408,1.635)}
\gppoint{gp mark 7}{(6.421,1.635)}
\gppoint{gp mark 7}{(6.433,1.635)}
\gppoint{gp mark 7}{(6.446,1.635)}
\gppoint{gp mark 7}{(6.459,1.635)}
\gppoint{gp mark 7}{(6.471,1.635)}
\gppoint{gp mark 7}{(6.484,1.635)}
\gppoint{gp mark 7}{(6.497,1.635)}
\gppoint{gp mark 7}{(6.510,1.635)}
\gppoint{gp mark 7}{(6.522,1.635)}
\gppoint{gp mark 7}{(6.535,1.635)}
\gppoint{gp mark 7}{(6.548,1.635)}
\gppoint{gp mark 7}{(6.560,1.635)}
\gppoint{gp mark 7}{(6.573,1.635)}
\gppoint{gp mark 7}{(6.586,1.635)}
\gppoint{gp mark 7}{(6.598,1.635)}
\gppoint{gp mark 7}{(6.611,1.635)}
\gppoint{gp mark 7}{(6.624,1.635)}
\gppoint{gp mark 7}{(6.636,1.635)}
\gppoint{gp mark 7}{(6.649,1.635)}
\gppoint{gp mark 7}{(6.662,1.635)}
\gppoint{gp mark 7}{(6.674,1.635)}
\gppoint{gp mark 7}{(6.687,1.635)}
\gppoint{gp mark 7}{(6.700,1.635)}
\gppoint{gp mark 7}{(6.712,1.635)}
\gppoint{gp mark 7}{(6.725,1.635)}
\gppoint{gp mark 7}{(6.738,1.635)}
\gppoint{gp mark 7}{(6.750,1.635)}
\gppoint{gp mark 7}{(6.763,1.635)}
\gppoint{gp mark 7}{(6.776,1.635)}
\gppoint{gp mark 7}{(6.788,1.635)}
\gppoint{gp mark 7}{(6.801,1.635)}
\gppoint{gp mark 7}{(6.814,1.635)}
\gppoint{gp mark 7}{(6.826,1.635)}
\gppoint{gp mark 7}{(6.839,1.635)}
\gppoint{gp mark 7}{(6.852,1.635)}
\gppoint{gp mark 7}{(6.865,1.635)}
\gppoint{gp mark 7}{(6.877,1.635)}
\gppoint{gp mark 7}{(6.890,1.635)}
\gppoint{gp mark 7}{(6.903,1.635)}
\gppoint{gp mark 7}{(6.915,1.635)}
\gppoint{gp mark 7}{(6.928,1.635)}
\gppoint{gp mark 7}{(6.941,1.635)}
\gppoint{gp mark 7}{(6.953,1.635)}
\gppoint{gp mark 7}{(6.966,1.635)}
\gppoint{gp mark 7}{(6.979,1.635)}
\gppoint{gp mark 7}{(6.991,1.635)}
\gppoint{gp mark 7}{(7.004,1.635)}
\gppoint{gp mark 7}{(7.017,1.635)}
\gppoint{gp mark 7}{(7.029,1.635)}
\gppoint{gp mark 7}{(7.042,1.635)}
\gppoint{gp mark 7}{(7.055,1.635)}
\gppoint{gp mark 7}{(7.067,1.635)}
\gppoint{gp mark 7}{(7.080,1.635)}
\gppoint{gp mark 7}{(7.093,1.635)}
\gppoint{gp mark 7}{(7.105,1.635)}
\gppoint{gp mark 7}{(7.118,1.635)}
\gppoint{gp mark 7}{(7.131,1.635)}
\gppoint{gp mark 7}{(7.143,1.635)}
\gppoint{gp mark 7}{(7.156,1.635)}
\gppoint{gp mark 7}{(7.169,1.635)}
\gppoint{gp mark 7}{(7.181,1.635)}
\gppoint{gp mark 7}{(7.194,1.635)}
\gppoint{gp mark 7}{(7.207,1.635)}
\gppoint{gp mark 7}{(7.220,1.635)}
\gppoint{gp mark 7}{(7.232,1.635)}
\gppoint{gp mark 7}{(7.245,1.635)}
\gppoint{gp mark 7}{(7.258,1.635)}
\gppoint{gp mark 7}{(7.270,1.635)}
\gppoint{gp mark 7}{(7.283,1.635)}
\gppoint{gp mark 7}{(7.296,1.635)}
\gppoint{gp mark 7}{(7.308,1.635)}
\gppoint{gp mark 7}{(7.321,1.635)}
\gppoint{gp mark 7}{(7.334,1.635)}
\gppoint{gp mark 7}{(7.346,1.635)}
\gppoint{gp mark 7}{(7.359,1.635)}
\gppoint{gp mark 7}{(7.372,1.635)}
\gppoint{gp mark 7}{(7.384,1.635)}
\gppoint{gp mark 7}{(7.397,1.635)}
\gppoint{gp mark 7}{(7.410,1.635)}
\gppoint{gp mark 7}{(7.422,1.635)}
\gppoint{gp mark 7}{(7.435,1.635)}
\gppoint{gp mark 7}{(7.448,1.635)}
\gppoint{gp mark 7}{(7.460,1.635)}
\gppoint{gp mark 7}{(7.473,1.635)}
\gppoint{gp mark 7}{(7.486,1.635)}
\gppoint{gp mark 7}{(7.498,1.635)}
\gppoint{gp mark 7}{(7.511,1.635)}
\gppoint{gp mark 7}{(7.524,1.635)}
\gppoint{gp mark 7}{(7.536,1.635)}
\gppoint{gp mark 7}{(7.549,1.635)}
\gppoint{gp mark 7}{(7.562,1.635)}
\gppoint{gp mark 7}{(7.575,1.635)}
\gppoint{gp mark 7}{(7.587,1.635)}
\gppoint{gp mark 7}{(7.600,1.635)}
\gppoint{gp mark 7}{(7.613,1.635)}
\gppoint{gp mark 7}{(7.625,1.635)}
\gppoint{gp mark 7}{(7.638,1.635)}
\gppoint{gp mark 7}{(7.651,1.635)}
\gppoint{gp mark 7}{(7.663,1.635)}
\gppoint{gp mark 7}{(7.676,1.635)}
\gppoint{gp mark 7}{(7.689,1.635)}
\gppoint{gp mark 7}{(7.701,1.635)}
\gppoint{gp mark 7}{(7.714,1.635)}
\gppoint{gp mark 7}{(7.727,1.635)}
\gppoint{gp mark 7}{(7.739,1.635)}
\gppoint{gp mark 7}{(7.752,1.635)}
\gppoint{gp mark 7}{(7.765,1.635)}
\gppoint{gp mark 7}{(7.777,1.635)}
\gppoint{gp mark 7}{(7.790,1.635)}
\gppoint{gp mark 7}{(7.803,1.635)}
\gppoint{gp mark 7}{(7.815,1.635)}
\gppoint{gp mark 7}{(7.828,1.635)}
\gppoint{gp mark 7}{(7.841,1.635)}
\gppoint{gp mark 7}{(7.853,1.635)}
\gppoint{gp mark 7}{(7.866,1.635)}
\gppoint{gp mark 7}{(7.879,1.635)}
\gppoint{gp mark 7}{(7.891,1.635)}
\gppoint{gp mark 7}{(7.904,1.635)}
\gppoint{gp mark 7}{(7.917,1.635)}
\gppoint{gp mark 7}{(7.930,1.635)}
\gpcolor{rgb color={0.000,0.000,0.000}}
\gpsetlinetype{gp lt plot 0}
\gpsetlinewidth{4.00}
\draw[gp path] (2.404,3.934)--(2.886,3.934);
\draw[gp path] (2.886,1.228)--(3.811,1.228);
\draw[gp path] (3.811,1.635)--(7.722,1.635);
\draw[gp path] (7.722,1.635)--(7.947,1.635);
\draw[gp path] (1.207,4.836)--(1.220,4.826)--(1.233,4.816)--(1.246,4.807)--(1.259,4.797)%
  --(1.271,4.787)--(1.284,4.777)--(1.297,4.768)--(1.310,4.758)--(1.323,4.748)--(1.336,4.739)%
  --(1.349,4.729)--(1.361,4.719)--(1.374,4.710)--(1.387,4.700)--(1.400,4.690)--(1.413,4.681)%
  --(1.426,4.671)--(1.439,4.661)--(1.452,4.652)--(1.464,4.642)--(1.477,4.632)--(1.490,4.622)%
  --(1.503,4.613)--(1.516,4.603)--(1.529,4.593)--(1.542,4.584)--(1.555,4.574)--(1.567,4.564)%
  --(1.580,4.555)--(1.593,4.545)--(1.606,4.535)--(1.619,4.526)--(1.632,4.516)--(1.645,4.506)%
  --(1.657,4.496)--(1.670,4.487)--(1.683,4.477)--(1.696,4.467)--(1.709,4.458)--(1.722,4.448)%
  --(1.735,4.438)--(1.748,4.429)--(1.760,4.419)--(1.773,4.409)--(1.786,4.400)--(1.799,4.390)%
  --(1.812,4.380)--(1.825,4.371)--(1.838,4.361)--(1.850,4.351)--(1.863,4.341)--(1.876,4.332)%
  --(1.889,4.322)--(1.902,4.312)--(1.915,4.303)--(1.928,4.293)--(1.941,4.283)--(1.953,4.274)%
  --(1.966,4.264)--(1.979,4.254)--(1.992,4.245)--(2.005,4.235)--(2.018,4.225)--(2.031,4.215)%
  --(2.044,4.206)--(2.056,4.196)--(2.069,4.186)--(2.082,4.177)--(2.095,4.167)--(2.108,4.157)%
  --(2.121,4.148)--(2.134,4.138)--(2.146,4.128)--(2.159,4.119)--(2.172,4.109)--(2.185,4.099)%
  --(2.198,4.090)--(2.211,4.080)--(2.224,4.070)--(2.237,4.060)--(2.249,4.051)--(2.262,4.041)%
  --(2.275,4.031)--(2.288,4.022)--(2.301,4.012)--(2.314,4.002)--(2.327,3.993)--(2.340,3.983)%
  --(2.352,3.973)--(2.365,3.964)--(2.378,3.954)--(2.391,3.944)--(2.404,3.934);
\draw[gp path] (2.886,3.934)--(2.886,1.228);
\draw[gp path] (3.811,1.228)--(3.811,1.635);
\draw[gp path] (3.793,3.695)--(4.572,3.695);
\gpcolor{rgb color={1.000,0.000,0.000}}
\gpsetlinewidth{0.50}
\gppoint{gp mark 7}{(4.182,2.921)}
\gpcolor{rgb color={0.502,0.502,0.502}}
\gppoint{gp mark 7}{(4.182,2.147)}
\gpcolor{rgb color={0.000,0.000,0.000}}
\node[gp node left,font={\fontsize{10pt}{12pt}\selectfont}] at (1.456,5.166) {\LARGE $B_y$};
\node[gp node left,font={\fontsize{10pt}{12pt}\selectfont}] at (5.740,5.166) {\large $\alpha = 3.0$};
\node[gp node left,font={\fontsize{10pt}{12pt}\selectfont}] at (4.831,3.695) {\large exact};
\node[gp node left,font={\fontsize{10pt}{12pt}\selectfont}] at (4.831,2.921) {\large HLLD-CWM};
\node[gp node left,font={\fontsize{10pt}{12pt}\selectfont}] at (4.831,2.147) {\large HLLD};
%% coordinates of the plot area
\gpdefrectangularnode{gp plot 1}{\pgfpoint{1.196cm}{0.985cm}}{\pgfpoint{7.947cm}{5.631cm}}
\end{tikzpicture}
%% gnuplot variables
} \\
\resizebox{0.5\linewidth}{!}{\tikzsetnextfilename{fast_coplanar_a_crsol_1}\begin{tikzpicture}[gnuplot]
%% generated with GNUPLOT 4.6p4 (Lua 5.1; terminal rev. 99, script rev. 100)
%% Sat 02 Aug 2014 10:08:42 AM EDT
\path (0.000,0.000) rectangle (8.500,6.000);
\gpfill{rgb color={1.000,1.000,1.000}} (1.196,0.985)--(7.946,0.985)--(7.946,5.630)--(1.196,5.630)--cycle;
\gpcolor{color=gp lt color border}
\gpsetlinetype{gp lt border}
\gpsetlinewidth{1.00}
\draw[gp path] (1.196,0.985)--(1.196,5.630)--(7.946,5.630)--(7.946,0.985)--cycle;
\gpcolor{color=gp lt color axes}
\gpsetlinetype{gp lt axes}
\gpsetlinewidth{2.00}
\draw[gp path] (1.196,1.275)--(7.947,1.275);
\gpcolor{color=gp lt color border}
\gpsetlinetype{gp lt border}
\draw[gp path] (1.196,1.275)--(1.268,1.275);
\draw[gp path] (7.947,1.275)--(7.875,1.275);
\gpcolor{rgb color={0.000,0.000,0.000}}
\node[gp node right,font={\fontsize{10pt}{12pt}\selectfont}] at (1.012,1.275) {0.65};
\gpcolor{color=gp lt color axes}
\gpsetlinetype{gp lt axes}
\draw[gp path] (1.196,2.001)--(7.947,2.001);
\gpcolor{color=gp lt color border}
\gpsetlinetype{gp lt border}
\draw[gp path] (1.196,2.001)--(1.268,2.001);
\draw[gp path] (7.947,2.001)--(7.875,2.001);
\gpcolor{rgb color={0.000,0.000,0.000}}
\node[gp node right,font={\fontsize{10pt}{12pt}\selectfont}] at (1.012,2.001) {0.7};
\gpcolor{color=gp lt color axes}
\gpsetlinetype{gp lt axes}
\draw[gp path] (1.196,2.727)--(7.947,2.727);
\gpcolor{color=gp lt color border}
\gpsetlinetype{gp lt border}
\draw[gp path] (1.196,2.727)--(1.268,2.727);
\draw[gp path] (7.947,2.727)--(7.875,2.727);
\gpcolor{rgb color={0.000,0.000,0.000}}
\node[gp node right,font={\fontsize{10pt}{12pt}\selectfont}] at (1.012,2.727) {0.75};
\gpcolor{color=gp lt color axes}
\gpsetlinetype{gp lt axes}
\draw[gp path] (1.196,3.453)--(7.947,3.453);
\gpcolor{color=gp lt color border}
\gpsetlinetype{gp lt border}
\draw[gp path] (1.196,3.453)--(1.268,3.453);
\draw[gp path] (7.947,3.453)--(7.875,3.453);
\gpcolor{rgb color={0.000,0.000,0.000}}
\node[gp node right,font={\fontsize{10pt}{12pt}\selectfont}] at (1.012,3.453) {0.8};
\gpcolor{color=gp lt color axes}
\gpsetlinetype{gp lt axes}
\draw[gp path] (1.196,4.179)--(7.947,4.179);
\gpcolor{color=gp lt color border}
\gpsetlinetype{gp lt border}
\draw[gp path] (1.196,4.179)--(1.268,4.179);
\draw[gp path] (7.947,4.179)--(7.875,4.179);
\gpcolor{rgb color={0.000,0.000,0.000}}
\node[gp node right,font={\fontsize{10pt}{12pt}\selectfont}] at (1.012,4.179) {0.85};
\gpcolor{color=gp lt color axes}
\gpsetlinetype{gp lt axes}
\draw[gp path] (1.196,4.905)--(7.947,4.905);
\gpcolor{color=gp lt color border}
\gpsetlinetype{gp lt border}
\draw[gp path] (1.196,4.905)--(1.268,4.905);
\draw[gp path] (7.947,4.905)--(7.875,4.905);
\gpcolor{rgb color={0.000,0.000,0.000}}
\node[gp node right,font={\fontsize{10pt}{12pt}\selectfont}] at (1.012,4.905) {0.9};
\gpcolor{color=gp lt color axes}
\gpsetlinetype{gp lt axes}
\draw[gp path] (1.196,5.631)--(7.947,5.631);
\gpcolor{color=gp lt color border}
\gpsetlinetype{gp lt border}
\draw[gp path] (1.196,5.631)--(1.268,5.631);
\draw[gp path] (7.947,5.631)--(7.875,5.631);
\gpcolor{rgb color={0.000,0.000,0.000}}
\node[gp node right,font={\fontsize{10pt}{12pt}\selectfont}] at (1.012,5.631) {0.95};
\gpcolor{color=gp lt color axes}
\gpsetlinetype{gp lt axes}
\draw[gp path] (1.196,0.985)--(1.196,5.631);
\gpcolor{color=gp lt color border}
\gpsetlinetype{gp lt border}
\draw[gp path] (1.196,0.985)--(1.196,1.057);
\draw[gp path] (1.196,5.631)--(1.196,5.559);
\gpcolor{rgb color={0.000,0.000,0.000}}
\node[gp node center,font={\fontsize{10pt}{12pt}\selectfont}] at (1.196,0.677) {0.3};
\gpcolor{color=gp lt color axes}
\gpsetlinetype{gp lt axes}
\draw[gp path] (2.494,0.985)--(2.494,5.631);
\gpcolor{color=gp lt color border}
\gpsetlinetype{gp lt border}
\draw[gp path] (2.494,0.985)--(2.494,1.057);
\draw[gp path] (2.494,5.631)--(2.494,5.559);
\gpcolor{rgb color={0.000,0.000,0.000}}
\node[gp node center,font={\fontsize{10pt}{12pt}\selectfont}] at (2.494,0.677) {0.35};
\gpcolor{color=gp lt color axes}
\gpsetlinetype{gp lt axes}
\draw[gp path] (3.793,0.985)--(3.793,5.631);
\gpcolor{color=gp lt color border}
\gpsetlinetype{gp lt border}
\draw[gp path] (3.793,0.985)--(3.793,1.057);
\draw[gp path] (3.793,5.631)--(3.793,5.559);
\gpcolor{rgb color={0.000,0.000,0.000}}
\node[gp node center,font={\fontsize{10pt}{12pt}\selectfont}] at (3.793,0.677) {0.4};
\gpcolor{color=gp lt color axes}
\gpsetlinetype{gp lt axes}
\draw[gp path] (5.091,0.985)--(5.091,5.631);
\gpcolor{color=gp lt color border}
\gpsetlinetype{gp lt border}
\draw[gp path] (5.091,0.985)--(5.091,1.057);
\draw[gp path] (5.091,5.631)--(5.091,5.559);
\gpcolor{rgb color={0.000,0.000,0.000}}
\node[gp node center,font={\fontsize{10pt}{12pt}\selectfont}] at (5.091,0.677) {0.45};
\gpcolor{color=gp lt color axes}
\gpsetlinetype{gp lt axes}
\draw[gp path] (6.389,0.985)--(6.389,5.631);
\gpcolor{color=gp lt color border}
\gpsetlinetype{gp lt border}
\draw[gp path] (6.389,0.985)--(6.389,1.057);
\draw[gp path] (6.389,5.631)--(6.389,5.559);
\gpcolor{rgb color={0.000,0.000,0.000}}
\node[gp node center,font={\fontsize{10pt}{12pt}\selectfont}] at (6.389,0.677) {0.5};
\gpcolor{color=gp lt color axes}
\gpsetlinetype{gp lt axes}
\draw[gp path] (7.687,0.985)--(7.687,5.631);
\gpcolor{color=gp lt color border}
\gpsetlinetype{gp lt border}
\draw[gp path] (7.687,0.985)--(7.687,1.057);
\draw[gp path] (7.687,5.631)--(7.687,5.559);
\gpcolor{rgb color={0.000,0.000,0.000}}
\node[gp node center,font={\fontsize{10pt}{12pt}\selectfont}] at (7.687,0.677) {0.55};
\gpcolor{color=gp lt color border}
\draw[gp path] (1.196,5.631)--(1.196,0.985)--(7.947,0.985)--(7.947,5.631)--cycle;
\gpcolor{rgb color={0.000,0.000,0.000}}
\node[gp node center,font={\fontsize{10pt}{12pt}\selectfont}] at (4.571,0.215) {\large $x$};
\gpcolor{rgb color={0.502,0.502,0.502}}
\gpsetlinewidth{0.50}
\gpsetpointsize{2.67}
\gppoint{gp mark 7}{(1.210,3.636)}
\gppoint{gp mark 7}{(1.223,3.621)}
\gppoint{gp mark 7}{(1.235,3.605)}
\gppoint{gp mark 7}{(1.248,3.590)}
\gppoint{gp mark 7}{(1.261,3.574)}
\gppoint{gp mark 7}{(1.273,3.559)}
\gppoint{gp mark 7}{(1.286,3.543)}
\gppoint{gp mark 7}{(1.299,3.528)}
\gppoint{gp mark 7}{(1.311,3.513)}
\gppoint{gp mark 7}{(1.324,3.497)}
\gppoint{gp mark 7}{(1.337,3.482)}
\gppoint{gp mark 7}{(1.349,3.466)}
\gppoint{gp mark 7}{(1.362,3.451)}
\gppoint{gp mark 7}{(1.375,3.436)}
\gppoint{gp mark 7}{(1.387,3.420)}
\gppoint{gp mark 7}{(1.400,3.405)}
\gppoint{gp mark 7}{(1.413,3.390)}
\gppoint{gp mark 7}{(1.425,3.374)}
\gppoint{gp mark 7}{(1.438,3.359)}
\gppoint{gp mark 7}{(1.451,3.344)}
\gppoint{gp mark 7}{(1.464,3.329)}
\gppoint{gp mark 7}{(1.476,3.313)}
\gppoint{gp mark 7}{(1.489,3.298)}
\gppoint{gp mark 7}{(1.502,3.283)}
\gppoint{gp mark 7}{(1.514,3.268)}
\gppoint{gp mark 7}{(1.527,3.252)}
\gppoint{gp mark 7}{(1.540,3.237)}
\gppoint{gp mark 7}{(1.552,3.222)}
\gppoint{gp mark 7}{(1.565,3.207)}
\gppoint{gp mark 7}{(1.578,3.192)}
\gppoint{gp mark 7}{(1.590,3.177)}
\gppoint{gp mark 7}{(1.603,3.162)}
\gppoint{gp mark 7}{(1.616,3.146)}
\gppoint{gp mark 7}{(1.628,3.131)}
\gppoint{gp mark 7}{(1.641,3.116)}
\gppoint{gp mark 7}{(1.654,3.101)}
\gppoint{gp mark 7}{(1.666,3.086)}
\gppoint{gp mark 7}{(1.679,3.071)}
\gppoint{gp mark 7}{(1.692,3.056)}
\gppoint{gp mark 7}{(1.704,3.041)}
\gppoint{gp mark 7}{(1.717,3.026)}
\gppoint{gp mark 7}{(1.730,3.011)}
\gppoint{gp mark 7}{(1.742,2.996)}
\gppoint{gp mark 7}{(1.755,2.981)}
\gppoint{gp mark 7}{(1.768,2.966)}
\gppoint{gp mark 7}{(1.780,2.951)}
\gppoint{gp mark 7}{(1.793,2.936)}
\gppoint{gp mark 7}{(1.806,2.921)}
\gppoint{gp mark 7}{(1.819,2.906)}
\gppoint{gp mark 7}{(1.831,2.891)}
\gppoint{gp mark 7}{(1.844,2.876)}
\gppoint{gp mark 7}{(1.857,2.861)}
\gppoint{gp mark 7}{(1.869,2.847)}
\gppoint{gp mark 7}{(1.882,2.832)}
\gppoint{gp mark 7}{(1.895,2.817)}
\gppoint{gp mark 7}{(1.907,2.802)}
\gppoint{gp mark 7}{(1.920,2.787)}
\gppoint{gp mark 7}{(1.933,2.772)}
\gppoint{gp mark 7}{(1.945,2.758)}
\gppoint{gp mark 7}{(1.958,2.743)}
\gppoint{gp mark 7}{(1.971,2.728)}
\gppoint{gp mark 7}{(1.983,2.713)}
\gppoint{gp mark 7}{(1.996,2.698)}
\gppoint{gp mark 7}{(2.009,2.684)}
\gppoint{gp mark 7}{(2.021,2.669)}
\gppoint{gp mark 7}{(2.034,2.654)}
\gppoint{gp mark 7}{(2.047,2.640)}
\gppoint{gp mark 7}{(2.059,2.625)}
\gppoint{gp mark 7}{(2.072,2.610)}
\gppoint{gp mark 7}{(2.085,2.595)}
\gppoint{gp mark 7}{(2.097,2.581)}
\gppoint{gp mark 7}{(2.110,2.566)}
\gppoint{gp mark 7}{(2.123,2.551)}
\gppoint{gp mark 7}{(2.135,2.537)}
\gppoint{gp mark 7}{(2.148,2.522)}
\gppoint{gp mark 7}{(2.161,2.508)}
\gppoint{gp mark 7}{(2.174,2.493)}
\gppoint{gp mark 7}{(2.186,2.478)}
\gppoint{gp mark 7}{(2.199,2.464)}
\gppoint{gp mark 7}{(2.212,2.449)}
\gppoint{gp mark 7}{(2.224,2.435)}
\gppoint{gp mark 7}{(2.237,2.420)}
\gppoint{gp mark 7}{(2.250,2.406)}
\gppoint{gp mark 7}{(2.262,2.391)}
\gppoint{gp mark 7}{(2.275,2.377)}
\gppoint{gp mark 7}{(2.288,2.362)}
\gppoint{gp mark 7}{(2.300,2.347)}
\gppoint{gp mark 7}{(2.313,2.333)}
\gppoint{gp mark 7}{(2.326,2.319)}
\gppoint{gp mark 7}{(2.338,2.304)}
\gppoint{gp mark 7}{(2.351,2.290)}
\gppoint{gp mark 7}{(2.364,2.275)}
\gppoint{gp mark 7}{(2.376,2.261)}
\gppoint{gp mark 7}{(2.389,2.246)}
\gppoint{gp mark 7}{(2.402,2.232)}
\gppoint{gp mark 7}{(2.414,2.217)}
\gppoint{gp mark 7}{(2.427,2.203)}
\gppoint{gp mark 7}{(2.440,2.189)}
\gppoint{gp mark 7}{(2.452,2.174)}
\gppoint{gp mark 7}{(2.465,2.160)}
\gppoint{gp mark 7}{(2.478,2.146)}
\gppoint{gp mark 7}{(2.490,2.131)}
\gppoint{gp mark 7}{(2.503,2.117)}
\gppoint{gp mark 7}{(2.516,2.103)}
\gppoint{gp mark 7}{(2.529,2.088)}
\gppoint{gp mark 7}{(2.541,2.074)}
\gppoint{gp mark 7}{(2.554,2.060)}
\gppoint{gp mark 7}{(2.567,2.046)}
\gppoint{gp mark 7}{(2.579,2.031)}
\gppoint{gp mark 7}{(2.592,2.017)}
\gppoint{gp mark 7}{(2.605,2.003)}
\gppoint{gp mark 7}{(2.617,1.989)}
\gppoint{gp mark 7}{(2.630,1.975)}
\gppoint{gp mark 7}{(2.643,1.961)}
\gppoint{gp mark 7}{(2.655,1.947)}
\gppoint{gp mark 7}{(2.668,1.933)}
\gppoint{gp mark 7}{(2.681,1.919)}
\gppoint{gp mark 7}{(2.693,1.905)}
\gppoint{gp mark 7}{(2.706,1.891)}
\gppoint{gp mark 7}{(2.719,1.877)}
\gppoint{gp mark 7}{(2.731,1.863)}
\gppoint{gp mark 7}{(2.744,1.850)}
\gppoint{gp mark 7}{(2.757,1.837)}
\gppoint{gp mark 7}{(2.769,1.823)}
\gppoint{gp mark 7}{(2.782,1.810)}
\gppoint{gp mark 7}{(2.795,1.798)}
\gppoint{gp mark 7}{(2.807,1.785)}
\gppoint{gp mark 7}{(2.820,1.774)}
\gppoint{gp mark 7}{(2.833,1.763)}
\gppoint{gp mark 7}{(2.845,1.753)}
\gppoint{gp mark 7}{(2.858,1.745)}
\gppoint{gp mark 7}{(2.871,1.737)}
\gppoint{gp mark 7}{(2.883,1.730)}
\gppoint{gp mark 7}{(2.896,1.721)}
\gppoint{gp mark 7}{(2.909,1.696)}
\gppoint{gp mark 7}{(2.922,1.609)}
\gppoint{gp mark 7}{(2.934,1.524)}
\gppoint{gp mark 7}{(2.947,1.745)}
\gppoint{gp mark 7}{(2.960,2.262)}
\gppoint{gp mark 7}{(2.972,2.340)}
\gppoint{gp mark 7}{(2.985,2.339)}
\gppoint{gp mark 7}{(2.998,2.339)}
\gppoint{gp mark 7}{(3.010,2.339)}
\gppoint{gp mark 7}{(3.023,2.340)}
\gppoint{gp mark 7}{(3.036,2.341)}
\gppoint{gp mark 7}{(3.048,2.341)}
\gppoint{gp mark 7}{(3.061,2.339)}
\gppoint{gp mark 7}{(3.074,2.337)}
\gppoint{gp mark 7}{(3.086,2.338)}
\gppoint{gp mark 7}{(3.099,2.339)}
\gppoint{gp mark 7}{(3.112,2.340)}
\gppoint{gp mark 7}{(3.124,2.340)}
\gppoint{gp mark 7}{(3.137,2.339)}
\gppoint{gp mark 7}{(3.150,2.339)}
\gppoint{gp mark 7}{(3.162,2.340)}
\gppoint{gp mark 7}{(3.175,2.342)}
\gppoint{gp mark 7}{(3.188,2.342)}
\gppoint{gp mark 7}{(3.200,2.341)}
\gppoint{gp mark 7}{(3.213,2.339)}
\gppoint{gp mark 7}{(3.226,2.339)}
\gppoint{gp mark 7}{(3.238,2.339)}
\gppoint{gp mark 7}{(3.251,2.339)}
\gppoint{gp mark 7}{(3.264,2.339)}
\gppoint{gp mark 7}{(3.277,2.338)}
\gppoint{gp mark 7}{(3.289,2.338)}
\gppoint{gp mark 7}{(3.302,2.338)}
\gppoint{gp mark 7}{(3.315,2.339)}
\gppoint{gp mark 7}{(3.327,2.340)}
\gppoint{gp mark 7}{(3.340,2.341)}
\gppoint{gp mark 7}{(3.353,2.341)}
\gppoint{gp mark 7}{(3.365,2.341)}
\gppoint{gp mark 7}{(3.378,2.341)}
\gppoint{gp mark 7}{(3.391,2.341)}
\gppoint{gp mark 7}{(3.403,2.341)}
\gppoint{gp mark 7}{(3.416,2.340)}
\gppoint{gp mark 7}{(3.429,2.339)}
\gppoint{gp mark 7}{(3.441,2.337)}
\gppoint{gp mark 7}{(3.454,2.337)}
\gppoint{gp mark 7}{(3.467,2.337)}
\gppoint{gp mark 7}{(3.479,2.337)}
\gppoint{gp mark 7}{(3.492,2.336)}
\gppoint{gp mark 7}{(3.505,2.336)}
\gppoint{gp mark 7}{(3.517,2.337)}
\gppoint{gp mark 7}{(3.530,2.338)}
\gppoint{gp mark 7}{(3.543,2.339)}
\gppoint{gp mark 7}{(3.555,2.340)}
\gppoint{gp mark 7}{(3.568,2.340)}
\gppoint{gp mark 7}{(3.581,2.340)}
\gppoint{gp mark 7}{(3.593,2.341)}
\gppoint{gp mark 7}{(3.606,2.341)}
\gppoint{gp mark 7}{(3.619,2.341)}
\gppoint{gp mark 7}{(3.632,2.340)}
\gppoint{gp mark 7}{(3.644,2.339)}
\gppoint{gp mark 7}{(3.657,2.339)}
\gppoint{gp mark 7}{(3.670,2.339)}
\gppoint{gp mark 7}{(3.682,2.340)}
\gppoint{gp mark 7}{(3.695,2.340)}
\gppoint{gp mark 7}{(3.708,2.341)}
\gppoint{gp mark 7}{(3.720,2.360)}
\gppoint{gp mark 7}{(3.733,2.597)}
\gppoint{gp mark 7}{(3.746,3.674)}
\gppoint{gp mark 7}{(3.758,4.580)}
\gppoint{gp mark 7}{(3.771,4.661)}
\gppoint{gp mark 7}{(3.784,4.658)}
\gppoint{gp mark 7}{(3.796,4.658)}
\gppoint{gp mark 7}{(3.809,4.657)}
\gppoint{gp mark 7}{(3.822,4.656)}
\gppoint{gp mark 7}{(3.834,4.657)}
\gppoint{gp mark 7}{(3.847,4.657)}
\gppoint{gp mark 7}{(3.860,4.657)}
\gppoint{gp mark 7}{(3.872,4.657)}
\gppoint{gp mark 7}{(3.885,4.657)}
\gppoint{gp mark 7}{(3.898,4.657)}
\gppoint{gp mark 7}{(3.910,4.656)}
\gppoint{gp mark 7}{(3.923,4.656)}
\gppoint{gp mark 7}{(3.936,4.656)}
\gppoint{gp mark 7}{(3.948,4.656)}
\gppoint{gp mark 7}{(3.961,4.656)}
\gppoint{gp mark 7}{(3.974,4.657)}
\gppoint{gp mark 7}{(3.987,4.657)}
\gppoint{gp mark 7}{(3.999,4.657)}
\gppoint{gp mark 7}{(4.012,4.658)}
\gppoint{gp mark 7}{(4.025,4.658)}
\gppoint{gp mark 7}{(4.037,4.657)}
\gppoint{gp mark 7}{(4.050,4.657)}
\gppoint{gp mark 7}{(4.063,4.656)}
\gppoint{gp mark 7}{(4.075,4.657)}
\gppoint{gp mark 7}{(4.088,4.657)}
\gppoint{gp mark 7}{(4.101,4.658)}
\gppoint{gp mark 7}{(4.113,4.658)}
\gppoint{gp mark 7}{(4.126,4.658)}
\gppoint{gp mark 7}{(4.139,4.658)}
\gppoint{gp mark 7}{(4.151,4.657)}
\gppoint{gp mark 7}{(4.164,4.657)}
\gppoint{gp mark 7}{(4.177,4.657)}
\gppoint{gp mark 7}{(4.189,4.656)}
\gppoint{gp mark 7}{(4.202,4.656)}
\gppoint{gp mark 7}{(4.215,4.657)}
\gppoint{gp mark 7}{(4.227,4.657)}
\gppoint{gp mark 7}{(4.240,4.658)}
\gppoint{gp mark 7}{(4.253,4.658)}
\gppoint{gp mark 7}{(4.265,4.657)}
\gppoint{gp mark 7}{(4.278,4.657)}
\gppoint{gp mark 7}{(4.291,4.657)}
\gppoint{gp mark 7}{(4.303,4.657)}
\gppoint{gp mark 7}{(4.316,4.657)}
\gppoint{gp mark 7}{(4.329,4.657)}
\gppoint{gp mark 7}{(4.342,4.657)}
\gppoint{gp mark 7}{(4.354,4.658)}
\gppoint{gp mark 7}{(4.367,4.658)}
\gppoint{gp mark 7}{(4.380,4.657)}
\gppoint{gp mark 7}{(4.392,4.657)}
\gppoint{gp mark 7}{(4.405,4.657)}
\gppoint{gp mark 7}{(4.418,4.656)}
\gppoint{gp mark 7}{(4.430,4.656)}
\gppoint{gp mark 7}{(4.443,4.656)}
\gppoint{gp mark 7}{(4.456,4.656)}
\gppoint{gp mark 7}{(4.468,4.657)}
\gppoint{gp mark 7}{(4.481,4.657)}
\gppoint{gp mark 7}{(4.494,4.658)}
\gppoint{gp mark 7}{(4.506,4.658)}
\gppoint{gp mark 7}{(4.519,4.657)}
\gppoint{gp mark 7}{(4.532,4.657)}
\gppoint{gp mark 7}{(4.544,4.657)}
\gppoint{gp mark 7}{(4.557,4.657)}
\gppoint{gp mark 7}{(4.570,4.657)}
\gppoint{gp mark 7}{(4.582,4.658)}
\gppoint{gp mark 7}{(4.595,4.658)}
\gppoint{gp mark 7}{(4.608,4.658)}
\gppoint{gp mark 7}{(4.620,4.658)}
\gppoint{gp mark 7}{(4.633,4.658)}
\gppoint{gp mark 7}{(4.646,4.658)}
\gppoint{gp mark 7}{(4.658,4.657)}
\gppoint{gp mark 7}{(4.671,4.657)}
\gppoint{gp mark 7}{(4.684,4.657)}
\gppoint{gp mark 7}{(4.697,4.658)}
\gppoint{gp mark 7}{(4.709,4.658)}
\gppoint{gp mark 7}{(4.722,4.658)}
\gppoint{gp mark 7}{(4.735,4.658)}
\gppoint{gp mark 7}{(4.747,4.658)}
\gppoint{gp mark 7}{(4.760,4.658)}
\gppoint{gp mark 7}{(4.773,4.657)}
\gppoint{gp mark 7}{(4.785,4.657)}
\gppoint{gp mark 7}{(4.798,4.657)}
\gppoint{gp mark 7}{(4.811,4.657)}
\gppoint{gp mark 7}{(4.823,4.658)}
\gppoint{gp mark 7}{(4.836,4.658)}
\gppoint{gp mark 7}{(4.849,4.658)}
\gppoint{gp mark 7}{(4.861,4.658)}
\gppoint{gp mark 7}{(4.874,4.658)}
\gppoint{gp mark 7}{(4.887,4.657)}
\gppoint{gp mark 7}{(4.899,4.657)}
\gppoint{gp mark 7}{(4.912,4.656)}
\gppoint{gp mark 7}{(4.925,4.656)}
\gppoint{gp mark 7}{(4.937,4.657)}
\gppoint{gp mark 7}{(4.950,4.657)}
\gppoint{gp mark 7}{(4.963,4.658)}
\gppoint{gp mark 7}{(4.975,4.658)}
\gppoint{gp mark 7}{(4.988,4.658)}
\gppoint{gp mark 7}{(5.001,4.658)}
\gppoint{gp mark 7}{(5.013,4.657)}
\gppoint{gp mark 7}{(5.026,4.657)}
\gppoint{gp mark 7}{(5.039,4.657)}
\gppoint{gp mark 7}{(5.052,4.657)}
\gppoint{gp mark 7}{(5.064,4.657)}
\gppoint{gp mark 7}{(5.077,4.658)}
\gppoint{gp mark 7}{(5.090,4.658)}
\gppoint{gp mark 7}{(5.102,4.658)}
\gppoint{gp mark 7}{(5.115,4.658)}
\gppoint{gp mark 7}{(5.128,4.657)}
\gppoint{gp mark 7}{(5.140,4.657)}
\gppoint{gp mark 7}{(5.153,4.657)}
\gppoint{gp mark 7}{(5.166,4.657)}
\gppoint{gp mark 7}{(5.178,4.657)}
\gppoint{gp mark 7}{(5.191,4.658)}
\gppoint{gp mark 7}{(5.204,4.658)}
\gppoint{gp mark 7}{(5.216,4.658)}
\gppoint{gp mark 7}{(5.229,4.658)}
\gppoint{gp mark 7}{(5.242,4.658)}
\gppoint{gp mark 7}{(5.254,4.658)}
\gppoint{gp mark 7}{(5.267,4.657)}
\gppoint{gp mark 7}{(5.280,4.657)}
\gppoint{gp mark 7}{(5.292,4.657)}
\gppoint{gp mark 7}{(5.305,4.657)}
\gppoint{gp mark 7}{(5.318,4.657)}
\gppoint{gp mark 7}{(5.330,4.658)}
\gppoint{gp mark 7}{(5.343,4.658)}
\gppoint{gp mark 7}{(5.356,4.658)}
\gppoint{gp mark 7}{(5.368,4.657)}
\gppoint{gp mark 7}{(5.381,4.657)}
\gppoint{gp mark 7}{(5.394,4.656)}
\gppoint{gp mark 7}{(5.407,4.656)}
\gppoint{gp mark 7}{(5.419,4.656)}
\gppoint{gp mark 7}{(5.432,4.656)}
\gppoint{gp mark 7}{(5.445,4.657)}
\gppoint{gp mark 7}{(5.457,4.657)}
\gppoint{gp mark 7}{(5.470,4.657)}
\gppoint{gp mark 7}{(5.483,4.657)}
\gppoint{gp mark 7}{(5.495,4.657)}
\gppoint{gp mark 7}{(5.508,4.657)}
\gppoint{gp mark 7}{(5.521,4.657)}
\gppoint{gp mark 7}{(5.533,4.657)}
\gppoint{gp mark 7}{(5.546,4.657)}
\gppoint{gp mark 7}{(5.559,4.657)}
\gppoint{gp mark 7}{(5.571,4.657)}
\gppoint{gp mark 7}{(5.584,4.657)}
\gppoint{gp mark 7}{(5.597,4.657)}
\gppoint{gp mark 7}{(5.609,4.657)}
\gppoint{gp mark 7}{(5.622,4.657)}
\gppoint{gp mark 7}{(5.635,4.657)}
\gppoint{gp mark 7}{(5.647,4.657)}
\gppoint{gp mark 7}{(5.660,4.657)}
\gppoint{gp mark 7}{(5.673,4.657)}
\gppoint{gp mark 7}{(5.685,4.657)}
\gppoint{gp mark 7}{(5.698,4.658)}
\gppoint{gp mark 7}{(5.711,4.658)}
\gppoint{gp mark 7}{(5.723,4.658)}
\gppoint{gp mark 7}{(5.736,4.658)}
\gppoint{gp mark 7}{(5.749,4.658)}
\gppoint{gp mark 7}{(5.761,4.657)}
\gppoint{gp mark 7}{(5.774,4.657)}
\gppoint{gp mark 7}{(5.787,4.657)}
\gppoint{gp mark 7}{(5.800,4.658)}
\gppoint{gp mark 7}{(5.812,4.658)}
\gppoint{gp mark 7}{(5.825,4.658)}
\gppoint{gp mark 7}{(5.838,4.658)}
\gppoint{gp mark 7}{(5.850,4.658)}
\gppoint{gp mark 7}{(5.863,4.658)}
\gppoint{gp mark 7}{(5.876,4.658)}
\gppoint{gp mark 7}{(5.888,4.658)}
\gppoint{gp mark 7}{(5.901,4.657)}
\gppoint{gp mark 7}{(5.914,4.657)}
\gppoint{gp mark 7}{(5.926,4.657)}
\gppoint{gp mark 7}{(5.939,4.657)}
\gppoint{gp mark 7}{(5.952,4.658)}
\gppoint{gp mark 7}{(5.964,4.658)}
\gppoint{gp mark 7}{(5.977,4.658)}
\gppoint{gp mark 7}{(5.990,4.657)}
\gppoint{gp mark 7}{(6.002,4.657)}
\gppoint{gp mark 7}{(6.015,4.657)}
\gppoint{gp mark 7}{(6.028,4.658)}
\gppoint{gp mark 7}{(6.040,4.658)}
\gppoint{gp mark 7}{(6.053,4.658)}
\gppoint{gp mark 7}{(6.066,4.658)}
\gppoint{gp mark 7}{(6.078,4.659)}
\gppoint{gp mark 7}{(6.091,4.659)}
\gppoint{gp mark 7}{(6.104,4.658)}
\gppoint{gp mark 7}{(6.116,4.658)}
\gppoint{gp mark 7}{(6.129,4.658)}
\gppoint{gp mark 7}{(6.142,4.658)}
\gppoint{gp mark 7}{(6.155,4.658)}
\gppoint{gp mark 7}{(6.167,4.658)}
\gppoint{gp mark 7}{(6.180,4.659)}
\gppoint{gp mark 7}{(6.193,4.659)}
\gppoint{gp mark 7}{(6.205,4.659)}
\gppoint{gp mark 7}{(6.218,4.659)}
\gppoint{gp mark 7}{(6.231,4.659)}
\gppoint{gp mark 7}{(6.243,4.659)}
\gppoint{gp mark 7}{(6.256,4.659)}
\gppoint{gp mark 7}{(6.269,4.658)}
\gppoint{gp mark 7}{(6.281,4.658)}
\gppoint{gp mark 7}{(6.294,4.659)}
\gppoint{gp mark 7}{(6.307,4.659)}
\gppoint{gp mark 7}{(6.319,4.659)}
\gppoint{gp mark 7}{(6.332,4.659)}
\gppoint{gp mark 7}{(6.345,4.659)}
\gppoint{gp mark 7}{(6.357,4.659)}
\gppoint{gp mark 7}{(6.370,4.659)}
\gppoint{gp mark 7}{(6.383,4.659)}
\gppoint{gp mark 7}{(6.395,4.659)}
\gppoint{gp mark 7}{(6.408,4.659)}
\gppoint{gp mark 7}{(6.421,4.659)}
\gppoint{gp mark 7}{(6.433,4.659)}
\gppoint{gp mark 7}{(6.446,4.658)}
\gppoint{gp mark 7}{(6.459,4.658)}
\gppoint{gp mark 7}{(6.471,4.658)}
\gppoint{gp mark 7}{(6.484,4.658)}
\gppoint{gp mark 7}{(6.497,4.658)}
\gppoint{gp mark 7}{(6.510,4.658)}
\gppoint{gp mark 7}{(6.522,4.658)}
\gppoint{gp mark 7}{(6.535,4.658)}
\gppoint{gp mark 7}{(6.548,4.658)}
\gppoint{gp mark 7}{(6.560,4.658)}
\gppoint{gp mark 7}{(6.573,4.658)}
\gppoint{gp mark 7}{(6.586,4.658)}
\gppoint{gp mark 7}{(6.598,4.658)}
\gppoint{gp mark 7}{(6.611,4.658)}
\gppoint{gp mark 7}{(6.624,4.658)}
\gppoint{gp mark 7}{(6.636,4.657)}
\gppoint{gp mark 7}{(6.649,4.657)}
\gppoint{gp mark 7}{(6.662,4.657)}
\gppoint{gp mark 7}{(6.674,4.657)}
\gppoint{gp mark 7}{(6.687,4.657)}
\gppoint{gp mark 7}{(6.700,4.657)}
\gppoint{gp mark 7}{(6.712,4.657)}
\gppoint{gp mark 7}{(6.725,4.657)}
\gppoint{gp mark 7}{(6.738,4.657)}
\gppoint{gp mark 7}{(6.750,4.656)}
\gppoint{gp mark 7}{(6.763,4.656)}
\gppoint{gp mark 7}{(6.776,4.656)}
\gppoint{gp mark 7}{(6.788,4.656)}
\gppoint{gp mark 7}{(6.801,4.655)}
\gppoint{gp mark 7}{(6.814,4.655)}
\gppoint{gp mark 7}{(6.826,4.655)}
\gppoint{gp mark 7}{(6.839,4.654)}
\gppoint{gp mark 7}{(6.852,4.654)}
\gppoint{gp mark 7}{(6.865,4.654)}
\gppoint{gp mark 7}{(6.877,4.654)}
\gppoint{gp mark 7}{(6.890,4.654)}
\gppoint{gp mark 7}{(6.903,4.654)}
\gppoint{gp mark 7}{(6.915,4.654)}
\gppoint{gp mark 7}{(6.928,4.654)}
\gppoint{gp mark 7}{(6.941,4.654)}
\gppoint{gp mark 7}{(6.953,4.653)}
\gppoint{gp mark 7}{(6.966,4.653)}
\gppoint{gp mark 7}{(6.979,4.652)}
\gppoint{gp mark 7}{(6.991,4.652)}
\gppoint{gp mark 7}{(7.004,4.652)}
\gppoint{gp mark 7}{(7.017,4.652)}
\gppoint{gp mark 7}{(7.029,4.652)}
\gppoint{gp mark 7}{(7.042,4.653)}
\gppoint{gp mark 7}{(7.055,4.653)}
\gppoint{gp mark 7}{(7.067,4.653)}
\gppoint{gp mark 7}{(7.080,4.653)}
\gppoint{gp mark 7}{(7.093,4.653)}
\gppoint{gp mark 7}{(7.105,4.653)}
\gppoint{gp mark 7}{(7.118,4.653)}
\gppoint{gp mark 7}{(7.131,4.653)}
\gppoint{gp mark 7}{(7.143,4.654)}
\gppoint{gp mark 7}{(7.156,4.654)}
\gppoint{gp mark 7}{(7.169,4.654)}
\gppoint{gp mark 7}{(7.181,4.654)}
\gppoint{gp mark 7}{(7.194,4.655)}
\gppoint{gp mark 7}{(7.207,4.655)}
\gppoint{gp mark 7}{(7.220,4.656)}
\gppoint{gp mark 7}{(7.232,4.657)}
\gppoint{gp mark 7}{(7.245,4.660)}
\gppoint{gp mark 7}{(7.258,4.661)}
\gppoint{gp mark 7}{(7.270,4.662)}
\gppoint{gp mark 7}{(7.283,4.662)}
\gppoint{gp mark 7}{(7.296,4.662)}
\gppoint{gp mark 7}{(7.308,4.663)}
\gppoint{gp mark 7}{(7.321,4.664)}
\gppoint{gp mark 7}{(7.334,4.666)}
\gppoint{gp mark 7}{(7.346,4.667)}
\gppoint{gp mark 7}{(7.359,4.667)}
\gppoint{gp mark 7}{(7.372,4.667)}
\gppoint{gp mark 7}{(7.384,4.668)}
\gppoint{gp mark 7}{(7.397,4.668)}
\gppoint{gp mark 7}{(7.410,4.667)}
\gppoint{gp mark 7}{(7.422,4.666)}
\gppoint{gp mark 7}{(7.435,4.664)}
\gppoint{gp mark 7}{(7.448,4.661)}
\gppoint{gp mark 7}{(7.460,4.657)}
\gppoint{gp mark 7}{(7.473,4.654)}
\gppoint{gp mark 7}{(7.486,4.650)}
\gppoint{gp mark 7}{(7.498,4.644)}
\gppoint{gp mark 7}{(7.511,4.635)}
\gppoint{gp mark 7}{(7.524,4.627)}
\gppoint{gp mark 7}{(7.536,4.620)}
\gppoint{gp mark 7}{(7.549,4.615)}
\gppoint{gp mark 7}{(7.562,4.610)}
\gppoint{gp mark 7}{(7.575,4.603)}
\gppoint{gp mark 7}{(7.587,4.594)}
\gppoint{gp mark 7}{(7.600,4.584)}
\gppoint{gp mark 7}{(7.613,4.577)}
\gppoint{gp mark 7}{(7.625,4.574)}
\gppoint{gp mark 7}{(7.638,4.573)}
\gppoint{gp mark 7}{(7.651,4.572)}
\gppoint{gp mark 7}{(7.663,4.569)}
\gppoint{gp mark 7}{(7.676,4.545)}
\gppoint{gp mark 7}{(7.689,4.377)}
\gppoint{gp mark 7}{(7.701,3.827)}
\gppoint{gp mark 7}{(7.714,2.973)}
\gppoint{gp mark 7}{(7.727,2.138)}
\gppoint{gp mark 7}{(7.739,1.570)}
\gppoint{gp mark 7}{(7.752,1.288)}
\gppoint{gp mark 7}{(7.765,1.183)}
\gppoint{gp mark 7}{(7.777,1.155)}
\gppoint{gp mark 7}{(7.790,1.150)}
\gppoint{gp mark 7}{(7.803,1.150)}
\gppoint{gp mark 7}{(7.815,1.150)}
\gppoint{gp mark 7}{(7.828,1.151)}
\gppoint{gp mark 7}{(7.841,1.155)}
\gppoint{gp mark 7}{(7.853,1.160)}
\gppoint{gp mark 7}{(7.866,1.163)}
\gppoint{gp mark 7}{(7.879,1.163)}
\gppoint{gp mark 7}{(7.891,1.163)}
\gppoint{gp mark 7}{(7.904,1.163)}
\gppoint{gp mark 7}{(7.917,1.164)}
\gppoint{gp mark 7}{(7.930,1.164)}
\gpcolor{rgb color={1.000,0.000,0.000}}
\gpsetpointsize{4.44}
\gppoint{gp mark 7}{(1.210,3.636)}
\gppoint{gp mark 7}{(1.223,3.620)}
\gppoint{gp mark 7}{(1.235,3.605)}
\gppoint{gp mark 7}{(1.248,3.590)}
\gppoint{gp mark 7}{(1.261,3.574)}
\gppoint{gp mark 7}{(1.273,3.559)}
\gppoint{gp mark 7}{(1.286,3.543)}
\gppoint{gp mark 7}{(1.299,3.528)}
\gppoint{gp mark 7}{(1.311,3.513)}
\gppoint{gp mark 7}{(1.324,3.497)}
\gppoint{gp mark 7}{(1.337,3.482)}
\gppoint{gp mark 7}{(1.349,3.467)}
\gppoint{gp mark 7}{(1.362,3.451)}
\gppoint{gp mark 7}{(1.375,3.436)}
\gppoint{gp mark 7}{(1.387,3.421)}
\gppoint{gp mark 7}{(1.400,3.405)}
\gppoint{gp mark 7}{(1.413,3.390)}
\gppoint{gp mark 7}{(1.425,3.375)}
\gppoint{gp mark 7}{(1.438,3.360)}
\gppoint{gp mark 7}{(1.451,3.344)}
\gppoint{gp mark 7}{(1.464,3.329)}
\gppoint{gp mark 7}{(1.476,3.314)}
\gppoint{gp mark 7}{(1.489,3.299)}
\gppoint{gp mark 7}{(1.502,3.284)}
\gppoint{gp mark 7}{(1.514,3.269)}
\gppoint{gp mark 7}{(1.527,3.253)}
\gppoint{gp mark 7}{(1.540,3.238)}
\gppoint{gp mark 7}{(1.552,3.223)}
\gppoint{gp mark 7}{(1.565,3.208)}
\gppoint{gp mark 7}{(1.578,3.193)}
\gppoint{gp mark 7}{(1.590,3.178)}
\gppoint{gp mark 7}{(1.603,3.163)}
\gppoint{gp mark 7}{(1.616,3.148)}
\gppoint{gp mark 7}{(1.628,3.133)}
\gppoint{gp mark 7}{(1.641,3.118)}
\gppoint{gp mark 7}{(1.654,3.103)}
\gppoint{gp mark 7}{(1.666,3.088)}
\gppoint{gp mark 7}{(1.679,3.073)}
\gppoint{gp mark 7}{(1.692,3.058)}
\gppoint{gp mark 7}{(1.704,3.043)}
\gppoint{gp mark 7}{(1.717,3.028)}
\gppoint{gp mark 7}{(1.730,3.013)}
\gppoint{gp mark 7}{(1.742,2.998)}
\gppoint{gp mark 7}{(1.755,2.983)}
\gppoint{gp mark 7}{(1.768,2.968)}
\gppoint{gp mark 7}{(1.780,2.953)}
\gppoint{gp mark 7}{(1.793,2.939)}
\gppoint{gp mark 7}{(1.806,2.924)}
\gppoint{gp mark 7}{(1.819,2.909)}
\gppoint{gp mark 7}{(1.831,2.894)}
\gppoint{gp mark 7}{(1.844,2.879)}
\gppoint{gp mark 7}{(1.857,2.865)}
\gppoint{gp mark 7}{(1.869,2.850)}
\gppoint{gp mark 7}{(1.882,2.835)}
\gppoint{gp mark 7}{(1.895,2.820)}
\gppoint{gp mark 7}{(1.907,2.806)}
\gppoint{gp mark 7}{(1.920,2.791)}
\gppoint{gp mark 7}{(1.933,2.776)}
\gppoint{gp mark 7}{(1.945,2.762)}
\gppoint{gp mark 7}{(1.958,2.747)}
\gppoint{gp mark 7}{(1.971,2.732)}
\gppoint{gp mark 7}{(1.983,2.718)}
\gppoint{gp mark 7}{(1.996,2.703)}
\gppoint{gp mark 7}{(2.009,2.688)}
\gppoint{gp mark 7}{(2.021,2.674)}
\gppoint{gp mark 7}{(2.034,2.659)}
\gppoint{gp mark 7}{(2.047,2.645)}
\gppoint{gp mark 7}{(2.059,2.630)}
\gppoint{gp mark 7}{(2.072,2.616)}
\gppoint{gp mark 7}{(2.085,2.601)}
\gppoint{gp mark 7}{(2.097,2.587)}
\gppoint{gp mark 7}{(2.110,2.572)}
\gppoint{gp mark 7}{(2.123,2.558)}
\gppoint{gp mark 7}{(2.135,2.544)}
\gppoint{gp mark 7}{(2.148,2.529)}
\gppoint{gp mark 7}{(2.161,2.515)}
\gppoint{gp mark 7}{(2.174,2.500)}
\gppoint{gp mark 7}{(2.186,2.486)}
\gppoint{gp mark 7}{(2.199,2.472)}
\gppoint{gp mark 7}{(2.212,2.457)}
\gppoint{gp mark 7}{(2.224,2.443)}
\gppoint{gp mark 7}{(2.237,2.429)}
\gppoint{gp mark 7}{(2.250,2.415)}
\gppoint{gp mark 7}{(2.262,2.401)}
\gppoint{gp mark 7}{(2.275,2.386)}
\gppoint{gp mark 7}{(2.288,2.372)}
\gppoint{gp mark 7}{(2.300,2.358)}
\gppoint{gp mark 7}{(2.313,2.344)}
\gppoint{gp mark 7}{(2.326,2.330)}
\gppoint{gp mark 7}{(2.338,2.316)}
\gppoint{gp mark 7}{(2.351,2.302)}
\gppoint{gp mark 7}{(2.364,2.288)}
\gppoint{gp mark 7}{(2.376,2.275)}
\gppoint{gp mark 7}{(2.389,2.261)}
\gppoint{gp mark 7}{(2.402,2.247)}
\gppoint{gp mark 7}{(2.414,2.234)}
\gppoint{gp mark 7}{(2.427,2.221)}
\gppoint{gp mark 7}{(2.440,2.209)}
\gppoint{gp mark 7}{(2.452,2.197)}
\gppoint{gp mark 7}{(2.465,2.188)}
\gppoint{gp mark 7}{(2.478,2.180)}
\gppoint{gp mark 7}{(2.490,2.176)}
\gppoint{gp mark 7}{(2.503,2.174)}
\gppoint{gp mark 7}{(2.516,2.174)}
\gppoint{gp mark 7}{(2.529,2.174)}
\gppoint{gp mark 7}{(2.541,2.174)}
\gppoint{gp mark 7}{(2.554,2.174)}
\gppoint{gp mark 7}{(2.567,2.175)}
\gppoint{gp mark 7}{(2.579,2.177)}
\gppoint{gp mark 7}{(2.592,2.180)}
\gppoint{gp mark 7}{(2.605,2.183)}
\gppoint{gp mark 7}{(2.617,2.184)}
\gppoint{gp mark 7}{(2.630,2.184)}
\gppoint{gp mark 7}{(2.643,2.184)}
\gppoint{gp mark 7}{(2.655,2.184)}
\gppoint{gp mark 7}{(2.668,2.185)}
\gppoint{gp mark 7}{(2.681,2.185)}
\gppoint{gp mark 7}{(2.693,2.186)}
\gppoint{gp mark 7}{(2.706,2.186)}
\gppoint{gp mark 7}{(2.719,2.186)}
\gppoint{gp mark 7}{(2.731,2.186)}
\gppoint{gp mark 7}{(2.744,2.186)}
\gppoint{gp mark 7}{(2.757,2.187)}
\gppoint{gp mark 7}{(2.769,2.187)}
\gppoint{gp mark 7}{(2.782,2.187)}
\gppoint{gp mark 7}{(2.795,2.187)}
\gppoint{gp mark 7}{(2.807,2.187)}
\gppoint{gp mark 7}{(2.820,2.187)}
\gppoint{gp mark 7}{(2.833,2.187)}
\gppoint{gp mark 7}{(2.845,2.187)}
\gppoint{gp mark 7}{(2.858,2.189)}
\gppoint{gp mark 7}{(2.871,2.206)}
\gppoint{gp mark 7}{(2.883,2.305)}
\gppoint{gp mark 7}{(2.896,2.204)}
\gppoint{gp mark 7}{(2.909,2.091)}
\gppoint{gp mark 7}{(2.922,2.199)}
\gppoint{gp mark 7}{(2.934,2.171)}
\gppoint{gp mark 7}{(2.947,2.180)}
\gppoint{gp mark 7}{(2.960,2.187)}
\gppoint{gp mark 7}{(2.972,2.186)}
\gppoint{gp mark 7}{(2.985,2.187)}
\gppoint{gp mark 7}{(2.998,2.190)}
\gppoint{gp mark 7}{(3.010,2.191)}
\gppoint{gp mark 7}{(3.023,2.190)}
\gppoint{gp mark 7}{(3.036,2.189)}
\gppoint{gp mark 7}{(3.048,2.189)}
\gppoint{gp mark 7}{(3.061,2.189)}
\gppoint{gp mark 7}{(3.074,2.189)}
\gppoint{gp mark 7}{(3.086,2.189)}
\gppoint{gp mark 7}{(3.099,2.187)}
\gppoint{gp mark 7}{(3.112,2.186)}
\gppoint{gp mark 7}{(3.124,2.186)}
\gppoint{gp mark 7}{(3.137,2.188)}
\gppoint{gp mark 7}{(3.150,2.189)}
\gppoint{gp mark 7}{(3.162,2.190)}
\gppoint{gp mark 7}{(3.175,2.189)}
\gppoint{gp mark 7}{(3.188,2.188)}
\gppoint{gp mark 7}{(3.200,2.187)}
\gppoint{gp mark 7}{(3.213,2.188)}
\gppoint{gp mark 7}{(3.226,2.189)}
\gppoint{gp mark 7}{(3.238,2.188)}
\gppoint{gp mark 7}{(3.251,2.188)}
\gppoint{gp mark 7}{(3.264,2.187)}
\gppoint{gp mark 7}{(3.277,2.187)}
\gppoint{gp mark 7}{(3.289,2.187)}
\gppoint{gp mark 7}{(3.302,2.187)}
\gppoint{gp mark 7}{(3.315,2.187)}
\gppoint{gp mark 7}{(3.327,2.187)}
\gppoint{gp mark 7}{(3.340,2.187)}
\gppoint{gp mark 7}{(3.353,2.187)}
\gppoint{gp mark 7}{(3.365,2.189)}
\gppoint{gp mark 7}{(3.378,2.189)}
\gppoint{gp mark 7}{(3.391,2.189)}
\gppoint{gp mark 7}{(3.403,2.188)}
\gppoint{gp mark 7}{(3.416,2.187)}
\gppoint{gp mark 7}{(3.429,2.187)}
\gppoint{gp mark 7}{(3.441,2.187)}
\gppoint{gp mark 7}{(3.454,2.188)}
\gppoint{gp mark 7}{(3.467,2.188)}
\gppoint{gp mark 7}{(3.479,2.189)}
\gppoint{gp mark 7}{(3.492,2.189)}
\gppoint{gp mark 7}{(3.505,2.190)}
\gppoint{gp mark 7}{(3.517,2.190)}
\gppoint{gp mark 7}{(3.530,2.191)}
\gppoint{gp mark 7}{(3.543,2.191)}
\gppoint{gp mark 7}{(3.555,2.190)}
\gppoint{gp mark 7}{(3.568,2.189)}
\gppoint{gp mark 7}{(3.581,2.188)}
\gppoint{gp mark 7}{(3.593,2.188)}
\gppoint{gp mark 7}{(3.606,2.188)}
\gppoint{gp mark 7}{(3.619,2.188)}
\gppoint{gp mark 7}{(3.632,2.189)}
\gppoint{gp mark 7}{(3.644,2.190)}
\gppoint{gp mark 7}{(3.657,2.191)}
\gppoint{gp mark 7}{(3.670,2.192)}
\gppoint{gp mark 7}{(3.682,2.192)}
\gppoint{gp mark 7}{(3.695,2.192)}
\gppoint{gp mark 7}{(3.708,2.192)}
\gppoint{gp mark 7}{(3.720,2.190)}
\gppoint{gp mark 7}{(3.733,2.188)}
\gppoint{gp mark 7}{(3.746,2.187)}
\gppoint{gp mark 7}{(3.758,2.191)}
\gppoint{gp mark 7}{(3.771,2.237)}
\gppoint{gp mark 7}{(3.784,2.647)}
\gppoint{gp mark 7}{(3.796,3.988)}
\gppoint{gp mark 7}{(3.809,4.643)}
\gppoint{gp mark 7}{(3.822,4.693)}
\gppoint{gp mark 7}{(3.834,4.692)}
\gppoint{gp mark 7}{(3.847,4.691)}
\gppoint{gp mark 7}{(3.860,4.690)}
\gppoint{gp mark 7}{(3.872,4.691)}
\gppoint{gp mark 7}{(3.885,4.692)}
\gppoint{gp mark 7}{(3.898,4.692)}
\gppoint{gp mark 7}{(3.910,4.691)}
\gppoint{gp mark 7}{(3.923,4.691)}
\gppoint{gp mark 7}{(3.936,4.692)}
\gppoint{gp mark 7}{(3.948,4.692)}
\gppoint{gp mark 7}{(3.961,4.692)}
\gppoint{gp mark 7}{(3.974,4.691)}
\gppoint{gp mark 7}{(3.987,4.691)}
\gppoint{gp mark 7}{(3.999,4.691)}
\gppoint{gp mark 7}{(4.012,4.692)}
\gppoint{gp mark 7}{(4.025,4.692)}
\gppoint{gp mark 7}{(4.037,4.692)}
\gppoint{gp mark 7}{(4.050,4.692)}
\gppoint{gp mark 7}{(4.063,4.693)}
\gppoint{gp mark 7}{(4.075,4.693)}
\gppoint{gp mark 7}{(4.088,4.693)}
\gppoint{gp mark 7}{(4.101,4.691)}
\gppoint{gp mark 7}{(4.113,4.691)}
\gppoint{gp mark 7}{(4.126,4.691)}
\gppoint{gp mark 7}{(4.139,4.692)}
\gppoint{gp mark 7}{(4.151,4.692)}
\gppoint{gp mark 7}{(4.164,4.692)}
\gppoint{gp mark 7}{(4.177,4.692)}
\gppoint{gp mark 7}{(4.189,4.692)}
\gppoint{gp mark 7}{(4.202,4.693)}
\gppoint{gp mark 7}{(4.215,4.693)}
\gppoint{gp mark 7}{(4.227,4.692)}
\gppoint{gp mark 7}{(4.240,4.692)}
\gppoint{gp mark 7}{(4.253,4.692)}
\gppoint{gp mark 7}{(4.265,4.691)}
\gppoint{gp mark 7}{(4.278,4.691)}
\gppoint{gp mark 7}{(4.291,4.691)}
\gppoint{gp mark 7}{(4.303,4.692)}
\gppoint{gp mark 7}{(4.316,4.693)}
\gppoint{gp mark 7}{(4.329,4.693)}
\gppoint{gp mark 7}{(4.342,4.693)}
\gppoint{gp mark 7}{(4.354,4.692)}
\gppoint{gp mark 7}{(4.367,4.692)}
\gppoint{gp mark 7}{(4.380,4.692)}
\gppoint{gp mark 7}{(4.392,4.692)}
\gppoint{gp mark 7}{(4.405,4.692)}
\gppoint{gp mark 7}{(4.418,4.692)}
\gppoint{gp mark 7}{(4.430,4.692)}
\gppoint{gp mark 7}{(4.443,4.692)}
\gppoint{gp mark 7}{(4.456,4.692)}
\gppoint{gp mark 7}{(4.468,4.692)}
\gppoint{gp mark 7}{(4.481,4.692)}
\gppoint{gp mark 7}{(4.494,4.692)}
\gppoint{gp mark 7}{(4.506,4.691)}
\gppoint{gp mark 7}{(4.519,4.691)}
\gppoint{gp mark 7}{(4.532,4.691)}
\gppoint{gp mark 7}{(4.544,4.691)}
\gppoint{gp mark 7}{(4.557,4.692)}
\gppoint{gp mark 7}{(4.570,4.692)}
\gppoint{gp mark 7}{(4.582,4.692)}
\gppoint{gp mark 7}{(4.595,4.692)}
\gppoint{gp mark 7}{(4.608,4.691)}
\gppoint{gp mark 7}{(4.620,4.691)}
\gppoint{gp mark 7}{(4.633,4.691)}
\gppoint{gp mark 7}{(4.646,4.691)}
\gppoint{gp mark 7}{(4.658,4.692)}
\gppoint{gp mark 7}{(4.671,4.692)}
\gppoint{gp mark 7}{(4.684,4.692)}
\gppoint{gp mark 7}{(4.697,4.692)}
\gppoint{gp mark 7}{(4.709,4.692)}
\gppoint{gp mark 7}{(4.722,4.692)}
\gppoint{gp mark 7}{(4.735,4.692)}
\gppoint{gp mark 7}{(4.747,4.691)}
\gppoint{gp mark 7}{(4.760,4.691)}
\gppoint{gp mark 7}{(4.773,4.691)}
\gppoint{gp mark 7}{(4.785,4.691)}
\gppoint{gp mark 7}{(4.798,4.691)}
\gppoint{gp mark 7}{(4.811,4.691)}
\gppoint{gp mark 7}{(4.823,4.691)}
\gppoint{gp mark 7}{(4.836,4.692)}
\gppoint{gp mark 7}{(4.849,4.692)}
\gppoint{gp mark 7}{(4.861,4.691)}
\gppoint{gp mark 7}{(4.874,4.691)}
\gppoint{gp mark 7}{(4.887,4.691)}
\gppoint{gp mark 7}{(4.899,4.692)}
\gppoint{gp mark 7}{(4.912,4.692)}
\gppoint{gp mark 7}{(4.925,4.692)}
\gppoint{gp mark 7}{(4.937,4.692)}
\gppoint{gp mark 7}{(4.950,4.692)}
\gppoint{gp mark 7}{(4.963,4.691)}
\gppoint{gp mark 7}{(4.975,4.691)}
\gppoint{gp mark 7}{(4.988,4.691)}
\gppoint{gp mark 7}{(5.001,4.691)}
\gppoint{gp mark 7}{(5.013,4.691)}
\gppoint{gp mark 7}{(5.026,4.692)}
\gppoint{gp mark 7}{(5.039,4.692)}
\gppoint{gp mark 7}{(5.052,4.692)}
\gppoint{gp mark 7}{(5.064,4.692)}
\gppoint{gp mark 7}{(5.077,4.692)}
\gppoint{gp mark 7}{(5.090,4.692)}
\gppoint{gp mark 7}{(5.102,4.692)}
\gppoint{gp mark 7}{(5.115,4.691)}
\gppoint{gp mark 7}{(5.128,4.691)}
\gppoint{gp mark 7}{(5.140,4.691)}
\gppoint{gp mark 7}{(5.153,4.691)}
\gppoint{gp mark 7}{(5.166,4.691)}
\gppoint{gp mark 7}{(5.178,4.692)}
\gppoint{gp mark 7}{(5.191,4.692)}
\gppoint{gp mark 7}{(5.204,4.692)}
\gppoint{gp mark 7}{(5.216,4.692)}
\gppoint{gp mark 7}{(5.229,4.691)}
\gppoint{gp mark 7}{(5.242,4.691)}
\gppoint{gp mark 7}{(5.254,4.691)}
\gppoint{gp mark 7}{(5.267,4.691)}
\gppoint{gp mark 7}{(5.280,4.691)}
\gppoint{gp mark 7}{(5.292,4.691)}
\gppoint{gp mark 7}{(5.305,4.691)}
\gppoint{gp mark 7}{(5.318,4.692)}
\gppoint{gp mark 7}{(5.330,4.692)}
\gppoint{gp mark 7}{(5.343,4.692)}
\gppoint{gp mark 7}{(5.356,4.691)}
\gppoint{gp mark 7}{(5.368,4.691)}
\gppoint{gp mark 7}{(5.381,4.691)}
\gppoint{gp mark 7}{(5.394,4.691)}
\gppoint{gp mark 7}{(5.407,4.692)}
\gppoint{gp mark 7}{(5.419,4.692)}
\gppoint{gp mark 7}{(5.432,4.692)}
\gppoint{gp mark 7}{(5.445,4.692)}
\gppoint{gp mark 7}{(5.457,4.692)}
\gppoint{gp mark 7}{(5.470,4.691)}
\gppoint{gp mark 7}{(5.483,4.691)}
\gppoint{gp mark 7}{(5.495,4.691)}
\gppoint{gp mark 7}{(5.508,4.691)}
\gppoint{gp mark 7}{(5.521,4.691)}
\gppoint{gp mark 7}{(5.533,4.692)}
\gppoint{gp mark 7}{(5.546,4.692)}
\gppoint{gp mark 7}{(5.559,4.692)}
\gppoint{gp mark 7}{(5.571,4.692)}
\gppoint{gp mark 7}{(5.584,4.692)}
\gppoint{gp mark 7}{(5.597,4.692)}
\gppoint{gp mark 7}{(5.609,4.692)}
\gppoint{gp mark 7}{(5.622,4.691)}
\gppoint{gp mark 7}{(5.635,4.691)}
\gppoint{gp mark 7}{(5.647,4.691)}
\gppoint{gp mark 7}{(5.660,4.691)}
\gppoint{gp mark 7}{(5.673,4.692)}
\gppoint{gp mark 7}{(5.685,4.692)}
\gppoint{gp mark 7}{(5.698,4.692)}
\gppoint{gp mark 7}{(5.711,4.692)}
\gppoint{gp mark 7}{(5.723,4.692)}
\gppoint{gp mark 7}{(5.736,4.691)}
\gppoint{gp mark 7}{(5.749,4.691)}
\gppoint{gp mark 7}{(5.761,4.691)}
\gppoint{gp mark 7}{(5.774,4.691)}
\gppoint{gp mark 7}{(5.787,4.691)}
\gppoint{gp mark 7}{(5.800,4.691)}
\gppoint{gp mark 7}{(5.812,4.692)}
\gppoint{gp mark 7}{(5.825,4.692)}
\gppoint{gp mark 7}{(5.838,4.692)}
\gppoint{gp mark 7}{(5.850,4.692)}
\gppoint{gp mark 7}{(5.863,4.692)}
\gppoint{gp mark 7}{(5.876,4.692)}
\gppoint{gp mark 7}{(5.888,4.692)}
\gppoint{gp mark 7}{(5.901,4.692)}
\gppoint{gp mark 7}{(5.914,4.692)}
\gppoint{gp mark 7}{(5.926,4.692)}
\gppoint{gp mark 7}{(5.939,4.692)}
\gppoint{gp mark 7}{(5.952,4.692)}
\gppoint{gp mark 7}{(5.964,4.692)}
\gppoint{gp mark 7}{(5.977,4.692)}
\gppoint{gp mark 7}{(5.990,4.691)}
\gppoint{gp mark 7}{(6.002,4.691)}
\gppoint{gp mark 7}{(6.015,4.691)}
\gppoint{gp mark 7}{(6.028,4.691)}
\gppoint{gp mark 7}{(6.040,4.691)}
\gppoint{gp mark 7}{(6.053,4.691)}
\gppoint{gp mark 7}{(6.066,4.692)}
\gppoint{gp mark 7}{(6.078,4.692)}
\gppoint{gp mark 7}{(6.091,4.692)}
\gppoint{gp mark 7}{(6.104,4.692)}
\gppoint{gp mark 7}{(6.116,4.691)}
\gppoint{gp mark 7}{(6.129,4.691)}
\gppoint{gp mark 7}{(6.142,4.691)}
\gppoint{gp mark 7}{(6.155,4.691)}
\gppoint{gp mark 7}{(6.167,4.691)}
\gppoint{gp mark 7}{(6.180,4.691)}
\gppoint{gp mark 7}{(6.193,4.691)}
\gppoint{gp mark 7}{(6.205,4.691)}
\gppoint{gp mark 7}{(6.218,4.691)}
\gppoint{gp mark 7}{(6.231,4.691)}
\gppoint{gp mark 7}{(6.243,4.691)}
\gppoint{gp mark 7}{(6.256,4.690)}
\gppoint{gp mark 7}{(6.269,4.690)}
\gppoint{gp mark 7}{(6.281,4.690)}
\gppoint{gp mark 7}{(6.294,4.690)}
\gppoint{gp mark 7}{(6.307,4.690)}
\gppoint{gp mark 7}{(6.319,4.690)}
\gppoint{gp mark 7}{(6.332,4.690)}
\gppoint{gp mark 7}{(6.345,4.690)}
\gppoint{gp mark 7}{(6.357,4.690)}
\gppoint{gp mark 7}{(6.370,4.690)}
\gppoint{gp mark 7}{(6.383,4.690)}
\gppoint{gp mark 7}{(6.395,4.690)}
\gppoint{gp mark 7}{(6.408,4.690)}
\gppoint{gp mark 7}{(6.421,4.690)}
\gppoint{gp mark 7}{(6.433,4.690)}
\gppoint{gp mark 7}{(6.446,4.690)}
\gppoint{gp mark 7}{(6.459,4.690)}
\gppoint{gp mark 7}{(6.471,4.690)}
\gppoint{gp mark 7}{(6.484,4.690)}
\gppoint{gp mark 7}{(6.497,4.690)}
\gppoint{gp mark 7}{(6.510,4.690)}
\gppoint{gp mark 7}{(6.522,4.690)}
\gppoint{gp mark 7}{(6.535,4.690)}
\gppoint{gp mark 7}{(6.548,4.690)}
\gppoint{gp mark 7}{(6.560,4.689)}
\gppoint{gp mark 7}{(6.573,4.689)}
\gppoint{gp mark 7}{(6.586,4.689)}
\gppoint{gp mark 7}{(6.598,4.689)}
\gppoint{gp mark 7}{(6.611,4.689)}
\gppoint{gp mark 7}{(6.624,4.688)}
\gppoint{gp mark 7}{(6.636,4.688)}
\gppoint{gp mark 7}{(6.649,4.688)}
\gppoint{gp mark 7}{(6.662,4.688)}
\gppoint{gp mark 7}{(6.674,4.688)}
\gppoint{gp mark 7}{(6.687,4.688)}
\gppoint{gp mark 7}{(6.700,4.688)}
\gppoint{gp mark 7}{(6.712,4.688)}
\gppoint{gp mark 7}{(6.725,4.687)}
\gppoint{gp mark 7}{(6.738,4.687)}
\gppoint{gp mark 7}{(6.750,4.687)}
\gppoint{gp mark 7}{(6.763,4.687)}
\gppoint{gp mark 7}{(6.776,4.687)}
\gppoint{gp mark 7}{(6.788,4.687)}
\gppoint{gp mark 7}{(6.801,4.687)}
\gppoint{gp mark 7}{(6.814,4.687)}
\gppoint{gp mark 7}{(6.826,4.687)}
\gppoint{gp mark 7}{(6.839,4.686)}
\gppoint{gp mark 7}{(6.852,4.686)}
\gppoint{gp mark 7}{(6.865,4.686)}
\gppoint{gp mark 7}{(6.877,4.686)}
\gppoint{gp mark 7}{(6.890,4.686)}
\gppoint{gp mark 7}{(6.903,4.686)}
\gppoint{gp mark 7}{(6.915,4.686)}
\gppoint{gp mark 7}{(6.928,4.686)}
\gppoint{gp mark 7}{(6.941,4.686)}
\gppoint{gp mark 7}{(6.953,4.686)}
\gppoint{gp mark 7}{(6.966,4.686)}
\gppoint{gp mark 7}{(6.979,4.687)}
\gppoint{gp mark 7}{(6.991,4.687)}
\gppoint{gp mark 7}{(7.004,4.687)}
\gppoint{gp mark 7}{(7.017,4.687)}
\gppoint{gp mark 7}{(7.029,4.687)}
\gppoint{gp mark 7}{(7.042,4.687)}
\gppoint{gp mark 7}{(7.055,4.687)}
\gppoint{gp mark 7}{(7.067,4.687)}
\gppoint{gp mark 7}{(7.080,4.688)}
\gppoint{gp mark 7}{(7.093,4.688)}
\gppoint{gp mark 7}{(7.105,4.689)}
\gppoint{gp mark 7}{(7.118,4.689)}
\gppoint{gp mark 7}{(7.131,4.689)}
\gppoint{gp mark 7}{(7.143,4.690)}
\gppoint{gp mark 7}{(7.156,4.691)}
\gppoint{gp mark 7}{(7.169,4.692)}
\gppoint{gp mark 7}{(7.181,4.692)}
\gppoint{gp mark 7}{(7.194,4.693)}
\gppoint{gp mark 7}{(7.207,4.694)}
\gppoint{gp mark 7}{(7.220,4.696)}
\gppoint{gp mark 7}{(7.232,4.699)}
\gppoint{gp mark 7}{(7.245,4.701)}
\gppoint{gp mark 7}{(7.258,4.703)}
\gppoint{gp mark 7}{(7.270,4.704)}
\gppoint{gp mark 7}{(7.283,4.705)}
\gppoint{gp mark 7}{(7.296,4.706)}
\gppoint{gp mark 7}{(7.308,4.708)}
\gppoint{gp mark 7}{(7.321,4.711)}
\gppoint{gp mark 7}{(7.334,4.715)}
\gppoint{gp mark 7}{(7.346,4.718)}
\gppoint{gp mark 7}{(7.359,4.719)}
\gppoint{gp mark 7}{(7.372,4.720)}
\gppoint{gp mark 7}{(7.384,4.720)}
\gppoint{gp mark 7}{(7.397,4.720)}
\gppoint{gp mark 7}{(7.410,4.720)}
\gppoint{gp mark 7}{(7.422,4.720)}
\gppoint{gp mark 7}{(7.435,4.719)}
\gppoint{gp mark 7}{(7.448,4.717)}
\gppoint{gp mark 7}{(7.460,4.712)}
\gppoint{gp mark 7}{(7.473,4.705)}
\gppoint{gp mark 7}{(7.486,4.698)}
\gppoint{gp mark 7}{(7.498,4.691)}
\gppoint{gp mark 7}{(7.511,4.685)}
\gppoint{gp mark 7}{(7.524,4.679)}
\gppoint{gp mark 7}{(7.536,4.672)}
\gppoint{gp mark 7}{(7.549,4.660)}
\gppoint{gp mark 7}{(7.562,4.643)}
\gppoint{gp mark 7}{(7.575,4.621)}
\gppoint{gp mark 7}{(7.587,4.601)}
\gppoint{gp mark 7}{(7.600,4.590)}
\gppoint{gp mark 7}{(7.613,4.586)}
\gppoint{gp mark 7}{(7.625,4.585)}
\gppoint{gp mark 7}{(7.638,4.584)}
\gppoint{gp mark 7}{(7.651,4.582)}
\gppoint{gp mark 7}{(7.663,4.570)}
\gppoint{gp mark 7}{(7.676,4.484)}
\gppoint{gp mark 7}{(7.689,4.088)}
\gppoint{gp mark 7}{(7.701,3.343)}
\gppoint{gp mark 7}{(7.714,2.509)}
\gppoint{gp mark 7}{(7.727,1.854)}
\gppoint{gp mark 7}{(7.739,1.459)}
\gppoint{gp mark 7}{(7.752,1.264)}
\gppoint{gp mark 7}{(7.765,1.183)}
\gppoint{gp mark 7}{(7.777,1.154)}
\gppoint{gp mark 7}{(7.790,1.146)}
\gppoint{gp mark 7}{(7.803,1.145)}
\gppoint{gp mark 7}{(7.815,1.145)}
\gppoint{gp mark 7}{(7.828,1.145)}
\gppoint{gp mark 7}{(7.841,1.147)}
\gppoint{gp mark 7}{(7.853,1.151)}
\gppoint{gp mark 7}{(7.866,1.153)}
\gppoint{gp mark 7}{(7.879,1.154)}
\gppoint{gp mark 7}{(7.891,1.155)}
\gppoint{gp mark 7}{(7.904,1.155)}
\gppoint{gp mark 7}{(7.917,1.155)}
\gppoint{gp mark 7}{(7.930,1.155)}
\gpcolor{rgb color={0.000,0.000,0.000}}
\gpsetlinetype{gp lt plot 0}
\gpsetlinewidth{4.00}
\draw[gp path] (2.440,2.203)--(2.899,2.203);
\draw[gp path] (2.899,2.203)--(3.805,2.203);
\draw[gp path] (3.805,4.694)--(7.720,4.694);
\draw[gp path] (7.720,1.153)--(7.947,1.153);
\draw[gp path] (1.204,3.442)--(1.217,3.426)--(1.230,3.411)--(1.243,3.396)--(1.256,3.380)%
  --(1.269,3.365)--(1.282,3.350)--(1.295,3.334)--(1.308,3.319)--(1.321,3.304)--(1.334,3.289)%
  --(1.347,3.274)--(1.360,3.259)--(1.373,3.244)--(1.386,3.229)--(1.399,3.214)--(1.412,3.199)%
  --(1.425,3.184)--(1.438,3.169)--(1.451,3.155)--(1.464,3.140)--(1.477,3.125)--(1.490,3.111)%
  --(1.503,3.096)--(1.516,3.082)--(1.529,3.067)--(1.542,3.053)--(1.555,3.038)--(1.568,3.024)%
  --(1.581,3.010)--(1.594,2.996)--(1.607,2.982)--(1.620,2.967)--(1.633,2.953)--(1.646,2.939)%
  --(1.659,2.925)--(1.672,2.911)--(1.686,2.898)--(1.699,2.884)--(1.712,2.870)--(1.725,2.856)%
  --(1.738,2.843)--(1.751,2.829)--(1.764,2.815)--(1.777,2.802)--(1.790,2.788)--(1.803,2.775)%
  --(1.816,2.762)--(1.829,2.748)--(1.842,2.735)--(1.855,2.722)--(1.868,2.709)--(1.881,2.696)%
  --(1.894,2.683)--(1.907,2.670)--(1.920,2.657)--(1.933,2.644)--(1.946,2.631)--(1.959,2.618)%
  --(1.972,2.605)--(1.985,2.593)--(1.998,2.580)--(2.011,2.567)--(2.024,2.555)--(2.037,2.542)%
  --(2.050,2.530)--(2.063,2.518)--(2.076,2.505)--(2.089,2.493)--(2.102,2.481)--(2.115,2.469)%
  --(2.128,2.457)--(2.141,2.445)--(2.154,2.433)--(2.167,2.421)--(2.180,2.409)--(2.193,2.397)%
  --(2.206,2.386)--(2.219,2.374)--(2.232,2.362)--(2.245,2.351)--(2.258,2.339)--(2.271,2.328)%
  --(2.284,2.317)--(2.297,2.305)--(2.310,2.294)--(2.323,2.283)--(2.336,2.272)--(2.349,2.261)%
  --(2.362,2.250)--(2.375,2.239)--(2.388,2.228)--(2.401,2.217)--(2.414,2.206)--(2.427,2.196)%
  --(2.440,2.203);
\draw[gp path] (3.805,2.203)--(3.805,4.694);
\draw[gp path] (7.720,4.694)--(7.720,1.153);
\node[gp node left,font={\fontsize{10pt}{12pt}\selectfont}] at (1.456,5.268) {\LARGE $\rho$};
\node[gp node left,font={\fontsize{10pt}{12pt}\selectfont}] at (5.740,5.268) {\large $\alpha = \pi$};
%% coordinates of the plot area
\gpdefrectangularnode{gp plot 1}{\pgfpoint{1.196cm}{0.985cm}}{\pgfpoint{7.947cm}{5.631cm}}
\end{tikzpicture}
%% gnuplot variables
} & 
\resizebox{0.5\linewidth}{!}{\tikzsetnextfilename{fast_coplanar_a_crsol_6}\begin{tikzpicture}[gnuplot]
%% generated with GNUPLOT 4.6p4 (Lua 5.1; terminal rev. 99, script rev. 100)
%% Sat 02 Aug 2014 10:08:43 AM EDT
\path (0.000,0.000) rectangle (8.500,6.000);
\gpfill{rgb color={1.000,1.000,1.000}} (1.196,0.985)--(7.946,0.985)--(7.946,5.630)--(1.196,5.630)--cycle;
\gpcolor{color=gp lt color border}
\gpsetlinetype{gp lt border}
\gpsetlinewidth{1.00}
\draw[gp path] (1.196,0.985)--(1.196,5.630)--(7.946,5.630)--(7.946,0.985)--cycle;
\gpcolor{color=gp lt color axes}
\gpsetlinetype{gp lt axes}
\gpsetlinewidth{2.00}
\draw[gp path] (1.196,0.985)--(7.947,0.985);
\gpcolor{color=gp lt color border}
\gpsetlinetype{gp lt border}
\draw[gp path] (1.196,0.985)--(1.268,0.985);
\draw[gp path] (7.947,0.985)--(7.875,0.985);
\gpcolor{rgb color={0.000,0.000,0.000}}
\node[gp node right,font={\fontsize{10pt}{12pt}\selectfont}] at (1.012,0.985) {-0.4};
\gpcolor{color=gp lt color axes}
\gpsetlinetype{gp lt axes}
\draw[gp path] (1.196,1.759)--(7.947,1.759);
\gpcolor{color=gp lt color border}
\gpsetlinetype{gp lt border}
\draw[gp path] (1.196,1.759)--(1.268,1.759);
\draw[gp path] (7.947,1.759)--(7.875,1.759);
\gpcolor{rgb color={0.000,0.000,0.000}}
\node[gp node right,font={\fontsize{10pt}{12pt}\selectfont}] at (1.012,1.759) {-0.2};
\gpcolor{color=gp lt color axes}
\gpsetlinetype{gp lt axes}
\draw[gp path] (1.196,2.534)--(7.947,2.534);
\gpcolor{color=gp lt color border}
\gpsetlinetype{gp lt border}
\draw[gp path] (1.196,2.534)--(1.268,2.534);
\draw[gp path] (7.947,2.534)--(7.875,2.534);
\gpcolor{rgb color={0.000,0.000,0.000}}
\node[gp node right,font={\fontsize{10pt}{12pt}\selectfont}] at (1.012,2.534) {0};
\gpcolor{color=gp lt color axes}
\gpsetlinetype{gp lt axes}
\draw[gp path] (1.196,3.308)--(7.947,3.308);
\gpcolor{color=gp lt color border}
\gpsetlinetype{gp lt border}
\draw[gp path] (1.196,3.308)--(1.268,3.308);
\draw[gp path] (7.947,3.308)--(7.875,3.308);
\gpcolor{rgb color={0.000,0.000,0.000}}
\node[gp node right,font={\fontsize{10pt}{12pt}\selectfont}] at (1.012,3.308) {0.2};
\gpcolor{color=gp lt color axes}
\gpsetlinetype{gp lt axes}
\draw[gp path] (1.196,4.082)--(7.947,4.082);
\gpcolor{color=gp lt color border}
\gpsetlinetype{gp lt border}
\draw[gp path] (1.196,4.082)--(1.268,4.082);
\draw[gp path] (7.947,4.082)--(7.875,4.082);
\gpcolor{rgb color={0.000,0.000,0.000}}
\node[gp node right,font={\fontsize{10pt}{12pt}\selectfont}] at (1.012,4.082) {0.4};
\gpcolor{color=gp lt color axes}
\gpsetlinetype{gp lt axes}
\draw[gp path] (1.196,4.857)--(7.947,4.857);
\gpcolor{color=gp lt color border}
\gpsetlinetype{gp lt border}
\draw[gp path] (1.196,4.857)--(1.268,4.857);
\draw[gp path] (7.947,4.857)--(7.875,4.857);
\gpcolor{rgb color={0.000,0.000,0.000}}
\node[gp node right,font={\fontsize{10pt}{12pt}\selectfont}] at (1.012,4.857) {0.6};
\gpcolor{color=gp lt color axes}
\gpsetlinetype{gp lt axes}
\draw[gp path] (1.196,5.631)--(7.947,5.631);
\gpcolor{color=gp lt color border}
\gpsetlinetype{gp lt border}
\draw[gp path] (1.196,5.631)--(1.268,5.631);
\draw[gp path] (7.947,5.631)--(7.875,5.631);
\gpcolor{rgb color={0.000,0.000,0.000}}
\node[gp node right,font={\fontsize{10pt}{12pt}\selectfont}] at (1.012,5.631) {0.8};
\gpcolor{color=gp lt color axes}
\gpsetlinetype{gp lt axes}
\draw[gp path] (1.196,0.985)--(1.196,5.631);
\gpcolor{color=gp lt color border}
\gpsetlinetype{gp lt border}
\draw[gp path] (1.196,0.985)--(1.196,1.057);
\draw[gp path] (1.196,5.631)--(1.196,5.559);
\gpcolor{rgb color={0.000,0.000,0.000}}
\node[gp node center,font={\fontsize{10pt}{12pt}\selectfont}] at (1.196,0.677) {0.3};
\gpcolor{color=gp lt color axes}
\gpsetlinetype{gp lt axes}
\draw[gp path] (2.494,0.985)--(2.494,5.631);
\gpcolor{color=gp lt color border}
\gpsetlinetype{gp lt border}
\draw[gp path] (2.494,0.985)--(2.494,1.057);
\draw[gp path] (2.494,5.631)--(2.494,5.559);
\gpcolor{rgb color={0.000,0.000,0.000}}
\node[gp node center,font={\fontsize{10pt}{12pt}\selectfont}] at (2.494,0.677) {0.35};
\gpcolor{color=gp lt color axes}
\gpsetlinetype{gp lt axes}
\draw[gp path] (3.793,0.985)--(3.793,5.631);
\gpcolor{color=gp lt color border}
\gpsetlinetype{gp lt border}
\draw[gp path] (3.793,0.985)--(3.793,1.057);
\draw[gp path] (3.793,5.631)--(3.793,5.559);
\gpcolor{rgb color={0.000,0.000,0.000}}
\node[gp node center,font={\fontsize{10pt}{12pt}\selectfont}] at (3.793,0.677) {0.4};
\gpcolor{color=gp lt color axes}
\gpsetlinetype{gp lt axes}
\draw[gp path] (5.091,0.985)--(5.091,5.631);
\gpcolor{color=gp lt color border}
\gpsetlinetype{gp lt border}
\draw[gp path] (5.091,0.985)--(5.091,1.057);
\draw[gp path] (5.091,5.631)--(5.091,5.559);
\gpcolor{rgb color={0.000,0.000,0.000}}
\node[gp node center,font={\fontsize{10pt}{12pt}\selectfont}] at (5.091,0.677) {0.45};
\gpcolor{color=gp lt color axes}
\gpsetlinetype{gp lt axes}
\draw[gp path] (6.389,0.985)--(6.389,5.631);
\gpcolor{color=gp lt color border}
\gpsetlinetype{gp lt border}
\draw[gp path] (6.389,0.985)--(6.389,1.057);
\draw[gp path] (6.389,5.631)--(6.389,5.559);
\gpcolor{rgb color={0.000,0.000,0.000}}
\node[gp node center,font={\fontsize{10pt}{12pt}\selectfont}] at (6.389,0.677) {0.5};
\gpcolor{color=gp lt color axes}
\gpsetlinetype{gp lt axes}
\draw[gp path] (7.687,0.985)--(7.687,5.631);
\gpcolor{color=gp lt color border}
\gpsetlinetype{gp lt border}
\draw[gp path] (7.687,0.985)--(7.687,1.057);
\draw[gp path] (7.687,5.631)--(7.687,5.559);
\gpcolor{rgb color={0.000,0.000,0.000}}
\node[gp node center,font={\fontsize{10pt}{12pt}\selectfont}] at (7.687,0.677) {0.55};
\gpcolor{color=gp lt color border}
\draw[gp path] (1.196,5.631)--(1.196,0.985)--(7.947,0.985)--(7.947,5.631)--cycle;
\gpcolor{rgb color={0.000,0.000,0.000}}
\node[gp node center,font={\fontsize{10pt}{12pt}\selectfont}] at (4.571,0.215) {\large $x$};
\gpcolor{rgb color={0.502,0.502,0.502}}
\gpsetlinewidth{0.50}
\gpsetpointsize{2.67}
\gppoint{gp mark 7}{(1.210,4.952)}
\gppoint{gp mark 7}{(1.223,4.943)}
\gppoint{gp mark 7}{(1.235,4.933)}
\gppoint{gp mark 7}{(1.248,4.924)}
\gppoint{gp mark 7}{(1.261,4.914)}
\gppoint{gp mark 7}{(1.273,4.905)}
\gppoint{gp mark 7}{(1.286,4.895)}
\gppoint{gp mark 7}{(1.299,4.885)}
\gppoint{gp mark 7}{(1.311,4.876)}
\gppoint{gp mark 7}{(1.324,4.866)}
\gppoint{gp mark 7}{(1.337,4.856)}
\gppoint{gp mark 7}{(1.349,4.847)}
\gppoint{gp mark 7}{(1.362,4.837)}
\gppoint{gp mark 7}{(1.375,4.827)}
\gppoint{gp mark 7}{(1.387,4.818)}
\gppoint{gp mark 7}{(1.400,4.808)}
\gppoint{gp mark 7}{(1.413,4.798)}
\gppoint{gp mark 7}{(1.425,4.788)}
\gppoint{gp mark 7}{(1.438,4.778)}
\gppoint{gp mark 7}{(1.451,4.769)}
\gppoint{gp mark 7}{(1.464,4.759)}
\gppoint{gp mark 7}{(1.476,4.749)}
\gppoint{gp mark 7}{(1.489,4.739)}
\gppoint{gp mark 7}{(1.502,4.729)}
\gppoint{gp mark 7}{(1.514,4.719)}
\gppoint{gp mark 7}{(1.527,4.709)}
\gppoint{gp mark 7}{(1.540,4.699)}
\gppoint{gp mark 7}{(1.552,4.689)}
\gppoint{gp mark 7}{(1.565,4.679)}
\gppoint{gp mark 7}{(1.578,4.669)}
\gppoint{gp mark 7}{(1.590,4.659)}
\gppoint{gp mark 7}{(1.603,4.649)}
\gppoint{gp mark 7}{(1.616,4.639)}
\gppoint{gp mark 7}{(1.628,4.629)}
\gppoint{gp mark 7}{(1.641,4.619)}
\gppoint{gp mark 7}{(1.654,4.608)}
\gppoint{gp mark 7}{(1.666,4.598)}
\gppoint{gp mark 7}{(1.679,4.588)}
\gppoint{gp mark 7}{(1.692,4.578)}
\gppoint{gp mark 7}{(1.704,4.567)}
\gppoint{gp mark 7}{(1.717,4.557)}
\gppoint{gp mark 7}{(1.730,4.547)}
\gppoint{gp mark 7}{(1.742,4.536)}
\gppoint{gp mark 7}{(1.755,4.526)}
\gppoint{gp mark 7}{(1.768,4.515)}
\gppoint{gp mark 7}{(1.780,4.505)}
\gppoint{gp mark 7}{(1.793,4.495)}
\gppoint{gp mark 7}{(1.806,4.484)}
\gppoint{gp mark 7}{(1.819,4.473)}
\gppoint{gp mark 7}{(1.831,4.463)}
\gppoint{gp mark 7}{(1.844,4.452)}
\gppoint{gp mark 7}{(1.857,4.442)}
\gppoint{gp mark 7}{(1.869,4.431)}
\gppoint{gp mark 7}{(1.882,4.420)}
\gppoint{gp mark 7}{(1.895,4.409)}
\gppoint{gp mark 7}{(1.907,4.399)}
\gppoint{gp mark 7}{(1.920,4.388)}
\gppoint{gp mark 7}{(1.933,4.377)}
\gppoint{gp mark 7}{(1.945,4.366)}
\gppoint{gp mark 7}{(1.958,4.355)}
\gppoint{gp mark 7}{(1.971,4.344)}
\gppoint{gp mark 7}{(1.983,4.333)}
\gppoint{gp mark 7}{(1.996,4.322)}
\gppoint{gp mark 7}{(2.009,4.311)}
\gppoint{gp mark 7}{(2.021,4.300)}
\gppoint{gp mark 7}{(2.034,4.288)}
\gppoint{gp mark 7}{(2.047,4.277)}
\gppoint{gp mark 7}{(2.059,4.266)}
\gppoint{gp mark 7}{(2.072,4.255)}
\gppoint{gp mark 7}{(2.085,4.243)}
\gppoint{gp mark 7}{(2.097,4.232)}
\gppoint{gp mark 7}{(2.110,4.220)}
\gppoint{gp mark 7}{(2.123,4.209)}
\gppoint{gp mark 7}{(2.135,4.197)}
\gppoint{gp mark 7}{(2.148,4.186)}
\gppoint{gp mark 7}{(2.161,4.174)}
\gppoint{gp mark 7}{(2.174,4.162)}
\gppoint{gp mark 7}{(2.186,4.150)}
\gppoint{gp mark 7}{(2.199,4.139)}
\gppoint{gp mark 7}{(2.212,4.127)}
\gppoint{gp mark 7}{(2.224,4.115)}
\gppoint{gp mark 7}{(2.237,4.103)}
\gppoint{gp mark 7}{(2.250,4.090)}
\gppoint{gp mark 7}{(2.262,4.078)}
\gppoint{gp mark 7}{(2.275,4.066)}
\gppoint{gp mark 7}{(2.288,4.054)}
\gppoint{gp mark 7}{(2.300,4.041)}
\gppoint{gp mark 7}{(2.313,4.029)}
\gppoint{gp mark 7}{(2.326,4.016)}
\gppoint{gp mark 7}{(2.338,4.004)}
\gppoint{gp mark 7}{(2.351,3.991)}
\gppoint{gp mark 7}{(2.364,3.978)}
\gppoint{gp mark 7}{(2.376,3.966)}
\gppoint{gp mark 7}{(2.389,3.953)}
\gppoint{gp mark 7}{(2.402,3.940)}
\gppoint{gp mark 7}{(2.414,3.927)}
\gppoint{gp mark 7}{(2.427,3.913)}
\gppoint{gp mark 7}{(2.440,3.900)}
\gppoint{gp mark 7}{(2.452,3.887)}
\gppoint{gp mark 7}{(2.465,3.873)}
\gppoint{gp mark 7}{(2.478,3.860)}
\gppoint{gp mark 7}{(2.490,3.846)}
\gppoint{gp mark 7}{(2.503,3.832)}
\gppoint{gp mark 7}{(2.516,3.818)}
\gppoint{gp mark 7}{(2.529,3.804)}
\gppoint{gp mark 7}{(2.541,3.790)}
\gppoint{gp mark 7}{(2.554,3.776)}
\gppoint{gp mark 7}{(2.567,3.762)}
\gppoint{gp mark 7}{(2.579,3.747)}
\gppoint{gp mark 7}{(2.592,3.733)}
\gppoint{gp mark 7}{(2.605,3.718)}
\gppoint{gp mark 7}{(2.617,3.703)}
\gppoint{gp mark 7}{(2.630,3.688)}
\gppoint{gp mark 7}{(2.643,3.673)}
\gppoint{gp mark 7}{(2.655,3.658)}
\gppoint{gp mark 7}{(2.668,3.642)}
\gppoint{gp mark 7}{(2.681,3.627)}
\gppoint{gp mark 7}{(2.693,3.611)}
\gppoint{gp mark 7}{(2.706,3.595)}
\gppoint{gp mark 7}{(2.719,3.580)}
\gppoint{gp mark 7}{(2.731,3.564)}
\gppoint{gp mark 7}{(2.744,3.547)}
\gppoint{gp mark 7}{(2.757,3.531)}
\gppoint{gp mark 7}{(2.769,3.515)}
\gppoint{gp mark 7}{(2.782,3.499)}
\gppoint{gp mark 7}{(2.795,3.483)}
\gppoint{gp mark 7}{(2.807,3.467)}
\gppoint{gp mark 7}{(2.820,3.452)}
\gppoint{gp mark 7}{(2.833,3.438)}
\gppoint{gp mark 7}{(2.845,3.425)}
\gppoint{gp mark 7}{(2.858,3.413)}
\gppoint{gp mark 7}{(2.871,3.403)}
\gppoint{gp mark 7}{(2.883,3.393)}
\gppoint{gp mark 7}{(2.896,3.380)}
\gppoint{gp mark 7}{(2.909,3.349)}
\gppoint{gp mark 7}{(2.922,3.249)}
\gppoint{gp mark 7}{(2.934,2.904)}
\gppoint{gp mark 7}{(2.947,2.044)}
\gppoint{gp mark 7}{(2.960,1.310)}
\gppoint{gp mark 7}{(2.972,1.202)}
\gppoint{gp mark 7}{(2.985,1.193)}
\gppoint{gp mark 7}{(2.998,1.191)}
\gppoint{gp mark 7}{(3.010,1.191)}
\gppoint{gp mark 7}{(3.023,1.192)}
\gppoint{gp mark 7}{(3.036,1.192)}
\gppoint{gp mark 7}{(3.048,1.192)}
\gppoint{gp mark 7}{(3.061,1.192)}
\gppoint{gp mark 7}{(3.074,1.193)}
\gppoint{gp mark 7}{(3.086,1.193)}
\gppoint{gp mark 7}{(3.099,1.193)}
\gppoint{gp mark 7}{(3.112,1.194)}
\gppoint{gp mark 7}{(3.124,1.194)}
\gppoint{gp mark 7}{(3.137,1.193)}
\gppoint{gp mark 7}{(3.150,1.193)}
\gppoint{gp mark 7}{(3.162,1.192)}
\gppoint{gp mark 7}{(3.175,1.192)}
\gppoint{gp mark 7}{(3.188,1.192)}
\gppoint{gp mark 7}{(3.200,1.192)}
\gppoint{gp mark 7}{(3.213,1.192)}
\gppoint{gp mark 7}{(3.226,1.191)}
\gppoint{gp mark 7}{(3.238,1.192)}
\gppoint{gp mark 7}{(3.251,1.192)}
\gppoint{gp mark 7}{(3.264,1.193)}
\gppoint{gp mark 7}{(3.277,1.193)}
\gppoint{gp mark 7}{(3.289,1.193)}
\gppoint{gp mark 7}{(3.302,1.193)}
\gppoint{gp mark 7}{(3.315,1.193)}
\gppoint{gp mark 7}{(3.327,1.193)}
\gppoint{gp mark 7}{(3.340,1.193)}
\gppoint{gp mark 7}{(3.353,1.192)}
\gppoint{gp mark 7}{(3.365,1.192)}
\gppoint{gp mark 7}{(3.378,1.191)}
\gppoint{gp mark 7}{(3.391,1.191)}
\gppoint{gp mark 7}{(3.403,1.192)}
\gppoint{gp mark 7}{(3.416,1.192)}
\gppoint{gp mark 7}{(3.429,1.192)}
\gppoint{gp mark 7}{(3.441,1.193)}
\gppoint{gp mark 7}{(3.454,1.193)}
\gppoint{gp mark 7}{(3.467,1.193)}
\gppoint{gp mark 7}{(3.479,1.193)}
\gppoint{gp mark 7}{(3.492,1.193)}
\gppoint{gp mark 7}{(3.505,1.192)}
\gppoint{gp mark 7}{(3.517,1.191)}
\gppoint{gp mark 7}{(3.530,1.191)}
\gppoint{gp mark 7}{(3.543,1.191)}
\gppoint{gp mark 7}{(3.555,1.191)}
\gppoint{gp mark 7}{(3.568,1.191)}
\gppoint{gp mark 7}{(3.581,1.191)}
\gppoint{gp mark 7}{(3.593,1.192)}
\gppoint{gp mark 7}{(3.606,1.193)}
\gppoint{gp mark 7}{(3.619,1.194)}
\gppoint{gp mark 7}{(3.632,1.194)}
\gppoint{gp mark 7}{(3.644,1.194)}
\gppoint{gp mark 7}{(3.657,1.194)}
\gppoint{gp mark 7}{(3.670,1.193)}
\gppoint{gp mark 7}{(3.682,1.192)}
\gppoint{gp mark 7}{(3.695,1.192)}
\gppoint{gp mark 7}{(3.708,1.191)}
\gppoint{gp mark 7}{(3.720,1.194)}
\gppoint{gp mark 7}{(3.733,1.239)}
\gppoint{gp mark 7}{(3.746,1.435)}
\gppoint{gp mark 7}{(3.758,1.592)}
\gppoint{gp mark 7}{(3.771,1.609)}
\gppoint{gp mark 7}{(3.784,1.610)}
\gppoint{gp mark 7}{(3.796,1.609)}
\gppoint{gp mark 7}{(3.809,1.610)}
\gppoint{gp mark 7}{(3.822,1.610)}
\gppoint{gp mark 7}{(3.834,1.610)}
\gppoint{gp mark 7}{(3.847,1.609)}
\gppoint{gp mark 7}{(3.860,1.609)}
\gppoint{gp mark 7}{(3.872,1.609)}
\gppoint{gp mark 7}{(3.885,1.609)}
\gppoint{gp mark 7}{(3.898,1.609)}
\gppoint{gp mark 7}{(3.910,1.609)}
\gppoint{gp mark 7}{(3.923,1.609)}
\gppoint{gp mark 7}{(3.936,1.609)}
\gppoint{gp mark 7}{(3.948,1.609)}
\gppoint{gp mark 7}{(3.961,1.609)}
\gppoint{gp mark 7}{(3.974,1.609)}
\gppoint{gp mark 7}{(3.987,1.609)}
\gppoint{gp mark 7}{(3.999,1.609)}
\gppoint{gp mark 7}{(4.012,1.609)}
\gppoint{gp mark 7}{(4.025,1.609)}
\gppoint{gp mark 7}{(4.037,1.609)}
\gppoint{gp mark 7}{(4.050,1.609)}
\gppoint{gp mark 7}{(4.063,1.609)}
\gppoint{gp mark 7}{(4.075,1.609)}
\gppoint{gp mark 7}{(4.088,1.609)}
\gppoint{gp mark 7}{(4.101,1.609)}
\gppoint{gp mark 7}{(4.113,1.609)}
\gppoint{gp mark 7}{(4.126,1.609)}
\gppoint{gp mark 7}{(4.139,1.609)}
\gppoint{gp mark 7}{(4.151,1.609)}
\gppoint{gp mark 7}{(4.164,1.609)}
\gppoint{gp mark 7}{(4.177,1.609)}
\gppoint{gp mark 7}{(4.189,1.609)}
\gppoint{gp mark 7}{(4.202,1.609)}
\gppoint{gp mark 7}{(4.215,1.609)}
\gppoint{gp mark 7}{(4.227,1.609)}
\gppoint{gp mark 7}{(4.240,1.609)}
\gppoint{gp mark 7}{(4.253,1.609)}
\gppoint{gp mark 7}{(4.265,1.609)}
\gppoint{gp mark 7}{(4.278,1.609)}
\gppoint{gp mark 7}{(4.291,1.609)}
\gppoint{gp mark 7}{(4.303,1.609)}
\gppoint{gp mark 7}{(4.316,1.609)}
\gppoint{gp mark 7}{(4.329,1.609)}
\gppoint{gp mark 7}{(4.342,1.609)}
\gppoint{gp mark 7}{(4.354,1.609)}
\gppoint{gp mark 7}{(4.367,1.609)}
\gppoint{gp mark 7}{(4.380,1.609)}
\gppoint{gp mark 7}{(4.392,1.609)}
\gppoint{gp mark 7}{(4.405,1.609)}
\gppoint{gp mark 7}{(4.418,1.609)}
\gppoint{gp mark 7}{(4.430,1.609)}
\gppoint{gp mark 7}{(4.443,1.609)}
\gppoint{gp mark 7}{(4.456,1.609)}
\gppoint{gp mark 7}{(4.468,1.609)}
\gppoint{gp mark 7}{(4.481,1.609)}
\gppoint{gp mark 7}{(4.494,1.609)}
\gppoint{gp mark 7}{(4.506,1.609)}
\gppoint{gp mark 7}{(4.519,1.609)}
\gppoint{gp mark 7}{(4.532,1.609)}
\gppoint{gp mark 7}{(4.544,1.609)}
\gppoint{gp mark 7}{(4.557,1.609)}
\gppoint{gp mark 7}{(4.570,1.609)}
\gppoint{gp mark 7}{(4.582,1.609)}
\gppoint{gp mark 7}{(4.595,1.609)}
\gppoint{gp mark 7}{(4.608,1.609)}
\gppoint{gp mark 7}{(4.620,1.609)}
\gppoint{gp mark 7}{(4.633,1.609)}
\gppoint{gp mark 7}{(4.646,1.609)}
\gppoint{gp mark 7}{(4.658,1.609)}
\gppoint{gp mark 7}{(4.671,1.609)}
\gppoint{gp mark 7}{(4.684,1.609)}
\gppoint{gp mark 7}{(4.697,1.609)}
\gppoint{gp mark 7}{(4.709,1.609)}
\gppoint{gp mark 7}{(4.722,1.609)}
\gppoint{gp mark 7}{(4.735,1.609)}
\gppoint{gp mark 7}{(4.747,1.609)}
\gppoint{gp mark 7}{(4.760,1.609)}
\gppoint{gp mark 7}{(4.773,1.609)}
\gppoint{gp mark 7}{(4.785,1.609)}
\gppoint{gp mark 7}{(4.798,1.609)}
\gppoint{gp mark 7}{(4.811,1.609)}
\gppoint{gp mark 7}{(4.823,1.609)}
\gppoint{gp mark 7}{(4.836,1.609)}
\gppoint{gp mark 7}{(4.849,1.609)}
\gppoint{gp mark 7}{(4.861,1.609)}
\gppoint{gp mark 7}{(4.874,1.609)}
\gppoint{gp mark 7}{(4.887,1.609)}
\gppoint{gp mark 7}{(4.899,1.609)}
\gppoint{gp mark 7}{(4.912,1.609)}
\gppoint{gp mark 7}{(4.925,1.609)}
\gppoint{gp mark 7}{(4.937,1.609)}
\gppoint{gp mark 7}{(4.950,1.609)}
\gppoint{gp mark 7}{(4.963,1.609)}
\gppoint{gp mark 7}{(4.975,1.609)}
\gppoint{gp mark 7}{(4.988,1.609)}
\gppoint{gp mark 7}{(5.001,1.609)}
\gppoint{gp mark 7}{(5.013,1.609)}
\gppoint{gp mark 7}{(5.026,1.609)}
\gppoint{gp mark 7}{(5.039,1.609)}
\gppoint{gp mark 7}{(5.052,1.609)}
\gppoint{gp mark 7}{(5.064,1.609)}
\gppoint{gp mark 7}{(5.077,1.609)}
\gppoint{gp mark 7}{(5.090,1.609)}
\gppoint{gp mark 7}{(5.102,1.609)}
\gppoint{gp mark 7}{(5.115,1.609)}
\gppoint{gp mark 7}{(5.128,1.609)}
\gppoint{gp mark 7}{(5.140,1.609)}
\gppoint{gp mark 7}{(5.153,1.609)}
\gppoint{gp mark 7}{(5.166,1.609)}
\gppoint{gp mark 7}{(5.178,1.609)}
\gppoint{gp mark 7}{(5.191,1.609)}
\gppoint{gp mark 7}{(5.204,1.609)}
\gppoint{gp mark 7}{(5.216,1.609)}
\gppoint{gp mark 7}{(5.229,1.609)}
\gppoint{gp mark 7}{(5.242,1.609)}
\gppoint{gp mark 7}{(5.254,1.609)}
\gppoint{gp mark 7}{(5.267,1.609)}
\gppoint{gp mark 7}{(5.280,1.609)}
\gppoint{gp mark 7}{(5.292,1.609)}
\gppoint{gp mark 7}{(5.305,1.609)}
\gppoint{gp mark 7}{(5.318,1.609)}
\gppoint{gp mark 7}{(5.330,1.609)}
\gppoint{gp mark 7}{(5.343,1.609)}
\gppoint{gp mark 7}{(5.356,1.609)}
\gppoint{gp mark 7}{(5.368,1.609)}
\gppoint{gp mark 7}{(5.381,1.609)}
\gppoint{gp mark 7}{(5.394,1.609)}
\gppoint{gp mark 7}{(5.407,1.609)}
\gppoint{gp mark 7}{(5.419,1.609)}
\gppoint{gp mark 7}{(5.432,1.609)}
\gppoint{gp mark 7}{(5.445,1.609)}
\gppoint{gp mark 7}{(5.457,1.609)}
\gppoint{gp mark 7}{(5.470,1.609)}
\gppoint{gp mark 7}{(5.483,1.609)}
\gppoint{gp mark 7}{(5.495,1.609)}
\gppoint{gp mark 7}{(5.508,1.609)}
\gppoint{gp mark 7}{(5.521,1.609)}
\gppoint{gp mark 7}{(5.533,1.609)}
\gppoint{gp mark 7}{(5.546,1.609)}
\gppoint{gp mark 7}{(5.559,1.609)}
\gppoint{gp mark 7}{(5.571,1.609)}
\gppoint{gp mark 7}{(5.584,1.609)}
\gppoint{gp mark 7}{(5.597,1.609)}
\gppoint{gp mark 7}{(5.609,1.609)}
\gppoint{gp mark 7}{(5.622,1.609)}
\gppoint{gp mark 7}{(5.635,1.609)}
\gppoint{gp mark 7}{(5.647,1.609)}
\gppoint{gp mark 7}{(5.660,1.609)}
\gppoint{gp mark 7}{(5.673,1.609)}
\gppoint{gp mark 7}{(5.685,1.609)}
\gppoint{gp mark 7}{(5.698,1.609)}
\gppoint{gp mark 7}{(5.711,1.609)}
\gppoint{gp mark 7}{(5.723,1.609)}
\gppoint{gp mark 7}{(5.736,1.609)}
\gppoint{gp mark 7}{(5.749,1.609)}
\gppoint{gp mark 7}{(5.761,1.609)}
\gppoint{gp mark 7}{(5.774,1.609)}
\gppoint{gp mark 7}{(5.787,1.609)}
\gppoint{gp mark 7}{(5.800,1.609)}
\gppoint{gp mark 7}{(5.812,1.609)}
\gppoint{gp mark 7}{(5.825,1.609)}
\gppoint{gp mark 7}{(5.838,1.609)}
\gppoint{gp mark 7}{(5.850,1.609)}
\gppoint{gp mark 7}{(5.863,1.609)}
\gppoint{gp mark 7}{(5.876,1.609)}
\gppoint{gp mark 7}{(5.888,1.609)}
\gppoint{gp mark 7}{(5.901,1.609)}
\gppoint{gp mark 7}{(5.914,1.609)}
\gppoint{gp mark 7}{(5.926,1.609)}
\gppoint{gp mark 7}{(5.939,1.609)}
\gppoint{gp mark 7}{(5.952,1.609)}
\gppoint{gp mark 7}{(5.964,1.609)}
\gppoint{gp mark 7}{(5.977,1.609)}
\gppoint{gp mark 7}{(5.990,1.609)}
\gppoint{gp mark 7}{(6.002,1.609)}
\gppoint{gp mark 7}{(6.015,1.609)}
\gppoint{gp mark 7}{(6.028,1.609)}
\gppoint{gp mark 7}{(6.040,1.609)}
\gppoint{gp mark 7}{(6.053,1.609)}
\gppoint{gp mark 7}{(6.066,1.609)}
\gppoint{gp mark 7}{(6.078,1.609)}
\gppoint{gp mark 7}{(6.091,1.609)}
\gppoint{gp mark 7}{(6.104,1.609)}
\gppoint{gp mark 7}{(6.116,1.609)}
\gppoint{gp mark 7}{(6.129,1.609)}
\gppoint{gp mark 7}{(6.142,1.609)}
\gppoint{gp mark 7}{(6.155,1.609)}
\gppoint{gp mark 7}{(6.167,1.609)}
\gppoint{gp mark 7}{(6.180,1.609)}
\gppoint{gp mark 7}{(6.193,1.609)}
\gppoint{gp mark 7}{(6.205,1.609)}
\gppoint{gp mark 7}{(6.218,1.609)}
\gppoint{gp mark 7}{(6.231,1.610)}
\gppoint{gp mark 7}{(6.243,1.610)}
\gppoint{gp mark 7}{(6.256,1.610)}
\gppoint{gp mark 7}{(6.269,1.609)}
\gppoint{gp mark 7}{(6.281,1.609)}
\gppoint{gp mark 7}{(6.294,1.609)}
\gppoint{gp mark 7}{(6.307,1.609)}
\gppoint{gp mark 7}{(6.319,1.609)}
\gppoint{gp mark 7}{(6.332,1.609)}
\gppoint{gp mark 7}{(6.345,1.609)}
\gppoint{gp mark 7}{(6.357,1.609)}
\gppoint{gp mark 7}{(6.370,1.609)}
\gppoint{gp mark 7}{(6.383,1.609)}
\gppoint{gp mark 7}{(6.395,1.609)}
\gppoint{gp mark 7}{(6.408,1.609)}
\gppoint{gp mark 7}{(6.421,1.609)}
\gppoint{gp mark 7}{(6.433,1.609)}
\gppoint{gp mark 7}{(6.446,1.609)}
\gppoint{gp mark 7}{(6.459,1.609)}
\gppoint{gp mark 7}{(6.471,1.609)}
\gppoint{gp mark 7}{(6.484,1.609)}
\gppoint{gp mark 7}{(6.497,1.609)}
\gppoint{gp mark 7}{(6.510,1.609)}
\gppoint{gp mark 7}{(6.522,1.609)}
\gppoint{gp mark 7}{(6.535,1.609)}
\gppoint{gp mark 7}{(6.548,1.609)}
\gppoint{gp mark 7}{(6.560,1.609)}
\gppoint{gp mark 7}{(6.573,1.609)}
\gppoint{gp mark 7}{(6.586,1.609)}
\gppoint{gp mark 7}{(6.598,1.609)}
\gppoint{gp mark 7}{(6.611,1.609)}
\gppoint{gp mark 7}{(6.624,1.609)}
\gppoint{gp mark 7}{(6.636,1.609)}
\gppoint{gp mark 7}{(6.649,1.609)}
\gppoint{gp mark 7}{(6.662,1.609)}
\gppoint{gp mark 7}{(6.674,1.609)}
\gppoint{gp mark 7}{(6.687,1.609)}
\gppoint{gp mark 7}{(6.700,1.609)}
\gppoint{gp mark 7}{(6.712,1.609)}
\gppoint{gp mark 7}{(6.725,1.609)}
\gppoint{gp mark 7}{(6.738,1.609)}
\gppoint{gp mark 7}{(6.750,1.609)}
\gppoint{gp mark 7}{(6.763,1.610)}
\gppoint{gp mark 7}{(6.776,1.610)}
\gppoint{gp mark 7}{(6.788,1.609)}
\gppoint{gp mark 7}{(6.801,1.609)}
\gppoint{gp mark 7}{(6.814,1.609)}
\gppoint{gp mark 7}{(6.826,1.609)}
\gppoint{gp mark 7}{(6.839,1.609)}
\gppoint{gp mark 7}{(6.852,1.609)}
\gppoint{gp mark 7}{(6.865,1.609)}
\gppoint{gp mark 7}{(6.877,1.609)}
\gppoint{gp mark 7}{(6.890,1.609)}
\gppoint{gp mark 7}{(6.903,1.609)}
\gppoint{gp mark 7}{(6.915,1.609)}
\gppoint{gp mark 7}{(6.928,1.609)}
\gppoint{gp mark 7}{(6.941,1.609)}
\gppoint{gp mark 7}{(6.953,1.609)}
\gppoint{gp mark 7}{(6.966,1.609)}
\gppoint{gp mark 7}{(6.979,1.609)}
\gppoint{gp mark 7}{(6.991,1.609)}
\gppoint{gp mark 7}{(7.004,1.609)}
\gppoint{gp mark 7}{(7.017,1.609)}
\gppoint{gp mark 7}{(7.029,1.609)}
\gppoint{gp mark 7}{(7.042,1.609)}
\gppoint{gp mark 7}{(7.055,1.609)}
\gppoint{gp mark 7}{(7.067,1.609)}
\gppoint{gp mark 7}{(7.080,1.609)}
\gppoint{gp mark 7}{(7.093,1.609)}
\gppoint{gp mark 7}{(7.105,1.609)}
\gppoint{gp mark 7}{(7.118,1.609)}
\gppoint{gp mark 7}{(7.131,1.609)}
\gppoint{gp mark 7}{(7.143,1.609)}
\gppoint{gp mark 7}{(7.156,1.609)}
\gppoint{gp mark 7}{(7.169,1.609)}
\gppoint{gp mark 7}{(7.181,1.609)}
\gppoint{gp mark 7}{(7.194,1.609)}
\gppoint{gp mark 7}{(7.207,1.609)}
\gppoint{gp mark 7}{(7.220,1.609)}
\gppoint{gp mark 7}{(7.232,1.609)}
\gppoint{gp mark 7}{(7.245,1.609)}
\gppoint{gp mark 7}{(7.258,1.609)}
\gppoint{gp mark 7}{(7.270,1.609)}
\gppoint{gp mark 7}{(7.283,1.609)}
\gppoint{gp mark 7}{(7.296,1.609)}
\gppoint{gp mark 7}{(7.308,1.609)}
\gppoint{gp mark 7}{(7.321,1.609)}
\gppoint{gp mark 7}{(7.334,1.609)}
\gppoint{gp mark 7}{(7.346,1.609)}
\gppoint{gp mark 7}{(7.359,1.609)}
\gppoint{gp mark 7}{(7.372,1.609)}
\gppoint{gp mark 7}{(7.384,1.609)}
\gppoint{gp mark 7}{(7.397,1.609)}
\gppoint{gp mark 7}{(7.410,1.609)}
\gppoint{gp mark 7}{(7.422,1.609)}
\gppoint{gp mark 7}{(7.435,1.609)}
\gppoint{gp mark 7}{(7.448,1.609)}
\gppoint{gp mark 7}{(7.460,1.609)}
\gppoint{gp mark 7}{(7.473,1.609)}
\gppoint{gp mark 7}{(7.486,1.609)}
\gppoint{gp mark 7}{(7.498,1.609)}
\gppoint{gp mark 7}{(7.511,1.609)}
\gppoint{gp mark 7}{(7.524,1.609)}
\gppoint{gp mark 7}{(7.536,1.609)}
\gppoint{gp mark 7}{(7.549,1.609)}
\gppoint{gp mark 7}{(7.562,1.609)}
\gppoint{gp mark 7}{(7.575,1.609)}
\gppoint{gp mark 7}{(7.587,1.609)}
\gppoint{gp mark 7}{(7.600,1.609)}
\gppoint{gp mark 7}{(7.613,1.609)}
\gppoint{gp mark 7}{(7.625,1.609)}
\gppoint{gp mark 7}{(7.638,1.609)}
\gppoint{gp mark 7}{(7.651,1.609)}
\gppoint{gp mark 7}{(7.663,1.609)}
\gppoint{gp mark 7}{(7.676,1.609)}
\gppoint{gp mark 7}{(7.689,1.609)}
\gppoint{gp mark 7}{(7.701,1.609)}
\gppoint{gp mark 7}{(7.714,1.609)}
\gppoint{gp mark 7}{(7.727,1.609)}
\gppoint{gp mark 7}{(7.739,1.609)}
\gppoint{gp mark 7}{(7.752,1.609)}
\gppoint{gp mark 7}{(7.765,1.609)}
\gppoint{gp mark 7}{(7.777,1.609)}
\gppoint{gp mark 7}{(7.790,1.609)}
\gppoint{gp mark 7}{(7.803,1.609)}
\gppoint{gp mark 7}{(7.815,1.609)}
\gppoint{gp mark 7}{(7.828,1.609)}
\gppoint{gp mark 7}{(7.841,1.609)}
\gppoint{gp mark 7}{(7.853,1.609)}
\gppoint{gp mark 7}{(7.866,1.609)}
\gppoint{gp mark 7}{(7.879,1.609)}
\gppoint{gp mark 7}{(7.891,1.609)}
\gppoint{gp mark 7}{(7.904,1.609)}
\gppoint{gp mark 7}{(7.917,1.609)}
\gppoint{gp mark 7}{(7.930,1.609)}
\gpcolor{rgb color={1.000,0.000,0.000}}
\gpsetpointsize{4.44}
\gppoint{gp mark 7}{(1.210,4.952)}
\gppoint{gp mark 7}{(1.223,4.943)}
\gppoint{gp mark 7}{(1.235,4.933)}
\gppoint{gp mark 7}{(1.248,4.924)}
\gppoint{gp mark 7}{(1.261,4.914)}
\gppoint{gp mark 7}{(1.273,4.905)}
\gppoint{gp mark 7}{(1.286,4.895)}
\gppoint{gp mark 7}{(1.299,4.885)}
\gppoint{gp mark 7}{(1.311,4.876)}
\gppoint{gp mark 7}{(1.324,4.866)}
\gppoint{gp mark 7}{(1.337,4.857)}
\gppoint{gp mark 7}{(1.349,4.847)}
\gppoint{gp mark 7}{(1.362,4.837)}
\gppoint{gp mark 7}{(1.375,4.828)}
\gppoint{gp mark 7}{(1.387,4.818)}
\gppoint{gp mark 7}{(1.400,4.808)}
\gppoint{gp mark 7}{(1.413,4.798)}
\gppoint{gp mark 7}{(1.425,4.789)}
\gppoint{gp mark 7}{(1.438,4.779)}
\gppoint{gp mark 7}{(1.451,4.769)}
\gppoint{gp mark 7}{(1.464,4.759)}
\gppoint{gp mark 7}{(1.476,4.749)}
\gppoint{gp mark 7}{(1.489,4.740)}
\gppoint{gp mark 7}{(1.502,4.730)}
\gppoint{gp mark 7}{(1.514,4.720)}
\gppoint{gp mark 7}{(1.527,4.710)}
\gppoint{gp mark 7}{(1.540,4.700)}
\gppoint{gp mark 7}{(1.552,4.690)}
\gppoint{gp mark 7}{(1.565,4.680)}
\gppoint{gp mark 7}{(1.578,4.670)}
\gppoint{gp mark 7}{(1.590,4.660)}
\gppoint{gp mark 7}{(1.603,4.650)}
\gppoint{gp mark 7}{(1.616,4.640)}
\gppoint{gp mark 7}{(1.628,4.630)}
\gppoint{gp mark 7}{(1.641,4.620)}
\gppoint{gp mark 7}{(1.654,4.610)}
\gppoint{gp mark 7}{(1.666,4.599)}
\gppoint{gp mark 7}{(1.679,4.589)}
\gppoint{gp mark 7}{(1.692,4.579)}
\gppoint{gp mark 7}{(1.704,4.569)}
\gppoint{gp mark 7}{(1.717,4.558)}
\gppoint{gp mark 7}{(1.730,4.548)}
\gppoint{gp mark 7}{(1.742,4.538)}
\gppoint{gp mark 7}{(1.755,4.528)}
\gppoint{gp mark 7}{(1.768,4.517)}
\gppoint{gp mark 7}{(1.780,4.507)}
\gppoint{gp mark 7}{(1.793,4.496)}
\gppoint{gp mark 7}{(1.806,4.486)}
\gppoint{gp mark 7}{(1.819,4.475)}
\gppoint{gp mark 7}{(1.831,4.465)}
\gppoint{gp mark 7}{(1.844,4.454)}
\gppoint{gp mark 7}{(1.857,4.444)}
\gppoint{gp mark 7}{(1.869,4.433)}
\gppoint{gp mark 7}{(1.882,4.423)}
\gppoint{gp mark 7}{(1.895,4.412)}
\gppoint{gp mark 7}{(1.907,4.401)}
\gppoint{gp mark 7}{(1.920,4.390)}
\gppoint{gp mark 7}{(1.933,4.380)}
\gppoint{gp mark 7}{(1.945,4.369)}
\gppoint{gp mark 7}{(1.958,4.358)}
\gppoint{gp mark 7}{(1.971,4.347)}
\gppoint{gp mark 7}{(1.983,4.336)}
\gppoint{gp mark 7}{(1.996,4.325)}
\gppoint{gp mark 7}{(2.009,4.314)}
\gppoint{gp mark 7}{(2.021,4.303)}
\gppoint{gp mark 7}{(2.034,4.292)}
\gppoint{gp mark 7}{(2.047,4.281)}
\gppoint{gp mark 7}{(2.059,4.270)}
\gppoint{gp mark 7}{(2.072,4.259)}
\gppoint{gp mark 7}{(2.085,4.248)}
\gppoint{gp mark 7}{(2.097,4.236)}
\gppoint{gp mark 7}{(2.110,4.225)}
\gppoint{gp mark 7}{(2.123,4.214)}
\gppoint{gp mark 7}{(2.135,4.203)}
\gppoint{gp mark 7}{(2.148,4.191)}
\gppoint{gp mark 7}{(2.161,4.180)}
\gppoint{gp mark 7}{(2.174,4.168)}
\gppoint{gp mark 7}{(2.186,4.157)}
\gppoint{gp mark 7}{(2.199,4.145)}
\gppoint{gp mark 7}{(2.212,4.133)}
\gppoint{gp mark 7}{(2.224,4.122)}
\gppoint{gp mark 7}{(2.237,4.110)}
\gppoint{gp mark 7}{(2.250,4.098)}
\gppoint{gp mark 7}{(2.262,4.086)}
\gppoint{gp mark 7}{(2.275,4.074)}
\gppoint{gp mark 7}{(2.288,4.063)}
\gppoint{gp mark 7}{(2.300,4.051)}
\gppoint{gp mark 7}{(2.313,4.039)}
\gppoint{gp mark 7}{(2.326,4.027)}
\gppoint{gp mark 7}{(2.338,4.014)}
\gppoint{gp mark 7}{(2.351,4.002)}
\gppoint{gp mark 7}{(2.364,3.990)}
\gppoint{gp mark 7}{(2.376,3.978)}
\gppoint{gp mark 7}{(2.389,3.966)}
\gppoint{gp mark 7}{(2.402,3.954)}
\gppoint{gp mark 7}{(2.414,3.942)}
\gppoint{gp mark 7}{(2.427,3.930)}
\gppoint{gp mark 7}{(2.440,3.919)}
\gppoint{gp mark 7}{(2.452,3.908)}
\gppoint{gp mark 7}{(2.465,3.899)}
\gppoint{gp mark 7}{(2.478,3.892)}
\gppoint{gp mark 7}{(2.490,3.888)}
\gppoint{gp mark 7}{(2.503,3.887)}
\gppoint{gp mark 7}{(2.516,3.887)}
\gppoint{gp mark 7}{(2.529,3.886)}
\gppoint{gp mark 7}{(2.541,3.887)}
\gppoint{gp mark 7}{(2.554,3.887)}
\gppoint{gp mark 7}{(2.567,3.887)}
\gppoint{gp mark 7}{(2.579,3.889)}
\gppoint{gp mark 7}{(2.592,3.892)}
\gppoint{gp mark 7}{(2.605,3.895)}
\gppoint{gp mark 7}{(2.617,3.896)}
\gppoint{gp mark 7}{(2.630,3.896)}
\gppoint{gp mark 7}{(2.643,3.896)}
\gppoint{gp mark 7}{(2.655,3.896)}
\gppoint{gp mark 7}{(2.668,3.896)}
\gppoint{gp mark 7}{(2.681,3.897)}
\gppoint{gp mark 7}{(2.693,3.897)}
\gppoint{gp mark 7}{(2.706,3.898)}
\gppoint{gp mark 7}{(2.719,3.898)}
\gppoint{gp mark 7}{(2.731,3.898)}
\gppoint{gp mark 7}{(2.744,3.898)}
\gppoint{gp mark 7}{(2.757,3.898)}
\gppoint{gp mark 7}{(2.769,3.898)}
\gppoint{gp mark 7}{(2.782,3.898)}
\gppoint{gp mark 7}{(2.795,3.898)}
\gppoint{gp mark 7}{(2.807,3.899)}
\gppoint{gp mark 7}{(2.820,3.899)}
\gppoint{gp mark 7}{(2.833,3.899)}
\gppoint{gp mark 7}{(2.845,3.899)}
\gppoint{gp mark 7}{(2.858,3.898)}
\gppoint{gp mark 7}{(2.871,3.896)}
\gppoint{gp mark 7}{(2.883,3.883)}
\gppoint{gp mark 7}{(2.896,3.379)}
\gppoint{gp mark 7}{(2.909,1.863)}
\gppoint{gp mark 7}{(2.922,1.213)}
\gppoint{gp mark 7}{(2.934,1.174)}
\gppoint{gp mark 7}{(2.947,1.174)}
\gppoint{gp mark 7}{(2.960,1.170)}
\gppoint{gp mark 7}{(2.972,1.168)}
\gppoint{gp mark 7}{(2.985,1.168)}
\gppoint{gp mark 7}{(2.998,1.169)}
\gppoint{gp mark 7}{(3.010,1.168)}
\gppoint{gp mark 7}{(3.023,1.168)}
\gppoint{gp mark 7}{(3.036,1.168)}
\gppoint{gp mark 7}{(3.048,1.168)}
\gppoint{gp mark 7}{(3.061,1.168)}
\gppoint{gp mark 7}{(3.074,1.168)}
\gppoint{gp mark 7}{(3.086,1.168)}
\gppoint{gp mark 7}{(3.099,1.168)}
\gppoint{gp mark 7}{(3.112,1.169)}
\gppoint{gp mark 7}{(3.124,1.169)}
\gppoint{gp mark 7}{(3.137,1.169)}
\gppoint{gp mark 7}{(3.150,1.168)}
\gppoint{gp mark 7}{(3.162,1.168)}
\gppoint{gp mark 7}{(3.175,1.167)}
\gppoint{gp mark 7}{(3.188,1.167)}
\gppoint{gp mark 7}{(3.200,1.167)}
\gppoint{gp mark 7}{(3.213,1.167)}
\gppoint{gp mark 7}{(3.226,1.167)}
\gppoint{gp mark 7}{(3.238,1.167)}
\gppoint{gp mark 7}{(3.251,1.168)}
\gppoint{gp mark 7}{(3.264,1.169)}
\gppoint{gp mark 7}{(3.277,1.169)}
\gppoint{gp mark 7}{(3.289,1.169)}
\gppoint{gp mark 7}{(3.302,1.169)}
\gppoint{gp mark 7}{(3.315,1.168)}
\gppoint{gp mark 7}{(3.327,1.168)}
\gppoint{gp mark 7}{(3.340,1.167)}
\gppoint{gp mark 7}{(3.353,1.167)}
\gppoint{gp mark 7}{(3.365,1.167)}
\gppoint{gp mark 7}{(3.378,1.167)}
\gppoint{gp mark 7}{(3.391,1.167)}
\gppoint{gp mark 7}{(3.403,1.168)}
\gppoint{gp mark 7}{(3.416,1.169)}
\gppoint{gp mark 7}{(3.429,1.169)}
\gppoint{gp mark 7}{(3.441,1.169)}
\gppoint{gp mark 7}{(3.454,1.169)}
\gppoint{gp mark 7}{(3.467,1.169)}
\gppoint{gp mark 7}{(3.479,1.168)}
\gppoint{gp mark 7}{(3.492,1.167)}
\gppoint{gp mark 7}{(3.505,1.167)}
\gppoint{gp mark 7}{(3.517,1.167)}
\gppoint{gp mark 7}{(3.530,1.167)}
\gppoint{gp mark 7}{(3.543,1.167)}
\gppoint{gp mark 7}{(3.555,1.168)}
\gppoint{gp mark 7}{(3.568,1.169)}
\gppoint{gp mark 7}{(3.581,1.170)}
\gppoint{gp mark 7}{(3.593,1.170)}
\gppoint{gp mark 7}{(3.606,1.170)}
\gppoint{gp mark 7}{(3.619,1.169)}
\gppoint{gp mark 7}{(3.632,1.168)}
\gppoint{gp mark 7}{(3.644,1.167)}
\gppoint{gp mark 7}{(3.657,1.167)}
\gppoint{gp mark 7}{(3.670,1.167)}
\gppoint{gp mark 7}{(3.682,1.167)}
\gppoint{gp mark 7}{(3.695,1.167)}
\gppoint{gp mark 7}{(3.708,1.167)}
\gppoint{gp mark 7}{(3.720,1.168)}
\gppoint{gp mark 7}{(3.733,1.169)}
\gppoint{gp mark 7}{(3.746,1.170)}
\gppoint{gp mark 7}{(3.758,1.171)}
\gppoint{gp mark 7}{(3.771,1.178)}
\gppoint{gp mark 7}{(3.784,1.241)}
\gppoint{gp mark 7}{(3.796,1.479)}
\gppoint{gp mark 7}{(3.809,1.600)}
\gppoint{gp mark 7}{(3.822,1.610)}
\gppoint{gp mark 7}{(3.834,1.610)}
\gppoint{gp mark 7}{(3.847,1.610)}
\gppoint{gp mark 7}{(3.860,1.610)}
\gppoint{gp mark 7}{(3.872,1.610)}
\gppoint{gp mark 7}{(3.885,1.610)}
\gppoint{gp mark 7}{(3.898,1.610)}
\gppoint{gp mark 7}{(3.910,1.610)}
\gppoint{gp mark 7}{(3.923,1.610)}
\gppoint{gp mark 7}{(3.936,1.610)}
\gppoint{gp mark 7}{(3.948,1.610)}
\gppoint{gp mark 7}{(3.961,1.610)}
\gppoint{gp mark 7}{(3.974,1.610)}
\gppoint{gp mark 7}{(3.987,1.610)}
\gppoint{gp mark 7}{(3.999,1.610)}
\gppoint{gp mark 7}{(4.012,1.610)}
\gppoint{gp mark 7}{(4.025,1.610)}
\gppoint{gp mark 7}{(4.037,1.610)}
\gppoint{gp mark 7}{(4.050,1.610)}
\gppoint{gp mark 7}{(4.063,1.610)}
\gppoint{gp mark 7}{(4.075,1.610)}
\gppoint{gp mark 7}{(4.088,1.610)}
\gppoint{gp mark 7}{(4.101,1.610)}
\gppoint{gp mark 7}{(4.113,1.610)}
\gppoint{gp mark 7}{(4.126,1.610)}
\gppoint{gp mark 7}{(4.139,1.610)}
\gppoint{gp mark 7}{(4.151,1.610)}
\gppoint{gp mark 7}{(4.164,1.610)}
\gppoint{gp mark 7}{(4.177,1.610)}
\gppoint{gp mark 7}{(4.189,1.610)}
\gppoint{gp mark 7}{(4.202,1.610)}
\gppoint{gp mark 7}{(4.215,1.610)}
\gppoint{gp mark 7}{(4.227,1.610)}
\gppoint{gp mark 7}{(4.240,1.610)}
\gppoint{gp mark 7}{(4.253,1.610)}
\gppoint{gp mark 7}{(4.265,1.610)}
\gppoint{gp mark 7}{(4.278,1.610)}
\gppoint{gp mark 7}{(4.291,1.610)}
\gppoint{gp mark 7}{(4.303,1.610)}
\gppoint{gp mark 7}{(4.316,1.610)}
\gppoint{gp mark 7}{(4.329,1.610)}
\gppoint{gp mark 7}{(4.342,1.610)}
\gppoint{gp mark 7}{(4.354,1.610)}
\gppoint{gp mark 7}{(4.367,1.610)}
\gppoint{gp mark 7}{(4.380,1.610)}
\gppoint{gp mark 7}{(4.392,1.610)}
\gppoint{gp mark 7}{(4.405,1.610)}
\gppoint{gp mark 7}{(4.418,1.610)}
\gppoint{gp mark 7}{(4.430,1.610)}
\gppoint{gp mark 7}{(4.443,1.610)}
\gppoint{gp mark 7}{(4.456,1.610)}
\gppoint{gp mark 7}{(4.468,1.610)}
\gppoint{gp mark 7}{(4.481,1.610)}
\gppoint{gp mark 7}{(4.494,1.610)}
\gppoint{gp mark 7}{(4.506,1.610)}
\gppoint{gp mark 7}{(4.519,1.610)}
\gppoint{gp mark 7}{(4.532,1.610)}
\gppoint{gp mark 7}{(4.544,1.610)}
\gppoint{gp mark 7}{(4.557,1.610)}
\gppoint{gp mark 7}{(4.570,1.610)}
\gppoint{gp mark 7}{(4.582,1.610)}
\gppoint{gp mark 7}{(4.595,1.610)}
\gppoint{gp mark 7}{(4.608,1.610)}
\gppoint{gp mark 7}{(4.620,1.610)}
\gppoint{gp mark 7}{(4.633,1.610)}
\gppoint{gp mark 7}{(4.646,1.610)}
\gppoint{gp mark 7}{(4.658,1.610)}
\gppoint{gp mark 7}{(4.671,1.610)}
\gppoint{gp mark 7}{(4.684,1.610)}
\gppoint{gp mark 7}{(4.697,1.610)}
\gppoint{gp mark 7}{(4.709,1.610)}
\gppoint{gp mark 7}{(4.722,1.610)}
\gppoint{gp mark 7}{(4.735,1.610)}
\gppoint{gp mark 7}{(4.747,1.610)}
\gppoint{gp mark 7}{(4.760,1.610)}
\gppoint{gp mark 7}{(4.773,1.610)}
\gppoint{gp mark 7}{(4.785,1.610)}
\gppoint{gp mark 7}{(4.798,1.610)}
\gppoint{gp mark 7}{(4.811,1.610)}
\gppoint{gp mark 7}{(4.823,1.610)}
\gppoint{gp mark 7}{(4.836,1.610)}
\gppoint{gp mark 7}{(4.849,1.610)}
\gppoint{gp mark 7}{(4.861,1.610)}
\gppoint{gp mark 7}{(4.874,1.610)}
\gppoint{gp mark 7}{(4.887,1.610)}
\gppoint{gp mark 7}{(4.899,1.610)}
\gppoint{gp mark 7}{(4.912,1.610)}
\gppoint{gp mark 7}{(4.925,1.610)}
\gppoint{gp mark 7}{(4.937,1.610)}
\gppoint{gp mark 7}{(4.950,1.610)}
\gppoint{gp mark 7}{(4.963,1.610)}
\gppoint{gp mark 7}{(4.975,1.610)}
\gppoint{gp mark 7}{(4.988,1.610)}
\gppoint{gp mark 7}{(5.001,1.610)}
\gppoint{gp mark 7}{(5.013,1.610)}
\gppoint{gp mark 7}{(5.026,1.610)}
\gppoint{gp mark 7}{(5.039,1.610)}
\gppoint{gp mark 7}{(5.052,1.610)}
\gppoint{gp mark 7}{(5.064,1.610)}
\gppoint{gp mark 7}{(5.077,1.610)}
\gppoint{gp mark 7}{(5.090,1.610)}
\gppoint{gp mark 7}{(5.102,1.610)}
\gppoint{gp mark 7}{(5.115,1.610)}
\gppoint{gp mark 7}{(5.128,1.610)}
\gppoint{gp mark 7}{(5.140,1.610)}
\gppoint{gp mark 7}{(5.153,1.610)}
\gppoint{gp mark 7}{(5.166,1.610)}
\gppoint{gp mark 7}{(5.178,1.610)}
\gppoint{gp mark 7}{(5.191,1.610)}
\gppoint{gp mark 7}{(5.204,1.610)}
\gppoint{gp mark 7}{(5.216,1.610)}
\gppoint{gp mark 7}{(5.229,1.610)}
\gppoint{gp mark 7}{(5.242,1.610)}
\gppoint{gp mark 7}{(5.254,1.610)}
\gppoint{gp mark 7}{(5.267,1.610)}
\gppoint{gp mark 7}{(5.280,1.610)}
\gppoint{gp mark 7}{(5.292,1.610)}
\gppoint{gp mark 7}{(5.305,1.610)}
\gppoint{gp mark 7}{(5.318,1.610)}
\gppoint{gp mark 7}{(5.330,1.610)}
\gppoint{gp mark 7}{(5.343,1.610)}
\gppoint{gp mark 7}{(5.356,1.610)}
\gppoint{gp mark 7}{(5.368,1.610)}
\gppoint{gp mark 7}{(5.381,1.610)}
\gppoint{gp mark 7}{(5.394,1.610)}
\gppoint{gp mark 7}{(5.407,1.610)}
\gppoint{gp mark 7}{(5.419,1.610)}
\gppoint{gp mark 7}{(5.432,1.610)}
\gppoint{gp mark 7}{(5.445,1.610)}
\gppoint{gp mark 7}{(5.457,1.610)}
\gppoint{gp mark 7}{(5.470,1.610)}
\gppoint{gp mark 7}{(5.483,1.610)}
\gppoint{gp mark 7}{(5.495,1.610)}
\gppoint{gp mark 7}{(5.508,1.610)}
\gppoint{gp mark 7}{(5.521,1.610)}
\gppoint{gp mark 7}{(5.533,1.610)}
\gppoint{gp mark 7}{(5.546,1.610)}
\gppoint{gp mark 7}{(5.559,1.610)}
\gppoint{gp mark 7}{(5.571,1.610)}
\gppoint{gp mark 7}{(5.584,1.610)}
\gppoint{gp mark 7}{(5.597,1.610)}
\gppoint{gp mark 7}{(5.609,1.610)}
\gppoint{gp mark 7}{(5.622,1.610)}
\gppoint{gp mark 7}{(5.635,1.610)}
\gppoint{gp mark 7}{(5.647,1.610)}
\gppoint{gp mark 7}{(5.660,1.610)}
\gppoint{gp mark 7}{(5.673,1.610)}
\gppoint{gp mark 7}{(5.685,1.610)}
\gppoint{gp mark 7}{(5.698,1.610)}
\gppoint{gp mark 7}{(5.711,1.610)}
\gppoint{gp mark 7}{(5.723,1.610)}
\gppoint{gp mark 7}{(5.736,1.610)}
\gppoint{gp mark 7}{(5.749,1.610)}
\gppoint{gp mark 7}{(5.761,1.610)}
\gppoint{gp mark 7}{(5.774,1.610)}
\gppoint{gp mark 7}{(5.787,1.610)}
\gppoint{gp mark 7}{(5.800,1.610)}
\gppoint{gp mark 7}{(5.812,1.610)}
\gppoint{gp mark 7}{(5.825,1.610)}
\gppoint{gp mark 7}{(5.838,1.610)}
\gppoint{gp mark 7}{(5.850,1.610)}
\gppoint{gp mark 7}{(5.863,1.610)}
\gppoint{gp mark 7}{(5.876,1.610)}
\gppoint{gp mark 7}{(5.888,1.610)}
\gppoint{gp mark 7}{(5.901,1.610)}
\gppoint{gp mark 7}{(5.914,1.610)}
\gppoint{gp mark 7}{(5.926,1.610)}
\gppoint{gp mark 7}{(5.939,1.610)}
\gppoint{gp mark 7}{(5.952,1.610)}
\gppoint{gp mark 7}{(5.964,1.610)}
\gppoint{gp mark 7}{(5.977,1.610)}
\gppoint{gp mark 7}{(5.990,1.610)}
\gppoint{gp mark 7}{(6.002,1.610)}
\gppoint{gp mark 7}{(6.015,1.610)}
\gppoint{gp mark 7}{(6.028,1.610)}
\gppoint{gp mark 7}{(6.040,1.610)}
\gppoint{gp mark 7}{(6.053,1.610)}
\gppoint{gp mark 7}{(6.066,1.610)}
\gppoint{gp mark 7}{(6.078,1.610)}
\gppoint{gp mark 7}{(6.091,1.610)}
\gppoint{gp mark 7}{(6.104,1.610)}
\gppoint{gp mark 7}{(6.116,1.610)}
\gppoint{gp mark 7}{(6.129,1.610)}
\gppoint{gp mark 7}{(6.142,1.610)}
\gppoint{gp mark 7}{(6.155,1.610)}
\gppoint{gp mark 7}{(6.167,1.610)}
\gppoint{gp mark 7}{(6.180,1.610)}
\gppoint{gp mark 7}{(6.193,1.610)}
\gppoint{gp mark 7}{(6.205,1.610)}
\gppoint{gp mark 7}{(6.218,1.610)}
\gppoint{gp mark 7}{(6.231,1.610)}
\gppoint{gp mark 7}{(6.243,1.610)}
\gppoint{gp mark 7}{(6.256,1.610)}
\gppoint{gp mark 7}{(6.269,1.610)}
\gppoint{gp mark 7}{(6.281,1.610)}
\gppoint{gp mark 7}{(6.294,1.610)}
\gppoint{gp mark 7}{(6.307,1.610)}
\gppoint{gp mark 7}{(6.319,1.610)}
\gppoint{gp mark 7}{(6.332,1.610)}
\gppoint{gp mark 7}{(6.345,1.610)}
\gppoint{gp mark 7}{(6.357,1.610)}
\gppoint{gp mark 7}{(6.370,1.610)}
\gppoint{gp mark 7}{(6.383,1.610)}
\gppoint{gp mark 7}{(6.395,1.610)}
\gppoint{gp mark 7}{(6.408,1.610)}
\gppoint{gp mark 7}{(6.421,1.610)}
\gppoint{gp mark 7}{(6.433,1.610)}
\gppoint{gp mark 7}{(6.446,1.610)}
\gppoint{gp mark 7}{(6.459,1.610)}
\gppoint{gp mark 7}{(6.471,1.610)}
\gppoint{gp mark 7}{(6.484,1.610)}
\gppoint{gp mark 7}{(6.497,1.610)}
\gppoint{gp mark 7}{(6.510,1.610)}
\gppoint{gp mark 7}{(6.522,1.610)}
\gppoint{gp mark 7}{(6.535,1.610)}
\gppoint{gp mark 7}{(6.548,1.610)}
\gppoint{gp mark 7}{(6.560,1.610)}
\gppoint{gp mark 7}{(6.573,1.610)}
\gppoint{gp mark 7}{(6.586,1.610)}
\gppoint{gp mark 7}{(6.598,1.610)}
\gppoint{gp mark 7}{(6.611,1.610)}
\gppoint{gp mark 7}{(6.624,1.610)}
\gppoint{gp mark 7}{(6.636,1.610)}
\gppoint{gp mark 7}{(6.649,1.610)}
\gppoint{gp mark 7}{(6.662,1.610)}
\gppoint{gp mark 7}{(6.674,1.610)}
\gppoint{gp mark 7}{(6.687,1.610)}
\gppoint{gp mark 7}{(6.700,1.610)}
\gppoint{gp mark 7}{(6.712,1.610)}
\gppoint{gp mark 7}{(6.725,1.610)}
\gppoint{gp mark 7}{(6.738,1.610)}
\gppoint{gp mark 7}{(6.750,1.610)}
\gppoint{gp mark 7}{(6.763,1.610)}
\gppoint{gp mark 7}{(6.776,1.610)}
\gppoint{gp mark 7}{(6.788,1.610)}
\gppoint{gp mark 7}{(6.801,1.610)}
\gppoint{gp mark 7}{(6.814,1.610)}
\gppoint{gp mark 7}{(6.826,1.610)}
\gppoint{gp mark 7}{(6.839,1.610)}
\gppoint{gp mark 7}{(6.852,1.610)}
\gppoint{gp mark 7}{(6.865,1.610)}
\gppoint{gp mark 7}{(6.877,1.610)}
\gppoint{gp mark 7}{(6.890,1.610)}
\gppoint{gp mark 7}{(6.903,1.610)}
\gppoint{gp mark 7}{(6.915,1.610)}
\gppoint{gp mark 7}{(6.928,1.610)}
\gppoint{gp mark 7}{(6.941,1.610)}
\gppoint{gp mark 7}{(6.953,1.610)}
\gppoint{gp mark 7}{(6.966,1.610)}
\gppoint{gp mark 7}{(6.979,1.610)}
\gppoint{gp mark 7}{(6.991,1.610)}
\gppoint{gp mark 7}{(7.004,1.610)}
\gppoint{gp mark 7}{(7.017,1.610)}
\gppoint{gp mark 7}{(7.029,1.610)}
\gppoint{gp mark 7}{(7.042,1.610)}
\gppoint{gp mark 7}{(7.055,1.610)}
\gppoint{gp mark 7}{(7.067,1.610)}
\gppoint{gp mark 7}{(7.080,1.610)}
\gppoint{gp mark 7}{(7.093,1.610)}
\gppoint{gp mark 7}{(7.105,1.610)}
\gppoint{gp mark 7}{(7.118,1.610)}
\gppoint{gp mark 7}{(7.131,1.610)}
\gppoint{gp mark 7}{(7.143,1.610)}
\gppoint{gp mark 7}{(7.156,1.610)}
\gppoint{gp mark 7}{(7.169,1.610)}
\gppoint{gp mark 7}{(7.181,1.610)}
\gppoint{gp mark 7}{(7.194,1.610)}
\gppoint{gp mark 7}{(7.207,1.610)}
\gppoint{gp mark 7}{(7.220,1.610)}
\gppoint{gp mark 7}{(7.232,1.610)}
\gppoint{gp mark 7}{(7.245,1.610)}
\gppoint{gp mark 7}{(7.258,1.610)}
\gppoint{gp mark 7}{(7.270,1.610)}
\gppoint{gp mark 7}{(7.283,1.610)}
\gppoint{gp mark 7}{(7.296,1.610)}
\gppoint{gp mark 7}{(7.308,1.610)}
\gppoint{gp mark 7}{(7.321,1.610)}
\gppoint{gp mark 7}{(7.334,1.610)}
\gppoint{gp mark 7}{(7.346,1.610)}
\gppoint{gp mark 7}{(7.359,1.610)}
\gppoint{gp mark 7}{(7.372,1.610)}
\gppoint{gp mark 7}{(7.384,1.610)}
\gppoint{gp mark 7}{(7.397,1.610)}
\gppoint{gp mark 7}{(7.410,1.610)}
\gppoint{gp mark 7}{(7.422,1.610)}
\gppoint{gp mark 7}{(7.435,1.610)}
\gppoint{gp mark 7}{(7.448,1.610)}
\gppoint{gp mark 7}{(7.460,1.610)}
\gppoint{gp mark 7}{(7.473,1.610)}
\gppoint{gp mark 7}{(7.486,1.610)}
\gppoint{gp mark 7}{(7.498,1.610)}
\gppoint{gp mark 7}{(7.511,1.610)}
\gppoint{gp mark 7}{(7.524,1.610)}
\gppoint{gp mark 7}{(7.536,1.610)}
\gppoint{gp mark 7}{(7.549,1.610)}
\gppoint{gp mark 7}{(7.562,1.610)}
\gppoint{gp mark 7}{(7.575,1.610)}
\gppoint{gp mark 7}{(7.587,1.610)}
\gppoint{gp mark 7}{(7.600,1.610)}
\gppoint{gp mark 7}{(7.613,1.610)}
\gppoint{gp mark 7}{(7.625,1.610)}
\gppoint{gp mark 7}{(7.638,1.610)}
\gppoint{gp mark 7}{(7.651,1.610)}
\gppoint{gp mark 7}{(7.663,1.610)}
\gppoint{gp mark 7}{(7.676,1.610)}
\gppoint{gp mark 7}{(7.689,1.610)}
\gppoint{gp mark 7}{(7.701,1.610)}
\gppoint{gp mark 7}{(7.714,1.610)}
\gppoint{gp mark 7}{(7.727,1.610)}
\gppoint{gp mark 7}{(7.739,1.610)}
\gppoint{gp mark 7}{(7.752,1.610)}
\gppoint{gp mark 7}{(7.765,1.610)}
\gppoint{gp mark 7}{(7.777,1.610)}
\gppoint{gp mark 7}{(7.790,1.610)}
\gppoint{gp mark 7}{(7.803,1.610)}
\gppoint{gp mark 7}{(7.815,1.610)}
\gppoint{gp mark 7}{(7.828,1.610)}
\gppoint{gp mark 7}{(7.841,1.610)}
\gppoint{gp mark 7}{(7.853,1.610)}
\gppoint{gp mark 7}{(7.866,1.610)}
\gppoint{gp mark 7}{(7.879,1.610)}
\gppoint{gp mark 7}{(7.891,1.610)}
\gppoint{gp mark 7}{(7.904,1.610)}
\gppoint{gp mark 7}{(7.917,1.610)}
\gppoint{gp mark 7}{(7.930,1.610)}
\gpcolor{rgb color={0.000,0.000,0.000}}
\gpsetlinetype{gp lt plot 0}
\gpsetlinewidth{4.00}
\draw[gp path] (2.440,3.897)--(2.899,3.897);
\draw[gp path] (2.899,1.171)--(3.805,1.171);
\draw[gp path] (3.805,1.610)--(7.720,1.610);
\draw[gp path] (7.720,1.610)--(7.947,1.610);
\draw[gp path] (1.204,4.831)--(1.217,4.821)--(1.230,4.812)--(1.243,4.802)--(1.256,4.792)%
  --(1.269,4.782)--(1.282,4.772)--(1.295,4.762)--(1.308,4.753)--(1.321,4.743)--(1.334,4.733)%
  --(1.347,4.723)--(1.360,4.713)--(1.373,4.703)--(1.386,4.694)--(1.399,4.684)--(1.412,4.674)%
  --(1.425,4.664)--(1.438,4.654)--(1.451,4.644)--(1.464,4.635)--(1.477,4.625)--(1.490,4.615)%
  --(1.503,4.605)--(1.516,4.595)--(1.529,4.585)--(1.542,4.576)--(1.555,4.566)--(1.568,4.556)%
  --(1.581,4.546)--(1.594,4.536)--(1.607,4.526)--(1.620,4.517)--(1.633,4.507)--(1.646,4.497)%
  --(1.659,4.487)--(1.672,4.477)--(1.686,4.467)--(1.699,4.457)--(1.712,4.448)--(1.725,4.438)%
  --(1.738,4.428)--(1.751,4.418)--(1.764,4.408)--(1.777,4.398)--(1.790,4.389)--(1.803,4.379)%
  --(1.816,4.369)--(1.829,4.359)--(1.842,4.349)--(1.855,4.339)--(1.868,4.330)--(1.881,4.320)%
  --(1.894,4.310)--(1.907,4.300)--(1.920,4.290)--(1.933,4.280)--(1.946,4.271)--(1.959,4.261)%
  --(1.972,4.251)--(1.985,4.241)--(1.998,4.231)--(2.011,4.221)--(2.024,4.212)--(2.037,4.202)%
  --(2.050,4.192)--(2.063,4.182)--(2.076,4.172)--(2.089,4.162)--(2.102,4.153)--(2.115,4.143)%
  --(2.128,4.133)--(2.141,4.123)--(2.154,4.113)--(2.167,4.103)--(2.180,4.093)--(2.193,4.084)%
  --(2.206,4.074)--(2.219,4.064)--(2.232,4.054)--(2.245,4.044)--(2.258,4.034)--(2.271,4.025)%
  --(2.284,4.015)--(2.297,4.005)--(2.310,3.995)--(2.323,3.985)--(2.336,3.975)--(2.349,3.966)%
  --(2.362,3.956)--(2.375,3.946)--(2.388,3.936)--(2.401,3.926)--(2.414,3.916)--(2.427,3.907)%
  --(2.440,3.897);
\draw[gp path] (2.899,3.897)--(2.899,1.171);
\draw[gp path] (3.805,1.171)--(3.805,1.610);
\draw[gp path] (3.793,3.695)--(4.572,3.695);
\gpcolor{rgb color={1.000,0.000,0.000}}
\gpsetlinewidth{0.50}
\gppoint{gp mark 7}{(4.182,2.921)}
\gpcolor{rgb color={0.502,0.502,0.502}}
\gppoint{gp mark 7}{(4.182,2.147)}
\gpcolor{rgb color={0.000,0.000,0.000}}
\node[gp node left,font={\fontsize{10pt}{12pt}\selectfont}] at (1.456,5.166) {\LARGE $B_y$};
\node[gp node left,font={\fontsize{10pt}{12pt}\selectfont}] at (5.740,5.166) {\large $\alpha = \pi$};
\node[gp node left,font={\fontsize{10pt}{12pt}\selectfont}] at (4.831,3.695) {\large exact};
\node[gp node left,font={\fontsize{10pt}{12pt}\selectfont}] at (4.831,2.921) {\large HLLD-CWM};
\node[gp node left,font={\fontsize{10pt}{12pt}\selectfont}] at (4.831,2.147) {\large HLLD};
%% coordinates of the plot area
\gpdefrectangularnode{gp plot 1}{\pgfpoint{1.196cm}{0.985cm}}{\pgfpoint{7.947cm}{5.631cm}}
\end{tikzpicture}
%% gnuplot variables
} \\
\end{tabular}
\caption{The fast rarefaction and rotational discontinuity solution found using HLLD-CWM without the (optional) flux redistribution step, HLLD, and the exact solver using $2048$ grid points for (top) a near-coplanar and pseudo-converging case and (bottom) the planar and non-converging (bottom) case.}
\label{fig:fast_coplanar_ab_crsol}
\end{figure}

For Test~6, the solutions of the coplanar and near coplanar cases obtained with  $A = 0.05$ using HLLD, and HLLD-CWM without any additional artificial viscosity for 512 grid points are shown in Figure~\ref{fig:fast_coplanar_ab_crsol_512}, and for 2048 grid points are shown in Figure~\ref{fig:fast_coplanar_ab_crsol}.  The approximate position of the left-going SS is improved with HLLD-CWM for both levels of grid refinement.  This test indicates HLLD-CWM performs extremely well for weak intermediate shocks, with minimal error through the transition.      

%-----------------------------------------------------------------
% Two fast compound 
%-----------------------------------------------------------------
\begin{figure}[htbp] 
\begin{tabular}{cc}
\resizebox{0.5\linewidth}{!}{\tikzsetnextfilename{AK7_crsol_1}\begin{tikzpicture}[gnuplot]
%% generated with GNUPLOT 4.6p4 (Lua 5.1; terminal rev. 99, script rev. 100)
%% Fri 22 Aug 2014 08:42:05 AM EDT
\path (0.000,0.000) rectangle (8.500,6.000);
\gpfill{rgb color={1.000,1.000,1.000}} (1.012,0.985)--(7.946,0.985)--(7.946,5.630)--(1.012,5.630)--cycle;
\gpcolor{color=gp lt color border}
\gpsetlinetype{gp lt border}
\gpsetlinewidth{1.00}
\draw[gp path] (1.012,0.985)--(1.012,5.630)--(7.946,5.630)--(7.946,0.985)--cycle;
\gpcolor{color=gp lt color axes}
\gpsetlinetype{gp lt axes}
\gpsetlinewidth{2.00}
\draw[gp path] (1.012,1.295)--(7.947,1.295);
\gpcolor{color=gp lt color border}
\gpsetlinetype{gp lt border}
\draw[gp path] (1.012,1.295)--(1.084,1.295);
\draw[gp path] (7.947,1.295)--(7.875,1.295);
\gpcolor{rgb color={0.000,0.000,0.000}}
\node[gp node right,font={\fontsize{10pt}{12pt}\selectfont}] at (0.828,1.295) {0.4};
\gpcolor{color=gp lt color axes}
\gpsetlinetype{gp lt axes}
\draw[gp path] (1.012,1.914)--(7.947,1.914);
\gpcolor{color=gp lt color border}
\gpsetlinetype{gp lt border}
\draw[gp path] (1.012,1.914)--(1.084,1.914);
\draw[gp path] (7.947,1.914)--(7.875,1.914);
\gpcolor{rgb color={0.000,0.000,0.000}}
\node[gp node right,font={\fontsize{10pt}{12pt}\selectfont}] at (0.828,1.914) {0.6};
\gpcolor{color=gp lt color axes}
\gpsetlinetype{gp lt axes}
\draw[gp path] (1.012,2.534)--(7.947,2.534);
\gpcolor{color=gp lt color border}
\gpsetlinetype{gp lt border}
\draw[gp path] (1.012,2.534)--(1.084,2.534);
\draw[gp path] (7.947,2.534)--(7.875,2.534);
\gpcolor{rgb color={0.000,0.000,0.000}}
\node[gp node right,font={\fontsize{10pt}{12pt}\selectfont}] at (0.828,2.534) {0.8};
\gpcolor{color=gp lt color axes}
\gpsetlinetype{gp lt axes}
\draw[gp path] (1.012,3.153)--(7.947,3.153);
\gpcolor{color=gp lt color border}
\gpsetlinetype{gp lt border}
\draw[gp path] (1.012,3.153)--(1.084,3.153);
\draw[gp path] (7.947,3.153)--(7.875,3.153);
\gpcolor{rgb color={0.000,0.000,0.000}}
\node[gp node right,font={\fontsize{10pt}{12pt}\selectfont}] at (0.828,3.153) {1};
\gpcolor{color=gp lt color axes}
\gpsetlinetype{gp lt axes}
\draw[gp path] (1.012,3.773)--(7.947,3.773);
\gpcolor{color=gp lt color border}
\gpsetlinetype{gp lt border}
\draw[gp path] (1.012,3.773)--(1.084,3.773);
\draw[gp path] (7.947,3.773)--(7.875,3.773);
\gpcolor{rgb color={0.000,0.000,0.000}}
\node[gp node right,font={\fontsize{10pt}{12pt}\selectfont}] at (0.828,3.773) {1.2};
\gpcolor{color=gp lt color axes}
\gpsetlinetype{gp lt axes}
\draw[gp path] (1.012,4.392)--(7.947,4.392);
\gpcolor{color=gp lt color border}
\gpsetlinetype{gp lt border}
\draw[gp path] (1.012,4.392)--(1.084,4.392);
\draw[gp path] (7.947,4.392)--(7.875,4.392);
\gpcolor{rgb color={0.000,0.000,0.000}}
\node[gp node right,font={\fontsize{10pt}{12pt}\selectfont}] at (0.828,4.392) {1.4};
\gpcolor{color=gp lt color axes}
\gpsetlinetype{gp lt axes}
\draw[gp path] (1.012,5.012)--(7.947,5.012);
\gpcolor{color=gp lt color border}
\gpsetlinetype{gp lt border}
\draw[gp path] (1.012,5.012)--(1.084,5.012);
\draw[gp path] (7.947,5.012)--(7.875,5.012);
\gpcolor{rgb color={0.000,0.000,0.000}}
\node[gp node right,font={\fontsize{10pt}{12pt}\selectfont}] at (0.828,5.012) {1.6};
\gpcolor{color=gp lt color axes}
\gpsetlinetype{gp lt axes}
\draw[gp path] (1.012,5.631)--(7.947,5.631);
\gpcolor{color=gp lt color border}
\gpsetlinetype{gp lt border}
\draw[gp path] (1.012,5.631)--(1.084,5.631);
\draw[gp path] (7.947,5.631)--(7.875,5.631);
\gpcolor{rgb color={0.000,0.000,0.000}}
\node[gp node right,font={\fontsize{10pt}{12pt}\selectfont}] at (0.828,5.631) {1.8};
\gpcolor{color=gp lt color axes}
\gpsetlinetype{gp lt axes}
\draw[gp path] (1.012,0.985)--(1.012,5.631);
\gpcolor{color=gp lt color border}
\gpsetlinetype{gp lt border}
\draw[gp path] (1.012,0.985)--(1.012,1.057);
\draw[gp path] (1.012,5.631)--(1.012,5.559);
\gpcolor{rgb color={0.000,0.000,0.000}}
\node[gp node center,font={\fontsize{10pt}{12pt}\selectfont}] at (1.012,0.677) {0.3};
\gpcolor{color=gp lt color axes}
\gpsetlinetype{gp lt axes}
\draw[gp path] (2.399,0.985)--(2.399,5.631);
\gpcolor{color=gp lt color border}
\gpsetlinetype{gp lt border}
\draw[gp path] (2.399,0.985)--(2.399,1.057);
\draw[gp path] (2.399,5.631)--(2.399,5.559);
\gpcolor{rgb color={0.000,0.000,0.000}}
\node[gp node center,font={\fontsize{10pt}{12pt}\selectfont}] at (2.399,0.677) {0.4};
\gpcolor{color=gp lt color axes}
\gpsetlinetype{gp lt axes}
\draw[gp path] (3.786,0.985)--(3.786,5.631);
\gpcolor{color=gp lt color border}
\gpsetlinetype{gp lt border}
\draw[gp path] (3.786,0.985)--(3.786,1.057);
\draw[gp path] (3.786,5.631)--(3.786,5.559);
\gpcolor{rgb color={0.000,0.000,0.000}}
\node[gp node center,font={\fontsize{10pt}{12pt}\selectfont}] at (3.786,0.677) {0.5};
\gpcolor{color=gp lt color axes}
\gpsetlinetype{gp lt axes}
\draw[gp path] (5.173,0.985)--(5.173,5.631);
\gpcolor{color=gp lt color border}
\gpsetlinetype{gp lt border}
\draw[gp path] (5.173,0.985)--(5.173,1.057);
\draw[gp path] (5.173,5.631)--(5.173,5.559);
\gpcolor{rgb color={0.000,0.000,0.000}}
\node[gp node center,font={\fontsize{10pt}{12pt}\selectfont}] at (5.173,0.677) {0.6};
\gpcolor{color=gp lt color axes}
\gpsetlinetype{gp lt axes}
\draw[gp path] (6.560,0.985)--(6.560,5.631);
\gpcolor{color=gp lt color border}
\gpsetlinetype{gp lt border}
\draw[gp path] (6.560,0.985)--(6.560,1.057);
\draw[gp path] (6.560,5.631)--(6.560,5.559);
\gpcolor{rgb color={0.000,0.000,0.000}}
\node[gp node center,font={\fontsize{10pt}{12pt}\selectfont}] at (6.560,0.677) {0.7};
\gpcolor{color=gp lt color axes}
\gpsetlinetype{gp lt axes}
\draw[gp path] (7.947,0.985)--(7.947,5.631);
\gpcolor{color=gp lt color border}
\gpsetlinetype{gp lt border}
\draw[gp path] (7.947,0.985)--(7.947,1.057);
\draw[gp path] (7.947,5.631)--(7.947,5.559);
\gpcolor{rgb color={0.000,0.000,0.000}}
\node[gp node center,font={\fontsize{10pt}{12pt}\selectfont}] at (7.947,0.677) {0.8};
\gpcolor{color=gp lt color border}
\draw[gp path] (1.012,5.631)--(1.012,0.985)--(7.947,0.985)--(7.947,5.631)--cycle;
\gpcolor{rgb color={0.000,0.000,0.000}}
\node[gp node center,font={\fontsize{10pt}{12pt}\selectfont}] at (4.479,0.215) {\large $x$};
\gpcolor{rgb color={0.502,0.502,0.502}}
\gpsetlinewidth{6.00}
\gpsetpointsize{2.67}
\gppoint{gp mark 7}{(1.019,2.865)}
\gppoint{gp mark 7}{(1.026,2.861)}
\gppoint{gp mark 7}{(1.033,2.857)}
\gppoint{gp mark 7}{(1.040,2.853)}
\gppoint{gp mark 7}{(1.047,2.848)}
\gppoint{gp mark 7}{(1.053,2.844)}
\gppoint{gp mark 7}{(1.060,2.840)}
\gppoint{gp mark 7}{(1.067,2.836)}
\gppoint{gp mark 7}{(1.074,2.832)}
\gppoint{gp mark 7}{(1.080,2.828)}
\gppoint{gp mark 7}{(1.087,2.824)}
\gppoint{gp mark 7}{(1.094,2.820)}
\gppoint{gp mark 7}{(1.101,2.816)}
\gppoint{gp mark 7}{(1.107,2.812)}
\gppoint{gp mark 7}{(1.114,2.807)}
\gppoint{gp mark 7}{(1.121,2.803)}
\gppoint{gp mark 7}{(1.128,2.799)}
\gppoint{gp mark 7}{(1.135,2.795)}
\gppoint{gp mark 7}{(1.141,2.791)}
\gppoint{gp mark 7}{(1.148,2.787)}
\gppoint{gp mark 7}{(1.155,2.783)}
\gppoint{gp mark 7}{(1.162,2.779)}
\gppoint{gp mark 7}{(1.168,2.775)}
\gppoint{gp mark 7}{(1.175,2.771)}
\gppoint{gp mark 7}{(1.182,2.767)}
\gppoint{gp mark 7}{(1.189,2.763)}
\gppoint{gp mark 7}{(1.196,2.758)}
\gppoint{gp mark 7}{(1.202,2.754)}
\gppoint{gp mark 7}{(1.209,2.750)}
\gppoint{gp mark 7}{(1.216,2.746)}
\gppoint{gp mark 7}{(1.223,2.742)}
\gppoint{gp mark 7}{(1.229,2.738)}
\gppoint{gp mark 7}{(1.236,2.734)}
\gppoint{gp mark 7}{(1.243,2.730)}
\gppoint{gp mark 7}{(1.250,2.726)}
\gppoint{gp mark 7}{(1.256,2.722)}
\gppoint{gp mark 7}{(1.263,2.718)}
\gppoint{gp mark 7}{(1.270,2.714)}
\gppoint{gp mark 7}{(1.277,2.710)}
\gppoint{gp mark 7}{(1.284,2.706)}
\gppoint{gp mark 7}{(1.290,2.702)}
\gppoint{gp mark 7}{(1.297,2.698)}
\gppoint{gp mark 7}{(1.304,2.694)}
\gppoint{gp mark 7}{(1.311,2.690)}
\gppoint{gp mark 7}{(1.317,2.686)}
\gppoint{gp mark 7}{(1.324,2.682)}
\gppoint{gp mark 7}{(1.331,2.678)}
\gppoint{gp mark 7}{(1.338,2.674)}
\gppoint{gp mark 7}{(1.345,2.670)}
\gppoint{gp mark 7}{(1.351,2.666)}
\gppoint{gp mark 7}{(1.358,2.662)}
\gppoint{gp mark 7}{(1.365,2.658)}
\gppoint{gp mark 7}{(1.372,2.654)}
\gppoint{gp mark 7}{(1.378,2.650)}
\gppoint{gp mark 7}{(1.385,2.646)}
\gppoint{gp mark 7}{(1.392,2.642)}
\gppoint{gp mark 7}{(1.399,2.638)}
\gppoint{gp mark 7}{(1.405,2.634)}
\gppoint{gp mark 7}{(1.412,2.630)}
\gppoint{gp mark 7}{(1.419,2.626)}
\gppoint{gp mark 7}{(1.426,2.622)}
\gppoint{gp mark 7}{(1.433,2.618)}
\gppoint{gp mark 7}{(1.439,2.614)}
\gppoint{gp mark 7}{(1.446,2.610)}
\gppoint{gp mark 7}{(1.453,2.606)}
\gppoint{gp mark 7}{(1.460,2.602)}
\gppoint{gp mark 7}{(1.466,2.598)}
\gppoint{gp mark 7}{(1.473,2.594)}
\gppoint{gp mark 7}{(1.480,2.590)}
\gppoint{gp mark 7}{(1.487,2.587)}
\gppoint{gp mark 7}{(1.494,2.583)}
\gppoint{gp mark 7}{(1.500,2.579)}
\gppoint{gp mark 7}{(1.507,2.575)}
\gppoint{gp mark 7}{(1.514,2.571)}
\gppoint{gp mark 7}{(1.521,2.567)}
\gppoint{gp mark 7}{(1.527,2.563)}
\gppoint{gp mark 7}{(1.534,2.559)}
\gppoint{gp mark 7}{(1.541,2.555)}
\gppoint{gp mark 7}{(1.548,2.551)}
\gppoint{gp mark 7}{(1.554,2.547)}
\gppoint{gp mark 7}{(1.561,2.543)}
\gppoint{gp mark 7}{(1.568,2.540)}
\gppoint{gp mark 7}{(1.575,2.536)}
\gppoint{gp mark 7}{(1.582,2.532)}
\gppoint{gp mark 7}{(1.588,2.528)}
\gppoint{gp mark 7}{(1.595,2.524)}
\gppoint{gp mark 7}{(1.602,2.520)}
\gppoint{gp mark 7}{(1.609,2.516)}
\gppoint{gp mark 7}{(1.615,2.512)}
\gppoint{gp mark 7}{(1.622,2.508)}
\gppoint{gp mark 7}{(1.629,2.505)}
\gppoint{gp mark 7}{(1.636,2.501)}
\gppoint{gp mark 7}{(1.643,2.497)}
\gppoint{gp mark 7}{(1.649,2.493)}
\gppoint{gp mark 7}{(1.656,2.489)}
\gppoint{gp mark 7}{(1.663,2.485)}
\gppoint{gp mark 7}{(1.670,2.481)}
\gppoint{gp mark 7}{(1.676,2.478)}
\gppoint{gp mark 7}{(1.683,2.474)}
\gppoint{gp mark 7}{(1.690,2.470)}
\gppoint{gp mark 7}{(1.697,2.466)}
\gppoint{gp mark 7}{(1.703,2.462)}
\gppoint{gp mark 7}{(1.710,2.459)}
\gppoint{gp mark 7}{(1.717,2.455)}
\gppoint{gp mark 7}{(1.724,2.451)}
\gppoint{gp mark 7}{(1.731,2.447)}
\gppoint{gp mark 7}{(1.737,2.443)}
\gppoint{gp mark 7}{(1.744,2.440)}
\gppoint{gp mark 7}{(1.751,2.436)}
\gppoint{gp mark 7}{(1.758,2.432)}
\gppoint{gp mark 7}{(1.764,2.428)}
\gppoint{gp mark 7}{(1.771,2.425)}
\gppoint{gp mark 7}{(1.778,2.421)}
\gppoint{gp mark 7}{(1.785,2.417)}
\gppoint{gp mark 7}{(1.792,2.414)}
\gppoint{gp mark 7}{(1.798,2.410)}
\gppoint{gp mark 7}{(1.805,2.406)}
\gppoint{gp mark 7}{(1.812,2.403)}
\gppoint{gp mark 7}{(1.819,2.399)}
\gppoint{gp mark 7}{(1.825,2.396)}
\gppoint{gp mark 7}{(1.832,2.392)}
\gppoint{gp mark 7}{(1.839,2.389)}
\gppoint{gp mark 7}{(1.846,2.386)}
\gppoint{gp mark 7}{(1.852,2.383)}
\gppoint{gp mark 7}{(1.859,2.380)}
\gppoint{gp mark 7}{(1.866,2.377)}
\gppoint{gp mark 7}{(1.873,2.375)}
\gppoint{gp mark 7}{(1.880,2.373)}
\gppoint{gp mark 7}{(1.886,2.371)}
\gppoint{gp mark 7}{(1.893,2.369)}
\gppoint{gp mark 7}{(1.900,2.368)}
\gppoint{gp mark 7}{(1.907,2.368)}
\gppoint{gp mark 7}{(1.913,2.367)}
\gppoint{gp mark 7}{(1.920,2.366)}
\gppoint{gp mark 7}{(1.927,2.365)}
\gppoint{gp mark 7}{(1.934,2.364)}
\gppoint{gp mark 7}{(1.941,2.363)}
\gppoint{gp mark 7}{(1.947,2.362)}
\gppoint{gp mark 7}{(1.954,2.374)}
\gppoint{gp mark 7}{(1.961,2.448)}
\gppoint{gp mark 7}{(1.968,2.644)}
\gppoint{gp mark 7}{(1.974,2.675)}
\gppoint{gp mark 7}{(1.981,2.648)}
\gppoint{gp mark 7}{(1.988,2.643)}
\gppoint{gp mark 7}{(1.995,2.645)}
\gppoint{gp mark 7}{(2.001,2.646)}
\gppoint{gp mark 7}{(2.008,2.646)}
\gppoint{gp mark 7}{(2.015,2.646)}
\gppoint{gp mark 7}{(2.022,2.646)}
\gppoint{gp mark 7}{(2.029,2.646)}
\gppoint{gp mark 7}{(2.035,2.646)}
\gppoint{gp mark 7}{(2.042,2.646)}
\gppoint{gp mark 7}{(2.049,2.646)}
\gppoint{gp mark 7}{(2.056,2.646)}
\gppoint{gp mark 7}{(2.062,2.646)}
\gppoint{gp mark 7}{(2.069,2.647)}
\gppoint{gp mark 7}{(2.076,2.647)}
\gppoint{gp mark 7}{(2.083,2.647)}
\gppoint{gp mark 7}{(2.089,2.647)}
\gppoint{gp mark 7}{(2.096,2.647)}
\gppoint{gp mark 7}{(2.103,2.647)}
\gppoint{gp mark 7}{(2.110,2.646)}
\gppoint{gp mark 7}{(2.117,2.646)}
\gppoint{gp mark 7}{(2.123,2.646)}
\gppoint{gp mark 7}{(2.130,2.646)}
\gppoint{gp mark 7}{(2.137,2.647)}
\gppoint{gp mark 7}{(2.144,2.648)}
\gppoint{gp mark 7}{(2.150,2.648)}
\gppoint{gp mark 7}{(2.157,2.648)}
\gppoint{gp mark 7}{(2.164,2.648)}
\gppoint{gp mark 7}{(2.171,2.648)}
\gppoint{gp mark 7}{(2.178,2.648)}
\gppoint{gp mark 7}{(2.184,2.647)}
\gppoint{gp mark 7}{(2.191,2.646)}
\gppoint{gp mark 7}{(2.198,2.646)}
\gppoint{gp mark 7}{(2.205,2.647)}
\gppoint{gp mark 7}{(2.211,2.647)}
\gppoint{gp mark 7}{(2.218,2.648)}
\gppoint{gp mark 7}{(2.225,2.647)}
\gppoint{gp mark 7}{(2.232,2.647)}
\gppoint{gp mark 7}{(2.238,2.647)}
\gppoint{gp mark 7}{(2.245,2.647)}
\gppoint{gp mark 7}{(2.252,2.647)}
\gppoint{gp mark 7}{(2.259,2.647)}
\gppoint{gp mark 7}{(2.266,2.647)}
\gppoint{gp mark 7}{(2.272,2.648)}
\gppoint{gp mark 7}{(2.279,2.648)}
\gppoint{gp mark 7}{(2.286,2.648)}
\gppoint{gp mark 7}{(2.293,2.648)}
\gppoint{gp mark 7}{(2.299,2.648)}
\gppoint{gp mark 7}{(2.306,2.648)}
\gppoint{gp mark 7}{(2.313,2.648)}
\gppoint{gp mark 7}{(2.320,2.647)}
\gppoint{gp mark 7}{(2.327,2.647)}
\gppoint{gp mark 7}{(2.333,2.647)}
\gppoint{gp mark 7}{(2.340,2.647)}
\gppoint{gp mark 7}{(2.347,2.647)}
\gppoint{gp mark 7}{(2.354,2.648)}
\gppoint{gp mark 7}{(2.360,2.648)}
\gppoint{gp mark 7}{(2.367,2.649)}
\gppoint{gp mark 7}{(2.374,2.649)}
\gppoint{gp mark 7}{(2.381,2.649)}
\gppoint{gp mark 7}{(2.387,2.649)}
\gppoint{gp mark 7}{(2.394,2.648)}
\gppoint{gp mark 7}{(2.401,2.648)}
\gppoint{gp mark 7}{(2.408,2.647)}
\gppoint{gp mark 7}{(2.415,2.647)}
\gppoint{gp mark 7}{(2.421,2.647)}
\gppoint{gp mark 7}{(2.428,2.647)}
\gppoint{gp mark 7}{(2.435,2.647)}
\gppoint{gp mark 7}{(2.442,2.648)}
\gppoint{gp mark 7}{(2.448,2.649)}
\gppoint{gp mark 7}{(2.455,2.649)}
\gppoint{gp mark 7}{(2.462,2.649)}
\gppoint{gp mark 7}{(2.469,2.649)}
\gppoint{gp mark 7}{(2.476,2.649)}
\gppoint{gp mark 7}{(2.482,2.648)}
\gppoint{gp mark 7}{(2.489,2.648)}
\gppoint{gp mark 7}{(2.496,2.648)}
\gppoint{gp mark 7}{(2.503,2.648)}
\gppoint{gp mark 7}{(2.509,2.648)}
\gppoint{gp mark 7}{(2.516,2.649)}
\gppoint{gp mark 7}{(2.523,2.649)}
\gppoint{gp mark 7}{(2.530,2.650)}
\gppoint{gp mark 7}{(2.536,2.650)}
\gppoint{gp mark 7}{(2.543,2.650)}
\gppoint{gp mark 7}{(2.550,2.649)}
\gppoint{gp mark 7}{(2.557,2.649)}
\gppoint{gp mark 7}{(2.564,2.649)}
\gppoint{gp mark 7}{(2.570,2.649)}
\gppoint{gp mark 7}{(2.577,2.649)}
\gppoint{gp mark 7}{(2.584,2.649)}
\gppoint{gp mark 7}{(2.591,2.650)}
\gppoint{gp mark 7}{(2.597,2.650)}
\gppoint{gp mark 7}{(2.604,2.650)}
\gppoint{gp mark 7}{(2.611,2.650)}
\gppoint{gp mark 7}{(2.618,2.651)}
\gppoint{gp mark 7}{(2.625,2.650)}
\gppoint{gp mark 7}{(2.631,2.650)}
\gppoint{gp mark 7}{(2.638,2.650)}
\gppoint{gp mark 7}{(2.645,2.650)}
\gppoint{gp mark 7}{(2.652,2.650)}
\gppoint{gp mark 7}{(2.658,2.650)}
\gppoint{gp mark 7}{(2.665,2.650)}
\gppoint{gp mark 7}{(2.672,2.650)}
\gppoint{gp mark 7}{(2.679,2.650)}
\gppoint{gp mark 7}{(2.685,2.650)}
\gppoint{gp mark 7}{(2.692,2.650)}
\gppoint{gp mark 7}{(2.699,2.650)}
\gppoint{gp mark 7}{(2.706,2.651)}
\gppoint{gp mark 7}{(2.713,2.651)}
\gppoint{gp mark 7}{(2.719,2.651)}
\gppoint{gp mark 7}{(2.726,2.651)}
\gppoint{gp mark 7}{(2.733,2.651)}
\gppoint{gp mark 7}{(2.740,2.651)}
\gppoint{gp mark 7}{(2.746,2.650)}
\gppoint{gp mark 7}{(2.753,2.650)}
\gppoint{gp mark 7}{(2.760,2.649)}
\gppoint{gp mark 7}{(2.767,2.649)}
\gppoint{gp mark 7}{(2.774,2.649)}
\gppoint{gp mark 7}{(2.780,2.650)}
\gppoint{gp mark 7}{(2.787,2.650)}
\gppoint{gp mark 7}{(2.794,2.650)}
\gppoint{gp mark 7}{(2.801,2.651)}
\gppoint{gp mark 7}{(2.807,2.651)}
\gppoint{gp mark 7}{(2.814,2.651)}
\gppoint{gp mark 7}{(2.821,2.651)}
\gppoint{gp mark 7}{(2.828,2.651)}
\gppoint{gp mark 7}{(2.834,2.650)}
\gppoint{gp mark 7}{(2.841,2.650)}
\gppoint{gp mark 7}{(2.848,2.650)}
\gppoint{gp mark 7}{(2.855,2.649)}
\gppoint{gp mark 7}{(2.862,2.649)}
\gppoint{gp mark 7}{(2.868,2.650)}
\gppoint{gp mark 7}{(2.875,2.650)}
\gppoint{gp mark 7}{(2.882,2.650)}
\gppoint{gp mark 7}{(2.889,2.650)}
\gppoint{gp mark 7}{(2.895,2.650)}
\gppoint{gp mark 7}{(2.902,2.650)}
\gppoint{gp mark 7}{(2.909,2.650)}
\gppoint{gp mark 7}{(2.916,2.650)}
\gppoint{gp mark 7}{(2.923,2.650)}
\gppoint{gp mark 7}{(2.929,2.650)}
\gppoint{gp mark 7}{(2.936,2.650)}
\gppoint{gp mark 7}{(2.943,2.650)}
\gppoint{gp mark 7}{(2.950,2.649)}
\gppoint{gp mark 7}{(2.956,2.649)}
\gppoint{gp mark 7}{(2.963,2.649)}
\gppoint{gp mark 7}{(2.970,2.649)}
\gppoint{gp mark 7}{(2.977,2.649)}
\gppoint{gp mark 7}{(2.983,2.649)}
\gppoint{gp mark 7}{(2.990,2.649)}
\gppoint{gp mark 7}{(2.997,2.649)}
\gppoint{gp mark 7}{(3.004,2.649)}
\gppoint{gp mark 7}{(3.011,2.649)}
\gppoint{gp mark 7}{(3.017,2.648)}
\gppoint{gp mark 7}{(3.024,2.648)}
\gppoint{gp mark 7}{(3.031,2.648)}
\gppoint{gp mark 7}{(3.038,2.648)}
\gppoint{gp mark 7}{(3.044,2.648)}
\gppoint{gp mark 7}{(3.051,2.649)}
\gppoint{gp mark 7}{(3.058,2.649)}
\gppoint{gp mark 7}{(3.065,2.649)}
\gppoint{gp mark 7}{(3.072,2.649)}
\gppoint{gp mark 7}{(3.078,2.649)}
\gppoint{gp mark 7}{(3.085,2.648)}
\gppoint{gp mark 7}{(3.092,2.648)}
\gppoint{gp mark 7}{(3.099,2.648)}
\gppoint{gp mark 7}{(3.105,2.648)}
\gppoint{gp mark 7}{(3.112,2.647)}
\gppoint{gp mark 7}{(3.119,2.647)}
\gppoint{gp mark 7}{(3.126,2.647)}
\gppoint{gp mark 7}{(3.132,2.648)}
\gppoint{gp mark 7}{(3.139,2.649)}
\gppoint{gp mark 7}{(3.146,2.650)}
\gppoint{gp mark 7}{(3.153,2.651)}
\gppoint{gp mark 7}{(3.160,2.651)}
\gppoint{gp mark 7}{(3.166,2.651)}
\gppoint{gp mark 7}{(3.173,2.650)}
\gppoint{gp mark 7}{(3.180,2.649)}
\gppoint{gp mark 7}{(3.187,2.649)}
\gppoint{gp mark 7}{(3.193,2.648)}
\gppoint{gp mark 7}{(3.200,2.648)}
\gppoint{gp mark 7}{(3.207,2.648)}
\gppoint{gp mark 7}{(3.214,2.648)}
\gppoint{gp mark 7}{(3.220,2.648)}
\gppoint{gp mark 7}{(3.227,2.651)}
\gppoint{gp mark 7}{(3.234,2.694)}
\gppoint{gp mark 7}{(3.241,3.101)}
\gppoint{gp mark 7}{(3.248,4.368)}
\gppoint{gp mark 7}{(3.254,4.886)}
\gppoint{gp mark 7}{(3.261,4.885)}
\gppoint{gp mark 7}{(3.268,4.865)}
\gppoint{gp mark 7}{(3.275,4.864)}
\gppoint{gp mark 7}{(3.281,4.863)}
\gppoint{gp mark 7}{(3.288,4.861)}
\gppoint{gp mark 7}{(3.295,4.860)}
\gppoint{gp mark 7}{(3.302,4.860)}
\gppoint{gp mark 7}{(3.309,4.857)}
\gppoint{gp mark 7}{(3.315,4.854)}
\gppoint{gp mark 7}{(3.322,4.852)}
\gppoint{gp mark 7}{(3.329,4.852)}
\gppoint{gp mark 7}{(3.336,4.852)}
\gppoint{gp mark 7}{(3.342,4.852)}
\gppoint{gp mark 7}{(3.349,4.850)}
\gppoint{gp mark 7}{(3.356,4.848)}
\gppoint{gp mark 7}{(3.363,4.845)}
\gppoint{gp mark 7}{(3.369,4.841)}
\gppoint{gp mark 7}{(3.376,4.838)}
\gppoint{gp mark 7}{(3.383,4.835)}
\gppoint{gp mark 7}{(3.390,4.832)}
\gppoint{gp mark 7}{(3.397,4.831)}
\gppoint{gp mark 7}{(3.403,4.831)}
\gppoint{gp mark 7}{(3.410,4.830)}
\gppoint{gp mark 7}{(3.417,4.828)}
\gppoint{gp mark 7}{(3.424,4.825)}
\gppoint{gp mark 7}{(3.430,4.822)}
\gppoint{gp mark 7}{(3.437,4.819)}
\gppoint{gp mark 7}{(3.444,4.819)}
\gppoint{gp mark 7}{(3.451,4.819)}
\gppoint{gp mark 7}{(3.458,4.820)}
\gppoint{gp mark 7}{(3.464,4.822)}
\gppoint{gp mark 7}{(3.471,4.823)}
\gppoint{gp mark 7}{(3.478,4.824)}
\gppoint{gp mark 7}{(3.485,4.824)}
\gppoint{gp mark 7}{(3.491,4.824)}
\gppoint{gp mark 7}{(3.498,4.823)}
\gppoint{gp mark 7}{(3.505,4.822)}
\gppoint{gp mark 7}{(3.512,4.822)}
\gppoint{gp mark 7}{(3.518,4.823)}
\gppoint{gp mark 7}{(3.525,4.825)}
\gppoint{gp mark 7}{(3.532,4.825)}
\gppoint{gp mark 7}{(3.539,4.825)}
\gppoint{gp mark 7}{(3.546,4.825)}
\gppoint{gp mark 7}{(3.552,4.824)}
\gppoint{gp mark 7}{(3.559,4.823)}
\gppoint{gp mark 7}{(3.566,4.822)}
\gppoint{gp mark 7}{(3.573,4.821)}
\gppoint{gp mark 7}{(3.579,4.820)}
\gppoint{gp mark 7}{(3.586,4.821)}
\gppoint{gp mark 7}{(3.593,4.821)}
\gppoint{gp mark 7}{(3.600,4.822)}
\gppoint{gp mark 7}{(3.607,4.823)}
\gppoint{gp mark 7}{(3.613,4.824)}
\gppoint{gp mark 7}{(3.620,4.825)}
\gppoint{gp mark 7}{(3.627,4.826)}
\gppoint{gp mark 7}{(3.634,4.827)}
\gppoint{gp mark 7}{(3.640,4.828)}
\gppoint{gp mark 7}{(3.647,4.829)}
\gppoint{gp mark 7}{(3.654,4.831)}
\gppoint{gp mark 7}{(3.661,4.833)}
\gppoint{gp mark 7}{(3.667,4.834)}
\gppoint{gp mark 7}{(3.674,4.834)}
\gppoint{gp mark 7}{(3.681,4.833)}
\gppoint{gp mark 7}{(3.688,4.832)}
\gppoint{gp mark 7}{(3.695,4.831)}
\gppoint{gp mark 7}{(3.701,4.826)}
\gppoint{gp mark 7}{(3.708,4.821)}
\gppoint{gp mark 7}{(3.715,4.815)}
\gppoint{gp mark 7}{(3.722,4.804)}
\gppoint{gp mark 7}{(3.728,4.794)}
\gppoint{gp mark 7}{(3.735,4.788)}
\gppoint{gp mark 7}{(3.742,4.776)}
\gppoint{gp mark 7}{(3.749,4.758)}
\gppoint{gp mark 7}{(3.756,4.748)}
\gppoint{gp mark 7}{(3.762,4.743)}
\gppoint{gp mark 7}{(3.769,4.730)}
\gppoint{gp mark 7}{(3.776,4.707)}
\gppoint{gp mark 7}{(3.783,4.694)}
\gppoint{gp mark 7}{(3.789,4.693)}
\gppoint{gp mark 7}{(3.796,4.693)}
\gppoint{gp mark 7}{(3.803,4.694)}
\gppoint{gp mark 7}{(3.810,4.693)}
\gppoint{gp mark 7}{(3.816,4.682)}
\gppoint{gp mark 7}{(3.823,4.588)}
\gppoint{gp mark 7}{(3.830,4.217)}
\gppoint{gp mark 7}{(3.837,3.925)}
\gppoint{gp mark 7}{(3.844,3.821)}
\gppoint{gp mark 7}{(3.850,3.669)}
\gppoint{gp mark 7}{(3.857,3.069)}
\gppoint{gp mark 7}{(3.864,2.184)}
\gppoint{gp mark 7}{(3.871,1.788)}
\gppoint{gp mark 7}{(3.877,1.748)}
\gppoint{gp mark 7}{(3.884,1.775)}
\gppoint{gp mark 7}{(3.891,2.022)}
\gppoint{gp mark 7}{(3.898,2.484)}
\gppoint{gp mark 7}{(3.905,2.769)}
\gppoint{gp mark 7}{(3.911,2.890)}
\gppoint{gp mark 7}{(3.918,2.948)}
\gppoint{gp mark 7}{(3.925,2.999)}
\gppoint{gp mark 7}{(3.932,3.037)}
\gppoint{gp mark 7}{(3.938,3.063)}
\gppoint{gp mark 7}{(3.945,3.085)}
\gppoint{gp mark 7}{(3.952,3.104)}
\gppoint{gp mark 7}{(3.959,3.120)}
\gppoint{gp mark 7}{(3.965,3.134)}
\gppoint{gp mark 7}{(3.972,3.147)}
\gppoint{gp mark 7}{(3.979,3.158)}
\gppoint{gp mark 7}{(3.986,3.174)}
\gppoint{gp mark 7}{(3.993,3.183)}
\gppoint{gp mark 7}{(3.999,3.192)}
\gppoint{gp mark 7}{(4.006,3.202)}
\gppoint{gp mark 7}{(4.013,3.211)}
\gppoint{gp mark 7}{(4.020,3.218)}
\gppoint{gp mark 7}{(4.026,3.231)}
\gppoint{gp mark 7}{(4.033,3.243)}
\gppoint{gp mark 7}{(4.040,3.252)}
\gppoint{gp mark 7}{(4.047,3.264)}
\gppoint{gp mark 7}{(4.054,3.276)}
\gppoint{gp mark 7}{(4.060,3.289)}
\gppoint{gp mark 7}{(4.067,3.300)}
\gppoint{gp mark 7}{(4.074,3.311)}
\gppoint{gp mark 7}{(4.081,3.329)}
\gppoint{gp mark 7}{(4.087,3.344)}
\gppoint{gp mark 7}{(4.094,3.360)}
\gppoint{gp mark 7}{(4.101,3.384)}
\gppoint{gp mark 7}{(4.108,3.400)}
\gppoint{gp mark 7}{(4.114,3.418)}
\gppoint{gp mark 7}{(4.121,3.437)}
\gppoint{gp mark 7}{(4.128,3.448)}
\gppoint{gp mark 7}{(4.135,3.458)}
\gppoint{gp mark 7}{(4.142,3.467)}
\gppoint{gp mark 7}{(4.148,3.474)}
\gppoint{gp mark 7}{(4.155,3.481)}
\gppoint{gp mark 7}{(4.162,3.486)}
\gppoint{gp mark 7}{(4.169,3.488)}
\gppoint{gp mark 7}{(4.175,3.492)}
\gppoint{gp mark 7}{(4.182,3.494)}
\gppoint{gp mark 7}{(4.189,3.496)}
\gppoint{gp mark 7}{(4.196,3.497)}
\gppoint{gp mark 7}{(4.203,3.498)}
\gppoint{gp mark 7}{(4.209,3.498)}
\gppoint{gp mark 7}{(4.216,3.499)}
\gppoint{gp mark 7}{(4.223,3.499)}
\gppoint{gp mark 7}{(4.230,3.500)}
\gppoint{gp mark 7}{(4.236,3.501)}
\gppoint{gp mark 7}{(4.243,3.502)}
\gppoint{gp mark 7}{(4.250,3.503)}
\gppoint{gp mark 7}{(4.257,3.504)}
\gppoint{gp mark 7}{(4.263,3.503)}
\gppoint{gp mark 7}{(4.270,3.503)}
\gppoint{gp mark 7}{(4.277,3.503)}
\gppoint{gp mark 7}{(4.284,3.504)}
\gppoint{gp mark 7}{(4.291,3.506)}
\gppoint{gp mark 7}{(4.297,3.507)}
\gppoint{gp mark 7}{(4.304,3.508)}
\gppoint{gp mark 7}{(4.311,3.510)}
\gppoint{gp mark 7}{(4.318,3.511)}
\gppoint{gp mark 7}{(4.324,3.511)}
\gppoint{gp mark 7}{(4.331,3.512)}
\gppoint{gp mark 7}{(4.338,3.514)}
\gppoint{gp mark 7}{(4.345,3.514)}
\gppoint{gp mark 7}{(4.352,3.514)}
\gppoint{gp mark 7}{(4.358,3.517)}
\gppoint{gp mark 7}{(4.365,3.519)}
\gppoint{gp mark 7}{(4.372,3.519)}
\gppoint{gp mark 7}{(4.379,3.520)}
\gppoint{gp mark 7}{(4.385,3.523)}
\gppoint{gp mark 7}{(4.392,3.524)}
\gppoint{gp mark 7}{(4.399,3.526)}
\gppoint{gp mark 7}{(4.406,3.529)}
\gppoint{gp mark 7}{(4.412,3.528)}
\gppoint{gp mark 7}{(4.419,3.526)}
\gppoint{gp mark 7}{(4.426,3.529)}
\gppoint{gp mark 7}{(4.433,3.534)}
\gppoint{gp mark 7}{(4.440,3.536)}
\gppoint{gp mark 7}{(4.446,3.532)}
\gppoint{gp mark 7}{(4.453,3.531)}
\gppoint{gp mark 7}{(4.460,3.531)}
\gppoint{gp mark 7}{(4.467,3.561)}
\gppoint{gp mark 7}{(4.473,3.624)}
\gppoint{gp mark 7}{(4.480,2.531)}
\gppoint{gp mark 7}{(4.487,1.493)}
\gppoint{gp mark 7}{(4.494,1.323)}
\gppoint{gp mark 7}{(4.500,1.330)}
\gppoint{gp mark 7}{(4.507,1.332)}
\gppoint{gp mark 7}{(4.514,1.332)}
\gppoint{gp mark 7}{(4.521,1.331)}
\gppoint{gp mark 7}{(4.528,1.331)}
\gppoint{gp mark 7}{(4.534,1.331)}
\gppoint{gp mark 7}{(4.541,1.331)}
\gppoint{gp mark 7}{(4.548,1.331)}
\gppoint{gp mark 7}{(4.555,1.331)}
\gppoint{gp mark 7}{(4.561,1.331)}
\gppoint{gp mark 7}{(4.568,1.331)}
\gppoint{gp mark 7}{(4.575,1.331)}
\gppoint{gp mark 7}{(4.582,1.331)}
\gppoint{gp mark 7}{(4.589,1.330)}
\gppoint{gp mark 7}{(4.595,1.330)}
\gppoint{gp mark 7}{(4.602,1.330)}
\gppoint{gp mark 7}{(4.609,1.330)}
\gppoint{gp mark 7}{(4.616,1.330)}
\gppoint{gp mark 7}{(4.622,1.330)}
\gppoint{gp mark 7}{(4.629,1.330)}
\gppoint{gp mark 7}{(4.636,1.330)}
\gppoint{gp mark 7}{(4.643,1.329)}
\gppoint{gp mark 7}{(4.649,1.329)}
\gppoint{gp mark 7}{(4.656,1.329)}
\gppoint{gp mark 7}{(4.663,1.329)}
\gppoint{gp mark 7}{(4.670,1.329)}
\gppoint{gp mark 7}{(4.677,1.329)}
\gppoint{gp mark 7}{(4.683,1.329)}
\gppoint{gp mark 7}{(4.690,1.329)}
\gppoint{gp mark 7}{(4.697,1.328)}
\gppoint{gp mark 7}{(4.704,1.328)}
\gppoint{gp mark 7}{(4.710,1.328)}
\gppoint{gp mark 7}{(4.717,1.328)}
\gppoint{gp mark 7}{(4.724,1.328)}
\gppoint{gp mark 7}{(4.731,1.328)}
\gppoint{gp mark 7}{(4.738,1.328)}
\gppoint{gp mark 7}{(4.744,1.328)}
\gppoint{gp mark 7}{(4.751,1.328)}
\gppoint{gp mark 7}{(4.758,1.328)}
\gppoint{gp mark 7}{(4.765,1.327)}
\gppoint{gp mark 7}{(4.771,1.327)}
\gppoint{gp mark 7}{(4.778,1.327)}
\gppoint{gp mark 7}{(4.785,1.327)}
\gppoint{gp mark 7}{(4.792,1.327)}
\gppoint{gp mark 7}{(4.798,1.327)}
\gppoint{gp mark 7}{(4.805,1.327)}
\gppoint{gp mark 7}{(4.812,1.327)}
\gppoint{gp mark 7}{(4.819,1.327)}
\gppoint{gp mark 7}{(4.826,1.327)}
\gppoint{gp mark 7}{(4.832,1.327)}
\gppoint{gp mark 7}{(4.839,1.327)}
\gppoint{gp mark 7}{(4.846,1.327)}
\gppoint{gp mark 7}{(4.853,1.327)}
\gppoint{gp mark 7}{(4.859,1.327)}
\gppoint{gp mark 7}{(4.866,1.327)}
\gppoint{gp mark 7}{(4.873,1.326)}
\gppoint{gp mark 7}{(4.880,1.326)}
\gppoint{gp mark 7}{(4.887,1.326)}
\gppoint{gp mark 7}{(4.893,1.326)}
\gppoint{gp mark 7}{(4.900,1.326)}
\gppoint{gp mark 7}{(4.907,1.326)}
\gppoint{gp mark 7}{(4.914,1.326)}
\gppoint{gp mark 7}{(4.920,1.326)}
\gppoint{gp mark 7}{(4.927,1.326)}
\gppoint{gp mark 7}{(4.934,1.326)}
\gppoint{gp mark 7}{(4.941,1.326)}
\gppoint{gp mark 7}{(4.947,1.326)}
\gppoint{gp mark 7}{(4.954,1.326)}
\gppoint{gp mark 7}{(4.961,1.326)}
\gppoint{gp mark 7}{(4.968,1.326)}
\gppoint{gp mark 7}{(4.975,1.326)}
\gppoint{gp mark 7}{(4.981,1.326)}
\gppoint{gp mark 7}{(4.988,1.326)}
\gppoint{gp mark 7}{(4.995,1.326)}
\gppoint{gp mark 7}{(5.002,1.325)}
\gppoint{gp mark 7}{(5.008,1.325)}
\gppoint{gp mark 7}{(5.015,1.325)}
\gppoint{gp mark 7}{(5.022,1.325)}
\gppoint{gp mark 7}{(5.029,1.325)}
\gppoint{gp mark 7}{(5.036,1.325)}
\gppoint{gp mark 7}{(5.042,1.325)}
\gppoint{gp mark 7}{(5.049,1.325)}
\gppoint{gp mark 7}{(5.056,1.325)}
\gppoint{gp mark 7}{(5.063,1.325)}
\gppoint{gp mark 7}{(5.069,1.325)}
\gppoint{gp mark 7}{(5.076,1.325)}
\gppoint{gp mark 7}{(5.083,1.325)}
\gppoint{gp mark 7}{(5.090,1.325)}
\gppoint{gp mark 7}{(5.096,1.325)}
\gppoint{gp mark 7}{(5.103,1.325)}
\gppoint{gp mark 7}{(5.110,1.325)}
\gppoint{gp mark 7}{(5.117,1.325)}
\gppoint{gp mark 7}{(5.124,1.325)}
\gppoint{gp mark 7}{(5.130,1.325)}
\gppoint{gp mark 7}{(5.137,1.325)}
\gppoint{gp mark 7}{(5.144,1.324)}
\gppoint{gp mark 7}{(5.151,1.324)}
\gppoint{gp mark 7}{(5.157,1.324)}
\gppoint{gp mark 7}{(5.164,1.324)}
\gppoint{gp mark 7}{(5.171,1.324)}
\gppoint{gp mark 7}{(5.178,1.324)}
\gppoint{gp mark 7}{(5.185,1.324)}
\gppoint{gp mark 7}{(5.191,1.324)}
\gppoint{gp mark 7}{(5.198,1.324)}
\gppoint{gp mark 7}{(5.205,1.324)}
\gppoint{gp mark 7}{(5.212,1.324)}
\gppoint{gp mark 7}{(5.218,1.324)}
\gppoint{gp mark 7}{(5.225,1.324)}
\gppoint{gp mark 7}{(5.232,1.324)}
\gppoint{gp mark 7}{(5.239,1.324)}
\gppoint{gp mark 7}{(5.245,1.324)}
\gppoint{gp mark 7}{(5.252,1.324)}
\gppoint{gp mark 7}{(5.259,1.324)}
\gppoint{gp mark 7}{(5.266,1.324)}
\gppoint{gp mark 7}{(5.273,1.324)}
\gppoint{gp mark 7}{(5.279,1.324)}
\gppoint{gp mark 7}{(5.286,1.324)}
\gppoint{gp mark 7}{(5.293,1.324)}
\gppoint{gp mark 7}{(5.300,1.324)}
\gppoint{gp mark 7}{(5.306,1.324)}
\gppoint{gp mark 7}{(5.313,1.324)}
\gppoint{gp mark 7}{(5.320,1.324)}
\gppoint{gp mark 7}{(5.327,1.324)}
\gppoint{gp mark 7}{(5.334,1.324)}
\gppoint{gp mark 7}{(5.340,1.324)}
\gppoint{gp mark 7}{(5.347,1.323)}
\gppoint{gp mark 7}{(5.354,1.323)}
\gppoint{gp mark 7}{(5.361,1.323)}
\gppoint{gp mark 7}{(5.367,1.323)}
\gppoint{gp mark 7}{(5.374,1.323)}
\gppoint{gp mark 7}{(5.381,1.323)}
\gppoint{gp mark 7}{(5.388,1.323)}
\gppoint{gp mark 7}{(5.394,1.323)}
\gppoint{gp mark 7}{(5.401,1.323)}
\gppoint{gp mark 7}{(5.408,1.323)}
\gppoint{gp mark 7}{(5.415,1.323)}
\gppoint{gp mark 7}{(5.422,1.323)}
\gppoint{gp mark 7}{(5.428,1.323)}
\gppoint{gp mark 7}{(5.435,1.323)}
\gppoint{gp mark 7}{(5.442,1.323)}
\gppoint{gp mark 7}{(5.449,1.323)}
\gppoint{gp mark 7}{(5.455,1.323)}
\gppoint{gp mark 7}{(5.462,1.323)}
\gppoint{gp mark 7}{(5.469,1.323)}
\gppoint{gp mark 7}{(5.476,1.323)}
\gppoint{gp mark 7}{(5.483,1.323)}
\gppoint{gp mark 7}{(5.489,1.323)}
\gppoint{gp mark 7}{(5.496,1.323)}
\gppoint{gp mark 7}{(5.503,1.323)}
\gppoint{gp mark 7}{(5.510,1.323)}
\gppoint{gp mark 7}{(5.516,1.323)}
\gppoint{gp mark 7}{(5.523,1.323)}
\gppoint{gp mark 7}{(5.530,1.323)}
\gppoint{gp mark 7}{(5.537,1.323)}
\gppoint{gp mark 7}{(5.543,1.323)}
\gppoint{gp mark 7}{(5.550,1.323)}
\gppoint{gp mark 7}{(5.557,1.323)}
\gppoint{gp mark 7}{(5.564,1.323)}
\gppoint{gp mark 7}{(5.571,1.323)}
\gppoint{gp mark 7}{(5.577,1.323)}
\gppoint{gp mark 7}{(5.584,1.323)}
\gppoint{gp mark 7}{(5.591,1.323)}
\gppoint{gp mark 7}{(5.598,1.323)}
\gppoint{gp mark 7}{(5.604,1.323)}
\gppoint{gp mark 7}{(5.611,1.323)}
\gppoint{gp mark 7}{(5.618,1.323)}
\gppoint{gp mark 7}{(5.625,1.323)}
\gppoint{gp mark 7}{(5.631,1.322)}
\gppoint{gp mark 7}{(5.638,1.322)}
\gppoint{gp mark 7}{(5.645,1.322)}
\gppoint{gp mark 7}{(5.652,1.322)}
\gppoint{gp mark 7}{(5.659,1.323)}
\gppoint{gp mark 7}{(5.665,1.323)}
\gppoint{gp mark 7}{(5.672,1.323)}
\gppoint{gp mark 7}{(5.679,1.322)}
\gppoint{gp mark 7}{(5.686,1.322)}
\gppoint{gp mark 7}{(5.692,1.322)}
\gppoint{gp mark 7}{(5.699,1.322)}
\gppoint{gp mark 7}{(5.706,1.322)}
\gppoint{gp mark 7}{(5.713,1.322)}
\gppoint{gp mark 7}{(5.720,1.322)}
\gppoint{gp mark 7}{(5.726,1.322)}
\gppoint{gp mark 7}{(5.733,1.322)}
\gppoint{gp mark 7}{(5.740,1.322)}
\gppoint{gp mark 7}{(5.747,1.322)}
\gppoint{gp mark 7}{(5.753,1.322)}
\gppoint{gp mark 7}{(5.760,1.322)}
\gppoint{gp mark 7}{(5.767,1.322)}
\gppoint{gp mark 7}{(5.774,1.322)}
\gppoint{gp mark 7}{(5.780,1.322)}
\gppoint{gp mark 7}{(5.787,1.322)}
\gppoint{gp mark 7}{(5.794,1.322)}
\gppoint{gp mark 7}{(5.801,1.322)}
\gppoint{gp mark 7}{(5.808,1.322)}
\gppoint{gp mark 7}{(5.814,1.322)}
\gppoint{gp mark 7}{(5.821,1.322)}
\gppoint{gp mark 7}{(5.828,1.322)}
\gppoint{gp mark 7}{(5.835,1.322)}
\gppoint{gp mark 7}{(5.841,1.322)}
\gppoint{gp mark 7}{(5.848,1.322)}
\gppoint{gp mark 7}{(5.855,1.322)}
\gppoint{gp mark 7}{(5.862,1.322)}
\gppoint{gp mark 7}{(5.869,1.322)}
\gppoint{gp mark 7}{(5.875,1.322)}
\gppoint{gp mark 7}{(5.882,1.322)}
\gppoint{gp mark 7}{(5.889,1.322)}
\gppoint{gp mark 7}{(5.896,1.322)}
\gppoint{gp mark 7}{(5.902,1.322)}
\gppoint{gp mark 7}{(5.909,1.322)}
\gppoint{gp mark 7}{(5.916,1.322)}
\gppoint{gp mark 7}{(5.923,1.322)}
\gppoint{gp mark 7}{(5.929,1.322)}
\gppoint{gp mark 7}{(5.936,1.322)}
\gppoint{gp mark 7}{(5.943,1.322)}
\gppoint{gp mark 7}{(5.950,1.322)}
\gppoint{gp mark 7}{(5.957,1.322)}
\gppoint{gp mark 7}{(5.963,1.322)}
\gppoint{gp mark 7}{(5.970,1.322)}
\gppoint{gp mark 7}{(5.977,1.322)}
\gppoint{gp mark 7}{(5.984,1.322)}
\gppoint{gp mark 7}{(5.990,1.322)}
\gppoint{gp mark 7}{(5.997,1.322)}
\gppoint{gp mark 7}{(6.004,1.322)}
\gppoint{gp mark 7}{(6.011,1.322)}
\gppoint{gp mark 7}{(6.018,1.322)}
\gppoint{gp mark 7}{(6.024,1.322)}
\gppoint{gp mark 7}{(6.031,1.322)}
\gppoint{gp mark 7}{(6.038,1.322)}
\gppoint{gp mark 7}{(6.045,1.322)}
\gppoint{gp mark 7}{(6.051,1.322)}
\gppoint{gp mark 7}{(6.058,1.322)}
\gppoint{gp mark 7}{(6.065,1.322)}
\gppoint{gp mark 7}{(6.072,1.322)}
\gppoint{gp mark 7}{(6.078,1.322)}
\gppoint{gp mark 7}{(6.085,1.322)}
\gppoint{gp mark 7}{(6.092,1.322)}
\gppoint{gp mark 7}{(6.099,1.322)}
\gppoint{gp mark 7}{(6.106,1.322)}
\gppoint{gp mark 7}{(6.112,1.322)}
\gppoint{gp mark 7}{(6.119,1.321)}
\gppoint{gp mark 7}{(6.126,1.321)}
\gppoint{gp mark 7}{(6.133,1.321)}
\gppoint{gp mark 7}{(6.139,1.322)}
\gppoint{gp mark 7}{(6.146,1.322)}
\gppoint{gp mark 7}{(6.153,1.322)}
\gppoint{gp mark 7}{(6.160,1.322)}
\gppoint{gp mark 7}{(6.167,1.321)}
\gppoint{gp mark 7}{(6.173,1.322)}
\gppoint{gp mark 7}{(6.180,1.322)}
\gppoint{gp mark 7}{(6.187,1.322)}
\gppoint{gp mark 7}{(6.194,1.321)}
\gppoint{gp mark 7}{(6.200,1.321)}
\gppoint{gp mark 7}{(6.207,1.321)}
\gppoint{gp mark 7}{(6.214,1.322)}
\gppoint{gp mark 7}{(6.221,1.322)}
\gppoint{gp mark 7}{(6.227,1.321)}
\gppoint{gp mark 7}{(6.234,1.321)}
\gppoint{gp mark 7}{(6.241,1.321)}
\gppoint{gp mark 7}{(6.248,1.321)}
\gppoint{gp mark 7}{(6.255,1.321)}
\gppoint{gp mark 7}{(6.261,1.321)}
\gppoint{gp mark 7}{(6.268,1.321)}
\gppoint{gp mark 7}{(6.275,1.321)}
\gppoint{gp mark 7}{(6.282,1.321)}
\gppoint{gp mark 7}{(6.288,1.322)}
\gppoint{gp mark 7}{(6.295,1.321)}
\gppoint{gp mark 7}{(6.302,1.321)}
\gppoint{gp mark 7}{(6.309,1.321)}
\gppoint{gp mark 7}{(6.316,1.321)}
\gppoint{gp mark 7}{(6.322,1.321)}
\gppoint{gp mark 7}{(6.329,1.321)}
\gppoint{gp mark 7}{(6.336,1.321)}
\gppoint{gp mark 7}{(6.343,1.321)}
\gppoint{gp mark 7}{(6.349,1.321)}
\gppoint{gp mark 7}{(6.356,1.322)}
\gppoint{gp mark 7}{(6.363,1.322)}
\gppoint{gp mark 7}{(6.370,1.321)}
\gppoint{gp mark 7}{(6.376,1.321)}
\gppoint{gp mark 7}{(6.383,1.321)}
\gppoint{gp mark 7}{(6.390,1.322)}
\gppoint{gp mark 7}{(6.397,1.321)}
\gppoint{gp mark 7}{(6.404,1.321)}
\gppoint{gp mark 7}{(6.410,1.321)}
\gppoint{gp mark 7}{(6.417,1.319)}
\gppoint{gp mark 7}{(6.424,1.318)}
\gppoint{gp mark 7}{(6.431,1.318)}
\gppoint{gp mark 7}{(6.437,1.312)}
\gppoint{gp mark 7}{(6.444,1.312)}
\gppoint{gp mark 7}{(6.451,1.314)}
\gppoint{gp mark 7}{(6.458,1.316)}
\gppoint{gp mark 7}{(6.465,1.316)}
\gppoint{gp mark 7}{(6.471,1.316)}
\gppoint{gp mark 7}{(6.478,1.316)}
\gppoint{gp mark 7}{(6.485,1.316)}
\gppoint{gp mark 7}{(6.492,1.316)}
\gppoint{gp mark 7}{(6.498,1.316)}
\gppoint{gp mark 7}{(6.505,1.316)}
\gppoint{gp mark 7}{(6.512,1.316)}
\gppoint{gp mark 7}{(6.519,1.316)}
\gppoint{gp mark 7}{(6.525,1.316)}
\gppoint{gp mark 7}{(6.532,1.315)}
\gppoint{gp mark 7}{(6.539,1.315)}
\gppoint{gp mark 7}{(6.546,1.315)}
\gppoint{gp mark 7}{(6.553,1.315)}
\gppoint{gp mark 7}{(6.559,1.315)}
\gppoint{gp mark 7}{(6.566,1.315)}
\gppoint{gp mark 7}{(6.573,1.315)}
\gppoint{gp mark 7}{(6.580,1.315)}
\gppoint{gp mark 7}{(6.586,1.315)}
\gppoint{gp mark 7}{(6.593,1.315)}
\gppoint{gp mark 7}{(6.600,1.315)}
\gppoint{gp mark 7}{(6.607,1.315)}
\gppoint{gp mark 7}{(6.614,1.315)}
\gppoint{gp mark 7}{(6.620,1.315)}
\gppoint{gp mark 7}{(6.627,1.315)}
\gppoint{gp mark 7}{(6.634,1.315)}
\gppoint{gp mark 7}{(6.641,1.315)}
\gppoint{gp mark 7}{(6.647,1.314)}
\gppoint{gp mark 7}{(6.654,1.314)}
\gppoint{gp mark 7}{(6.661,1.314)}
\gppoint{gp mark 7}{(6.668,1.314)}
\gppoint{gp mark 7}{(6.674,1.314)}
\gppoint{gp mark 7}{(6.681,1.314)}
\gppoint{gp mark 7}{(6.688,1.314)}
\gppoint{gp mark 7}{(6.695,1.314)}
\gppoint{gp mark 7}{(6.702,1.314)}
\gppoint{gp mark 7}{(6.708,1.313)}
\gppoint{gp mark 7}{(6.715,1.313)}
\gppoint{gp mark 7}{(6.722,1.313)}
\gppoint{gp mark 7}{(6.729,1.313)}
\gppoint{gp mark 7}{(6.735,1.313)}
\gppoint{gp mark 7}{(6.742,1.313)}
\gppoint{gp mark 7}{(6.749,1.312)}
\gppoint{gp mark 7}{(6.756,1.312)}
\gppoint{gp mark 7}{(6.762,1.312)}
\gppoint{gp mark 7}{(6.769,1.312)}
\gppoint{gp mark 7}{(6.776,1.311)}
\gppoint{gp mark 7}{(6.783,1.311)}
\gppoint{gp mark 7}{(6.790,1.311)}
\gppoint{gp mark 7}{(6.796,1.311)}
\gppoint{gp mark 7}{(6.803,1.308)}
\gppoint{gp mark 7}{(6.810,1.299)}
\gppoint{gp mark 7}{(6.817,1.287)}
\gppoint{gp mark 7}{(6.823,1.283)}
\gppoint{gp mark 7}{(6.830,1.283)}
\gppoint{gp mark 7}{(6.837,1.283)}
\gppoint{gp mark 7}{(6.844,1.284)}
\gppoint{gp mark 7}{(6.851,1.286)}
\gppoint{gp mark 7}{(6.857,1.289)}
\gppoint{gp mark 7}{(6.864,1.291)}
\gppoint{gp mark 7}{(6.871,1.292)}
\gppoint{gp mark 7}{(6.878,1.293)}
\gppoint{gp mark 7}{(6.884,1.294)}
\gppoint{gp mark 7}{(6.891,1.296)}
\gppoint{gp mark 7}{(6.898,1.297)}
\gppoint{gp mark 7}{(6.905,1.298)}
\gppoint{gp mark 7}{(6.911,1.300)}
\gppoint{gp mark 7}{(6.918,1.301)}
\gppoint{gp mark 7}{(6.925,1.302)}
\gppoint{gp mark 7}{(6.932,1.304)}
\gppoint{gp mark 7}{(6.939,1.305)}
\gppoint{gp mark 7}{(6.945,1.307)}
\gppoint{gp mark 7}{(6.952,1.308)}
\gppoint{gp mark 7}{(6.959,1.309)}
\gppoint{gp mark 7}{(6.966,1.311)}
\gppoint{gp mark 7}{(6.972,1.312)}
\gppoint{gp mark 7}{(6.979,1.314)}
\gppoint{gp mark 7}{(6.986,1.315)}
\gppoint{gp mark 7}{(6.993,1.316)}
\gppoint{gp mark 7}{(7.000,1.318)}
\gppoint{gp mark 7}{(7.006,1.319)}
\gppoint{gp mark 7}{(7.013,1.320)}
\gppoint{gp mark 7}{(7.020,1.322)}
\gppoint{gp mark 7}{(7.027,1.323)}
\gppoint{gp mark 7}{(7.033,1.325)}
\gppoint{gp mark 7}{(7.040,1.326)}
\gppoint{gp mark 7}{(7.047,1.327)}
\gppoint{gp mark 7}{(7.054,1.329)}
\gppoint{gp mark 7}{(7.060,1.330)}
\gppoint{gp mark 7}{(7.067,1.332)}
\gppoint{gp mark 7}{(7.074,1.333)}
\gppoint{gp mark 7}{(7.081,1.335)}
\gppoint{gp mark 7}{(7.088,1.336)}
\gppoint{gp mark 7}{(7.094,1.337)}
\gppoint{gp mark 7}{(7.101,1.339)}
\gppoint{gp mark 7}{(7.108,1.340)}
\gppoint{gp mark 7}{(7.115,1.342)}
\gppoint{gp mark 7}{(7.121,1.343)}
\gppoint{gp mark 7}{(7.128,1.344)}
\gppoint{gp mark 7}{(7.135,1.346)}
\gppoint{gp mark 7}{(7.142,1.347)}
\gppoint{gp mark 7}{(7.149,1.349)}
\gppoint{gp mark 7}{(7.155,1.350)}
\gppoint{gp mark 7}{(7.162,1.351)}
\gppoint{gp mark 7}{(7.169,1.353)}
\gppoint{gp mark 7}{(7.176,1.354)}
\gppoint{gp mark 7}{(7.182,1.356)}
\gppoint{gp mark 7}{(7.189,1.357)}
\gppoint{gp mark 7}{(7.196,1.359)}
\gppoint{gp mark 7}{(7.203,1.360)}
\gppoint{gp mark 7}{(7.209,1.361)}
\gppoint{gp mark 7}{(7.216,1.363)}
\gppoint{gp mark 7}{(7.223,1.364)}
\gppoint{gp mark 7}{(7.230,1.366)}
\gppoint{gp mark 7}{(7.237,1.367)}
\gppoint{gp mark 7}{(7.243,1.368)}
\gppoint{gp mark 7}{(7.250,1.370)}
\gppoint{gp mark 7}{(7.257,1.371)}
\gppoint{gp mark 7}{(7.264,1.373)}
\gppoint{gp mark 7}{(7.270,1.374)}
\gppoint{gp mark 7}{(7.277,1.376)}
\gppoint{gp mark 7}{(7.284,1.377)}
\gppoint{gp mark 7}{(7.291,1.378)}
\gppoint{gp mark 7}{(7.298,1.380)}
\gppoint{gp mark 7}{(7.304,1.381)}
\gppoint{gp mark 7}{(7.311,1.383)}
\gppoint{gp mark 7}{(7.318,1.384)}
\gppoint{gp mark 7}{(7.325,1.386)}
\gppoint{gp mark 7}{(7.331,1.387)}
\gppoint{gp mark 7}{(7.338,1.389)}
\gppoint{gp mark 7}{(7.345,1.390)}
\gppoint{gp mark 7}{(7.352,1.391)}
\gppoint{gp mark 7}{(7.358,1.393)}
\gppoint{gp mark 7}{(7.365,1.394)}
\gppoint{gp mark 7}{(7.372,1.396)}
\gppoint{gp mark 7}{(7.379,1.397)}
\gppoint{gp mark 7}{(7.386,1.399)}
\gppoint{gp mark 7}{(7.392,1.400)}
\gppoint{gp mark 7}{(7.399,1.401)}
\gppoint{gp mark 7}{(7.406,1.403)}
\gppoint{gp mark 7}{(7.413,1.404)}
\gppoint{gp mark 7}{(7.419,1.406)}
\gppoint{gp mark 7}{(7.426,1.407)}
\gppoint{gp mark 7}{(7.433,1.409)}
\gppoint{gp mark 7}{(7.440,1.410)}
\gppoint{gp mark 7}{(7.447,1.412)}
\gppoint{gp mark 7}{(7.453,1.413)}
\gppoint{gp mark 7}{(7.460,1.414)}
\gppoint{gp mark 7}{(7.467,1.416)}
\gppoint{gp mark 7}{(7.474,1.417)}
\gppoint{gp mark 7}{(7.480,1.419)}
\gppoint{gp mark 7}{(7.487,1.420)}
\gppoint{gp mark 7}{(7.494,1.422)}
\gppoint{gp mark 7}{(7.501,1.423)}
\gppoint{gp mark 7}{(7.507,1.425)}
\gppoint{gp mark 7}{(7.514,1.426)}
\gppoint{gp mark 7}{(7.521,1.427)}
\gppoint{gp mark 7}{(7.528,1.429)}
\gppoint{gp mark 7}{(7.535,1.430)}
\gppoint{gp mark 7}{(7.541,1.432)}
\gppoint{gp mark 7}{(7.548,1.433)}
\gppoint{gp mark 7}{(7.555,1.435)}
\gppoint{gp mark 7}{(7.562,1.436)}
\gppoint{gp mark 7}{(7.568,1.438)}
\gppoint{gp mark 7}{(7.575,1.439)}
\gppoint{gp mark 7}{(7.582,1.441)}
\gppoint{gp mark 7}{(7.589,1.442)}
\gppoint{gp mark 7}{(7.596,1.443)}
\gppoint{gp mark 7}{(7.602,1.445)}
\gppoint{gp mark 7}{(7.609,1.446)}
\gppoint{gp mark 7}{(7.616,1.448)}
\gppoint{gp mark 7}{(7.623,1.449)}
\gppoint{gp mark 7}{(7.629,1.451)}
\gppoint{gp mark 7}{(7.636,1.452)}
\gppoint{gp mark 7}{(7.643,1.454)}
\gppoint{gp mark 7}{(7.650,1.455)}
\gppoint{gp mark 7}{(7.656,1.457)}
\gppoint{gp mark 7}{(7.663,1.458)}
\gppoint{gp mark 7}{(7.670,1.460)}
\gppoint{gp mark 7}{(7.677,1.461)}
\gppoint{gp mark 7}{(7.684,1.463)}
\gppoint{gp mark 7}{(7.690,1.464)}
\gppoint{gp mark 7}{(7.697,1.465)}
\gppoint{gp mark 7}{(7.704,1.467)}
\gppoint{gp mark 7}{(7.711,1.468)}
\gppoint{gp mark 7}{(7.717,1.470)}
\gppoint{gp mark 7}{(7.724,1.471)}
\gppoint{gp mark 7}{(7.731,1.473)}
\gppoint{gp mark 7}{(7.738,1.474)}
\gppoint{gp mark 7}{(7.745,1.476)}
\gppoint{gp mark 7}{(7.751,1.477)}
\gppoint{gp mark 7}{(7.758,1.479)}
\gppoint{gp mark 7}{(7.765,1.480)}
\gppoint{gp mark 7}{(7.772,1.482)}
\gppoint{gp mark 7}{(7.778,1.483)}
\gppoint{gp mark 7}{(7.785,1.485)}
\gppoint{gp mark 7}{(7.792,1.486)}
\gppoint{gp mark 7}{(7.799,1.488)}
\gppoint{gp mark 7}{(7.805,1.489)}
\gppoint{gp mark 7}{(7.812,1.491)}
\gppoint{gp mark 7}{(7.819,1.492)}
\gppoint{gp mark 7}{(7.826,1.494)}
\gppoint{gp mark 7}{(7.833,1.495)}
\gppoint{gp mark 7}{(7.839,1.496)}
\gppoint{gp mark 7}{(7.846,1.498)}
\gppoint{gp mark 7}{(7.853,1.499)}
\gppoint{gp mark 7}{(7.860,1.501)}
\gppoint{gp mark 7}{(7.866,1.502)}
\gppoint{gp mark 7}{(7.873,1.504)}
\gppoint{gp mark 7}{(7.880,1.505)}
\gppoint{gp mark 7}{(7.887,1.507)}
\gppoint{gp mark 7}{(7.893,1.508)}
\gppoint{gp mark 7}{(7.900,1.510)}
\gppoint{gp mark 7}{(7.907,1.511)}
\gppoint{gp mark 7}{(7.914,1.513)}
\gppoint{gp mark 7}{(7.921,1.514)}
\gppoint{gp mark 7}{(7.927,1.516)}
\gppoint{gp mark 7}{(7.934,1.517)}
\gppoint{gp mark 7}{(7.941,1.519)}
\gpcolor{rgb color={1.000,0.000,0.000}}
\gpsetlinewidth{4.00}
\gppoint{gp mark 7}{(1.019,2.865)}
\gppoint{gp mark 7}{(1.026,2.861)}
\gppoint{gp mark 7}{(1.033,2.857)}
\gppoint{gp mark 7}{(1.040,2.853)}
\gppoint{gp mark 7}{(1.047,2.849)}
\gppoint{gp mark 7}{(1.053,2.845)}
\gppoint{gp mark 7}{(1.060,2.840)}
\gppoint{gp mark 7}{(1.067,2.836)}
\gppoint{gp mark 7}{(1.074,2.832)}
\gppoint{gp mark 7}{(1.080,2.828)}
\gppoint{gp mark 7}{(1.087,2.824)}
\gppoint{gp mark 7}{(1.094,2.820)}
\gppoint{gp mark 7}{(1.101,2.816)}
\gppoint{gp mark 7}{(1.107,2.812)}
\gppoint{gp mark 7}{(1.114,2.808)}
\gppoint{gp mark 7}{(1.121,2.804)}
\gppoint{gp mark 7}{(1.128,2.800)}
\gppoint{gp mark 7}{(1.135,2.796)}
\gppoint{gp mark 7}{(1.141,2.792)}
\gppoint{gp mark 7}{(1.148,2.787)}
\gppoint{gp mark 7}{(1.155,2.783)}
\gppoint{gp mark 7}{(1.162,2.779)}
\gppoint{gp mark 7}{(1.168,2.775)}
\gppoint{gp mark 7}{(1.175,2.771)}
\gppoint{gp mark 7}{(1.182,2.767)}
\gppoint{gp mark 7}{(1.189,2.763)}
\gppoint{gp mark 7}{(1.196,2.759)}
\gppoint{gp mark 7}{(1.202,2.755)}
\gppoint{gp mark 7}{(1.209,2.751)}
\gppoint{gp mark 7}{(1.216,2.747)}
\gppoint{gp mark 7}{(1.223,2.743)}
\gppoint{gp mark 7}{(1.229,2.739)}
\gppoint{gp mark 7}{(1.236,2.735)}
\gppoint{gp mark 7}{(1.243,2.731)}
\gppoint{gp mark 7}{(1.250,2.727)}
\gppoint{gp mark 7}{(1.256,2.723)}
\gppoint{gp mark 7}{(1.263,2.719)}
\gppoint{gp mark 7}{(1.270,2.715)}
\gppoint{gp mark 7}{(1.277,2.711)}
\gppoint{gp mark 7}{(1.284,2.707)}
\gppoint{gp mark 7}{(1.290,2.703)}
\gppoint{gp mark 7}{(1.297,2.699)}
\gppoint{gp mark 7}{(1.304,2.695)}
\gppoint{gp mark 7}{(1.311,2.691)}
\gppoint{gp mark 7}{(1.317,2.687)}
\gppoint{gp mark 7}{(1.324,2.683)}
\gppoint{gp mark 7}{(1.331,2.679)}
\gppoint{gp mark 7}{(1.338,2.675)}
\gppoint{gp mark 7}{(1.345,2.671)}
\gppoint{gp mark 7}{(1.351,2.667)}
\gppoint{gp mark 7}{(1.358,2.663)}
\gppoint{gp mark 7}{(1.365,2.659)}
\gppoint{gp mark 7}{(1.372,2.655)}
\gppoint{gp mark 7}{(1.378,2.651)}
\gppoint{gp mark 7}{(1.385,2.647)}
\gppoint{gp mark 7}{(1.392,2.643)}
\gppoint{gp mark 7}{(1.399,2.639)}
\gppoint{gp mark 7}{(1.405,2.635)}
\gppoint{gp mark 7}{(1.412,2.631)}
\gppoint{gp mark 7}{(1.419,2.627)}
\gppoint{gp mark 7}{(1.426,2.623)}
\gppoint{gp mark 7}{(1.433,2.619)}
\gppoint{gp mark 7}{(1.439,2.615)}
\gppoint{gp mark 7}{(1.446,2.611)}
\gppoint{gp mark 7}{(1.453,2.607)}
\gppoint{gp mark 7}{(1.460,2.603)}
\gppoint{gp mark 7}{(1.466,2.599)}
\gppoint{gp mark 7}{(1.473,2.595)}
\gppoint{gp mark 7}{(1.480,2.591)}
\gppoint{gp mark 7}{(1.487,2.587)}
\gppoint{gp mark 7}{(1.494,2.583)}
\gppoint{gp mark 7}{(1.500,2.579)}
\gppoint{gp mark 7}{(1.507,2.575)}
\gppoint{gp mark 7}{(1.514,2.571)}
\gppoint{gp mark 7}{(1.521,2.568)}
\gppoint{gp mark 7}{(1.527,2.564)}
\gppoint{gp mark 7}{(1.534,2.560)}
\gppoint{gp mark 7}{(1.541,2.557)}
\gppoint{gp mark 7}{(1.548,2.554)}
\gppoint{gp mark 7}{(1.554,2.553)}
\gppoint{gp mark 7}{(1.561,2.552)}
\gppoint{gp mark 7}{(1.568,2.552)}
\gppoint{gp mark 7}{(1.575,2.552)}
\gppoint{gp mark 7}{(1.582,2.552)}
\gppoint{gp mark 7}{(1.588,2.553)}
\gppoint{gp mark 7}{(1.595,2.554)}
\gppoint{gp mark 7}{(1.602,2.555)}
\gppoint{gp mark 7}{(1.609,2.556)}
\gppoint{gp mark 7}{(1.615,2.556)}
\gppoint{gp mark 7}{(1.622,2.556)}
\gppoint{gp mark 7}{(1.629,2.556)}
\gppoint{gp mark 7}{(1.636,2.556)}
\gppoint{gp mark 7}{(1.643,2.556)}
\gppoint{gp mark 7}{(1.649,2.556)}
\gppoint{gp mark 7}{(1.656,2.556)}
\gppoint{gp mark 7}{(1.663,2.556)}
\gppoint{gp mark 7}{(1.670,2.556)}
\gppoint{gp mark 7}{(1.676,2.557)}
\gppoint{gp mark 7}{(1.683,2.557)}
\gppoint{gp mark 7}{(1.690,2.557)}
\gppoint{gp mark 7}{(1.697,2.557)}
\gppoint{gp mark 7}{(1.703,2.557)}
\gppoint{gp mark 7}{(1.710,2.558)}
\gppoint{gp mark 7}{(1.717,2.558)}
\gppoint{gp mark 7}{(1.724,2.558)}
\gppoint{gp mark 7}{(1.731,2.558)}
\gppoint{gp mark 7}{(1.737,2.558)}
\gppoint{gp mark 7}{(1.744,2.558)}
\gppoint{gp mark 7}{(1.751,2.558)}
\gppoint{gp mark 7}{(1.758,2.558)}
\gppoint{gp mark 7}{(1.764,2.558)}
\gppoint{gp mark 7}{(1.771,2.558)}
\gppoint{gp mark 7}{(1.778,2.558)}
\gppoint{gp mark 7}{(1.785,2.559)}
\gppoint{gp mark 7}{(1.792,2.559)}
\gppoint{gp mark 7}{(1.798,2.559)}
\gppoint{gp mark 7}{(1.805,2.560)}
\gppoint{gp mark 7}{(1.812,2.560)}
\gppoint{gp mark 7}{(1.819,2.560)}
\gppoint{gp mark 7}{(1.825,2.560)}
\gppoint{gp mark 7}{(1.832,2.560)}
\gppoint{gp mark 7}{(1.839,2.560)}
\gppoint{gp mark 7}{(1.846,2.560)}
\gppoint{gp mark 7}{(1.852,2.560)}
\gppoint{gp mark 7}{(1.859,2.559)}
\gppoint{gp mark 7}{(1.866,2.559)}
\gppoint{gp mark 7}{(1.873,2.559)}
\gppoint{gp mark 7}{(1.880,2.559)}
\gppoint{gp mark 7}{(1.886,2.559)}
\gppoint{gp mark 7}{(1.893,2.559)}
\gppoint{gp mark 7}{(1.900,2.559)}
\gppoint{gp mark 7}{(1.907,2.559)}
\gppoint{gp mark 7}{(1.913,2.527)}
\gppoint{gp mark 7}{(1.920,2.278)}
\gppoint{gp mark 7}{(1.927,2.155)}
\gppoint{gp mark 7}{(1.934,2.411)}
\gppoint{gp mark 7}{(1.941,2.548)}
\gppoint{gp mark 7}{(1.947,2.561)}
\gppoint{gp mark 7}{(1.954,2.563)}
\gppoint{gp mark 7}{(1.961,2.566)}
\gppoint{gp mark 7}{(1.968,2.582)}
\gppoint{gp mark 7}{(1.974,2.582)}
\gppoint{gp mark 7}{(1.981,2.563)}
\gppoint{gp mark 7}{(1.988,2.547)}
\gppoint{gp mark 7}{(1.995,2.549)}
\gppoint{gp mark 7}{(2.001,2.566)}
\gppoint{gp mark 7}{(2.008,2.582)}
\gppoint{gp mark 7}{(2.015,2.583)}
\gppoint{gp mark 7}{(2.022,2.576)}
\gppoint{gp mark 7}{(2.029,2.549)}
\gppoint{gp mark 7}{(2.035,2.541)}
\gppoint{gp mark 7}{(2.042,2.543)}
\gppoint{gp mark 7}{(2.049,2.551)}
\gppoint{gp mark 7}{(2.056,2.564)}
\gppoint{gp mark 7}{(2.062,2.580)}
\gppoint{gp mark 7}{(2.069,2.587)}
\gppoint{gp mark 7}{(2.076,2.587)}
\gppoint{gp mark 7}{(2.083,2.579)}
\gppoint{gp mark 7}{(2.089,2.560)}
\gppoint{gp mark 7}{(2.096,2.553)}
\gppoint{gp mark 7}{(2.103,2.556)}
\gppoint{gp mark 7}{(2.110,2.571)}
\gppoint{gp mark 7}{(2.117,2.578)}
\gppoint{gp mark 7}{(2.123,2.579)}
\gppoint{gp mark 7}{(2.130,2.570)}
\gppoint{gp mark 7}{(2.137,2.555)}
\gppoint{gp mark 7}{(2.144,2.551)}
\gppoint{gp mark 7}{(2.150,2.551)}
\gppoint{gp mark 7}{(2.157,2.555)}
\gppoint{gp mark 7}{(2.164,2.558)}
\gppoint{gp mark 7}{(2.171,2.557)}
\gppoint{gp mark 7}{(2.178,2.556)}
\gppoint{gp mark 7}{(2.184,2.557)}
\gppoint{gp mark 7}{(2.191,2.558)}
\gppoint{gp mark 7}{(2.198,2.562)}
\gppoint{gp mark 7}{(2.205,2.571)}
\gppoint{gp mark 7}{(2.211,2.580)}
\gppoint{gp mark 7}{(2.218,2.581)}
\gppoint{gp mark 7}{(2.225,2.579)}
\gppoint{gp mark 7}{(2.232,2.573)}
\gppoint{gp mark 7}{(2.238,2.571)}
\gppoint{gp mark 7}{(2.245,2.571)}
\gppoint{gp mark 7}{(2.252,2.572)}
\gppoint{gp mark 7}{(2.259,2.573)}
\gppoint{gp mark 7}{(2.266,2.568)}
\gppoint{gp mark 7}{(2.272,2.553)}
\gppoint{gp mark 7}{(2.279,2.543)}
\gppoint{gp mark 7}{(2.286,2.542)}
\gppoint{gp mark 7}{(2.293,2.545)}
\gppoint{gp mark 7}{(2.299,2.557)}
\gppoint{gp mark 7}{(2.306,2.562)}
\gppoint{gp mark 7}{(2.313,2.562)}
\gppoint{gp mark 7}{(2.320,2.557)}
\gppoint{gp mark 7}{(2.327,2.548)}
\gppoint{gp mark 7}{(2.333,2.545)}
\gppoint{gp mark 7}{(2.340,2.546)}
\gppoint{gp mark 7}{(2.347,2.557)}
\gppoint{gp mark 7}{(2.354,2.574)}
\gppoint{gp mark 7}{(2.360,2.579)}
\gppoint{gp mark 7}{(2.367,2.579)}
\gppoint{gp mark 7}{(2.374,2.574)}
\gppoint{gp mark 7}{(2.381,2.565)}
\gppoint{gp mark 7}{(2.387,2.563)}
\gppoint{gp mark 7}{(2.394,2.564)}
\gppoint{gp mark 7}{(2.401,2.569)}
\gppoint{gp mark 7}{(2.408,2.573)}
\gppoint{gp mark 7}{(2.415,2.574)}
\gppoint{gp mark 7}{(2.421,2.572)}
\gppoint{gp mark 7}{(2.428,2.568)}
\gppoint{gp mark 7}{(2.435,2.568)}
\gppoint{gp mark 7}{(2.442,2.569)}
\gppoint{gp mark 7}{(2.448,2.568)}
\gppoint{gp mark 7}{(2.455,2.566)}
\gppoint{gp mark 7}{(2.462,2.564)}
\gppoint{gp mark 7}{(2.469,2.561)}
\gppoint{gp mark 7}{(2.476,2.561)}
\gppoint{gp mark 7}{(2.482,2.566)}
\gppoint{gp mark 7}{(2.489,2.571)}
\gppoint{gp mark 7}{(2.496,2.573)}
\gppoint{gp mark 7}{(2.503,2.572)}
\gppoint{gp mark 7}{(2.509,2.567)}
\gppoint{gp mark 7}{(2.516,2.564)}
\gppoint{gp mark 7}{(2.523,2.563)}
\gppoint{gp mark 7}{(2.530,2.567)}
\gppoint{gp mark 7}{(2.536,2.574)}
\gppoint{gp mark 7}{(2.543,2.576)}
\gppoint{gp mark 7}{(2.550,2.576)}
\gppoint{gp mark 7}{(2.557,2.571)}
\gppoint{gp mark 7}{(2.564,2.567)}
\gppoint{gp mark 7}{(2.570,2.567)}
\gppoint{gp mark 7}{(2.577,2.569)}
\gppoint{gp mark 7}{(2.584,2.574)}
\gppoint{gp mark 7}{(2.591,2.576)}
\gppoint{gp mark 7}{(2.597,2.576)}
\gppoint{gp mark 7}{(2.604,2.572)}
\gppoint{gp mark 7}{(2.611,2.569)}
\gppoint{gp mark 7}{(2.618,2.568)}
\gppoint{gp mark 7}{(2.625,2.569)}
\gppoint{gp mark 7}{(2.631,2.571)}
\gppoint{gp mark 7}{(2.638,2.572)}
\gppoint{gp mark 7}{(2.645,2.571)}
\gppoint{gp mark 7}{(2.652,2.569)}
\gppoint{gp mark 7}{(2.658,2.569)}
\gppoint{gp mark 7}{(2.665,2.569)}
\gppoint{gp mark 7}{(2.672,2.570)}
\gppoint{gp mark 7}{(2.679,2.569)}
\gppoint{gp mark 7}{(2.685,2.569)}
\gppoint{gp mark 7}{(2.692,2.566)}
\gppoint{gp mark 7}{(2.699,2.566)}
\gppoint{gp mark 7}{(2.706,2.567)}
\gppoint{gp mark 7}{(2.713,2.568)}
\gppoint{gp mark 7}{(2.719,2.570)}
\gppoint{gp mark 7}{(2.726,2.569)}
\gppoint{gp mark 7}{(2.733,2.567)}
\gppoint{gp mark 7}{(2.740,2.564)}
\gppoint{gp mark 7}{(2.746,2.563)}
\gppoint{gp mark 7}{(2.753,2.565)}
\gppoint{gp mark 7}{(2.760,2.568)}
\gppoint{gp mark 7}{(2.767,2.570)}
\gppoint{gp mark 7}{(2.774,2.570)}
\gppoint{gp mark 7}{(2.780,2.568)}
\gppoint{gp mark 7}{(2.787,2.564)}
\gppoint{gp mark 7}{(2.794,2.563)}
\gppoint{gp mark 7}{(2.801,2.564)}
\gppoint{gp mark 7}{(2.807,2.566)}
\gppoint{gp mark 7}{(2.814,2.569)}
\gppoint{gp mark 7}{(2.821,2.569)}
\gppoint{gp mark 7}{(2.828,2.568)}
\gppoint{gp mark 7}{(2.834,2.565)}
\gppoint{gp mark 7}{(2.841,2.564)}
\gppoint{gp mark 7}{(2.848,2.564)}
\gppoint{gp mark 7}{(2.855,2.566)}
\gppoint{gp mark 7}{(2.862,2.566)}
\gppoint{gp mark 7}{(2.868,2.567)}
\gppoint{gp mark 7}{(2.875,2.566)}
\gppoint{gp mark 7}{(2.882,2.564)}
\gppoint{gp mark 7}{(2.889,2.564)}
\gppoint{gp mark 7}{(2.895,2.564)}
\gppoint{gp mark 7}{(2.902,2.564)}
\gppoint{gp mark 7}{(2.909,2.563)}
\gppoint{gp mark 7}{(2.916,2.563)}
\gppoint{gp mark 7}{(2.923,2.562)}
\gppoint{gp mark 7}{(2.929,2.563)}
\gppoint{gp mark 7}{(2.936,2.563)}
\gppoint{gp mark 7}{(2.943,2.564)}
\gppoint{gp mark 7}{(2.950,2.564)}
\gppoint{gp mark 7}{(2.956,2.563)}
\gppoint{gp mark 7}{(2.963,2.561)}
\gppoint{gp mark 7}{(2.970,2.559)}
\gppoint{gp mark 7}{(2.977,2.560)}
\gppoint{gp mark 7}{(2.983,2.561)}
\gppoint{gp mark 7}{(2.990,2.562)}
\gppoint{gp mark 7}{(2.997,2.562)}
\gppoint{gp mark 7}{(3.004,2.561)}
\gppoint{gp mark 7}{(3.011,2.559)}
\gppoint{gp mark 7}{(3.017,2.557)}
\gppoint{gp mark 7}{(3.024,2.557)}
\gppoint{gp mark 7}{(3.031,2.558)}
\gppoint{gp mark 7}{(3.038,2.559)}
\gppoint{gp mark 7}{(3.044,2.560)}
\gppoint{gp mark 7}{(3.051,2.559)}
\gppoint{gp mark 7}{(3.058,2.557)}
\gppoint{gp mark 7}{(3.065,2.555)}
\gppoint{gp mark 7}{(3.072,2.555)}
\gppoint{gp mark 7}{(3.078,2.557)}
\gppoint{gp mark 7}{(3.085,2.560)}
\gppoint{gp mark 7}{(3.092,2.563)}
\gppoint{gp mark 7}{(3.099,2.567)}
\gppoint{gp mark 7}{(3.105,2.568)}
\gppoint{gp mark 7}{(3.112,2.568)}
\gppoint{gp mark 7}{(3.119,2.568)}
\gppoint{gp mark 7}{(3.126,2.570)}
\gppoint{gp mark 7}{(3.132,2.571)}
\gppoint{gp mark 7}{(3.139,2.571)}
\gppoint{gp mark 7}{(3.146,2.571)}
\gppoint{gp mark 7}{(3.153,2.571)}
\gppoint{gp mark 7}{(3.160,2.571)}
\gppoint{gp mark 7}{(3.166,2.571)}
\gppoint{gp mark 7}{(3.173,2.571)}
\gppoint{gp mark 7}{(3.180,2.570)}
\gppoint{gp mark 7}{(3.187,2.569)}
\gppoint{gp mark 7}{(3.193,2.568)}
\gppoint{gp mark 7}{(3.200,2.568)}
\gppoint{gp mark 7}{(3.207,2.568)}
\gppoint{gp mark 7}{(3.214,2.568)}
\gppoint{gp mark 7}{(3.220,2.568)}
\gppoint{gp mark 7}{(3.227,2.568)}
\gppoint{gp mark 7}{(3.234,2.566)}
\gppoint{gp mark 7}{(3.241,2.565)}
\gppoint{gp mark 7}{(3.248,2.566)}
\gppoint{gp mark 7}{(3.254,2.568)}
\gppoint{gp mark 7}{(3.261,2.573)}
\gppoint{gp mark 7}{(3.268,2.577)}
\gppoint{gp mark 7}{(3.275,2.578)}
\gppoint{gp mark 7}{(3.281,2.577)}
\gppoint{gp mark 7}{(3.288,2.576)}
\gppoint{gp mark 7}{(3.295,2.577)}
\gppoint{gp mark 7}{(3.302,2.597)}
\gppoint{gp mark 7}{(3.309,2.868)}
\gppoint{gp mark 7}{(3.315,4.452)}
\gppoint{gp mark 7}{(3.322,5.489)}
\gppoint{gp mark 7}{(3.329,5.519)}
\gppoint{gp mark 7}{(3.336,5.523)}
\gppoint{gp mark 7}{(3.342,5.517)}
\gppoint{gp mark 7}{(3.349,5.510)}
\gppoint{gp mark 7}{(3.356,5.508)}
\gppoint{gp mark 7}{(3.363,5.509)}
\gppoint{gp mark 7}{(3.369,5.506)}
\gppoint{gp mark 7}{(3.376,5.491)}
\gppoint{gp mark 7}{(3.383,5.494)}
\gppoint{gp mark 7}{(3.390,5.504)}
\gppoint{gp mark 7}{(3.397,5.511)}
\gppoint{gp mark 7}{(3.403,5.516)}
\gppoint{gp mark 7}{(3.410,5.525)}
\gppoint{gp mark 7}{(3.417,5.524)}
\gppoint{gp mark 7}{(3.424,5.518)}
\gppoint{gp mark 7}{(3.430,5.513)}
\gppoint{gp mark 7}{(3.437,5.510)}
\gppoint{gp mark 7}{(3.444,5.510)}
\gppoint{gp mark 7}{(3.451,5.513)}
\gppoint{gp mark 7}{(3.458,5.516)}
\gppoint{gp mark 7}{(3.464,5.516)}
\gppoint{gp mark 7}{(3.471,5.507)}
\gppoint{gp mark 7}{(3.478,5.490)}
\gppoint{gp mark 7}{(3.485,5.488)}
\gppoint{gp mark 7}{(3.491,5.490)}
\gppoint{gp mark 7}{(3.498,5.489)}
\gppoint{gp mark 7}{(3.505,5.486)}
\gppoint{gp mark 7}{(3.512,5.483)}
\gppoint{gp mark 7}{(3.518,5.481)}
\gppoint{gp mark 7}{(3.525,5.476)}
\gppoint{gp mark 7}{(3.532,5.465)}
\gppoint{gp mark 7}{(3.539,5.452)}
\gppoint{gp mark 7}{(3.546,5.443)}
\gppoint{gp mark 7}{(3.552,5.438)}
\gppoint{gp mark 7}{(3.559,5.424)}
\gppoint{gp mark 7}{(3.566,5.411)}
\gppoint{gp mark 7}{(3.573,5.412)}
\gppoint{gp mark 7}{(3.579,5.421)}
\gppoint{gp mark 7}{(3.586,5.431)}
\gppoint{gp mark 7}{(3.593,5.435)}
\gppoint{gp mark 7}{(3.600,5.437)}
\gppoint{gp mark 7}{(3.607,5.439)}
\gppoint{gp mark 7}{(3.613,5.445)}
\gppoint{gp mark 7}{(3.620,5.452)}
\gppoint{gp mark 7}{(3.627,5.458)}
\gppoint{gp mark 7}{(3.634,5.470)}
\gppoint{gp mark 7}{(3.640,5.488)}
\gppoint{gp mark 7}{(3.647,5.505)}
\gppoint{gp mark 7}{(3.654,5.506)}
\gppoint{gp mark 7}{(3.661,5.504)}
\gppoint{gp mark 7}{(3.667,5.498)}
\gppoint{gp mark 7}{(3.674,5.477)}
\gppoint{gp mark 7}{(3.681,5.457)}
\gppoint{gp mark 7}{(3.688,5.437)}
\gppoint{gp mark 7}{(3.695,5.412)}
\gppoint{gp mark 7}{(3.701,5.358)}
\gppoint{gp mark 7}{(3.708,5.308)}
\gppoint{gp mark 7}{(3.715,5.288)}
\gppoint{gp mark 7}{(3.722,5.285)}
\gppoint{gp mark 7}{(3.728,5.288)}
\gppoint{gp mark 7}{(3.735,5.323)}
\gppoint{gp mark 7}{(3.742,5.353)}
\gppoint{gp mark 7}{(3.749,5.355)}
\gppoint{gp mark 7}{(3.756,5.342)}
\gppoint{gp mark 7}{(3.762,5.279)}
\gppoint{gp mark 7}{(3.769,5.245)}
\gppoint{gp mark 7}{(3.776,5.246)}
\gppoint{gp mark 7}{(3.783,5.264)}
\gppoint{gp mark 7}{(3.789,5.281)}
\gppoint{gp mark 7}{(3.796,5.277)}
\gppoint{gp mark 7}{(3.803,5.214)}
\gppoint{gp mark 7}{(3.810,4.790)}
\gppoint{gp mark 7}{(3.816,4.283)}
\gppoint{gp mark 7}{(3.823,4.099)}
\gppoint{gp mark 7}{(3.830,3.886)}
\gppoint{gp mark 7}{(3.837,3.090)}
\gppoint{gp mark 7}{(3.844,1.995)}
\gppoint{gp mark 7}{(3.850,1.564)}
\gppoint{gp mark 7}{(3.857,1.527)}
\gppoint{gp mark 7}{(3.864,1.558)}
\gppoint{gp mark 7}{(3.871,1.706)}
\gppoint{gp mark 7}{(3.877,1.919)}
\gppoint{gp mark 7}{(3.884,2.158)}
\gppoint{gp mark 7}{(3.891,2.419)}
\gppoint{gp mark 7}{(3.898,2.633)}
\gppoint{gp mark 7}{(3.905,2.767)}
\gppoint{gp mark 7}{(3.911,2.864)}
\gppoint{gp mark 7}{(3.918,2.962)}
\gppoint{gp mark 7}{(3.925,3.074)}
\gppoint{gp mark 7}{(3.932,3.160)}
\gppoint{gp mark 7}{(3.938,3.189)}
\gppoint{gp mark 7}{(3.945,3.194)}
\gppoint{gp mark 7}{(3.952,3.193)}
\gppoint{gp mark 7}{(3.959,3.187)}
\gppoint{gp mark 7}{(3.965,3.186)}
\gppoint{gp mark 7}{(3.972,3.188)}
\gppoint{gp mark 7}{(3.979,3.204)}
\gppoint{gp mark 7}{(3.986,3.228)}
\gppoint{gp mark 7}{(3.993,3.253)}
\gppoint{gp mark 7}{(3.999,3.278)}
\gppoint{gp mark 7}{(4.006,3.289)}
\gppoint{gp mark 7}{(4.013,3.304)}
\gppoint{gp mark 7}{(4.020,3.335)}
\gppoint{gp mark 7}{(4.026,3.356)}
\gppoint{gp mark 7}{(4.033,3.371)}
\gppoint{gp mark 7}{(4.040,3.406)}
\gppoint{gp mark 7}{(4.047,3.455)}
\gppoint{gp mark 7}{(4.054,3.475)}
\gppoint{gp mark 7}{(4.060,3.476)}
\gppoint{gp mark 7}{(4.067,3.468)}
\gppoint{gp mark 7}{(4.074,3.457)}
\gppoint{gp mark 7}{(4.081,3.455)}
\gppoint{gp mark 7}{(4.087,3.462)}
\gppoint{gp mark 7}{(4.094,3.472)}
\gppoint{gp mark 7}{(4.101,3.474)}
\gppoint{gp mark 7}{(4.108,3.471)}
\gppoint{gp mark 7}{(4.114,3.470)}
\gppoint{gp mark 7}{(4.121,3.473)}
\gppoint{gp mark 7}{(4.128,3.486)}
\gppoint{gp mark 7}{(4.135,3.499)}
\gppoint{gp mark 7}{(4.142,3.508)}
\gppoint{gp mark 7}{(4.148,3.511)}
\gppoint{gp mark 7}{(4.155,3.509)}
\gppoint{gp mark 7}{(4.162,3.511)}
\gppoint{gp mark 7}{(4.169,3.523)}
\gppoint{gp mark 7}{(4.175,3.533)}
\gppoint{gp mark 7}{(4.182,3.538)}
\gppoint{gp mark 7}{(4.189,3.537)}
\gppoint{gp mark 7}{(4.196,3.535)}
\gppoint{gp mark 7}{(4.203,3.535)}
\gppoint{gp mark 7}{(4.209,3.535)}
\gppoint{gp mark 7}{(4.216,3.536)}
\gppoint{gp mark 7}{(4.223,3.537)}
\gppoint{gp mark 7}{(4.230,3.534)}
\gppoint{gp mark 7}{(4.236,3.525)}
\gppoint{gp mark 7}{(4.243,3.520)}
\gppoint{gp mark 7}{(4.250,3.519)}
\gppoint{gp mark 7}{(4.257,3.520)}
\gppoint{gp mark 7}{(4.263,3.520)}
\gppoint{gp mark 7}{(4.270,3.519)}
\gppoint{gp mark 7}{(4.277,3.519)}
\gppoint{gp mark 7}{(4.284,3.521)}
\gppoint{gp mark 7}{(4.291,3.524)}
\gppoint{gp mark 7}{(4.297,3.528)}
\gppoint{gp mark 7}{(4.304,3.532)}
\gppoint{gp mark 7}{(4.311,3.529)}
\gppoint{gp mark 7}{(4.318,3.525)}
\gppoint{gp mark 7}{(4.324,3.525)}
\gppoint{gp mark 7}{(4.331,3.536)}
\gppoint{gp mark 7}{(4.338,3.541)}
\gppoint{gp mark 7}{(4.345,3.535)}
\gppoint{gp mark 7}{(4.352,3.528)}
\gppoint{gp mark 7}{(4.358,3.530)}
\gppoint{gp mark 7}{(4.365,3.543)}
\gppoint{gp mark 7}{(4.372,3.547)}
\gppoint{gp mark 7}{(4.379,3.542)}
\gppoint{gp mark 7}{(4.385,3.533)}
\gppoint{gp mark 7}{(4.392,3.536)}
\gppoint{gp mark 7}{(4.399,3.550)}
\gppoint{gp mark 7}{(4.406,3.550)}
\gppoint{gp mark 7}{(4.412,3.540)}
\gppoint{gp mark 7}{(4.419,3.527)}
\gppoint{gp mark 7}{(4.426,3.531)}
\gppoint{gp mark 7}{(4.433,3.546)}
\gppoint{gp mark 7}{(4.440,3.551)}
\gppoint{gp mark 7}{(4.446,3.535)}
\gppoint{gp mark 7}{(4.453,2.664)}
\gppoint{gp mark 7}{(4.460,1.494)}
\gppoint{gp mark 7}{(4.467,1.335)}
\gppoint{gp mark 7}{(4.473,1.324)}
\gppoint{gp mark 7}{(4.480,1.323)}
\gppoint{gp mark 7}{(4.487,1.322)}
\gppoint{gp mark 7}{(4.494,1.320)}
\gppoint{gp mark 7}{(4.500,1.320)}
\gppoint{gp mark 7}{(4.507,1.320)}
\gppoint{gp mark 7}{(4.514,1.323)}
\gppoint{gp mark 7}{(4.521,1.327)}
\gppoint{gp mark 7}{(4.528,1.329)}
\gppoint{gp mark 7}{(4.534,1.330)}
\gppoint{gp mark 7}{(4.541,1.331)}
\gppoint{gp mark 7}{(4.548,1.331)}
\gppoint{gp mark 7}{(4.555,1.329)}
\gppoint{gp mark 7}{(4.561,1.324)}
\gppoint{gp mark 7}{(4.568,1.321)}
\gppoint{gp mark 7}{(4.575,1.320)}
\gppoint{gp mark 7}{(4.582,1.320)}
\gppoint{gp mark 7}{(4.589,1.320)}
\gppoint{gp mark 7}{(4.595,1.320)}
\gppoint{gp mark 7}{(4.602,1.321)}
\gppoint{gp mark 7}{(4.609,1.323)}
\gppoint{gp mark 7}{(4.616,1.323)}
\gppoint{gp mark 7}{(4.622,1.324)}
\gppoint{gp mark 7}{(4.629,1.324)}
\gppoint{gp mark 7}{(4.636,1.323)}
\gppoint{gp mark 7}{(4.643,1.322)}
\gppoint{gp mark 7}{(4.649,1.322)}
\gppoint{gp mark 7}{(4.656,1.321)}
\gppoint{gp mark 7}{(4.663,1.322)}
\gppoint{gp mark 7}{(4.670,1.323)}
\gppoint{gp mark 7}{(4.677,1.323)}
\gppoint{gp mark 7}{(4.683,1.323)}
\gppoint{gp mark 7}{(4.690,1.324)}
\gppoint{gp mark 7}{(4.697,1.324)}
\gppoint{gp mark 7}{(4.704,1.324)}
\gppoint{gp mark 7}{(4.710,1.324)}
\gppoint{gp mark 7}{(4.717,1.323)}
\gppoint{gp mark 7}{(4.724,1.321)}
\gppoint{gp mark 7}{(4.731,1.320)}
\gppoint{gp mark 7}{(4.738,1.319)}
\gppoint{gp mark 7}{(4.744,1.319)}
\gppoint{gp mark 7}{(4.751,1.319)}
\gppoint{gp mark 7}{(4.758,1.320)}
\gppoint{gp mark 7}{(4.765,1.321)}
\gppoint{gp mark 7}{(4.771,1.321)}
\gppoint{gp mark 7}{(4.778,1.322)}
\gppoint{gp mark 7}{(4.785,1.322)}
\gppoint{gp mark 7}{(4.792,1.322)}
\gppoint{gp mark 7}{(4.798,1.323)}
\gppoint{gp mark 7}{(4.805,1.323)}
\gppoint{gp mark 7}{(4.812,1.323)}
\gppoint{gp mark 7}{(4.819,1.322)}
\gppoint{gp mark 7}{(4.826,1.321)}
\gppoint{gp mark 7}{(4.832,1.320)}
\gppoint{gp mark 7}{(4.839,1.320)}
\gppoint{gp mark 7}{(4.846,1.320)}
\gppoint{gp mark 7}{(4.853,1.320)}
\gppoint{gp mark 7}{(4.859,1.320)}
\gppoint{gp mark 7}{(4.866,1.320)}
\gppoint{gp mark 7}{(4.873,1.321)}
\gppoint{gp mark 7}{(4.880,1.322)}
\gppoint{gp mark 7}{(4.887,1.322)}
\gppoint{gp mark 7}{(4.893,1.322)}
\gppoint{gp mark 7}{(4.900,1.322)}
\gppoint{gp mark 7}{(4.907,1.322)}
\gppoint{gp mark 7}{(4.914,1.321)}
\gppoint{gp mark 7}{(4.920,1.321)}
\gppoint{gp mark 7}{(4.927,1.321)}
\gppoint{gp mark 7}{(4.934,1.321)}
\gppoint{gp mark 7}{(4.941,1.321)}
\gppoint{gp mark 7}{(4.947,1.321)}
\gppoint{gp mark 7}{(4.954,1.321)}
\gppoint{gp mark 7}{(4.961,1.320)}
\gppoint{gp mark 7}{(4.968,1.321)}
\gppoint{gp mark 7}{(4.975,1.321)}
\gppoint{gp mark 7}{(4.981,1.321)}
\gppoint{gp mark 7}{(4.988,1.321)}
\gppoint{gp mark 7}{(4.995,1.320)}
\gppoint{gp mark 7}{(5.002,1.320)}
\gppoint{gp mark 7}{(5.008,1.320)}
\gppoint{gp mark 7}{(5.015,1.320)}
\gppoint{gp mark 7}{(5.022,1.320)}
\gppoint{gp mark 7}{(5.029,1.320)}
\gppoint{gp mark 7}{(5.036,1.320)}
\gppoint{gp mark 7}{(5.042,1.320)}
\gppoint{gp mark 7}{(5.049,1.319)}
\gppoint{gp mark 7}{(5.056,1.319)}
\gppoint{gp mark 7}{(5.063,1.319)}
\gppoint{gp mark 7}{(5.069,1.319)}
\gppoint{gp mark 7}{(5.076,1.319)}
\gppoint{gp mark 7}{(5.083,1.319)}
\gppoint{gp mark 7}{(5.090,1.319)}
\gppoint{gp mark 7}{(5.096,1.319)}
\gppoint{gp mark 7}{(5.103,1.319)}
\gppoint{gp mark 7}{(5.110,1.319)}
\gppoint{gp mark 7}{(5.117,1.319)}
\gppoint{gp mark 7}{(5.124,1.319)}
\gppoint{gp mark 7}{(5.130,1.319)}
\gppoint{gp mark 7}{(5.137,1.319)}
\gppoint{gp mark 7}{(5.144,1.319)}
\gppoint{gp mark 7}{(5.151,1.319)}
\gppoint{gp mark 7}{(5.157,1.319)}
\gppoint{gp mark 7}{(5.164,1.319)}
\gppoint{gp mark 7}{(5.171,1.319)}
\gppoint{gp mark 7}{(5.178,1.319)}
\gppoint{gp mark 7}{(5.185,1.319)}
\gppoint{gp mark 7}{(5.191,1.319)}
\gppoint{gp mark 7}{(5.198,1.320)}
\gppoint{gp mark 7}{(5.205,1.320)}
\gppoint{gp mark 7}{(5.212,1.319)}
\gppoint{gp mark 7}{(5.218,1.319)}
\gppoint{gp mark 7}{(5.225,1.319)}
\gppoint{gp mark 7}{(5.232,1.319)}
\gppoint{gp mark 7}{(5.239,1.319)}
\gppoint{gp mark 7}{(5.245,1.320)}
\gppoint{gp mark 7}{(5.252,1.320)}
\gppoint{gp mark 7}{(5.259,1.319)}
\gppoint{gp mark 7}{(5.266,1.319)}
\gppoint{gp mark 7}{(5.273,1.319)}
\gppoint{gp mark 7}{(5.279,1.319)}
\gppoint{gp mark 7}{(5.286,1.320)}
\gppoint{gp mark 7}{(5.293,1.321)}
\gppoint{gp mark 7}{(5.300,1.322)}
\gppoint{gp mark 7}{(5.306,1.322)}
\gppoint{gp mark 7}{(5.313,1.322)}
\gppoint{gp mark 7}{(5.320,1.322)}
\gppoint{gp mark 7}{(5.327,1.322)}
\gppoint{gp mark 7}{(5.334,1.323)}
\gppoint{gp mark 7}{(5.340,1.323)}
\gppoint{gp mark 7}{(5.347,1.322)}
\gppoint{gp mark 7}{(5.354,1.321)}
\gppoint{gp mark 7}{(5.361,1.321)}
\gppoint{gp mark 7}{(5.367,1.321)}
\gppoint{gp mark 7}{(5.374,1.322)}
\gppoint{gp mark 7}{(5.381,1.322)}
\gppoint{gp mark 7}{(5.388,1.323)}
\gppoint{gp mark 7}{(5.394,1.323)}
\gppoint{gp mark 7}{(5.401,1.322)}
\gppoint{gp mark 7}{(5.408,1.321)}
\gppoint{gp mark 7}{(5.415,1.321)}
\gppoint{gp mark 7}{(5.422,1.322)}
\gppoint{gp mark 7}{(5.428,1.323)}
\gppoint{gp mark 7}{(5.435,1.323)}
\gppoint{gp mark 7}{(5.442,1.322)}
\gppoint{gp mark 7}{(5.449,1.320)}
\gppoint{gp mark 7}{(5.455,1.317)}
\gppoint{gp mark 7}{(5.462,1.315)}
\gppoint{gp mark 7}{(5.469,1.315)}
\gppoint{gp mark 7}{(5.476,1.315)}
\gppoint{gp mark 7}{(5.483,1.315)}
\gppoint{gp mark 7}{(5.489,1.316)}
\gppoint{gp mark 7}{(5.496,1.317)}
\gppoint{gp mark 7}{(5.503,1.317)}
\gppoint{gp mark 7}{(5.510,1.318)}
\gppoint{gp mark 7}{(5.516,1.317)}
\gppoint{gp mark 7}{(5.523,1.317)}
\gppoint{gp mark 7}{(5.530,1.316)}
\gppoint{gp mark 7}{(5.537,1.316)}
\gppoint{gp mark 7}{(5.543,1.317)}
\gppoint{gp mark 7}{(5.550,1.317)}
\gppoint{gp mark 7}{(5.557,1.318)}
\gppoint{gp mark 7}{(5.564,1.317)}
\gppoint{gp mark 7}{(5.571,1.316)}
\gppoint{gp mark 7}{(5.577,1.316)}
\gppoint{gp mark 7}{(5.584,1.316)}
\gppoint{gp mark 7}{(5.591,1.317)}
\gppoint{gp mark 7}{(5.598,1.319)}
\gppoint{gp mark 7}{(5.604,1.319)}
\gppoint{gp mark 7}{(5.611,1.319)}
\gppoint{gp mark 7}{(5.618,1.319)}
\gppoint{gp mark 7}{(5.625,1.319)}
\gppoint{gp mark 7}{(5.631,1.319)}
\gppoint{gp mark 7}{(5.638,1.320)}
\gppoint{gp mark 7}{(5.645,1.321)}
\gppoint{gp mark 7}{(5.652,1.321)}
\gppoint{gp mark 7}{(5.659,1.320)}
\gppoint{gp mark 7}{(5.665,1.319)}
\gppoint{gp mark 7}{(5.672,1.318)}
\gppoint{gp mark 7}{(5.679,1.318)}
\gppoint{gp mark 7}{(5.686,1.320)}
\gppoint{gp mark 7}{(5.692,1.320)}
\gppoint{gp mark 7}{(5.699,1.320)}
\gppoint{gp mark 7}{(5.706,1.320)}
\gppoint{gp mark 7}{(5.713,1.319)}
\gppoint{gp mark 7}{(5.720,1.319)}
\gppoint{gp mark 7}{(5.726,1.319)}
\gppoint{gp mark 7}{(5.733,1.319)}
\gppoint{gp mark 7}{(5.740,1.320)}
\gppoint{gp mark 7}{(5.747,1.320)}
\gppoint{gp mark 7}{(5.753,1.319)}
\gppoint{gp mark 7}{(5.760,1.319)}
\gppoint{gp mark 7}{(5.767,1.319)}
\gppoint{gp mark 7}{(5.774,1.320)}
\gppoint{gp mark 7}{(5.780,1.320)}
\gppoint{gp mark 7}{(5.787,1.320)}
\gppoint{gp mark 7}{(5.794,1.319)}
\gppoint{gp mark 7}{(5.801,1.319)}
\gppoint{gp mark 7}{(5.808,1.319)}
\gppoint{gp mark 7}{(5.814,1.320)}
\gppoint{gp mark 7}{(5.821,1.320)}
\gppoint{gp mark 7}{(5.828,1.320)}
\gppoint{gp mark 7}{(5.835,1.319)}
\gppoint{gp mark 7}{(5.841,1.318)}
\gppoint{gp mark 7}{(5.848,1.318)}
\gppoint{gp mark 7}{(5.855,1.319)}
\gppoint{gp mark 7}{(5.862,1.322)}
\gppoint{gp mark 7}{(5.869,1.322)}
\gppoint{gp mark 7}{(5.875,1.322)}
\gppoint{gp mark 7}{(5.882,1.322)}
\gppoint{gp mark 7}{(5.889,1.321)}
\gppoint{gp mark 7}{(5.896,1.321)}
\gppoint{gp mark 7}{(5.902,1.322)}
\gppoint{gp mark 7}{(5.909,1.323)}
\gppoint{gp mark 7}{(5.916,1.322)}
\gppoint{gp mark 7}{(5.923,1.319)}
\gppoint{gp mark 7}{(5.929,1.315)}
\gppoint{gp mark 7}{(5.936,1.311)}
\gppoint{gp mark 7}{(5.943,1.310)}
\gppoint{gp mark 7}{(5.950,1.311)}
\gppoint{gp mark 7}{(5.957,1.313)}
\gppoint{gp mark 7}{(5.963,1.314)}
\gppoint{gp mark 7}{(5.970,1.313)}
\gppoint{gp mark 7}{(5.977,1.313)}
\gppoint{gp mark 7}{(5.984,1.315)}
\gppoint{gp mark 7}{(5.990,1.322)}
\gppoint{gp mark 7}{(5.997,1.328)}
\gppoint{gp mark 7}{(6.004,1.330)}
\gppoint{gp mark 7}{(6.011,1.329)}
\gppoint{gp mark 7}{(6.018,1.326)}
\gppoint{gp mark 7}{(6.024,1.321)}
\gppoint{gp mark 7}{(6.031,1.318)}
\gppoint{gp mark 7}{(6.038,1.317)}
\gppoint{gp mark 7}{(6.045,1.318)}
\gppoint{gp mark 7}{(6.051,1.318)}
\gppoint{gp mark 7}{(6.058,1.316)}
\gppoint{gp mark 7}{(6.065,1.314)}
\gppoint{gp mark 7}{(6.072,1.314)}
\gppoint{gp mark 7}{(6.078,1.314)}
\gppoint{gp mark 7}{(6.085,1.316)}
\gppoint{gp mark 7}{(6.092,1.318)}
\gppoint{gp mark 7}{(6.099,1.318)}
\gppoint{gp mark 7}{(6.106,1.318)}
\gppoint{gp mark 7}{(6.112,1.318)}
\gppoint{gp mark 7}{(6.119,1.320)}
\gppoint{gp mark 7}{(6.126,1.321)}
\gppoint{gp mark 7}{(6.133,1.320)}
\gppoint{gp mark 7}{(6.139,1.319)}
\gppoint{gp mark 7}{(6.146,1.317)}
\gppoint{gp mark 7}{(6.153,1.317)}
\gppoint{gp mark 7}{(6.160,1.318)}
\gppoint{gp mark 7}{(6.167,1.321)}
\gppoint{gp mark 7}{(6.173,1.323)}
\gppoint{gp mark 7}{(6.180,1.322)}
\gppoint{gp mark 7}{(6.187,1.318)}
\gppoint{gp mark 7}{(6.194,1.314)}
\gppoint{gp mark 7}{(6.200,1.315)}
\gppoint{gp mark 7}{(6.207,1.317)}
\gppoint{gp mark 7}{(6.214,1.323)}
\gppoint{gp mark 7}{(6.221,1.324)}
\gppoint{gp mark 7}{(6.227,1.323)}
\gppoint{gp mark 7}{(6.234,1.316)}
\gppoint{gp mark 7}{(6.241,1.312)}
\gppoint{gp mark 7}{(6.248,1.312)}
\gppoint{gp mark 7}{(6.255,1.318)}
\gppoint{gp mark 7}{(6.261,1.325)}
\gppoint{gp mark 7}{(6.268,1.325)}
\gppoint{gp mark 7}{(6.275,1.323)}
\gppoint{gp mark 7}{(6.282,1.315)}
\gppoint{gp mark 7}{(6.288,1.313)}
\gppoint{gp mark 7}{(6.295,1.313)}
\gppoint{gp mark 7}{(6.302,1.318)}
\gppoint{gp mark 7}{(6.309,1.324)}
\gppoint{gp mark 7}{(6.316,1.324)}
\gppoint{gp mark 7}{(6.322,1.321)}
\gppoint{gp mark 7}{(6.329,1.319)}
\gppoint{gp mark 7}{(6.336,1.315)}
\gppoint{gp mark 7}{(6.343,1.314)}
\gppoint{gp mark 7}{(6.349,1.317)}
\gppoint{gp mark 7}{(6.356,1.320)}
\gppoint{gp mark 7}{(6.363,1.322)}
\gppoint{gp mark 7}{(6.370,1.323)}
\gppoint{gp mark 7}{(6.376,1.322)}
\gppoint{gp mark 7}{(6.383,1.314)}
\gppoint{gp mark 7}{(6.390,1.312)}
\gppoint{gp mark 7}{(6.397,1.319)}
\gppoint{gp mark 7}{(6.404,1.323)}
\gppoint{gp mark 7}{(6.410,1.325)}
\gppoint{gp mark 7}{(6.417,1.308)}
\gppoint{gp mark 7}{(6.424,1.264)}
\gppoint{gp mark 7}{(6.431,1.188)}
\gppoint{gp mark 7}{(6.437,1.229)}
\gppoint{gp mark 7}{(6.444,1.309)}
\gppoint{gp mark 7}{(6.451,1.317)}
\gppoint{gp mark 7}{(6.458,1.317)}
\gppoint{gp mark 7}{(6.465,1.317)}
\gppoint{gp mark 7}{(6.471,1.317)}
\gppoint{gp mark 7}{(6.478,1.317)}
\gppoint{gp mark 7}{(6.485,1.317)}
\gppoint{gp mark 7}{(6.492,1.317)}
\gppoint{gp mark 7}{(6.498,1.317)}
\gppoint{gp mark 7}{(6.505,1.317)}
\gppoint{gp mark 7}{(6.512,1.317)}
\gppoint{gp mark 7}{(6.519,1.317)}
\gppoint{gp mark 7}{(6.525,1.317)}
\gppoint{gp mark 7}{(6.532,1.317)}
\gppoint{gp mark 7}{(6.539,1.317)}
\gppoint{gp mark 7}{(6.546,1.317)}
\gppoint{gp mark 7}{(6.553,1.317)}
\gppoint{gp mark 7}{(6.559,1.317)}
\gppoint{gp mark 7}{(6.566,1.316)}
\gppoint{gp mark 7}{(6.573,1.316)}
\gppoint{gp mark 7}{(6.580,1.316)}
\gppoint{gp mark 7}{(6.586,1.316)}
\gppoint{gp mark 7}{(6.593,1.316)}
\gppoint{gp mark 7}{(6.600,1.316)}
\gppoint{gp mark 7}{(6.607,1.316)}
\gppoint{gp mark 7}{(6.614,1.316)}
\gppoint{gp mark 7}{(6.620,1.316)}
\gppoint{gp mark 7}{(6.627,1.316)}
\gppoint{gp mark 7}{(6.634,1.316)}
\gppoint{gp mark 7}{(6.641,1.316)}
\gppoint{gp mark 7}{(6.647,1.316)}
\gppoint{gp mark 7}{(6.654,1.316)}
\gppoint{gp mark 7}{(6.661,1.316)}
\gppoint{gp mark 7}{(6.668,1.316)}
\gppoint{gp mark 7}{(6.674,1.316)}
\gppoint{gp mark 7}{(6.681,1.316)}
\gppoint{gp mark 7}{(6.688,1.316)}
\gppoint{gp mark 7}{(6.695,1.316)}
\gppoint{gp mark 7}{(6.702,1.316)}
\gppoint{gp mark 7}{(6.708,1.316)}
\gppoint{gp mark 7}{(6.715,1.316)}
\gppoint{gp mark 7}{(6.722,1.316)}
\gppoint{gp mark 7}{(6.729,1.316)}
\gppoint{gp mark 7}{(6.735,1.316)}
\gppoint{gp mark 7}{(6.742,1.315)}
\gppoint{gp mark 7}{(6.749,1.315)}
\gppoint{gp mark 7}{(6.756,1.315)}
\gppoint{gp mark 7}{(6.762,1.315)}
\gppoint{gp mark 7}{(6.769,1.315)}
\gppoint{gp mark 7}{(6.776,1.315)}
\gppoint{gp mark 7}{(6.783,1.315)}
\gppoint{gp mark 7}{(6.790,1.315)}
\gppoint{gp mark 7}{(6.796,1.314)}
\gppoint{gp mark 7}{(6.803,1.314)}
\gppoint{gp mark 7}{(6.810,1.314)}
\gppoint{gp mark 7}{(6.817,1.314)}
\gppoint{gp mark 7}{(6.823,1.313)}
\gppoint{gp mark 7}{(6.830,1.313)}
\gppoint{gp mark 7}{(6.837,1.313)}
\gppoint{gp mark 7}{(6.844,1.313)}
\gppoint{gp mark 7}{(6.851,1.313)}
\gppoint{gp mark 7}{(6.857,1.313)}
\gppoint{gp mark 7}{(6.864,1.313)}
\gppoint{gp mark 7}{(6.871,1.313)}
\gppoint{gp mark 7}{(6.878,1.311)}
\gppoint{gp mark 7}{(6.884,1.307)}
\gppoint{gp mark 7}{(6.891,1.303)}
\gppoint{gp mark 7}{(6.898,1.301)}
\gppoint{gp mark 7}{(6.905,1.301)}
\gppoint{gp mark 7}{(6.911,1.301)}
\gppoint{gp mark 7}{(6.918,1.301)}
\gppoint{gp mark 7}{(6.925,1.301)}
\gppoint{gp mark 7}{(6.932,1.302)}
\gppoint{gp mark 7}{(6.939,1.304)}
\gppoint{gp mark 7}{(6.945,1.306)}
\gppoint{gp mark 7}{(6.952,1.307)}
\gppoint{gp mark 7}{(6.959,1.309)}
\gppoint{gp mark 7}{(6.966,1.311)}
\gppoint{gp mark 7}{(6.972,1.312)}
\gppoint{gp mark 7}{(6.979,1.314)}
\gppoint{gp mark 7}{(6.986,1.315)}
\gppoint{gp mark 7}{(6.993,1.316)}
\gppoint{gp mark 7}{(7.000,1.318)}
\gppoint{gp mark 7}{(7.006,1.319)}
\gppoint{gp mark 7}{(7.013,1.321)}
\gppoint{gp mark 7}{(7.020,1.322)}
\gppoint{gp mark 7}{(7.027,1.323)}
\gppoint{gp mark 7}{(7.033,1.325)}
\gppoint{gp mark 7}{(7.040,1.326)}
\gppoint{gp mark 7}{(7.047,1.327)}
\gppoint{gp mark 7}{(7.054,1.329)}
\gppoint{gp mark 7}{(7.060,1.330)}
\gppoint{gp mark 7}{(7.067,1.332)}
\gppoint{gp mark 7}{(7.074,1.333)}
\gppoint{gp mark 7}{(7.081,1.334)}
\gppoint{gp mark 7}{(7.088,1.336)}
\gppoint{gp mark 7}{(7.094,1.337)}
\gppoint{gp mark 7}{(7.101,1.339)}
\gppoint{gp mark 7}{(7.108,1.340)}
\gppoint{gp mark 7}{(7.115,1.341)}
\gppoint{gp mark 7}{(7.121,1.343)}
\gppoint{gp mark 7}{(7.128,1.344)}
\gppoint{gp mark 7}{(7.135,1.346)}
\gppoint{gp mark 7}{(7.142,1.347)}
\gppoint{gp mark 7}{(7.149,1.348)}
\gppoint{gp mark 7}{(7.155,1.350)}
\gppoint{gp mark 7}{(7.162,1.351)}
\gppoint{gp mark 7}{(7.169,1.353)}
\gppoint{gp mark 7}{(7.176,1.354)}
\gppoint{gp mark 7}{(7.182,1.355)}
\gppoint{gp mark 7}{(7.189,1.357)}
\gppoint{gp mark 7}{(7.196,1.358)}
\gppoint{gp mark 7}{(7.203,1.360)}
\gppoint{gp mark 7}{(7.209,1.361)}
\gppoint{gp mark 7}{(7.216,1.362)}
\gppoint{gp mark 7}{(7.223,1.364)}
\gppoint{gp mark 7}{(7.230,1.365)}
\gppoint{gp mark 7}{(7.237,1.367)}
\gppoint{gp mark 7}{(7.243,1.368)}
\gppoint{gp mark 7}{(7.250,1.369)}
\gppoint{gp mark 7}{(7.257,1.371)}
\gppoint{gp mark 7}{(7.264,1.372)}
\gppoint{gp mark 7}{(7.270,1.374)}
\gppoint{gp mark 7}{(7.277,1.375)}
\gppoint{gp mark 7}{(7.284,1.377)}
\gppoint{gp mark 7}{(7.291,1.378)}
\gppoint{gp mark 7}{(7.298,1.379)}
\gppoint{gp mark 7}{(7.304,1.381)}
\gppoint{gp mark 7}{(7.311,1.382)}
\gppoint{gp mark 7}{(7.318,1.384)}
\gppoint{gp mark 7}{(7.325,1.385)}
\gppoint{gp mark 7}{(7.331,1.387)}
\gppoint{gp mark 7}{(7.338,1.388)}
\gppoint{gp mark 7}{(7.345,1.389)}
\gppoint{gp mark 7}{(7.352,1.391)}
\gppoint{gp mark 7}{(7.358,1.392)}
\gppoint{gp mark 7}{(7.365,1.394)}
\gppoint{gp mark 7}{(7.372,1.395)}
\gppoint{gp mark 7}{(7.379,1.397)}
\gppoint{gp mark 7}{(7.386,1.398)}
\gppoint{gp mark 7}{(7.392,1.399)}
\gppoint{gp mark 7}{(7.399,1.401)}
\gppoint{gp mark 7}{(7.406,1.402)}
\gppoint{gp mark 7}{(7.413,1.404)}
\gppoint{gp mark 7}{(7.419,1.405)}
\gppoint{gp mark 7}{(7.426,1.407)}
\gppoint{gp mark 7}{(7.433,1.408)}
\gppoint{gp mark 7}{(7.440,1.410)}
\gppoint{gp mark 7}{(7.447,1.411)}
\gppoint{gp mark 7}{(7.453,1.412)}
\gppoint{gp mark 7}{(7.460,1.414)}
\gppoint{gp mark 7}{(7.467,1.415)}
\gppoint{gp mark 7}{(7.474,1.417)}
\gppoint{gp mark 7}{(7.480,1.418)}
\gppoint{gp mark 7}{(7.487,1.420)}
\gppoint{gp mark 7}{(7.494,1.421)}
\gppoint{gp mark 7}{(7.501,1.423)}
\gppoint{gp mark 7}{(7.507,1.424)}
\gppoint{gp mark 7}{(7.514,1.425)}
\gppoint{gp mark 7}{(7.521,1.427)}
\gppoint{gp mark 7}{(7.528,1.428)}
\gppoint{gp mark 7}{(7.535,1.430)}
\gppoint{gp mark 7}{(7.541,1.431)}
\gppoint{gp mark 7}{(7.548,1.433)}
\gppoint{gp mark 7}{(7.555,1.434)}
\gppoint{gp mark 7}{(7.562,1.436)}
\gppoint{gp mark 7}{(7.568,1.437)}
\gppoint{gp mark 7}{(7.575,1.439)}
\gppoint{gp mark 7}{(7.582,1.440)}
\gppoint{gp mark 7}{(7.589,1.441)}
\gppoint{gp mark 7}{(7.596,1.443)}
\gppoint{gp mark 7}{(7.602,1.444)}
\gppoint{gp mark 7}{(7.609,1.446)}
\gppoint{gp mark 7}{(7.616,1.447)}
\gppoint{gp mark 7}{(7.623,1.449)}
\gppoint{gp mark 7}{(7.629,1.450)}
\gppoint{gp mark 7}{(7.636,1.452)}
\gppoint{gp mark 7}{(7.643,1.453)}
\gppoint{gp mark 7}{(7.650,1.455)}
\gppoint{gp mark 7}{(7.656,1.456)}
\gppoint{gp mark 7}{(7.663,1.458)}
\gppoint{gp mark 7}{(7.670,1.459)}
\gppoint{gp mark 7}{(7.677,1.460)}
\gppoint{gp mark 7}{(7.684,1.462)}
\gppoint{gp mark 7}{(7.690,1.463)}
\gppoint{gp mark 7}{(7.697,1.465)}
\gppoint{gp mark 7}{(7.704,1.466)}
\gppoint{gp mark 7}{(7.711,1.468)}
\gppoint{gp mark 7}{(7.717,1.469)}
\gppoint{gp mark 7}{(7.724,1.471)}
\gppoint{gp mark 7}{(7.731,1.472)}
\gppoint{gp mark 7}{(7.738,1.474)}
\gppoint{gp mark 7}{(7.745,1.475)}
\gppoint{gp mark 7}{(7.751,1.477)}
\gppoint{gp mark 7}{(7.758,1.478)}
\gppoint{gp mark 7}{(7.765,1.480)}
\gppoint{gp mark 7}{(7.772,1.481)}
\gppoint{gp mark 7}{(7.778,1.483)}
\gppoint{gp mark 7}{(7.785,1.484)}
\gppoint{gp mark 7}{(7.792,1.486)}
\gppoint{gp mark 7}{(7.799,1.487)}
\gppoint{gp mark 7}{(7.805,1.488)}
\gppoint{gp mark 7}{(7.812,1.490)}
\gppoint{gp mark 7}{(7.819,1.491)}
\gppoint{gp mark 7}{(7.826,1.493)}
\gppoint{gp mark 7}{(7.833,1.494)}
\gppoint{gp mark 7}{(7.839,1.496)}
\gppoint{gp mark 7}{(7.846,1.497)}
\gppoint{gp mark 7}{(7.853,1.499)}
\gppoint{gp mark 7}{(7.860,1.500)}
\gppoint{gp mark 7}{(7.866,1.502)}
\gppoint{gp mark 7}{(7.873,1.503)}
\gppoint{gp mark 7}{(7.880,1.505)}
\gppoint{gp mark 7}{(7.887,1.506)}
\gppoint{gp mark 7}{(7.893,1.508)}
\gppoint{gp mark 7}{(7.900,1.509)}
\gppoint{gp mark 7}{(7.907,1.511)}
\gppoint{gp mark 7}{(7.914,1.512)}
\gppoint{gp mark 7}{(7.921,1.514)}
\gppoint{gp mark 7}{(7.927,1.515)}
\gppoint{gp mark 7}{(7.934,1.517)}
\gppoint{gp mark 7}{(7.941,1.518)}
\gpcolor{rgb color={0.000,0.000,0.000}}
\gpsetlinetype{gp lt plot 0}
\draw[gp path] (1.531,2.562)--(1.929,2.562);
\draw[gp path] (1.929,2.562)--(3.366,2.562);
\draw[gp path] (3.366,5.536)--(3.855,5.536);
\draw[gp path] (3.855,3.568)--(4.388,3.568);
\draw[gp path] (4.388,1.318)--(6.434,1.318);
\draw[gp path] (6.434,1.318)--(6.997,1.318);
\draw[gp path] (1.013,2.853)--(1.017,2.850)--(1.021,2.848)--(1.025,2.846)--(1.028,2.843)%
  --(1.032,2.841)--(1.036,2.839)--(1.040,2.836)--(1.044,2.834)--(1.048,2.832)--(1.051,2.829)%
  --(1.055,2.827)--(1.059,2.825)--(1.063,2.822)--(1.067,2.820)--(1.070,2.818)--(1.074,2.815)%
  --(1.078,2.813)--(1.082,2.811)--(1.086,2.808)--(1.089,2.806)--(1.093,2.804)--(1.097,2.801)%
  --(1.101,2.799)--(1.105,2.797)--(1.108,2.795)--(1.112,2.792)--(1.116,2.790)--(1.120,2.788)%
  --(1.124,2.785)--(1.127,2.783)--(1.131,2.781)--(1.135,2.779)--(1.139,2.776)--(1.143,2.774)%
  --(1.147,2.772)--(1.150,2.770)--(1.154,2.767)--(1.158,2.765)--(1.162,2.763)--(1.166,2.761)%
  --(1.169,2.758)--(1.173,2.756)--(1.177,2.754)--(1.181,2.752)--(1.185,2.749)--(1.188,2.747)%
  --(1.192,2.745)--(1.196,2.743)--(1.200,2.741)--(1.204,2.738)--(1.207,2.736)--(1.211,2.734)%
  --(1.215,2.732)--(1.219,2.730)--(1.223,2.727)--(1.226,2.725)--(1.230,2.723)--(1.234,2.721)%
  --(1.238,2.719)--(1.242,2.716)--(1.246,2.714)--(1.249,2.712)--(1.253,2.710)--(1.257,2.708)%
  --(1.261,2.706)--(1.265,2.703)--(1.268,2.701)--(1.272,2.699)--(1.276,2.697)--(1.280,2.695)%
  --(1.284,2.693)--(1.287,2.691)--(1.291,2.688)--(1.295,2.686)--(1.299,2.684)--(1.303,2.682)%
  --(1.306,2.680)--(1.310,2.678)--(1.314,2.676)--(1.318,2.674)--(1.322,2.672)--(1.325,2.669)%
  --(1.329,2.667)--(1.333,2.665)--(1.337,2.663)--(1.341,2.661)--(1.345,2.659)--(1.348,2.657)%
  --(1.352,2.655)--(1.356,2.653)--(1.360,2.651)--(1.364,2.649)--(1.367,2.647)--(1.371,2.645)%
  --(1.375,2.643)--(1.379,2.640)--(1.383,2.638)--(1.386,2.636)--(1.390,2.634)--(1.394,2.632)%
  --(1.398,2.630)--(1.402,2.628)--(1.405,2.626)--(1.409,2.624)--(1.413,2.622)--(1.417,2.620)%
  --(1.421,2.618)--(1.424,2.616)--(1.428,2.614)--(1.432,2.612)--(1.436,2.610)--(1.440,2.608)%
  --(1.443,2.606)--(1.447,2.604)--(1.451,2.602)--(1.455,2.600)--(1.459,2.598)--(1.463,2.596)%
  --(1.466,2.594)--(1.470,2.593)--(1.474,2.591)--(1.478,2.589)--(1.482,2.587)--(1.485,2.585)%
  --(1.489,2.583)--(1.493,2.581)--(1.497,2.579)--(1.501,2.577)--(1.504,2.575)--(1.508,2.573)%
  --(1.512,2.571)--(1.516,2.569)--(1.520,2.567)--(1.523,2.566)--(1.527,2.564)--(1.531,2.562);
\draw[gp path] (3.366,2.562)--(3.366,5.536);
\draw[gp path] (3.855,5.536)--(3.855,3.568);
\draw[gp path] (4.388,3.568)--(4.388,1.318);
\draw[gp path] (6.997,1.318)--(7.003,1.319)--(7.008,1.319)--(7.013,1.320)--(7.018,1.321)%
  --(7.023,1.322)--(7.029,1.323)--(7.034,1.324)--(7.039,1.325)--(7.044,1.326)--(7.049,1.327)%
  --(7.055,1.328)--(7.060,1.329)--(7.065,1.330)--(7.070,1.331)--(7.075,1.332)--(7.080,1.333)%
  --(7.086,1.334)--(7.091,1.334)--(7.096,1.335)--(7.101,1.336)--(7.106,1.337)--(7.112,1.338)%
  --(7.117,1.339)--(7.122,1.340)--(7.127,1.341)--(7.132,1.342)--(7.138,1.343)--(7.143,1.344)%
  --(7.148,1.345)--(7.153,1.346)--(7.158,1.347)--(7.164,1.348)--(7.169,1.349)--(7.174,1.350)%
  --(7.179,1.351)--(7.184,1.352)--(7.189,1.353)--(7.195,1.354)--(7.200,1.355)--(7.205,1.356)%
  --(7.210,1.357)--(7.215,1.358)--(7.221,1.359)--(7.226,1.360)--(7.231,1.361)--(7.236,1.362)%
  --(7.241,1.363)--(7.247,1.364)--(7.252,1.365)--(7.257,1.366)--(7.262,1.367)--(7.267,1.368)%
  --(7.272,1.369)--(7.278,1.370)--(7.283,1.371)--(7.288,1.372)--(7.293,1.373)--(7.298,1.374)%
  --(7.304,1.375)--(7.309,1.376)--(7.314,1.377)--(7.319,1.378)--(7.324,1.379)--(7.330,1.380)%
  --(7.335,1.381)--(7.340,1.382)--(7.345,1.383)--(7.350,1.384)--(7.355,1.385)--(7.361,1.387)%
  --(7.366,1.388)--(7.371,1.389)--(7.376,1.390)--(7.381,1.391)--(7.387,1.392)--(7.392,1.393)%
  --(7.397,1.394)--(7.402,1.395)--(7.407,1.396)--(7.413,1.397)--(7.418,1.398)--(7.423,1.399)%
  --(7.428,1.400)--(7.433,1.401)--(7.439,1.402)--(7.444,1.403)--(7.449,1.405)--(7.454,1.406)%
  --(7.459,1.407)--(7.464,1.408)--(7.470,1.409)--(7.475,1.410)--(7.480,1.411)--(7.485,1.412)%
  --(7.490,1.413)--(7.496,1.414)--(7.501,1.415)--(7.506,1.416)--(7.511,1.418)--(7.516,1.419)%
  --(7.522,1.420)--(7.527,1.421)--(7.532,1.422)--(7.537,1.423)--(7.542,1.424)--(7.547,1.425)%
  --(7.553,1.426)--(7.558,1.427)--(7.563,1.429)--(7.568,1.430)--(7.573,1.431)--(7.579,1.432)%
  --(7.584,1.433)--(7.589,1.434)--(7.594,1.435)--(7.599,1.436)--(7.605,1.437)--(7.610,1.439)%
  --(7.615,1.440)--(7.620,1.441)--(7.625,1.442)--(7.631,1.443)--(7.636,1.444)--(7.641,1.445)%
  --(7.646,1.446)--(7.651,1.448)--(7.656,1.449)--(7.662,1.450)--(7.667,1.451)--(7.672,1.452)%
  --(7.677,1.453)--(7.682,1.454)--(7.688,1.456)--(7.693,1.457)--(7.698,1.458)--(7.703,1.459)%
  --(7.708,1.460)--(7.714,1.461)--(7.719,1.462)--(7.724,1.464)--(7.729,1.465)--(7.734,1.466)%
  --(7.739,1.467)--(7.745,1.468)--(7.750,1.469)--(7.755,1.470)--(7.760,1.472)--(7.765,1.473)%
  --(7.771,1.474)--(7.776,1.475)--(7.781,1.476)--(7.786,1.477)--(7.791,1.479)--(7.797,1.480)%
  --(7.802,1.481)--(7.807,1.482)--(7.812,1.483)--(7.817,1.485)--(7.823,1.486)--(7.828,1.487)%
  --(7.833,1.488)--(7.838,1.489)--(7.843,1.490)--(7.848,1.492)--(7.854,1.493)--(7.859,1.494)%
  --(7.864,1.495)--(7.869,1.496)--(7.874,1.498)--(7.880,1.499)--(7.885,1.500)--(7.890,1.501)%
  --(7.895,1.502)--(7.900,1.504)--(7.906,1.505)--(7.911,1.506)--(7.916,1.507)--(7.921,1.508)%
  --(7.926,1.510)--(7.931,1.511)--(7.937,1.512)--(7.942,1.513);
\node[gp node left,font={\fontsize{10pt}{12pt}\selectfont}] at (1.151,5.321) {\LARGE $\rho$};
\node[gp node left,font={\fontsize{10pt}{12pt}\selectfont}] at (6.005,5.321) {\large $\alpha = 2.95$};
%% coordinates of the plot area
\gpdefrectangularnode{gp plot 1}{\pgfpoint{1.012cm}{0.985cm}}{\pgfpoint{7.947cm}{5.631cm}}
\end{tikzpicture}
%% gnuplot variables
} & 
\resizebox{0.5\linewidth}{!}{\tikzsetnextfilename{AK7_crsol_6}\begin{tikzpicture}[gnuplot]
%% generated with GNUPLOT 4.6p4 (Lua 5.1; terminal rev. 99, script rev. 100)
%% Fri 22 Aug 2014 08:42:06 AM EDT
\path (0.000,0.000) rectangle (8.500,6.000);
\gpfill{rgb color={1.000,1.000,1.000}} (0.828,0.985)--(7.946,0.985)--(7.946,5.630)--(0.828,5.630)--cycle;
\gpcolor{color=gp lt color border}
\gpsetlinetype{gp lt border}
\gpsetlinewidth{1.00}
\draw[gp path] (0.828,0.985)--(0.828,5.630)--(7.946,5.630)--(7.946,0.985)--cycle;
\gpcolor{color=gp lt color axes}
\gpsetlinetype{gp lt axes}
\gpsetlinewidth{2.00}
\draw[gp path] (0.828,0.985)--(7.947,0.985);
\gpcolor{color=gp lt color border}
\gpsetlinetype{gp lt border}
\draw[gp path] (0.828,0.985)--(0.900,0.985);
\draw[gp path] (7.947,0.985)--(7.875,0.985);
\gpcolor{rgb color={0.000,0.000,0.000}}
\node[gp node right,font={\fontsize{10pt}{12pt}\selectfont}] at (0.644,0.985) {-2};
\gpcolor{color=gp lt color axes}
\gpsetlinetype{gp lt axes}
\draw[gp path] (0.828,1.914)--(7.947,1.914);
\gpcolor{color=gp lt color border}
\gpsetlinetype{gp lt border}
\draw[gp path] (0.828,1.914)--(0.900,1.914);
\draw[gp path] (7.947,1.914)--(7.875,1.914);
\gpcolor{rgb color={0.000,0.000,0.000}}
\node[gp node right,font={\fontsize{10pt}{12pt}\selectfont}] at (0.644,1.914) {-1};
\gpcolor{color=gp lt color axes}
\gpsetlinetype{gp lt axes}
\draw[gp path] (0.828,2.843)--(7.947,2.843);
\gpcolor{color=gp lt color border}
\gpsetlinetype{gp lt border}
\draw[gp path] (0.828,2.843)--(0.900,2.843);
\draw[gp path] (7.947,2.843)--(7.875,2.843);
\gpcolor{rgb color={0.000,0.000,0.000}}
\node[gp node right,font={\fontsize{10pt}{12pt}\selectfont}] at (0.644,2.843) {0};
\gpcolor{color=gp lt color axes}
\gpsetlinetype{gp lt axes}
\draw[gp path] (0.828,3.773)--(7.947,3.773);
\gpcolor{color=gp lt color border}
\gpsetlinetype{gp lt border}
\draw[gp path] (0.828,3.773)--(0.900,3.773);
\draw[gp path] (7.947,3.773)--(7.875,3.773);
\gpcolor{rgb color={0.000,0.000,0.000}}
\node[gp node right,font={\fontsize{10pt}{12pt}\selectfont}] at (0.644,3.773) {1};
\gpcolor{color=gp lt color axes}
\gpsetlinetype{gp lt axes}
\draw[gp path] (0.828,4.702)--(7.947,4.702);
\gpcolor{color=gp lt color border}
\gpsetlinetype{gp lt border}
\draw[gp path] (0.828,4.702)--(0.900,4.702);
\draw[gp path] (7.947,4.702)--(7.875,4.702);
\gpcolor{rgb color={0.000,0.000,0.000}}
\node[gp node right,font={\fontsize{10pt}{12pt}\selectfont}] at (0.644,4.702) {2};
\gpcolor{color=gp lt color axes}
\gpsetlinetype{gp lt axes}
\draw[gp path] (0.828,5.631)--(7.947,5.631);
\gpcolor{color=gp lt color border}
\gpsetlinetype{gp lt border}
\draw[gp path] (0.828,5.631)--(0.900,5.631);
\draw[gp path] (7.947,5.631)--(7.875,5.631);
\gpcolor{rgb color={0.000,0.000,0.000}}
\node[gp node right,font={\fontsize{10pt}{12pt}\selectfont}] at (0.644,5.631) {3};
\gpcolor{color=gp lt color axes}
\gpsetlinetype{gp lt axes}
\draw[gp path] (0.828,0.985)--(0.828,5.631);
\gpcolor{color=gp lt color border}
\gpsetlinetype{gp lt border}
\draw[gp path] (0.828,0.985)--(0.828,1.057);
\draw[gp path] (0.828,5.631)--(0.828,5.559);
\gpcolor{rgb color={0.000,0.000,0.000}}
\node[gp node center,font={\fontsize{10pt}{12pt}\selectfont}] at (0.828,0.677) {0.3};
\gpcolor{color=gp lt color axes}
\gpsetlinetype{gp lt axes}
\draw[gp path] (2.252,0.985)--(2.252,5.631);
\gpcolor{color=gp lt color border}
\gpsetlinetype{gp lt border}
\draw[gp path] (2.252,0.985)--(2.252,1.057);
\draw[gp path] (2.252,5.631)--(2.252,5.559);
\gpcolor{rgb color={0.000,0.000,0.000}}
\node[gp node center,font={\fontsize{10pt}{12pt}\selectfont}] at (2.252,0.677) {0.4};
\gpcolor{color=gp lt color axes}
\gpsetlinetype{gp lt axes}
\draw[gp path] (3.676,0.985)--(3.676,5.631);
\gpcolor{color=gp lt color border}
\gpsetlinetype{gp lt border}
\draw[gp path] (3.676,0.985)--(3.676,1.057);
\draw[gp path] (3.676,5.631)--(3.676,5.559);
\gpcolor{rgb color={0.000,0.000,0.000}}
\node[gp node center,font={\fontsize{10pt}{12pt}\selectfont}] at (3.676,0.677) {0.5};
\gpcolor{color=gp lt color axes}
\gpsetlinetype{gp lt axes}
\draw[gp path] (5.099,0.985)--(5.099,5.631);
\gpcolor{color=gp lt color border}
\gpsetlinetype{gp lt border}
\draw[gp path] (5.099,0.985)--(5.099,1.057);
\draw[gp path] (5.099,5.631)--(5.099,5.559);
\gpcolor{rgb color={0.000,0.000,0.000}}
\node[gp node center,font={\fontsize{10pt}{12pt}\selectfont}] at (5.099,0.677) {0.6};
\gpcolor{color=gp lt color axes}
\gpsetlinetype{gp lt axes}
\draw[gp path] (6.523,0.985)--(6.523,5.631);
\gpcolor{color=gp lt color border}
\gpsetlinetype{gp lt border}
\draw[gp path] (6.523,0.985)--(6.523,1.057);
\draw[gp path] (6.523,5.631)--(6.523,5.559);
\gpcolor{rgb color={0.000,0.000,0.000}}
\node[gp node center,font={\fontsize{10pt}{12pt}\selectfont}] at (6.523,0.677) {0.7};
\gpcolor{color=gp lt color axes}
\gpsetlinetype{gp lt axes}
\draw[gp path] (7.947,0.985)--(7.947,5.631);
\gpcolor{color=gp lt color border}
\gpsetlinetype{gp lt border}
\draw[gp path] (7.947,0.985)--(7.947,1.057);
\draw[gp path] (7.947,5.631)--(7.947,5.559);
\gpcolor{rgb color={0.000,0.000,0.000}}
\node[gp node center,font={\fontsize{10pt}{12pt}\selectfont}] at (7.947,0.677) {0.8};
\gpcolor{color=gp lt color border}
\draw[gp path] (0.828,5.631)--(0.828,0.985)--(7.947,0.985)--(7.947,5.631)--cycle;
\gpcolor{rgb color={0.000,0.000,0.000}}
\node[gp node center,font={\fontsize{10pt}{12pt}\selectfont}] at (4.387,0.215) {\large $x$};
\gpcolor{rgb color={0.502,0.502,0.502}}
\gpsetlinewidth{6.00}
\gpsetpointsize{2.67}
\gppoint{gp mark 7}{(0.836,5.285)}
\gppoint{gp mark 7}{(0.843,5.276)}
\gppoint{gp mark 7}{(0.850,5.267)}
\gppoint{gp mark 7}{(0.857,5.258)}
\gppoint{gp mark 7}{(0.863,5.249)}
\gppoint{gp mark 7}{(0.870,5.240)}
\gppoint{gp mark 7}{(0.877,5.232)}
\gppoint{gp mark 7}{(0.884,5.223)}
\gppoint{gp mark 7}{(0.891,5.214)}
\gppoint{gp mark 7}{(0.898,5.205)}
\gppoint{gp mark 7}{(0.905,5.196)}
\gppoint{gp mark 7}{(0.912,5.187)}
\gppoint{gp mark 7}{(0.919,5.178)}
\gppoint{gp mark 7}{(0.926,5.169)}
\gppoint{gp mark 7}{(0.933,5.160)}
\gppoint{gp mark 7}{(0.940,5.151)}
\gppoint{gp mark 7}{(0.947,5.142)}
\gppoint{gp mark 7}{(0.954,5.133)}
\gppoint{gp mark 7}{(0.961,5.124)}
\gppoint{gp mark 7}{(0.968,5.115)}
\gppoint{gp mark 7}{(0.975,5.105)}
\gppoint{gp mark 7}{(0.982,5.096)}
\gppoint{gp mark 7}{(0.989,5.087)}
\gppoint{gp mark 7}{(0.996,5.078)}
\gppoint{gp mark 7}{(1.002,5.069)}
\gppoint{gp mark 7}{(1.009,5.060)}
\gppoint{gp mark 7}{(1.016,5.050)}
\gppoint{gp mark 7}{(1.023,5.041)}
\gppoint{gp mark 7}{(1.030,5.032)}
\gppoint{gp mark 7}{(1.037,5.023)}
\gppoint{gp mark 7}{(1.044,5.013)}
\gppoint{gp mark 7}{(1.051,5.004)}
\gppoint{gp mark 7}{(1.058,4.995)}
\gppoint{gp mark 7}{(1.065,4.985)}
\gppoint{gp mark 7}{(1.072,4.976)}
\gppoint{gp mark 7}{(1.079,4.967)}
\gppoint{gp mark 7}{(1.086,4.957)}
\gppoint{gp mark 7}{(1.093,4.948)}
\gppoint{gp mark 7}{(1.100,4.938)}
\gppoint{gp mark 7}{(1.107,4.929)}
\gppoint{gp mark 7}{(1.114,4.919)}
\gppoint{gp mark 7}{(1.121,4.910)}
\gppoint{gp mark 7}{(1.128,4.900)}
\gppoint{gp mark 7}{(1.135,4.891)}
\gppoint{gp mark 7}{(1.142,4.881)}
\gppoint{gp mark 7}{(1.148,4.872)}
\gppoint{gp mark 7}{(1.155,4.862)}
\gppoint{gp mark 7}{(1.162,4.852)}
\gppoint{gp mark 7}{(1.169,4.843)}
\gppoint{gp mark 7}{(1.176,4.833)}
\gppoint{gp mark 7}{(1.183,4.823)}
\gppoint{gp mark 7}{(1.190,4.813)}
\gppoint{gp mark 7}{(1.197,4.804)}
\gppoint{gp mark 7}{(1.204,4.794)}
\gppoint{gp mark 7}{(1.211,4.784)}
\gppoint{gp mark 7}{(1.218,4.774)}
\gppoint{gp mark 7}{(1.225,4.764)}
\gppoint{gp mark 7}{(1.232,4.754)}
\gppoint{gp mark 7}{(1.239,4.744)}
\gppoint{gp mark 7}{(1.246,4.734)}
\gppoint{gp mark 7}{(1.253,4.724)}
\gppoint{gp mark 7}{(1.260,4.714)}
\gppoint{gp mark 7}{(1.267,4.704)}
\gppoint{gp mark 7}{(1.274,4.694)}
\gppoint{gp mark 7}{(1.281,4.684)}
\gppoint{gp mark 7}{(1.288,4.674)}
\gppoint{gp mark 7}{(1.294,4.664)}
\gppoint{gp mark 7}{(1.301,4.653)}
\gppoint{gp mark 7}{(1.308,4.643)}
\gppoint{gp mark 7}{(1.315,4.633)}
\gppoint{gp mark 7}{(1.322,4.622)}
\gppoint{gp mark 7}{(1.329,4.612)}
\gppoint{gp mark 7}{(1.336,4.602)}
\gppoint{gp mark 7}{(1.343,4.591)}
\gppoint{gp mark 7}{(1.350,4.581)}
\gppoint{gp mark 7}{(1.357,4.570)}
\gppoint{gp mark 7}{(1.364,4.560)}
\gppoint{gp mark 7}{(1.371,4.549)}
\gppoint{gp mark 7}{(1.378,4.538)}
\gppoint{gp mark 7}{(1.385,4.528)}
\gppoint{gp mark 7}{(1.392,4.517)}
\gppoint{gp mark 7}{(1.399,4.506)}
\gppoint{gp mark 7}{(1.406,4.495)}
\gppoint{gp mark 7}{(1.413,4.485)}
\gppoint{gp mark 7}{(1.420,4.474)}
\gppoint{gp mark 7}{(1.427,4.463)}
\gppoint{gp mark 7}{(1.434,4.452)}
\gppoint{gp mark 7}{(1.440,4.441)}
\gppoint{gp mark 7}{(1.447,4.430)}
\gppoint{gp mark 7}{(1.454,4.419)}
\gppoint{gp mark 7}{(1.461,4.407)}
\gppoint{gp mark 7}{(1.468,4.396)}
\gppoint{gp mark 7}{(1.475,4.385)}
\gppoint{gp mark 7}{(1.482,4.373)}
\gppoint{gp mark 7}{(1.489,4.362)}
\gppoint{gp mark 7}{(1.496,4.351)}
\gppoint{gp mark 7}{(1.503,4.339)}
\gppoint{gp mark 7}{(1.510,4.328)}
\gppoint{gp mark 7}{(1.517,4.316)}
\gppoint{gp mark 7}{(1.524,4.304)}
\gppoint{gp mark 7}{(1.531,4.293)}
\gppoint{gp mark 7}{(1.538,4.281)}
\gppoint{gp mark 7}{(1.545,4.269)}
\gppoint{gp mark 7}{(1.552,4.257)}
\gppoint{gp mark 7}{(1.559,4.245)}
\gppoint{gp mark 7}{(1.566,4.233)}
\gppoint{gp mark 7}{(1.573,4.221)}
\gppoint{gp mark 7}{(1.580,4.209)}
\gppoint{gp mark 7}{(1.586,4.197)}
\gppoint{gp mark 7}{(1.593,4.184)}
\gppoint{gp mark 7}{(1.600,4.172)}
\gppoint{gp mark 7}{(1.607,4.160)}
\gppoint{gp mark 7}{(1.614,4.147)}
\gppoint{gp mark 7}{(1.621,4.135)}
\gppoint{gp mark 7}{(1.628,4.122)}
\gppoint{gp mark 7}{(1.635,4.110)}
\gppoint{gp mark 7}{(1.642,4.097)}
\gppoint{gp mark 7}{(1.649,4.085)}
\gppoint{gp mark 7}{(1.656,4.072)}
\gppoint{gp mark 7}{(1.663,4.060)}
\gppoint{gp mark 7}{(1.670,4.048)}
\gppoint{gp mark 7}{(1.677,4.036)}
\gppoint{gp mark 7}{(1.684,4.024)}
\gppoint{gp mark 7}{(1.691,4.013)}
\gppoint{gp mark 7}{(1.698,4.002)}
\gppoint{gp mark 7}{(1.705,3.992)}
\gppoint{gp mark 7}{(1.712,3.983)}
\gppoint{gp mark 7}{(1.719,3.975)}
\gppoint{gp mark 7}{(1.726,3.968)}
\gppoint{gp mark 7}{(1.732,3.963)}
\gppoint{gp mark 7}{(1.739,3.959)}
\gppoint{gp mark 7}{(1.746,3.956)}
\gppoint{gp mark 7}{(1.753,3.953)}
\gppoint{gp mark 7}{(1.760,3.951)}
\gppoint{gp mark 7}{(1.767,3.947)}
\gppoint{gp mark 7}{(1.774,3.942)}
\gppoint{gp mark 7}{(1.781,3.937)}
\gppoint{gp mark 7}{(1.788,3.932)}
\gppoint{gp mark 7}{(1.795,3.838)}
\gppoint{gp mark 7}{(1.802,3.115)}
\gppoint{gp mark 7}{(1.809,1.858)}
\gppoint{gp mark 7}{(1.816,1.300)}
\gppoint{gp mark 7}{(1.823,1.186)}
\gppoint{gp mark 7}{(1.830,1.170)}
\gppoint{gp mark 7}{(1.837,1.170)}
\gppoint{gp mark 7}{(1.844,1.169)}
\gppoint{gp mark 7}{(1.851,1.168)}
\gppoint{gp mark 7}{(1.858,1.168)}
\gppoint{gp mark 7}{(1.865,1.168)}
\gppoint{gp mark 7}{(1.872,1.168)}
\gppoint{gp mark 7}{(1.878,1.168)}
\gppoint{gp mark 7}{(1.885,1.168)}
\gppoint{gp mark 7}{(1.892,1.168)}
\gppoint{gp mark 7}{(1.899,1.168)}
\gppoint{gp mark 7}{(1.906,1.169)}
\gppoint{gp mark 7}{(1.913,1.169)}
\gppoint{gp mark 7}{(1.920,1.169)}
\gppoint{gp mark 7}{(1.927,1.169)}
\gppoint{gp mark 7}{(1.934,1.169)}
\gppoint{gp mark 7}{(1.941,1.169)}
\gppoint{gp mark 7}{(1.948,1.169)}
\gppoint{gp mark 7}{(1.955,1.169)}
\gppoint{gp mark 7}{(1.962,1.169)}
\gppoint{gp mark 7}{(1.969,1.169)}
\gppoint{gp mark 7}{(1.976,1.169)}
\gppoint{gp mark 7}{(1.983,1.169)}
\gppoint{gp mark 7}{(1.990,1.169)}
\gppoint{gp mark 7}{(1.997,1.170)}
\gppoint{gp mark 7}{(2.004,1.170)}
\gppoint{gp mark 7}{(2.011,1.170)}
\gppoint{gp mark 7}{(2.018,1.170)}
\gppoint{gp mark 7}{(2.024,1.170)}
\gppoint{gp mark 7}{(2.031,1.170)}
\gppoint{gp mark 7}{(2.038,1.170)}
\gppoint{gp mark 7}{(2.045,1.170)}
\gppoint{gp mark 7}{(2.052,1.170)}
\gppoint{gp mark 7}{(2.059,1.170)}
\gppoint{gp mark 7}{(2.066,1.170)}
\gppoint{gp mark 7}{(2.073,1.170)}
\gppoint{gp mark 7}{(2.080,1.170)}
\gppoint{gp mark 7}{(2.087,1.170)}
\gppoint{gp mark 7}{(2.094,1.170)}
\gppoint{gp mark 7}{(2.101,1.170)}
\gppoint{gp mark 7}{(2.108,1.170)}
\gppoint{gp mark 7}{(2.115,1.170)}
\gppoint{gp mark 7}{(2.122,1.170)}
\gppoint{gp mark 7}{(2.129,1.170)}
\gppoint{gp mark 7}{(2.136,1.171)}
\gppoint{gp mark 7}{(2.143,1.171)}
\gppoint{gp mark 7}{(2.150,1.171)}
\gppoint{gp mark 7}{(2.157,1.170)}
\gppoint{gp mark 7}{(2.164,1.171)}
\gppoint{gp mark 7}{(2.170,1.171)}
\gppoint{gp mark 7}{(2.177,1.171)}
\gppoint{gp mark 7}{(2.184,1.171)}
\gppoint{gp mark 7}{(2.191,1.171)}
\gppoint{gp mark 7}{(2.198,1.171)}
\gppoint{gp mark 7}{(2.205,1.171)}
\gppoint{gp mark 7}{(2.212,1.171)}
\gppoint{gp mark 7}{(2.219,1.171)}
\gppoint{gp mark 7}{(2.226,1.171)}
\gppoint{gp mark 7}{(2.233,1.171)}
\gppoint{gp mark 7}{(2.240,1.171)}
\gppoint{gp mark 7}{(2.247,1.171)}
\gppoint{gp mark 7}{(2.254,1.171)}
\gppoint{gp mark 7}{(2.261,1.171)}
\gppoint{gp mark 7}{(2.268,1.171)}
\gppoint{gp mark 7}{(2.275,1.171)}
\gppoint{gp mark 7}{(2.282,1.171)}
\gppoint{gp mark 7}{(2.289,1.171)}
\gppoint{gp mark 7}{(2.296,1.171)}
\gppoint{gp mark 7}{(2.303,1.171)}
\gppoint{gp mark 7}{(2.310,1.171)}
\gppoint{gp mark 7}{(2.316,1.171)}
\gppoint{gp mark 7}{(2.323,1.171)}
\gppoint{gp mark 7}{(2.330,1.172)}
\gppoint{gp mark 7}{(2.337,1.172)}
\gppoint{gp mark 7}{(2.344,1.171)}
\gppoint{gp mark 7}{(2.351,1.171)}
\gppoint{gp mark 7}{(2.358,1.171)}
\gppoint{gp mark 7}{(2.365,1.172)}
\gppoint{gp mark 7}{(2.372,1.172)}
\gppoint{gp mark 7}{(2.379,1.172)}
\gppoint{gp mark 7}{(2.386,1.172)}
\gppoint{gp mark 7}{(2.393,1.172)}
\gppoint{gp mark 7}{(2.400,1.172)}
\gppoint{gp mark 7}{(2.407,1.172)}
\gppoint{gp mark 7}{(2.414,1.172)}
\gppoint{gp mark 7}{(2.421,1.172)}
\gppoint{gp mark 7}{(2.428,1.172)}
\gppoint{gp mark 7}{(2.435,1.172)}
\gppoint{gp mark 7}{(2.442,1.172)}
\gppoint{gp mark 7}{(2.449,1.172)}
\gppoint{gp mark 7}{(2.455,1.172)}
\gppoint{gp mark 7}{(2.462,1.172)}
\gppoint{gp mark 7}{(2.469,1.172)}
\gppoint{gp mark 7}{(2.476,1.172)}
\gppoint{gp mark 7}{(2.483,1.172)}
\gppoint{gp mark 7}{(2.490,1.172)}
\gppoint{gp mark 7}{(2.497,1.172)}
\gppoint{gp mark 7}{(2.504,1.172)}
\gppoint{gp mark 7}{(2.511,1.172)}
\gppoint{gp mark 7}{(2.518,1.172)}
\gppoint{gp mark 7}{(2.525,1.172)}
\gppoint{gp mark 7}{(2.532,1.172)}
\gppoint{gp mark 7}{(2.539,1.172)}
\gppoint{gp mark 7}{(2.546,1.172)}
\gppoint{gp mark 7}{(2.553,1.172)}
\gppoint{gp mark 7}{(2.560,1.172)}
\gppoint{gp mark 7}{(2.567,1.172)}
\gppoint{gp mark 7}{(2.574,1.172)}
\gppoint{gp mark 7}{(2.581,1.172)}
\gppoint{gp mark 7}{(2.588,1.172)}
\gppoint{gp mark 7}{(2.595,1.172)}
\gppoint{gp mark 7}{(2.601,1.172)}
\gppoint{gp mark 7}{(2.608,1.172)}
\gppoint{gp mark 7}{(2.615,1.172)}
\gppoint{gp mark 7}{(2.622,1.172)}
\gppoint{gp mark 7}{(2.629,1.172)}
\gppoint{gp mark 7}{(2.636,1.172)}
\gppoint{gp mark 7}{(2.643,1.172)}
\gppoint{gp mark 7}{(2.650,1.172)}
\gppoint{gp mark 7}{(2.657,1.172)}
\gppoint{gp mark 7}{(2.664,1.172)}
\gppoint{gp mark 7}{(2.671,1.172)}
\gppoint{gp mark 7}{(2.678,1.172)}
\gppoint{gp mark 7}{(2.685,1.172)}
\gppoint{gp mark 7}{(2.692,1.172)}
\gppoint{gp mark 7}{(2.699,1.172)}
\gppoint{gp mark 7}{(2.706,1.172)}
\gppoint{gp mark 7}{(2.713,1.172)}
\gppoint{gp mark 7}{(2.720,1.172)}
\gppoint{gp mark 7}{(2.727,1.172)}
\gppoint{gp mark 7}{(2.734,1.172)}
\gppoint{gp mark 7}{(2.741,1.172)}
\gppoint{gp mark 7}{(2.747,1.172)}
\gppoint{gp mark 7}{(2.754,1.172)}
\gppoint{gp mark 7}{(2.761,1.172)}
\gppoint{gp mark 7}{(2.768,1.172)}
\gppoint{gp mark 7}{(2.775,1.172)}
\gppoint{gp mark 7}{(2.782,1.172)}
\gppoint{gp mark 7}{(2.789,1.172)}
\gppoint{gp mark 7}{(2.796,1.172)}
\gppoint{gp mark 7}{(2.803,1.172)}
\gppoint{gp mark 7}{(2.810,1.172)}
\gppoint{gp mark 7}{(2.817,1.172)}
\gppoint{gp mark 7}{(2.824,1.172)}
\gppoint{gp mark 7}{(2.831,1.172)}
\gppoint{gp mark 7}{(2.838,1.172)}
\gppoint{gp mark 7}{(2.845,1.172)}
\gppoint{gp mark 7}{(2.852,1.172)}
\gppoint{gp mark 7}{(2.859,1.172)}
\gppoint{gp mark 7}{(2.866,1.172)}
\gppoint{gp mark 7}{(2.873,1.172)}
\gppoint{gp mark 7}{(2.880,1.172)}
\gppoint{gp mark 7}{(2.887,1.172)}
\gppoint{gp mark 7}{(2.893,1.172)}
\gppoint{gp mark 7}{(2.900,1.172)}
\gppoint{gp mark 7}{(2.907,1.172)}
\gppoint{gp mark 7}{(2.914,1.172)}
\gppoint{gp mark 7}{(2.921,1.172)}
\gppoint{gp mark 7}{(2.928,1.172)}
\gppoint{gp mark 7}{(2.935,1.172)}
\gppoint{gp mark 7}{(2.942,1.172)}
\gppoint{gp mark 7}{(2.949,1.173)}
\gppoint{gp mark 7}{(2.956,1.172)}
\gppoint{gp mark 7}{(2.963,1.172)}
\gppoint{gp mark 7}{(2.970,1.172)}
\gppoint{gp mark 7}{(2.977,1.172)}
\gppoint{gp mark 7}{(2.984,1.172)}
\gppoint{gp mark 7}{(2.991,1.172)}
\gppoint{gp mark 7}{(2.998,1.172)}
\gppoint{gp mark 7}{(3.005,1.173)}
\gppoint{gp mark 7}{(3.012,1.173)}
\gppoint{gp mark 7}{(3.019,1.172)}
\gppoint{gp mark 7}{(3.026,1.172)}
\gppoint{gp mark 7}{(3.033,1.172)}
\gppoint{gp mark 7}{(3.039,1.172)}
\gppoint{gp mark 7}{(3.046,1.172)}
\gppoint{gp mark 7}{(3.053,1.173)}
\gppoint{gp mark 7}{(3.060,1.173)}
\gppoint{gp mark 7}{(3.067,1.173)}
\gppoint{gp mark 7}{(3.074,1.172)}
\gppoint{gp mark 7}{(3.081,1.172)}
\gppoint{gp mark 7}{(3.088,1.172)}
\gppoint{gp mark 7}{(3.095,1.172)}
\gppoint{gp mark 7}{(3.102,1.173)}
\gppoint{gp mark 7}{(3.109,1.177)}
\gppoint{gp mark 7}{(3.116,1.210)}
\gppoint{gp mark 7}{(3.123,1.304)}
\gppoint{gp mark 7}{(3.130,1.344)}
\gppoint{gp mark 7}{(3.137,1.342)}
\gppoint{gp mark 7}{(3.144,1.339)}
\gppoint{gp mark 7}{(3.151,1.339)}
\gppoint{gp mark 7}{(3.158,1.342)}
\gppoint{gp mark 7}{(3.165,1.342)}
\gppoint{gp mark 7}{(3.172,1.341)}
\gppoint{gp mark 7}{(3.179,1.340)}
\gppoint{gp mark 7}{(3.185,1.339)}
\gppoint{gp mark 7}{(3.192,1.340)}
\gppoint{gp mark 7}{(3.199,1.341)}
\gppoint{gp mark 7}{(3.206,1.341)}
\gppoint{gp mark 7}{(3.213,1.339)}
\gppoint{gp mark 7}{(3.220,1.339)}
\gppoint{gp mark 7}{(3.227,1.339)}
\gppoint{gp mark 7}{(3.234,1.340)}
\gppoint{gp mark 7}{(3.241,1.340)}
\gppoint{gp mark 7}{(3.248,1.339)}
\gppoint{gp mark 7}{(3.255,1.338)}
\gppoint{gp mark 7}{(3.262,1.338)}
\gppoint{gp mark 7}{(3.269,1.338)}
\gppoint{gp mark 7}{(3.276,1.339)}
\gppoint{gp mark 7}{(3.283,1.338)}
\gppoint{gp mark 7}{(3.290,1.337)}
\gppoint{gp mark 7}{(3.297,1.337)}
\gppoint{gp mark 7}{(3.304,1.337)}
\gppoint{gp mark 7}{(3.311,1.338)}
\gppoint{gp mark 7}{(3.318,1.337)}
\gppoint{gp mark 7}{(3.325,1.337)}
\gppoint{gp mark 7}{(3.331,1.337)}
\gppoint{gp mark 7}{(3.338,1.337)}
\gppoint{gp mark 7}{(3.345,1.337)}
\gppoint{gp mark 7}{(3.352,1.338)}
\gppoint{gp mark 7}{(3.359,1.337)}
\gppoint{gp mark 7}{(3.366,1.337)}
\gppoint{gp mark 7}{(3.373,1.338)}
\gppoint{gp mark 7}{(3.380,1.338)}
\gppoint{gp mark 7}{(3.387,1.338)}
\gppoint{gp mark 7}{(3.394,1.338)}
\gppoint{gp mark 7}{(3.401,1.338)}
\gppoint{gp mark 7}{(3.408,1.338)}
\gppoint{gp mark 7}{(3.415,1.338)}
\gppoint{gp mark 7}{(3.422,1.338)}
\gppoint{gp mark 7}{(3.429,1.338)}
\gppoint{gp mark 7}{(3.436,1.338)}
\gppoint{gp mark 7}{(3.443,1.338)}
\gppoint{gp mark 7}{(3.450,1.338)}
\gppoint{gp mark 7}{(3.457,1.338)}
\gppoint{gp mark 7}{(3.464,1.338)}
\gppoint{gp mark 7}{(3.471,1.338)}
\gppoint{gp mark 7}{(3.477,1.338)}
\gppoint{gp mark 7}{(3.484,1.338)}
\gppoint{gp mark 7}{(3.491,1.338)}
\gppoint{gp mark 7}{(3.498,1.338)}
\gppoint{gp mark 7}{(3.505,1.338)}
\gppoint{gp mark 7}{(3.512,1.338)}
\gppoint{gp mark 7}{(3.519,1.338)}
\gppoint{gp mark 7}{(3.526,1.338)}
\gppoint{gp mark 7}{(3.533,1.338)}
\gppoint{gp mark 7}{(3.540,1.338)}
\gppoint{gp mark 7}{(3.547,1.338)}
\gppoint{gp mark 7}{(3.554,1.338)}
\gppoint{gp mark 7}{(3.561,1.338)}
\gppoint{gp mark 7}{(3.568,1.338)}
\gppoint{gp mark 7}{(3.575,1.338)}
\gppoint{gp mark 7}{(3.582,1.338)}
\gppoint{gp mark 7}{(3.589,1.338)}
\gppoint{gp mark 7}{(3.596,1.338)}
\gppoint{gp mark 7}{(3.603,1.338)}
\gppoint{gp mark 7}{(3.610,1.338)}
\gppoint{gp mark 7}{(3.617,1.338)}
\gppoint{gp mark 7}{(3.623,1.338)}
\gppoint{gp mark 7}{(3.630,1.338)}
\gppoint{gp mark 7}{(3.637,1.338)}
\gppoint{gp mark 7}{(3.644,1.338)}
\gppoint{gp mark 7}{(3.651,1.338)}
\gppoint{gp mark 7}{(3.658,1.338)}
\gppoint{gp mark 7}{(3.665,1.338)}
\gppoint{gp mark 7}{(3.672,1.338)}
\gppoint{gp mark 7}{(3.679,1.338)}
\gppoint{gp mark 7}{(3.686,1.338)}
\gppoint{gp mark 7}{(3.693,1.338)}
\gppoint{gp mark 7}{(3.700,1.338)}
\gppoint{gp mark 7}{(3.707,1.338)}
\gppoint{gp mark 7}{(3.714,1.338)}
\gppoint{gp mark 7}{(3.721,1.338)}
\gppoint{gp mark 7}{(3.728,1.338)}
\gppoint{gp mark 7}{(3.735,1.338)}
\gppoint{gp mark 7}{(3.742,1.338)}
\gppoint{gp mark 7}{(3.749,1.338)}
\gppoint{gp mark 7}{(3.756,1.338)}
\gppoint{gp mark 7}{(3.763,1.338)}
\gppoint{gp mark 7}{(3.769,1.338)}
\gppoint{gp mark 7}{(3.776,1.338)}
\gppoint{gp mark 7}{(3.783,1.338)}
\gppoint{gp mark 7}{(3.790,1.338)}
\gppoint{gp mark 7}{(3.797,1.338)}
\gppoint{gp mark 7}{(3.804,1.338)}
\gppoint{gp mark 7}{(3.811,1.338)}
\gppoint{gp mark 7}{(3.818,1.338)}
\gppoint{gp mark 7}{(3.825,1.338)}
\gppoint{gp mark 7}{(3.832,1.338)}
\gppoint{gp mark 7}{(3.839,1.338)}
\gppoint{gp mark 7}{(3.846,1.338)}
\gppoint{gp mark 7}{(3.853,1.338)}
\gppoint{gp mark 7}{(3.860,1.338)}
\gppoint{gp mark 7}{(3.867,1.338)}
\gppoint{gp mark 7}{(3.874,1.338)}
\gppoint{gp mark 7}{(3.881,1.338)}
\gppoint{gp mark 7}{(3.888,1.338)}
\gppoint{gp mark 7}{(3.895,1.338)}
\gppoint{gp mark 7}{(3.902,1.337)}
\gppoint{gp mark 7}{(3.908,1.337)}
\gppoint{gp mark 7}{(3.915,1.337)}
\gppoint{gp mark 7}{(3.922,1.337)}
\gppoint{gp mark 7}{(3.929,1.337)}
\gppoint{gp mark 7}{(3.936,1.337)}
\gppoint{gp mark 7}{(3.943,1.338)}
\gppoint{gp mark 7}{(3.950,1.338)}
\gppoint{gp mark 7}{(3.957,1.338)}
\gppoint{gp mark 7}{(3.964,1.338)}
\gppoint{gp mark 7}{(3.971,1.337)}
\gppoint{gp mark 7}{(3.978,1.337)}
\gppoint{gp mark 7}{(3.985,1.337)}
\gppoint{gp mark 7}{(3.992,1.337)}
\gppoint{gp mark 7}{(3.999,1.337)}
\gppoint{gp mark 7}{(4.006,1.337)}
\gppoint{gp mark 7}{(4.013,1.337)}
\gppoint{gp mark 7}{(4.020,1.337)}
\gppoint{gp mark 7}{(4.027,1.337)}
\gppoint{gp mark 7}{(4.034,1.337)}
\gppoint{gp mark 7}{(4.041,1.337)}
\gppoint{gp mark 7}{(4.048,1.337)}
\gppoint{gp mark 7}{(4.054,1.337)}
\gppoint{gp mark 7}{(4.061,1.337)}
\gppoint{gp mark 7}{(4.068,1.337)}
\gppoint{gp mark 7}{(4.075,1.337)}
\gppoint{gp mark 7}{(4.082,1.337)}
\gppoint{gp mark 7}{(4.089,1.338)}
\gppoint{gp mark 7}{(4.096,1.338)}
\gppoint{gp mark 7}{(4.103,1.337)}
\gppoint{gp mark 7}{(4.110,1.337)}
\gppoint{gp mark 7}{(4.117,1.337)}
\gppoint{gp mark 7}{(4.124,1.338)}
\gppoint{gp mark 7}{(4.131,1.338)}
\gppoint{gp mark 7}{(4.138,1.338)}
\gppoint{gp mark 7}{(4.145,1.338)}
\gppoint{gp mark 7}{(4.152,1.338)}
\gppoint{gp mark 7}{(4.159,1.338)}
\gppoint{gp mark 7}{(4.166,1.337)}
\gppoint{gp mark 7}{(4.173,1.337)}
\gppoint{gp mark 7}{(4.180,1.337)}
\gppoint{gp mark 7}{(4.187,1.337)}
\gppoint{gp mark 7}{(4.194,1.338)}
\gppoint{gp mark 7}{(4.200,1.337)}
\gppoint{gp mark 7}{(4.207,1.337)}
\gppoint{gp mark 7}{(4.214,1.337)}
\gppoint{gp mark 7}{(4.221,1.338)}
\gppoint{gp mark 7}{(4.228,1.337)}
\gppoint{gp mark 7}{(4.235,1.337)}
\gppoint{gp mark 7}{(4.242,1.337)}
\gppoint{gp mark 7}{(4.249,1.337)}
\gppoint{gp mark 7}{(4.256,1.338)}
\gppoint{gp mark 7}{(4.263,1.338)}
\gppoint{gp mark 7}{(4.270,1.337)}
\gppoint{gp mark 7}{(4.277,1.337)}
\gppoint{gp mark 7}{(4.284,1.338)}
\gppoint{gp mark 7}{(4.291,1.339)}
\gppoint{gp mark 7}{(4.298,1.338)}
\gppoint{gp mark 7}{(4.305,1.336)}
\gppoint{gp mark 7}{(4.312,1.337)}
\gppoint{gp mark 7}{(4.319,1.339)}
\gppoint{gp mark 7}{(4.326,1.340)}
\gppoint{gp mark 7}{(4.333,1.337)}
\gppoint{gp mark 7}{(4.340,1.335)}
\gppoint{gp mark 7}{(4.346,1.336)}
\gppoint{gp mark 7}{(4.353,1.341)}
\gppoint{gp mark 7}{(4.360,1.341)}
\gppoint{gp mark 7}{(4.367,1.334)}
\gppoint{gp mark 7}{(4.374,1.335)}
\gppoint{gp mark 7}{(4.381,1.345)}
\gppoint{gp mark 7}{(4.388,1.249)}
\gppoint{gp mark 7}{(4.395,1.141)}
\gppoint{gp mark 7}{(4.402,1.121)}
\gppoint{gp mark 7}{(4.409,1.121)}
\gppoint{gp mark 7}{(4.416,1.122)}
\gppoint{gp mark 7}{(4.423,1.123)}
\gppoint{gp mark 7}{(4.430,1.123)}
\gppoint{gp mark 7}{(4.437,1.122)}
\gppoint{gp mark 7}{(4.444,1.122)}
\gppoint{gp mark 7}{(4.451,1.123)}
\gppoint{gp mark 7}{(4.458,1.123)}
\gppoint{gp mark 7}{(4.465,1.122)}
\gppoint{gp mark 7}{(4.472,1.122)}
\gppoint{gp mark 7}{(4.479,1.123)}
\gppoint{gp mark 7}{(4.486,1.123)}
\gppoint{gp mark 7}{(4.492,1.122)}
\gppoint{gp mark 7}{(4.499,1.122)}
\gppoint{gp mark 7}{(4.506,1.122)}
\gppoint{gp mark 7}{(4.513,1.122)}
\gppoint{gp mark 7}{(4.520,1.122)}
\gppoint{gp mark 7}{(4.527,1.122)}
\gppoint{gp mark 7}{(4.534,1.122)}
\gppoint{gp mark 7}{(4.541,1.122)}
\gppoint{gp mark 7}{(4.548,1.122)}
\gppoint{gp mark 7}{(4.555,1.122)}
\gppoint{gp mark 7}{(4.562,1.122)}
\gppoint{gp mark 7}{(4.569,1.122)}
\gppoint{gp mark 7}{(4.576,1.122)}
\gppoint{gp mark 7}{(4.583,1.122)}
\gppoint{gp mark 7}{(4.590,1.122)}
\gppoint{gp mark 7}{(4.597,1.122)}
\gppoint{gp mark 7}{(4.604,1.122)}
\gppoint{gp mark 7}{(4.611,1.122)}
\gppoint{gp mark 7}{(4.618,1.122)}
\gppoint{gp mark 7}{(4.625,1.122)}
\gppoint{gp mark 7}{(4.632,1.122)}
\gppoint{gp mark 7}{(4.638,1.122)}
\gppoint{gp mark 7}{(4.645,1.122)}
\gppoint{gp mark 7}{(4.652,1.122)}
\gppoint{gp mark 7}{(4.659,1.122)}
\gppoint{gp mark 7}{(4.666,1.122)}
\gppoint{gp mark 7}{(4.673,1.122)}
\gppoint{gp mark 7}{(4.680,1.122)}
\gppoint{gp mark 7}{(4.687,1.122)}
\gppoint{gp mark 7}{(4.694,1.122)}
\gppoint{gp mark 7}{(4.701,1.122)}
\gppoint{gp mark 7}{(4.708,1.122)}
\gppoint{gp mark 7}{(4.715,1.122)}
\gppoint{gp mark 7}{(4.722,1.122)}
\gppoint{gp mark 7}{(4.729,1.122)}
\gppoint{gp mark 7}{(4.736,1.122)}
\gppoint{gp mark 7}{(4.743,1.122)}
\gppoint{gp mark 7}{(4.750,1.122)}
\gppoint{gp mark 7}{(4.757,1.122)}
\gppoint{gp mark 7}{(4.764,1.122)}
\gppoint{gp mark 7}{(4.771,1.122)}
\gppoint{gp mark 7}{(4.778,1.122)}
\gppoint{gp mark 7}{(4.784,1.122)}
\gppoint{gp mark 7}{(4.791,1.122)}
\gppoint{gp mark 7}{(4.798,1.122)}
\gppoint{gp mark 7}{(4.805,1.122)}
\gppoint{gp mark 7}{(4.812,1.122)}
\gppoint{gp mark 7}{(4.819,1.122)}
\gppoint{gp mark 7}{(4.826,1.122)}
\gppoint{gp mark 7}{(4.833,1.122)}
\gppoint{gp mark 7}{(4.840,1.122)}
\gppoint{gp mark 7}{(4.847,1.122)}
\gppoint{gp mark 7}{(4.854,1.122)}
\gppoint{gp mark 7}{(4.861,1.122)}
\gppoint{gp mark 7}{(4.868,1.122)}
\gppoint{gp mark 7}{(4.875,1.122)}
\gppoint{gp mark 7}{(4.882,1.122)}
\gppoint{gp mark 7}{(4.889,1.122)}
\gppoint{gp mark 7}{(4.896,1.122)}
\gppoint{gp mark 7}{(4.903,1.122)}
\gppoint{gp mark 7}{(4.910,1.122)}
\gppoint{gp mark 7}{(4.917,1.122)}
\gppoint{gp mark 7}{(4.924,1.122)}
\gppoint{gp mark 7}{(4.930,1.122)}
\gppoint{gp mark 7}{(4.937,1.122)}
\gppoint{gp mark 7}{(4.944,1.122)}
\gppoint{gp mark 7}{(4.951,1.122)}
\gppoint{gp mark 7}{(4.958,1.122)}
\gppoint{gp mark 7}{(4.965,1.122)}
\gppoint{gp mark 7}{(4.972,1.122)}
\gppoint{gp mark 7}{(4.979,1.122)}
\gppoint{gp mark 7}{(4.986,1.122)}
\gppoint{gp mark 7}{(4.993,1.122)}
\gppoint{gp mark 7}{(5.000,1.122)}
\gppoint{gp mark 7}{(5.007,1.122)}
\gppoint{gp mark 7}{(5.014,1.122)}
\gppoint{gp mark 7}{(5.021,1.122)}
\gppoint{gp mark 7}{(5.028,1.122)}
\gppoint{gp mark 7}{(5.035,1.122)}
\gppoint{gp mark 7}{(5.042,1.122)}
\gppoint{gp mark 7}{(5.049,1.122)}
\gppoint{gp mark 7}{(5.056,1.122)}
\gppoint{gp mark 7}{(5.063,1.122)}
\gppoint{gp mark 7}{(5.070,1.122)}
\gppoint{gp mark 7}{(5.076,1.122)}
\gppoint{gp mark 7}{(5.083,1.122)}
\gppoint{gp mark 7}{(5.090,1.122)}
\gppoint{gp mark 7}{(5.097,1.122)}
\gppoint{gp mark 7}{(5.104,1.122)}
\gppoint{gp mark 7}{(5.111,1.122)}
\gppoint{gp mark 7}{(5.118,1.122)}
\gppoint{gp mark 7}{(5.125,1.121)}
\gppoint{gp mark 7}{(5.132,1.121)}
\gppoint{gp mark 7}{(5.139,1.121)}
\gppoint{gp mark 7}{(5.146,1.122)}
\gppoint{gp mark 7}{(5.153,1.122)}
\gppoint{gp mark 7}{(5.160,1.122)}
\gppoint{gp mark 7}{(5.167,1.122)}
\gppoint{gp mark 7}{(5.174,1.122)}
\gppoint{gp mark 7}{(5.181,1.122)}
\gppoint{gp mark 7}{(5.188,1.122)}
\gppoint{gp mark 7}{(5.195,1.122)}
\gppoint{gp mark 7}{(5.202,1.122)}
\gppoint{gp mark 7}{(5.209,1.122)}
\gppoint{gp mark 7}{(5.216,1.122)}
\gppoint{gp mark 7}{(5.222,1.122)}
\gppoint{gp mark 7}{(5.229,1.121)}
\gppoint{gp mark 7}{(5.236,1.121)}
\gppoint{gp mark 7}{(5.243,1.121)}
\gppoint{gp mark 7}{(5.250,1.121)}
\gppoint{gp mark 7}{(5.257,1.122)}
\gppoint{gp mark 7}{(5.264,1.122)}
\gppoint{gp mark 7}{(5.271,1.122)}
\gppoint{gp mark 7}{(5.278,1.122)}
\gppoint{gp mark 7}{(5.285,1.122)}
\gppoint{gp mark 7}{(5.292,1.121)}
\gppoint{gp mark 7}{(5.299,1.122)}
\gppoint{gp mark 7}{(5.306,1.122)}
\gppoint{gp mark 7}{(5.313,1.122)}
\gppoint{gp mark 7}{(5.320,1.122)}
\gppoint{gp mark 7}{(5.327,1.122)}
\gppoint{gp mark 7}{(5.334,1.121)}
\gppoint{gp mark 7}{(5.341,1.121)}
\gppoint{gp mark 7}{(5.348,1.121)}
\gppoint{gp mark 7}{(5.355,1.121)}
\gppoint{gp mark 7}{(5.361,1.121)}
\gppoint{gp mark 7}{(5.368,1.122)}
\gppoint{gp mark 7}{(5.375,1.122)}
\gppoint{gp mark 7}{(5.382,1.122)}
\gppoint{gp mark 7}{(5.389,1.122)}
\gppoint{gp mark 7}{(5.396,1.121)}
\gppoint{gp mark 7}{(5.403,1.121)}
\gppoint{gp mark 7}{(5.410,1.121)}
\gppoint{gp mark 7}{(5.417,1.121)}
\gppoint{gp mark 7}{(5.424,1.121)}
\gppoint{gp mark 7}{(5.431,1.121)}
\gppoint{gp mark 7}{(5.438,1.121)}
\gppoint{gp mark 7}{(5.445,1.121)}
\gppoint{gp mark 7}{(5.452,1.121)}
\gppoint{gp mark 7}{(5.459,1.121)}
\gppoint{gp mark 7}{(5.466,1.121)}
\gppoint{gp mark 7}{(5.473,1.122)}
\gppoint{gp mark 7}{(5.480,1.122)}
\gppoint{gp mark 7}{(5.487,1.122)}
\gppoint{gp mark 7}{(5.494,1.122)}
\gppoint{gp mark 7}{(5.501,1.121)}
\gppoint{gp mark 7}{(5.507,1.121)}
\gppoint{gp mark 7}{(5.514,1.121)}
\gppoint{gp mark 7}{(5.521,1.121)}
\gppoint{gp mark 7}{(5.528,1.121)}
\gppoint{gp mark 7}{(5.535,1.121)}
\gppoint{gp mark 7}{(5.542,1.121)}
\gppoint{gp mark 7}{(5.549,1.121)}
\gppoint{gp mark 7}{(5.556,1.121)}
\gppoint{gp mark 7}{(5.563,1.121)}
\gppoint{gp mark 7}{(5.570,1.121)}
\gppoint{gp mark 7}{(5.577,1.122)}
\gppoint{gp mark 7}{(5.584,1.122)}
\gppoint{gp mark 7}{(5.591,1.121)}
\gppoint{gp mark 7}{(5.598,1.121)}
\gppoint{gp mark 7}{(5.605,1.121)}
\gppoint{gp mark 7}{(5.612,1.121)}
\gppoint{gp mark 7}{(5.619,1.121)}
\gppoint{gp mark 7}{(5.626,1.121)}
\gppoint{gp mark 7}{(5.633,1.121)}
\gppoint{gp mark 7}{(5.640,1.121)}
\gppoint{gp mark 7}{(5.647,1.122)}
\gppoint{gp mark 7}{(5.653,1.122)}
\gppoint{gp mark 7}{(5.660,1.121)}
\gppoint{gp mark 7}{(5.667,1.121)}
\gppoint{gp mark 7}{(5.674,1.121)}
\gppoint{gp mark 7}{(5.681,1.121)}
\gppoint{gp mark 7}{(5.688,1.121)}
\gppoint{gp mark 7}{(5.695,1.121)}
\gppoint{gp mark 7}{(5.702,1.121)}
\gppoint{gp mark 7}{(5.709,1.121)}
\gppoint{gp mark 7}{(5.716,1.121)}
\gppoint{gp mark 7}{(5.723,1.121)}
\gppoint{gp mark 7}{(5.730,1.121)}
\gppoint{gp mark 7}{(5.737,1.121)}
\gppoint{gp mark 7}{(5.744,1.121)}
\gppoint{gp mark 7}{(5.751,1.121)}
\gppoint{gp mark 7}{(5.758,1.121)}
\gppoint{gp mark 7}{(5.765,1.121)}
\gppoint{gp mark 7}{(5.772,1.121)}
\gppoint{gp mark 7}{(5.779,1.121)}
\gppoint{gp mark 7}{(5.786,1.121)}
\gppoint{gp mark 7}{(5.793,1.121)}
\gppoint{gp mark 7}{(5.799,1.121)}
\gppoint{gp mark 7}{(5.806,1.121)}
\gppoint{gp mark 7}{(5.813,1.121)}
\gppoint{gp mark 7}{(5.820,1.121)}
\gppoint{gp mark 7}{(5.827,1.121)}
\gppoint{gp mark 7}{(5.834,1.121)}
\gppoint{gp mark 7}{(5.841,1.121)}
\gppoint{gp mark 7}{(5.848,1.121)}
\gppoint{gp mark 7}{(5.855,1.121)}
\gppoint{gp mark 7}{(5.862,1.121)}
\gppoint{gp mark 7}{(5.869,1.121)}
\gppoint{gp mark 7}{(5.876,1.121)}
\gppoint{gp mark 7}{(5.883,1.121)}
\gppoint{gp mark 7}{(5.890,1.121)}
\gppoint{gp mark 7}{(5.897,1.121)}
\gppoint{gp mark 7}{(5.904,1.121)}
\gppoint{gp mark 7}{(5.911,1.121)}
\gppoint{gp mark 7}{(5.918,1.121)}
\gppoint{gp mark 7}{(5.925,1.121)}
\gppoint{gp mark 7}{(5.932,1.121)}
\gppoint{gp mark 7}{(5.939,1.121)}
\gppoint{gp mark 7}{(5.945,1.121)}
\gppoint{gp mark 7}{(5.952,1.121)}
\gppoint{gp mark 7}{(5.959,1.121)}
\gppoint{gp mark 7}{(5.966,1.121)}
\gppoint{gp mark 7}{(5.973,1.121)}
\gppoint{gp mark 7}{(5.980,1.121)}
\gppoint{gp mark 7}{(5.987,1.121)}
\gppoint{gp mark 7}{(5.994,1.121)}
\gppoint{gp mark 7}{(6.001,1.121)}
\gppoint{gp mark 7}{(6.008,1.121)}
\gppoint{gp mark 7}{(6.015,1.121)}
\gppoint{gp mark 7}{(6.022,1.121)}
\gppoint{gp mark 7}{(6.029,1.121)}
\gppoint{gp mark 7}{(6.036,1.121)}
\gppoint{gp mark 7}{(6.043,1.121)}
\gppoint{gp mark 7}{(6.050,1.121)}
\gppoint{gp mark 7}{(6.057,1.121)}
\gppoint{gp mark 7}{(6.064,1.121)}
\gppoint{gp mark 7}{(6.071,1.121)}
\gppoint{gp mark 7}{(6.078,1.121)}
\gppoint{gp mark 7}{(6.085,1.121)}
\gppoint{gp mark 7}{(6.091,1.121)}
\gppoint{gp mark 7}{(6.098,1.121)}
\gppoint{gp mark 7}{(6.105,1.121)}
\gppoint{gp mark 7}{(6.112,1.121)}
\gppoint{gp mark 7}{(6.119,1.121)}
\gppoint{gp mark 7}{(6.126,1.121)}
\gppoint{gp mark 7}{(6.133,1.121)}
\gppoint{gp mark 7}{(6.140,1.121)}
\gppoint{gp mark 7}{(6.147,1.121)}
\gppoint{gp mark 7}{(6.154,1.121)}
\gppoint{gp mark 7}{(6.161,1.121)}
\gppoint{gp mark 7}{(6.168,1.121)}
\gppoint{gp mark 7}{(6.175,1.121)}
\gppoint{gp mark 7}{(6.182,1.121)}
\gppoint{gp mark 7}{(6.189,1.121)}
\gppoint{gp mark 7}{(6.196,1.121)}
\gppoint{gp mark 7}{(6.203,1.121)}
\gppoint{gp mark 7}{(6.210,1.121)}
\gppoint{gp mark 7}{(6.217,1.121)}
\gppoint{gp mark 7}{(6.224,1.121)}
\gppoint{gp mark 7}{(6.231,1.121)}
\gppoint{gp mark 7}{(6.237,1.121)}
\gppoint{gp mark 7}{(6.244,1.121)}
\gppoint{gp mark 7}{(6.251,1.120)}
\gppoint{gp mark 7}{(6.258,1.120)}
\gppoint{gp mark 7}{(6.265,1.120)}
\gppoint{gp mark 7}{(6.272,1.120)}
\gppoint{gp mark 7}{(6.279,1.120)}
\gppoint{gp mark 7}{(6.286,1.120)}
\gppoint{gp mark 7}{(6.293,1.120)}
\gppoint{gp mark 7}{(6.300,1.120)}
\gppoint{gp mark 7}{(6.307,1.120)}
\gppoint{gp mark 7}{(6.314,1.119)}
\gppoint{gp mark 7}{(6.321,1.119)}
\gppoint{gp mark 7}{(6.328,1.119)}
\gppoint{gp mark 7}{(6.335,1.120)}
\gppoint{gp mark 7}{(6.342,1.119)}
\gppoint{gp mark 7}{(6.349,1.119)}
\gppoint{gp mark 7}{(6.356,1.120)}
\gppoint{gp mark 7}{(6.363,1.124)}
\gppoint{gp mark 7}{(6.370,1.166)}
\gppoint{gp mark 7}{(6.377,1.324)}
\gppoint{gp mark 7}{(6.383,1.697)}
\gppoint{gp mark 7}{(6.390,2.279)}
\gppoint{gp mark 7}{(6.397,2.965)}
\gppoint{gp mark 7}{(6.404,3.549)}
\gppoint{gp mark 7}{(6.411,3.901)}
\gppoint{gp mark 7}{(6.418,4.043)}
\gppoint{gp mark 7}{(6.425,4.072)}
\gppoint{gp mark 7}{(6.432,4.074)}
\gppoint{gp mark 7}{(6.439,4.074)}
\gppoint{gp mark 7}{(6.446,4.074)}
\gppoint{gp mark 7}{(6.453,4.073)}
\gppoint{gp mark 7}{(6.460,4.073)}
\gppoint{gp mark 7}{(6.467,4.072)}
\gppoint{gp mark 7}{(6.474,4.072)}
\gppoint{gp mark 7}{(6.481,4.072)}
\gppoint{gp mark 7}{(6.488,4.072)}
\gppoint{gp mark 7}{(6.495,4.073)}
\gppoint{gp mark 7}{(6.502,4.073)}
\gppoint{gp mark 7}{(6.509,4.073)}
\gppoint{gp mark 7}{(6.516,4.073)}
\gppoint{gp mark 7}{(6.523,4.073)}
\gppoint{gp mark 7}{(6.529,4.073)}
\gppoint{gp mark 7}{(6.536,4.072)}
\gppoint{gp mark 7}{(6.543,4.072)}
\gppoint{gp mark 7}{(6.550,4.071)}
\gppoint{gp mark 7}{(6.557,4.071)}
\gppoint{gp mark 7}{(6.564,4.070)}
\gppoint{gp mark 7}{(6.571,4.070)}
\gppoint{gp mark 7}{(6.578,4.070)}
\gppoint{gp mark 7}{(6.585,4.069)}
\gppoint{gp mark 7}{(6.592,4.069)}
\gppoint{gp mark 7}{(6.599,4.069)}
\gppoint{gp mark 7}{(6.606,4.069)}
\gppoint{gp mark 7}{(6.613,4.069)}
\gppoint{gp mark 7}{(6.620,4.069)}
\gppoint{gp mark 7}{(6.627,4.068)}
\gppoint{gp mark 7}{(6.634,4.068)}
\gppoint{gp mark 7}{(6.641,4.067)}
\gppoint{gp mark 7}{(6.648,4.067)}
\gppoint{gp mark 7}{(6.655,4.066)}
\gppoint{gp mark 7}{(6.662,4.066)}
\gppoint{gp mark 7}{(6.668,4.065)}
\gppoint{gp mark 7}{(6.675,4.065)}
\gppoint{gp mark 7}{(6.682,4.064)}
\gppoint{gp mark 7}{(6.689,4.063)}
\gppoint{gp mark 7}{(6.696,4.063)}
\gppoint{gp mark 7}{(6.703,4.062)}
\gppoint{gp mark 7}{(6.710,4.061)}
\gppoint{gp mark 7}{(6.717,4.061)}
\gppoint{gp mark 7}{(6.724,4.060)}
\gppoint{gp mark 7}{(6.731,4.059)}
\gppoint{gp mark 7}{(6.738,4.058)}
\gppoint{gp mark 7}{(6.745,4.057)}
\gppoint{gp mark 7}{(6.752,4.056)}
\gppoint{gp mark 7}{(6.759,4.055)}
\gppoint{gp mark 7}{(6.766,4.054)}
\gppoint{gp mark 7}{(6.773,4.045)}
\gppoint{gp mark 7}{(6.780,4.008)}
\gppoint{gp mark 7}{(6.787,3.961)}
\gppoint{gp mark 7}{(6.794,3.943)}
\gppoint{gp mark 7}{(6.801,3.941)}
\gppoint{gp mark 7}{(6.808,3.942)}
\gppoint{gp mark 7}{(6.814,3.945)}
\gppoint{gp mark 7}{(6.821,3.955)}
\gppoint{gp mark 7}{(6.828,3.966)}
\gppoint{gp mark 7}{(6.835,3.975)}
\gppoint{gp mark 7}{(6.842,3.980)}
\gppoint{gp mark 7}{(6.849,3.985)}
\gppoint{gp mark 7}{(6.856,3.990)}
\gppoint{gp mark 7}{(6.863,3.995)}
\gppoint{gp mark 7}{(6.870,4.000)}
\gppoint{gp mark 7}{(6.877,4.005)}
\gppoint{gp mark 7}{(6.884,4.011)}
\gppoint{gp mark 7}{(6.891,4.016)}
\gppoint{gp mark 7}{(6.898,4.022)}
\gppoint{gp mark 7}{(6.905,4.027)}
\gppoint{gp mark 7}{(6.912,4.033)}
\gppoint{gp mark 7}{(6.919,4.038)}
\gppoint{gp mark 7}{(6.926,4.044)}
\gppoint{gp mark 7}{(6.933,4.049)}
\gppoint{gp mark 7}{(6.940,4.054)}
\gppoint{gp mark 7}{(6.947,4.060)}
\gppoint{gp mark 7}{(6.954,4.065)}
\gppoint{gp mark 7}{(6.960,4.071)}
\gppoint{gp mark 7}{(6.967,4.076)}
\gppoint{gp mark 7}{(6.974,4.081)}
\gppoint{gp mark 7}{(6.981,4.087)}
\gppoint{gp mark 7}{(6.988,4.092)}
\gppoint{gp mark 7}{(6.995,4.097)}
\gppoint{gp mark 7}{(7.002,4.103)}
\gppoint{gp mark 7}{(7.009,4.108)}
\gppoint{gp mark 7}{(7.016,4.113)}
\gppoint{gp mark 7}{(7.023,4.118)}
\gppoint{gp mark 7}{(7.030,4.124)}
\gppoint{gp mark 7}{(7.037,4.129)}
\gppoint{gp mark 7}{(7.044,4.134)}
\gppoint{gp mark 7}{(7.051,4.139)}
\gppoint{gp mark 7}{(7.058,4.144)}
\gppoint{gp mark 7}{(7.065,4.150)}
\gppoint{gp mark 7}{(7.072,4.155)}
\gppoint{gp mark 7}{(7.079,4.160)}
\gppoint{gp mark 7}{(7.086,4.165)}
\gppoint{gp mark 7}{(7.093,4.170)}
\gppoint{gp mark 7}{(7.100,4.175)}
\gppoint{gp mark 7}{(7.106,4.180)}
\gppoint{gp mark 7}{(7.113,4.186)}
\gppoint{gp mark 7}{(7.120,4.191)}
\gppoint{gp mark 7}{(7.127,4.196)}
\gppoint{gp mark 7}{(7.134,4.201)}
\gppoint{gp mark 7}{(7.141,4.206)}
\gppoint{gp mark 7}{(7.148,4.211)}
\gppoint{gp mark 7}{(7.155,4.216)}
\gppoint{gp mark 7}{(7.162,4.221)}
\gppoint{gp mark 7}{(7.169,4.226)}
\gppoint{gp mark 7}{(7.176,4.231)}
\gppoint{gp mark 7}{(7.183,4.236)}
\gppoint{gp mark 7}{(7.190,4.241)}
\gppoint{gp mark 7}{(7.197,4.246)}
\gppoint{gp mark 7}{(7.204,4.251)}
\gppoint{gp mark 7}{(7.211,4.256)}
\gppoint{gp mark 7}{(7.218,4.261)}
\gppoint{gp mark 7}{(7.225,4.266)}
\gppoint{gp mark 7}{(7.232,4.271)}
\gppoint{gp mark 7}{(7.239,4.276)}
\gppoint{gp mark 7}{(7.246,4.281)}
\gppoint{gp mark 7}{(7.252,4.286)}
\gppoint{gp mark 7}{(7.259,4.290)}
\gppoint{gp mark 7}{(7.266,4.295)}
\gppoint{gp mark 7}{(7.273,4.300)}
\gppoint{gp mark 7}{(7.280,4.305)}
\gppoint{gp mark 7}{(7.287,4.310)}
\gppoint{gp mark 7}{(7.294,4.315)}
\gppoint{gp mark 7}{(7.301,4.320)}
\gppoint{gp mark 7}{(7.308,4.324)}
\gppoint{gp mark 7}{(7.315,4.329)}
\gppoint{gp mark 7}{(7.322,4.334)}
\gppoint{gp mark 7}{(7.329,4.339)}
\gppoint{gp mark 7}{(7.336,4.344)}
\gppoint{gp mark 7}{(7.343,4.348)}
\gppoint{gp mark 7}{(7.350,4.353)}
\gppoint{gp mark 7}{(7.357,4.358)}
\gppoint{gp mark 7}{(7.364,4.363)}
\gppoint{gp mark 7}{(7.371,4.368)}
\gppoint{gp mark 7}{(7.378,4.372)}
\gppoint{gp mark 7}{(7.385,4.377)}
\gppoint{gp mark 7}{(7.392,4.382)}
\gppoint{gp mark 7}{(7.398,4.387)}
\gppoint{gp mark 7}{(7.405,4.391)}
\gppoint{gp mark 7}{(7.412,4.396)}
\gppoint{gp mark 7}{(7.419,4.401)}
\gppoint{gp mark 7}{(7.426,4.405)}
\gppoint{gp mark 7}{(7.433,4.410)}
\gppoint{gp mark 7}{(7.440,4.415)}
\gppoint{gp mark 7}{(7.447,4.420)}
\gppoint{gp mark 7}{(7.454,4.424)}
\gppoint{gp mark 7}{(7.461,4.429)}
\gppoint{gp mark 7}{(7.468,4.434)}
\gppoint{gp mark 7}{(7.475,4.438)}
\gppoint{gp mark 7}{(7.482,4.443)}
\gppoint{gp mark 7}{(7.489,4.447)}
\gppoint{gp mark 7}{(7.496,4.452)}
\gppoint{gp mark 7}{(7.503,4.457)}
\gppoint{gp mark 7}{(7.510,4.461)}
\gppoint{gp mark 7}{(7.517,4.466)}
\gppoint{gp mark 7}{(7.524,4.471)}
\gppoint{gp mark 7}{(7.531,4.475)}
\gppoint{gp mark 7}{(7.538,4.480)}
\gppoint{gp mark 7}{(7.544,4.484)}
\gppoint{gp mark 7}{(7.551,4.489)}
\gppoint{gp mark 7}{(7.558,4.494)}
\gppoint{gp mark 7}{(7.565,4.498)}
\gppoint{gp mark 7}{(7.572,4.503)}
\gppoint{gp mark 7}{(7.579,4.507)}
\gppoint{gp mark 7}{(7.586,4.512)}
\gppoint{gp mark 7}{(7.593,4.516)}
\gppoint{gp mark 7}{(7.600,4.521)}
\gppoint{gp mark 7}{(7.607,4.526)}
\gppoint{gp mark 7}{(7.614,4.530)}
\gppoint{gp mark 7}{(7.621,4.535)}
\gppoint{gp mark 7}{(7.628,4.539)}
\gppoint{gp mark 7}{(7.635,4.544)}
\gppoint{gp mark 7}{(7.642,4.548)}
\gppoint{gp mark 7}{(7.649,4.553)}
\gppoint{gp mark 7}{(7.656,4.557)}
\gppoint{gp mark 7}{(7.663,4.562)}
\gppoint{gp mark 7}{(7.670,4.566)}
\gppoint{gp mark 7}{(7.677,4.571)}
\gppoint{gp mark 7}{(7.684,4.575)}
\gppoint{gp mark 7}{(7.690,4.580)}
\gppoint{gp mark 7}{(7.697,4.584)}
\gppoint{gp mark 7}{(7.704,4.589)}
\gppoint{gp mark 7}{(7.711,4.593)}
\gppoint{gp mark 7}{(7.718,4.598)}
\gppoint{gp mark 7}{(7.725,4.602)}
\gppoint{gp mark 7}{(7.732,4.607)}
\gppoint{gp mark 7}{(7.739,4.611)}
\gppoint{gp mark 7}{(7.746,4.615)}
\gppoint{gp mark 7}{(7.753,4.620)}
\gppoint{gp mark 7}{(7.760,4.624)}
\gppoint{gp mark 7}{(7.767,4.629)}
\gppoint{gp mark 7}{(7.774,4.633)}
\gppoint{gp mark 7}{(7.781,4.638)}
\gppoint{gp mark 7}{(7.788,4.642)}
\gppoint{gp mark 7}{(7.795,4.646)}
\gppoint{gp mark 7}{(7.802,4.651)}
\gppoint{gp mark 7}{(7.809,4.655)}
\gppoint{gp mark 7}{(7.816,4.660)}
\gppoint{gp mark 7}{(7.823,4.664)}
\gppoint{gp mark 7}{(7.830,4.668)}
\gppoint{gp mark 7}{(7.836,4.673)}
\gppoint{gp mark 7}{(7.843,4.677)}
\gppoint{gp mark 7}{(7.850,4.682)}
\gppoint{gp mark 7}{(7.857,4.686)}
\gppoint{gp mark 7}{(7.864,4.690)}
\gppoint{gp mark 7}{(7.871,4.695)}
\gppoint{gp mark 7}{(7.878,4.699)}
\gppoint{gp mark 7}{(7.885,4.703)}
\gppoint{gp mark 7}{(7.892,4.708)}
\gppoint{gp mark 7}{(7.899,4.712)}
\gppoint{gp mark 7}{(7.906,4.717)}
\gppoint{gp mark 7}{(7.913,4.721)}
\gppoint{gp mark 7}{(7.920,4.725)}
\gppoint{gp mark 7}{(7.927,4.730)}
\gppoint{gp mark 7}{(7.934,4.734)}
\gppoint{gp mark 7}{(7.941,4.738)}
\gpcolor{rgb color={1.000,0.000,0.000}}
\gpsetlinewidth{4.00}
\gppoint{gp mark 7}{(0.836,5.285)}
\gppoint{gp mark 7}{(0.843,5.276)}
\gppoint{gp mark 7}{(0.850,5.267)}
\gppoint{gp mark 7}{(0.857,5.259)}
\gppoint{gp mark 7}{(0.863,5.250)}
\gppoint{gp mark 7}{(0.870,5.241)}
\gppoint{gp mark 7}{(0.877,5.232)}
\gppoint{gp mark 7}{(0.884,5.223)}
\gppoint{gp mark 7}{(0.891,5.214)}
\gppoint{gp mark 7}{(0.898,5.205)}
\gppoint{gp mark 7}{(0.905,5.196)}
\gppoint{gp mark 7}{(0.912,5.188)}
\gppoint{gp mark 7}{(0.919,5.179)}
\gppoint{gp mark 7}{(0.926,5.170)}
\gppoint{gp mark 7}{(0.933,5.161)}
\gppoint{gp mark 7}{(0.940,5.152)}
\gppoint{gp mark 7}{(0.947,5.143)}
\gppoint{gp mark 7}{(0.954,5.134)}
\gppoint{gp mark 7}{(0.961,5.125)}
\gppoint{gp mark 7}{(0.968,5.116)}
\gppoint{gp mark 7}{(0.975,5.107)}
\gppoint{gp mark 7}{(0.982,5.097)}
\gppoint{gp mark 7}{(0.989,5.088)}
\gppoint{gp mark 7}{(0.996,5.079)}
\gppoint{gp mark 7}{(1.002,5.070)}
\gppoint{gp mark 7}{(1.009,5.061)}
\gppoint{gp mark 7}{(1.016,5.052)}
\gppoint{gp mark 7}{(1.023,5.043)}
\gppoint{gp mark 7}{(1.030,5.033)}
\gppoint{gp mark 7}{(1.037,5.024)}
\gppoint{gp mark 7}{(1.044,5.015)}
\gppoint{gp mark 7}{(1.051,5.006)}
\gppoint{gp mark 7}{(1.058,4.996)}
\gppoint{gp mark 7}{(1.065,4.987)}
\gppoint{gp mark 7}{(1.072,4.978)}
\gppoint{gp mark 7}{(1.079,4.968)}
\gppoint{gp mark 7}{(1.086,4.959)}
\gppoint{gp mark 7}{(1.093,4.950)}
\gppoint{gp mark 7}{(1.100,4.940)}
\gppoint{gp mark 7}{(1.107,4.931)}
\gppoint{gp mark 7}{(1.114,4.921)}
\gppoint{gp mark 7}{(1.121,4.912)}
\gppoint{gp mark 7}{(1.128,4.902)}
\gppoint{gp mark 7}{(1.135,4.893)}
\gppoint{gp mark 7}{(1.142,4.883)}
\gppoint{gp mark 7}{(1.148,4.874)}
\gppoint{gp mark 7}{(1.155,4.864)}
\gppoint{gp mark 7}{(1.162,4.854)}
\gppoint{gp mark 7}{(1.169,4.845)}
\gppoint{gp mark 7}{(1.176,4.835)}
\gppoint{gp mark 7}{(1.183,4.825)}
\gppoint{gp mark 7}{(1.190,4.816)}
\gppoint{gp mark 7}{(1.197,4.806)}
\gppoint{gp mark 7}{(1.204,4.796)}
\gppoint{gp mark 7}{(1.211,4.786)}
\gppoint{gp mark 7}{(1.218,4.776)}
\gppoint{gp mark 7}{(1.225,4.767)}
\gppoint{gp mark 7}{(1.232,4.757)}
\gppoint{gp mark 7}{(1.239,4.747)}
\gppoint{gp mark 7}{(1.246,4.737)}
\gppoint{gp mark 7}{(1.253,4.727)}
\gppoint{gp mark 7}{(1.260,4.717)}
\gppoint{gp mark 7}{(1.267,4.706)}
\gppoint{gp mark 7}{(1.274,4.696)}
\gppoint{gp mark 7}{(1.281,4.686)}
\gppoint{gp mark 7}{(1.288,4.676)}
\gppoint{gp mark 7}{(1.294,4.666)}
\gppoint{gp mark 7}{(1.301,4.655)}
\gppoint{gp mark 7}{(1.308,4.645)}
\gppoint{gp mark 7}{(1.315,4.635)}
\gppoint{gp mark 7}{(1.322,4.624)}
\gppoint{gp mark 7}{(1.329,4.614)}
\gppoint{gp mark 7}{(1.336,4.603)}
\gppoint{gp mark 7}{(1.343,4.593)}
\gppoint{gp mark 7}{(1.350,4.583)}
\gppoint{gp mark 7}{(1.357,4.572)}
\gppoint{gp mark 7}{(1.364,4.563)}
\gppoint{gp mark 7}{(1.371,4.554)}
\gppoint{gp mark 7}{(1.378,4.547)}
\gppoint{gp mark 7}{(1.385,4.543)}
\gppoint{gp mark 7}{(1.392,4.541)}
\gppoint{gp mark 7}{(1.399,4.541)}
\gppoint{gp mark 7}{(1.406,4.541)}
\gppoint{gp mark 7}{(1.413,4.541)}
\gppoint{gp mark 7}{(1.420,4.542)}
\gppoint{gp mark 7}{(1.427,4.545)}
\gppoint{gp mark 7}{(1.434,4.549)}
\gppoint{gp mark 7}{(1.440,4.552)}
\gppoint{gp mark 7}{(1.447,4.553)}
\gppoint{gp mark 7}{(1.454,4.553)}
\gppoint{gp mark 7}{(1.461,4.553)}
\gppoint{gp mark 7}{(1.468,4.553)}
\gppoint{gp mark 7}{(1.475,4.553)}
\gppoint{gp mark 7}{(1.482,4.553)}
\gppoint{gp mark 7}{(1.489,4.553)}
\gppoint{gp mark 7}{(1.496,4.553)}
\gppoint{gp mark 7}{(1.503,4.553)}
\gppoint{gp mark 7}{(1.510,4.553)}
\gppoint{gp mark 7}{(1.517,4.553)}
\gppoint{gp mark 7}{(1.524,4.554)}
\gppoint{gp mark 7}{(1.531,4.555)}
\gppoint{gp mark 7}{(1.538,4.555)}
\gppoint{gp mark 7}{(1.545,4.556)}
\gppoint{gp mark 7}{(1.552,4.556)}
\gppoint{gp mark 7}{(1.559,4.556)}
\gppoint{gp mark 7}{(1.566,4.556)}
\gppoint{gp mark 7}{(1.573,4.556)}
\gppoint{gp mark 7}{(1.580,4.556)}
\gppoint{gp mark 7}{(1.586,4.556)}
\gppoint{gp mark 7}{(1.593,4.556)}
\gppoint{gp mark 7}{(1.600,4.557)}
\gppoint{gp mark 7}{(1.607,4.557)}
\gppoint{gp mark 7}{(1.614,4.558)}
\gppoint{gp mark 7}{(1.621,4.559)}
\gppoint{gp mark 7}{(1.628,4.560)}
\gppoint{gp mark 7}{(1.635,4.561)}
\gppoint{gp mark 7}{(1.642,4.562)}
\gppoint{gp mark 7}{(1.649,4.562)}
\gppoint{gp mark 7}{(1.656,4.563)}
\gppoint{gp mark 7}{(1.663,4.563)}
\gppoint{gp mark 7}{(1.670,4.563)}
\gppoint{gp mark 7}{(1.677,4.563)}
\gppoint{gp mark 7}{(1.684,4.562)}
\gppoint{gp mark 7}{(1.691,4.561)}
\gppoint{gp mark 7}{(1.698,4.561)}
\gppoint{gp mark 7}{(1.705,4.560)}
\gppoint{gp mark 7}{(1.712,4.560)}
\gppoint{gp mark 7}{(1.719,4.559)}
\gppoint{gp mark 7}{(1.726,4.559)}
\gppoint{gp mark 7}{(1.732,4.559)}
\gppoint{gp mark 7}{(1.739,4.559)}
\gppoint{gp mark 7}{(1.746,4.558)}
\gppoint{gp mark 7}{(1.753,4.554)}
\gppoint{gp mark 7}{(1.760,4.532)}
\gppoint{gp mark 7}{(1.767,3.298)}
\gppoint{gp mark 7}{(1.774,1.496)}
\gppoint{gp mark 7}{(1.781,1.163)}
\gppoint{gp mark 7}{(1.788,1.136)}
\gppoint{gp mark 7}{(1.795,1.142)}
\gppoint{gp mark 7}{(1.802,1.148)}
\gppoint{gp mark 7}{(1.809,1.150)}
\gppoint{gp mark 7}{(1.816,1.151)}
\gppoint{gp mark 7}{(1.823,1.150)}
\gppoint{gp mark 7}{(1.830,1.149)}
\gppoint{gp mark 7}{(1.837,1.150)}
\gppoint{gp mark 7}{(1.844,1.151)}
\gppoint{gp mark 7}{(1.851,1.152)}
\gppoint{gp mark 7}{(1.858,1.152)}
\gppoint{gp mark 7}{(1.865,1.151)}
\gppoint{gp mark 7}{(1.872,1.150)}
\gppoint{gp mark 7}{(1.878,1.150)}
\gppoint{gp mark 7}{(1.885,1.151)}
\gppoint{gp mark 7}{(1.892,1.152)}
\gppoint{gp mark 7}{(1.899,1.153)}
\gppoint{gp mark 7}{(1.906,1.153)}
\gppoint{gp mark 7}{(1.913,1.153)}
\gppoint{gp mark 7}{(1.920,1.152)}
\gppoint{gp mark 7}{(1.927,1.152)}
\gppoint{gp mark 7}{(1.934,1.152)}
\gppoint{gp mark 7}{(1.941,1.152)}
\gppoint{gp mark 7}{(1.948,1.152)}
\gppoint{gp mark 7}{(1.955,1.151)}
\gppoint{gp mark 7}{(1.962,1.151)}
\gppoint{gp mark 7}{(1.969,1.152)}
\gppoint{gp mark 7}{(1.976,1.152)}
\gppoint{gp mark 7}{(1.983,1.152)}
\gppoint{gp mark 7}{(1.990,1.152)}
\gppoint{gp mark 7}{(1.997,1.152)}
\gppoint{gp mark 7}{(2.004,1.152)}
\gppoint{gp mark 7}{(2.011,1.153)}
\gppoint{gp mark 7}{(2.018,1.153)}
\gppoint{gp mark 7}{(2.024,1.153)}
\gppoint{gp mark 7}{(2.031,1.153)}
\gppoint{gp mark 7}{(2.038,1.153)}
\gppoint{gp mark 7}{(2.045,1.153)}
\gppoint{gp mark 7}{(2.052,1.153)}
\gppoint{gp mark 7}{(2.059,1.153)}
\gppoint{gp mark 7}{(2.066,1.153)}
\gppoint{gp mark 7}{(2.073,1.153)}
\gppoint{gp mark 7}{(2.080,1.153)}
\gppoint{gp mark 7}{(2.087,1.153)}
\gppoint{gp mark 7}{(2.094,1.153)}
\gppoint{gp mark 7}{(2.101,1.153)}
\gppoint{gp mark 7}{(2.108,1.153)}
\gppoint{gp mark 7}{(2.115,1.153)}
\gppoint{gp mark 7}{(2.122,1.153)}
\gppoint{gp mark 7}{(2.129,1.153)}
\gppoint{gp mark 7}{(2.136,1.153)}
\gppoint{gp mark 7}{(2.143,1.153)}
\gppoint{gp mark 7}{(2.150,1.153)}
\gppoint{gp mark 7}{(2.157,1.153)}
\gppoint{gp mark 7}{(2.164,1.153)}
\gppoint{gp mark 7}{(2.170,1.153)}
\gppoint{gp mark 7}{(2.177,1.152)}
\gppoint{gp mark 7}{(2.184,1.152)}
\gppoint{gp mark 7}{(2.191,1.152)}
\gppoint{gp mark 7}{(2.198,1.152)}
\gppoint{gp mark 7}{(2.205,1.152)}
\gppoint{gp mark 7}{(2.212,1.153)}
\gppoint{gp mark 7}{(2.219,1.153)}
\gppoint{gp mark 7}{(2.226,1.153)}
\gppoint{gp mark 7}{(2.233,1.153)}
\gppoint{gp mark 7}{(2.240,1.153)}
\gppoint{gp mark 7}{(2.247,1.153)}
\gppoint{gp mark 7}{(2.254,1.153)}
\gppoint{gp mark 7}{(2.261,1.153)}
\gppoint{gp mark 7}{(2.268,1.153)}
\gppoint{gp mark 7}{(2.275,1.153)}
\gppoint{gp mark 7}{(2.282,1.153)}
\gppoint{gp mark 7}{(2.289,1.153)}
\gppoint{gp mark 7}{(2.296,1.152)}
\gppoint{gp mark 7}{(2.303,1.152)}
\gppoint{gp mark 7}{(2.310,1.152)}
\gppoint{gp mark 7}{(2.316,1.153)}
\gppoint{gp mark 7}{(2.323,1.153)}
\gppoint{gp mark 7}{(2.330,1.154)}
\gppoint{gp mark 7}{(2.337,1.154)}
\gppoint{gp mark 7}{(2.344,1.154)}
\gppoint{gp mark 7}{(2.351,1.154)}
\gppoint{gp mark 7}{(2.358,1.153)}
\gppoint{gp mark 7}{(2.365,1.153)}
\gppoint{gp mark 7}{(2.372,1.153)}
\gppoint{gp mark 7}{(2.379,1.153)}
\gppoint{gp mark 7}{(2.386,1.153)}
\gppoint{gp mark 7}{(2.393,1.153)}
\gppoint{gp mark 7}{(2.400,1.153)}
\gppoint{gp mark 7}{(2.407,1.153)}
\gppoint{gp mark 7}{(2.414,1.153)}
\gppoint{gp mark 7}{(2.421,1.153)}
\gppoint{gp mark 7}{(2.428,1.153)}
\gppoint{gp mark 7}{(2.435,1.153)}
\gppoint{gp mark 7}{(2.442,1.153)}
\gppoint{gp mark 7}{(2.449,1.154)}
\gppoint{gp mark 7}{(2.455,1.154)}
\gppoint{gp mark 7}{(2.462,1.154)}
\gppoint{gp mark 7}{(2.469,1.154)}
\gppoint{gp mark 7}{(2.476,1.154)}
\gppoint{gp mark 7}{(2.483,1.154)}
\gppoint{gp mark 7}{(2.490,1.153)}
\gppoint{gp mark 7}{(2.497,1.153)}
\gppoint{gp mark 7}{(2.504,1.153)}
\gppoint{gp mark 7}{(2.511,1.153)}
\gppoint{gp mark 7}{(2.518,1.153)}
\gppoint{gp mark 7}{(2.525,1.153)}
\gppoint{gp mark 7}{(2.532,1.153)}
\gppoint{gp mark 7}{(2.539,1.153)}
\gppoint{gp mark 7}{(2.546,1.154)}
\gppoint{gp mark 7}{(2.553,1.154)}
\gppoint{gp mark 7}{(2.560,1.154)}
\gppoint{gp mark 7}{(2.567,1.154)}
\gppoint{gp mark 7}{(2.574,1.154)}
\gppoint{gp mark 7}{(2.581,1.154)}
\gppoint{gp mark 7}{(2.588,1.154)}
\gppoint{gp mark 7}{(2.595,1.153)}
\gppoint{gp mark 7}{(2.601,1.153)}
\gppoint{gp mark 7}{(2.608,1.153)}
\gppoint{gp mark 7}{(2.615,1.153)}
\gppoint{gp mark 7}{(2.622,1.153)}
\gppoint{gp mark 7}{(2.629,1.153)}
\gppoint{gp mark 7}{(2.636,1.153)}
\gppoint{gp mark 7}{(2.643,1.153)}
\gppoint{gp mark 7}{(2.650,1.154)}
\gppoint{gp mark 7}{(2.657,1.154)}
\gppoint{gp mark 7}{(2.664,1.154)}
\gppoint{gp mark 7}{(2.671,1.154)}
\gppoint{gp mark 7}{(2.678,1.154)}
\gppoint{gp mark 7}{(2.685,1.154)}
\gppoint{gp mark 7}{(2.692,1.154)}
\gppoint{gp mark 7}{(2.699,1.154)}
\gppoint{gp mark 7}{(2.706,1.154)}
\gppoint{gp mark 7}{(2.713,1.154)}
\gppoint{gp mark 7}{(2.720,1.154)}
\gppoint{gp mark 7}{(2.727,1.154)}
\gppoint{gp mark 7}{(2.734,1.153)}
\gppoint{gp mark 7}{(2.741,1.153)}
\gppoint{gp mark 7}{(2.747,1.153)}
\gppoint{gp mark 7}{(2.754,1.153)}
\gppoint{gp mark 7}{(2.761,1.153)}
\gppoint{gp mark 7}{(2.768,1.154)}
\gppoint{gp mark 7}{(2.775,1.154)}
\gppoint{gp mark 7}{(2.782,1.154)}
\gppoint{gp mark 7}{(2.789,1.154)}
\gppoint{gp mark 7}{(2.796,1.154)}
\gppoint{gp mark 7}{(2.803,1.154)}
\gppoint{gp mark 7}{(2.810,1.154)}
\gppoint{gp mark 7}{(2.817,1.154)}
\gppoint{gp mark 7}{(2.824,1.153)}
\gppoint{gp mark 7}{(2.831,1.153)}
\gppoint{gp mark 7}{(2.838,1.153)}
\gppoint{gp mark 7}{(2.845,1.153)}
\gppoint{gp mark 7}{(2.852,1.153)}
\gppoint{gp mark 7}{(2.859,1.153)}
\gppoint{gp mark 7}{(2.866,1.154)}
\gppoint{gp mark 7}{(2.873,1.154)}
\gppoint{gp mark 7}{(2.880,1.154)}
\gppoint{gp mark 7}{(2.887,1.154)}
\gppoint{gp mark 7}{(2.893,1.154)}
\gppoint{gp mark 7}{(2.900,1.154)}
\gppoint{gp mark 7}{(2.907,1.154)}
\gppoint{gp mark 7}{(2.914,1.153)}
\gppoint{gp mark 7}{(2.921,1.153)}
\gppoint{gp mark 7}{(2.928,1.153)}
\gppoint{gp mark 7}{(2.935,1.153)}
\gppoint{gp mark 7}{(2.942,1.153)}
\gppoint{gp mark 7}{(2.949,1.153)}
\gppoint{gp mark 7}{(2.956,1.153)}
\gppoint{gp mark 7}{(2.963,1.153)}
\gppoint{gp mark 7}{(2.970,1.153)}
\gppoint{gp mark 7}{(2.977,1.154)}
\gppoint{gp mark 7}{(2.984,1.155)}
\gppoint{gp mark 7}{(2.991,1.155)}
\gppoint{gp mark 7}{(2.998,1.155)}
\gppoint{gp mark 7}{(3.005,1.155)}
\gppoint{gp mark 7}{(3.012,1.154)}
\gppoint{gp mark 7}{(3.019,1.153)}
\gppoint{gp mark 7}{(3.026,1.153)}
\gppoint{gp mark 7}{(3.033,1.153)}
\gppoint{gp mark 7}{(3.039,1.153)}
\gppoint{gp mark 7}{(3.046,1.154)}
\gppoint{gp mark 7}{(3.053,1.154)}
\gppoint{gp mark 7}{(3.060,1.154)}
\gppoint{gp mark 7}{(3.067,1.154)}
\gppoint{gp mark 7}{(3.074,1.154)}
\gppoint{gp mark 7}{(3.081,1.153)}
\gppoint{gp mark 7}{(3.088,1.154)}
\gppoint{gp mark 7}{(3.095,1.154)}
\gppoint{gp mark 7}{(3.102,1.154)}
\gppoint{gp mark 7}{(3.109,1.155)}
\gppoint{gp mark 7}{(3.116,1.155)}
\gppoint{gp mark 7}{(3.123,1.154)}
\gppoint{gp mark 7}{(3.130,1.153)}
\gppoint{gp mark 7}{(3.137,1.152)}
\gppoint{gp mark 7}{(3.144,1.153)}
\gppoint{gp mark 7}{(3.151,1.153)}
\gppoint{gp mark 7}{(3.158,1.154)}
\gppoint{gp mark 7}{(3.165,1.155)}
\gppoint{gp mark 7}{(3.172,1.156)}
\gppoint{gp mark 7}{(3.179,1.156)}
\gppoint{gp mark 7}{(3.185,1.171)}
\gppoint{gp mark 7}{(3.192,1.281)}
\gppoint{gp mark 7}{(3.199,1.336)}
\gppoint{gp mark 7}{(3.206,1.334)}
\gppoint{gp mark 7}{(3.213,1.334)}
\gppoint{gp mark 7}{(3.220,1.335)}
\gppoint{gp mark 7}{(3.227,1.336)}
\gppoint{gp mark 7}{(3.234,1.336)}
\gppoint{gp mark 7}{(3.241,1.335)}
\gppoint{gp mark 7}{(3.248,1.335)}
\gppoint{gp mark 7}{(3.255,1.336)}
\gppoint{gp mark 7}{(3.262,1.338)}
\gppoint{gp mark 7}{(3.269,1.338)}
\gppoint{gp mark 7}{(3.276,1.338)}
\gppoint{gp mark 7}{(3.283,1.337)}
\gppoint{gp mark 7}{(3.290,1.337)}
\gppoint{gp mark 7}{(3.297,1.337)}
\gppoint{gp mark 7}{(3.304,1.337)}
\gppoint{gp mark 7}{(3.311,1.337)}
\gppoint{gp mark 7}{(3.318,1.336)}
\gppoint{gp mark 7}{(3.325,1.335)}
\gppoint{gp mark 7}{(3.331,1.335)}
\gppoint{gp mark 7}{(3.338,1.336)}
\gppoint{gp mark 7}{(3.345,1.336)}
\gppoint{gp mark 7}{(3.352,1.336)}
\gppoint{gp mark 7}{(3.359,1.336)}
\gppoint{gp mark 7}{(3.366,1.336)}
\gppoint{gp mark 7}{(3.373,1.336)}
\gppoint{gp mark 7}{(3.380,1.336)}
\gppoint{gp mark 7}{(3.387,1.336)}
\gppoint{gp mark 7}{(3.394,1.336)}
\gppoint{gp mark 7}{(3.401,1.335)}
\gppoint{gp mark 7}{(3.408,1.335)}
\gppoint{gp mark 7}{(3.415,1.335)}
\gppoint{gp mark 7}{(3.422,1.335)}
\gppoint{gp mark 7}{(3.429,1.335)}
\gppoint{gp mark 7}{(3.436,1.335)}
\gppoint{gp mark 7}{(3.443,1.334)}
\gppoint{gp mark 7}{(3.450,1.334)}
\gppoint{gp mark 7}{(3.457,1.334)}
\gppoint{gp mark 7}{(3.464,1.335)}
\gppoint{gp mark 7}{(3.471,1.335)}
\gppoint{gp mark 7}{(3.477,1.335)}
\gppoint{gp mark 7}{(3.484,1.335)}
\gppoint{gp mark 7}{(3.491,1.335)}
\gppoint{gp mark 7}{(3.498,1.336)}
\gppoint{gp mark 7}{(3.505,1.336)}
\gppoint{gp mark 7}{(3.512,1.336)}
\gppoint{gp mark 7}{(3.519,1.336)}
\gppoint{gp mark 7}{(3.526,1.335)}
\gppoint{gp mark 7}{(3.533,1.335)}
\gppoint{gp mark 7}{(3.540,1.335)}
\gppoint{gp mark 7}{(3.547,1.336)}
\gppoint{gp mark 7}{(3.554,1.336)}
\gppoint{gp mark 7}{(3.561,1.336)}
\gppoint{gp mark 7}{(3.568,1.336)}
\gppoint{gp mark 7}{(3.575,1.337)}
\gppoint{gp mark 7}{(3.582,1.337)}
\gppoint{gp mark 7}{(3.589,1.337)}
\gppoint{gp mark 7}{(3.596,1.337)}
\gppoint{gp mark 7}{(3.603,1.336)}
\gppoint{gp mark 7}{(3.610,1.336)}
\gppoint{gp mark 7}{(3.617,1.336)}
\gppoint{gp mark 7}{(3.623,1.336)}
\gppoint{gp mark 7}{(3.630,1.336)}
\gppoint{gp mark 7}{(3.637,1.336)}
\gppoint{gp mark 7}{(3.644,1.336)}
\gppoint{gp mark 7}{(3.651,1.336)}
\gppoint{gp mark 7}{(3.658,1.336)}
\gppoint{gp mark 7}{(3.665,1.336)}
\gppoint{gp mark 7}{(3.672,1.336)}
\gppoint{gp mark 7}{(3.679,1.336)}
\gppoint{gp mark 7}{(3.686,1.336)}
\gppoint{gp mark 7}{(3.693,1.336)}
\gppoint{gp mark 7}{(3.700,1.336)}
\gppoint{gp mark 7}{(3.707,1.336)}
\gppoint{gp mark 7}{(3.714,1.335)}
\gppoint{gp mark 7}{(3.721,1.335)}
\gppoint{gp mark 7}{(3.728,1.335)}
\gppoint{gp mark 7}{(3.735,1.335)}
\gppoint{gp mark 7}{(3.742,1.335)}
\gppoint{gp mark 7}{(3.749,1.335)}
\gppoint{gp mark 7}{(3.756,1.335)}
\gppoint{gp mark 7}{(3.763,1.335)}
\gppoint{gp mark 7}{(3.769,1.336)}
\gppoint{gp mark 7}{(3.776,1.336)}
\gppoint{gp mark 7}{(3.783,1.336)}
\gppoint{gp mark 7}{(3.790,1.336)}
\gppoint{gp mark 7}{(3.797,1.336)}
\gppoint{gp mark 7}{(3.804,1.336)}
\gppoint{gp mark 7}{(3.811,1.336)}
\gppoint{gp mark 7}{(3.818,1.335)}
\gppoint{gp mark 7}{(3.825,1.335)}
\gppoint{gp mark 7}{(3.832,1.335)}
\gppoint{gp mark 7}{(3.839,1.335)}
\gppoint{gp mark 7}{(3.846,1.336)}
\gppoint{gp mark 7}{(3.853,1.336)}
\gppoint{gp mark 7}{(3.860,1.336)}
\gppoint{gp mark 7}{(3.867,1.336)}
\gppoint{gp mark 7}{(3.874,1.336)}
\gppoint{gp mark 7}{(3.881,1.336)}
\gppoint{gp mark 7}{(3.888,1.335)}
\gppoint{gp mark 7}{(3.895,1.335)}
\gppoint{gp mark 7}{(3.902,1.335)}
\gppoint{gp mark 7}{(3.908,1.335)}
\gppoint{gp mark 7}{(3.915,1.335)}
\gppoint{gp mark 7}{(3.922,1.335)}
\gppoint{gp mark 7}{(3.929,1.335)}
\gppoint{gp mark 7}{(3.936,1.335)}
\gppoint{gp mark 7}{(3.943,1.335)}
\gppoint{gp mark 7}{(3.950,1.335)}
\gppoint{gp mark 7}{(3.957,1.335)}
\gppoint{gp mark 7}{(3.964,1.335)}
\gppoint{gp mark 7}{(3.971,1.334)}
\gppoint{gp mark 7}{(3.978,1.334)}
\gppoint{gp mark 7}{(3.985,1.334)}
\gppoint{gp mark 7}{(3.992,1.334)}
\gppoint{gp mark 7}{(3.999,1.334)}
\gppoint{gp mark 7}{(4.006,1.335)}
\gppoint{gp mark 7}{(4.013,1.336)}
\gppoint{gp mark 7}{(4.020,1.336)}
\gppoint{gp mark 7}{(4.027,1.336)}
\gppoint{gp mark 7}{(4.034,1.336)}
\gppoint{gp mark 7}{(4.041,1.336)}
\gppoint{gp mark 7}{(4.048,1.335)}
\gppoint{gp mark 7}{(4.054,1.335)}
\gppoint{gp mark 7}{(4.061,1.335)}
\gppoint{gp mark 7}{(4.068,1.335)}
\gppoint{gp mark 7}{(4.075,1.335)}
\gppoint{gp mark 7}{(4.082,1.335)}
\gppoint{gp mark 7}{(4.089,1.335)}
\gppoint{gp mark 7}{(4.096,1.336)}
\gppoint{gp mark 7}{(4.103,1.336)}
\gppoint{gp mark 7}{(4.110,1.336)}
\gppoint{gp mark 7}{(4.117,1.336)}
\gppoint{gp mark 7}{(4.124,1.336)}
\gppoint{gp mark 7}{(4.131,1.336)}
\gppoint{gp mark 7}{(4.138,1.336)}
\gppoint{gp mark 7}{(4.145,1.336)}
\gppoint{gp mark 7}{(4.152,1.336)}
\gppoint{gp mark 7}{(4.159,1.336)}
\gppoint{gp mark 7}{(4.166,1.336)}
\gppoint{gp mark 7}{(4.173,1.336)}
\gppoint{gp mark 7}{(4.180,1.336)}
\gppoint{gp mark 7}{(4.187,1.337)}
\gppoint{gp mark 7}{(4.194,1.337)}
\gppoint{gp mark 7}{(4.200,1.337)}
\gppoint{gp mark 7}{(4.207,1.337)}
\gppoint{gp mark 7}{(4.214,1.337)}
\gppoint{gp mark 7}{(4.221,1.337)}
\gppoint{gp mark 7}{(4.228,1.337)}
\gppoint{gp mark 7}{(4.235,1.337)}
\gppoint{gp mark 7}{(4.242,1.336)}
\gppoint{gp mark 7}{(4.249,1.336)}
\gppoint{gp mark 7}{(4.256,1.336)}
\gppoint{gp mark 7}{(4.263,1.336)}
\gppoint{gp mark 7}{(4.270,1.336)}
\gppoint{gp mark 7}{(4.277,1.335)}
\gppoint{gp mark 7}{(4.284,1.336)}
\gppoint{gp mark 7}{(4.291,1.336)}
\gppoint{gp mark 7}{(4.298,1.337)}
\gppoint{gp mark 7}{(4.305,1.336)}
\gppoint{gp mark 7}{(4.312,1.335)}
\gppoint{gp mark 7}{(4.319,1.335)}
\gppoint{gp mark 7}{(4.326,1.337)}
\gppoint{gp mark 7}{(4.333,1.337)}
\gppoint{gp mark 7}{(4.340,1.335)}
\gppoint{gp mark 7}{(4.346,1.333)}
\gppoint{gp mark 7}{(4.353,1.330)}
\gppoint{gp mark 7}{(4.360,1.262)}
\gppoint{gp mark 7}{(4.367,1.146)}
\gppoint{gp mark 7}{(4.374,1.132)}
\gppoint{gp mark 7}{(4.381,1.128)}
\gppoint{gp mark 7}{(4.388,1.127)}
\gppoint{gp mark 7}{(4.395,1.127)}
\gppoint{gp mark 7}{(4.402,1.127)}
\gppoint{gp mark 7}{(4.409,1.127)}
\gppoint{gp mark 7}{(4.416,1.128)}
\gppoint{gp mark 7}{(4.423,1.129)}
\gppoint{gp mark 7}{(4.430,1.129)}
\gppoint{gp mark 7}{(4.437,1.129)}
\gppoint{gp mark 7}{(4.444,1.128)}
\gppoint{gp mark 7}{(4.451,1.127)}
\gppoint{gp mark 7}{(4.458,1.127)}
\gppoint{gp mark 7}{(4.465,1.128)}
\gppoint{gp mark 7}{(4.472,1.128)}
\gppoint{gp mark 7}{(4.479,1.128)}
\gppoint{gp mark 7}{(4.486,1.129)}
\gppoint{gp mark 7}{(4.492,1.128)}
\gppoint{gp mark 7}{(4.499,1.128)}
\gppoint{gp mark 7}{(4.506,1.127)}
\gppoint{gp mark 7}{(4.513,1.127)}
\gppoint{gp mark 7}{(4.520,1.128)}
\gppoint{gp mark 7}{(4.527,1.128)}
\gppoint{gp mark 7}{(4.534,1.128)}
\gppoint{gp mark 7}{(4.541,1.128)}
\gppoint{gp mark 7}{(4.548,1.128)}
\gppoint{gp mark 7}{(4.555,1.127)}
\gppoint{gp mark 7}{(4.562,1.127)}
\gppoint{gp mark 7}{(4.569,1.127)}
\gppoint{gp mark 7}{(4.576,1.128)}
\gppoint{gp mark 7}{(4.583,1.128)}
\gppoint{gp mark 7}{(4.590,1.128)}
\gppoint{gp mark 7}{(4.597,1.128)}
\gppoint{gp mark 7}{(4.604,1.128)}
\gppoint{gp mark 7}{(4.611,1.127)}
\gppoint{gp mark 7}{(4.618,1.127)}
\gppoint{gp mark 7}{(4.625,1.127)}
\gppoint{gp mark 7}{(4.632,1.127)}
\gppoint{gp mark 7}{(4.638,1.128)}
\gppoint{gp mark 7}{(4.645,1.128)}
\gppoint{gp mark 7}{(4.652,1.128)}
\gppoint{gp mark 7}{(4.659,1.128)}
\gppoint{gp mark 7}{(4.666,1.127)}
\gppoint{gp mark 7}{(4.673,1.127)}
\gppoint{gp mark 7}{(4.680,1.127)}
\gppoint{gp mark 7}{(4.687,1.127)}
\gppoint{gp mark 7}{(4.694,1.128)}
\gppoint{gp mark 7}{(4.701,1.128)}
\gppoint{gp mark 7}{(4.708,1.128)}
\gppoint{gp mark 7}{(4.715,1.128)}
\gppoint{gp mark 7}{(4.722,1.128)}
\gppoint{gp mark 7}{(4.729,1.127)}
\gppoint{gp mark 7}{(4.736,1.127)}
\gppoint{gp mark 7}{(4.743,1.127)}
\gppoint{gp mark 7}{(4.750,1.127)}
\gppoint{gp mark 7}{(4.757,1.127)}
\gppoint{gp mark 7}{(4.764,1.128)}
\gppoint{gp mark 7}{(4.771,1.128)}
\gppoint{gp mark 7}{(4.778,1.128)}
\gppoint{gp mark 7}{(4.784,1.127)}
\gppoint{gp mark 7}{(4.791,1.127)}
\gppoint{gp mark 7}{(4.798,1.127)}
\gppoint{gp mark 7}{(4.805,1.127)}
\gppoint{gp mark 7}{(4.812,1.128)}
\gppoint{gp mark 7}{(4.819,1.128)}
\gppoint{gp mark 7}{(4.826,1.128)}
\gppoint{gp mark 7}{(4.833,1.128)}
\gppoint{gp mark 7}{(4.840,1.128)}
\gppoint{gp mark 7}{(4.847,1.127)}
\gppoint{gp mark 7}{(4.854,1.127)}
\gppoint{gp mark 7}{(4.861,1.127)}
\gppoint{gp mark 7}{(4.868,1.127)}
\gppoint{gp mark 7}{(4.875,1.128)}
\gppoint{gp mark 7}{(4.882,1.128)}
\gppoint{gp mark 7}{(4.889,1.128)}
\gppoint{gp mark 7}{(4.896,1.128)}
\gppoint{gp mark 7}{(4.903,1.127)}
\gppoint{gp mark 7}{(4.910,1.127)}
\gppoint{gp mark 7}{(4.917,1.127)}
\gppoint{gp mark 7}{(4.924,1.127)}
\gppoint{gp mark 7}{(4.930,1.128)}
\gppoint{gp mark 7}{(4.937,1.128)}
\gppoint{gp mark 7}{(4.944,1.128)}
\gppoint{gp mark 7}{(4.951,1.128)}
\gppoint{gp mark 7}{(4.958,1.127)}
\gppoint{gp mark 7}{(4.965,1.127)}
\gppoint{gp mark 7}{(4.972,1.127)}
\gppoint{gp mark 7}{(4.979,1.127)}
\gppoint{gp mark 7}{(4.986,1.127)}
\gppoint{gp mark 7}{(4.993,1.128)}
\gppoint{gp mark 7}{(5.000,1.128)}
\gppoint{gp mark 7}{(5.007,1.128)}
\gppoint{gp mark 7}{(5.014,1.128)}
\gppoint{gp mark 7}{(5.021,1.127)}
\gppoint{gp mark 7}{(5.028,1.127)}
\gppoint{gp mark 7}{(5.035,1.127)}
\gppoint{gp mark 7}{(5.042,1.127)}
\gppoint{gp mark 7}{(5.049,1.127)}
\gppoint{gp mark 7}{(5.056,1.128)}
\gppoint{gp mark 7}{(5.063,1.127)}
\gppoint{gp mark 7}{(5.070,1.127)}
\gppoint{gp mark 7}{(5.076,1.127)}
\gppoint{gp mark 7}{(5.083,1.127)}
\gppoint{gp mark 7}{(5.090,1.127)}
\gppoint{gp mark 7}{(5.097,1.128)}
\gppoint{gp mark 7}{(5.104,1.128)}
\gppoint{gp mark 7}{(5.111,1.128)}
\gppoint{gp mark 7}{(5.118,1.127)}
\gppoint{gp mark 7}{(5.125,1.127)}
\gppoint{gp mark 7}{(5.132,1.127)}
\gppoint{gp mark 7}{(5.139,1.127)}
\gppoint{gp mark 7}{(5.146,1.127)}
\gppoint{gp mark 7}{(5.153,1.127)}
\gppoint{gp mark 7}{(5.160,1.127)}
\gppoint{gp mark 7}{(5.167,1.127)}
\gppoint{gp mark 7}{(5.174,1.127)}
\gppoint{gp mark 7}{(5.181,1.127)}
\gppoint{gp mark 7}{(5.188,1.127)}
\gppoint{gp mark 7}{(5.195,1.127)}
\gppoint{gp mark 7}{(5.202,1.127)}
\gppoint{gp mark 7}{(5.209,1.127)}
\gppoint{gp mark 7}{(5.216,1.127)}
\gppoint{gp mark 7}{(5.222,1.127)}
\gppoint{gp mark 7}{(5.229,1.127)}
\gppoint{gp mark 7}{(5.236,1.127)}
\gppoint{gp mark 7}{(5.243,1.127)}
\gppoint{gp mark 7}{(5.250,1.127)}
\gppoint{gp mark 7}{(5.257,1.127)}
\gppoint{gp mark 7}{(5.264,1.127)}
\gppoint{gp mark 7}{(5.271,1.127)}
\gppoint{gp mark 7}{(5.278,1.127)}
\gppoint{gp mark 7}{(5.285,1.127)}
\gppoint{gp mark 7}{(5.292,1.127)}
\gppoint{gp mark 7}{(5.299,1.127)}
\gppoint{gp mark 7}{(5.306,1.127)}
\gppoint{gp mark 7}{(5.313,1.127)}
\gppoint{gp mark 7}{(5.320,1.127)}
\gppoint{gp mark 7}{(5.327,1.127)}
\gppoint{gp mark 7}{(5.334,1.127)}
\gppoint{gp mark 7}{(5.341,1.127)}
\gppoint{gp mark 7}{(5.348,1.127)}
\gppoint{gp mark 7}{(5.355,1.127)}
\gppoint{gp mark 7}{(5.361,1.127)}
\gppoint{gp mark 7}{(5.368,1.127)}
\gppoint{gp mark 7}{(5.375,1.128)}
\gppoint{gp mark 7}{(5.382,1.128)}
\gppoint{gp mark 7}{(5.389,1.128)}
\gppoint{gp mark 7}{(5.396,1.127)}
\gppoint{gp mark 7}{(5.403,1.127)}
\gppoint{gp mark 7}{(5.410,1.127)}
\gppoint{gp mark 7}{(5.417,1.127)}
\gppoint{gp mark 7}{(5.424,1.127)}
\gppoint{gp mark 7}{(5.431,1.127)}
\gppoint{gp mark 7}{(5.438,1.127)}
\gppoint{gp mark 7}{(5.445,1.127)}
\gppoint{gp mark 7}{(5.452,1.127)}
\gppoint{gp mark 7}{(5.459,1.127)}
\gppoint{gp mark 7}{(5.466,1.127)}
\gppoint{gp mark 7}{(5.473,1.127)}
\gppoint{gp mark 7}{(5.480,1.127)}
\gppoint{gp mark 7}{(5.487,1.127)}
\gppoint{gp mark 7}{(5.494,1.127)}
\gppoint{gp mark 7}{(5.501,1.127)}
\gppoint{gp mark 7}{(5.507,1.127)}
\gppoint{gp mark 7}{(5.514,1.127)}
\gppoint{gp mark 7}{(5.521,1.127)}
\gppoint{gp mark 7}{(5.528,1.127)}
\gppoint{gp mark 7}{(5.535,1.127)}
\gppoint{gp mark 7}{(5.542,1.127)}
\gppoint{gp mark 7}{(5.549,1.127)}
\gppoint{gp mark 7}{(5.556,1.127)}
\gppoint{gp mark 7}{(5.563,1.127)}
\gppoint{gp mark 7}{(5.570,1.128)}
\gppoint{gp mark 7}{(5.577,1.128)}
\gppoint{gp mark 7}{(5.584,1.127)}
\gppoint{gp mark 7}{(5.591,1.127)}
\gppoint{gp mark 7}{(5.598,1.127)}
\gppoint{gp mark 7}{(5.605,1.127)}
\gppoint{gp mark 7}{(5.612,1.127)}
\gppoint{gp mark 7}{(5.619,1.127)}
\gppoint{gp mark 7}{(5.626,1.127)}
\gppoint{gp mark 7}{(5.633,1.127)}
\gppoint{gp mark 7}{(5.640,1.127)}
\gppoint{gp mark 7}{(5.647,1.127)}
\gppoint{gp mark 7}{(5.653,1.127)}
\gppoint{gp mark 7}{(5.660,1.127)}
\gppoint{gp mark 7}{(5.667,1.127)}
\gppoint{gp mark 7}{(5.674,1.128)}
\gppoint{gp mark 7}{(5.681,1.128)}
\gppoint{gp mark 7}{(5.688,1.128)}
\gppoint{gp mark 7}{(5.695,1.128)}
\gppoint{gp mark 7}{(5.702,1.127)}
\gppoint{gp mark 7}{(5.709,1.127)}
\gppoint{gp mark 7}{(5.716,1.127)}
\gppoint{gp mark 7}{(5.723,1.127)}
\gppoint{gp mark 7}{(5.730,1.127)}
\gppoint{gp mark 7}{(5.737,1.127)}
\gppoint{gp mark 7}{(5.744,1.127)}
\gppoint{gp mark 7}{(5.751,1.127)}
\gppoint{gp mark 7}{(5.758,1.127)}
\gppoint{gp mark 7}{(5.765,1.127)}
\gppoint{gp mark 7}{(5.772,1.127)}
\gppoint{gp mark 7}{(5.779,1.127)}
\gppoint{gp mark 7}{(5.786,1.127)}
\gppoint{gp mark 7}{(5.793,1.127)}
\gppoint{gp mark 7}{(5.799,1.127)}
\gppoint{gp mark 7}{(5.806,1.127)}
\gppoint{gp mark 7}{(5.813,1.127)}
\gppoint{gp mark 7}{(5.820,1.127)}
\gppoint{gp mark 7}{(5.827,1.127)}
\gppoint{gp mark 7}{(5.834,1.127)}
\gppoint{gp mark 7}{(5.841,1.127)}
\gppoint{gp mark 7}{(5.848,1.127)}
\gppoint{gp mark 7}{(5.855,1.127)}
\gppoint{gp mark 7}{(5.862,1.127)}
\gppoint{gp mark 7}{(5.869,1.127)}
\gppoint{gp mark 7}{(5.876,1.127)}
\gppoint{gp mark 7}{(5.883,1.127)}
\gppoint{gp mark 7}{(5.890,1.127)}
\gppoint{gp mark 7}{(5.897,1.127)}
\gppoint{gp mark 7}{(5.904,1.127)}
\gppoint{gp mark 7}{(5.911,1.127)}
\gppoint{gp mark 7}{(5.918,1.127)}
\gppoint{gp mark 7}{(5.925,1.127)}
\gppoint{gp mark 7}{(5.932,1.127)}
\gppoint{gp mark 7}{(5.939,1.127)}
\gppoint{gp mark 7}{(5.945,1.127)}
\gppoint{gp mark 7}{(5.952,1.127)}
\gppoint{gp mark 7}{(5.959,1.127)}
\gppoint{gp mark 7}{(5.966,1.127)}
\gppoint{gp mark 7}{(5.973,1.127)}
\gppoint{gp mark 7}{(5.980,1.127)}
\gppoint{gp mark 7}{(5.987,1.127)}
\gppoint{gp mark 7}{(5.994,1.127)}
\gppoint{gp mark 7}{(6.001,1.127)}
\gppoint{gp mark 7}{(6.008,1.127)}
\gppoint{gp mark 7}{(6.015,1.127)}
\gppoint{gp mark 7}{(6.022,1.127)}
\gppoint{gp mark 7}{(6.029,1.127)}
\gppoint{gp mark 7}{(6.036,1.127)}
\gppoint{gp mark 7}{(6.043,1.127)}
\gppoint{gp mark 7}{(6.050,1.127)}
\gppoint{gp mark 7}{(6.057,1.127)}
\gppoint{gp mark 7}{(6.064,1.127)}
\gppoint{gp mark 7}{(6.071,1.127)}
\gppoint{gp mark 7}{(6.078,1.127)}
\gppoint{gp mark 7}{(6.085,1.127)}
\gppoint{gp mark 7}{(6.091,1.127)}
\gppoint{gp mark 7}{(6.098,1.127)}
\gppoint{gp mark 7}{(6.105,1.127)}
\gppoint{gp mark 7}{(6.112,1.127)}
\gppoint{gp mark 7}{(6.119,1.127)}
\gppoint{gp mark 7}{(6.126,1.126)}
\gppoint{gp mark 7}{(6.133,1.126)}
\gppoint{gp mark 7}{(6.140,1.126)}
\gppoint{gp mark 7}{(6.147,1.126)}
\gppoint{gp mark 7}{(6.154,1.126)}
\gppoint{gp mark 7}{(6.161,1.127)}
\gppoint{gp mark 7}{(6.168,1.127)}
\gppoint{gp mark 7}{(6.175,1.127)}
\gppoint{gp mark 7}{(6.182,1.127)}
\gppoint{gp mark 7}{(6.189,1.127)}
\gppoint{gp mark 7}{(6.196,1.127)}
\gppoint{gp mark 7}{(6.203,1.127)}
\gppoint{gp mark 7}{(6.210,1.127)}
\gppoint{gp mark 7}{(6.217,1.127)}
\gppoint{gp mark 7}{(6.224,1.127)}
\gppoint{gp mark 7}{(6.231,1.127)}
\gppoint{gp mark 7}{(6.237,1.126)}
\gppoint{gp mark 7}{(6.244,1.126)}
\gppoint{gp mark 7}{(6.251,1.126)}
\gppoint{gp mark 7}{(6.258,1.126)}
\gppoint{gp mark 7}{(6.265,1.126)}
\gppoint{gp mark 7}{(6.272,1.126)}
\gppoint{gp mark 7}{(6.279,1.126)}
\gppoint{gp mark 7}{(6.286,1.126)}
\gppoint{gp mark 7}{(6.293,1.126)}
\gppoint{gp mark 7}{(6.300,1.125)}
\gppoint{gp mark 7}{(6.307,1.124)}
\gppoint{gp mark 7}{(6.314,1.123)}
\gppoint{gp mark 7}{(6.321,1.123)}
\gppoint{gp mark 7}{(6.328,1.123)}
\gppoint{gp mark 7}{(6.335,1.124)}
\gppoint{gp mark 7}{(6.342,1.125)}
\gppoint{gp mark 7}{(6.349,1.125)}
\gppoint{gp mark 7}{(6.356,1.125)}
\gppoint{gp mark 7}{(6.363,1.126)}
\gppoint{gp mark 7}{(6.370,1.127)}
\gppoint{gp mark 7}{(6.377,1.160)}
\gppoint{gp mark 7}{(6.383,1.366)}
\gppoint{gp mark 7}{(6.390,2.202)}
\gppoint{gp mark 7}{(6.397,3.190)}
\gppoint{gp mark 7}{(6.404,3.593)}
\gppoint{gp mark 7}{(6.411,3.822)}
\gppoint{gp mark 7}{(6.418,3.956)}
\gppoint{gp mark 7}{(6.425,4.025)}
\gppoint{gp mark 7}{(6.432,4.057)}
\gppoint{gp mark 7}{(6.439,4.070)}
\gppoint{gp mark 7}{(6.446,4.076)}
\gppoint{gp mark 7}{(6.453,4.078)}
\gppoint{gp mark 7}{(6.460,4.078)}
\gppoint{gp mark 7}{(6.467,4.078)}
\gppoint{gp mark 7}{(6.474,4.079)}
\gppoint{gp mark 7}{(6.481,4.079)}
\gppoint{gp mark 7}{(6.488,4.078)}
\gppoint{gp mark 7}{(6.495,4.078)}
\gppoint{gp mark 7}{(6.502,4.078)}
\gppoint{gp mark 7}{(6.509,4.078)}
\gppoint{gp mark 7}{(6.516,4.077)}
\gppoint{gp mark 7}{(6.523,4.077)}
\gppoint{gp mark 7}{(6.529,4.076)}
\gppoint{gp mark 7}{(6.536,4.076)}
\gppoint{gp mark 7}{(6.543,4.076)}
\gppoint{gp mark 7}{(6.550,4.076)}
\gppoint{gp mark 7}{(6.557,4.076)}
\gppoint{gp mark 7}{(6.564,4.076)}
\gppoint{gp mark 7}{(6.571,4.076)}
\gppoint{gp mark 7}{(6.578,4.076)}
\gppoint{gp mark 7}{(6.585,4.076)}
\gppoint{gp mark 7}{(6.592,4.076)}
\gppoint{gp mark 7}{(6.599,4.076)}
\gppoint{gp mark 7}{(6.606,4.076)}
\gppoint{gp mark 7}{(6.613,4.076)}
\gppoint{gp mark 7}{(6.620,4.076)}
\gppoint{gp mark 7}{(6.627,4.076)}
\gppoint{gp mark 7}{(6.634,4.076)}
\gppoint{gp mark 7}{(6.641,4.075)}
\gppoint{gp mark 7}{(6.648,4.075)}
\gppoint{gp mark 7}{(6.655,4.075)}
\gppoint{gp mark 7}{(6.662,4.074)}
\gppoint{gp mark 7}{(6.668,4.074)}
\gppoint{gp mark 7}{(6.675,4.073)}
\gppoint{gp mark 7}{(6.682,4.073)}
\gppoint{gp mark 7}{(6.689,4.073)}
\gppoint{gp mark 7}{(6.696,4.073)}
\gppoint{gp mark 7}{(6.703,4.073)}
\gppoint{gp mark 7}{(6.710,4.073)}
\gppoint{gp mark 7}{(6.717,4.072)}
\gppoint{gp mark 7}{(6.724,4.072)}
\gppoint{gp mark 7}{(6.731,4.071)}
\gppoint{gp mark 7}{(6.738,4.071)}
\gppoint{gp mark 7}{(6.745,4.071)}
\gppoint{gp mark 7}{(6.752,4.070)}
\gppoint{gp mark 7}{(6.759,4.069)}
\gppoint{gp mark 7}{(6.766,4.068)}
\gppoint{gp mark 7}{(6.773,4.066)}
\gppoint{gp mark 7}{(6.780,4.065)}
\gppoint{gp mark 7}{(6.787,4.065)}
\gppoint{gp mark 7}{(6.794,4.065)}
\gppoint{gp mark 7}{(6.801,4.065)}
\gppoint{gp mark 7}{(6.808,4.065)}
\gppoint{gp mark 7}{(6.814,4.065)}
\gppoint{gp mark 7}{(6.821,4.065)}
\gppoint{gp mark 7}{(6.828,4.065)}
\gppoint{gp mark 7}{(6.835,4.065)}
\gppoint{gp mark 7}{(6.842,4.064)}
\gppoint{gp mark 7}{(6.849,4.056)}
\gppoint{gp mark 7}{(6.856,4.039)}
\gppoint{gp mark 7}{(6.863,4.024)}
\gppoint{gp mark 7}{(6.870,4.018)}
\gppoint{gp mark 7}{(6.877,4.016)}
\gppoint{gp mark 7}{(6.884,4.016)}
\gppoint{gp mark 7}{(6.891,4.016)}
\gppoint{gp mark 7}{(6.898,4.017)}
\gppoint{gp mark 7}{(6.905,4.021)}
\gppoint{gp mark 7}{(6.912,4.028)}
\gppoint{gp mark 7}{(6.919,4.035)}
\gppoint{gp mark 7}{(6.926,4.042)}
\gppoint{gp mark 7}{(6.933,4.048)}
\gppoint{gp mark 7}{(6.940,4.054)}
\gppoint{gp mark 7}{(6.947,4.060)}
\gppoint{gp mark 7}{(6.954,4.066)}
\gppoint{gp mark 7}{(6.960,4.071)}
\gppoint{gp mark 7}{(6.967,4.076)}
\gppoint{gp mark 7}{(6.974,4.082)}
\gppoint{gp mark 7}{(6.981,4.087)}
\gppoint{gp mark 7}{(6.988,4.092)}
\gppoint{gp mark 7}{(6.995,4.097)}
\gppoint{gp mark 7}{(7.002,4.103)}
\gppoint{gp mark 7}{(7.009,4.108)}
\gppoint{gp mark 7}{(7.016,4.113)}
\gppoint{gp mark 7}{(7.023,4.118)}
\gppoint{gp mark 7}{(7.030,4.123)}
\gppoint{gp mark 7}{(7.037,4.129)}
\gppoint{gp mark 7}{(7.044,4.134)}
\gppoint{gp mark 7}{(7.051,4.139)}
\gppoint{gp mark 7}{(7.058,4.144)}
\gppoint{gp mark 7}{(7.065,4.149)}
\gppoint{gp mark 7}{(7.072,4.154)}
\gppoint{gp mark 7}{(7.079,4.159)}
\gppoint{gp mark 7}{(7.086,4.164)}
\gppoint{gp mark 7}{(7.093,4.169)}
\gppoint{gp mark 7}{(7.100,4.174)}
\gppoint{gp mark 7}{(7.106,4.180)}
\gppoint{gp mark 7}{(7.113,4.185)}
\gppoint{gp mark 7}{(7.120,4.190)}
\gppoint{gp mark 7}{(7.127,4.195)}
\gppoint{gp mark 7}{(7.134,4.200)}
\gppoint{gp mark 7}{(7.141,4.205)}
\gppoint{gp mark 7}{(7.148,4.210)}
\gppoint{gp mark 7}{(7.155,4.215)}
\gppoint{gp mark 7}{(7.162,4.220)}
\gppoint{gp mark 7}{(7.169,4.225)}
\gppoint{gp mark 7}{(7.176,4.230)}
\gppoint{gp mark 7}{(7.183,4.235)}
\gppoint{gp mark 7}{(7.190,4.240)}
\gppoint{gp mark 7}{(7.197,4.245)}
\gppoint{gp mark 7}{(7.204,4.249)}
\gppoint{gp mark 7}{(7.211,4.254)}
\gppoint{gp mark 7}{(7.218,4.259)}
\gppoint{gp mark 7}{(7.225,4.264)}
\gppoint{gp mark 7}{(7.232,4.269)}
\gppoint{gp mark 7}{(7.239,4.274)}
\gppoint{gp mark 7}{(7.246,4.279)}
\gppoint{gp mark 7}{(7.252,4.284)}
\gppoint{gp mark 7}{(7.259,4.289)}
\gppoint{gp mark 7}{(7.266,4.294)}
\gppoint{gp mark 7}{(7.273,4.298)}
\gppoint{gp mark 7}{(7.280,4.303)}
\gppoint{gp mark 7}{(7.287,4.308)}
\gppoint{gp mark 7}{(7.294,4.313)}
\gppoint{gp mark 7}{(7.301,4.318)}
\gppoint{gp mark 7}{(7.308,4.323)}
\gppoint{gp mark 7}{(7.315,4.327)}
\gppoint{gp mark 7}{(7.322,4.332)}
\gppoint{gp mark 7}{(7.329,4.337)}
\gppoint{gp mark 7}{(7.336,4.342)}
\gppoint{gp mark 7}{(7.343,4.347)}
\gppoint{gp mark 7}{(7.350,4.351)}
\gppoint{gp mark 7}{(7.357,4.356)}
\gppoint{gp mark 7}{(7.364,4.361)}
\gppoint{gp mark 7}{(7.371,4.366)}
\gppoint{gp mark 7}{(7.378,4.370)}
\gppoint{gp mark 7}{(7.385,4.375)}
\gppoint{gp mark 7}{(7.392,4.380)}
\gppoint{gp mark 7}{(7.398,4.385)}
\gppoint{gp mark 7}{(7.405,4.389)}
\gppoint{gp mark 7}{(7.412,4.394)}
\gppoint{gp mark 7}{(7.419,4.399)}
\gppoint{gp mark 7}{(7.426,4.404)}
\gppoint{gp mark 7}{(7.433,4.408)}
\gppoint{gp mark 7}{(7.440,4.413)}
\gppoint{gp mark 7}{(7.447,4.418)}
\gppoint{gp mark 7}{(7.454,4.422)}
\gppoint{gp mark 7}{(7.461,4.427)}
\gppoint{gp mark 7}{(7.468,4.432)}
\gppoint{gp mark 7}{(7.475,4.436)}
\gppoint{gp mark 7}{(7.482,4.441)}
\gppoint{gp mark 7}{(7.489,4.446)}
\gppoint{gp mark 7}{(7.496,4.450)}
\gppoint{gp mark 7}{(7.503,4.455)}
\gppoint{gp mark 7}{(7.510,4.460)}
\gppoint{gp mark 7}{(7.517,4.464)}
\gppoint{gp mark 7}{(7.524,4.469)}
\gppoint{gp mark 7}{(7.531,4.473)}
\gppoint{gp mark 7}{(7.538,4.478)}
\gppoint{gp mark 7}{(7.544,4.483)}
\gppoint{gp mark 7}{(7.551,4.487)}
\gppoint{gp mark 7}{(7.558,4.492)}
\gppoint{gp mark 7}{(7.565,4.496)}
\gppoint{gp mark 7}{(7.572,4.501)}
\gppoint{gp mark 7}{(7.579,4.505)}
\gppoint{gp mark 7}{(7.586,4.510)}
\gppoint{gp mark 7}{(7.593,4.515)}
\gppoint{gp mark 7}{(7.600,4.519)}
\gppoint{gp mark 7}{(7.607,4.524)}
\gppoint{gp mark 7}{(7.614,4.528)}
\gppoint{gp mark 7}{(7.621,4.533)}
\gppoint{gp mark 7}{(7.628,4.537)}
\gppoint{gp mark 7}{(7.635,4.542)}
\gppoint{gp mark 7}{(7.642,4.546)}
\gppoint{gp mark 7}{(7.649,4.551)}
\gppoint{gp mark 7}{(7.656,4.555)}
\gppoint{gp mark 7}{(7.663,4.560)}
\gppoint{gp mark 7}{(7.670,4.564)}
\gppoint{gp mark 7}{(7.677,4.569)}
\gppoint{gp mark 7}{(7.684,4.573)}
\gppoint{gp mark 7}{(7.690,4.578)}
\gppoint{gp mark 7}{(7.697,4.582)}
\gppoint{gp mark 7}{(7.704,4.587)}
\gppoint{gp mark 7}{(7.711,4.591)}
\gppoint{gp mark 7}{(7.718,4.596)}
\gppoint{gp mark 7}{(7.725,4.600)}
\gppoint{gp mark 7}{(7.732,4.605)}
\gppoint{gp mark 7}{(7.739,4.609)}
\gppoint{gp mark 7}{(7.746,4.614)}
\gppoint{gp mark 7}{(7.753,4.618)}
\gppoint{gp mark 7}{(7.760,4.623)}
\gppoint{gp mark 7}{(7.767,4.627)}
\gppoint{gp mark 7}{(7.774,4.631)}
\gppoint{gp mark 7}{(7.781,4.636)}
\gppoint{gp mark 7}{(7.788,4.640)}
\gppoint{gp mark 7}{(7.795,4.645)}
\gppoint{gp mark 7}{(7.802,4.649)}
\gppoint{gp mark 7}{(7.809,4.653)}
\gppoint{gp mark 7}{(7.816,4.658)}
\gppoint{gp mark 7}{(7.823,4.662)}
\gppoint{gp mark 7}{(7.830,4.667)}
\gppoint{gp mark 7}{(7.836,4.671)}
\gppoint{gp mark 7}{(7.843,4.675)}
\gppoint{gp mark 7}{(7.850,4.680)}
\gppoint{gp mark 7}{(7.857,4.684)}
\gppoint{gp mark 7}{(7.864,4.689)}
\gppoint{gp mark 7}{(7.871,4.693)}
\gppoint{gp mark 7}{(7.878,4.697)}
\gppoint{gp mark 7}{(7.885,4.702)}
\gppoint{gp mark 7}{(7.892,4.706)}
\gppoint{gp mark 7}{(7.899,4.710)}
\gppoint{gp mark 7}{(7.906,4.715)}
\gppoint{gp mark 7}{(7.913,4.719)}
\gppoint{gp mark 7}{(7.920,4.723)}
\gppoint{gp mark 7}{(7.927,4.728)}
\gppoint{gp mark 7}{(7.934,4.732)}
\gppoint{gp mark 7}{(7.941,4.737)}
\gpcolor{rgb color={0.000,0.000,0.000}}
\gpsetlinetype{gp lt plot 0}
\draw[gp path] (1.361,4.567)--(1.769,4.567);
\draw[gp path] (1.769,1.153)--(3.245,1.153);
\draw[gp path] (3.245,1.335)--(3.746,1.335);
\draw[gp path] (3.746,1.335)--(4.294,1.335);
\draw[gp path] (4.294,1.127)--(6.394,1.127);
\draw[gp path] (6.394,4.081)--(6.972,4.081);
\draw[gp path] (0.829,5.258)--(0.833,5.253)--(0.837,5.248)--(0.841,5.243)--(0.845,5.238)%
  --(0.849,5.233)--(0.853,5.228)--(0.857,5.223)--(0.861,5.218)--(0.864,5.213)--(0.868,5.208)%
  --(0.872,5.203)--(0.876,5.197)--(0.880,5.192)--(0.884,5.187)--(0.888,5.182)--(0.892,5.177)%
  --(0.896,5.172)--(0.900,5.167)--(0.904,5.162)--(0.907,5.157)--(0.911,5.152)--(0.915,5.147)%
  --(0.919,5.142)--(0.923,5.136)--(0.927,5.131)--(0.931,5.126)--(0.935,5.121)--(0.939,5.116)%
  --(0.943,5.111)--(0.947,5.106)--(0.950,5.101)--(0.954,5.096)--(0.958,5.091)--(0.962,5.086)%
  --(0.966,5.081)--(0.970,5.075)--(0.974,5.070)--(0.978,5.065)--(0.982,5.060)--(0.986,5.055)%
  --(0.990,5.050)--(0.993,5.045)--(0.997,5.040)--(1.001,5.035)--(1.005,5.030)--(1.009,5.025)%
  --(1.013,5.020)--(1.017,5.014)--(1.021,5.009)--(1.025,5.004)--(1.029,4.999)--(1.033,4.994)%
  --(1.036,4.989)--(1.040,4.984)--(1.044,4.979)--(1.048,4.974)--(1.052,4.969)--(1.056,4.964)%
  --(1.060,4.959)--(1.064,4.953)--(1.068,4.948)--(1.072,4.943)--(1.076,4.938)--(1.079,4.933)%
  --(1.083,4.928)--(1.087,4.923)--(1.091,4.918)--(1.095,4.913)--(1.099,4.908)--(1.103,4.903)%
  --(1.107,4.898)--(1.111,4.892)--(1.115,4.887)--(1.119,4.882)--(1.122,4.877)--(1.126,4.872)%
  --(1.130,4.867)--(1.134,4.862)--(1.138,4.857)--(1.142,4.852)--(1.146,4.847)--(1.150,4.842)%
  --(1.154,4.837)--(1.158,4.831)--(1.162,4.826)--(1.165,4.821)--(1.169,4.816)--(1.173,4.811)%
  --(1.177,4.806)--(1.181,4.801)--(1.185,4.796)--(1.189,4.791)--(1.193,4.786)--(1.197,4.781)%
  --(1.201,4.776)--(1.204,4.770)--(1.208,4.765)--(1.212,4.760)--(1.216,4.755)--(1.220,4.750)%
  --(1.224,4.745)--(1.228,4.740)--(1.232,4.735)--(1.236,4.730)--(1.240,4.725)--(1.244,4.720)%
  --(1.247,4.715)--(1.251,4.709)--(1.255,4.704)--(1.259,4.699)--(1.263,4.694)--(1.267,4.689)%
  --(1.271,4.684)--(1.275,4.679)--(1.279,4.674)--(1.283,4.669)--(1.287,4.664)--(1.290,4.659)%
  --(1.294,4.654)--(1.298,4.648)--(1.302,4.643)--(1.306,4.638)--(1.310,4.633)--(1.314,4.628)%
  --(1.318,4.623)--(1.322,4.618)--(1.326,4.613)--(1.330,4.608)--(1.333,4.603)--(1.337,4.598)%
  --(1.341,4.593)--(1.345,4.587)--(1.349,4.582)--(1.353,4.577)--(1.357,4.572)--(1.361,4.567);
\draw[gp path] (1.769,4.567)--(1.769,1.153);
\draw[gp path] (3.245,1.153)--(3.245,1.335);
\draw[gp path] (4.294,1.335)--(4.294,1.127);
\draw[gp path] (6.394,1.127)--(6.394,4.081);
\draw[gp path] (6.972,4.081)--(6.978,4.084)--(6.983,4.088)--(6.988,4.091)--(6.994,4.095)%
  --(6.999,4.099)--(7.004,4.102)--(7.010,4.106)--(7.015,4.109)--(7.020,4.113)--(7.026,4.116)%
  --(7.031,4.120)--(7.036,4.123)--(7.042,4.127)--(7.047,4.130)--(7.052,4.134)--(7.057,4.137)%
  --(7.063,4.141)--(7.068,4.144)--(7.073,4.148)--(7.079,4.151)--(7.084,4.155)--(7.089,4.158)%
  --(7.095,4.162)--(7.100,4.165)--(7.105,4.169)--(7.111,4.172)--(7.116,4.176)--(7.121,4.179)%
  --(7.127,4.183)--(7.132,4.187)--(7.137,4.190)--(7.143,4.194)--(7.148,4.197)--(7.153,4.201)%
  --(7.159,4.204)--(7.164,4.208)--(7.169,4.211)--(7.175,4.215)--(7.180,4.218)--(7.185,4.222)%
  --(7.191,4.225)--(7.196,4.229)--(7.201,4.232)--(7.207,4.236)--(7.212,4.239)--(7.217,4.243)%
  --(7.223,4.246)--(7.228,4.250)--(7.233,4.253)--(7.239,4.257)--(7.244,4.260)--(7.249,4.264)%
  --(7.255,4.268)--(7.260,4.271)--(7.265,4.275)--(7.271,4.278)--(7.276,4.282)--(7.281,4.285)%
  --(7.287,4.289)--(7.292,4.292)--(7.297,4.296)--(7.303,4.299)--(7.308,4.303)--(7.313,4.306)%
  --(7.318,4.310)--(7.324,4.313)--(7.329,4.317)--(7.334,4.320)--(7.340,4.324)--(7.345,4.327)%
  --(7.350,4.331)--(7.356,4.334)--(7.361,4.338)--(7.366,4.341)--(7.372,4.345)--(7.377,4.349)%
  --(7.382,4.352)--(7.388,4.356)--(7.393,4.359)--(7.398,4.363)--(7.404,4.366)--(7.409,4.370)%
  --(7.414,4.373)--(7.420,4.377)--(7.425,4.380)--(7.430,4.384)--(7.436,4.387)--(7.441,4.391)%
  --(7.446,4.394)--(7.452,4.398)--(7.457,4.401)--(7.462,4.405)--(7.468,4.408)--(7.473,4.412)%
  --(7.478,4.415)--(7.484,4.419)--(7.489,4.422)--(7.494,4.426)--(7.500,4.429)--(7.505,4.433)%
  --(7.510,4.437)--(7.516,4.440)--(7.521,4.444)--(7.526,4.447)--(7.532,4.451)--(7.537,4.454)%
  --(7.542,4.458)--(7.548,4.461)--(7.553,4.465)--(7.558,4.468)--(7.564,4.472)--(7.569,4.475)%
  --(7.574,4.479)--(7.580,4.482)--(7.585,4.486)--(7.590,4.489)--(7.595,4.493)--(7.601,4.496)%
  --(7.606,4.500)--(7.611,4.503)--(7.617,4.507)--(7.622,4.510)--(7.627,4.514)--(7.633,4.518)%
  --(7.638,4.521)--(7.643,4.525)--(7.649,4.528)--(7.654,4.532)--(7.659,4.535)--(7.665,4.539)%
  --(7.670,4.542)--(7.675,4.546)--(7.681,4.549)--(7.686,4.553)--(7.691,4.556)--(7.697,4.560)%
  --(7.702,4.563)--(7.707,4.567)--(7.713,4.570)--(7.718,4.574)--(7.723,4.577)--(7.729,4.581)%
  --(7.734,4.584)--(7.739,4.588)--(7.745,4.591)--(7.750,4.595)--(7.755,4.599)--(7.761,4.602)%
  --(7.766,4.606)--(7.771,4.609)--(7.777,4.613)--(7.782,4.616)--(7.787,4.620)--(7.793,4.623)%
  --(7.798,4.627)--(7.803,4.630)--(7.809,4.634)--(7.814,4.637)--(7.819,4.641)--(7.825,4.644)%
  --(7.830,4.648)--(7.835,4.651)--(7.841,4.655)--(7.846,4.658)--(7.851,4.662)--(7.856,4.665)%
  --(7.862,4.669)--(7.867,4.672)--(7.872,4.676)--(7.878,4.679)--(7.883,4.683)--(7.888,4.687)%
  --(7.894,4.690)--(7.899,4.694)--(7.904,4.697)--(7.910,4.701)--(7.915,4.704)--(7.920,4.708)%
  --(7.926,4.711)--(7.931,4.715)--(7.936,4.718)--(7.942,4.722);
\draw[gp path] (2.252,3.540)--(2.679,3.540);
\gpcolor{rgb color={1.000,0.000,0.000}}
\gppoint{gp mark 7}{(2.465,3.076)}
\gpcolor{rgb color={0.502,0.502,0.502}}
\gppoint{gp mark 7}{(2.465,2.611)}
\gpcolor{rgb color={0.000,0.000,0.000}}
\node[gp node left,font={\fontsize{10pt}{12pt}\selectfont}] at (0.970,5.166) {\LARGE $B_y$};
\node[gp node left,font={\fontsize{10pt}{12pt}\selectfont}] at (5.954,5.166) {\large $\alpha = 2.95$};
\node[gp node left,font={\fontsize{10pt}{12pt}\selectfont}] at (2.821,3.540) {\large exact};
\node[gp node left,font={\fontsize{10pt}{12pt}\selectfont}] at (2.821,3.076) {\large HLLD-CWM};
\node[gp node left,font={\fontsize{10pt}{12pt}\selectfont}] at (2.821,2.611) {\large HLLD};
%% coordinates of the plot area
\gpdefrectangularnode{gp plot 1}{\pgfpoint{0.828cm}{0.985cm}}{\pgfpoint{7.947cm}{5.631cm}}
\end{tikzpicture}
%% gnuplot variables
}
\end{tabular}
\caption{Solution consisting of two fast rarefactions and two rotational discontinuities found using HLLD-CWM without the (optional) flux redistribution step, HLLD, and the exact solver using $2048$ grid points for the non-planar case.}
\label{fig:AK7_crsol}
\end{figure}

The solution of Test~7 with $A = -0.1$ and 4096 grid points using HLLD, and HLLD-CWM without any additional artificial viscosity is shown in Figure~\ref{fig:AK7_crsol}.  The right-going compound wave is no longer present in either the HLLD or HLLD-CWM solution.  A small deviation at $x\approx 0.691$ is present in the CWM solution.  This can be eliminated by increasing the threshold value $\beta_T$ for when CWM is applied.  However, if $\beta_T$ were increased so that CWM was not applied to the right-going compound wave, the error in the approximate solution at the tail of the right-going rarefaction located at $x \approx 0.732$ would increase.  This is a desirable property of the CWM method, the adjustment applied is reduced as the compound wave disappears.  The jump across the CD is still incorrect with and without CWM, however, the state downstream of the CD is better approximated with CWM because the jump across the left-going SS at $x\approx 0.47$ is more accurate.  The maximum value of $\rho$ approximated with CWM is $1.7657$ and $1.5596$ without it.  The percent difference form the exact solution downstream of the CD (row 4 of Table~\ref{tab:AK7}) is reduce from $12\%$ to $0.2\%$ when CWM is enabled.

Test~7 differs from Tests 5 and 6 in that the base scheme does not approximate the positions of the SS downstream of a CW correctly.  The SCW of Tests 5a and 5b removed the SS from the approximate solution.  In Tests 6a and 6b, the approximated position of the SS is relatively accurate, although slight improvement is seen when CWM is enabled.  In Test~7, the position of both the left- and right-going SS is incorrect.  Although still incorrect, when CWM is enabled, the approximation of the shock positions improves.  

The increase in accuracy obtained with CWM was demonstrated with three different test problems.  In each case, when CWM enabled, the jump across and position of each wave affected by pseudo-convergence was more accurately captured.  In the next section, the increase in accuracy provided by CWM is quantified with RMSE calculations where CWM is shown to reduce the RMSE for both coplanar and near coplanar cases.

%----------------------------------------------------------------------------------------
% Error analysis
%----------------------------------------------------------------------------------------
\section[Error analysis]{Error analysis}
\label{sec:error}

The appearance of regular structures in the approximate solution is also independent of grid refinement in the region $x=[0.254,0.481]$ of Figure~\ref{fig:coplanar_ab_crsol} for Test 5 and $x = [0.348,0.551]$ of Figure~\ref{fig:fast_coplanar_ab_crsol}, where the c-solution differs form the r-solution.  This is the region between the tail of the left-going fast rarefaction and the right-going contact discontinuity (CD).  The region includes either a left-going slow shock and rotational discontinuity or, a slow compound wave for Test~5 or a fast compound wave and slow shock for Test~6.  

The RMSE was computed using
\begin{gather*}
\text{RMSE} = \sqrt{\sum_{i=1}^M \mathcal{E}_i^2} \text{ ,}
\end{gather*}
and
\begin{gather*}
\mathcal{E} = \frac{1}{N}\sum_{i=1}^N \left|U_i - U_{ex}(x_i)\right|
\end{gather*} 
where $M$ is the number of conservative state variables, $N$ is the number of grid points, $U_i$ is an approximated conservative state variable, and $U_{ex}$ is the exact solution for the conservative state variable.  

%-----------------------------------------------------------------
% RMS-error Slow
%-----------------------------------------------------------------
\begin{figure}[htbp] 
\begin{tabular}{cc}
\resizebox{0.5\linewidth}{!}{\tikzsetnextfilename{coplanar_b_L1_err_1}\begin{tikzpicture}[gnuplot]
%% generated with GNUPLOT 4.6p4 (Lua 5.1; terminal rev. 99, script rev. 100)
%% Mon 02 Jun 2014 05:44:56 PM EDT
\path (0.000,0.000) rectangle (8.500,6.000);
\gpfill{rgb color={1.000,1.000,1.000}} (1.196,0.985)--(7.946,0.985)--(7.946,5.630)--(1.196,5.630)--cycle;
\gpcolor{color=gp lt color border}
\gpsetlinetype{gp lt border}
\gpsetlinewidth{1.00}
\draw[gp path] (1.196,0.985)--(1.196,5.630)--(7.946,5.630)--(7.946,0.985)--cycle;
\gpsetlinewidth{2.00}
\draw[gp path] (1.196,0.985)--(1.250,0.985);
\draw[gp path] (7.947,0.985)--(7.893,0.985);
\draw[gp path] (1.196,1.108)--(1.250,1.108);
\draw[gp path] (7.947,1.108)--(7.893,1.108);
\draw[gp path] (1.196,1.215)--(1.250,1.215);
\draw[gp path] (7.947,1.215)--(7.893,1.215);
\draw[gp path] (1.196,1.309)--(1.250,1.309);
\draw[gp path] (7.947,1.309)--(7.893,1.309);
\draw[gp path] (1.196,1.394)--(1.304,1.394);
\draw[gp path] (7.947,1.394)--(7.839,1.394);
\gpcolor{rgb color={0.000,0.000,0.000}}
\node[gp node right,font={\fontsize{10pt}{12pt}\selectfont}] at (1.012,1.394) {$10^{-3}$};
\gpcolor{color=gp lt color border}
\draw[gp path] (1.196,1.394)--(1.250,1.394);
\draw[gp path] (7.947,1.394)--(7.893,1.394);
\draw[gp path] (1.196,1.948)--(1.250,1.948);
\draw[gp path] (7.947,1.948)--(7.893,1.948);
\draw[gp path] (1.196,2.272)--(1.250,2.272);
\draw[gp path] (7.947,2.272)--(7.893,2.272);
\draw[gp path] (1.196,2.502)--(1.250,2.502);
\draw[gp path] (7.947,2.502)--(7.893,2.502);
\draw[gp path] (1.196,2.681)--(1.250,2.681);
\draw[gp path] (7.947,2.681)--(7.893,2.681);
\draw[gp path] (1.196,2.827)--(1.250,2.827);
\draw[gp path] (7.947,2.827)--(7.893,2.827);
\draw[gp path] (1.196,2.950)--(1.250,2.950);
\draw[gp path] (7.947,2.950)--(7.893,2.950);
\draw[gp path] (1.196,3.057)--(1.250,3.057);
\draw[gp path] (7.947,3.057)--(7.893,3.057);
\draw[gp path] (1.196,3.151)--(1.250,3.151);
\draw[gp path] (7.947,3.151)--(7.893,3.151);
\draw[gp path] (1.196,3.235)--(1.304,3.235);
\draw[gp path] (7.947,3.235)--(7.839,3.235);
\gpcolor{rgb color={0.000,0.000,0.000}}
\node[gp node right,font={\fontsize{10pt}{12pt}\selectfont}] at (1.012,3.235) {$10^{-2}$};
\gpcolor{color=gp lt color border}
\draw[gp path] (1.196,3.235)--(1.250,3.235);
\draw[gp path] (7.947,3.235)--(7.893,3.235);
\draw[gp path] (1.196,3.789)--(1.250,3.789);
\draw[gp path] (7.947,3.789)--(7.893,3.789);
\draw[gp path] (1.196,4.114)--(1.250,4.114);
\draw[gp path] (7.947,4.114)--(7.893,4.114);
\draw[gp path] (1.196,4.344)--(1.250,4.344);
\draw[gp path] (7.947,4.344)--(7.893,4.344);
\draw[gp path] (1.196,4.522)--(1.250,4.522);
\draw[gp path] (7.947,4.522)--(7.893,4.522);
\draw[gp path] (1.196,4.668)--(1.250,4.668);
\draw[gp path] (7.947,4.668)--(7.893,4.668);
\draw[gp path] (1.196,4.791)--(1.250,4.791);
\draw[gp path] (7.947,4.791)--(7.893,4.791);
\draw[gp path] (1.196,4.898)--(1.250,4.898);
\draw[gp path] (7.947,4.898)--(7.893,4.898);
\draw[gp path] (1.196,4.992)--(1.250,4.992);
\draw[gp path] (7.947,4.992)--(7.893,4.992);
\draw[gp path] (1.196,5.077)--(1.304,5.077);
\draw[gp path] (7.947,5.077)--(7.839,5.077);
\gpcolor{rgb color={0.000,0.000,0.000}}
\node[gp node right,font={\fontsize{10pt}{12pt}\selectfont}] at (1.012,5.077) {$10^{-1}$};
\gpcolor{color=gp lt color border}
\draw[gp path] (1.196,5.077)--(1.250,5.077);
\draw[gp path] (7.947,5.077)--(7.893,5.077);
\draw[gp path] (1.196,5.631)--(1.250,5.631);
\draw[gp path] (7.947,5.631)--(7.893,5.631);
\draw[gp path] (1.511,0.985)--(1.511,1.093);
\draw[gp path] (1.511,5.631)--(1.511,5.523);
\gpcolor{rgb color={0.000,0.000,0.000}}
\node[gp node center,font={\fontsize{10pt}{12pt}\selectfont}] at (1.511,0.677) {$2^{6}$};
\gpcolor{color=gp lt color border}
\draw[gp path] (2.394,0.985)--(2.394,1.093);
\draw[gp path] (2.394,5.631)--(2.394,5.523);
\gpcolor{rgb color={0.000,0.000,0.000}}
\node[gp node center,font={\fontsize{10pt}{12pt}\selectfont}] at (2.394,0.677) {$2^{7}$};
\gpcolor{color=gp lt color border}
\draw[gp path] (3.277,0.985)--(3.277,1.093);
\draw[gp path] (3.277,5.631)--(3.277,5.523);
\gpcolor{rgb color={0.000,0.000,0.000}}
\node[gp node center,font={\fontsize{10pt}{12pt}\selectfont}] at (3.277,0.677) {$2^{8}$};
\gpcolor{color=gp lt color border}
\draw[gp path] (4.160,0.985)--(4.160,1.093);
\draw[gp path] (4.160,5.631)--(4.160,5.523);
\gpcolor{rgb color={0.000,0.000,0.000}}
\node[gp node center,font={\fontsize{10pt}{12pt}\selectfont}] at (4.160,0.677) {$2^{9}$};
\gpcolor{color=gp lt color border}
\draw[gp path] (5.043,0.985)--(5.043,1.093);
\draw[gp path] (5.043,5.631)--(5.043,5.523);
\gpcolor{rgb color={0.000,0.000,0.000}}
\node[gp node center,font={\fontsize{10pt}{12pt}\selectfont}] at (5.043,0.677) {$2^{10}$};
\gpcolor{color=gp lt color border}
\draw[gp path] (5.927,0.985)--(5.927,1.093);
\draw[gp path] (5.927,5.631)--(5.927,5.523);
\gpcolor{rgb color={0.000,0.000,0.000}}
\node[gp node center,font={\fontsize{10pt}{12pt}\selectfont}] at (5.927,0.677) {$2^{11}$};
\gpcolor{color=gp lt color border}
\draw[gp path] (6.810,0.985)--(6.810,1.093);
\draw[gp path] (6.810,5.631)--(6.810,5.523);
\gpcolor{rgb color={0.000,0.000,0.000}}
\node[gp node center,font={\fontsize{10pt}{12pt}\selectfont}] at (6.810,0.677) {$2^{12}$};
\gpcolor{color=gp lt color border}
\draw[gp path] (7.693,0.985)--(7.693,1.093);
\draw[gp path] (7.693,5.631)--(7.693,5.523);
\gpcolor{rgb color={0.000,0.000,0.000}}
\node[gp node center,font={\fontsize{10pt}{12pt}\selectfont}] at (7.693,0.677) {$2^{13}$};
\gpcolor{color=gp lt color border}
\draw[gp path] (1.196,5.631)--(1.196,0.985)--(7.947,0.985)--(7.947,5.631)--cycle;
\gpcolor{rgb color={0.000,0.000,0.000}}
\node[gp node center,font={\fontsize{10pt}{12pt}\selectfont}] at (4.571,0.215) {\large \# of grid pts.};
\gpcolor{rgb color={1.000,0.000,0.000}}
\gpsetlinetype{gp lt plot 0}
\gpsetlinewidth{3.00}
\draw[gp path] (1.511,5.015)--(2.394,4.554)--(3.277,4.135)--(4.160,3.435)--(5.043,2.895)%
  --(5.927,2.564)--(6.810,2.371)--(7.693,2.084);
\gpsetpointsize{5.33}
\gppoint{gp mark 9}{(1.511,5.015)}
\gppoint{gp mark 9}{(2.394,4.554)}
\gppoint{gp mark 9}{(3.277,4.135)}
\gppoint{gp mark 9}{(4.160,3.435)}
\gppoint{gp mark 9}{(5.043,2.895)}
\gppoint{gp mark 9}{(5.927,2.564)}
\gppoint{gp mark 9}{(6.810,2.371)}
\gppoint{gp mark 9}{(7.693,2.084)}
\gpcolor{rgb color={0.000,0.000,0.000}}
\gpsetlinewidth{1.00}
\gppoint{gp mark 8}{(1.511,5.015)}
\gppoint{gp mark 8}{(2.394,4.554)}
\gppoint{gp mark 8}{(3.277,4.135)}
\gppoint{gp mark 8}{(4.160,3.435)}
\gppoint{gp mark 8}{(5.043,2.895)}
\gppoint{gp mark 8}{(5.927,2.564)}
\gppoint{gp mark 8}{(6.810,2.371)}
\gppoint{gp mark 8}{(7.693,2.084)}
\gpcolor{rgb color={0.000,0.000,1.000}}
\gpsetlinewidth{2.00}
\draw[gp path] (1.511,5.057)--(2.394,4.881)--(3.277,4.741)--(4.160,4.558)--(5.043,4.431)%
  --(5.927,4.182)--(6.810,3.830)--(7.693,3.430);
\gppoint{gp mark 9}{(1.511,5.057)}
\gppoint{gp mark 9}{(2.394,4.881)}
\gppoint{gp mark 9}{(3.277,4.741)}
\gppoint{gp mark 9}{(4.160,4.558)}
\gppoint{gp mark 9}{(5.043,4.431)}
\gppoint{gp mark 9}{(5.927,4.182)}
\gppoint{gp mark 9}{(6.810,3.830)}
\gppoint{gp mark 9}{(7.693,3.430)}
\gpcolor{rgb color={0.000,0.000,0.000}}
\gpsetlinewidth{1.00}
\gppoint{gp mark 8}{(1.511,5.057)}
\gppoint{gp mark 8}{(2.394,4.881)}
\gppoint{gp mark 8}{(3.277,4.741)}
\gppoint{gp mark 8}{(4.160,4.558)}
\gppoint{gp mark 8}{(5.043,4.431)}
\gppoint{gp mark 8}{(5.927,4.182)}
\gppoint{gp mark 8}{(6.810,3.830)}
\gppoint{gp mark 8}{(7.693,3.430)}
\node[gp node left,font={\fontsize{10pt}{12pt}\selectfont}] at (6.181,5.077) {\large RMSE};
\node[gp node left,font={\fontsize{10pt}{12pt}\selectfont}] at (6.181,4.522) {\large $\alpha = 3.0$};
%% coordinates of the plot area
\gpdefrectangularnode{gp plot 1}{\pgfpoint{1.196cm}{0.985cm}}{\pgfpoint{7.947cm}{5.631cm}}
\end{tikzpicture}
%% gnuplot variables
} &
\resizebox{0.5\linewidth}{!}{\tikzsetnextfilename{coplanar_a_L1_err_1}\begin{tikzpicture}[gnuplot]
%% generated with GNUPLOT 4.6p4 (Lua 5.1; terminal rev. 99, script rev. 100)
%% Mon 02 Jun 2014 05:46:18 PM EDT
\path (0.000,0.000) rectangle (8.500,6.000);
\gpfill{rgb color={1.000,1.000,1.000}} (1.196,0.985)--(7.946,0.985)--(7.946,5.630)--(1.196,5.630)--cycle;
\gpcolor{color=gp lt color border}
\gpsetlinetype{gp lt border}
\gpsetlinewidth{1.00}
\draw[gp path] (1.196,0.985)--(1.196,5.630)--(7.946,5.630)--(7.946,0.985)--cycle;
\gpsetlinewidth{2.00}
\draw[gp path] (1.196,0.985)--(1.250,0.985);
\draw[gp path] (7.947,0.985)--(7.893,0.985);
\draw[gp path] (1.196,1.108)--(1.250,1.108);
\draw[gp path] (7.947,1.108)--(7.893,1.108);
\draw[gp path] (1.196,1.215)--(1.250,1.215);
\draw[gp path] (7.947,1.215)--(7.893,1.215);
\draw[gp path] (1.196,1.309)--(1.250,1.309);
\draw[gp path] (7.947,1.309)--(7.893,1.309);
\draw[gp path] (1.196,1.394)--(1.304,1.394);
\draw[gp path] (7.947,1.394)--(7.839,1.394);
\gpcolor{rgb color={0.000,0.000,0.000}}
\node[gp node right,font={\fontsize{10pt}{12pt}\selectfont}] at (1.012,1.394) {$10^{-3}$};
\gpcolor{color=gp lt color border}
\draw[gp path] (1.196,1.394)--(1.250,1.394);
\draw[gp path] (7.947,1.394)--(7.893,1.394);
\draw[gp path] (1.196,1.948)--(1.250,1.948);
\draw[gp path] (7.947,1.948)--(7.893,1.948);
\draw[gp path] (1.196,2.272)--(1.250,2.272);
\draw[gp path] (7.947,2.272)--(7.893,2.272);
\draw[gp path] (1.196,2.502)--(1.250,2.502);
\draw[gp path] (7.947,2.502)--(7.893,2.502);
\draw[gp path] (1.196,2.681)--(1.250,2.681);
\draw[gp path] (7.947,2.681)--(7.893,2.681);
\draw[gp path] (1.196,2.827)--(1.250,2.827);
\draw[gp path] (7.947,2.827)--(7.893,2.827);
\draw[gp path] (1.196,2.950)--(1.250,2.950);
\draw[gp path] (7.947,2.950)--(7.893,2.950);
\draw[gp path] (1.196,3.057)--(1.250,3.057);
\draw[gp path] (7.947,3.057)--(7.893,3.057);
\draw[gp path] (1.196,3.151)--(1.250,3.151);
\draw[gp path] (7.947,3.151)--(7.893,3.151);
\draw[gp path] (1.196,3.235)--(1.304,3.235);
\draw[gp path] (7.947,3.235)--(7.839,3.235);
\gpcolor{rgb color={0.000,0.000,0.000}}
\node[gp node right,font={\fontsize{10pt}{12pt}\selectfont}] at (1.012,3.235) {$10^{-2}$};
\gpcolor{color=gp lt color border}
\draw[gp path] (1.196,3.235)--(1.250,3.235);
\draw[gp path] (7.947,3.235)--(7.893,3.235);
\draw[gp path] (1.196,3.789)--(1.250,3.789);
\draw[gp path] (7.947,3.789)--(7.893,3.789);
\draw[gp path] (1.196,4.114)--(1.250,4.114);
\draw[gp path] (7.947,4.114)--(7.893,4.114);
\draw[gp path] (1.196,4.344)--(1.250,4.344);
\draw[gp path] (7.947,4.344)--(7.893,4.344);
\draw[gp path] (1.196,4.522)--(1.250,4.522);
\draw[gp path] (7.947,4.522)--(7.893,4.522);
\draw[gp path] (1.196,4.668)--(1.250,4.668);
\draw[gp path] (7.947,4.668)--(7.893,4.668);
\draw[gp path] (1.196,4.791)--(1.250,4.791);
\draw[gp path] (7.947,4.791)--(7.893,4.791);
\draw[gp path] (1.196,4.898)--(1.250,4.898);
\draw[gp path] (7.947,4.898)--(7.893,4.898);
\draw[gp path] (1.196,4.992)--(1.250,4.992);
\draw[gp path] (7.947,4.992)--(7.893,4.992);
\draw[gp path] (1.196,5.077)--(1.304,5.077);
\draw[gp path] (7.947,5.077)--(7.839,5.077);
\gpcolor{rgb color={0.000,0.000,0.000}}
\node[gp node right,font={\fontsize{10pt}{12pt}\selectfont}] at (1.012,5.077) {$10^{-1}$};
\gpcolor{color=gp lt color border}
\draw[gp path] (1.196,5.077)--(1.250,5.077);
\draw[gp path] (7.947,5.077)--(7.893,5.077);
\draw[gp path] (1.196,5.631)--(1.250,5.631);
\draw[gp path] (7.947,5.631)--(7.893,5.631);
\draw[gp path] (1.511,0.985)--(1.511,1.093);
\draw[gp path] (1.511,5.631)--(1.511,5.523);
\gpcolor{rgb color={0.000,0.000,0.000}}
\node[gp node center,font={\fontsize{10pt}{12pt}\selectfont}] at (1.511,0.677) {$2^{6}$};
\gpcolor{color=gp lt color border}
\draw[gp path] (2.394,0.985)--(2.394,1.093);
\draw[gp path] (2.394,5.631)--(2.394,5.523);
\gpcolor{rgb color={0.000,0.000,0.000}}
\node[gp node center,font={\fontsize{10pt}{12pt}\selectfont}] at (2.394,0.677) {$2^{7}$};
\gpcolor{color=gp lt color border}
\draw[gp path] (3.277,0.985)--(3.277,1.093);
\draw[gp path] (3.277,5.631)--(3.277,5.523);
\gpcolor{rgb color={0.000,0.000,0.000}}
\node[gp node center,font={\fontsize{10pt}{12pt}\selectfont}] at (3.277,0.677) {$2^{8}$};
\gpcolor{color=gp lt color border}
\draw[gp path] (4.160,0.985)--(4.160,1.093);
\draw[gp path] (4.160,5.631)--(4.160,5.523);
\gpcolor{rgb color={0.000,0.000,0.000}}
\node[gp node center,font={\fontsize{10pt}{12pt}\selectfont}] at (4.160,0.677) {$2^{9}$};
\gpcolor{color=gp lt color border}
\draw[gp path] (5.043,0.985)--(5.043,1.093);
\draw[gp path] (5.043,5.631)--(5.043,5.523);
\gpcolor{rgb color={0.000,0.000,0.000}}
\node[gp node center,font={\fontsize{10pt}{12pt}\selectfont}] at (5.043,0.677) {$2^{10}$};
\gpcolor{color=gp lt color border}
\draw[gp path] (5.927,0.985)--(5.927,1.093);
\draw[gp path] (5.927,5.631)--(5.927,5.523);
\gpcolor{rgb color={0.000,0.000,0.000}}
\node[gp node center,font={\fontsize{10pt}{12pt}\selectfont}] at (5.927,0.677) {$2^{11}$};
\gpcolor{color=gp lt color border}
\draw[gp path] (6.810,0.985)--(6.810,1.093);
\draw[gp path] (6.810,5.631)--(6.810,5.523);
\gpcolor{rgb color={0.000,0.000,0.000}}
\node[gp node center,font={\fontsize{10pt}{12pt}\selectfont}] at (6.810,0.677) {$2^{12}$};
\gpcolor{color=gp lt color border}
\draw[gp path] (7.693,0.985)--(7.693,1.093);
\draw[gp path] (7.693,5.631)--(7.693,5.523);
\gpcolor{rgb color={0.000,0.000,0.000}}
\node[gp node center,font={\fontsize{10pt}{12pt}\selectfont}] at (7.693,0.677) {$2^{13}$};
\gpcolor{color=gp lt color border}
\draw[gp path] (1.196,5.631)--(1.196,0.985)--(7.947,0.985)--(7.947,5.631)--cycle;
\gpcolor{rgb color={0.000,0.000,0.000}}
\node[gp node center,font={\fontsize{10pt}{12pt}\selectfont}] at (4.571,0.215) {\large \# of grid pts.};
\gpcolor{rgb color={1.000,0.000,0.000}}
\gpsetlinetype{gp lt plot 0}
\gpsetlinewidth{3.00}
\draw[gp path] (1.511,5.031)--(2.394,4.532)--(3.277,4.097)--(4.160,3.302)--(5.043,2.952)%
  --(5.927,2.609)--(6.810,2.268)--(7.693,2.233);
\gpsetpointsize{5.33}
\gppoint{gp mark 9}{(1.511,5.031)}
\gppoint{gp mark 9}{(2.394,4.532)}
\gppoint{gp mark 9}{(3.277,4.097)}
\gppoint{gp mark 9}{(4.160,3.302)}
\gppoint{gp mark 9}{(5.043,2.952)}
\gppoint{gp mark 9}{(5.927,2.609)}
\gppoint{gp mark 9}{(6.810,2.268)}
\gppoint{gp mark 9}{(7.693,2.233)}
\gpcolor{rgb color={0.000,0.000,0.000}}
\gpsetlinewidth{1.00}
\gppoint{gp mark 8}{(1.511,5.031)}
\gppoint{gp mark 8}{(2.394,4.532)}
\gppoint{gp mark 8}{(3.277,4.097)}
\gppoint{gp mark 8}{(4.160,3.302)}
\gppoint{gp mark 8}{(5.043,2.952)}
\gppoint{gp mark 8}{(5.927,2.609)}
\gppoint{gp mark 8}{(6.810,2.268)}
\gppoint{gp mark 8}{(7.693,2.233)}
\gpcolor{rgb color={0.000,0.000,1.000}}
\gpsetlinewidth{2.00}
\draw[gp path] (1.511,5.069)--(2.394,4.877)--(3.277,4.740)--(4.160,4.584)--(5.043,4.529)%
  --(5.927,4.523)--(6.810,4.529)--(7.693,4.517);
\gppoint{gp mark 9}{(1.511,5.069)}
\gppoint{gp mark 9}{(2.394,4.877)}
\gppoint{gp mark 9}{(3.277,4.740)}
\gppoint{gp mark 9}{(4.160,4.584)}
\gppoint{gp mark 9}{(5.043,4.529)}
\gppoint{gp mark 9}{(5.927,4.523)}
\gppoint{gp mark 9}{(6.810,4.529)}
\gppoint{gp mark 9}{(7.693,4.517)}
\gpcolor{rgb color={0.000,0.000,0.000}}
\gpsetlinewidth{1.00}
\gppoint{gp mark 8}{(1.511,5.069)}
\gppoint{gp mark 8}{(2.394,4.877)}
\gppoint{gp mark 8}{(3.277,4.740)}
\gppoint{gp mark 8}{(4.160,4.584)}
\gppoint{gp mark 8}{(5.043,4.529)}
\gppoint{gp mark 8}{(5.927,4.523)}
\gppoint{gp mark 8}{(6.810,4.529)}
\gppoint{gp mark 8}{(7.693,4.517)}
\node[gp node left,font={\fontsize{10pt}{12pt}\selectfont}] at (6.181,5.077) {\large RMSE};
\node[gp node left,font={\fontsize{10pt}{12pt}\selectfont}] at (6.302,4.114) {\large $\alpha = \pi$};
%% coordinates of the plot area
\gpdefrectangularnode{gp plot 1}{\pgfpoint{1.196cm}{0.985cm}}{\pgfpoint{7.947cm}{5.631cm}}
\end{tikzpicture}
%% gnuplot variables
} 
\end{tabular}
\caption{RMSE in $x=[0.254,0.481]$ using HLLD (blue) and HLLD-CWM (red) for Tests (left) 5a and (right) 5b.}
\label{fig:coplanar_b_err_rms}
\end{figure}

The RMSEs of the original HLLD scheme and the modified HLLD-CWM scheme for both the coplanar and near-coplanar cases of Test~5 are shown in Figure~\ref{fig:coplanar_b_err_rms}.  For the near-coplanar case, a reduction in RMSE through grid refinement occurs with both HLLD and HLLD-CWM.  Initially convergence is much quicker with HLLD-CWM.  As numerical diffusion is decreased through grid refinement, the compound wave breaks apart and the convergence rate using HLLD increases at about $2^{10}$ grid points.  For the coplanar case, the RMSE is reduced through grid refinement only when the HLLD-CWM flux is used.  Note that the domain used in this analysis differs slightly from $x = [0.2,0.4]$ used in \citep{Torrilhon:2003b}; the domain used here covers the largest differences between the exact and non-converging solution.

%-----------------------------------------------------------------
% RMS-error Fast
%-----------------------------------------------------------------
\begin{figure}[htbp] 
\begin{tabular}{cc}
\resizebox{0.5\linewidth}{!}{\tikzsetnextfilename{fast_coplanar_b_L1_err_1}\begin{tikzpicture}[gnuplot]
%% generated with GNUPLOT 4.6p4 (Lua 5.1; terminal rev. 99, script rev. 100)
%% Fri 22 Aug 2014 10:58:53 AM EDT
\path (0.000,0.000) rectangle (8.500,6.000);
\gpfill{rgb color={1.000,1.000,1.000}} (1.196,0.985)--(7.946,0.985)--(7.946,5.630)--(1.196,5.630)--cycle;
\gpcolor{color=gp lt color border}
\gpsetlinetype{gp lt border}
\gpsetlinewidth{1.00}
\draw[gp path] (1.196,0.985)--(1.196,5.630)--(7.946,5.630)--(7.946,0.985)--cycle;
\gpsetlinewidth{2.00}
\draw[gp path] (1.196,0.985)--(1.250,0.985);
\draw[gp path] (7.947,0.985)--(7.893,0.985);
\draw[gp path] (1.196,1.108)--(1.250,1.108);
\draw[gp path] (7.947,1.108)--(7.893,1.108);
\draw[gp path] (1.196,1.215)--(1.250,1.215);
\draw[gp path] (7.947,1.215)--(7.893,1.215);
\draw[gp path] (1.196,1.309)--(1.250,1.309);
\draw[gp path] (7.947,1.309)--(7.893,1.309);
\draw[gp path] (1.196,1.394)--(1.304,1.394);
\draw[gp path] (7.947,1.394)--(7.839,1.394);
\gpcolor{rgb color={0.000,0.000,0.000}}
\node[gp node right,font={\fontsize{10pt}{12pt}\selectfont}] at (1.012,1.394) {$10^{-3}$};
\gpcolor{color=gp lt color border}
\draw[gp path] (1.196,1.394)--(1.250,1.394);
\draw[gp path] (7.947,1.394)--(7.893,1.394);
\draw[gp path] (1.196,1.948)--(1.250,1.948);
\draw[gp path] (7.947,1.948)--(7.893,1.948);
\draw[gp path] (1.196,2.272)--(1.250,2.272);
\draw[gp path] (7.947,2.272)--(7.893,2.272);
\draw[gp path] (1.196,2.502)--(1.250,2.502);
\draw[gp path] (7.947,2.502)--(7.893,2.502);
\draw[gp path] (1.196,2.681)--(1.250,2.681);
\draw[gp path] (7.947,2.681)--(7.893,2.681);
\draw[gp path] (1.196,2.827)--(1.250,2.827);
\draw[gp path] (7.947,2.827)--(7.893,2.827);
\draw[gp path] (1.196,2.950)--(1.250,2.950);
\draw[gp path] (7.947,2.950)--(7.893,2.950);
\draw[gp path] (1.196,3.057)--(1.250,3.057);
\draw[gp path] (7.947,3.057)--(7.893,3.057);
\draw[gp path] (1.196,3.151)--(1.250,3.151);
\draw[gp path] (7.947,3.151)--(7.893,3.151);
\draw[gp path] (1.196,3.235)--(1.304,3.235);
\draw[gp path] (7.947,3.235)--(7.839,3.235);
\gpcolor{rgb color={0.000,0.000,0.000}}
\node[gp node right,font={\fontsize{10pt}{12pt}\selectfont}] at (1.012,3.235) {$10^{-2}$};
\gpcolor{color=gp lt color border}
\draw[gp path] (1.196,3.235)--(1.250,3.235);
\draw[gp path] (7.947,3.235)--(7.893,3.235);
\draw[gp path] (1.196,3.789)--(1.250,3.789);
\draw[gp path] (7.947,3.789)--(7.893,3.789);
\draw[gp path] (1.196,4.114)--(1.250,4.114);
\draw[gp path] (7.947,4.114)--(7.893,4.114);
\draw[gp path] (1.196,4.344)--(1.250,4.344);
\draw[gp path] (7.947,4.344)--(7.893,4.344);
\draw[gp path] (1.196,4.522)--(1.250,4.522);
\draw[gp path] (7.947,4.522)--(7.893,4.522);
\draw[gp path] (1.196,4.668)--(1.250,4.668);
\draw[gp path] (7.947,4.668)--(7.893,4.668);
\draw[gp path] (1.196,4.791)--(1.250,4.791);
\draw[gp path] (7.947,4.791)--(7.893,4.791);
\draw[gp path] (1.196,4.898)--(1.250,4.898);
\draw[gp path] (7.947,4.898)--(7.893,4.898);
\draw[gp path] (1.196,4.992)--(1.250,4.992);
\draw[gp path] (7.947,4.992)--(7.893,4.992);
\draw[gp path] (1.196,5.077)--(1.304,5.077);
\draw[gp path] (7.947,5.077)--(7.839,5.077);
\gpcolor{rgb color={0.000,0.000,0.000}}
\node[gp node right,font={\fontsize{10pt}{12pt}\selectfont}] at (1.012,5.077) {$10^{-1}$};
\gpcolor{color=gp lt color border}
\draw[gp path] (1.196,5.077)--(1.250,5.077);
\draw[gp path] (7.947,5.077)--(7.893,5.077);
\draw[gp path] (1.196,5.631)--(1.250,5.631);
\draw[gp path] (7.947,5.631)--(7.893,5.631);
\draw[gp path] (1.511,0.985)--(1.511,1.093);
\draw[gp path] (1.511,5.631)--(1.511,5.523);
\gpcolor{rgb color={0.000,0.000,0.000}}
\node[gp node center,font={\fontsize{10pt}{12pt}\selectfont}] at (1.511,0.677) {$2^{6}$};
\gpcolor{color=gp lt color border}
\draw[gp path] (2.394,0.985)--(2.394,1.093);
\draw[gp path] (2.394,5.631)--(2.394,5.523);
\gpcolor{rgb color={0.000,0.000,0.000}}
\node[gp node center,font={\fontsize{10pt}{12pt}\selectfont}] at (2.394,0.677) {$2^{7}$};
\gpcolor{color=gp lt color border}
\draw[gp path] (3.277,0.985)--(3.277,1.093);
\draw[gp path] (3.277,5.631)--(3.277,5.523);
\gpcolor{rgb color={0.000,0.000,0.000}}
\node[gp node center,font={\fontsize{10pt}{12pt}\selectfont}] at (3.277,0.677) {$2^{8}$};
\gpcolor{color=gp lt color border}
\draw[gp path] (4.160,0.985)--(4.160,1.093);
\draw[gp path] (4.160,5.631)--(4.160,5.523);
\gpcolor{rgb color={0.000,0.000,0.000}}
\node[gp node center,font={\fontsize{10pt}{12pt}\selectfont}] at (4.160,0.677) {$2^{9}$};
\gpcolor{color=gp lt color border}
\draw[gp path] (5.043,0.985)--(5.043,1.093);
\draw[gp path] (5.043,5.631)--(5.043,5.523);
\gpcolor{rgb color={0.000,0.000,0.000}}
\node[gp node center,font={\fontsize{10pt}{12pt}\selectfont}] at (5.043,0.677) {$2^{10}$};
\gpcolor{color=gp lt color border}
\draw[gp path] (5.927,0.985)--(5.927,1.093);
\draw[gp path] (5.927,5.631)--(5.927,5.523);
\gpcolor{rgb color={0.000,0.000,0.000}}
\node[gp node center,font={\fontsize{10pt}{12pt}\selectfont}] at (5.927,0.677) {$2^{11}$};
\gpcolor{color=gp lt color border}
\draw[gp path] (6.810,0.985)--(6.810,1.093);
\draw[gp path] (6.810,5.631)--(6.810,5.523);
\gpcolor{rgb color={0.000,0.000,0.000}}
\node[gp node center,font={\fontsize{10pt}{12pt}\selectfont}] at (6.810,0.677) {$2^{12}$};
\gpcolor{color=gp lt color border}
\draw[gp path] (7.693,0.985)--(7.693,1.093);
\draw[gp path] (7.693,5.631)--(7.693,5.523);
\gpcolor{rgb color={0.000,0.000,0.000}}
\node[gp node center,font={\fontsize{10pt}{12pt}\selectfont}] at (7.693,0.677) {$2^{13}$};
\gpcolor{color=gp lt color border}
\draw[gp path] (1.196,5.631)--(1.196,0.985)--(7.947,0.985)--(7.947,5.631)--cycle;
\gpcolor{rgb color={0.000,0.000,0.000}}
\node[gp node center,font={\fontsize{10pt}{12pt}\selectfont}] at (4.571,0.215) {\large \# of grid pts.};
\gpcolor{rgb color={1.000,0.000,0.000}}
\gpsetlinetype{gp lt plot 0}
\gpsetlinewidth{3.00}
\draw[gp path] (1.511,5.070)--(2.394,4.601)--(3.277,4.125)--(4.160,3.381)--(5.043,3.000)%
  --(5.927,2.537)--(6.810,2.344)--(7.693,2.048);
\gpsetpointsize{5.33}
\gppoint{gp mark 9}{(1.511,5.070)}
\gppoint{gp mark 9}{(2.394,4.601)}
\gppoint{gp mark 9}{(3.277,4.125)}
\gppoint{gp mark 9}{(4.160,3.381)}
\gppoint{gp mark 9}{(5.043,3.000)}
\gppoint{gp mark 9}{(5.927,2.537)}
\gppoint{gp mark 9}{(6.810,2.344)}
\gppoint{gp mark 9}{(7.693,2.048)}
\gpcolor{rgb color={0.000,0.000,0.000}}
\gpsetlinewidth{1.00}
\gppoint{gp mark 8}{(1.511,5.070)}
\gppoint{gp mark 8}{(2.394,4.601)}
\gppoint{gp mark 8}{(3.277,4.125)}
\gppoint{gp mark 8}{(4.160,3.381)}
\gppoint{gp mark 8}{(5.043,3.000)}
\gppoint{gp mark 8}{(5.927,2.537)}
\gppoint{gp mark 8}{(6.810,2.344)}
\gppoint{gp mark 8}{(7.693,2.048)}
\gpcolor{rgb color={0.000,0.000,1.000}}
\gpsetlinewidth{2.00}
\draw[gp path] (1.511,5.124)--(2.394,4.797)--(3.277,4.577)--(4.160,4.161)--(5.043,4.032)%
  --(5.927,3.939)--(6.810,3.727)--(7.693,3.353);
\gppoint{gp mark 9}{(1.511,5.124)}
\gppoint{gp mark 9}{(2.394,4.797)}
\gppoint{gp mark 9}{(3.277,4.577)}
\gppoint{gp mark 9}{(4.160,4.161)}
\gppoint{gp mark 9}{(5.043,4.032)}
\gppoint{gp mark 9}{(5.927,3.939)}
\gppoint{gp mark 9}{(6.810,3.727)}
\gppoint{gp mark 9}{(7.693,3.353)}
\gpcolor{rgb color={0.000,0.000,0.000}}
\gpsetlinewidth{1.00}
\gppoint{gp mark 8}{(1.511,5.124)}
\gppoint{gp mark 8}{(2.394,4.797)}
\gppoint{gp mark 8}{(3.277,4.577)}
\gppoint{gp mark 8}{(4.160,4.161)}
\gppoint{gp mark 8}{(5.043,4.032)}
\gppoint{gp mark 8}{(5.927,3.939)}
\gppoint{gp mark 8}{(6.810,3.727)}
\gppoint{gp mark 8}{(7.693,3.353)}
\node[gp node left,font={\fontsize{10pt}{12pt}\selectfont}] at (6.181,5.077) {\large RMSE};
\node[gp node left,font={\fontsize{10pt}{12pt}\selectfont}] at (6.181,4.522) {\large $\alpha = 3.0$};
%% coordinates of the plot area
\gpdefrectangularnode{gp plot 1}{\pgfpoint{1.196cm}{0.985cm}}{\pgfpoint{7.947cm}{5.631cm}}
\end{tikzpicture}
%% gnuplot variables
} &
\resizebox{0.5\linewidth}{!}{\tikzsetnextfilename{fast_coplanar_a_L1_err_1}\begin{tikzpicture}[gnuplot]
%% generated with GNUPLOT 4.6p4 (Lua 5.1; terminal rev. 99, script rev. 100)
%% Fri 22 Aug 2014 11:11:52 AM EDT
\path (0.000,0.000) rectangle (8.500,6.000);
\gpfill{rgb color={1.000,1.000,1.000}} (1.196,0.985)--(7.946,0.985)--(7.946,5.630)--(1.196,5.630)--cycle;
\gpcolor{color=gp lt color border}
\gpsetlinetype{gp lt border}
\gpsetlinewidth{1.00}
\draw[gp path] (1.196,0.985)--(1.196,5.630)--(7.946,5.630)--(7.946,0.985)--cycle;
\gpsetlinewidth{2.00}
\draw[gp path] (1.196,0.985)--(1.250,0.985);
\draw[gp path] (7.947,0.985)--(7.893,0.985);
\draw[gp path] (1.196,1.108)--(1.250,1.108);
\draw[gp path] (7.947,1.108)--(7.893,1.108);
\draw[gp path] (1.196,1.215)--(1.250,1.215);
\draw[gp path] (7.947,1.215)--(7.893,1.215);
\draw[gp path] (1.196,1.309)--(1.250,1.309);
\draw[gp path] (7.947,1.309)--(7.893,1.309);
\draw[gp path] (1.196,1.394)--(1.304,1.394);
\draw[gp path] (7.947,1.394)--(7.839,1.394);
\gpcolor{rgb color={0.000,0.000,0.000}}
\node[gp node right,font={\fontsize{10pt}{12pt}\selectfont}] at (1.012,1.394) {$10^{-3}$};
\gpcolor{color=gp lt color border}
\draw[gp path] (1.196,1.394)--(1.250,1.394);
\draw[gp path] (7.947,1.394)--(7.893,1.394);
\draw[gp path] (1.196,1.948)--(1.250,1.948);
\draw[gp path] (7.947,1.948)--(7.893,1.948);
\draw[gp path] (1.196,2.272)--(1.250,2.272);
\draw[gp path] (7.947,2.272)--(7.893,2.272);
\draw[gp path] (1.196,2.502)--(1.250,2.502);
\draw[gp path] (7.947,2.502)--(7.893,2.502);
\draw[gp path] (1.196,2.681)--(1.250,2.681);
\draw[gp path] (7.947,2.681)--(7.893,2.681);
\draw[gp path] (1.196,2.827)--(1.250,2.827);
\draw[gp path] (7.947,2.827)--(7.893,2.827);
\draw[gp path] (1.196,2.950)--(1.250,2.950);
\draw[gp path] (7.947,2.950)--(7.893,2.950);
\draw[gp path] (1.196,3.057)--(1.250,3.057);
\draw[gp path] (7.947,3.057)--(7.893,3.057);
\draw[gp path] (1.196,3.151)--(1.250,3.151);
\draw[gp path] (7.947,3.151)--(7.893,3.151);
\draw[gp path] (1.196,3.235)--(1.304,3.235);
\draw[gp path] (7.947,3.235)--(7.839,3.235);
\gpcolor{rgb color={0.000,0.000,0.000}}
\node[gp node right,font={\fontsize{10pt}{12pt}\selectfont}] at (1.012,3.235) {$10^{-2}$};
\gpcolor{color=gp lt color border}
\draw[gp path] (1.196,3.235)--(1.250,3.235);
\draw[gp path] (7.947,3.235)--(7.893,3.235);
\draw[gp path] (1.196,3.789)--(1.250,3.789);
\draw[gp path] (7.947,3.789)--(7.893,3.789);
\draw[gp path] (1.196,4.114)--(1.250,4.114);
\draw[gp path] (7.947,4.114)--(7.893,4.114);
\draw[gp path] (1.196,4.344)--(1.250,4.344);
\draw[gp path] (7.947,4.344)--(7.893,4.344);
\draw[gp path] (1.196,4.522)--(1.250,4.522);
\draw[gp path] (7.947,4.522)--(7.893,4.522);
\draw[gp path] (1.196,4.668)--(1.250,4.668);
\draw[gp path] (7.947,4.668)--(7.893,4.668);
\draw[gp path] (1.196,4.791)--(1.250,4.791);
\draw[gp path] (7.947,4.791)--(7.893,4.791);
\draw[gp path] (1.196,4.898)--(1.250,4.898);
\draw[gp path] (7.947,4.898)--(7.893,4.898);
\draw[gp path] (1.196,4.992)--(1.250,4.992);
\draw[gp path] (7.947,4.992)--(7.893,4.992);
\draw[gp path] (1.196,5.077)--(1.304,5.077);
\draw[gp path] (7.947,5.077)--(7.839,5.077);
\gpcolor{rgb color={0.000,0.000,0.000}}
\node[gp node right,font={\fontsize{10pt}{12pt}\selectfont}] at (1.012,5.077) {$10^{-1}$};
\gpcolor{color=gp lt color border}
\draw[gp path] (1.196,5.077)--(1.250,5.077);
\draw[gp path] (7.947,5.077)--(7.893,5.077);
\draw[gp path] (1.196,5.631)--(1.250,5.631);
\draw[gp path] (7.947,5.631)--(7.893,5.631);
\draw[gp path] (1.511,0.985)--(1.511,1.093);
\draw[gp path] (1.511,5.631)--(1.511,5.523);
\gpcolor{rgb color={0.000,0.000,0.000}}
\node[gp node center,font={\fontsize{10pt}{12pt}\selectfont}] at (1.511,0.677) {$2^{6}$};
\gpcolor{color=gp lt color border}
\draw[gp path] (2.394,0.985)--(2.394,1.093);
\draw[gp path] (2.394,5.631)--(2.394,5.523);
\gpcolor{rgb color={0.000,0.000,0.000}}
\node[gp node center,font={\fontsize{10pt}{12pt}\selectfont}] at (2.394,0.677) {$2^{7}$};
\gpcolor{color=gp lt color border}
\draw[gp path] (3.277,0.985)--(3.277,1.093);
\draw[gp path] (3.277,5.631)--(3.277,5.523);
\gpcolor{rgb color={0.000,0.000,0.000}}
\node[gp node center,font={\fontsize{10pt}{12pt}\selectfont}] at (3.277,0.677) {$2^{8}$};
\gpcolor{color=gp lt color border}
\draw[gp path] (4.160,0.985)--(4.160,1.093);
\draw[gp path] (4.160,5.631)--(4.160,5.523);
\gpcolor{rgb color={0.000,0.000,0.000}}
\node[gp node center,font={\fontsize{10pt}{12pt}\selectfont}] at (4.160,0.677) {$2^{9}$};
\gpcolor{color=gp lt color border}
\draw[gp path] (5.043,0.985)--(5.043,1.093);
\draw[gp path] (5.043,5.631)--(5.043,5.523);
\gpcolor{rgb color={0.000,0.000,0.000}}
\node[gp node center,font={\fontsize{10pt}{12pt}\selectfont}] at (5.043,0.677) {$2^{10}$};
\gpcolor{color=gp lt color border}
\draw[gp path] (5.927,0.985)--(5.927,1.093);
\draw[gp path] (5.927,5.631)--(5.927,5.523);
\gpcolor{rgb color={0.000,0.000,0.000}}
\node[gp node center,font={\fontsize{10pt}{12pt}\selectfont}] at (5.927,0.677) {$2^{11}$};
\gpcolor{color=gp lt color border}
\draw[gp path] (6.810,0.985)--(6.810,1.093);
\draw[gp path] (6.810,5.631)--(6.810,5.523);
\gpcolor{rgb color={0.000,0.000,0.000}}
\node[gp node center,font={\fontsize{10pt}{12pt}\selectfont}] at (6.810,0.677) {$2^{12}$};
\gpcolor{color=gp lt color border}
\draw[gp path] (7.693,0.985)--(7.693,1.093);
\draw[gp path] (7.693,5.631)--(7.693,5.523);
\gpcolor{rgb color={0.000,0.000,0.000}}
\node[gp node center,font={\fontsize{10pt}{12pt}\selectfont}] at (7.693,0.677) {$2^{13}$};
\gpcolor{color=gp lt color border}
\draw[gp path] (1.196,5.631)--(1.196,0.985)--(7.947,0.985)--(7.947,5.631)--cycle;
\gpcolor{rgb color={0.000,0.000,0.000}}
\node[gp node center,font={\fontsize{10pt}{12pt}\selectfont}] at (4.571,0.215) {\large \# of grid pts.};
\gpcolor{rgb color={1.000,0.000,0.000}}
\gpsetlinetype{gp lt plot 0}
\gpsetlinewidth{3.00}
\draw[gp path] (1.511,5.083)--(2.394,4.639)--(3.277,4.140)--(4.160,3.443)--(5.043,2.990)%
  --(5.927,2.535)--(6.810,2.422)--(7.693,2.162);
\gpsetpointsize{5.33}
\gppoint{gp mark 9}{(1.511,5.083)}
\gppoint{gp mark 9}{(2.394,4.639)}
\gppoint{gp mark 9}{(3.277,4.140)}
\gppoint{gp mark 9}{(4.160,3.443)}
\gppoint{gp mark 9}{(5.043,2.990)}
\gppoint{gp mark 9}{(5.927,2.535)}
\gppoint{gp mark 9}{(6.810,2.422)}
\gppoint{gp mark 9}{(7.693,2.162)}
\gpcolor{rgb color={0.000,0.000,0.000}}
\gpsetlinewidth{1.00}
\gppoint{gp mark 8}{(1.511,5.083)}
\gppoint{gp mark 8}{(2.394,4.639)}
\gppoint{gp mark 8}{(3.277,4.140)}
\gppoint{gp mark 8}{(4.160,3.443)}
\gppoint{gp mark 8}{(5.043,2.990)}
\gppoint{gp mark 8}{(5.927,2.535)}
\gppoint{gp mark 8}{(6.810,2.422)}
\gppoint{gp mark 8}{(7.693,2.162)}
\gpcolor{rgb color={0.000,0.000,1.000}}
\gpsetlinewidth{2.00}
\draw[gp path] (1.511,5.127)--(2.394,4.820)--(3.277,4.454)--(4.160,4.195)--(5.043,4.040)%
  --(5.927,3.919)--(6.810,3.911)--(7.693,3.882);
\gppoint{gp mark 9}{(1.511,5.127)}
\gppoint{gp mark 9}{(2.394,4.820)}
\gppoint{gp mark 9}{(3.277,4.454)}
\gppoint{gp mark 9}{(4.160,4.195)}
\gppoint{gp mark 9}{(5.043,4.040)}
\gppoint{gp mark 9}{(5.927,3.919)}
\gppoint{gp mark 9}{(6.810,3.911)}
\gppoint{gp mark 9}{(7.693,3.882)}
\gpcolor{rgb color={0.000,0.000,0.000}}
\gpsetlinewidth{1.00}
\gppoint{gp mark 8}{(1.511,5.127)}
\gppoint{gp mark 8}{(2.394,4.820)}
\gppoint{gp mark 8}{(3.277,4.454)}
\gppoint{gp mark 8}{(4.160,4.195)}
\gppoint{gp mark 8}{(5.043,4.040)}
\gppoint{gp mark 8}{(5.927,3.919)}
\gppoint{gp mark 8}{(6.810,3.911)}
\gppoint{gp mark 8}{(7.693,3.882)}
\node[gp node left,font={\fontsize{10pt}{12pt}\selectfont}] at (6.181,5.077) {\large RMSE};
\node[gp node left,font={\fontsize{10pt}{12pt}\selectfont}] at (6.302,4.522) {\large $\alpha = \pi$};
%% coordinates of the plot area
\gpdefrectangularnode{gp plot 1}{\pgfpoint{1.196cm}{0.985cm}}{\pgfpoint{7.947cm}{5.631cm}}
\end{tikzpicture}
%% gnuplot variables
} 
\end{tabular}
\caption{RMSE in $x=[0.348,0.551]$ using HLLD (blue) and HLLD-CWM (red) for Tests (left) 6a and (right) 6b.}
\label{fig:fast_coplanar_b_err_rms}
\end{figure}

The RMSEs of the original HLLD scheme and the modified HLLD-CWM scheme for both the coplanar and near-coplanar cases of Test~6 are shown in Figure~\ref{fig:fast_coplanar_b_err_rms}.  Unlike Test~5, a regular slow shock is located in the domain in which the RMSEs were computed.  The presence of the regular shock is the reason for the initial reduction in error with HLLD for the coplanar case.  For the near-coplanar case, a reduction in RMSE through grid refinement occurs with both HLLD and HLLD-CWM.  As with Test~5, convergence is faster with HLLD-CWM until between $2^{10}$ and $2^{11}$ grid points at which point the rate using HLLD increases at about.  For the coplanar case, after the initial convergence, between $2^{8}$ and $2^{9}$ grid points, only HLLD-CWM is able to reduce the RMSE.  The CWM should be tested with high order WENO schemes.  This has the potential to reduce the need for artificial viscosity downstream of the rotational discontinuity at higher Courant numbers.

The new method known as CWM, which eliminated the compound wave from the solution to coplanar and near coplanar cases at all grid resolutions.  The CWM method was shown to localize the effects of a compound wave to the transition layer of a rotational discontinuity while maintaining conservation for a multiple of test problems.  For weak intermediate shocks, as in Test~6, the compound wave was removed with minimal error through the transition.  In Test~7, CWM was shown to drastically reduce the effects of a compound wave in the presence of other numerical inaccuracies.  These results have never been achieved prior to CWM without an exact solver.  In the next section, a description of numerical methods for two-dimensional HD and MHD is given.  \Gls{ct} is incorporated into the ideal MHD solver to maintain $\divergebf{B} = 0$ the machine precision.  The process of implementing the solvers for shared memory parallelism is described.  Results concluding the GPU out preforms the CPU by two to three times are reported.  




